\section{Moy--Prasad filtration}
\label{sec:moyprasad}

Recall that all supercuspidal representations are conjecturally induced from compact subgroups.
Moy--Prasad filtration gives a partial answer to the question; it is a filtration of $G$ attached to each point of $\scB(G)$, which is used to define \emph{depth} of representations.
Especially, we can classify all the \emph{depth-zero} supercuspidal representations, where all of them arise as inductions of \emph{depth-zero minimal $K$ types}.

Before we define the Moy--Prasad filtration, we introduce the notion of \emph{parahoric subgroups}, which are groups $G_{x}$ associated to each point $x \in \scB(G)$.
When $x \in \scA(G, T)$, we define $G_{x}$ and its (pro-)unipotent radical $G_{x}^+$ as follows:
% For each $x \in \scB(G)$, we associate a \emph{parahoric}\footnote{\textbf{para}bolic + Iwa\textbf{horic}, indeed.} subgroup (an affine analogue of parabolic subgroups) $G_{x}$ and its unipotent radical $G_{x}^+$ as
\begin{align}
    G_{x} &:= \langle T(\bZ_p), x_\alpha(p^{-\lfloor \alpha(x) \rfloor}) : \alpha \in \Phi \rangle \\
    G_{x}^+ &:= \langle T(1 + p\bZ_p), x_\alpha(p^{1 - \lceil \alpha(x) \rceil}) : \alpha \in \Phi \rangle
\end{align}
Now, for an arbitary point $x \in \scB(G)$, take $x_0 \in \scA(G)$ and $g \in G$ with $x = g \cdot x_0$.
Then we define $G_{x} := g G_{x_0} g^{-1}$ and $G_{x}^+ := g G_{x_0}^+ g^{-1}$, which is independent of the choice of $x_0$ and $g$.
Both of the groups only depends on the facet $\mathscr{F}$ containing $x$, and sometimes we denote them by $G_{\mathscr{F}}$ and $G_{\mathscr{F}}^+$.
Note that the quotient $G_{x} / G_{x}^+$ is always a Lie group of finite type, which we denote as $\mathscr{G}_{x}$.
%  finite group, and we denote it by $F_{x}$.

Moy--Prasad filtration is a filtration of these two groups.

\begin{definition}[Moy--Prasad filtration]
Let $x \in \scA(G, T)$.
Moy--Prasad filtration of $G_x$ and $G_x^+$ is given by, for each $r \in \bR_{\ge 0}$,
\begin{align}
    G_{x, r} &:= \langle T(1 + p^{\lceil r \rceil}), x_{\alpha}(p^{-\lfloor \alpha(x) - r \rfloor}) : \alpha \in \Phi \rangle \subset G_x \\
    G_{x, r^+} &:= \bigcap_{s > r} G_{x, s}
    %  \langle T(1 + p^{\lfloor r \rfloor + 1}), x_{\alpha}(p^{1 - \lceil \alpha(x) - r\rceil}) : \alpha \in \Phi \rangle \subset G_{x, r}
\end{align}
\end{definition}
We can generalize the definition to any $x \in \scB(G)$ similarly as before.
It is easy to check that for any $x$, $G_{x, 0} = G_{x}$, $G_{x, 0^+} = G_{x}^+$, and $\{G_{x, r}\}_{r \ge 0}$ form a decreasing filtration and a basis of open compact neighborhoods of the identity in $G$.
The parameter $r$ of the filtration is additive in the following sense: we have $[G_{x, r}, G_{x, r'}] \subseteq G_{x, r + r'}$.
Also, we have analogous filtrations $\frg_{x, r^+} \subset \frg_{x, r} \subset \frg_{x}$ for the Lie algebra $\frg = \mathrm{Lie}(G)$.

% \textcolor{red}{Add filtrations for $\SL_{2}(\bQ_2)$}
For example, consider $G = \SL_2(\bQ_3)$ again.
There are essentially two (or three) different possibilities: vertices or the points on the middle of the edges.
There are two different types of vertices, correspond to two different conjugacy classes of maximal compact subgroups, which are
\[
    x_1 \leftrightarrow \begin{bmatrix} \bZ_3 & \bZ_3 \\ \bZ_3 & \bZ_3 \end{bmatrix}, \qquad x_2 \leftrightarrow \begin{bmatrix} \bZ_3 & 3 \bZ_3 \\ \frac{1}{3} \bZ_3 & \bZ_3 \end{bmatrix}.
\]
For the point $y$ in the middle of $x_1$ and $x_2$, it corresponds to
\[
    y \leftrightarrow \begin{bmatrix}
        \bZ_3 & 3 \bZ_3 \\ \bZ_3 & \bZ_3
    \end{bmatrix}.
\]

The corresponding Moy--Prasad filtrations are given by
% \begin{align*}
%     G_{x_1,0} = \begin{bmatrix}
%         \bZ_3 & \bZ_3 \\ \bZ_3 & \bZ_3
%     \end{bmatrix} \qquad G_{y,0} &= \begin{bmatrix}
%         \bZ_3 & 3 \bZ_3 \\ \bZ_3 & \bZ_3
%     \end{bmatrix} \qquad G_{x_2} = \begin{bmatrix}
%         \bZ_3 & 3 \bZ_3 \\ \frac{1}{2} \bZ_3 & \bZ_3
%     \end{bmatrix} \\
%     G_{y, 0.5} &= \begin{bmatrix}
%         1 + 3 \bZ_3 & 3 \bZ_3 \\ \bZ_3 & 1 + 3 \bZ_3
%     \end{bmatrix} \\
%     G_{x_1, 1} = \begin{bmatrix}
%         1 + 3 \bZ_3 & 3 \bZ_3 \\ 3 \bZ_3 & 1 + 3 \bZ_3
%     \end{bmatrix} \qquad G_{y, 1} &= \begin{bmatrix}
%         1 + 3 \bZ_3 & 3^3 \bZ_3 \\ 3 \bZ_3 & 1 + 3 \bZ_3
%     \end{bmatrix} \qquad G_{x_2, 1} = \begin{bmatrix}
%         1 + 3 \bZ_3 & 3^3 \bZ_3 \\ 3 \bZ_3 & 1 + 3 \bZ_3 
%     \end{bmatrix} \\
%     G_{y, 1.5} &= \begin{bmatrix}
%         1 + 3^3 \bZ_3 & 3^3 \bZ_3 \\ 3 \bZ_3 & 1 + 3^3 \bZ_3
%     \end{bmatrix}
% \end{align*}
% the one with $x_{\alpha}(p^{-1})$ and the one with $x_{\alpha}(p^{-2})$.

\begin{table}[h]
    \footnotesize
    \begin{center}
        \begin{tabular}{ccc}
        % \begin{tabular}{c|c|l}
            $G_{x_1,0} = \begin{bmatrix}
                \bZ_3 & \bZ_3 \\ \bZ_3 & \bZ_3
            \end{bmatrix}$ & $G_{y,0} = \begin{bmatrix}
                \bZ_3 & 3 \bZ_3 \\ \bZ_3 & \bZ_3
            \end{bmatrix}$ & $G_{x_2} = \begin{bmatrix}
                \bZ_3 & 3 \bZ_3 \\ \frac{1}{2} \bZ_3 & \bZ_3
            \end{bmatrix}$ \\
            & & \\
            & $\bigcup$ & \\
            & & \\
            $\bigcup$ & $G_{y, 0.5} = \begin{bmatrix}
                1 + 3 \bZ_3 & 3 \bZ_3 \\ \bZ_3 & 1 + 3 \bZ_3
            \end{bmatrix}$ & $\bigcup$ \\
            & & \\
            & $\bigcup$ & \\
            & & \\
            $G_{x_1, 1} = \begin{bmatrix}
                1 + 3 \bZ_3 & 3 \bZ_3 \\ 3 \bZ_3 & 1 + 3 \bZ_3
            \end{bmatrix}$ & $G_{y, 1} = \begin{bmatrix}
                1 + 3 \bZ_3 & 3^3 \bZ_3 \\ 3 \bZ_3 & 1 + 3 \bZ_3
            \end{bmatrix}$ & $G_{x_2, 1} = \begin{bmatrix}
                1 + 3 \bZ_3 & 3^3 \bZ_3 \\ 3 \bZ_3 & 1 + 3 \bZ_3 
            \end{bmatrix}$ \\
            & & \\
            & $\bigcup$ & \\
            & & \\
            $\bigcup$ & $G_{y, 1.5} = \begin{bmatrix}
                1 + 3^3 \bZ_3 & 3^3 \bZ_3 \\ 3 \bZ_3 & 1 + 3^3 \bZ_3
            \end{bmatrix}$ & $\bigcup$ \\
            $\vdots$ & $\vdots$ & $\vdots$
        \end{tabular}
        % \caption{Known optimal bounds for uncertainty principle}
        \label{tab:uncertainty_optimal_bounds}
    \end{center}
\end{table}

They proved that any (smooth) representation of $G(F)$ possesses a vector fixed by a group in the filtration.

\begin{theorem}[Moy--Prasad {\cite[Theorem 5.2]{moy1994unrefined}}]
\label{thm:mpdepth}
If $(\pi, V)$ is a smooth representation of $G$, then there is a nonnegative real number $r = \varrho(\pi)$ with the property that $r$ is the minimal number such that $V^{G_{x, r^+}} \ne \{0\}$ for some $x \in \scB(G)$.
\end{theorem}

We call the number $\varrho(\pi)$ as the \emph{depth} of $\pi$.
Espeically, depth-zero representation $(\pi, V)$ is a representation with $V^{G_x^+} \ne \{0\}$.
Note that the depth of a representation does not need to be an integer: we will see an example of depth $\frac{1}{2}$ representation in Section \ref{sec:yu}.

Moreover, they proved that all the depth-zero supercuspidal representations arise from the representations of finite groups, generalizing the construction of representations of $\GL_2(\bQ_p)$ in Section \ref{sec:intro}.

\begin{definition}
A \emph{depth-zero minimal $K$-type} is a pair $(G_x, \tau)$ where $x \in \scB(G)$ and $\tau$ is a cuspidal representation of the finite group $G_{x} / G_{x}^+$, inflated to $G_{x}$.
\end{definition}

\begin{theorem}[Moy--Prasad \cite{moy1994unrefined,moy1996jacquet}]
\label{thm:mpd0}
Let $(G_x, \sigma)$ be a minimal $K$-type and $\scE(\sigma)$ be the set of irreducible representations of $F_{x} = N_G(G_x)$ (up to equivalence) whose restriction to $G_x$ contains $\sigma$.
Then for any $\tau \in \scE(\sigma)$, $\Ind_{F_x}^{G}(\tau)$ an irreducible supercuspidal representation of $G$ (necessarily of depth-zero), and any depth-zero irreducible supercuspidal representations arises from some $x \in \scB(G)$ in this way.
\end{theorem}
% The theorem tells us that the construction of depth-zero supercuspidal representations reduce to the construction of cuspidal representations of finite groups.
The construction of the cuspidal representations of ``finite groups of Lie type'' (e.g. has a form of $\scG(\bF_q)$ for some $\scG_{/\bF_q}$) are usually understood by the Deligne--Lusztig theory ($\ell$-adic cohomology of a variety $X_{/\bF_q}$ with $\scG(\bF_q)$-action on it).