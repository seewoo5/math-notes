
\section{Theta correspondence and Howe duality}

\subsection{Half-integral weight modular forms}
The theta series, defined as 
$$
\theta(z) = \sum_{n\in\mathbb{Z}} q^{n^{2}}, \qquad q = e^{2\pi i z}
$$
is regarded as a powerful to study lattices and quadratic forms.
For example, power of the theta series $\theta(z)^{k}$ is a generating function of $r_{k}(n)$,
the number of ways to represent an integral $n$ as a sum of $k$ squares.
For \emph{even} $k$, $\theta(z)^{k}$ is a weight $k/2$ modular form,
which makes us to analyze $\theta(z)^{k}$ more closely and even find the formula for
$r_{k}(n)$.
For example, $\theta(z)^{2}$ is a weight 1 modular form on $\Gamma_{1}(4)$ with character (Nebentypus)
$\chi_{4}$, the primitive Dirichlet character of level 4.
The space $S_{1}(\Gamma_{1}(4), \chi_{4})$ of such modular forms has dimension 1, so that $\theta(z)^{2}$
is actually a non-zero multiple of certain weight 1 Eisenstein series, and this gives a formula
$$
r_{2}(n) = 4 \sum_{2\nmid d | n} (-1)^{(d-1)/2}
$$
and this gives a one-line proof for the Fermat's theorem on sum of two squares.
Similarly, $\theta(z)^{4}$ is also a modular form (of weight 2 on $\Gamma_{0}(4)$),
and the similar argument gives a formula
$$
r_{4}(n) = 8 \sum_{4\nmid d | n} d
$$
and Lagrange's four square theorem is a direct consequence of this
(see Zagier's article \emph{Elliptic Modular Forms and Their Applications}
in the book \cite{bruinier20081} for details).

How about the \emph{odd} powers of $\theta(z)$? For example, what is $\theta(z)$ itself?

\subsection{Shimura correspondence and Shintani lift}
In \cite{shimura1973onmodular}, Shimura established a lift from the space of 
half integral weight modular forms to the space of integral weight modular forms.
\begin{theorem}[Shimura, \cite{shimura1973onmodular}]
\end{theorem}

\subsection{Waldspurger's work}

\subsection{Theta correspondence and Howe duality}

\subsection{Gan-Gross-Prasad conjecture}