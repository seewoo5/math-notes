\newpage

\newcommand\px[1]{\partial x_{#1}}
\newcommand\py[1]{\partial y_{#1}}
\newcommand\pd{\partial}
\newcommand\dx[1]{\frac{\partial}{\partial x_{#1}}}
\newcommand\dy[1]{\frac{\partial}{\partial y_{#1}}}
\newcommand\dxx[1]{\frac{\partial^2}{\partial x_{#1}^2}}
\newcommand\ddxx[2]{\frac{\partial^2}{\partial x_{#1}x_{#2}}}
\newcommand\dyy[1]{\frac{\partial^2}{\partial y_{#1}^2}}
\newcommand\ddyy[2]{\frac{\partial^2}{\partial y_{#1}y_{#2}}}

\newcommand\Sym{\mathrm{Sym}}


\section{Symmetric power lifting}

Automorphic forms on $\mathrm{GL}(2)$ are often \emph{classified} into two kinds of objects: \emph{modular forms} and \emph{Maass forms}\footnote{and constant functions.}.
These functions oftenly considered as a starting point for studying automorphic forms and representations for $\mathrm{GL}(n)$ and other groups.
However, there are not many references for $\mathrm{GL}(3)$.

\subsection{Automorphic forms on $\mathrm{GL}(3, \mathbb{R})$}

We first introduce the theory of automorphic forms on $\mathrm{GL}(3)$ (We follow the Bump's book \cite{bump2006automorphic}).
We only consider the level 1 automorphic forms.
Before we start, let's revisit the $\GL(2)$.
Modular forms and Maass forms are certain functions defined on the complex upper half plane $\mathfrak{H}$, and one can lift the functions as a function on $\mathrm{GL}(2, \mathbb{R})$ by viewing $\mathfrak{H}$ as a symmetric space
$$
\mathfrak{H} \simeq  \GL(2, \mathbb{R})/Z(\mathbb{R})\mathrm{O}(2).
$$
Here $Z(\mathbb{R}) \simeq \mathbb{R}^{\times}$ is the center of $\GL(2, \mathbb{R})$ and $\mathrm{O}(2)$ is a group of orthogonal matrices. 
The above isomorphism holds since $\GL(2, \mathbb{R})$ acts on $\mathbb{H}$ transitively and the stabilizer of $i$ is $Z(\mathbb{R})\mathrm{O}(2)$.
To develop a theory of automorphic forms on $\GL(3)$, it is natural to consider them as a function defined on the symmetric space 
$$
    \mathfrak{H}_{3} := \GL(3,\mathbb{R}) / Z(\mathbb{R})\mathrm{O}(3).
$$
Using Iwasawa decomposition, each coset has a unique representation of the form 
$$
\begin{pmatrix} 1 & x_{2} & x_{3} \\ 0 & 1 & x_{1} \\ 0 & 0 & 1 \end{pmatrix} \begin{pmatrix}
y_{1}y_{2} & & \\ & y_{1} & \\ & & 1
\end{pmatrix}, \quad y_{1}, y_{2} > 0.
$$
Especially, the space is parametrized with 5 real variables and has a real dimension 5, so we can't expect any \emph{holomorphic} automorphic form over $\GL(3)$, in constrast to the $\GL(2)$ case.
Also, we have an involution $\iota$ on $\mathfrak{H}_{3}$ defined as 
$$
\begin{pmatrix} 1 & x_{2} & x_{3} \\ 0 & 1 & x_{1} \\ 0 & 0 & 1 \end{pmatrix} \begin{pmatrix}
y_{1}y_{2} & & \\ & y_{1} & \\ & & 1
\end{pmatrix} \mapsto
\begin{pmatrix} 1 & -x_{1} & x_{1}x_{2} - x_{3} \\ 0 & 1 & -x_{2} \\ 0 & 0 & 1 \end{pmatrix} \begin{pmatrix}
y_{1}y_{2} & & \\ & y_{1} & \\ & & 1
\end{pmatrix}.
$$
% For $\GL_{2}$ case, Maass forms have a Fourier expansion in terms of \emph{Whittaker function}, which is a second order differential equation where two solutions are expressed as Bessel functions.
What about the Fourier expansion of $\GL(3)$ automorphic forms?
In case of $\GL(2)$, the algebra of $\GL(2, \mathbb{R})$-invariant differential operators on $\mathfrak{h}_{2}$ is isomorphic to a polynomial ring of single variable, generated by the following hyperbolic Laplacian
$$
\Delta = -y^{2} \left(\frac{\partial^{2}}{\partial x^{2}} + \frac{\partial^{2}}{\partial y^{2}}\right).
$$
A 1-periodic function $f(z) = \sum_{n\geq 0} a_{n}(y)e^{2\pi i n x}$ become an eigenfunction with respect to $\Delta$ if the coefficients $a_{n}(y)$ satisfy certain degree 2 linear differential equations.
More precisely, when $\Delta f = \left( \frac{1}{4} - \nu^{2}\right) f$, the $n$-th coefficient $a_{n}(y)$ satisfies
$$
y^{2} \frac{\partial^{2}}{\partial y^{2}}a_{n}(y) + \left(\frac{1}{4} - \nu^{2} - 4\pi^{2}n^{2}y\right)a_{n}(y) = 0.
$$
Among two linearly independent solutions, only one satisfies the required growth condition (the other one grows exponentially), which can be expressed with a Bessel function of second kind:
$$
a_{n}(y) = c_{n} \sqrt{y}K_{\nu}(2\pi |n|y), \quad K_{\nu}(y) := \frac{1}{2}\int_{0}^{\infty}e^{\frac{y(t + t^{-1})}{2}}t^{\nu} \frac{dt}{t}.
$$

For $\GL(3)$, the algebra of $\GL(3, \mathbb{R})$-invariant differential operators on $\mathfrak{h}_{3}$ is a polynomial ring in \emph{two} variables, with two specific generators $\Delta_{1}, \Delta_{2}$.
Then the automorphic forms of $\GL(3, \mathbb{R})$ would be defined as functions that are eigenforms with respect to these two operators.
Then the coefficients of the Fourier expansion (which will be defined explicitly later) of the automorphic forms would satisfy specific differential equations.
In fact, for given $\lambda$ and $\mu$, there are 6 linearly independent functions that are
\begin{enumerate}
    \item eigenfunctions with respect to $\Delta_{1}, \Delta_{2}$, i.e. 
    \begin{align*}
        \Delta_{1}F = \lambda F \\
        \Delta_{2}F = \mu F
    \end{align*}
    \item and satisfies the equation
    $$
        F\left(\begin{pmatrix}
        1 & x_{1} & x_{3} \\ 0 & 1 & x_{2} \\ 0 & 0 & 1
        \end{pmatrix}\tau\right) = e(x_{1} + x_{2})F(\tau)
    $$
    for all $\tau$ and $x_{1}, x_{2}, x_{3} \in \mathbb{R}$, where $e(x):= \exp(2\pi i x)$.
\end{enumerate}
Among these 6 solutions, only one of them \emph{decays rapidly}, which is the appropriate substitute of $K_{\nu}(y)$ for $\GL(3)$ (This is multiplicity one theorem for $\GL(3)$).
It can be written as an inverse Mellin transform of a certain 2-variable function $V(s_{1}, s_{2})$,
\begin{align*}
W(y_1, y_2) &= W_{\nu_{1}, \nu_{2}}(y_{1}, y_{2}) \\
&= \frac{1}{4} \frac{1}{(2\pi i)^{2}} \int_{\sigma - i\infty}^{\sigma + i\infty} \int_{\sigma - i\infty}^{\sigma + i\infty} V(s_1, s_2)(\pi y_{1})^{1-s_{1}} (\pi y_{2})^{1-s_{2}} ds_{1}ds_{2}
\end{align*}
where
$$
V(s_{1}, s_{2}) = \frac{\Gamma\left(\frac{s_{1} + \alpha}{2}\right)\Gamma\left(\frac{s_{1} + \beta}{2}\right)\Gamma\left(\frac{s_{1} + \gamma}{2}\right)\Gamma\left(\frac{s_{2} - \alpha}{2}\right)\Gamma\left(\frac{s_{2} - \beta}{2}\right)\Gamma\left(\frac{s_{2} - \gamma}{2}\right)}{\Gamma\left(\frac{s_{1} + s_{2}}{2}\right)}.
$$
Here $\alpha, \beta, \gamma$ are auxillary parameters satisfy
\begin{align*}
    \alpha &= -\nu_{1} - 2\nu_{2} +1 \\
    \beta &= -\nu_{1} + \nu_{2} \\\
    \gamma &= 2\nu_{1} + \nu_{2} - 1 \\
    \lambda &= -1 - \alpha\beta - \beta\gamma - \gamma\alpha \\
    \mu &= -\alpha\beta\gamma.
\end{align*}
The Whittaker function $W(y_1, y_2)$ also can be written as an integral of Bessel functions
\begin{align*}
    &W_{\nu_{1}, \nu_{2}}(y_{1}, y_{2}) = 4y_{1}^{1 - \frac{\beta}{2}}y_{2}^{1 + \frac{\beta}{2}} \int_{0}^{\infty} K_{\frac{\gamma - \alpha}{2}}(2\pi y_{2} \sqrt{1 + u^{-2}}) K_{\frac{\gamma - \alpha}{2}}(2\pi y_{1} \sqrt{1 + u^{2}}) u^{\frac{-3\beta}{2}} \frac{du}{u}.
\end{align*}


\begin{definition}[Automorphic form on $\GL(3, \mathbb{R})$]
Let $\nu_{1}, \nu_{2} \in \mathbb{C}$.
An automorphic form of type $(\nu_{1}, \nu_{2})$ on $\GL(3, \mathbb{R})$ is a function $\phi: \mathfrak{H}_{3} \to \mathbb{C}$ such that 
\begin{enumerate}
    \item $\phi(g\tau) = \phi(\tau)$ for all $g\in \GL(3, \mathbb{Z})$ and $\tau \in \GL_{3}(\mathbb{R})$.
    \item $\phi$ is an eigenfunction of $\Delta_{1}, \Delta_{2}$ with eigenvalues $\lambda, \mu$ defined above,
    \item there exists $n_{1}, n_{2}$ such that 
    $$
    y_{1}^{n_{1}}y_{2}^{n_{2}} \phi\left(\begin{pmatrix}
    y_{1}y_{2} & & \\ & y_{1} & \\ & & 1
    \end{pmatrix}\right)
    $$
    is bounded on the subset of $\mathfrak{H}_{2}$ determined by $y_{1}, y_{2} > 1$.
\end{enumerate}
In addition, for all $\tau\in \mathfrak{H}_{3}$, if
$$
\int_{0}^{1} \int_{0}^{1} \phi\left(\begin{pmatrix}
1 & & \xi_{3} \\ & 1 & \xi_{1} \\ & & 1
\end{pmatrix}\tau\right) d\xi_{1} d\xi_{3} = 0
$$
and
$$
\int_{0}^{1} \int_{0}^{1} \phi\left(\begin{pmatrix}
1 & \xi_{2} & \xi_{3} \\ & 1 & \\ & & 1
\end{pmatrix}\tau\right) d\xi_{1} d\xi_{3} = 0
$$
then $\phi$ is called \emph{cusp form}.
\end{definition}
Note that, for a given automorphic form $\phi$ of type $(\nu_1, \nu_2)$, the \emph{dual} form $\tilde{\phi}(\tau) := \phi({}^{\iota}\tau)$ is also an automorphic form, but of type $(\nu_2, \nu_1)$.

Any $\phi$ has a Fourier expansion with double indices
$$
\phi(\tau) = \sum_{g\in \Gamma_{\infty}^{2}\backslash \Gamma^{2}}\sum_{n_{1} \geq 1} \sum_{n_{2}\geq 1} \hat{\phi}_{n_{1}, n_{2}}\left(\begin{pmatrix} g & \\ & 1 \end{pmatrix}z\right)
$$
where $\Gamma_{\infty}^{2}$, $\Gamma^{2}$ are the subgroups of $\GL(3, \mathbb{Z})$ defined as
$$
\Gamma^{2} = \left\{\begin{pmatrix}
* & * & \\ * & * & \\ & & 1
\end{pmatrix} \in \GL(2, \mathbb{Z})\right\}, \Gamma_{\infty}^{2} = \Gamma^{2} \cap \left\{\begin{pmatrix}
1 & * & * \\ & 1 & * \\ & & 1
\end{pmatrix} \in \GL(2, \mathbb{Z})\right\}
$$
and $\hat{\phi}_{n_{1}, n_{2}}(z)$ is 
$$
\hat{\phi}_{n_{1}, n_{2}}(z) = \int_{0}^{1}\int_{0}^{1} \int_{0}^{1} \phi(xz) e^{-2\pi i (n_{1}x_{1} + n_{2}x_{2})} dx
$$
where
$$
x = \begin{pmatrix}
1 & x_{2} & x_{3} \\ & 1 & x_{1} \\ & & 1
\end{pmatrix}
$$
and $dx = dx_{1}dx_{2}dx_{3}$.
One can check that $\hat{\phi}_{n_{1}, n{2}}(z)$ is a rapidly decreasing Whittaker function, and multiplicity one theorem gives us that it should be a multiple of (suitable modification of) $W_{\nu_{1}, \nu_{2}}(y_{1}, y_{2})$, that is
$$
\phi(\tau) = \sum_{g\in \Gamma_{\infty}^{2}\backslash \Gamma^{2}}\sum_{n_{1} \geq 1} \sum_{n_{2}\geq 1} \frac{a_{n_{1}, n_{2}}}{n_{1}n_{2}} W_{1, 1}^{(\nu_1, \nu_2)}\left(\begin{pmatrix}
n_{1}n_{2} & & \\ & n_{1} & \\ & & 1
\end{pmatrix} g\tau \right).
$$
Here
$$
W_{1, 1}^{(\nu_{1}, \nu_{2})}(\tau) = W_{\nu_{1},\nu_{2}}(y_{1}, y_{2}) e(x_{1} + x_{2}).
$$
We call $\{a_{n_{1}, n_{2}}\}$ the matrix of Fourier coefficients of $\phi$.
From this, we define the corresponding $L$-function as
$$
L(s,\phi) = \sum_{n \geq 1} \frac{a_{1, n}}{n^{s}}.
$$
As we expect, this function admits an analytic continuation and satisfies certain functional equation.
\begin{theorem}[$L$-function of an automorphic form  on $\GL(3, \mathbb{R})$]
The $L$-function $L(s, \phi)$ of an $\GL(3, \mathbb{R})$ automorphic form $\phi$ admits an analytic continuation for all $\mathbb{C}$ and satisfies the functional equation
$$
\Phi(s)L(s, \phi) = \tilde{\Phi}(1-s)L(1-s, \tilde{\phi})
$$
where $\Phi(s), \tilde{\Phi}(s)$ are Gamma factors
\begin{align*}
    \Phi(s) &= \pi^{-\frac{3s}{2}} \Gamma\left(\frac{s -\alpha}{2}\right)
    \Gamma\left(\frac{s -\beta}{2}\right)
    \Gamma\left(\frac{s - \gamma}{2}\right) \\
    \tilde{\Phi}(s) &= \pi^{-\frac{3s}{2}} \Gamma\left(\frac{s + \alpha}{2}\right)
    \Gamma\left(\frac{s + \beta}{2}\right)
    \Gamma\left(\frac{s + \gamma}{2}\right)
\end{align*}
and $L(s, \tilde{\phi})$ is the $L$-function of the dual automorphic form which equals
$$
L(s, \tilde{\phi}) = \sum_{n \geq 1} \frac{a_{n, 1}}{n^{s}}.
$$
\end{theorem}
It is also possible to define Hecke operators on the space of $\GL(3, \mathbb{R})$ automorphic forms.
We define them via double cosets, and the ring of Hecke operators became commutative.
Note that, for each $n\geq 1$, there are \emph{two} Hecke operators $T_{n}, S_{n}$, where 
\begin{definition}[Hecke operators]
Let $\mathcal{H} = \mathbb{Z}[\Gamma \backslash G / \Gamma]$ be a $\mathbb{Z}$-algebra of double cosets where $G = \GL(3, \mathbb{R})$ and $\Gamma = \GL(3, \mathbb{Z})$, which is called \emph{Hecke algebra}.
It decomposes as a (internal) tensor product
$$
\mathcal{H} = \bigotimes_{p} \mathcal{H}_{p}
$$
where $\mathcal{H}_{p}$ is a subalgebra of $\mathcal{H}$ corresponding to the double cosets whose elementary divisors are powers of a given prime $p$.
For each prime $p$, $\mathcal{H}_{p}$ is generated by three elements
\begin{align*}
    T_{p} := \Gamma \begin{pmatrix}
    p & & \\ & 1 & \\ & & 1
    \end{pmatrix} \Gamma, \quad
    S_{p} := \Gamma \begin{pmatrix}
    p & & \\ & p & \\ & & 1
    \end{pmatrix} \Gamma, \quad
    R_{p} := \Gamma \begin{pmatrix}
    p & & \\ & p & \\ & & p
    \end{pmatrix} \Gamma.
\end{align*}
The whole $\mathcal{H}$ is generated by the operators $T_{n}, S_{n}, R_{n}$ where 
\begin{align*}
    T_{n} &:= \sum_{n_{0}^{3} n_{1}^{2} n_{2} = n} \Gamma \begin{pmatrix}
    n_{0}n_{1}n_{2} & & \\ & n_{0}n_{1} & \\ & & 1
    \end{pmatrix} \Gamma \\
    S_{n} &:= \sum_{n_{0}^{3} n_{1}^{2} n_{2} = n} \Gamma \begin{pmatrix}
    n_{0}^{2}n_{1}^{2}n_{2} & & \\ & n_{0}^{2}n_{1}n_{2} & \\ & & n_{0}^{2}n_{1}
    \end{pmatrix} \Gamma \\
    R_{n} &:= \Gamma \begin{pmatrix}
    n & & \\ & n & \\ & & n
    \end{pmatrix} \Gamma
\end{align*}
which satisfies certain relations given as the formal power series
$$
\sum_{n \geq 1} \frac{T_{n}}{n^{s}} = \prod_{p} \frac{1}{1 - T_{p}\cdot p^{-s} +S_{p}\cdot p^{1-2s} - R_{p}\cdot p^{3-3s}}.
$$
If $\phi$ is an automorphic form on $\GL(3, \mathbb{R})$, then the action of Hecke algebra on the form is defined as 
$$
(\phi | \Gamma \alpha \Gamma)(\tau) := \sum_{i} \phi(\alpha_{i} \tau)
$$
where $\alpha_{i}$'s are the representatives of the double coset $\Gamma \alpha \Gamma$, i.e. $\Gamma \alpha \Gamma = \cup_{i} \Gamma\alpha_{i}$.
\end{definition}
Note that the Hecke operators also commutes with the differential operators $\Delta_{1}$ and $\Delta_{2}$, so the space of automorphic forms of type $(\nu_{1}, \nu_{2})$ has a basis consisting of simultaneous eigenforms for all Hecke operators.
Also, the coefficients of $\phi|T_{n}$ and $\phi|S_{n}$ can be expressed as certain sums of coefficients of $\phi$ - see the equations (9.8) and (9.9) in \cite{bump2006automorphic}.

In $\GL(2)$, $L$-function attached to an automorphic form admits an Euler product if and only if the form is Hecke eigenform, and the local factors has a form of $P_{p}(p^{-s})^{-1}$, where $P_{p}(x)$ is a polynomial of degree 2.
The same thing also holds for $\GL(3, \mathbb{R})$, where the local factors are inverses of cubic polynomials in $p^{-s}$.
\begin{theorem}[Euler product of $L$-function]
If $\phi$ is a normalized Hecke eigenform on $\GL(3, \mathbb{R})$ with matrix coefficients $\{a_{n_{1}, n_{2}}\}$, then its $L$-function has an Euler product
$$
L(s,\phi) = \prod_{p} \frac{1}{1 - a_{1, p}p^{-s} + a_{p, 1}p^{-2s} - p^{-3s}}.
$$
\end{theorem}
% Two specific generators are given as follows\footnote{We follow the notation in \cite{bump2006automorphic}, which has an opposite sign compared to Laplacian for $\GL_{2}$.}:
% \begin{align*}
%     \Delta_{1} &= y_{1}^{2} \left(\dxx{1} + \dyy{1} \right) + y_{2}^{2}\left( \dxx{2} + \dyy{2}\right) - y_{1}y_{2} \ddyy{1}{2} \\
%     &+y_{1}^{2}(x_{1}^{2} + y_{2}^{2})\dxx{3} + 2y_{1}^{2}x_{2}\ddxx{1}{3} \\
%     \Delta_{2} &= -y_{1}^{2}y_{2}\frac{\pd^{3}}{\py{1}^{2} \py{2}} + y_{1}y_{2}^{2} \frac{\pd^{3}}{\py{1} \py{2}^{2}} - y_{1}^{3} y_{2}^{2} \frac{\pd^{3}}{\px{3}^{2} \py{1}} + y_{1}y_{2}^{2} \frac{\pd^{3}}{\px{2}^{2} \py{1}} \\
%     & -2y_{1}^{2} y_{2} x_{2} \frac{\pd^{3}}{\px{1} \px{3} \py{2}} + (-x_{2}^{2} + y_{2}^{2})y_{1}^{2}y_{2} \frac{\pd^{3}}{\px{3}^{2} \py{2}} - y_{1}^{2} y_{2} \frac{\pd^{3}}{\px{1}^{2} \py{2}} \\
%     & + 2y_{1}^{2}y_{2}^{2} \frac{\pd^{3}}{\px{1}\px{2}\px{3}} + 2y_{1}^{2}y_{2}x_{2} \frac{\pd^{3}}{\px{2}\px{3}^{2}} + 2y_{1}^{2}x_{2}\frac{\pd^{2}}{\px{1}\px{3}} + y_{1}^{2}(x_{1}^{2} + y_{1}^{2}) \dxx{3} \\
%     & + y_{1}^{2} \left(\dxx{1} + \dyy{1}\right) - y_{2}^{2}\left( \dxx{2} + \dyy{2}\right).
% \end{align*}



\subsection{Symmetric square lifting by Gelbart-Jacquet}

Let $f$ be a level 1 Maass cusp form (of weight 0) on $\GL(2,\mathbb{R})$ with eigenvalue $\lambda = \nu(1-\nu)$ which is also a normalized eigenform.
Let $\{a_{n}\}_{n\geq 1}$ be Fourier coefficients of $f$.
Consider the Rankin-Selberg $L$-function of $f \times f$, i.e.
\begin{align*}
L(s, f \times f) = \zeta(2s)\sum_{n\geq 1} \frac{|a_{n}|^{2}}{n^{s}}
= \zeta(2s)\prod_{p}  \frac{1}{(1-\alpha_{p}^{2}p^{-s})(1-p^{-s})^{2}(1-\beta_{p}^{2}p^{-s})}
\end{align*}
where $a_{p} = \alpha_{p} + \beta_{p}, \alpha_{p}\beta_{p} =1$.
By the theory of Rankin-Selberg convolution, the $L$-function admits an analytic continuation to $\mathbb{C}$ with functional equation
$$
\Lambda(s, f \times f) = G(s)L(s, f\times f) = \Lambda(1-s, f\times f)
$$
where $G(s)$ is the Gamma factor
$$
G(s) := \pi^{-2s}\Gamma\left(\frac{s + 1 - 2\nu}{2}\right)\Gamma\left(\frac{s}{2}\right)^{2} \Gamma\left(\frac{s - 1 + 2\nu}{2}\right)
$$
If we divide the functional equation of $\zeta(s)$ from both sides, we get
\begin{align*}
    &\pi^{-\frac{3s}{2}} \Gamma\left(\frac{s + 1 - 2\nu}{2}\right)\Gamma\left(\frac{s}{2}\right) \Gamma\left(\frac{s - 1 + 2\nu}{2}\right) \frac{L(s, f\times f)}{\zeta(s)} \\
    &=\pi^{-\frac{3(1-s)}{2}} \Gamma\left(\frac{(1-s) + 1 - 2\nu}{2}\right)\Gamma\left(\frac{1-s}{2}\right) \Gamma\left(\frac{(1-s) - 1 + 2\nu}{2}\right) \frac{L(1-s, f\times f)}{\zeta(1-s)}
\end{align*}
One may expect that the degree 3 $L$-function $L(s, f\times f) / \zeta(s)$ is attached to certain self-dual $\GL(3)$ Maass eigenform of type $(2\nu/3, 2\nu/3)$.
Indeed, the twisted $L$-functions by Dirichlet characters admits Euler product, satisfies EBV (entire and bounded in vertical strips) conditions and certain functional equation, 
so the $\GL(3)$ converse theorem gives the desired result.
The detailed proof can be found in Chapter 7 of Goldfeld's book \cite{goldfeld2006automorphic},
where the proof of EBV condition is based on Shimura's brilliant idea that considers Rankin-Selberg product of $f$ 
with a theta function (see also \cite{shimura1975holomorphy}).
Let's write the corresponding $\GL(3)$ Maass form as $\phi = \phi(\tau)$.
From $L(s, \phi)\zeta(s) = L(s, f\times f)$, the Fourier coefficients matrix $\{b_{n_{1}, n_{2}}\}$ of $\phi$ and the Fourier coefficients of $f(z)$ should be related as
$$
a_{n}^{2} = \sum_{d|n} b_{d, 1} \Longleftrightarrow b_{n, 1} = \sum_{d|n} \mu(d) a_{n/d}^{2}.
$$

Now we will interpret the situation in the context of representation theory.
Let $\pi = \otimes_{v} \pi_{v}$ be an automorphic representation of $\GL_{2}(\mathbb{A})$, and let $\varphi_{v}$ be the 2-dimensional representations of the Weil-Deligne group $W_{v}:=W_{F_{v}}$ of $F_{v}$ attached to $\pi_{v}$ via Local Langlands correspondence.
The symmetric square representation of $\GL(2, \mathbb{C})$
\begin{align*}
\Sym^{2}: \GL(2, \mathbb{C}) \to \GL(3, \mathbb{C}), \qquad
\begin{pmatrix}
a & b\\ c & d \end{pmatrix} \mapsto \begin{pmatrix}
a^{2} & 2ab & b^{2}\\ ac & ad + bc & bd\\ c^{2} & 2cd & d^{2} \end{pmatrix}
\end{align*}
gives a 3-dimensional representation $\Sym^{2}(\varphi_{v}):=\Sym^{2}\circ\varphi$ of $W_{v}$, which should corresponds to an irreducible admissible representation of $\GL(3, F_{v})$ via Local Langlands correspondence again.
Global Langlands correspondences predicts that the representation $\Sym^{2}(\pi) := \otimes_{v} \Sym^{2}(\pi_{v})$ is an automorphic represenetation of $\GL(3, \mathbb{A})$, which is proven by Gelbart-Jacquet.

\begin{theorem}[Gelbart-Jacquet, \cite{gelbart1976relation}]
Let $F$ be a number field and $\pi$ be a cuspidal automorphic representation of $\GL(2, \mathbb{A})$ with $\mathbb{A} = \mathbb{A}_{F}$.
Then $\Sym^{2}(\pi)$ is an automorphic representation of $\GL(2, \mathbb{A})$.
% It is cuspidal unless $\pi$ is monomial, i.e. $\pi \not\simeq \pi \otimes \eta$ where $\eta\neq 1$ is a gr\"ossencharacter of $F$.
\end{theorem}

% Here we give a sketch of proof.
% Basically, the proof uses converse theorem for $\GL_{3}$ proven by 

In \cite{ramakrishnan2014exercise}, Ramakrishnan proved the following converse of the Gelbart-Jacquet, by using $L$-functions.
\begin{theorem}[Ramakrishnan, \cite{ramakrishnan2014exercise}]
Let $F$ be a number field and $\Pi$ be a cuspidal automorphic representation of $GL(3, \mathbb{A}_{F})$, which is self-dual.
Then, up to quadratic twist, it can be realized as an adjoint of a $GL(2, \mathbb{A}_{F})$ automorphic representation.
More precisely, there exists an automorphic form $\pi$ of $\GL(2, \mathbb{A}_{F})$ and a gr\"ossencharacter $\eta$ of $F$ with $\eta^{2} = 1$ such that
$$
\Pi = \Ad(\pi) \otimes \eta
$$
where $\Ad(\pi) = \Sym^{2}(\pi) \otimes \omega_{\pi}^{-1}$.
\end{theorem}
\subsection{Higher symmetric power}
Since symmetric power map $\Sym^{r}:\GL(2) \to \GL(r+1)$ is defined for arbitrary power $r$, we expect the presence of lifting from $\GL(2)$ automorphic representations to $\GL(r+1)$ automorphic representations.
Until now, this is proved for $r = 3, 4$ cases.

\begin{theorem}[Kim-Shahidi, \cite{kim2002functorial}]
\label{cubicpow}
Let $F$ be a number field and $\pi$ be a cuspidal automorphic representation of $\GL(2, \mathbb{A})$ with $\mathbb{A} = \mathbb{A}_{F}$.
Then $\Sym^{3}(\pi)$ is an automorphic representation of $\GL(4, \mathbb{A})$.
$\Sym^{3}(\pi)$ is cuspidal unless $\pi$ is either dihedral or tetrahedral type.
In particular, if $F = \mathbb{Q}$ and $\pi$ is the automorphic representation attached to nondihedral modular form of level $\geq 2$, then $\Sym^{3}(f)$ is cuspidal.
\end{theorem}


\begin{theorem}[Kim, \cite{kim2003functoriality}]
\label{quarticpow}
Let $F$ be a number field and $\pi$ be a cuspidal automorphic representation of $\GL(2, \mathbb{A})$ with $\mathbb{A} = \mathbb{A}_{F}$.
Then $\Sym^{4}(\pi)$ is an automorphic representation of $\GL(5, \mathbb{A})$.
If $\Sym^{3}(\pi)$ is cuspidal, then $\Sym^{4}(\pi)$ is either cuspidal or induced from cuspidal representation of $\GL(2, \mathbb{A})$ and $\GL(3, \mathbb{A})$.
\end{theorem}

To prove Theorem \ref{cubicpow}, Kim and Shahidi first proved the functoriality $\GL(2) \times \GL(3) \to \GL(6)$, i.e. existence of an automorphic representation $\pi_{1} \boxtimes \pi_{2}$ for $\GL(2)$ automorphic representation $\pi_{1}$ and $\GL(3)$ automorphic representation $\pi_{2}$.
Then they obtained the result by applying it for $\pi_{1} = \pi$ and $\pi_{2}  = \Ad(\pi_{1})$, where $\Ad$ is the automorphic representation of $\GL(3, \mathbb{A})$ obtained with adjoint representation $\Ad: \GL(2) \to \PGL(2) \to \GL(3)$.
Note that $\Sym^{2}(\pi) = \Ad(\pi) \otimes \omega_{\pi}$, where $\omega_{\pi}$ is the central character of $\pi$. 

For Theorem \ref{quarticpow}, Kim first proved exterior square lifting for $\GL(4)$, which corresponds to the map $\wedge^{2}:\GL(4, \mathbb{C}) \to \GL(6, \mathbb{C})$.
Then he obtained the result on the fourth power by applying exterior square to $\Sym^{3}(\pi) \otimes \omega_{\pi}^{-1}$, showing that 
$$
\wedge^{2}(\Sym^{3}(\pi) \otimes \omega_{\pi}^{-1}) = (\Sym^{4}(\pi) \otimes \omega_{\pi}^{-1}) \boxplus \omega_{\pi}.
$$

Recently, it is proved that the functoriality holds for arbitrary power when $\pi$ is a \emph{regular algebraic cuspidal} representation, which corresponds to twists of cuspidal modular forms.
\begin{theorem}[Newton-Thorne \cite{newton2021symmetric, newton2021symmetric2}]
Let $\pi$ be a regular algebraic cuspidal representation of $\GL(2, \mathbb{A}_{\mathbb{Q}})$ of level 1, or without complex multiplication.
For any $n\geq 1$, $\mathrm{Sym}^{n}(f)$ is a regular algebraic cuspidal representation of $\GL(n + 1, \mathbb{A})$.
\end{theorem}


\subsection{Ramanujan's conjecture and Selberg's 1/4 conjecture}

The importance of symmetric power lifting is due to it's application on Ramanujan's conjecture and Selberg's eigenvalue conjecture.

In 1916, Ramanujan conjectured that the Fourier coefficients $\tau(n)$ of disciminant function
$$
\Delta(z) = q\prod_{n\geq 1}(1 - q^{n})^{24} = \sum_{n\geq 1}\tau(n)q^{n}, \qquad q = e^{2\pi i z}
$$
(which is a weight 12 and level 1 holomorphic cusp form) satisfy
\begin{enumerate}
    \item $\tau(mn) = \tau(m)\tau(n)$ for $(m, n) =1$ (i.e. $\tau$ is multiplicative),
    \item $\tau(p^{k+1}) = \tau(p)\tau(p^{k}) - p^{11}\tau(p^{k-1})$ for prime $p$ and $k\geq 1$,
    \item $|\tau(p)|\leq p^{11/2}$.
\end{enumerate}
The first two statements were proved by Mordell in 1917, and Hecke showed that 
similar phenomena can be found in other modular forms, by defining so-called \emph{Hecke operators}.
Third statement remained as a conjecture until 1974 when it was proved by Deligne
as a consequence of his proof of Weil's conjecture.
Weil's conjecture, which is an analogue of Riemann's hypothesis for varieties over finite fields,
states that the \emph{Zeta function} of a curve is a rational function which can be written as
an alternating product of certain polynomials, and the zeros of each polynomial have the same norm.

In \cite{satake1966spherical}, Satake formulated the Ramanujan's conjecture
in terms of automorphic representations.
He conjectured that, for any automorphic representation $\pi = \otimes_{v}\pi_{v}$ of $\GL(2)$,
the local components $\pi_{v}$ are \emph{tempered}, i.e. the matrix coefficients
of $\pi_{v}$ are in $L^{2+\epsilon}(\GL(2, F_{v}))$ for all $\epsilon > 0$.
For example, let's consider the automorphic representation $\pi = \pi_{\Delta}$ associated
to the disciminant function $\Delta$.
For $p\neq 2, 3$, $\pi_{p}$ is an unramified principal series representation, which is 
\begin{align*}
\pi(\chi_{1}, \chi_{2}) &= \mathrm{Ind}_{B(\mathbb{Q}_{p})}^{\GL(2, \mathbb{Q}_{p})}(\chi_{1}\boxtimes \chi_{2})\\ 
&= \left\{f: \GL(2, F_{v})\to\mathbb{C}:f(bg) = \bigg|\frac{b_{1}}{b_{2}}\bigg|^{\frac{1}{2}}\chi_{1}(b_{1})\chi_{2}(b_{2})f(g),b = \begin{pmatrix}b_{1}&*\\&b_{2}\end{pmatrix}\right\}
\end{align*}
where $\chi_{1}, \chi_{2}: \mathbb{Q}_{p}^{\times}\to \mathbb{C}^{\times}$ are unramified characters ($\mathbb{Z}_{p}^{\times}\subseteq \ker\chi_{i}$)
with $\chi_{i}(p) = \alpha_{i}$ where $\alpha_{1}, \alpha_{2}$ are zeros of the polynomial $x^{2} - \tau(p)x + p^{11}$.


Several authors found counterexamples in some groups like $\mathrm{U}(2, 1)$ and $\mathrm{Sp}(4)$
by constructing corresponding automorphic forms that are non-tempered almost everywhere.


\begin{conjecture}[Selberg's 1/4 conjecture]
For any Maass form on a congruence subgroup $\Gamma \subseteq \mathrm{SL}(2, \mathbb{Z})$, its eigenvalue is at least $1/4$.
\end{conjecture}
It is known that the conjecture is false for non-congruence subgroups (See \cite{sarnak1995selberg} for Sarnak's argument). 
Also, it is widely believed that the Maass forms with eigenvalue $1/4$ are \emph{algebraic}, in a sense that they comes from \emph{even} 2-dimensional Galois representations.
Selberg himself proved a weaker bound $3/16$ for $\Gamma = \Gamma(N)$ in \cite{selberg1965estimation}, 

\begin{proposition}
\label{powerselberg}
Assume that symmetric power lifting holds for arbitrary power, i.e. for any cuspidal automorphic representation $\pi$ on $\GL(2,\mathbb{A})$,  $\Sym^{r}(\pi)$ is an automorphic representation of $\GL(r + 1, \mathbb{A})$ for any $r$.
Then the Ramanujan's conjecture and Selberg's conjecture are true.
\end{proposition}
\begin{proof}
By Jacquet-Shalika [REFERENCE], it is proven that the Satake parameters of automorphic forms of $\GL(n, \mathbb{A})$ satisfy 
$$
q_{v}^{-1/2} < |\alpha_{i,v}| < q_{v}^{1/2}
$$
for all $1\leq i\leq n$ and unramified places $v$ (including archimedean places).
Now, assume that symmetric power lifting holds for arbitrary power. 
If $\Pi = \otimes_{v}\Pi_{v} = \Sym^{r}(\pi)$ is the corresponding representation, then the Satake parameters at place $v$ are given as
$$
\begin{pmatrix}
\alpha_{1, v}^{r} & & & & \\ 
& \alpha_{1, v}^{r-1}\alpha_{2, v} & & & \\ 
& & \ddots & & & \\
& & & \alpha_{1, v}\alpha_{2, v}^{r-1} & \\
& & & & \alpha_{2, v}^{r}
\end{pmatrix}
$$
and Jacquet-Shalika's bound gives 
$$
q_{v}^{-1/2} < |\alpha_{i, v}^{r}| < q_{v}^{1/2} \Longleftrightarrow q_{v}^{-1/2r} < |\alpha_{i, v}| < q_{v}^{1/2r}
$$
for all $r\geq 1$. Now taking the limit $r\to \infty$ proves both conjecture.
% For details, see \cite{sarnak2005notes}.
\end{proof}

Combined with Theorem \ref{quarticpow}, Proposition \ref{powerselberg} gives the current best bound for the Selberg's conjecture.
\begin{corollary}
Eigenvalus of Maass forms on a congruence subgroup is at least 
$$
\frac{1}{4} - \left(\frac{7}{64}\right)^{2} = \frac{975}{4096} \approx 0.238037\dots
$$
\end{corollary}

The current best known bound for $\GL(2, \mathbb{A}_{\mathbb{Q}})$ is the following result from Kim and Sarnak \cite{kim2003refined}.

\begin{theorem}
Let $\pi = \otimes_{p}'\pi_{p}$ be an automorphic cusp form on $\GL(2, \mathbb{A}_{\mathbb{Q}})$.
If $\pi$ is unramified at $p$, then its Satake parameter $\alpha_{1, p}, \alpha_{2, p}$ satisfies
$$
|\log_{p}|\alpha_{j, p}|| \leq \frac{7}{64},\,\, j = 1, 2.
$$
If $\pi_{\infty}$ is unramified, then
$$
|\Re(\mu_{j, \infty})|\leq \frac{7}{64},\,\,j = 1, 2.
$$
\end{theorem}
In terms of classical Masss forms, this implies: if $f$ is a classical Maass form,
on a congruence subgroup $\Gamma \subseteq \SL(2, \mathbb{Z})$,
then its' Fourier coefficient satisfies
$$
|a_{p}| \leq p^{7/64} + p^{-7/64}
$$
and its eigenvalue is at least
$$
\frac{1}{4} - \left(\frac{7}{64}\right)^{2} = \frac{975}{4096} \approx 0.238037\cdots.
$$
\begin{proof}
When $\Pi$ is a non self-dual $\GL(n)$ automorphic form, it is known that the symmetric square $L$-function $L(s, \Pi, \Sym^{2})$
satisfies desired analytic properties by Kim and Shahidi \cite{kim1999langlands,shahidi1988ramanujan}.
Using this, the authors proved that the Satake parameters of unramified $\Pi_{p}$ satisfy
$$
|\log_{p}|\alpha_{j, p}|| \leq \frac{1}{2} - \frac{1}{\frac{n(n+1)}{2} + 1},\,\, 1\leq j \leq n.
$$
for Satake parameters of unramified $\Pi_{p}$ with $p <\infty$, and
$$
|\Re(\mu_{j, \infty})|\leq \frac{1}{2} - \frac{1}{\frac{n(n+1)}{2} + 1},\,\, 1\leq j \leq n.
$$
for Satake parameters $\mu_{j, \infty}$ when $\Pi_{\infty}$ is unramified.
By Theorem \ref{quarticpow}, $\Pi = \Sym^{4}(\pi)$ is automorphic 
and applying the above bound for $\Pi$ gives the desired bound.
\end{proof}


\subsection{Sato-Tate conjecture}

Another consequence of symmetric power lifting is Sato-Tate conjecture.
It is a conjecture about equidistribution of \emph{Frobenius angle}, which we are going to explain briefly.
Let $E$ be an elliptic curve over $\mathbb{Q}$ without CM.
For every prime $p$ of good reduction for $E$, the number of points over $\mathbb{F}_{p}$ satisfies an inequality
$$
|\#E(\mathbb{F}_{p}) - (p + 1)| \leq 2\sqrt{p}
$$
which was proven by Hasse in 1933. The quantity $t_{p} : = (p + 1) - \#E(\mathbb{F}_{p})$ is called
\emph{trace}, since it is actually the trace of $\rho_{E, \ell}(\mathrm{Frob}_{p})$, where
$\rho_{E, \ell}$ is a (family of) $\ell$-adic Galois representation attached to $E$ and $\mathrm{Frob}_{p} \in \mathrm{Gal}(\overline{\mathbb{Q}} / \mathbb{Q})$
is a Frobenius automorphism.
Because of Hasse's inequality, we can write $t_{p}$ as $t_{p} = 2\sqrt{p}\cos\theta_{p}$ for some $\theta_{p} \in [0, \pi]$,
where we call $\theta_{p}$ as \emph{Frobenius angle}.
Sato-Tate conjecture states that the Frobenius angle is equidistributed over $[0, \pi]$,
and it was proven in 2011 by Barnet-Lamb, Geraghty, Harris, and Taylor \cite{barnet2011family}.

\begin{theorem}[Sato-Tate conjecture, \cite{harris2010family}]
\label{sato-tate}
Let $E/\mathbb{Q}$ be an elliptic curve without CM.
The sequence of Frobenius angles $\{\theta_{p}\}$ is uniformly distributed on the
interval $[0, \pi]$. In terms of traces $\{t_{p}\}$, for every subinterval $[a, b]$ of $[-2, 2]$,
$$
\lim_{B\to \infty} \frac{\#\{p\leq B\,:\, t_{p} \in [a, b]\}}{\#\{p\leq B\}} = \int_{a}^{b} \frac{1}{2\pi} \sqrt{4 - t^{2}} dt.
$$
\end{theorem}

A key idea for the proof of Theorem \ref{sato-tate} is the following equivalence between 
equidistribution and holomorphicity \& nonzeroness of a $L$-function.
Let $G$ be a compact group and $X = \mathrm{conj}(G)$ be a space of
conjugacy classes of $G$. Let $K$ be a number field, and $P = (\mathfrak{p}_{1}, \mathfrak{p}_{2}, \dots)$
be a sequence of all but finitely many primes of $K$ ordered by norm.
Let $(x_{\mathfrak{p}})_{\mathfrak{p}\in P}$ be a sequence in $X$ indexed by $P$, and for each
irreducible representation $\rho: G \to \GL(d, \mathbb{C})$, define the $L$-function
$$
L(s, \rho) := \prod_{\mathfrak{p}\in P} \det(1 - \rho(x_{\mathfrak{p}})N(\mathfrak{p})^{-s})^{-1},
$$
for $s\in \mathbb{C}$ with $\Re s > 1$.

\begin{theorem}
\label{equidistribution}
Let $G$ and $(x_{\mathfrak{p}})$ as above, and assume that $L(s, \rho)$ is meromorphic on $\Re s \geq 1$
with nozeros or poles except possibly at $s=1$, for every irreducible representation $\rho$ of $G$.
The sequence $(x_{\mathfrak{p}})$ is equidistributed if and only if for each $\rho \neq 1$, the $L$-function
$L(\rho, s)$ extends analytically to a function that is holomorphic and nonvanishing on $\Re s \geq 1$.
\end{theorem}
\begin{proof}
See Theorem 2.3 of \cite{fite2015equidistribution}.
\end{proof}

Now let $G = \mathrm{SU}(2)$ (compact group of complex 2 by 2 unitary matrices of determinant 1)
and $K = \mathbb{Q}$. The irreducible representations of $\mathrm{SU}(2)$ are of the form $\rho_{m} = \mathrm{Sym}^{m}\rho_{1}$
where $\rho_{1}: \mathrm{SU}(2) \hookrightarrow \GL(2, \mathbb{C})$ is the representation given by inclusion.
In this case, each element of $X = \mathrm{conj}(\mathrm{SU}(2))$ has a representative of the form
$$
\begin{pmatrix}
    e^{i\theta} & 0 \\ 0 & e^{-i\theta}
\end{pmatrix}
$$
with $\theta \in [0, \pi]$, and the $L$-function of $\rho_{m}$ can be written as
$$
L(s, \rho_{m}) = \prod_{p\nmid N} \det(1 - \rho_{m}(x_{p})p^{-s})^{-1} = \prod_{p\nmid N}\prod_{k=0}^{m} (1 - e^{i(m - 2k)\theta_{p}}p^{-s})^{-1}
$$
where $\theta_{p}$ is Frobenius angle and $N$ is a conductor of elliptic curve $E$.
If we set $\alpha_{p} := e^{i\theta_{p}}p^{1/2}$ and
$$
L_{m}^{1}(s):= \prod_{p\nmid N}\prod_{k=0}^{m} (1 - \alpha_{p}^{m-k}\bar{\alpha}_{p}^{k}p^{-s})^{-1},
$$
then we have $L(s, \rho_{m}) = L_{m}^{1}(s - m/2)$.
By Theorem \ref{equidistribution}, Sato-Tate theorem (for non-CM elliptic curve) would follow from
holomorphicity and nonzeroness of $L_{m}^{1}(s - m/2)$ for $\Re s \geq \frac{m}{2} + 1$.
Now, the celebrated modularity theorem by several mathematicians (Wiles, Taylor, Brueil, Conrad, \dots)
states that one can find a modular form $f$ whose $L$-function $L(s, f)$ coincides with
the Hasse-Weil $L$-function $L(s, E)$ attached to $E$, and both coincides with $L_{1}^{1}(s)$ up to
finitely many Euler factors at bad primes.
It is easy to show holomorphicity and nonvanishing property of $L(s, f)$, and
one essentially have $L_{m}^{1}(s) = L(s, \mathrm{Sym}^{m}f)$.
Hence, Sato-Tate conjecture reduces to some analytic properties of $L(s, \mathrm{Sym}^{m}f)$.

Sato-Tate conjecture is also true for CM elliptic curves, but with different measures.
Actually it is much easier to prove for such cases
since the corresponding \emph{Sato-Tate group} is $U(1)$ (embedded in $\GL(2, \mathbb{C})$ via
$u\mapsto \left(\begin{smallmatrix}
    u & 0 \\ 0 & \bar{u}
\end{smallmatrix}\right)$) or $N(U(1))$ (normalizer of $U(1)$ in $GL(2, \mathbb{C})$) and $U(1)$ is abelian,
so all the irreducible representations are 1-dimensional.
Also, it is possible to define \emph{a} Sato-Tate group $\mathrm{ST}(A)$ and formulate the analogous conjecture
for abelian varieties $A$: the normalized images of the Frobenius elements in a Sato-Tate group
is equidistributed with respect to the pushforward of the Haar measure of $\mathrm{ST}(A)$ to its
space of conjugacy classes.
However, we don't even know whether \emph{the} Sato-Tate group is well-defined or not for $g\geq 2$, and only 
few examples of abelian varieties are known to be that the conjecture holds (e.g. \cite{fite2014sato}).
One can find more details in Sutherland's excellent survey paper \cite{sutherland2013sato}.