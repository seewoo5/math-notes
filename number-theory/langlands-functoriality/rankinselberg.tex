\newpage
\section{Rankin-Selberg product}

\subsection{Rankin-Selberg convolution of modular forms}

Let $f, g$ be two holomorphic cusp forms of weight $k$ and level 1.
Assume that two forms have Fourier expansions
$$
    f(z) = \sum_{n\geq 1} a_{n}e^{2\pi i n}, \quad g(z) = \sum_{n\geq 1} b_{n}e^{2\pi i n}.
$$
If $f,g$ are Hecke eigenforms, then their $L$-functions admit euler products as
\begin{align*}
    L(s, f) &= \sum_{n\geq 1}\frac{a_{n}}{n^{s}} = \prod_{p} \frac{1}{(1 - \alpha_{p}p^{-s})(1 - \beta_{p}p^{-s})} \\
    L(s, g) &= \sum_{n\geq 1}\frac{b_{n}}{n^{s}} = \prod_{p} \frac{1}{(1 - \alpha_{p}'p^{-s})(1 - \beta_{p}'p^{-s})}
\end{align*}
where $\alpha_{p} + \beta_{p} = a_{p}$, $\alpha_{p}' + \beta_{p}' = b_{p}$, and $\alpha_{p}\beta_{p} = \alpha_{p}'\beta_{p}' = p^{k-1}$ for all $p$.
Rankin (1939) and Selberg (1940) independently studied the \emph{convolution} of two $L$-series attached to $f$ and $g$, which is
\begin{align*}
    L(s, f\times g) &= \sum_{n\geq 1} \frac{a_{n}\overline{b_{n}}}{n^{s}} \\
    &= \prod_{p} \frac{1}{(1 - \alpha_{p}\alpha_{p}'p^{-s})(1 - \alpha_{p}\beta_{p}'p^{-s})(1 - \beta_{p}\alpha_{p}'p^{-s})(1 - \beta_{p}\beta_{p}'p^{-s})}
\end{align*}
and studied its analytic properties.
They proved that the new $L$-function also satisfy similar properties as original $L$-functions $L(s, f)$ and $L(s, g)$: it admits a meromorphic continuation, bounded on vertical strips,
and satisfy a functional equation.

\begin{theorem}[Rankin-Selberg convolution]
Let $f, g$ be two holomorphic cusp eigenforms of weight $k$ on $\Gamma = \mathrm{SL}(2, \mathbb{Z})$.
Let
$$
\Lambda(s, f \times g) = (2\pi)^{-2s}\Gamma(s)\Gamma(s - k + 1)\zeta(2s-2k+2)L(s, f\times g)
$$
be a completed $L$-function.
Then $\Lambda(s, f\times g)$, which is originally defined for large $\Re(s)$, admits a meromorphic continuation to all $s$ except for at most simple poles at $s = k$ and $k-1$.
Also, it satisfies a functional equation
$$
\Lambda(s, f\times g) = \Lambda(2k-1-s, f\times g).
$$
\end{theorem}
Essence of the proof is using real-analytic Eisenstein series with \emph{unfolding trick}.
For $s \in \mathbb{C}$, define a real-analytic Eisenstein series $E_{s}(z)$ as
$$
E_{s}(z):= \sum_{\gamma \in P\backslash \mathrm{SL}(2, \mathbb{Z})} \Im(\gamma z)^{s}
$$
where $P = \left\{\left(\begin{smallmatrix} * & * \\ 0 & * \end{smallmatrix}\right)\right\}\subset \Gamma$ is a standard parabolic subgroup of $\Gamma$.
Clearly, this is a $\Gamma$-invariant function, and it converges for $\Re(s) > 1$. 
Also, using $\Delta y^{s} = s(1-s)y^{s}$, one can show that $E_{s}(z)$ is also a Maass form with eigenvalue $s(1-s)$ (but not a cusp form).
By computing its Fourier expansion, we can see that $E_{s}(z)$, as a function in $s$, satisfies the functional equation 
$$
    \xi(2s)E_{s}(z) = \xi(2 - 2s)E_{1-s}(z)
$$
where $\xi(s)$ is the completed zeta function
$$
    \xi(s) = \pi^{-s/2} \Gamma\left(\frac{s}{2}\right)\zeta(s)
$$
that satisfies $\xi(s) = \xi(1-s)$ for all $s$.
\begin{proposition}
$$    
    \langle f\cdot E_{s}, g\rangle = (4\pi)^{-(s+2k-1)}\Gamma(s + 2k-1)\sum_{n\geq 1} L(s+2k-1, f\times g)
$$
where $\langle -,-\rangle$ is the Petersson inner product.
\end{proposition}
\begin{proof}
The idea is to unfold the integral.
If $\varphi$ is a $P$-invariant function on $\mathfrak{H}$, then Fubini's theorem gives
$$
\int_{\Gamma\backslash \mathfrak{H}} \sum_{\gamma \in P \backslash \Gamma}\varphi(\gamma z) \frac{dxdy}{y^{2}}
=\int_{P \backslash \mathfrak{H}} \varphi(z) \frac{dxdy}{y^{2}}
= \int_{0}^{\infty} \int_{0}^{1} \varphi(z) \frac{dxdy}{y^{2}}
$$
(the fundamental domain of $P\backslash \mathfrak{H}$ is $\{z = x + iy \in \mathfrak{H}\,:\, 0\leq x < 1\}$.)
Once we apply this for $\varphi(z) = y^{s} f(z)\overline{g(z)}y^{2k}$, we get
\begin{align*}
    \langle f\cdot E_{s}, g\rangle &= \int_{0}^{\infty}\int_{0}^{1} y^{s}f(z)\overline{g(z)} y^{2k} \frac{dxdy}{y^{2}} \\
    &= \sum_{m, n \geq 1} a_{m} \overline{b_{n}}y^{s + 2k - 1} e^{-2\pi(m+n)y}  \left(\int_{0}^{1} e^{2\pi i (m - n)x}dx\right)\frac{dy}{y} \\
    &= \sum_{n\geq 1} a_{n}\overline{b_{n}} \int_{0}^{\infty} y^{s + 2k - 1} e^{-4\pi n y} \frac{dy}{y} \\
    &= (4\pi)^{-(s+2k-1)}\Gamma(s + 2k - 1) L(s+2k-1, f\times g).
\end{align*}
\end{proof}

\subsection{Modularity of $\GL(2) \times \GL(2)$}
It is also possible to construct Rankin-Selberg $L$-function attached to two Maass cusp forms with similar properties.
In general, for given automorphic representations $\pi_{1}, \pi_{2}$ on $\GL(2)$,
one can define Rankin-Selberg $L$-function $L(s, \pi_{1}\times \pi_{2})$.
According to the philosophy of Langlands, there should exists a $\GL(4)$ automorphic representation whose $L$-function 
is $L(s, \pi_{1}\times \pi_{2})$.
This is proven by Ramakrishnan in 2000.
\begin{theorem}[Ramakrishnan, \cite{ramakrishnan2000modularity}]
    \label{rankin-selberg-modularity}
    Let $\pi_{1}, \pi_{2}$ be automorphic forms on $\GL(2, \mathbb{A})$.
    Then there exists an automorphic representation $\pi_{1} \boxtimes \pi_{2}$ on $\GL(4, \mathbb{A})$
    whose $L$-function equals the Rankin-Selberg $L$-function, i.e 
    $$
        L(s, \pi_{1}\boxtimes \pi_{2}) = L(s, \pi_{1} \times \pi_{2}).
    $$
\end{theorem}

As a corollary of Theorem \ref{rankin-selberg-modularity}, he proved multiplicity one result for $\SL(2)$.
This was conjectured by Labesse and Langlands before \cite{labesse1979indistinguishability}.
\begin{theorem}[Ramakrishnan, \cite{ramakrishnan2000modularity}]
\label{sl2-multiplicity-one}
Multiplicity one theorem holds for $\SL(2)$. 
More precisely, any smooth irreducible admissible representation of $\SL(2, \mathbb{A})$ occurs with 
multiplicity at most one in the space of cusp forms $L^{2}_{0}(\SL(2, F)\backslash\SL(2, \mathbb{A}_{F}))$.
\end{theorem}
In the context of modular forms, this implies the following.
Let $f, g$ be cusp forms of level $N$ and $M$ respectively.
Assume that, for all but finitely many $p$, we have
$$
    a_{p}^{2} = b_{p}^{2}
$$
where $a_{p}$ (resp. $b_{p}$) is the $p$-th Fourier coefficient of $f$ (resp. $g$).
Then there exists a quadratic Dirichlet character $\chi$ such that
$$
    a_{p} = \chi(p)b_{p}
$$
for all but finitely many $p$.
Also, if $N, M$ are in addition square-free, then $\chi = 1$ and so $f = g$ by strong multiplicity one.

Proof of Theorem \ref{sl2-multiplicity-one} goes as follows. 
In \cite{labesse1979indistinguishability}, the authors proved that the multiplicity one theorem for $\SL(2)$ 
holds if one can show the following.
\begin{theorem}[Ramakrishnan, \cite{ramakrishnan2000modularity}]
If two automorphic representation $\pi, \pi'$ on $\GL(2, \mathbb{A}_{F})$ satisfy
$\Ad(\pi)\simeq \Ad(\pi')$, then $\pi' \simeq \pi \otimes \chi$ for some idele class character $\chi$ of $F$.
Here $\Ad$ is the adjoint lift from $\GL(2)$ to $\GL(3)$ \cite{gelbart1976relation}.
\end{theorem}
He first show this when at least one of $\pi$ or $\pi'$ is \emph{dihedral} (i.e. has a form of $\mathrm{AI}_{K}^{F}(\mu)$ for some 
quadratic extension $K$ of $F$ and idele class character $\mu$ of $K$).
If both $\pi$, $\pi'$ are not dihedral, then he proved that $\pi \boxtimes \pi'$ is not cuspidal by
analyzing poles of $L$-functions at $s = 1$.
which shows that $\pi' \simeq \pi \otimes \chi$ for some $\chi$ by cuspidality criterion that is also proved by him in loc. cit.


\subsection{$\GL(n)\times \GL(n)$}

