
\section{Automorphic induction}

\subsection{From $\GL_{1}/K$ to $\GL_{2}/\mathbb{Q}$}
Let $K$ be a quadratic field (over $\mathbb{Q}$) and $\xi$ be a Hecke character for $K$. 
By Hecke and Maass, it was proven that one can associate $\GL_{2}$ automorphic forms. 
Hecke attached modular forms to Hecke characters for imaginary quadratic fields, and Maass attached Maass forms to those for real quadratic fields.
More precisely, they proved the following:
\begin{theorem}[Hecke]
Let $\xi$ (mod $\mathfrak{m}$) be a primitive Hecke character of $K = \mathbb{Q}(\sqrt{D})$ of discriminant $D < 0$ such that 
$$
\xi((a)) = \left(\frac{a}{|a|}\right)^{u} \quad \mathrm{if}\, a\equiv 1 \,(\mathrm{mod}\,\mathfrak{m})
$$
where $u$ is a non-negative integer. Then 
$$
f(z) = \sum_{\mathfrak{a} \subset \mathcal{O}_{K}}\xi(\mathfrak{a}) (N\mathfrak{a})^{\frac{u}{2}} e^{2\pi (N\mathfrak{a})z} 
$$
is a modular form of weight $k = u + 1$ and level $N = |D|\cdot N\mathfrak{m}$ with Nebentypus $\chi\,(\mathrm{mod}\,N)$, which is a Dirichlet character defined as 
$$
\chi(n) = \chi_{D}(n)\xi((n))\quad n \in \mathbb{Z}.
$$
\end{theorem}

\begin{theorem}[Maass]
Let $K = \mathbb{Q}(\sqrt{D})$ be a real quadratic field of discriminant $D > 0$ and $\xi\,(\mathrm{mod}\,\mathfrak{m})$ a Hecke character such that
$$
\xi((a)) = \frac{a}{|a|}\quad \mathrm{if}\, a \equiv 1\,(\mathrm{mod}\,\mathfrak{m})
$$
or
$$
\xi((a)) = \frac{a'}{|a'|}\quad \mathrm{if}\, a \equiv 1\,(\mathrm{mod}\,\mathfrak{m})
$$
where $a'$ is a conjugate of $a$ over $\mathbb{Q}$. 
Then 
$$
u(z) = \sum_{\mathfrak{a}\in \mathcal{O}_{K}}\xi(\mathfrak{a})y^{\frac{1}{2}}e^{2\pi i (N\mathfrak{a})z}, \quad z = x + yi
$$
is a Maass form of level $N$, eigenvalue $1/4$, and a Nebentypus $\chi\,(\mathrm{mod}\,N)$ for $N = D\cdot N\mathfrak{m}$.
\end{theorem}

Both theorem can be proved using converse theorems for $L$-functions.
By showing that the $L$-function attached to Hecke character $\xi$ satisfies suitable functional equations, converse theorem shows that the $L$ function should coincides with one comes from modular forms or Maass forms.

\subsection{From $\GL_{n}/K$ to $\GL_{rn}/F$}
In view of Langlands functoriality conjecture, Hecke and Maass' results can be considered as a special case when $G = \mathrm{GL}_{1} / K$ and $G' = \GL_{2}/\mathbb{Q}$. 
Automorphic induction, which is a vast generalization of this, is a functoriality from $\GL_{n}/K$ to $\GL_{rn}/F$, where $K/F$ is a degree $r$ extension.
% It is more natural to see Galois counterpart for the functoriality. 
On Galois side, this actually corresponds to the \emph{induction} of Galois representation of $G_{K} = \Gal(\overline{K}/K)$ to $G_{F} = \Gal(\overline{F}/F)$.
In other words, if one has a $\GL_{n}/K$ automorphic representation $\pi$ with corresponding Galois representation $\sigma = \sigma(\pi): G_{K} \to \GL_{n}(\mathbb{C})$, 
then automorphic induction predicts the existence of $\GL_{rn}/F$ automorphic representation $\Pi$ that corresponds to the Galois representation
$$
\Sigma = \mathrm{Ind}_{G_{K}}^{G_{F}} \sigma : G_{F} \to \GL_{rn}(\mathbb{C})
$$
via Langlands correspondence.

\begin{conjecture}[Automorphic induction]
Let $K/F$ be a degree $r$ extension of number fields. 
Let $\pi$ be a $\GL_{n}/K$ automorphic representation. 
Then there exists a $\GL_{rn}/F$ automorphic representation $\Pi = \AI_{F}^{K}(\pi)$ such that 
\begin{enumerate}
    \item the Galois representations 
    $$
    \sigma = \sigma(\pi): G_{K} \to \GL_{n}(\mathbb{C}), \quad \Sigma = \Sigma(\Pi): G_{F} \to \GL_{rn}(\mathbb{C})
    $$
    corresponds to $\pi$ and $\Pi$ via Langlands correspondence satisfies 
    $$
    \Sigma \simeq \mathrm{Ind}_{G_{K}}^{G_{F}} \sigma.
    $$
    \item Local $L$-functions of $\pi$ and $\Pi$ are related as
    $$
    L(s, \Pi_{v}) = \prod_{w|v} L(s, \pi_{w})
    $$
    for all but finitely many $v$. 
\end{enumerate}
\end{conjecture}

This is open in general, but proven to be true for some cases.
% Here are brief reviews of proofs for known cases. 
We give a sketch of proofs for known cases.

\subsubsection{Local automorphic induction (Henniart-Herb)}

\subsubsection{Cyclic Galois extension of prime degree (Arthur-Clozel)}

\begin{theorem}[Arthur-Clozel, \cite{arthur2016simple}]

\end{theorem}
% This includes:
% \begin{itemize}
%     \item Over local fields
%     \item Cyclic Galois extension of prime degree
%     \item Non-normal cubic extension
%     \item Non-normal extensions with solvable Galois closure for certain Hecke characters
%     \item Non-normal quintic extension with non-solvable closure
% \end{itemize}
