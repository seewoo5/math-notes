\chapter{Jacquet-Langlands correspondence}

\section{Basis problem and quaternionic modular forms}

Here's a fundamental question that we can ask about modular forms.

\begin{question}
Find a basis of a space $S_{k}(N)$ for given $k, N$.
\end{question}

For $N = 1$, it is possible to construct a basis using the Eisenstein series $E_{4}$ and $E_{6}$.
In particular, they are algebraically independent over $\mathbb{C}$ (i.e. there's no nonzero polynomial $p(x, y) \in \mathbb{C}[x, y]$)
such that $p(E_{4}, E_{6}) = 0$ identically), and they generate the space of modular forms of level 1 (with trivial characters).
We have $\Delta = \frac{1}{1728}(E_{4}^{3} - E_{6}^{2})$, where $\Delta(z) = e^{2\pi i z}\prod_{n\geq 1}(1 - e^{2\pi i n z})^{24}$ is a 
discriminant function, and multiplying $\Delta$ induces an isomorphism $M_{k}(1)$ to $S_{k+6}(1)$.
With $S_{k}(1) = 0$ for $k = 2, 4, 6, 8, 10$, we can inductively construct the basis of $S_{k}(1)$.
(See \cite{serre2012course} or Zagier's article in \cite{bruinier20081} for details.)

For general $N$, Hecke \cite{Hecke1940AnalytischeAD} conjectured that one can construct a basis using \emph{theta series associated to
orders in certain quaternion algebra}.
Before we give definition of such theta series, let's introduce some notations.
For a prime $p$, let $B_{p}$ be a quaternion algebra over $\mathbb{Q}$ which is ramified at $p$ and infinity.
It known that quaternion algebra over $\mathbb{Q}$ is uniquely determined by the set of primes that ramify.

\cite{eichler1973basis}


\section{Jacquet-Langlands correspondence}

\section{Basis problem by means of Jacquet-Langlands correspondence}
In this section, we will give a solution to the basis problem by using Jacquet-Langlands correspondence.
Our goal is to prove the following theorem.
\begin{theorem}[Kimball, \cite{martin2020basis}]
    Let $F$ be a totally real number field of degree $d = [F:\mathbb{Q}]$ and $B$ be a quaternion algebra over $F$ with discriminant $\mathfrak{D}$.
    Let $\mathcal{O}$ be a special order of level $\mathfrak{N}$.
    This means that $\mathcal{O}_{\mathfrak{p}}$ is an Eichler order for $\mathfrak{p}\nmid \mathfrak{D}$
    and $\mathcal{O}_{\mathfrak{p}}$ contains the ring of integers of a quadratic extension
    $E_{\mathfrak{p}} / F_{\mathfrak{p}}$ for $\mathfrak{p} | \mathfrak{D}$, 
    and that the product of levels of the local orders is $\mathfrak{N}$.
    Let $\mathbf{k} = (k_{1}, \dots, k_{d}) \in (\mathbb{Z}_{\geq 0})^{d}$ and $S_{\mathbf{k}}(\mathcal{O})$
    be the space of quaternionic cusp forms of level $\mathbf{k}$.

    There is a Hecke-module homomorphism 
    $$\mathrm{JL}: S_{\mathbf{k}}(\mathcal{O}) \to S_{\mathbf{k} + \mathbf{2}}(\mathfrak{N})$$
    such that 
    \begin{enumerate}
        \item any newform $f \in S_{\mathbf{k} + \mathbf{2}}(\mathfrak{N})$ which is $\mathfrak{p}$-primitive 
        for $\mathfrak{p}|\mathfrak{D}$ is contained in the image;
        \item if $v_{\mathfrak{p}}(\mathfrak{N})$ is odd for all $\mathfrak{p}|\mathfrak{D}$,
        then $\mathrm{JL}$ is injective and yields an isomorphism
        $$
            S_{\mathbf{k}}(\mathcal{O}) \simeq \bigoplus_{\mathfrak{d}} S_{\mathbf{k} + \mathbf{2}}^{\mathfrak{d}- new}(\mathfrak{dM})
        $$
        where $\mathfrak{d}$ runs over all divisors of $\mathfrak{N}'$ such that $v_{\mathfrak{p}}(\mathfrak{d})$ 
        is odd for all $\mathfrak{p} | \mathfrak{D}$, and $\mathfrak{M}$ is $\mathfrak{D}$-prime part of $\mathfrak{N}$.
        Here $S_{\mathbf{k} + \mathbf{2}}^{\mathfrak{d}-new}$ is the subspace of $S_{\mathbf{k} + \mathbf{2}}(\mathfrak{N})$consisting of forms that are $\mathfrak{p}$-new
        for all $\mathfrak{p} | \mathfrak{d}$.
    \end{enumerate}

\end{theorem}
This section closely follow Kimball's article \cite{martin2020basis}.