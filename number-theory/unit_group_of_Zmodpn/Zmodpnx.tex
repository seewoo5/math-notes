\documentclass{article}
\usepackage{amsfonts, amssymb, amsmath, amsthm}
\usepackage{tikz}
\usepackage{tikz-cd}
\usepackage{hyperref}
\usepackage{comment}
\usepackage{mathtools}
\usetikzlibrary{positioning}

\title{Structure of $(\mathbb{Z}/p^{n}\mathbb{Z})^{\times}$}
\author{Seewoo Lee}


\newtheorem{theorem}{Theorem}
\newtheorem{lemma}{Lemma}
\newtheorem{corollary}{Corollary}
\newtheorem{proposition}{Proposition}

\newcommand{\Mod}[1]{(\text{mod }#1)}

\begin{document}
\maketitle

We are going to prove the following well-known result:
\begin{theorem}
Let $p$ be a prime and $n\geq 1$ be an integer. Then 
$$
(\mathbb{Z}/p^{n}\mathbb{Z})^{\times} \simeq \begin{cases} \mathbb{Z}/p^{n-1}(p-1)\mathbb{Z} & p>2 \\ \mathbb{Z}/2\mathbb{Z}\times \mathbb{Z}/2^{n-2}\mathbb{Z} & p=2, \, n\geq 2 \\
1 & p=2, \, n=1\end{cases}
$$
\end{theorem}

In other words, unit group of a ring $\mathbb{Z}/p^{n}\mathbb{Z}$ is cyclic for odd prime $p$ and product of two cyclic groups for $p=2$. 
There are some elementary proofs of this, but they are all complicated. Here we introduce $p$-adic proof of this. 

First, we'll deal with odd prime $p$, since $p=2$ case needs more concern. 

\begin{proposition}
For an odd prime $p$, we have 
$$
\mathbb{Z}_{p}^{\times}\simeq \mathbb{Z}/(p-1)\mathbb{Z}\times \mathbb{Z}_{p}
$$
\end{proposition}
\begin{proof}
%%%We follow Lubin's proof on \href{https://math.stackexchange.com/questions/2099514/units-of-p-adic-integers}{this} MSE question. 
Consider the following exact sequence of abelian groups
$$
1\to 1+p\mathbb{Z}_{p} \to \mathbb{Z}_{p}^{\times} \xrightarrow{\mathrm{mod}\,p} (\mathbb{Z}/p\mathbb{Z})^{\times} \to 1.
$$
Surprisingly, there exists a section $\omega:\mathbb{F}_{p}^{\times} \to \mathbb{Z}_{p}^{\times}$ of mod $p$ map, which is called the Teichm\"uller character. This map is defined as   
$$
\omega(x):= \lim_{n\to \infty} x^{p^{n}}, 
$$
which converges. 
(This can be regarded as a unique solution of $\omega(x)^{p} = \omega(x)$ that is congruent to $x$ mod $p$.) Hence the sequence splits and we have
$$
\mathbb{Z}_{p}^{\times}\simeq (\mathbb{Z}/p\mathbb{Z})^{\times}\times (1+p\mathbb{Z}_{p}) \simeq \mathbb{Z}/(p-1)\mathbb{Z}\times (1+p\mathbb{Z}_{p})
$$
since $(\mathbb{Z}/p\mathbb{Z})^{\times}$ is cyclic. 
To prove $(1+p\mathbb{Z}_{p}, \times)\simeq (\mathbb{Z}_{p}, +)$, we use the logarithm map, defined as a power series. For $x\in p\mathbb{Z}_{p}$, the series
$$
\log(1+x) = x- \frac{x^{2}}{2} + \frac{x^{3}}{3} - \frac{x^{4}}{4} +\cdots 
$$
converges and satisfies $\log((1+x)(1+y)) = \log(1+x) + \log (1+y)$. One can show that this gives an isomorphism between $1+p\mathbb{Z}_{p}$ and $p\mathbb{Z}_{p}$, and the inverse map corresponds to an exponential map
$$
\exp:p\mathbb{Z}_{p}\to 1+p\mathbb{Z}_{p}, \quad z\mapsto 1+ z+ \frac{z^{2}}{2!} + \frac{z^{3}}{3!} \cdots .
$$
Hence we have an isomorphism 
$$
(1+p\mathbb{Z}_{p}, \times)\to (\mathbb{Z}_{p}, +), \quad 1+x\mapsto \frac{1}{p}\log(1+x) = \frac{1}{p}\left(x-\frac{x^{2}}{2} + \frac{x^{3}}{3}-\cdots\right). 
$$
Note that by unfolding all of these, the resulting isomorphism $\mathbb{Z}/(p-1)\mathbb{Z}\times \mathbb{Z}_{p}\to \mathbb{Z}_{p}^{\times}$ can be written as 
$$
(m, z)\mapsto \omega(g_{p}^{m})\exp(pz) = \omega(g_{p})^{m}\exp(pz)
$$
where $g_{p}$ is a generator of the cyclic group $(\mathbb{Z}/p\mathbb{Z})^{\times}$. 
\end{proof}

Now we can prove our theorem for the odd prime case. Consider the following diagram:
\begin{center}
\begin{tikzcd}
1 \rar & 1+p\mathbb{Z}_{p} \rar \arrow[d, swap, "\text{mod }p^{n}"]& \mathbb{Z}_{p}^{\times} \arrow[r, "\text{mod }p"]  \arrow[d, swap, "\text{mod }p^{n}"]& (\mathbb{Z}/p\mathbb{Z})^{\times}  \arrow[l, bend left = 18, dotted, "\omega"]\dar[equal] \rar & 1 \\
1 \rar & U \rar & (\mathbb{Z}/p^{n}\mathbb{Z})^{\times} \arrow[r, "\text{mod }p"] & (\mathbb{Z}/p\mathbb{Z})^{\times} \arrow[l, bend left = 18, dotted, "\omega\text{ mod }p^{n}"] \rar & 1
\end{tikzcd}
\end{center}
where 
$$
U = \{a\equiv 1\,(\text{mod }p)\} \subset (\mathbb{Z}/p^{n}\mathbb{Z})^{\times}.
$$
This is a commutative diagram and the first row is exact, and it is easy to check that second row is also exact. Hence $(\mathbb{Z}/p^{n})^{\times} \simeq U\times (\mathbb{Z}/p\mathbb{Z})^{\times}$. 
Since $U$ and $(\mathbb{Z}/p\mathbb{Z})^{\times}$ have coprime orders, we only need to check that $U$ is a cyclic group. 

We've already showed that $(\mathbb{Z}_{p}, +)\simeq (1+p\mathbb{Z}_{p}, \times)$, and $\mathbb{Z}$ is a cyclic dense subgroup of $\mathbb{Z}_{p}$. Now we have a following diagram
$$
\mathbb{Z}\hookrightarrow \mathbb{Z}_{p} \simeq 1+p\mathbb{Z}_{p} \to U
$$
where the last map is a mod $p^{n}$ reduction map. All of these maps are continuous when we endow $U$ with a discrete topology. 
Hence the image of $\mathbb{Z}$ under this map is a dense subgroup of $U$, so is $U$ itself. In other words, the image of 1,  $\exp(p)$, is a generator of $U$. 

By combining all of these maps, we can compute a generator of the cyclic group $(\mathbb{Z}/p^{n}\mathbb{Z})^{\times}$: 
$$
g_{p, n} = \omega(g_{p})\exp(p)\text{ mod }p^{n}
$$
is a generator of $(\mathbb{Z}/p^{n}\mathbb{Z})^{\times}$. 

For example, let $p = 5$ and $n = 4$. Then $(\mathbb{Z}/5\mathbb{Z})^{\times} = \langle 2\rangle$. We have 
\begin{align*}
\omega(2) &= \lim_{n\to \infty} 2^{5^{n}} \\
&\equiv 2^{5^{4}}\quad \Mod{5^{4}} \\
&\equiv 182 \quad \Mod{5^{4}}
\end{align*}
and
\begin{align*}
\exp(5) &= 1 + 5 + \frac{5^{2}}{2} + \frac{5^{3}}{6} + \cdots  \\
&\equiv 1 + 5 - 2\cdot 5^{2}\cdot (1+5+5^{2}+\cdots) + 5^{3}\cdot(1-5+5^{2}-\cdots) \quad \Mod{5^{4}}\\
&\equiv 1 + 5 - 2\cdot 5^{2} + 3\cdot 5^{3}\quad \Mod{5^{4}} \\
&\equiv 71\quad \Mod{5^{4}}
\end{align*}
Hence 
$$
182\cdot 71 \equiv 422\quad\Mod{5^{4}}
$$
is a generator of $(\mathbb{Z}/5^{4}\mathbb{Z})^{\times}$. 

In case of $p=2$, exponential function doesn't converges on $2\mathbb{Z}_{2}\subset \mathbb{Z}_{2}$, so we have to study more carefully. We have the following:
\begin{proposition}
$$
\mathbb{Z}_{2}^{\times} \simeq \mathbb{Z}/2\mathbb{Z} \times \mathbb{Z}_{2}
$$
\end{proposition}
\begin{proof}
Consider the following exact sequence of abelian groups
$$
1 \to 1+4\mathbb{Z}_{2} \to \mathbb{Z}_{2}^{\times} \xrightarrow{\text{mod }4} (\mathbb{Z}/4\mathbb{Z})^{\times} \to 1.
$$
This exact sequence splits since there exists a section $(\mathbb{Z}/4\mathbb{Z})^{\times} \to \mathbb{Z}_{2}^{\times}$ defined as $3\mapsto -1$. Hence we have
$$
\mathbb{Z}_{2}^{\times} \simeq (\mathbb{Z}/4\mathbb{Z})^{\times}\times (1+4\mathbb{Z}_{2}) \simeq \mathbb{Z}/2\mathbb{Z}\times (1+4\mathbb{Z}_{2}). 
$$
Now, as before, logarithm and exponential maps give isomorphism between $(1+4\mathbb{Z}_{2}, \times)$ and $(\mathbb{Z}_{2}, +)$. We have 
$$
(1+4\mathbb{Z}_{2}, \times) \simeq (\mathbb{Z}_{2}, +), \quad 1+x \mapsto \frac{1}{4}\log(1+x) = \frac{1}{4}\left(x-\frac{x^{2}}{2} + \frac{x^{3}}{3} - \cdots\right)
$$
with an inverse 
$$
(\mathbb{Z}_{2}, +)\simeq (1+4\mathbb{Z}_{2}, \times), \quad z\mapsto \exp(4z) = 1 + 4z + \frac{(4z)^{2}}{2!} + \frac{(4z)^{3}}{3!} + \cdots
$$
\end{proof}
Now we can prove our theorem for $p=2$. Assume that $n>1$. 
Consider the following diagram:
\begin{center}
\begin{tikzcd}
1 \rar & 1+4\mathbb{Z}_{2} \rar \arrow[d, swap, "\text{mod }2^{n}"]& \mathbb{Z}_{2}^{\times} \arrow[r, "\text{mod }4"]  \arrow[d, swap, "\text{mod }2^{n}"]& (\mathbb{Z}/4\mathbb{Z})^{\times}  \arrow[l, bend left = 18, dotted, "\omega"]\dar[equal] \rar & 1 \\
1 \rar & U \rar & (\mathbb{Z}/2^{n}\mathbb{Z})^{\times} \arrow[r, "\text{mod }4"] & (\mathbb{Z}/4\mathbb{Z})^{\times} \arrow[l, bend left = 18, dotted, "\omega\text{ mod }2^{n}"] \rar & 1
\end{tikzcd}
\end{center}
where 
$$
U = \{a\equiv 1\,(\text{mod }4)\} \subset (\mathbb{Z}/2^{n}\mathbb{Z})^{\times}.
$$
As before, the diagram commutes and both rows are split exact. By the same argument as $p>2$, $U$ is a cyclic group and we have $$(\mathbb{Z}/2^{n}\mathbb{Z})^{\times} \simeq (\mathbb{Z}/4\mathbb{Z})^{\times}\times  U\simeq \mathbb{Z}/2\mathbb{Z}\times \mathbb{Z}/2^{n-2}\mathbb{Z}.$$
Hence $(\mathbb{Z}/2^{n}\mathbb{Z})^{\times}$ has index 2 cyclic subgroup, which is generated by $\exp(4)$. 

For example, if $n = 6$, then 
\begin{align*}
\exp(4) &= 1 + 4 + \frac{4^{2}}{2} + \frac{4^{3}}{6} + \frac{4^{4}}{24} + \frac{4^{5}}{120} + \cdots \\
&\equiv 1 + 2^{2} + 2^{3} + 2^{5}\cdot (1-2 + 2^{2} - \cdots) +  2^{5}\cdot (1-2 + 2^{2} - \cdots) \quad\Mod{2^{6}} \\
&\equiv 1 + 2^{2} + 2^{3} + 2\cdot 2^{5} \quad\Mod{2^{6}} \\
&\equiv 13\quad\Mod{2^{6}}
\end{align*}
so $13$ is an element of $(\mathbb{Z}/2^{6}\mathbb{Z})^{\times}$ of order $16 = 2^{4}$, and $(\mathbb{Z}/2^{6}\mathbb{Z})^{\times}$ is generated by $13$ and $-1$. 
\begin{comment}
We can actually compute this with SAGE. Exponential function is already implemented in SAGE, so we made a simple code for the Teichm\"uller character. 
\end{comment}


\end{document}