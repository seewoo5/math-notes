\documentclass{amsart}
\usepackage[utf8]{inputenc}
\usepackage{amsmath, amsthm, amssymb}
\usepackage{url}
\usepackage{mathrsfs}
\usepackage{tikz-cd}

\newtheorem{theorem}{Theorem}[section]
\newtheorem{conjecture}{Conjecture}[section]
% \newtheorem*{conjecture*}{Conjecture}[section]
\newtheorem{lemma}{Lemma}
\newtheorem{proposition}{Proposition}[section]
\newtheorem{corollary}{Corollary}[section]
\newtheorem{definition}{Definition}[section]


\newtheoremstyle{problemstyle}  % <name>
        {3pt}                                               % <space above>
        {3pt}                                               % <space below>
        {\normalfont}                               % <body font>
        {}                                                  % <indent amount}
        {\bfseries\itshape}                 % <theorem head font>
        {\normalfont\bfseries}
        {.5em}                                          % <space after theorem head>
        {}                                                  % <theorem head spec (can be left empty, meaning `normal')>
\theoremstyle{problemstyle}
\newtheorem{problem}{Problem}[section]

\DeclareMathOperator{\GL}{\mathrm{GL}}
\DeclareMathOperator{\SL}{\mathrm{SL}}
\DeclareMathOperator{\PGL}{\mathrm{PGL}}
\DeclareMathOperator{\Ad}{\mathrm{Ad}}
\DeclareMathOperator{\Oo}{\mathrm{O}}
\DeclareMathOperator{\Gal}{\mathrm{Gal}}
\DeclareMathOperator{\AI}{\mathrm{AI}}
\DeclareMathOperator{\L2}{\mathrm{L}^{2}}
\DeclareMathOperator{\Spec}{\mathrm{Spec}}
\DeclareMathOperator{\Hom}{\mathrm{Hom}}
\DeclareMathOperator{\Lie}{\mathrm{Lie}}
\DeclareMathOperator{\diag}{\mathrm{diag}}

\newcommand{\notfinish}{\textcolor{red}{NOT FINISHED}}

\title{Solution for ``An Introduction to Automorphic Representation"}
\author{Seewoo Lee}


\begin{document}

\begin{abstract}
This is a solution for the exercises in J. Getz and H. Hahn's book \emph{An Introduction to Automorphic 
Representation with a view toward Trace Formulae}.
\end{abstract}

\maketitle
% \tableofcontents

\newpage
\section{Chapter 1}


\begin{problem} ***
By Yoneda lemma, the morphism $\Spec(B) \to \Spec(A)$ of affine schemes
corresponds to the $k$-algebra morphism $\phi: A \to B$.
This induces a map on the underlying topological spaces by sending a prime ideal
$\mathfrak{p} \subset B$ to $\phi^{-1}(\mathfrak{p}) \subset A$, which is also prime.
\end{problem}

\begin{problem} ***
\end{problem}

\begin{problem}
By Yoneda lemma, we have 
$$
\mathrm{Mor}(\Spec(B), \Spec(A)) \simeq \mathrm{Nat}(h^{B}, h^{A}) \simeq h^{B}(A) = \Hom_{k}(A, B)
$$
which gives an equivalence between $\mathbf{AffSch}_{k}^{\mathrm{op}}$ and $\mathbf{Alg}_{k}$.
\end{problem}

\begin{problem}
    \begin{itemize}
        \item Nonreduced: $\Spec(\mathbb{C}[x]/(x^{2}))$
        \item Reducible: $\Spec(\mathbb{C}[x, y]/(x, y))$
        \item Reduced and irreducible (i.e. integral): $\Spec(\mathbb{C}[x])$
    \end{itemize}
\end{problem}

\begin{problem} 
We can assume that $Y = \Spec(A)$ and $X = \Spec(A/I)$ for some $k$-algebra $A$ and an ideal $I$ of $A$.
Then it is enough to show that the map $\Hom(A/I, R) \to \Hom(A, R)$, given by composing with
the natural map $\pi: A \to A/I$, is injective.
This follows from the surjectivity of $\pi$.
\end{problem}

\begin{problem}
Let $X = \Spec(A), Y = \Spec(B), Z = \Spec(C)$. Then the statement is equivalent to
$$
    \Hom(A \otimes_{B}C, R) \simeq \Hom(A, R) \times_{\Hom(B, R)} \Hom(C, R).
$$
We can define a bijection as follows. First, consider the following commutative diagram:
\begin{center}
\begin{tikzcd}
    A \arrow[r, "\iota_{A}"]
    & A \otimes_B C \\
    B \arrow[r, "\gamma"] \arrow[u, "\alpha"]
    &  C \arrow[u, "\iota_{C}"]
\end{tikzcd}
\end{center}
Using the maps above, we define a map from LHS to RHS as $\phi \mapsto (\phi \iota_{A}, \phi\iota_{C})$.
Since $\iota_{A}\alpha = \iota_{C}\gamma$, we have $\phi\iota_{A}\alpha = \phi\iota_{C}\gamma$ and the
map is well-defined.
For the other direction, for given $(f, g): A\times C \to R$ with $f\alpha = g \gamma$, universal property of the tensor product
gives a unique map $\phi: A\otimes_{B} C \to R$ with $f = \phi\iota_{A}$ and $g = \phi\iota_{C}$.
We can check that these maps are inverses for each other.
\end{problem}

\begin{problem} ***
\end{problem}

\begin{problem} ***
We define an $\mathbb{R}$-algebra $A$ as
$$
    A = \mathbb{R}[(x_{ij}, y_{ij})_{1 \leq i, j \leq n}] / I
$$
where $I$ is an ideal generated by elements of the form
\begin{align*}
    &\left(\sum_{k=1}^{n} (x_{ik}^{2} + y_{ik}^{2}) \right) - 1, \\
    &\sum_{k=1}^{n} (x_{ik}x_{jk} - y_{ik}y_{jk}), \quad i \neq j \\
    &\sum_{k=1}^{n} (x_{ij}y_{jk} + y_{ik}x_{jk}), \quad i \neq j
\end{align*}
for $1\leq i, j \leq n$.
Then we can identify $\mathrm{U}_{n}(R)$ with $\Hom(A, R)$ as follows: for given $\phi : A \to R$,
let $\alpha_{ij} = \phi(x_{ij})$ and $\beta_{ij} = \phi(y_{ij})$.
Then a matrix $g = (g_{ij})_{1\leq i, j \leq n}$ with $g_{ij} = 1 \otimes \alpha_{ij} + \sqrt{-1} \otimes \beta_{ij}$ becomes an element
of $U_{n}(R)$ by the relations of $x_{ij}$ and $y_{ij}$s defined by the ideal $I$.
Similarly, for given $g = (g_{ij}) \in \mathrm{U}_{n}(R)$, we can write $g_{ij} = (a_{ij} + \sqrt{-1}b_{ij}) \otimes r_{ij} = 1 \otimes a_{ij}r_{ij} + \sqrt{-1} \otimes b_{ij}r_{ij}$
and we have a corresponding map $\phi : A \to R$ sending $x_{ij}$ to $a_{ij}r_{ij}$ and $y_{ij}$ to $b_{ij}r_{ij}$.

The group $\mathrm{U}_n(\mathbb{R})$ is a compact group (as a topological subgroup of $\GL_{n}(\mathbb{C})$)
since it is closed (it is an inverse image of point $I$ of a continuous map $g \to g \overline{g}^{t}$) and bounded (each row and column vectors have norm 1).

At last, NOT FINISHED
\end{problem}

\begin{problem}
Consider the following short exact sequence:
$$
    0 \to \ker(\epsilon) / \ker(\epsilon)^{2} \to \mathcal{O}(G) / \ker(\epsilon)^{2} \to k \to 0.
$$
The map $O(G) / \ker(\epsilon)^{2} \to k$ is defined as a composition of the natural map $\mathcal{O}(G)/\ker(\epsilon)^{2} \to \mathcal{O}(G)/\ker(\epsilon)$
followed by $\epsilon$.
Then we have a section $k \to \mathcal{O}(G)/\ker(\epsilon)$ which is the composition $k \to \mathcal{O}(G) \to \mathcal{O}(G)/\ker(\epsilon)^{2}$ and the above sequence splits.
\end{problem}

\begin{problem}
Let $g = (g_{ij}) \in GL_{n}(R)$ and $J = (\alpha_{ij}) \in \GL_{n}(k)$.
Then $g^{t}Jg = J$ is equivalent to
$$
    \sum_{k, l = 1}^{n} \alpha_{kl}g_{ki}g_{lj} = \alpha_{ij}
$$
for all $1\leq i, j \leq n$.
Hence $G$ is an affine algebraic group with a coordinate ring
$$
A = k[(X_{ij})_{1\leq i, j \leq n}]/\left(\sum_{k, l =1}^{n} \alpha_{kl}X_{ki}X_{lj} - \alpha_{ij}, 1\leq i, j \leq n\right).
$$
Since $\Lie G = \ker(G(k[t]/t^{2}) \to G(k))$, the elements of $\Lie G$ have a form of $I + tX$ for some $X\in \mathrm{M}_{n}(k)$.
Then the defining equation $g^{t}Jg = J$ is equivalent to
$$
    (I + tX)^{t}J(I + tX) = J \Leftrightarrow J + tX^{t}J + tJX + t^{2}X^{t}JX = J + t(X^{t}J + JX) = J,
$$
(here every elements are in $\GL_{n}(k[t]/t^{2})$) so we should have $X^{t}J + JX = 0$.
In other words, we have a map
$$
\Lie G \xrightarrow{\sim} \{X \in \mathfrak{gl}_{n}(k)\,:\, X^{t}J + JX\}, \quad I + tX \to X.
$$
\end{problem}

\begin{problem} ***
\end{problem}

\begin{problem} ***
\end{problem}

\begin{problem}
Using the equivalence of $\mathbf{Spl}_{k}$ and $\mathbf{RRD}$, it is enough to check that the dual of the root datum
of $\GL_{n}$ is isomorphic to itself in $\mathbf{RRD}$.
Recall that the root datum of $\GL_{n}$ with torus $T$ of diagonal elements is given as follows: (Example 1.12)
\begin{itemize}
    \item $X^{*}(T) = \{
        \alpha_{k_1, \dots, k_{n}}: \mathrm{diag}(t_{1}, \dots, t_{n}) \mapsto \prod_{1\leq j \leq n}t_{j}^{k_{j}}, \, k_{1}, \dots, k_{n} \in \mathbb{Z}
    \}\simeq \mathbb{Z}^{n}$
    \item $X_{*}(T) = \{\beta_{k_{1}, \dots, k_{n}}:t \mapsto \diag(t^{k_{1}}, \dots, t^{k_{n}}),\, t_{1}, \dots, t_{n} \in \mathbb{Z}\} \simeq \mathbb{Z}^{n}$
    \item $\Phi(\GL_{n}, T) = \{e_{ij},\,1\leq i\neq j \leq n\},\,e_{ij}(\diag(t_{1}, \dots, t_{n})) = t_{i}t_{j}^{-1}$
    \item $\Phi^{\vee}(\GL_{n}, T) = \{e_{ij},\,1\leq i\neq j \leq n\},\,e_{ij}^{\vee}(t) = \diag(1, \dots, t, \dots, t^{-1}, \dots, 1)$ ($t$ in the $i$-th entry, $t^{-1}$ in the $j$-th entry, 1 for other entries)
\end{itemize}
Then we define a map $f: X_{*}(T) \to X^{*}(T)$ and $\iota: \Phi(\GL_{n}, T) \to \Phi^{\vee}(\GL_{n}, T)$ as
$$
    f(\beta_{k_1, \dots, k_n}) = \alpha_{k_1, \dots, k_n}, \quad \iota(e_{ij}) = e_{ij}^{\vee}.
$$
and define $f^{\vee}: X^{*}(T) \to X_{*}(T)$ and $\iota^{\vee}: \Phi^{\vee}(\GL_{n}, T) \to \Phi(\GL_{n}, T)$ similarly.
Then these maps are inverse to each other and gives an isomorphism between two root data
$$
    (X^{*}(T), X_{*}(T), \Phi(\GL_{n}, T), \Phi^{\vee}(\GL_{n}, T)) \simeq (X_{*}(T), X^{*}(T), \Phi^{\vee}(\GL_{n}, T), \Phi(\GL_{n}, T))
$$
(they are central isogenies) so we get $\widehat{\GL_{n}} = \GL_{n\mathbb{C}}$.
\end{problem}

\begin{problem} ***
\end{problem}

\begin{problem} ***
\end{problem}

\newpage
\section{Chapter 2}

\begin{problem} \notfinish
\end{problem}
\begin{problem} \notfinish
\end{problem}

\begin{problem}
It is compact since it is an intersection of closed subset $G(F)$ of $\GL_{n}(F)$ ($G \hookrightarrow \GL_{n}$ is closed immersion)
and intersection of closed set with compact set is again compact.
Openness follows from continuity of $G(F) \hookrightarrow \GL_{n}(F)$: $\rho(G(F)) \cap K$ is an inverse image of $K$ under $G(F) \hookrightarrow \GL_{n}(F)$.
\end{problem}

\begin{problem} \notfinish
\end{problem}

\begin{problem} Using the anti-equivalence of category $\mathbf{AffSch}_{k}$ and $\mathbf{Alg}_{k}$,
we can reformulate the situation in terms of algebra as follows.
Let $A = \mathcal{O}(Y)$ be $\mathfrak{o}$-algebra and $A_F:= A\otimes_{\mathfrak{o}} F$.
Let $X = \Spec (A_F/I)$ and $\mathcal{X}$ be schematic closure of $X$ in $Y$, so that
$\mathcal{O}(\mathcal{X}) = \mathrm{Im}(\pi^{I}\iota)$ where $\iota: A\hookrightarrow A_F$ and $\pi^{I}: A_F \twoheadrightarrow A_F / I$.
Let $\mathcal{Z} = \Spec A/J$ (we have closed immersion $\mathcal{Z} \hookrightarrow \mathcal{Y}$),
and we assume that the map on generic fibre, which corresponds to $A_F \twoheadrightarrow (A/J)_F$, induces an isomorphism
$A_F / I = \mathcal{O}(X) \simeq \mathcal{O}(\mathcal{Z}) = (A/J)_F$.
This means that there exists an isomorphism $\phi: A_F/I \to (A/J)_F$ such that the following diagram commutes:
\begin{center}
    \begin{tikzcd}
        (A/J)_F \\
        A_F / I \arrow[u, "\phi"]
        &  A_F \arrow[lu, "\pi^{J}_{F}",swap, twoheadrightarrow] \arrow[l, "\pi^{I}", twoheadrightarrow]
    \end{tikzcd}
\end{center}
Now our goal is to show that there exists a unique map
$$
f: \mathcal{O}(\mathcal{X}) = \mathrm{Im}(\pi^{I}\iota) \to \mathcal{O}(\mathcal{Z}) = A/J
$$
such that the following diagram commutes:
\begin{center}
    \begin{tikzcd}
        A/J \\
        \mathrm{Im}(\pi^{I}\iota) \arrow[u, "f"]
        &  A \arrow[lu, "\pi^{J}",swap, twoheadrightarrow] \arrow[l, "\pi^{I}\iota", twoheadrightarrow]
    \end{tikzcd}
\end{center}
The only way to define $f$ that the above diagram commutes is following: for $x \in \mathrm{Im}(\pi^{I}\iota)$, 
choose $a\in A$ with $x = \pi^{I}\iota(a)$ and define $f(x):= \pi^{J}(a)$.
Then we only need to show that the map is well-defined regardless of the choice of $a$.
Let $a_{1}, a_{2} \in A$ such that $\pi^{I}\iota(a_1) = \pi^{I}\iota(a_2) = x$.
Since $\iota^{J}: A/J \hookrightarrow (A/J)_F$ is an injection, it is enough to show that $\iota^{J}\pi^{J}(a_1) =\iota^{J}\pi^{J}(a_2)$.
By the commutativity of the following diagram
\begin{center}
    \begin{tikzcd}
        A/J \arrow[d, hookrightarrow, "\iota^{J}", swap] & A \arrow[d, hookrightarrow, "\iota"] \arrow[l, twoheadrightarrow, "\pi^{J}", swap]\\
        (A/J)_F & A_F \arrow[l, twoheadrightarrow, "\pi^{J}_F", swap]
    \end{tikzcd}
\end{center}
we have $\iota^{J}\pi^{J} = \pi_{F}^{J} \iota = \phi\pi^{I}\iota$, and this proves
$$
\iota^{J}\pi^{J}(a_1) = \phi\pi^{I}\iota(a_1) = \phi(x) = \phi\pi^{I}\iota(a_2) = \iota^{J}\pi^{J}(a_2),
$$
i.e. the map is well-defined.

\end{problem}
\begin{problem} \notfinish
\end{problem}
\begin{problem} \notfinish
\end{problem}

\begin{problem}
Note that the coordinate ring of $\GL_{n,\mathbb{Q}}$ is 
$$
B =\mathcal{O}(\GL_{n, \mathbb{Q}}) = \mathbb{Q}[x_{ij}, y]_{1\leq i,j \leq n} / (\det(x_{ij})y - 1).
$$
To show that $\mathcal{G}$ is a model of $\GL_{n, \mathbb{Q}}$ over $\mathbb{Z}$, we need to show that
$A \hookrightarrow B$ and $A \otimes_{\mathbb{Z}} \mathbb{Q} \simeq B$.
Latter isomorphism easily follows from 
$$
A \otimes \mathbb{Q} = \mathbb{Q}[x_{ij}, t_{ij}, y] / (\det(x_{ij})y - 1, \{x_{ij} - \delta_{ij} - mt_{ij}\}) \simeq B
$$
since we can invert $m > 1$ in $\mathbb{Q}$ and get an isomorphism $A\otimes\mathbb{Q} \to B$ via $t_{ij} \mapsto (1 - x_{ij})/m$.
Shoing $A \hookrightarrow B \simeq A \otimes_{\mathbb{Z}}\mathbb{Q}$ is equivalent to
showing that $A$ is a torsion-free $\mathbb{Z}$-module.
Assume that we have $z\in \mathbb{Z}[x_{ij}, t_{ij}, y]$ and $0 \neq a\in\mathbb{Z}$
such that $az = 0$ in $A$.
Then there exists $\alpha, \beta_{ij} \in \mathbb{Z}$ for $1 \leq i, j \leq n$ s.t.
\begin{align*}
az &= \alpha(\det(x_{ij})y - 1) + \sum_{ij}\beta_{ij}(x_{ij} - \delta_{ij} - mt_{ij}) \\
\Leftrightarrow z &= \frac{\alpha}{a} \det(x_{ij})y + \sum_{i, j} \frac{\beta_{ij}}{a}x_{ij} - \sum_{i, j} \frac{m\beta_{ij}}{a}t_{ij} - \frac{\alpha + \sum_{i}\beta_{ii}}{a}
\end{align*}
which implies $a |\alpha$ and $a|\beta_{ij}$, i.e. $z = 0$ in $A$.
Hence $\mathcal{G}$ is a model of $\GL_{n,\mathbb{Q}}$ over $\mathbb{Z}$.

The set of $\mathbb{Z}$-points $\mathcal{G}(\mathbb{Z}) = \Hom(A, \mathbb{Z})$ can be identified with the set via map
\begin{align*}
\Hom(A, \mathbb{Z}) &\to \{g \in \GL_{n}(\mathbb{Z})\,:\,g \equiv I_{n} \,(\mathrm{mod}\, m \mathrm{M}_{n}(\mathbb{Z}))\} \\
\phi &\mapsto (g_{ij} = \phi(x_{ij}))
\end{align*}
since $\phi(x_{ij}) = \delta_{ij} + m\phi(t_{ij}) \Rightarrow g - I_{n} \in m\mathrm{M}_{n}(\mathbb{Z})$.
\end{problem}

\begin{problem} \notfinish
It is not hard to prove that if $Z_{1}, Z_{2}$ are dense subsets of a topological space $Y_{1}, Y_{2}$ respectively, then
$Z_{1} \times Z_{2}$ is dense in $Y_{1} \times Y_{2}$.
Combining with Exercise 1.6 and Theorem 2.2.1 (b), we get the desired results for both weak and strong approximation.

\end{problem}


\begin{problem}
By Exercise 2.7 and 2.9, $\mathrm{M}_{n} \simeq \mathbb{G}_{a}^{n^{2}}$ admits weak approximation over $F$.
With embedding $\GL_{n} \hookrightarrow \mathrm{M}_{n}$ with $\GL_{n}(F) = \mathrm{M}_{n}(F) \cap \GL_{n}(F_S) \subset \mathrm{M}_{n}(F_{S})$,
we also have $\GL_{n}(F)$ dense in $\GL_{n}(F_S)$.
\end{problem}


\begin{problem} \notfinish
\end{problem}
\begin{problem} \notfinish
\end{problem}
\begin{problem} \notfinish
\end{problem}
\begin{problem} \notfinish
\end{problem}
\begin{problem} \notfinish
\end{problem}
\begin{problem} \notfinish
\end{problem}
\begin{problem} \notfinish
\end{problem}
\begin{problem} \notfinish
\end{problem}

\begin{problem}
Let $N = p_{1}^{e_{1}}\cdots p_{r}^{e_r}$ be a prime factorization of $N$. Define $K_N \leq \GL_{n}(\mathbb{A}_{\mathbb{Q}}^{\infty})$ as
$$
K_N = \prod_{i=1}^{r} (I_{n} + p_{i}^{e_{i}}\mathrm{M}_n(\mathbb{Z}_{p_i})) \times  \prod_{p \neq p_{i}} \GL_{n}(\mathbb{Z}_p).
$$
Then $K_N$ is an open compact subgroup of $\GL_{n}(\mathbb{A}_{\mathbb{Q}}^{\infty})$ such that $K_N \cap \GL_{n}(\mathbb{Q}) = \Gamma(N)$.

($\Rightarrow$) Let $H$ be a congruence subgroup of $\GL_{n}(\mathbb{Q})$, which means that there exists an open compact subgroup $K_H \leq \GL_{n}(\mathbb{A}_{\mathbb{Q}}^{\infty})$
such that $H = K_H \cap \GL_{n}(\mathbb{Q})$.
Then we can find an open compact neighborhood $U \leq K_H$ of $I_n$  which has a form of
$$
U = \prod_{p\in S} (I_n + p^{e_p}\mathrm{M}_n(\mathbb{Z}_p)) \times \prod_{p \not \in S} \GL_n(\mathbb{Z}_p)
$$
for some finite set of primes $S$ (Note that $\{I_n + p^{k}\mathrm{M}_{n}(\mathbb{Z}_{p})\}_{k\geq 1}$ is a decreasing sequence of open compact neighborhoods of $I_n$, which is also a subgroup of $\GL_{n}(\mathbb{Z}_p)$).
Then $U = K_N$ for $N = \prod_{p \in S}p^{e_p}$, i.e. $U$ is also an open compact subgroup of $\GL_{n}(\mathbb{A}_{\mathbb{Q}}^{\infty})$, and it is a finite index subgroup of $K_H$
since $K_H$ is open and compact (consider all the cosets of $K_N$ in $K_H$, which are all homeomorphic to $K_N$).
Then $[H: \Gamma(N)] = [K_H: K_N]$ implies that $H$ contains $\Gamma(N)$ as a finite index subgroup.

($\Leftarrow$) 
Let $H$ be a subgroup of $\GL_{n}(\mathbb{Q})$ contains $\Gamma(N)$ with $[H:\Gamma(N)] < \infty$.
Let $K_H$ be an image of $H$ in $\GL_{n}(\mathbb{A}_{\mathbb{Q}}^{\infty})$ under the diagonal embedding $\GL_{n}(\mathbb{Q}) \hookrightarrow \GL_{n}(\mathbb{A}_{\mathbb{Q}}^{\infty})$
so that $K_H \cap \GL_{n}(\mathbb{Q}) = H$.
Then $K_H$ contains $K_N$ and $[K_H: K_N] = [H:\Gamma(N)]$, so $K_N$ is a finite index subgroup of $K_H$.
for coset representatives $g_{1}, g_{2}, \dots, g_{t}$ of $K_{H} / K_{N}$, $K_{H} = \cup_{j=1}^{t} g_{j}K_{N}$ and by openness (resp. compactness) of $K_N$, $K_H$ is also open (resp. compact) subgroup.

\end{problem}
\newpage
\section{Chapter 3}

\begin{problem} \notfinish
\end{problem}

\begin{problem}
Since $G$ is compact, the image of the modular quasi-character $\delta_{G}: G \to \mathbb{R}_{>0}^{\times}$ is a compact subgroup of $\mathbb{R}_{>0}^{\times}$.
Then it should be trivial - otherwise, there exists $g \in G$ with $\delta_{G}(g) > 1$ (we can choose $g$ or $g^{-1}$), and then $\delta_{G}(g^{n}) = \delta_{G}(g)^{n} \to \infty$
as $n\to\infty$, i.e. the image is not bounded.
Hence $G$ is unimodular.
\end{problem}

\begin{problem} \notfinish
\end{problem}

\begin{problem}
By invariance of Haar measure, $dx(a + \varpi^k \mathcal{O}_F) = dx(\varpi^k \mathcal{O}_F)$ for all $a$.
From $\mathcal{O}_F = \cup_{a \in \mathcal{O}_F / \varpi^k \mathcal{O}_F}(a + \varpi^k \mathcal{O}_F)$, we have $dx(\mathcal{O}_F) = \sum_{a \in \mathcal{O}_F / \varpi^k \mathcal{O}_F} dx(a + \varpi^k \mathcal{O}_F) = |\mathcal{O}_F / \varpi^k \mathcal{O}_F| dx(\varpi^k \mathcal{O}_F)$.
Since the absolute value is normalized, $q = |\varpi|^{-1} = |\mathcal{O}_F/\varpi\mathcal{O}_F|$ and
$$
|\mathcal{O}_F / \varpi^k \mathcal{O}_F| = |\mathcal{O}_F / \varpi \mathcal{O}_F| \cdots |\varpi^{k-1} \mathcal{O}_F / \varpi^{k} \mathcal{O}_F| = q^{k}
$$
which shows $dx(a + \varpi^k \mathcal{O}_F) = dx(\varpi^k \mathcal{O}_F) = q^{-k} dx(\mathcal{O}_F)$.
\end{problem}

\begin{problem}
Since $G$ is reductive, $G(\mathbb{A}_F)$ is unimodular by Lemma 3.6.4.
$P(\mathbb{A}_F)$ and $K$ are both closed in $G(\mathbb{A}_F)$.
At last, $P(\mathbb{A}_F) \cap K$ is compact since $K$ is.
Hence we can normalize the Haar measure so that $dg = d_{\ell}p dk$ holds by Proposition 3.2.1.
\end{problem}

\begin{problem} \notfinish
\end{problem}

\begin{problem}
Let $k$ be a residue field and $\varpi$ be a uniformizer of $F$.
We have $\mathcal{O}_{F}^{\times} = \coprod_{a\in k^{\times}} (a + \varpi \mathcal{O}_{F})$ and
\begin{align*}
    d^{\times}x(\mathcal{O}_{F}^{\times}) &= \int_{\mathcal{O}_{F}^{\times}} \frac{dx}{|x|} \\
    &= \int_{\mathcal{O}_{F}^{\times}} dx \\
    &= dx(\mathcal{O}_{F}^{\times}) \\
    &= \sum_{a\in k^{\times}} dx(a + \varpi \mathcal{O}_{F}) \\
    &= \sum_{a\in k^{\times}} q^{-1}dx(\mathcal{O}_{F}) \\
    &= (q - 1)q^{-1}dx(\mathcal{O}_{F}) = (1 - q^{-1}) dx(\mathcal{O}_F).
\end{align*}
\end{problem}

\begin{problem} \notfinish
\end{problem}
\begin{problem} \notfinish
\end{problem}
\begin{problem} \notfinish
\end{problem}

\begin{problem}
Let $x, g, y \in \GL_n(F)$ with $y = xg$ (regard $g$ as a constant matrix).
Then we have $y_{ij} = \sum_{1\leq k \leq n} x_{ik}g_{kj}$ and $dy_{ij} = \sum_{1\leq k \leq n} g_{kj}dx_{ik}$.
This gives
\begin{align*}
&dy_{11} \wedge dy_{12} \wedge \cdots \wedge dy_{1n} \\
&= (g_{11}dx_{11} + g_{21}dx_{12} + \cdots + g_{n1}dx_{1n}) \wedge \cdots \wedge (g_{1n}dx_{11} + \cdots + g_{nn}dx_{1n}) \\
&= |\det(g^{t})| dx_{11} \wedge \cdots \wedge dx_{1n} \\
&= |\det(g)| dx_{11} \wedge \cdots \wedge dx_{1n}
\end{align*}
and along with $\det(xg) = \det(x) \det(g)$, we have
\begin{align*}
    \frac{\wedge_{i, j}dy_{ij}}{|\det(y)|^{n}} = \frac{|\det(g)|^{n} \wedge_{i, j} dx_{ij}}{|\det(xg)|^{n}} = \frac{\wedge_{i, j} dx_{ij}}{|\det(x)|^{n}}
\end{align*}
so $d(x_{ij})$ is right Haar measure. Since $\GL_{n}$ is reductive, it is unimodular and so $d(x_{ij})$ is also a left Haar measure.
\end{problem}

\begin{problem} \notfinish
\end{problem}
\begin{problem} \notfinish
\end{problem}

\begin{problem} \notfinish
Consider a reduction map $\GL_n(\mathcal{O}_{F_v}) \twoheadrightarrow \GL_n(k_v)$ where $k_v$ is a residue field of $F_v$ with $\#k_v = q_v$,
which is surjective.
The kernel $H$ of the map is $1 + \varpi_v\mathrm{M}_{n}(\mathcal{O}_{F_v})$ where $\varpi_v$ is a uniformizer of $F_v$.
Then we have 
$$
|\omega|_v(\GL_{n}(\mathcal{O}_{F_v})) = |\omega|_v(H) \cdot \# \GL_{n}(k_v).
$$
The order of $\GL_n(k_v)$ is $(q_v^2 - 1)(q_v^2 - q_v)$: there are $q_v^2 - 1$ choices for the first column vector (all but zero vector), and $q_v^2 - q_v$
choices for the second column vector (all but vectors which are multiples of the first column vector).
Also, for $h \in H$, we have
\begin{align*}
h &= \begin{pmatrix}
    1 + \varpi_v x_{11} & \varpi_v x_{12} \\ 
    \varpi_v x_{21} & 1 + \varpi_v x_{22}
\end{pmatrix}  \\
\Rightarrow |\det(h)|_v &= |1 + \varpi_v(x_{11} + x_{22}) + \varpi_v^{2} (x_{11}x_{22} - x_{12}x_{21})|_v = 1
\end{align*}
So
\begin{align*}
    |\omega|_v(H) &= \int_{\mathcal{O}_{F_v}^{4}} d(\varpi_v x_{11}) \wedge \cdots \wedge d(\varpi_v x_{22}) \\
    &= q_{v}^{-4}\int_{\mathcal{O}_{F_v}^{4}} dx_{11} \wedge \cdots \wedge dx_{22} = q_{v}^{-4}
\end{align*}
and the measure is $|\omega|_{v}(\GL_n(\mathcal{O}_{F_v})) = (1 - q_{v}^{-1})(1 - q_{v}^{-2})$.

When $F$ is a number field, then the \emph{Dedekind zeta function} of $F$, defined as
$$
\zeta_F(s) := \sum_{0 \neq I \subseteq \mathcal{O}_F} \frac{1}{N_{F/\mathbb{Q}}(I)^{s}}
$$
admits an Euler product for $\Re s > 1$:
$$
\zeta_F(s) = \prod_{\mathfrak{p} \subseteq \mathcal{O}_{F}} \frac{1}{1 - N_{F/\mathbb{Q}}(\mathfrak{p})^{s}}.
$$
Then the product is 
$$
\prod_{v\nmid \infty} \left( 1 - \frac{1}{q_v}\right) \left(1 - \frac{1}{q_{v}^{2}}\right)
$$
and this diverges since $\prod_{v\nmid \infty} (1 - q_{v}^{-1})$ does and $\prod_{v\nmid \infty}(1 - q_{v}^{-2}) = \zeta_F(2)^{-1}$ does not.
However, the normalized product
$$
\prod_{v\nmid \infty} (1 - q_{v}^{-1})^{-1} |\omega|_{v}(\GL_{n}(\mathcal{O}_{F_v})) = \prod_{v\nmid \infty} (1 - q_{v}^{-2})
$$
conveges to $\zeta_{F}(2)^{-1}$.

Now assume that $F$ is a function field.
\end{problem}

\begin{problem}
(Note that this is a theorem of Maschke.)
It is enough to show the following:

\textbf{Claim.} Let $\rho: G \to \GL(V)$ be a complex representation of finite group $G$, and let $U$ be a subrepresentation of $\rho$,
i.e. invariant under $\rho$.
Then there exists $W \leq V$ such that $U \cap W = \{0\}$ and $U \oplus W = V$.

Applying the above claim repeatedly shows that any representation of a finite group is completely decomposable.
To show the lemma, let $W'$ be \emph{any} subspace of $V$ such that $U \cap W' = \{0\}$ and $U \oplus W' = V$.
Let $\pi': V \to U$ be a corresponding projection.
Then define $\pi: V \to V$ as
$$
\pi(v) = \frac{1}{|G|} \sum_{g \in G} g^{-1} \pi'(gv)
$$
whose image is in $U$ ($gv:=\rho(g)v$).
Our claim is that $W = \ker\pi$ is the desired subspace: $W$ is $\rho$-invariant and $U \oplus W = V$.
First of all, since $\pi'|_U$ is identity on $U$ and $U$ is $\rho$-invariant, $\pi|_U$ is also an identity map on $U$.
Then we have $W \cap U = 0$, and by dimension counting we get $V = U \oplus W$.
Hence we only need to show that $W$ is $\rho$-invariant: for $h \in G$ and $v \pi W = \ker \pi$, 
\begin{align*}
    \pi(hv) &= \frac{1}{|G|} \sum_{g \in G} g^{-1} \pi'(ghv) \\
    &= \frac{1}{|G|} \sum_{g' \in G} hg'^{-1}\pi'(g'v) \quad (g' =gh) \\
    &= h \pi(v) = 0
\end{align*}
so $hv \in W$.
\end{problem}

\begin{problem}
Assume that the representation $\rho: B(\mathbb{C}) \to \GL_{2}(\mathbb{C})$ is completely reducible.
Since the representation is 2-dimensional, it should be decomposed as $\chi_{1} \oplus \chi_{2}$ for some 
characters $\chi_1, \chi_2 : \mathbb{C}^{\times} \to \mathbb{C}^{\times}$.
In other words, there exists $g_{0} \in \GL_{2}(\mathbb{C})$ such that
$$
\rho(g) = g_{0} \begin{pmatrix}
    \chi_1(g) & 0 \\ 0 & \chi_2(g)
\end{pmatrix} g_{0}^{-1}.
$$
This implies $\rho(gh) = \rho(hg)$, which is not true since $B(\mathbb{C})$ is not commutative.
\end{problem}

\begin{problem}
For any $g \in G$,
\begin{align*}
    ((f_1 * f_2) * f_3)(g) &= \int_{G}(f_1 * f_2)(gh_1^{-1})f_{3}(h_{1})d_{r}h_{1} \\
    &= \int_{G} \int_{G} f_{1}(gh_{1}^{-1}h_{2}^{-1})f_{2}(h_{2})d_{r}h_{2}f_{3}(h_{1})d_{r}h_{1} \\
    &= \int_{G} \int_{G} f_{1}(gh_{1}^{-1}h_{2}^{-1})f_{2}(h_{2})f_{3}(h_{1}) d_{r}h_{2} d_{r}h_{1} \\
    &= \int_{G} \int_{G} f_{1}(gh_{3}^{-1})f_{2}(h_{3}h_{1}^{-1})f_{3}(h_{1}) d_{r}h_{3}d_{r}h_{1} \quad (h_{3} = h_{2}h_{1},\,d_{r}h_{3} = d_{r}h_{2}) \\
    &= \int_{G} \int_{G} f_{1}(gh_{3}^{-1})f_{2}(h_{3}h_{1}^{-1})f_{3}(h_{1}) d_{r}h_{1}d_{r}h_{3} \quad (\text{Fubini's theorem}) \\
    &= \int_{G} f_{1}(gh_{3}^{-1}) \left(\int_{G} f_{2}(h_{3}h_{1}^{-1})f_{3}(h_{1}) d_{r}h_{1}\right) d_{r}h_{3} \\
    &= \int_{G} f_{1}(gh_{3}^{-1}) (f_{2} * f_{3})(h_{3})d_{r}h_{3} \\
    &= (f_{1} * (f_{2} * f_{3}))(g).
\end{align*}
\end{problem}

\begin{problem}
By Corollary 3.3.2, for $\varphi_1 ,\varphi_2 \in V$ and $f \in L^1(G)$ we have
\begin{align*}
    \|\pi(f)(\varphi_1 - \varphi_2)\| &= \bigg\|\int_G f(g)\pi(g)(\varphi_1 - \varphi_2)\bigg\|_2 dg \\
    & \leq \int_G \| f(g) \pi(g) (\varphi_1 - \varphi_2)\|_2 dg \\
    & = \int_G |f(g)| \| \pi(g) (\varphi_1 - \varphi_2)\|_2 dg \\
    & = \int_G |f(g)| \| \varphi_1 - \varphi_2 \|_2 dg \\
    &= \| \varphi_1 - \varphi_2\|_2 \|f\|_1
\end{align*}
and so $\pi(f): V \to V$ is continuous.
\end{problem}

\begin{problem}
\begin{align*}
    \pi(f_1 * f_2)\varphi &= \int_G (f_1 * f_2)(g) \pi(g) \varphi d_{r}g \\
    &= \int_G \int_G f_1(gh^{-1})f_{2}(h)\pi(g)\varphi d_{r}h d_{r}g \\
    &= \int_G \int_G f_1(gh^{-1})f_{2}(h)\pi(g)\varphi d_{r}g d_{r}h \quad (\text{Fubini's theorem}) \\
    &= \int_G \int_G f_{1}(g_{1})f_{2}(h)\pi(g_{1}h)\varphi d_{r}g_{1} d_{r}h\quad (g_1 = gh^{-1}, d_{r}g_1 = d_{r}g) \\
    &= \int_G \int_G f_{1}(g_{1})f_{2}(h)\pi(g_{1}h)\varphi d_{r}h d_{r}g_1\quad (\text{Fubini's theorem}) \\
    &= \int_G f_1(g_1) \pi(g_1) \left(\int_G f_2(h)\pi(h)\varphi d_{r}h \right)d_{r}g_{1} \\
    &= \int_G f_1(g_1) \pi(g_1) \pi(f_2)\varphi d_{r}g_1 \\
    &= (\pi(f_1) \circ \pi(f_2))\varphi
\end{align*}
\end{problem}

\newpage
\section{Chapter 4}

\begin{problem} \notfinish
\end{problem}

\begin{problem} \notfinish
\end{problem}

\begin{problem} \notfinish
% (This solution is from Mateusz Wasilewski's answer in MO \cite{unitaryequiv})

% Let $\alpha:V_1 \to V_2$ be an intertwining map that gives an equivalence, and let $M$ be a corresponding matrix with respect
% to certain basis of $V_1$ and $V_2$.
% then $$

% Since $K$ is compact, the irreducible representations are finite dimensional.
% Let $\alpha: V_1 \to V_2$ be an intertwining map that gives equivalence of two representations.
% Let $M_{\alpha}$ be a matrix form of $\alpha$ with respect to certain basis of $V_1$ and $V_2$.
% Consider its polar decomposition 
% $$
% M_\alpha = U_{0}P_{0} = (UV^{H})(V\Sigma V^{H}) = U\Sigma V^{H}
% $$
% where $U, V$ are unitary matrices and $\Sigma$ is a diagonal matrix with nonnegative real entries.
% \begin{lemma}
% For any $i=1,2$ and $k \in K$,
% $A_{i} = (V\Sigma V^{H}) P_{i}(k)$ is a normal matrix, where $P_i(k)$ is a matrix form of $\pi_i(k): V_i \to V_i$.
% \end{lemma}
% \begin{proof}
    
% \end{proof}
\end{problem}

\begin{problem} \notfinish
\end{problem}

\begin{problem}
By Schur's lemma, any elements in a center $z\in Z_G(F)$ acts as a (nonzero) scalar, let's say, $\omega_\pi(z) \in \mathbb{C}^{\times}$.
Then $\omega_\pi: Z_G(F) \to \mathbb{C}^{\times}$ is a character since $\omega_G = \pi|_{Z_F(G)}$.

Let $\chi: G(F) \to \mathbb{C}^{\times}$ be a quasi-character.
The representation $\pi \otimes \chi$ is defined as $(\pi \otimes \chi)(g)v = \chi(g)\cdot \pi(g)v$,
and it's restriction on the center becomes $\chi|_{Z_G(F)} \cdot \omega_\pi$, which is the centeral character $\omega_{\pi \otimes \chi}$ of $\pi \otimes \chi$.
\end{problem}

\begin{problem} \notfinish
\end{problem}

\begin{problem} \notfinish
\end{problem}

\begin{problem} \notfinish
\end{problem}

\begin{problem} \notfinish
\end{problem}

\begin{problem} \notfinish
\end{problem}

\begin{problem} \notfinish
\end{problem}

\begin{problem} \notfinish
\end{problem}

\begin{problem} \notfinish
\end{problem}



\bibliographystyle{plain}
\bibliography{ref}
\end{document}