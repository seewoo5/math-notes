\newpage
\section{Chapter 17}

\begin{problem} \notfinish
\end{problem}

\begin{problem} \notfinish
\end{problem}

\begin{problem} \notfinish
\end{problem}

\begin{problem} \notfinish
\end{problem}

\begin{problem} \notfinish
\end{problem}

\begin{problem} \notfinish
\end{problem}

\begin{problem} \notfinish
\end{problem}

\begin{problem} \notfinish
\end{problem}

\begin{problem} \notfinish
\end{problem}

\begin{problem} \notfinish
\end{problem}

\begin{problem}
For well-definedness, we only need to check that $g_2 \in C_{B_{\sigma}(\gamma), G}(R)$ when $(g_1, g_2) \in H_\gamma(R)$.
(Note that $(g_1, g_2) \in H_{\gamma}(R)$ automatically implies $g_{2} \in G^{\sigma}(R)$.)
We have 
$$
g_1^{-1}\gamma g_{2} = \gamma \Rightarrow g_{1}^{-1}\sigma(\gamma)g_{2} = \sigma(g_{1}^{-1}\gamma g_{2}) = \sigma(\gamma)
$$
and we can ``cancel out'' $g_1$ from two equations to get $g_{2}^{-1}\sigma(\gamma)^{-1}\gamma g_{2} = \sigma(\gamma)^{-1}\gamma$, i.e. $g_{2} \in C_{\sigma(\gamma)^{-1}\gamma, G}(R) = C_{B_{\sigma}(\gamma), G}(R)$.
It is obvious that the map is a homomorphism.
The map is injective since when $g_2 = 1$, $g_{1}^{-1}\gamma g_{2} = g_{1}^{-1}\gamma = \gamma$ implies $g_{1} = 1$.
Also, surjectivity follows from $(\gamma g_2 \gamma^{-1}, g_2)\mapsto g_2$.
\end{problem}

\begin{problem}
For well-definedness, wee need to show that a) the map $\sigma \mapsto h\sigma(h^{-1})$ is 1-cocycle,
b) the map lies in the kernel of $\rH^{1}(k, I) \to \rH^1(k, H)$, and c)
it does not depend on the choice of the right coset representative.
a) follows from the direct computation, and b) is also clear from the definition of cohomologousness and neutral element.
For c), assume that $h_2 = ih_1$ for $h_1, h_2 \in H(k^{\sep})$ and $i \in H(k^{\sep})$.
Then from $(ih_1)\sigma(ih_1)^{-1} = i h_1 \sigma(h_1)^{-1} \sigma(i)^{-1}$, we have $c_1(\sigma) = i^{-1} c_{2}(\sigma) \sigma(i)$
for $c_i(\sigma):= h_i \sigma(h_i)^{-1}$, hence they lies in the same equivalence class in $\rH^{1}(k, I)$.

For surjectivity, if $c \in \mathcal{D}(k, I, H)$, then $c: \Gal_k \to I(k^{\sep}) \to H(k^{\sep})$ that is cohomologous to the neutral element,
and this implies that there exist $h \in H(k^{\sep})$ such that $c(\sigma) = h\sigma(h)^{-1}$.

Now, let's assume that $h_1$ and $h_2$ maps to the same element in $\mathcal{D}(k, I, H)$, so that there exists $i \in I(k^{\sep})$ s.t.
$h_{2}\sigma(h_{2})^{-1} = ih_{1}\sigma(h_{1})^{-1}\sigma(i)^{-1} = ih_{1} \sigma(ih_1)^{-1}$.
This gives $\sigma((ih_{1})^{-1}h_{2}) = (ih_{1})^{-1}h_{2}$ for all $\sigma \in \Gal_k$, hence $h_{0}:=(ih)^{-1}h_{2}$ is in $H(k^{\sep})^{\Gal_k} = H(k)$,
and implies that $I(k^{\sep})h_1$ and $I(k^{\sep})h_2$ are in the same $H(k)$-orbit (under the action by right multiplication).

\end{problem}

\begin{problem}
First, $I_{h, k^{\sep}}$ is a form of $I_{k^{\sep}}$: for any $k^{\sep}$-algebra $R$, if we denote $h_R$ for the
image of $h$ in $H(R)$ under the natural map $H(k^{\sep}) \to H(R)$, then $\varphi(R): I_{k^{\sep}}(R) \to I_{h, k^{\sep}}(R)$
is given by $\varphi(R)(i):= h_{R}^{-1}ih_{R}$ gives an isomorphism.
Under this $\varphi$, we can check that $(\varphi^{-1}\circ \sigma \circ \varphi \circ \sigma^{-1})(i) = (h\sigma^{-1}(h))i (h\sigma^{-1}(h))^{-1}$,
hence it is an inner automorphism and $I_{h}$ is an inner form of $I$.
\end{problem}

\begin{problem} \notfinish
\end{problem}


\begin{problem}
For each $x \in \Omega$, choose a compact open neighborhood $U_x$ (note that $X_\alpha$ are locally compact).
Then $\cup_{x\in \Omega} U_x = \Omega$, and compactness of $\Omega$ implies that $\Omega = \cup_{1\leq i \leq n} U_i$ for some $U_i = U_{x_i}$.
For each $i$, $U_i$ is a form of
$$
U_i = U_{A_i} \times \prod_{\alpha \in A - A_i} K_{\alpha}
$$
for some finite set $A_i$ and compact open $U_{A_i} \subseteq \prod_{\alpha \in A_i}X_\alpha$.
If we put $A' = \cup_{1\leq i \leq n} A_i$, then we can rewrite $U_i$ as
$$
U_i = \left(U_{A_i} \times \prod_{\alpha \in A' - A_i} K_{\alpha}\right) \times \prod_{\alpha \in A - A'} K_\alpha =: U_{i}' \times \prod_{\alpha \in A - A'} K_\alpha
$$
and this gives $\Omega = \Omega_{A'} \times \prod_{\alpha \in A - A'} K_\alpha$ for $\Omega_{A'} = \cup_{1\leq i \leq n} U_{i}' \subseteq \prod_{\alpha \in A'} X_\alpha$.
\end{problem}


\begin{problem}
For $\sigma, \sigma'\in \Gal_k$, we have
$$
c_1(\sigma\sigma')c_2(\sigma\sigma') = c_1(\sigma) \sigma(c_1(\sigma')) c_2(\sigma) \sigma(c_2(\sigma')) = c_1(\sigma)c_2(\sigma) \sigma(c_1(\sigma')c_2(\sigma'))
$$
so $\sigma\mapsto c_1(\sigma)c_2(\sigma)$ is a 1-cocycle.
This gives a well-defined abelian group structure on $\rH^1(k, G)$ as follows: if $c_1 \sim c_1'$ and $c_2 \sim c_2'$, there exists
$g_1, g_2\in G$ such that $c_i'(\sigma) = c_i(\sigma) g_i^{-1} \sigma(g_i)$ for all $\sigma \in \Gal_k$ and $i=1,2$.
Then 
$$
(c_1'c_2')(\sigma) = c_1(\sigma) g_{1}^{-1}\sigma(g_1) c_2(\sigma) g_2^{-1} \sigma(g_2) = (c_1 c_2)(\sigma)(g_1g_2)^{-1} \sigma(g_1g_2)
$$
so the multiplication is well-defined. It is easy to check that the multiplication is commutative and associative.
Also, for $[c] \in \rH^1(k, G)$, its inverse is given by $[c]^{-1}:=[c^{-1}]$, $c^{-1}(\sigma):=c(\sigma)^{-1}$.
\end{problem}

\begin{problem} \notfinish
\end{problem}

\begin{problem} \notfinish
\end{problem}