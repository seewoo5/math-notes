\newpage
\section{Chapter 3}

\begin{problem} \notfinish
\end{problem}

\begin{problem}
Since $G$ is compact, the image of the modular quasi-character $\delta_{G}: G \to \mathbb{R}_{>0}^{\times}$ is a compact subgroup of $\mathbb{R}_{>0}^{\times}$.
Then it should be trivial - otherwise, there exists $g \in G$ with $\delta_{G}(g) > 1$ (we can choose $g$ or $g^{-1}$), and then $\delta_{G}(g^{n}) = \delta_{G}(g)^{n} \to \infty$
as $n\to\infty$, i.e. the image is not bounded.
Hence $G$ is unimodular.
\end{problem}

\begin{problem} \notfinish
\end{problem}

\begin{problem} \notfinish
\end{problem}

\begin{problem} \notfinish
\end{problem}

\begin{problem} \notfinish
\end{problem}

\begin{problem}
Let $k$ be a residue field and $\varpi$ be a uniformizer of $F$.
We have $\mathcal{O}_{F}^{\times} = \coprod_{a\in k^{\times}} (a + \varpi \mathcal{O}_{F})$ and
\begin{align*}
    d^{\times}x(\mathcal{O}_{F}^{\times}) &= \int_{\mathcal{O}_{F}^{\times}} \frac{dx}{|x|} \\
    &= \int_{\mathcal{O}_{F}^{\times}} dx \\
    &= dx(\mathcal{O}_{F}^{\times}) \\
    &= \sum_{a\in k^{\times}} dx(a + \varpi \mathcal{O}_{F}) \\
    &= \sum_{a\in k^{\times}} q^{-1}dx(\mathcal{O}_{F}) \\
    &= (q - 1)q^{-1}dx(\mathcal{O}_{F}) = (1 - q^{-1}) dx(\mathcal{O}_F).
\end{align*}
\end{problem}

\begin{problem} \notfinish
\end{problem}
\begin{problem} \notfinish
\end{problem}
\begin{problem} \notfinish
\end{problem}

\begin{problem}
Let $x, g, y \in \GL_n(F)$ with $y = xg$ (regard $g$ as a constant matrix).
Then we have $y_{ij} = \sum_{1\leq k \leq n} x_{ik}g_{kj}$ and $dy_{ij} = \sum_{1\leq k \leq n} g_{kj}dx_{ik}$.
This gives
\begin{align*}
&dy_{11} \wedge dy_{12} \wedge \cdots \wedge dy_{1n} \\
&= (g_{11}dx_{11} + g_{21}dx_{12} + \cdots + g_{n1}dx_{1n}) \wedge \cdots \wedge (g_{1n}dx_{11} + \cdots + g_{nn}dx_{1n}) \\
&= |\det(g^{t})| dx_{11} \wedge \cdots \wedge dx_{1n} \\
&= |\det(g)| dx_{11} \wedge \cdots \wedge dx_{1n}
\end{align*}
and along with $\det(xg) = \det(x) \det(g)$, we have
\begin{align*}
    \frac{\wedge_{i, j}dy_{ij}}{|\det(y)|^{n}} = \frac{|\det(g)|^{n} \wedge_{i, j} dx_{ij}}{|\det(xg)|^{n}} = \frac{\wedge_{i, j} dx_{ij}}{|\det(x)|^{n}}
\end{align*}
so $d(x_{ij})$ is right Haar measure. Since $\GL_{n}$ is reductive, it is unimodular and so $d(x_{ij})$ is also a left Haar measure.
\end{problem}

\begin{problem} \notfinish
\end{problem}
\begin{problem} \notfinish
\end{problem}

\begin{problem} \notfinish
Consider a reduction map $\GL_n(\mathcal{O}_{F_v}) \twoheadrightarrow \GL_n(k_v)$ where $k_v$ is a residue field of $F_v$ with $\#k_v = q_v$,
which is surjective.
The kernel $H$ of the map is $1 + \varpi_v\mathrm{M}_{n}(\mathcal{O}_{F_v})$ where $\varpi_v$ is a uniformizer of $F_v$.
Then we have 
$$
|\omega|_v(\GL_{n}(\mathcal{O}_{F_v})) = |\omega|_v(H) \cdot \# \GL_{n}(k_v).
$$
The order of $\GL_n(k_v)$ is $(q_v^2 - 1)(q_v^2 - q_v)$: there are $q_v^2 - 1$ choices for the first column vector (all but zero vector), and $q_v^2 - q_v$
choices for the second column vector (all but vectors which are multiples of the first column vector).
Also, for $h \in H$, we have
\begin{align*}
h &= \begin{pmatrix}
    1 + \varpi_v x_{11} & \varpi_v x_{12} \\ 
    \varpi_v x_{21} & 1 + \varpi_v x_{22}
\end{pmatrix}  \\
\Rightarrow |\det(h)|_v &= |1 + \varpi_v(x_{11} + x_{22}) + \varpi_v^{2} (x_{11}x_{22} - x_{12}x_{21})|_v = 1
\end{align*}
So
\begin{align*}
    |\omega|_v(H) &= \int_{\mathcal{O}_{F_v}^{4}} d(\varpi_v x_{11}) \wedge \cdots \wedge d(\varpi_v x_{22}) \\
    &= q_{v}^{-4}\int_{\mathcal{O}_{F_v}^{4}} dx_{11} \wedge \cdots \wedge dx_{22} = q_{v}^{-4}
\end{align*}
and the measure is $|\omega|_{v}(\GL_n(\mathcal{O}_{F_v})) = (1 - q_{v}^{-1})(1 - q_{v}^{-2})$.

When $F$ is a number field, then the \emph{Dedekind zeta function} of $F$, defined as
$$
\zeta_F(s) := \sum_{0 \neq I \subseteq \mathcal{O}_F} \frac{1}{N_{F/\mathbb{Q}}(I)^{s}}
$$
admits an Euler product for $\Re s > 1$:
$$
\zeta_F(s) = \prod_{\mathfrak{p} \subseteq \mathcal{O}_{F}} \frac{1}{1 - N_{F/\mathbb{Q}}(\mathfrak{p})^{s}}.
$$
Then the product is 
$$
\prod_{v\nmid \infty} \left( 1 - \frac{1}{q_v}\right) \left(1 - \frac{1}{q_{v}^{2}}\right)
$$
and this diverges since $\prod_{v\nmid \infty} (1 - q_{v}^{-1})$ does and $\prod_{v\nmid \infty}(1 - q_{v}^{-2}) = \zeta_F(2)^{-1}$ does not.
However, the normalized product
$$
\prod_{v\nmid \infty} (1 - q_{v}^{-1})^{-1} |\omega|_{v}(\GL_{n}(\mathcal{O}_{F_v})) = \prod_{v\nmid \infty} (1 - q_{v}^{-2})
$$
conveges to $\zeta_{F}(2)^{-1}$.

Now assume that $F$ is a function field.
\end{problem}

\begin{problem}
(Note that this is a theorem of Maschke.)
It is enough to show the following:

\textbf{Claim.} Let $\rho: G \to \GL(V)$ be a complex representation of finite group $G$, and let $U$ be a subrepresentation of $\rho$,
i.e. invariant under $\rho$.
Then there exists $W \leq V$ such that $U \cap W = \{0\}$ and $U \oplus W = V$.

Applying the above claim repeatedly shows that any representation of a finite group is completely decomposable.
To show the lemma, let $W'$ be \emph{any} subspace of $V$ such that $U \cap W' = \{0\}$ and $U \oplus W' = V$.
Let $\pi': V \to U$ be a corresponding projection.
Then define $\pi: V \to V$ as
$$
\pi(v) = \frac{1}{|G|} \sum_{g \in G} g^{-1} \pi'(gv)
$$
whose image is in $U$ ($gv:=\rho(g)v$).
Our claim is that $W = \ker\pi$ is the desired subspace: $W$ is $\rho$-invariant and $U \oplus W = V$.
First of all, since $\pi'|_U$ is identity on $U$ and $U$ is $\rho$-invariant, $\pi|_U$ is also an identity map on $U$.
Then we have $W \cap U = 0$, and by dimension counting we get $V = U \oplus W$.
Hence we only need to show that $W$ is $\rho$-invariant: for $h \in G$ and $v \pi W = \ker \pi$, 
\begin{align*}
    \pi(hv) &= \frac{1}{|G|} \sum_{g \in G} g^{-1} \pi'(ghv) \\
    &= \frac{1}{|G|} \sum_{g' \in G} hg'^{-1}\pi'(g'v) \quad (g' =gh) \\
    &= h \pi(v) = 0
\end{align*}
so $hv \in W$.
\end{problem}

\begin{problem}
Assume that the representation $\rho: B(\mathbb{C}) \to \GL_{2}(\mathbb{C})$ is completely reducible.
Since the representation is 2-dimensional, it should be decomposed as $\chi_{1} \oplus \chi_{2}$ for some 
characters $\chi_1, \chi_2 : \mathbb{C}^{\times} \to \mathbb{C}^{\times}$.
In other words, there exists $g_{0} \in \GL_{2}(\mathbb{C})$ such that
$$
\rho(g) = g_{0} \begin{pmatrix}
    \chi_1(g) & 0 \\ 0 & \chi_2(g)
\end{pmatrix} g_{0}^{-1}.
$$
This implies $\rho(gh) = \rho(hg)$, which is not true since $B(\mathbb{C})$ is not commutative.
\end{problem}

\begin{problem}
For any $g \in G$,
\begin{align*}
    ((f_1 * f_2) * f_3)(g) &= \int_{G}(f_1 * f_2)(gh_1^{-1})f_{3}(h_{1})d_{r}h_{1} \\
    &= \int_{G} \int_{G} f_{1}(gh_{1}^{-1}h_{2}^{-1})f_{2}(h_{2})d_{r}h_{2}f_{3}(h_{1})d_{r}h_{1} \\
    &= \int_{G} \int_{G} f_{1}(gh_{1}^{-1}h_{2}^{-1})f_{2}(h_{2})f_{3}(h_{1}) d_{r}h_{2} d_{r}h_{1} \\
    &= \int_{G} \int_{G} f_{1}(gh_{3}^{-1})f_{2}(h_{3}h_{1}^{-1})f_{3}(h_{1}) d_{r}h_{3}d_{r}h_{1} \quad (h_{3} = h_{2}h_{1},\,d_{r}h_{3} = d_{r}h_{2}) \\
    &= \int_{G} \int_{G} f_{1}(gh_{3}^{-1})f_{2}(h_{3}h_{1}^{-1})f_{3}(h_{1}) d_{r}h_{1}d_{r}h_{3} \quad (\text{Fubini's theorem}) \\
    &= \int_{G} f_{1}(gh_{3}^{-1}) \left(\int_{G} f_{2}(h_{3}h_{1}^{-1})f_{3}(h_{1}) d_{r}h_{1}\right) d_{r}h_{3} \\
    &= \int_{G} f_{1}(gh_{3}^{-1}) (f_{2} * f_{3})(h_{3})d_{r}h_{3} \\
    &= (f_{1} * (f_{2} * f_{3}))(g).
\end{align*}
\end{problem}

\begin{problem} \notfinish
\end{problem}

\begin{problem}
\begin{align*}
    \pi(f_1 * f_2)\varphi &= \int_G (f_1 * f_2)(g) \pi(g) \varphi d_{r}g \\
    &= \int_G \int_G f_1(gh^{-1})f_{2}(h)\pi(g)\varphi d_{r}h d_{r}g \\
    &= \int_G \int_G f_1(gh^{-1})f_{2}(h)\pi(g)\varphi d_{r}g d_{r}h \quad (\text{Fubini's theorem}) \\
    &= \int_G \int_G f_{1}(g_{1})f_{2}(h)\pi(g_{1}h)\varphi d_{r}g_{1} d_{r}h\quad (g_1 = gh^{-1}, d_{r}g_1 = d_{r}g) \\
    &= \int_G \int_G f_{1}(g_{1})f_{2}(h)\pi(g_{1}h)\varphi d_{r}h d_{r}g_1\quad (\text{Fubini's theorem}) \\
    &= \int_G f_1(g_1) \pi(g_1) \left(\int_G f_2(h)\pi(h)\varphi d_{r}h \right)d_{r}g_{1} \\
    &= \int_G f_1(g_1) \pi(g_1) \pi(f_2)\varphi d_{r}g_1 \\
    &= (\pi(f_1) \circ \pi(f_2))\varphi
\end{align*}
\end{problem}
