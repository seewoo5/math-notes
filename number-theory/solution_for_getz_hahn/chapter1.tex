\newpage
\section{Chapter 1}


\begin{problem} ***
By Yoneda lemma, the morphism $\Spec(B) \to \Spec(A)$ of affine schemes
corresponds to the $k$-algebra morphism $\phi: A \to B$.
This induces a map on the underlying topological spaces by sending a prime ideal
$\mathfrak{p} \subset B$ to $\phi^{-1}(\mathfrak{p}) \subset A$, which is also prime.
\end{problem}

\begin{problem} ***
\end{problem}

\begin{problem}
By Yoneda lemma, we have 
$$
\mathrm{Mor}(\Spec(B), \Spec(A)) \simeq \mathrm{Nat}(h^{B}, h^{A}) \simeq h^{B}(A) = \Hom_{k}(A, B)
$$
which gives an equivalence between $\mathbf{AffSch}_{k}^{\mathrm{op}}$ and $\mathbf{Alg}_{k}$.
\end{problem}

\begin{problem}
    \begin{itemize}
        \item Nonreduced: $\Spec(\mathbb{C}[x]/(x^{2}))$
        \item Reducible: $\Spec(\mathbb{C}[x, y]/(x, y))$
        \item Reduced and irreducible (i.e. integral): $\Spec(\mathbb{C}[x])$
    \end{itemize}
\end{problem}

\begin{problem} 
We can assume that $Y = \Spec(A)$ and $X = \Spec(A/I)$ for some $k$-algebra $A$ and an ideal $I$ of $A$.
Then it is enough to show that the map $\Hom(A/I, R) \to \Hom(A, R)$, given by composing with
the natural map $\pi: A \to A/I$, is injective.
This follows from the surjectivity of $\pi$.
\end{problem}

\begin{problem}
Let $X = \Spec(A), Y = \Spec(B), Z = \Spec(C)$. Then the statement is equivalent to
$$
    \Hom(A \otimes_{B}C, R) \simeq \Hom(A, R) \times_{\Hom(B, R)} \Hom(C, R).
$$
We can define a bijection as follows. First, consider the following commutative diagram:
\begin{center}
\begin{tikzcd}
    A \arrow[r, "\iota_{A}"]
    & A \otimes_B C \\
    B \arrow[r, "\gamma"] \arrow[u, "\alpha"]
    &  C \arrow[u, "\iota_{C}"]
\end{tikzcd}
\end{center}
Using the maps above, we define a map from LHS to RHS as $\phi \mapsto (\phi \iota_{A}, \phi\iota_{C})$.
Since $\iota_{A}\alpha = \iota_{C}\gamma$, we have $\phi\iota_{A}\alpha = \phi\iota_{C}\gamma$ and the
map is well-defined.
For the other direction, for given $(f, g): A\times C \to R$ with $f\alpha = g \gamma$, universal property of the tensor product
gives a unique map $\phi: A\otimes_{B} C \to R$ with $f = \phi\iota_{A}$ and $g = \phi\iota_{C}$.
We can check that these maps are inverses for each other.
\end{problem}

\begin{problem} ***
\end{problem}

\begin{problem} ***
We define an $\mathbb{R}$-algebra $A$ as
$$
    A = \mathbb{R}[(x_{ij}, y_{ij})_{1 \leq i, j \leq n}] / I
$$
where $I$ is an ideal generated by elements of the form
\begin{align*}
    &\left(\sum_{k=1}^{n} (x_{ik}^{2} + y_{ik}^{2}) \right) - 1, \\
    &\sum_{k=1}^{n} (x_{ik}x_{jk} - y_{ik}y_{jk}), \quad i \neq j \\
    &\sum_{k=1}^{n} (x_{ij}y_{jk} + y_{ik}x_{jk}), \quad i \neq j
\end{align*}
for $1\leq i, j \leq n$.
\end{problem}
\begin{problem} ***
\end{problem}
\begin{problem} ***
\end{problem}
\begin{problem} ***
\end{problem}
\begin{problem} ***
\end{problem}
\begin{problem} ***
\end{problem}
\begin{problem} ***
\end{problem}
\begin{problem} ***
\end{problem}
