\newpage
\section{Chapter 6}

\begin{problem} \notfinish
\end{problem}

\begin{problem}
(a) We have
\begin{align*}
    \iota(gh)_{i,j} = \begin{cases}
        \sum_{1\leq k \leq n} \iota'(g)_{i, k}\iota'(h)_{k, j} & 1\leq i, j \leq n \\ 
        \sum_{1\leq k \leq n} \iota'(g^{-1})_{k, i-n}\iota'(h^{-1})_{j-n, k} & n+1 \leq i, j \leq 2n \\
        0 & \text{otherwise}.
    \end{cases}
\end{align*}
From above, for $1 \leq i, j \leq n$, we have
$$
\|\iota(gh)_{i, j} \|_v = \bigg\| \sum_{1\leq k \leq n} \iota'(g)_{i, k} \iota'(h)_{k, j}\bigg\|_v 
\leq\sum_{1\leq k \leq n} \| \iota'(g)_{i, k} \iota'(h)_{k, j}  \|_v \leq n \|g\|_v \|h\|_v
$$
and similar inequality holds for $n+1 \leq i, j \leq 2n$. Hence we have $\|gh\|_{v} \leq n \|g\|_v \|h\|_v$. 
Note that we can replace the constant $n$ by $1$ for a non-archimedean place $v$. \\
(b) Since the entries of $\iota(g)$ and $\iota(g^{-1})$ are the same (just rearranged), $\|g\|_v = \|g^{-1}\|_v$. \\
(c) By (a), we have 
$$
\frac{\|gh\|_v}{\|g\|_v} \leq \frac{n\|g\|_v\|h\|_v}{\|g\|_v} = n\|h\|_v
$$
and the last term is bounded since $\Omega$ is compact. We can take $c_2(\Omega):= 2n\sup_{h\in \Omega}\|h\|_v$.
For the other direction, we have
$$
\|g\|_v = \|(gh)h^{-1}\|_v \leq n \|gh\|_v \|h^{-1}\|_v = n \|gh\|_v \|h\|_v
$$
by (a) and (b), which gives $\frac{\|gh\|}{\|g\|} \geq \frac{1}{n\|h\|_v} > c_1(\Omega)$ for $c_1(\Omega):= 1/c_2(\Omega)$.
\end{problem}

\begin{problem} \notfinish
\end{problem}


\begin{problem} \notfinish
\end{problem}


\begin{problem} \notfinish
\end{problem}


\begin{problem} \notfinish
\end{problem}


\begin{problem} \notfinish
\end{problem}


\begin{problem} \notfinish
\end{problem}


\begin{problem}
This requires the theory of Hecke operators. We'll only provide a sketch.
For the complete proof, see any textbook on modular forms (e.g. Diamond-Shurman).

Define $\alpha_p, \beta_p$ to be complex numbers satisfying $(1 -\alpha_p p^{-t})(1 - \beta_p p^{-t}) = 1 - \tau(p)p^{-t} + p^{11 - 2t}$.
Then we have
$$
\frac{1}{1 - \tau(p)p^{-t} + p^{11 - 2t}} = \frac{1}{(1 - \alpha_p p^{-t})(1 - \beta_p p^{-t})} = \sum_{k=0}^{\infty} \left( \sum_{j=0}^{k} \alpha_p^{j} \beta_{p}^{k-j} \right)p^{-kt}
$$
and the equation 6.28 is equivalent to the following statements:
\begin{enumerate}
    \item $\tau(n)$ is multiplicative, i.e. $\tau(mn) = \tau(m)\tau(n)$ for all coprime $m, n$.
    \item For all prime $p$ and $k \geq 1$, $\tau(p^{k+2}) = \tau(p)\tau(p^{k+1}) - p^{11} \tau(p^{k})$.
\end{enumerate}
Now there are Hecke operators $T_n$ for each $n \geq 1$ that acts on the space $S_{12}(\SL_2(\mathbb{Z}))$ satisfying
\begin{enumerate}
    \item $T_{mn} = T_{m}T_{n} = T_{n}T_{m}$ for all coprime $m, n$.
    \item $T_{p^{k+2}} = T_{p}T_{p^{k+1}} - p^{11}T_{p^k}$.
\end{enumerate}
Since $S_{12}(\SL_2(\mathbb{Z}))$ is 1-dimensional generated by $\Delta(z)$,  every $T_n$ acts as a constant and $\Delta(z)$ becomes
the eigenform of all $T_n$'s. Once we have $a_1 =1$, it is also known that $T_n f(z) = a_n(f)\cdot f(z)$
for Hecke eigenforms $f(z)$ (the common eigenvector of all $T_n$'s), hence $\tau(n)$ satisfy the same relation as above.

\end{problem}


\begin{problem} \notfinish
\end{problem}
