\section{Holomorphic modular forms and automorphic representations of $\GL_2$}
\label{sec:gl2}

In this section, we recall the theory of automorphic representations of $\GL_2(\bA_\bQ)$, and how to associate such an automorphic representation with a holomorphic modular form.
The main references are Bump \cite{bump1998automorphic}, Getz--Hahn \cite{getz2023introduction}, and Booher's note \cite{booher}.
We assume that the readers are familiar with the theory of holomorphic modular forms - if not, the standard references for the classical theory are Serre \cite{serre2012course}, Diamond--Shurman \cite{diamond2005first}, Zagier \cite{zagier2008elliptic}, and Bump \cite[Chapter 1]{bump1998automorphic} again.

\subsection{Adelizing holomorphic modular forms}
\label{subsec:gl2auto}

Let $f : \bH \to \bC$ be a holomorphic modular form of (even) weight $k$ and level 1, defined on the complex upper-half plane $\bH = \{z  \in \bC: \Im (z) > 0\}$.
We can upgrade $f$ to a function $\varphi_f$ on $\GL_2(\bA)$ as follows.
By the strong approximation theorem, any $g \in \GL_2(\bA)$ can be written as $g = \gamma g_\infty g_\fin$ with $\gamma \in \GL_2(\bQ)$ (diagonally embedded), $g_\infty \in \GL_2^+(\bR)$, $g_\fin \in \GL_2(\what{\bZ}) = \prod_{p < \infty} \GL_2(\bZ_p)$.
Then we define
$$
\varphi_f(\gamma g_\infty g_\fin) = (f|_k g_\infty)(i) = (ad - bc)^{k/2} (ci + d)^{-k} f\left(\frac{ai + b}{ci + d}\right)
$$
where $g_\infty = \left(\begin{smallmatrix}
    a & b \\ c & d
\end{smallmatrix}\right) \in \GL_2(\bR)$.
Using the transformation law of $f$, one can show that $\varphi_f$ is well-defined and becomes an \emph{automorphic form on $\GL_2(\bA)$}: it is
\begin{itemize}
    \item left $\GL_2(\bQ)$-invariant,
    \item right $K = K_\infty K_\fin = \rO(2) \GL_2(\what{\bZ})$-finite,
    \item has moderate growth,
    \item $Z = Z(\GL_2(\bA))$-invariant,
    \item $\mathcal{Z} = \mathcal{Z}(\gl_2(\bR))$-finite,
    \item if $f$ is a cusp form, then
    $$
        \int_{\bQ \backslash \bA} \varphi_f \left(\begin{pmatrix}
            1 & x \\ 0 & 1
        \end{pmatrix}g\right) \dd x = 0
    $$
    for all $g \in \GL_2(\bA)$.
\end{itemize}
We denote the space of functions on $\GL_2(\bA)$ satisfying the above conditions (resp. including the last condition) as $\cA(\GL_2)$ (resp. $\cA_0(\GL_2)$).
It admits an action of $(\gl_2, O(2)) \times \GL_2(\bA_\fin)$ via \emph{differentiation and right translation}
$$
    ((X, k, g_\fin), \varphi) \mapsto (x \mapsto X \varphi(xkg_\fin)).
$$
Now, we can associate a representation of $(\gl_2, \rO(2)) \times \GL_2(\bA_\fin)$ with $f$ by considering the space generated by the right translations of $\varphi_f$, denoted as $\pi = \pi_f$.
The representation is irreducible if and only if $f$ is a \emph{Hecke eigenform}, and the corresponding $\pi$ becomes an \emph{automorphic representation}, i.e. irreducible and admissible subquotient of the space $\cA(\GL_2)$ (with right translation).
Especially, any irreducible admissible automorphic representation factors as a (restricted) tensor product of local representations $\pi \simeq \otimes_{p \le \infty}' \pi_p$, proven by Flath.
Theorem \ref{thm:gl2factor} tells you how these local components are related to the original modular form $f$.


\begin{theorem}
\label{thm:gl2factor}
Let $f = \sum_{n \ge 0} a_n(f) q^n$ be an eigenform of weight $k$ and level 1.
Then the associated automorphic representation factors as $\pi =  \otimes'_{p \le \infty} \pi_{p}$ where
\begin{enumerate}
    \item $\pi_\infty$ is a discrete series of weight $k$.
    \item $\pi_p$ is an unramified principal series of $\GL_2(\bQ_p)$ induced from (unramified) characters $\chi_1, \chi_2$ with $\chi_i(p) = \alpha_i / p^{\frac{k - 1}{2}}$ satisfying
    $$
        \alpha_1 + \alpha_2 = a_p(f), \quad \alpha_1 \alpha_2 = p^{k-1}.
    $$
\end{enumerate}
\end{theorem}

The definitions of \emph{discrete series} and \emph{unramified principal series} will be explained in the following sections \ref{subsec:gl2arch} and \ref{subsec:gl2nonarch}, along with the classification of local representations.

If we consider modular forms of higher level or Maass wave forms, then we can get more interesting local representations (e.g. Steinberg or supercuspidal at finite places or other principal series at the archimedean place).
See \cite{loeffler2012computation} for the computation of local representations at nonarchimedean places when they are supercuspidal.



\subsection{Archimedean representation theory of $\GL_2$}
\label{subsec:gl2arch}

To study a representation of $\GL_2(\bR)$ or a general Lie group $G(\bR)$, we study the associated $(\frg, K)$-modules instead, which are more ``algebraic'' in nature.
Here $\frg$ is (the complexification of) the Lie algebra of $G$, and $K$ is a maximal compact subgroup of $G$.
Then $(\frg, K)$-module is a vector space equipped with actions of $\frg$ and $K$ which are compatible in a certain sense.
A $(\frg, K)$-module $V$ is \emph{admissible} if $V(\sigma)$ ($\sigma$-isotypic part of $V$) is finite dimensional for any unitary representation $\sigma$ of $K$.
Now, for any Hilbert space representation $\pi$ of $G(\bR)$, we can always associate a $(\frg, K)$-module (by differentiating and restricting the original representation), and it determines the original representation when $\pi$ is unitary.

In case of $G = \GL_2$, we have a complete classification of irreducible admissible $(\mathfrak{gl}_{2}, \rO(2))$-modules.
Representations of $\gl_2(\bC)$ is well-understood; especially, the center of the universal enveloping algebra $\mathcal{Z}(\gl_2(\bC)) = \mathcal{Z}(\mathcal{U}(\gl_2(\bC)))$ is generated by the two elements $Z = \left(\begin{smallmatrix} 1 & 0 \\ 0 & 1 \end{smallmatrix}\right)$ and $\Delta$ (Casimir operator), and it is enough to understand how these elements act (which are just constants since they live in the center).
All irreducible representations of $\rO(2)$ are either trivial, determinant, or $\tau_n = \Ind_{\SO(2)}^{\rO(2)} (\epsilon_n)$ induced from 1-dimensional characters $\epsilon_n : \SO(2) \to \bS^1, \kappa_\theta = \left( \begin{smallmatrix} \cos \theta & -\sin \theta \\ \sin \theta & \cos \theta \end{smallmatrix} \right) \mapsto e^{i n \theta}$ for $n \ge 1$.
Using this, we can obtain the following classification result.

\begin{theorem}
Irreducible admissible $(\mathfrak{gl}_2, O(2))$-modules is one of the following:
\begin{enumerate}
    \item Principal series $\pi_{s, \mu, \varepsilon} = \pInd_{B(\bR)}^{\GL_2(\bR)} (\chi_{1} \boxtimes \chi_2)$, with
    $$
        \chi_i: \bR^\times \to \bR^\times, \quad y \mapsto \sgn(y)^{\varepsilon_i} |y|^{s_i}
    $$
    where $\varepsilon_1, \varepsilon_2 \in \{0, 1\}$ and $s_1, s_2 \in \bC$ satisfy $\varepsilon \equiv \varepsilon_1 + \varepsilon_2 \pmod{2}$, $\mu = s_1 + s_2, s = \frac{1}{2}(s_1 - s_2 + 1)$.
    $Z$ (resp. $\Delta$) acts as $\mu$ (resp. $\lambda = s (1 - s)$).
    \item Discrete series $\pi_{k, \mu}$ for $k \ge 2$.
\end{enumerate}
\end{theorem}
Here $\pInd$ stands for \emph{parabolic induction}.
Discrete series appear discretely in the decomposition of the regular representation of $\GL_2(\bR)$ on $L^2(\GL_2(\bR))$.
These can be realized as a representation on a space of holomorphic functions (and they are often called \emph{holomorphic} discrete series) of $\bH$ with Petersson norm.
Especially, they are unitarizable.
As we see in Theorem \ref{thm:gl2factor}, discrete series appear as archimedean component of the automorphic representation associated with a holomorphic modular form of weight $k$.
Note that there is a \emph{limit} of discrete series $\pi_{1, \mu}$, which is actually the principal series $\pi_{\frac{1}{2}, \mu, \varepsilon}$.

For general $G$, we can still use ``parabolic induction'' technique by inducing (tempered) representations of Levi subgroups, and this gives all irreducible admissible representations by Langlands \cite[Theorem 4.9.2]{getz2023introduction}.

\subsection{Nonarchimedean representation theory of $\GL_2$}
\label{subsec:gl2nonarch}


As in the archimedean case, we are only interested in the special class of representations of $\GL_2(\bQ_p)$, which are \emph{admissible} representations.
A representation $\pi: G \to \GL(V)$ of $G$ on a complex vector space $V$ is \emph{smooth} if the corresponding map $G \times V \to V$ is continuous where we endow $V$ with the discrete topology.
A representation is \emph{admissible} if it is smooth and $\dim V^K < \infty$ for any compact open subgroup $K \le G$.
We have a classification of \emph{admissible} representations of $G = \GL_2(\bQ_p)$, which is somewhat similar to the classification of $(\frg, K)$-modules for $\GL_2(\bR)$:

\begin{theorem}
Irreducible admissible representation of $\GL_2(\bQ_p)$ is one of the following:
\begin{enumerate}
    \item Principal series $\pi(\chi_1, \chi_2) = \pInd_{B(\bQ_p)}^{\GL_2(\bQ_p)}(\chi_1 \boxtimes \chi_2)$, for quasicharacters $\chi_1, \chi_2: \bQ_p^\times \to \bC^\times$ with $\chi_1 \chi_2^{-1} \ne |\cdot|^{\pm 1}$,
    \item (Twisted) Steinberg representations,
    \item 1-dimensional representations of the form $g \mapsto \chi(\det (g))$ for $\chi : \bQ_p^\times \to \bC^\times$,
    \item Supercuspidal representations.
\end{enumerate}
\end{theorem}

Among the above representations, we will focus on the principal series.
Especially, when $p \nmid N$ and $\pi_p$ is a local component of an automorphic representation associated to a level $N$ modular form, then $\pi_p$ is an \emph{unramified} principal series, i.e. it has a nonzero $K = \GL_2(\bZ_p)$-fixed vector (\emph{spherical vector}).
For such representations, one can ``linearize'' it as a representation of the \emph{spherical Hecke algebra}
$$
\cH_p = \cH_p(\GL_2) := C_c^\infty(K \backslash \GL_2(\bQ_p) / K),
$$
where the multiplication on $\cH_p$ is given by the convolution, and $e_p := \frac{1}{|K|} \dso_{K}$ is an identity.
$\cH_p$ acts on the space of $K$-fixed vectors $\pi_p^K$ via
$$
\pi_p(f)v = \int_{\GL_2(\bQ_p)} f(g)\pi_p(g)v \dd g.
$$
One can show that $\cH_p$ is commutative: in fact, Cartan decomposition allows us to understand the structure of $\cH_p$ fairly well.
\begin{theorem}
$\cH_p$ is generated by three characteristic functions:
\begin{align*}
    T_p &= \dso_{K \left(\begin{smallmatrix} p & \\ & 1 \end{smallmatrix}\right) K} \\
    R_p &= \dso_{K \left(\begin{smallmatrix} p & \\ & p \end{smallmatrix}\right) K} \\
    R_p^{-1} &= \dso_{K \left(\begin{smallmatrix} p^{-1} & \\ & p^{-1} \end{smallmatrix}\right) K} .
\end{align*}
\end{theorem}
By the theorem, we have $\dim \pi_p^{K} = 1$ and the representation of $\cH_p$ is just a character.
In fact, unramified representations are completely determined by the associated characters of $\cH_p$ and it is enough to understand the characters.
Again, using Cartan decomposition, we can explicitly write down the structure of $\cH_p$ in terms of characteristic functions of certain double cosets, and write down the action of these explicitly.


\begin{theorem}
Let $\pi$ be an irreducible admissible unramified representation of $\GL_2(\bQ_p)$. Then $\pi$ is either 1-dimensional representation of the form $g \mapsto \chi(\det(g))$ for an unramified quasicharacter $\chi: \bQ_p^\times \to \bC^\times$, or a principal series associated to unramified quasicharacters $\chi_1, \chi_2$.
For the latter case, the associated Hecke character acts on a spherical vector $\phi^\circ \in \pi^{K}$ via
$$
T_p \phi^\circ = p^{1/2}(\chi_1(p) + \chi_2(p)) \phi^\circ, \quad R_p \phi^\circ = \chi_1(p) \chi_2(p) \phi^\circ.
$$
\end{theorem}

Under the Satake isomorphism, spherical Hecke algebra is isomorphic to the representation ring of the Langlands dual group $\what{\GL_2} = \GL_2(\bC)$ \cite{gross1998satake}.
The isomorphism $\cS: \cH_p \simeq R(\what{\GL_2})$ is given by
\begin{align*}
\cS(T_p) &= p^{1/2} \cdot \chi, \\
\cS(R_p) &= \det, \\
\cS(R_p^{-1}) &= \det{}^{-1}
\end{align*}
where $\chi = \chi_{\mathrm{std}}$ is the character of the standard representation.


\subsection{Fourier expansion and Whittaker model}
\label{subsec:gl2whit}

Recall that holomorphic modular forms of level 1 are 1-periodic and admit a Fourier expansion of the form
$$
f (z) = \sum_{n \ge 0} a_n(f) q^n =  \sum_{n \ge 0} a_n (f) e^{2 \pi i n z}
$$
for $z \in \bH$ and $q = e^{2 \pi i z}$.
One can upgrade $f$ as a function on $\GL_2(\bR)^+ = \{ g \in \GL_2(\bR): \det(g) > 0\}$ via
$$
\phi_f(g) = (ad - bc)^{k/2}(ci+d)^{-k} f\left(\frac{ai + b}{ci + d}\right), \quad g = \begin{pmatrix}
    a & b \\ c & d
\end{pmatrix} \in \GL_2(\bR)^+
$$
and consider a ``Fourier expansion'' of this function as follows.

Let $\phi: \GL_2(\bR)^+ \to \bC$ be an automorphic form on $\GL_2(\bR)$\footnote{Although I only explained the definition of $\GL_2(\bA)$ automorphic forms, we have a similar definition of automorphic forms on Lie groups. See Booher's note \cite{booher} for the case of $\GL_2(\bR)$.}.
For a fixed $g \in \GL_2(\bR)^+$, consider the following function
$$
\phi_g: \bR \to \bC, \quad  x \mapsto \phi\left(\begin{pmatrix}
    1 & x \\ 0 & 1
\end{pmatrix} g\right).
$$
By the automorphy of $\phi$, $\phi_g$ becomes a function on $\bZ \backslash \bR \simeq \bS^1$.
Hence $\phi_g$ admits a Fourier expansion
$$
    \phi_g(x) = \sum_{\psi \in \widehat{\bZ \backslash \bR}} W_{\phi, \psi}(g) \psi(x)
$$
where
$$
    W_{\phi, \psi}(g) = \int_{0}^{1} \phi_g(x) \overline{\psi(x)} \dd x.
$$
Here the sum is over all (unitary) characters of $\bZ \backslash \bR$.
Any such characters are of the form $\psi_n(x) = e^{2 \pi i n x}$, and by the moderate growth condition on $\phi$, it gives
$$
\phi_g(x) = \sum_{n \ge 0} a_n(\phi_g) e^{2 \pi i n x}
$$
where
$$
    a_{n}(\phi_g) = \int_{0}^{1} \phi_g(x) \overline{\psi_n(x)} \dd x = \int_{0}^{1} \phi\left(\begin{pmatrix}
        1 & x \\ 0 & 1
    \end{pmatrix}g\right) e^{-2 \pi i n x} \dd x
$$
which recovers the usual Fourier expansion when $\phi = \phi_f$ is associated with a holomorphic modular form $f(z)$.
Also, there is an adelic version of this (replace $\bZ, \bR, \GL_2(\bR)^+$ with $\bQ, \bA, \GL_2(\bA)$), but we will concentrate on the \emph{archimedean} version for the latter purpose;  Fourier expansions of the modular forms on $G_2(\bR)$ will be introduced in Section \ref{subsec:g2modfourier}.

One can easily check that the function $W_{\phi, \psi}: \GL_2(\bR)^+ \to \bC$ satisfy the equation
$$
   W_{\phi, \psi}\left(\begin{pmatrix}
       1 & x \\ 0 & 1
   \end{pmatrix}g\right) = \psi(x) W_{\phi, \psi}(g)
$$
for all $x \in \bR$.
Let $\cW(\psi)$ be the space of functions in $\GL_2(\bR)^+$ satisfying the above equation and having moderate growth, where $\GL_2(\bR)^+$ acts as a right translation.
One can understood a representation $V$ of $\GL_2(\bR)^+$ or the associated $(\gl_2, \rO(2))$-modules through an embedding $V \hookrightarrow \cW(\psi)$, i.e. via \emph{Whittaker model}, when such $\psi$ exists.
Equivalently, we can consider the space of \emph{Whittaker functionals}
$$
\Wh(\pi, \psi) = \Hom_{N(\bR)}(\pi, \psi) = \{\ell: \pi \to \bC, \ell(\pi(n)v) = \psi(n) \ell(v)\,\forall n \in N(\bR)\}
$$
for a representation $\pi$ of $\GL_2(\bR)^+$, where $N(\bR) = \{(\begin{smallmatrix}
    1 & x \\ 0 & 1
\end{smallmatrix}): x \in \bR\}$ and $\psi((\begin{smallmatrix}
    1 & x \\ 0 & 1
\end{smallmatrix})) = \psi(x)$.
Then we have a \emph{uniqueness} of a Whittaker model.
\begin{theorem}
Let $\pi$ be an irreducible admissible $(\frg, K)$-module for $\GL_2(\bR)$.
Then $\dim_{\bC} \Wh(\pi, \psi) \le 1$.
When the dimension is one, the corresponding Whittaker model can be given as a solution of a certain differential equation, which can be expressed as a Bessel function.
\end{theorem}
Note that local multiplicity one is true for $\GL_n(F)$ when $n \ge 2$ is arbitrary and $F$ is a local field (both archimedean and non-archimedean).
In fact, we also have a \emph{global} uniqueness result, which implies (strong) multiplicity one result for $\GL_n$ (see \cite[Chapter 4]{cogdell2004lectures}).
