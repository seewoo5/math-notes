\section{Introduction}

The main goal of this note is to give evidences (not proof) of the following claim:

\begin{claim}
Modular forms on the exceptional group $G_2$ are as much interesting as classical modular forms.
\end{claim}
While reading the note, you will find that many of the features of the classical modular forms also possessed by the modular forms on $G_2$, including:
\begin{itemize}
    \item Fourier coefficients and expansions via cubic rings (Section \ref{subsec:cubicring} and \ref{subsec:g2modfourier})
    \item Local representation theory (Section \ref{sec:g2localrep})
    \item Eisenstein series and theta series (Section \ref{subsec:g2modex})
    \item Hecke operators (Section \ref{subsec:g2heckefourier})
\end{itemize}
Gan--Gross--Savin \cite{gan2002fourier} initiated the theory of $G_2$ modular forms, based on the structure theory \cite{baez2002octonions}, \emph{quaternionic} discrete series \cite{gross1996quaternionic}, Hecke algebra \cite{gross1998satake}, multiplicity one \cite{wallach2003generalized}, etc.
We start with covering the basic theory of automorphic forms on $\GL_{2, \bQ}$ in Section \ref{sec:gl2}.
Section \ref{sec:g2def} covers the definition of (split) $G_2$ and its root system.
We define the Heisenberg parabolic subgroup of $G_2$ in Section \ref{sec:g2heisenberg}, along with its relation with the space of cubic rings.
Archimedean and non-archimedean representation theory of $G_2$ is coverd in Section \ref{sec:g2localrep}, which will be used to define \emph{weights} of modular forms on $G_2$ in Section \ref{sec:g2mod}.
We end the note by introducing related works that are not covered here (Section \ref{sec:further}) for the readers who are interested in the topic.
