\section{Heisenberg parabolic subgroup and cubic rings}
\label{sec:g2heisenberg}

For $G_2$, we have a very special parabolic subgroup called \emph{Heisenberg parabolic subgroup}, which encode all information of Fourier coefficients of modular forms on $G_2$ (Section \ref{subsec:g2modfourier}).
Especially, the orbit of the character group under the adjoint action (of Levi component) is in bijection with the isomorphism classes of the \emph{cubic rings}, and these will parametrize Fourier coefficients of modualar forms on $G_2$.
Most of the discussions in the following can be found in Gan--Gross--Savin \cite{gan2002fourier}.


\subsection{Heisenberg parabolic subgroup of $G_2$}

Here $G_2$ is considered as an algebraic group over $\bZ$.
Recall that we have a bijection between (conjugacy classes of) parabolic subgroups as follows.
Let $G$ be a split group with a maximal (split) torus $T$ and a Borel subgroup $B$.
Let $J \subseteq \Delta$ be a subset and define $\Phi(J) := \bZ J \cap \Phi(G, T)$, where $\Phi(G, T)$ is the set of roots of $T \subset G$.
Then there exists a unique parabolic subgroup $P_J \supseteq B$ with a unipotent radical $U_J$ such that
$$
    \Lie U_J = \bigoplus_{\alpha \in \Phi^+ - (\Phi(J) \cap \Phi^+)} \frg_\alpha
$$
\begin{theorem}
We have a bijection
\begin{align*}
    \{ J \subseteq \Delta\} &\leftrightarrow \{\text{parabolic subgroups of }G \text{ containing }B\}     \\
    J &\mapsto P_J.
\end{align*}
\end{theorem}
This is an order-preserving bijection: we have $P_{\emptyset} = B$ and $P_{\Delta} = G$.
The Levi subgroup $L_J$ of $P_J$ is also equal to the subgroup of $G$ generated by the centralizer $C_G(T)$ and $G_\alpha$ for $\alpha \in J$.
See \cite[Theorem 1.9.2]{getz2023introduction} or Chapter 1.9 of loc. cit. for a general theory that covers quasi-split $G$.

Let's specialize it to the \emph{maximal} parabolic subgroups.
These subgroups correspond to the subset of $\Delta$ of the form $\theta = \Delta - \{\alpha\}$ for some $\alpha \in \Delta$.
The parabolic subgroup $P = P_\theta$ admits a Levi decomposition $P = LU$, where
$$
\Lie L = \Lie T \oplus \left(\bigoplus_{\beta: m_\alpha(\beta) = 0} \frg_\beta \right)
$$
where $m_\alpha(\beta)$ is the multiplicity of $\alpha$ in the root $\beta$.
The unipotent radical $U$ has a Lie algebra
$$
V = \Lie U = \bigoplus_{\beta: m_\alpha(\beta) > 0} \frg_\beta.
$$
It admits an action of the center $Z(L) \simeq \bG_m$, which gives a grading of $V$:
\begin{align*}
    V &= \bigoplus_{n \ge 1} V_n, \\
    V_n &= \bigoplus_{\beta: m_\alpha(\beta) = n} \frg_\beta.
\end{align*}
We have a canonical $L$-stable filtration of $U$, corresponds to the filtration of $V$: we have
$$
U = U_1 \supset U_2 \supset \cdots \supset U_d \supset \{1\}
$$
with $\Lie(U_i / U_{i+1}) = V_i$. The commutator on $U$ gives a map $U_i \times U_j \to U_{i+j}$, and this corresponds to the Lie bracket $V_i \times V_j \to V_{i+j}$ by passing to the quotients.

In our case, we first fix our set of simple roots $\Delta = \{\alpha, \alpha'\}$ above and have a corresponding Borel subgroup $B \subset G_2$.
We consider the maximal parabolic subgroup $P = LU$ associated with the subset $J = \Delta - \{\alpha\} = \{\alpha'\} \subset \Delta$.
Then
$$
\Phi^+ - (\Phi(J) \cap \Phi^+) = \{\alpha, \alpha + \alpha', \alpha + 2 \alpha', \alpha + 3 \alpha', \beta_0\}
$$
where the first four of them has $m_\alpha(-) = 1$ (contribute to $V_1$) and $m_\alpha(\beta_0) = 2$ (contribute to $V_2$).
The filtration of $U$ is
$$
U = U_1 \supset U_2 = U_{\beta_0} \supset \{1\}
$$
where $U_{\beta_0} = Z(U) = [U, U]$, and $U^{\mathrm{ab}} = U / [U, U] \simeq V_1$ has dimension 4.



\subsection{Cubic rings}

\label{subsec:cubicring}

The Levi subgroup $L$ acts on $U$ by conjugation, and this induces an action of $L$ on $\Hom(U, \bG_a) = \Hom(U^{\mathrm{ab}}, \bG_a)$.
We have the following description of the representation:
\begin{proposition}
\label{prop:repL}
The representation $L$ on the space $\Hom(U, \bG_a) = \Hom(U^\mathrm{ab} , \bG_a)$ is isomorphic to the twisted representation of $\GL_2$ on the space of binary cubic forms
$$
p(x, y) = ax^3 + bx^2 y + cxy^2 + dy^3
$$
defined as
$$
\begin{pmatrix}
    A & B \\ C & D
\end{pmatrix} \cdot p(x, y) = \frac{1}{\det(\gamma)} \cdot p(Ax + Cy, Bx + Dy), \quad \gamma = \begin{pmatrix}
    A & B \\ C & D
\end{pmatrix} \in \GL_2.
$$
\end{proposition}

Later, this space will serve as a parametrizing space of the Fourier coefficients of modular forms on $G_2$.
Proposition \ref{prop:repL} is even true over $\bZ$, and in fact, each orbit is in bijection with an isomorphism class of \emph{cubic rings}: a ring which is a free $\bZ$-module of rank 3.

\begin{proposition}[Delone--Fadeev \cite{delone1964theory}, Gan--Gross--Savin \cite{gan2002fourier}]
\label{prop:bcfcubic}
There is a bijection between the $\GL_2(\bZ)$-orbits of the space of binary cubic forms with integer coefficients and the set of isomorphism class of cubic rings.
This bijection preserves discriminants.
\end{proposition}
\begin{proof}
Here we only introduce the bijection without proof.
For a cubic ring $A$, we can always find a \emph{good basis} $(1, \alpha, \beta)$ so that $A = \bZ + \bZ \cdot \alpha + \bZ \cdot \beta$ and
$$
\begin{cases}
    \alpha \beta = -ad \\
    \alpha^2 = -ac + b \alpha - a \beta \\
    \beta^2 = -bd - d \alpha + c \beta
\end{cases}
$$
for $a, b, c, d \in \bZ$, and the corresponding binary cubic form is $f(x, y) = ax^3 + bx^2 y + cxy^2 + dy^3$.
Both $A$ and $f$ has the same discriminant
$$
    \Delta = b^2 c^2 + 18abcd - 4ac^3 - 4db^3 - 27 a^2 d^2.
$$
\end{proof}
For a binary cubic form $f(x, y) = ax^3 + bx^2 y + cx y^2 + dy^3$, we define it's \emph{content} simply as $e = \gcd(a, b, c, d)$.
Then the associated cubic ring $A$ can be written as $A = \bZ + e A_0$ for some other cubic ring $A_0$.
Especially, $f$ is primitive (i.e. the content is 1) if and only if the corresponding cubic ring is \emph{Gorenstein} (i.e. $\Hom(A, \bZ)$ is projective).
We also have a local variant of the content, namely $p$-depth for a prime $p$, which is the exponent of $p$ in the prime factorization of $e$.
