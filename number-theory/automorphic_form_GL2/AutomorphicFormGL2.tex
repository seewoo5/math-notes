\documentclass{article}
\usepackage{amsfonts, amssymb, amsmath, amsthm, comment}
\usepackage{tikz}
\usepackage{mathdots}
\usepackage{hyperref}
\usepackage{makeidx}
\usepackage{dsfont}
\usepackage{accents}

\usepackage{mathtools}
\usepackage[shortlabels]{enumitem}
\usetikzlibrary{positioning}
\title{Automorphic forms and representations of $\GL(2)$}
\author{Seewoo Lee}

\newlength{\dhatheight}
\newcommand{\SST}{\mathrm{SST}}
\newcommand{\wt}{\mathrm{wt}}
\newtheorem{theorem}{Theorem}[section]
\newtheorem{lemma}{Lemma}[section]
\newtheorem{definition}{Definition}[section]

\newcommand{\pre}[1]{\prescript{#1}{}}

\newcommand{\nc}{\newcommand}

\nc{\dpl}{\displaylimits_}
\nc{\bs}{\backslash}
\nc{\resp}{\sideset{}{'}\prod}

\nc{\frag}{\mathfrak{g}}
\nc{\fraH}{\mathfrak{H}}
\nc{\fraK}{\mathfrak{k}}
\nc{\frap}{\mathfrak{p}}
\nc{\fraq}{\mathfrak{q}}
\nc{\fraa}{\mathfrak{a}}
\nc{\fraD}{\mathfrak{D}}
\nc{\gl}{\mathfrak{gl}}

\newcommand{\Cl}{\mathrm{Cl}}
\newcommand{\GL}{\mathrm{GL}}
\newcommand{\SL}{\mathrm{SL}}
\nc{\SO}{\mathrm{SO}}
\nc{\PGL}{\mathrm{PGL}}
\nc{\GO}{\mathrm{GO}}
\nc{\End}{\mathrm{End}}
\nc{\Tr}{\mathrm{Tr}}
\nc{\sgn}{\mathrm{sgn}}
\nc{\rmO}{\mathrm{O}}
\nc{\rO}{\rmO}
\nc{\PU}{\mathrm{PU}}
\nc{\rU}{\mathrm{U}}
\nc{\fin}{\mathrm{fin}}
\nc{\Sym}{\mathrm{Sym}}
\nc{\Ind}{\mathrm{Ind}}
\nc{\cInd}{\mathrm{cInd}}
\nc{\Ad}{\mathrm{Ad}}
\nc{\Stab}{\mathrm{Stab}}
\nc{\Mat}{\mathrm{Mat}}
\nc{\Hom}{\mathrm{Hom}}
\nc{\ad}{\mathrm{ad}}
\nc{\supp}{\mathrm{supp}\,}
\nc{\id}{\mathrm{id}}
\nc{\ord}{\mathrm{ord}}
\nc{\Img}{\mathrm{Im}}
\nc{\alg}{\mathrm{alg}}
\nc{\rH}{\mathrm{H}}

\nc{\ol}{\overline}
\nc{\ul}{\underline}
\nc{\wh}{\widehat}

\nc{\calD}{\mathcal{D}}
\nc{\calW}{\mathcal{W}}
\nc{\calP}{\mathcal{P}}
\nc{\calH}{\mathcal{H}}
\nc{\calO}{\mathcal{O}}
\nc{\calZ}{\mathcal{Z}}
\nc{\calG}{\mathcal{G}}
\nc{\calI}{\mathcal{I}}
\nc{\calA}{\mathcal{A}}
\nc{\calF}{\mathcal{F}}
\nc{\calM}{\mathcal{M}}
\nc{\calK}{\mathcal{K}}
\nc{\calS}{\mathcal{S}}
\nc{\calB}{\mathcal{B}}
\nc{\bra}[2]{\langle #1, #2\rangle}
\nc{\dbra}[2]{\langle\langle #1, #2\rangle\rangle}

\nc{\Qq}{\mathbb{Q}}
\nc{\Zz}{\mathbb{Z}}
\nc{\Rr}{\mathbb{R}}
\nc{\Aa}{\mathbb{A}}
\nc{\Cc}{\mathbb{C}}
\nc{\Ff}{\mathbb{F}}
\nc{\Tt}{\mathbb{T}}



\nc{\chf}{\mathds{1}}



\nc{\parz}{\frac{\partial}{\partial z}}
\nc{\parzb}{\frac{\partial}{\partial \ol{z}}}

\newcommand{\doublehat}[1]{%
    \settoheight{\dhatheight}{\ensuremath{\hat{#1}}}%
    \addtolength{\dhatheight}{-0.35ex}%
    \hat{\vphantom{\rule{1pt}{\dhatheight}}%
    \smash{\hat{#1}}}}
    

\nc{\pmat}[4]{\begin{pmatrix} #1 & #2 \\ #3 & #4 \end{pmatrix}}

\newcommand{\Mod}[1]{\,(\mathrm{mod}\,#1)}
\newcommand{\smat}[4]{\left(\begin{smallmatrix} #1 & #2 \\ #3 & #4 \end{smallmatrix}\right)}

\newtheorem{corollary}{Corollary}[section]
\newtheorem{proposition}{Proposition}[section]


\begin{document}

\maketitle
In this note, we study automorphic forms and representations of $\GL(2)$. First, we describe local theory, archimedean and non-archimedean, and then global theory. This note is mainly a summary of a  part of  Bump's \emph{Automorphic forms and representations} \cite{bu}, from chapter 2 to 4. 


\tableofcontents

\newpage
\section{Introduction}

Modular forms and Maass wave forms are certain functions defined on the complex upper half plane that satisfies $\SL(2, \Zz)$-transformations laws (or more generally, transform under congruence subgroups $\Gamma_{0}(N)$). 
There are a lot of applications of modular forms in number theory, such as sum of squares and the irrationality of $\zeta(3)$, and the Wiles' famous proof of Fermat's Last Theorem. 
There are also applications in other subjects, such as combinatorics (partition numbers), physics, representation theory (monstrous moonshine), knot theory, etc. 
 
In this note, we will study how to interpret such functions (so-called classical automorphic forms) as a representation of ad\'ele groups $\GL(2, \Aa)$ (here $\Aa$ is a ring of ad\'eles of global fields such as $\Qq$), and study representation theory of it. 
This can be a starting point of the \emph{Langlands' Program}, which connects representation of Galois groups, algebraic geometry, and automorphic forms (representations). 

To study such representations, we first study local representations. 
There are two kinds of local representations - archimedean and non-archimedean. 
For the archimedean cases, we study representation theory of $\GL(2, \Rr)$  via so-called ($\frag, K$)-modules. $(\frag, K)$-module is a vector space with compatible $\frag_{\Cc} = \mathfrak{gl}(2, \Cc)$ and $K = \rO(2)$-actions. It is easier to study $(\frag, K)$-modules than studying the representation of $\GL(2, \Rr)$ directly since $(\frag, K)$-modules are more \emph{algebraic}. We will classify $(\frag, K)$-modules for $\GL(2, \Rr)$ and also study which of them are unitarizable, since we are interested in the representation that lives in $L^{2}$ space. Also, we will see how these representations are related to classical automorphic forms (such as modular forms and Maass wave forms). 

We also have non-archimedean representations - which are representation of $p$-adic groups $\GL(2, \Qq_{p})$ for a prime $p$. They are very different from archimedean cases because of their topology. This makes the situation easier or harder, but anyway, we will also classify all the representations of such groups and study their unitarizability. 

When we finish the local theories, we can \emph{glue} these representations to obtain the representation of the ad\'ele group $\GL(2, \Aa)$. (In fact, this is not a true representation of $\GL(2, \Aa)$, but a representation of $(\frag_{\infty}, K_{\infty}) \times \GL(2, \Aa_{\fin})$.) 
While we are studying such representations (local or global), we will only concentrate on some \emph{nice} representations (\emph{admissible} representations) that are close to the representation of finite groups. 
\emph{Automorphic} representations are some nice representations that also satisfies some analytic conditions on growth. 
Later, we will see that Flath's decomposition theorem tells us that it is enough to study such glued representations to study automorphic representations. 

Before we get into the representation theory of $\GL(2, \Aa)$, we will study $\GL(1, \Aa)$ first, which are  completed by Tate in his celebrated thesis. He find a natural way to prove the analytic continuation and the functional equation of Hecke's $L$-function using local-global principle, and such idea will be used to define $L$-functions attached to automorphic representations of $\GL(2,\Aa)$. 

It may be hard to study an abstract representation of a given group (such as $\GL(2, \Rr), \GL(2, \Qq_{p})$ or $\GL(2, \Aa)$). 
Whittaker model (or Whittaker functional) help us to study such representations as a very concrete representation that functions on the group lives (and the group acts as a right translation). 
Most case, such Whittaker model exist and unique, and such results are called (local or global) multiplicity one theorem. 
In the last section, we will see how the multiplicity one theorem is related to the classical modular forms. 

%%%%%%%%%%%%%%%%%%%%%%%%%%%%
\newpage

\section{Archimedean theory}
In this section, we will study representation theory of the group $\GL(2, \Rr)$. 
Usually, it is easier to study representation of compact groups than non-compact groups because it is not much different from the representation theory of finite groups. First, any finite dimensional representations are unitarizable, by taking average of arbitrary hermitian inner product on the space over all group with respect to Haar measure, which is finite for compact groups. Also, we have celebrated Peter-Weyl theorem, which claims that any unitary representation (including infinite dimensional representation) on a complex Hilbert space is semisimple, i.e. can be decomposed as a direct sum of irreducible dimensional unitary representations, and these are all finite dimensional and mutually orthogonal. 
It is also known that representation of compact group are completely determined by its character. 

Also, Lie algebra representations of $\frag = \mathfrak{gl}(2, \Rr)$ (or its complexification $\frag_{\Cc} = \mathfrak{gl}(2, \Cc)$) are much easier than studying the representation of Lie group, because it is a linearlized version of original representation and we have a lot of tools to use. We even have a complete classification of semisimple Lie algebra over $\Cc$, which is a very rich theory itself. 

Instead of studying representations of $\GL(2, \Rr)$ directly, we will study representation theory of its maximal compact group $\rmO(2)$ and Lie algebra representation of $\mathfrak{gl}(2,\Rr)$. Eventually, we  will consider so-called $(\frag, K)$-module, which is a vector space with compatible actions of $\frag$ and $K$, and the space is not so big to deal with, i.e. admissible. 
We give complete classification of $(\frag, K)$-module for $\GL(2, \Rr)^{+}$ and $\GL(2, \Rr)$, and investigate which of them are unitarizable. 
Since unitary representation of $\GL(2, \Rr)$ is completely determined by associated $(\frag, K)$-module, we also get a complete classification of unitary representations. 

In the last subsection, we will also see how the representation theory of $\GL(2, \Rr)$ can be used to study spectral problems (of classical automorphic forms). 




\subsection{Representation theory of $\mathfrak{gl}(2, \Rr)$}
Geometrically, Lie algebra of a Lie group is a tangent space at the identity, and it has a structure of Lie algebra given by a Lie bracket. In case of $G = \GL(n, \Rr)^{+}$ and $\GL(n, \Rr)$, their Lie algebra is $\frag = \mathfrak{gl}(n, \Rr) = \Mat(n, \Rr)$, the space of $n\times n$ real matrices with the Lie bracket $[X, Y]:= XY - YX$. The most important point is that any representation of Lie group induces a Lie algebra representation. 
\begin{proposition}
Let $G$ be a Lie group and $\frag$ be a Lie algebra of $G$. Let $(\pi, V)$ be a finite dimensional representation of $G$ such that $g\mapsto \pi(g)v$ is a smooth function for all $v\in V$. Then we have an induced Lie algebra representation $d\pi:\frag\to \End(V)$ given by 
$$
(d\pi X)v = \frac{d}{dt}\Big|_{t=0} \pi(\exp(tX))v
$$
where $\exp:\frag \to G$ is the exponential map. 
\end{proposition}
The finite dimensionality assumption is non really necessary. 
In fact, we will only consider special kind of representation: right regular representation on $C^{\infty}(G)$. 
The statement is also true for this case, even if the space is not finite dimensional.
\begin{proposition}
The map $d:\frag \to \End(C^{\infty}(G))$ defined as 
$$
(dXf)(g) = \frac{d}{dt}\Big|_{t=0} f(g\exp(tX))
$$
is a Lie algebra homomorphism, i.e. $d$ is a Lie algebra representation of $\frag$ on $C^{\infty}(G)$. 
\end{proposition}

By the universal property of universal enveloping algebra $U\frag$, any Lie algebra representation $\pi:\frag \to \End(V)$ can be extended to a representation of $U\frag$. 
We will regard $U\frag$ as a ring of differential operators, which are left-invariant since Lie algebra action is obtained by differentiating right regular representation. 
When the element is in the center $Z(U\frag)$ of the universal enveloping algebra $U\frag$, it is both  invariant under the left and right regular representations.

\begin{theorem}
\label{center}
Let $G = \GL(n, \Rr)^{+}$ and let $\frag = \gl(n, \Rr)$. If $D$ is an element of $U\frag$, then $D$ is invariant under both the left and right regular representations of $G$. 
\end{theorem}
\begin{proof}
The proof is a little technical. We need the following lemma:
\begin{lemma}
Let $G = \GL(n, \Rr)^{+}$ and let $X\in \frag = \gl(n, \Rr)$. Suppose that $\phi\in C^{\infty}(G\times \Rr)$ satisfies 
$$
\frac{\partial}{\partial t} \phi(g, t) = dX \phi(g, t)
$$
and the boundary condition $\phi(g, 0) = 0$. Then $\phi(g, t) = 0$ for all $t\in \Rr$. 
\end{lemma}
\begin{proof}
This can be done by method of characteristic. Let $\phi_{g}(u, t) = \phi(g \exp(uX),t)$ for $g\in G$. If we make the change of variables as $t = v + w$ and $u = v-w$, the equation is equivalent to 
$$
\frac{\partial}{\partial w}\phi_{g}(v-w, v+w) = 0
$$
so $\phi_{g}(v-w, v+w)$ is independent of $w$ and $\phi_{g}(v-w, v+w) = F_{g}(v)$ for some $F_{g}\in C^{\infty}(\Rr)$. This gives $\phi_{g}(u, t) = F_{g}((u+t)/2)$ and the boundary condition implies that $F_{g} = 0$, so $\phi_{g} = 0$. 
\end{proof}
Now apply the lemma for the function
$$
\phi(g, t) = (D\rho(\exp(tX))f - \rho(\exp(tX))Df)(g)
$$
and we get the result. Note that $G$ is generated by $\exp(\frag)$.
\end{proof}

Now we will concentrate on $n = 2$. $\frag = \gl(2, \Rr)$ is generated by the elements
$$
\wh{R} = \pmat{0}{1}{0}{0}, \quad \wh{L} = \pmat{0}{0}{1}{0}, \quad \wh{H} = \pmat{1}{0}{0}{-1}, \quad Z = \pmat{1}{0}{0}{1}
$$
with relations 
$$
[\wh{H}, \wh{R}] = 2\wh{R}, \quad [\wh{H}, \wh{L}] = -2\wh{L}, \quad [\wh{R}, \wh{L}] = \wh{H}. 
$$
Now let
$$
\Delta = -\frac{1}{4} ( \wh{H}^{2} + 2\wh{R}\wh{L} + 2\wh{L}\wh{R})
$$
be an element in $U\frag$, where the multiplication is in $U\frag$, not a matrix multiplication. This is a very special element in $U\frag$, which is called the \emph{Cacimir element}. The element is in the center of $U\frag$, and in fact the center is generated by $\Delta$ and $Z$. 
\begin{theorem}
$\Delta$ lies in the center of $U\frag = U\gl(2, \Rr)$. 
\end{theorem}
\begin{proof}
This follows from direct computations and relations among $\wh{R}, \wh{L}, \wh{H}$. 
\end{proof}
We will consider the complexification $\frag_{\Cc} = \gl(2, \Cc)$ of $\frag$ and slightly modify the elements $\wh{R}, \wh{L}, \wh{H}$ in $\gl(2,\Cc)$ as
$$
R = \frac{1}{2} \pmat{1}{i}{i}{-1} , \quad L = \frac{1}{2} \pmat{1}{-i}{-i}{-1}, \quad H = -i\pmat{0}{1}{-1}{0}.
$$
Then they satisfy the same relations as $\wh{R}, \wh{L}$, and $\wh{H}$. Indeed, we have
$$
CHC^{-1} = \wh{H}, \quad CRC^{-1} = \wh{R}, \quad CLC^{-1} = \wh{L}
$$
where 
$$
C = -\frac{1+i}{2} \pmat{i}{1}{i}{-1}
$$
is the Cayley transform. We will see the reason why we are using $R, L, H$ instead of $\wh{R}, \wh{L}$, and $\wh{H}$, in section 2.6. 



For an arbitrary representation $(\pi, \fraH)$ of $G$, there may not exists a corresponding Lie algebra action on $\fraH$ since the limit may not exists. 
We will define $\fraH^{\infty}$ as a largest subspace where such action exists, i.e. the limit $\pi(X)f = Xf = \frac{d}{dt}|_{t=0} \pi(\exp(tX)) f$ exists for all $X\in \frag$ and $f \in \fraH^{\infty}$. We will call such $f$ as \emph{smooth} vector, and we can easily check that such space is invariant under the action of $G$ from the equation
$$
\pi(X)\pi(g)f = \pi(g)\left( \lim_{t\to 0} \frac{1}{t} (\pi(\exp( t\Ad(g^{-1})X))f - f)\right).
$$  
Also, the action of $\frag$ on $\fraH^{\infty}$ is a Lie algebra representation. 
We define the action of $C^{\infty}_{c}(G)$ on $\fraH$ as 
$$
\pi(\phi)f = \int_{G} \phi(g)\pi(g)f dg
$$
for $\phi\in C_{c}^{\infty}(G)$. We can show that the subspace $\fraH^{\infty}$ of smooth vectors is not so small, indeed, it is dense in $\fraH$. 

\begin{proposition}
Let $(\pi, \fraH)$ be a Hilbert space representation of $G = \GL(n, \Rr)$ or $G = \GL(n, \Rr)^{+}$. 
\begin{enumerate}
\item If $\phi\in C_{c}^{\infty}(G)$ and $f\in \fraH$, then $\pi(\phi)f \in \fraH^{\infty}$. 
\item $\fraH^{\infty}$ is dense in $\fraH$. 
\end{enumerate}
\end{proposition}
\begin{proof}
For 1, we can check that $\pi(X)\pi(\phi)f = \pi(\phi_{X})f$ where $$\phi_{X}(g) = \frac{d}{dt}\Big|_{t=0} \phi(\exp(-tX)g).$$ 
Hence $\pi(\phi)f$ is differentiable and we can repeat this to get $\pi(\phi)f\in \fraH^{\infty}$. 

For 2, we use 1 with appropriate function $\phi$. For given $\epsilon>0$, continuity of $(g, f)\to \pi(g)f$ implies that there exists an open neighborhood of the identity of $G$ such that $|\pi(g)f - f|<\epsilon$ for all $g\in U$. 
Now take $\phi\in C_{c}^{\infty}(G)$ to be a nonnegative function with $\supp(\phi)\subset U$ and $\int_{G}\phi(g) dg = 1$, so that 
$$
|\pi(\phi)f - f| \leq \int_{G} \phi(g)|\pi(g)f - f|dg \leq \epsilon
$$
which proves that $\fraH^{\infty}$ is dense in $\fraH$. 
\end{proof}








\subsection{Representation theory of compact group}
In this section, we will see how representations of compact groups well-behaves. We will prove the Peter-Weyl theorem, which claims that every representation of a compact group decomposes as a direct sum of finite dimensional irreducible representations. 

For any finite group $G$ and it's irreducible representation $(\pi, V)$ (which has finite degree), we can construct a $G$-invariant inner product on $V$: choose any inner product  $\bra{\,}{\,}_{1}:V\times V\to \Cc$ and define a new pairing $\bra{\,}{\,}:V\times V \to \Cc$ as 
$$
\bra{v}{w} = \sum_{g\in G} \bra{\pi(g)v}{\pi(g)w}_{1}.
$$
Then this pairing is also an inner product on $V$ and it is $G$-invariant by definition. We can do the same thing for a representation of compact group $K$ on a Hilbert space $\fraH$,  by integrating a given inner product on over $K$ with respect to its Haar measure. (Note that compact group has a finite Haar measure.) This induces same topology as before. 
\begin{lemma}
Let $(\pi, \fraH)$ be a representation of a compact group $K$ on a Hilbert space $(\fraH, \bra{\,}{\,}_{1})$. 
There exists a Hermitian inner product $\bra{\,}{\,}$ on $\fraH$ inducing the same topology as the original one and $K$-invariant. 
\end{lemma}
\begin{proof}
We define such inner product on $\fraH$ as 
$$
\bra{v}{w} = \int_{K} \bra{\pi(\kappa)v}{\pi(\kappa)w}_{1}d\kappa. 
$$
It is easy to check that this defines a new inner product which is $K$-invariant. By Banach-Steinhaus theorem,  we can found a constant $C>0$ such that $C^{-1}|v|_{1} \leq |\pi(\kappa)v|_{1} \leq C|v|_{1}$ for all $v\in \fraH$ and $\kappa \in K$, and this proves $C^{-1}|v|_{1} \leq |v| \leq C|v|_{1}$ for all $v$. Hence topologies are same. 
\end{proof}


Now we will prove the most important theorem in the representation theory of compact groups, Peter-Weyl theorem. For a representation $(\pi, \fraH)$ on a Hilbert space $\fraH$ of $G$, a \emph{matrix coefficient} of the representation is a function on $G$ of the form $g\mapsto \langle \pi(g)x, y\rangle$. 
We need the following proposition:
\begin{proposition}
\label{nonzeroint}
Let $G$ be a compact group and $(\pi_{1}, H_{1}), (\pi_{2}, H_{2})$ be representations where $(\pi_{2}, H_{2})$ is unitary. 
If there exists matrix coefficients $f_{1}, f_{2}$ of $\pi_{1}$ and $\pi_{2}$ that are not orthogonal in $L^{2}(G)$, then there exists a nonzero intertwining operator $L:H_{1}\to H_{2}$. 
\end{proposition}
\begin{proof}
Assume that $f_{i} = \bra{\pi_{i}(g)x_{i}}{y_{i}}$ such that
$$
\ol{\dbra{f_{1}}{f_{2}}} := \int_{G} \ol{f_{1}(g)} f_{2}(g) dg = \int_{G} \ol{\bra{\pi_{1}(g)x_{1}}{y_{1}}}\bra{\pi_{2}(g)x_{2}}{y_{2}} dg \neq 0. 
$$
Then the bounded linear map $L:H_{1}\to H_{2}$ defined as
$$
L(v) = \int_{G} \bra{\pi_{1}(g)v}{y_{1}} \pi_{2}(g^{-1})y_{2} dg
$$
gives a nonzero intertwining operator, since $\bra{x_{2}}{L(x_{1})} = \ol{\dbra{f_{1}}{f_{2}}}$. 
\end{proof}
\begin{theorem}[Peter-Weyl]
Let $K$ be a compact subgroup of $\GL(n, \Cc)$.
\begin{enumerate}
\item The matrix coefficiens of finite dimensional unitary representation of $K$ is dense in $C(K)$ and $L^{p}(K)$ for all $1\leq p <\infty$. 
\item Any irreducible unitary representation of $K$ is finite dimensional. 
\item Any unitary representation of $K$ is semisimple, i.e. decomposes as a Hilbert direct sum of (finite dimensional) irreducible representations. 
\end{enumerate}
\end{theorem}
\begin{proof}
By embedding $\GL(n, \Cc)\hookrightarrow \GL(2n, \Rr)$, we can assume that $K$ is a subgroup of $\GL(n, \Rr)$ for some $n$. We call a function on $K$ a polynomial function if it sis a polynomial with complex coefficients in terms of $n^{2}$ entries of matrices in $K\subset \Mat(n, \Rr)$. We first show that any polynomial function on $K$ is a matrix coefficient of a finite dimensional representation. 
Indeed, let $r\in \Zz_{>0}$ and $(\rho, R)$ be the representation of $K$ where $R$ is a space of polynomial functions of degree $\leq r$ on $\Mat(n, \Rr)$, where $K$ acts by right translation. 
We can find a Hiermitian inner product on $R$ which is $K$-invariant, and by Riesz representation theorem there exists $f_{0}\in R$ such that $f(1) = \bra{f}{f_{0}}$ for all $f\in R$, since $f\mapsto f(1)$ is a bounded linear functional on $R$. Then 
$$
f(g) = (\rho(g)f)(1) = \bra{\rho(g)f}{f_{0}}
$$
so the function $f$ is a matrix coefficient of $R$. 

Now we prove 1. It is known that $C(K)$ is dense in $L^{p}(K)$ for any $1\leq p <\infty$, and Stone-Weierstrass theorem implies that any continuous function on $K$ can be uniformly approximated by polynomial functions, which are matrix coefficients. 

To show 2 and 3, it is enough to show that any nonzero unitary representation $(\pi, \fraH)$ of $K$ admits a nonzero finite dimensional invariant subspace. 
Choose any nonzero matrix coefficient $\phi$ of $\fraH$ and approximate it by a polynomial function $\phi_{0}$, so that $\phi$ and $\phi_{0}$ are not orthogonal. Then the proposition \ref{nonzeroint} shows that there is a nonzero intertwining map $L:R\to \fraH$ for a finite dimensional representation $R$ of polynomial functions, and the image of $L$ is a finite dimensional invariant subspace of $\fraH$. 
This proves 2, and 3 also follows from this with applying Zorn's lemma. 
\end{proof}

Using Peter-Weyl theorem, we can define admissibility of representation of $G$ for $G = \GL(n, \Rr)^{+}$ or $\GL(n,\Rr)$. 
A representation $(\pi, \fraH)$ of $G$ is \emph{admissible} if each isomorphism class of finite dimensional representations of $K$ occurs only finitely many times in a decomposition of $\pi|_{K}$. 
This implies that for each irreducible representation $\rho$ of $K$, the isotypic component $\fraH(\rho)$ of $(\pi|_{K}, \fraH)$, the direct sum of all the subrepresentations of $(\pi|_{K}, \fraH)$ isomorphic to $\rho$, is finite dimensional. 
We can check that multiplicity of a given finite dimensional representation does not depend on the decomposition. 
Also, it is a right category to study since it is known that any irreducible unitary representation is admissible. 

The next result shows that in the decomposition of irreducible admissible unitary representation $\fraH$ over $K$, the multiplicity of the trivial representation of $K$ is at most one. 
To prove this, we need the result about commutativity of Hecke algebra $C^{\infty}_{c}(K\backslash G/K)$ which can be proved by Gelfand's trick with Cartan decomposition. 

\begin{theorem}[Gelfand]
\label{archec}
Let $G = \GL(n, \Rr)$ and $K = \rmO(n)$, or $G = \GL(n, \Rr)^{+}$ and $K = \SO(n)$. Let $C_{c}^{\infty}(K\backslash G/K)$ be a subalgebra of $C_{c}^{\infty}(G)$ which are $K$-bi-invariant, i.e. $\phi(\kappa_{1}g\kappa_{2})=\phi(g)$ for all $g\in G$ and $\kappa_{1}, \kappa_{2}\in K$, where the multiplication is given by convolution. 
Then $C_{c}^{\infty}(K\backslash G/K)$ is commutative. 
\end{theorem}
Note that $C_{c}^{\infty}(G)$ is non-commutative. 
\begin{proof}
We need the following decomposition theorem of Cartan, which we will not going to prove. Basically, this follows from the induction on $n$. 
\begin{proposition}[Cartan]
Let $G = \GL(n, \Rr)$ and $K = \rmO(n)$, or $G = \GL(n, \Rr)^{+}$ and $K = \SO(n)$. 
In either case, every double coset in $K\backslash G/K$ has a unique representative of the form 
$$
\begin{pmatrix} d_{1} & & \\ & \ddots & \\ & & d_{n}\end{pmatrix}, \quad d_{i}\in \Rr, \quad d_{1} \geq d_{2} \geq \dots \geq d_{n} >0.
$$
\end{proposition}
Now let $\iota:C^{\infty}_{c}(K\backslash G / K)\to C^{\infty}_{c}(K\backslash G /K)$ be a map defined as $\iota(\phi(g))= \wh{\phi}(g) := \phi(\pre{T}{g})$. 
Then this map ins an anti-involution of $C^{\infty}_{c}(K\backslash G / K)$:
\begin{align*}
\wh{(\phi_{1}*\phi_{2})}(g) &= \int_{G} \phi_{1}(\pre{T}{}gh)\phi_{2}(h^{-1})dh \\
&= \int_{G} \wh{\phi_{2}}({}^{T}h^{-1}) \wh{\phi_{1}}({}^{T}hg) dh \\
&= \int_{G} \wh{\phi_{2}}(h) \wh{\phi_{1}}(h^{-1}g)dh = (\wh{\phi_{2}} * \wh{\phi_{1}})(g). 
\end{align*}
By the way, Cartan's decomposition theorem allow us to decompose $g$ as $g = \kappa_{1} d\kappa_{2}$ where $\kappa_{1}, \kappa_{2}\in K$ and $d$ is a diagonal matrix. Then 
$\phi(g) = \phi(d) = \wh{\phi}(d) = \wh{\phi}(g)$, so that $\iota = \id$ and $\phi_{1} * \phi_{2} = \phi_{2} * \phi_{1}$, i.e. $C_{c}^{\infty}(K\backslash G /K)$ is commutative. 
\end{proof}

For $n = 2$, we can prove a similar result when we consider the subalgebra of $C_{c}^{\infty}(G)$ where $K$ acts as a nontrivial character $\sigma$, i.e. $\phi(\kappa_{1} g\kappa_{2}) = \sigma(\kappa_{1}) \phi(g) \sigma(\kappa_{2})$. Let $C_{c}^{\infty}(K\backslash G/K, \sigma)$ be a subalgebra of such functions. 
\begin{proposition}
\label{commchar}
Let $G = \GL(2, \Rr)^{+}$ and $K = \SO(2)$. Let $\sigma$ be a character of $K$. Then $C_{c}^{\infty}(K\backslash G / K, \sigma)$ is commutative. 
\end{proposition}
\begin{proof}
The proof is almost same, but we use the following involution
$$
\wh{\phi}(g) = \phi\left(\pmat{-1}{}{}{1} \pre{T}{}g \pmat{-1}{}{}{1}\right). 
$$
\end{proof}

Now we can prove the uniqueness of the $K$-fixed vector. 
\begin{theorem}
\label{mult1char}
Let $G = \GL(n, \Rr)$ and $K = \rmO(n)$, or let $G = \GL(n, \Rr)^{+}$ and $K =\SO(n)$. Let $(\pi, \fraH)$ be an irreducible admissible unitary representation of $G$. Then $\dim \fraH^{K} \leq 1$. 
Similarly, $\dim \fraH_{k} \leq 1$ for each $k\in \Zz$, where $\fraH_{k} = \{v\in \fraH\,:\, \pi(\kappa_{\theta})v = \sigma_{k}(\kappa_{\theta})v\}$ for $\sigma_{k}(\kappa_{\theta}) = e^{ik\theta}$. 
\end{theorem}
\begin{proof}
By admissibility, we know that $\fraH^{K}$ is finite dimensional. $C_{c}^{\infty}(K\backslash G/K)$ can be realized as a commutative family of normal operators on a finite dimensional space, which are simultaneously diagonalizable. Therefore there is a one dimensional invariant subspace $V_{0}$ of $\fraH^{K}$, which should be whole $\fraH^{K}$ by irreducibility. 
The proof is almost same for $\fraH_{k}$ except that we use commutativity of $C_{c}^{\infty}(K\bs G/K, \sigma_{k})$ instead of $C_{c}^{\infty}(K\backslash G/K)$.
\end{proof}
Note that the admissibility condition is unnecessary because any irreducible unitary representation is admissible (as we mentioned above). 


\subsection{$(\frag, K)$-module for $\GL(2, \Rr)$ and classification}
Now we can define the $(\frag, K)$-module, which is a thing what we really want to study. In some sense, the subspace $\fraH^{\infty}$ of smooth vectors is still too big to study. We will consider much smaller space, the space of $K$-finite vectors $\fraH_{\fin}$, which is also dense in $\fraH$ but much easier to study algebraically. 

\begin{definition}
Let $(\pi, \fraH)$ be an admissible representation of $G = \GL(n, \Rr)$ or $\GL(n, \Rr)^{+}$. 
We may assume that $\sigma|_{K}$ is a unitary representation of $K$, so that $\sigma|_{K}$ decomposes as a Hilbert space direct sum of the isotypic parts $\fraH(\sigma)$ for each $\sigma\in \wh{K}$. 
Now let $\fraH_{\fin}$ be the algebraic direct sum of the $\fraH(\sigma)$. 
We call $f\in \fraH_{\fin}$ as $K$-finite vectors. 
\end{definition}

\begin{proposition}
For $f\in \fraH$, TFAE:
\begin{enumerate}
\item $f\in \fraH_{\fin}$.
\item $\langle \pi(\kappa)f\,:\, \kappa\in K\rangle$ is finite dimensional.
\item $\langle Xf\,:\, X\in \mathfrak{k}\rangle$ is finite dimensional (here $\mathfrak{k} = \mathrm{Lie}(K)$). 
\end{enumerate}
\end{proposition}
\begin{proposition}
Let $(\pi, \fraH)$ be an admissible Hilbert space representation of $G = \GL(n, \Rr)$ or $G = \GL(n, \Rr)^{+}$. 
The $K$-finite vectors are smooth, and $\fraH_{\fin}$ is dense $G$-invariant subspace of $\fraH^{\infty}$.
\end{proposition}
\begin{proof}
Let $\fraH_{0} = \fraH^{\infty} \cap \fraH_{\fin}$. 
We will first show that $\fraH_{0}$ is dense in $\fraH^{\infty}$. 
For given $f\in \fraH$, we will find suitable $\phi \in C^{\infty}(G)$ such that $\pi(\phi)f$ is sufficiently close to $f$ and $\pi(\phi)f \in \fraH_{0}$. 
To do this, let $U$ be a small open neighborhood of the identity in $G$ and let $\epsilon >0$ be a given constant. 
Choose $U_{1} \subset U$ and $V\subset K$ such that $VU_{1} \subset U$. 
Let $\phi_1$ be a smooth positive-valued function with $\supp(\phi_1)\subset U_1$ and $\int_{F} \phi_{1}(g) dg = 1$. Also, by Peter-Weyl theorem, we can find a matrix coefficient $\phi_0$ of a finite dimensionalunitary representation $(\rho, R)$ of $K$ such that $\int_{K} \phi_0(\kappa)d\kappa = 1$ and $\int_{K\bs V} |\phi_{0}(\kappa)| d\kappa < \epsilon$. 
Now let
$$
\phi(g) := \int\dpl{K} \phi_{0}(\kappa) \phi_{1}(\kappa^{-1}g)d\kappa.
$$
Then one can check that $\int_{G\bs U} |\phi(g)|dg < \epsilon$, so that $\pi(\phi)f$ is sufficiently close to $f$. 
To show that $\pi(\phi)f$ is $K$-finite, let $\phi_{0}(\kappa) = \bra{\rho(\kappa)\xi}{\eta}$ where $\xi, \eta$ are vectors in $R$. 
Then for $\kappa_1\in K$, we have
\begin{align*}
\phi_{1}(\kappa^{-1}g) = \int\dpl{K} \dbra{\rho(\kappa)\xi}{\rho(\kappa_1)\eta} \phi_1(\kappa^{-1}g)d\kappa
\end{align*}
so the space of functions $\phi(\kappa_{1}^{-1}g)$ lies in the finite dimensionalspace spanned by functions of the form 
$$
g\mapsto \int\dpl{K} \dbra{\rho(\kappa)\xi}{\zeta} \phi_{1}(\kappa^{-1}g)d\kappa, \quad \zeta\in R.
$$
This is a finite dimensionalspace of functions, so the space spanned by the vectors 
$$
\pi(\kappa_1) \pi(\phi)f = \int\dpl{G}\phi(g)\pi(\kappa_{1}g)fdg = \int\dpl{G} \phi(\kappa_{1}^{-1}g)\pi(g)fdg
$$
is finite dimensional. Hence $\pi(\phi)f\in \fraH_{\fin}$ by the previous proposition. This shows $\fraH_0$ is dense in $\fraH$. 

To show $\fraH_\fin \subseteq \fraH^{\infty}$, it is enough to show that $\fraH_{0}(\sigma) = \fraH(\sigma)$ for all irreducible representation $\sigma$ of $K$. 
Clearly, $\fraH_{0}(\sigma) \subseteq \fraH(\sigma)$, and if they are not same for some $\sigma$, then we can find $0\neq f\in \fraH(\sigma)$ orthogonal to $\fraH_{0}(\sigma)$, 
Then this $f$ is orthogonal to $\fraH_{0}(\tau)$ for all $\tau \neq \sigma$, which contradicts to the denseness of $\fraH_{0}$ in $\fraH^{\infty}$. 

For $\frag$-invariance, let $f\in R\subset \fraH$ be a $K$-finite vector where $R$ is a finite dimensional  $\fraK$-invariant subspace. 
Let $R_{1}$ be a space generated by $Yf$ for $Y\in \frag$ and $f\in R$, which is also a finite dimensional space. For $X\in \fraK$ and $Y\in \frag$, $X(Y\phi) = [X, Y]\phi + Y(X\phi)$ shows that $R_{1}$ is $\fraK$-invariant so $Yf$ is a $K$-finite vector. 
\end{proof}
Motivated by this, we define a notion of $(\frag, K)$-module, which is a vector space of $K$-finite vectors with compatible $\frag, K$ actions.
\begin{definition}
Let $G, K, \frag, \fraK$ as above. A vector space $V$ with representations $\pi$ of $K$ and $\frag$ is called $(\frag, K)$-module if
\begin{enumerate}
\item $V$ is $K$-finite, i.e. $V$ decomposes into an algebraic direct sum of finite dimensional invariant subspaces under the action of $K$. 
\item The representations of $\frag$ and $K$ are compatible in the sense that 
$$
\pi(X)f = \frac{d}{dt}\Big|_{t=0} \pi(\exp(tX))f
$$
for all $f\in V$ and $X\in \fraK$. 
\item The representations are compatible with adjoint action in the sense that
$$
\pi(g) \pi(X)\pi(g^{-1}) f = \pi(\Ad(g)X)f
$$
for all $f\in V$, $g\in K$, and $X\in \frag$.
\end{enumerate}
\end{definition}


For example, if $(\pi, \fraH)$ is an admissible representation of $\GL(2, \Rr)$, then $\fraH_{\fin}$ is a $(\frag, K)$-module. 
We will classify all the irreducible admissible $(\frag, K)$-module for $\GL(2, \Rr)$.
First, we will do for $G = \GL(2, \Rr)^{+}$ with $K = \SO(2)$, and modify it to get the result for $\GL(2, \Rr)$ with $K = \rmO(2)$. 

Let $V$ be a irreducible admissible $(\frag, K)$-module, so that it can be decomposed as an algebraic sum of isotypic parts 
$$
V = \bigoplus _{\sigma} V(\sigma)\quad (\text{algebraic sum})
$$
where each $V(\sigma)$ is finite dimensional. 
Since $K = \SO(2)$ is abelian, all the irreducible representations are 1-dimensional, and they are parametrized by integers as $\sigma_{k}(\kappa_{\theta}) = e^{ik\theta}$. Hence we can write $V$ as
$$
V = \bigoplus_{k\in \Zz} V(k)
$$
where $V(k) = V(\sigma_{k})$. 
Each $V(k)$ is at most 1-dimensional by Theorem \ref{mult1char}.   Now we can extend the $\frag$-action to $U\frag_{\Cc}$-action naturally. 
The set $\Sigma = \{k\in \Zz\,:\, V(k)\neq 0\}$ is called the set of $K$-types. 
We have the following Schur's lemma for $(\frag, K)$-modules. 
\begin{proposition}
Let $V$ be an irreducible admissible $(\frag, K)$-module. If $D\in Z(U\frag_{\Cc})$ is an element in a center of $U\frag_{\Cc}$, then $D$ acts as a scalar on $V$. 
\end{proposition}
\begin{proof}
We can naturally extend the adjoint action $\Ad$ of $G$ on $\frag$ to $U\frag_{\Cc}$ by 
$$
\Ad(g) (x_{1}\otimes \cdots \otimes x_{r}) = \Ad(g)x_{1} \otimes \cdots \otimes \Ad(g)x_{r}. 
$$
One can check that $D$ is fixed by this action by the third condition of $(\frag, K)$-module, so that $\pi(\kappa)\circ D = D\circ \pi(\kappa)$ for $\kappa\in K$. 
Consequently, the isotypic subspaces $V(\sigma)$ are stable under $D$. 
Choose any nonzero $V(\sigma)$. Since it has finite dimension, there exists a nonzero eigenvector $x_{0}\in V(\sigma)$ with an eigenvalue $\lambda$. Let $V_{0}\subseteq V(\sigma)$ be an eigenspace of $\lambda$. 
Since $D$ is in the center, it commutes with the action of $\frag$ and $K$, so that $V_{0}$ is a nonzero invariant subspace. Thus we have $V = V(\sigma) = V_{0}$. 
\end{proof}
This proposition shows that the elements $Z, \Delta$ acts as scalars on $V$. (This will be the parameter to classify $(\frag, K)$-modules later.) 
The following proposition gives a description how the elements $R, L, H, Z, \Delta \in U\frag_{\Cc}$ acts . 

\begin{proposition}
Let $V$ be an irreducible admissible $(\frag, K)$-module for $\GL(2, \Rr)^{+}$. 
\begin{enumerate}
\item $V(k)$ is the eigenspace for $H$ with an eigenvalue $k$. 
\item $R(V(k)) \subseteq V(k+2)$ and $L(V(k))\subseteq V(k-2)$. 
\item If $0\neq x\in V(k)$, then $V(k) = \Cc. x, V(k+2n) = \Cc. R^{n}x, V(k-2n) =\Cc. L^{n}x$ for $n>0$ and 
$$
V = \Cc. x \oplus \bigoplus _{n>0} \Cc. R^{n}x \oplus \bigoplus_{n>0} \Cc. L^{n}x.
$$
\item $\dim V(k) \leq 1$ and if $V(k), V(l)$ are both nonzero, then $k\equiv l\Mod{2}$. 
\item Let $\lambda$ be an eigenvalue of $\Delta$ on $V$. If $x\in V(k)$, then 
$$
LRx=\left( -\lambda - \frac{k}{2}\left(1 + \frac{k}{2}\right)\right) x, \quad RLx = \left( - \lambda + \frac{k}{2}\left( 1- \frac{k}{2}\right)\right)x.
$$
\item Let $\lambda$ be an eigenvalue of $\Delta$ on $V$. If $0\neq x\in V(k)$ and $Rx = 0$, then $\lambda = -\frac{k}{2}\left( 1 +\frac{k}{2}\right)$, while if $Lx = 0$, then $\lambda = \frac{k}{2}\left(1-\frac{k}{2}\right)$. 
\item Suppose that $\lambda = \frac{k}{2}\left( 1-\frac{k}{2}\right)$ and $x\in V(l)$. If $Rx =0$, then either $l =-k$ or $l = k-2$, and if $Lx = 0$, then either $l = k$ or $l = 2-k$. 
\end{enumerate}
\end{proposition}
\begin{proof}
Every statement follows form the relations among $R, L, H$ and $\Delta$. 
\end{proof}
By the proposition, we have that the set of $K$-types of $V$ is all even or all odd. This defines a parity of $V$, even or odd. 
The following theorem tells us the uniqueness of representations, whether we don't know the existence yet. 
\begin{theorem}
Let $\lambda, \mu$ be complex numbers. 
\begin{enumerate}
\item Assume that $\lambda \neq \frac{k}{2}\left( 1-\frac{k}{2}\right)$ for all $k$ even (resp. odd). Then There exists at most one isomorphism class of even (resp. odd) $(\frag, K)$-modules $V$ such that $\Delta, Z$ acts as scalars $\lambda, \mu$. For such $V$, the set of $K$-types consists of all even (resp. odd) $k$. 
\item Assume that $\lambda = \frac{k}{2}\left(1-\frac{k}{2}\right)$ for some integer $k\geq 1$. Then there are three possible sets of $K$-types:
\begin{align*}
\Sigma^{+}(k) &= \{l\in \Zz\,:\, l\equiv k\Mod{2}, l\geq k\} \\
\Sigma^{-}(k) &= \{l\in \Zz\,:\, l\equiv k\Mod{2}, l\leq -k\} \\
\Sigma^{0}(k) &= \{l\in \Zz\,:\, l\equiv k\Mod{2}, -k < l < k\}
\end{align*}
\end{enumerate}
\end{theorem}
\begin{proof}
Basically, all of these follows from the previous proposition, 6 and 7. For the uniqueness, we will only show the first case. 
Let $V, V'$ be two irreducible admissible $(\frag, K)$-module with the same set of $K$-types. 
Choose $0\neq x\in V(k)$ and $0\neq x'\in V'(k)$, then $x, L^{n}x, R^{n}x$ (for $n>0$) form a basis of $V$, and similarly $x', L^{n}x', R^{n}x'$ ($n>0$) form a basis of $V'$. 
Now if we define $\phi:V\to V'$ by $\phi(x) = x', \phi(L^{n}x) = L^{n}x'$ and $\phi(R^{n}x) = R^{n}x'$, then we can easily check that this is a nonzero $(\frag, K)$-module homomorphism from $V$ to $V'$. 
\end{proof}


Now we will give a construction of such representation with given parameters, which will finish the classification. Let $\epsilon = 0$ or $1$, which represents parity of a representation, and let $s_{1}, s_{2}$ be two complex numbers. Let $\lambda = s(1-s)$ and $\mu = s_{1} + s_{2}$, where $s = \frac{1}{2}(s_{1} - s_{2} + 1)$. As you expect, these will be scalars corresponding to $\Delta$ and $Z$. 

\begin{definition}
$H^{\infty}(s_{1}, s_{2}, \epsilon)$ be the space of smooth functions $f:\GL(2, \Rr)^{+}\to \Cc$ satisfying 
\begin{align*}
f\left(\pmat{y_{1}}{x}{}{y_{2}} g\right) &= y_{1}^{s_{1} + 1/2}y_{2}^{s_{2} - 1/2}f(g), \quad y_{1}. y_{2} >0 \\
f\left( \pmat{-1}{}{}{-1}g \right) &= (-1)^{\epsilon} f(g).
\end{align*}
We let $G$ acts by right translation. We also give a Hermitian inner product by 
$$
\langle f_{1}, f_{2}\rangle = \frac{1}{2\pi} \int_{0}^{2\pi} f_{1}(\kappa_{\theta})\overline{f_{2}(\kappa_{\theta})} d\theta
$$
and let $H(s_{1}, s_{2}, \epsilon)$ be the Hilbert space completion of $H^{\infty}(s_{1}, s_{2}, \epsilon)$. 
\end{definition}
Note that the right translation action (regular action) extends to $H(s_{1}, s_{2}, \epsilon)$. We can also prove that $H^{\infty}(s_{1}, s_{2}, \epsilon)$ is the space of smooth vectors for this representation. 
By Iwasawa decomposition, we have 
$$
f(g) = f\left( \pmat{u}{}{}{u} \pmat{y^{1/2}}{xy^{-1/2}}{}{y^{-1/2}} \kappa_{\theta}\right) = u^{s_{1} + s_{2}} y^{s} f(\kappa_{\theta})
$$
for $f\in H(s_{1}, s_{2}, \epsilon)$, so each $f$ is determined by its value on $K = \SO(2)$, and $f|_{K}$ can be any smooth function, subject to the condition $f(\kappa_{\theta+ \pi}) = (-1)^{\epsilon}f(\kappa_{\theta})$. 

In fact, the representation is an example of an example of induced representation. 
For a locally compact Hausdorff group $G$ and its subgroup $H$, we can obtain a representation of $G$ from a representation of $H$ in a canonical way: if $(\rho, V)$ is a representation of $H$, then define 
$$
V^{G} = \left\{f:G\to \Cc\,:\, f(hg) = \left( \frac{\delta_{H}(h)}{\delta_{G}(g)}\right)^{1/2}\rho(h)f(g)\right\}
$$
where $\delta_{H}, \delta_{G}$ are modular characters. If we give $G$-action on $V^{G}$ by right translation, then this gives a representation of $G$. We denote such representation by $\Ind_{H}^{G} (\rho)$. 
Now let $G = \GL(2, \Rr)^{+}$, $H = B(\Rr)^{+}$ (the subgroup of upper triangular matrices in $G$), and let $\chi:B(\Rr)^{+}\to \mathbb{C}^{\times}$ be a character defined as 
$$
\chi\pmat{y_{1}}{x}{}{y_{2}} = \sgn(y_{1})^{\epsilon}|y_{1}|^{s_{1}}|y_{2}|^{s_{2}}.
$$
Then, by definition, the representation $H(s_{1}, s_{2}, \epsilon)$ is just $\Ind_{H}^{G}(\chi)$. 
Note that $G$ is a unimodular group (so that $\delta_{G}$ is trivial) and 
$$
\delta_{B(\Rr)^{+}} \pmat{y_{1}}{x}{}{y_{2}} = \frac{y_{1}}{y_{2}}. 
$$

Now we want to study $(\frag, K)$-module of $K$-finite vectors in $\fraH = H(s_{1}, s_{2}, \epsilon)$. 
If $l\equiv \epsilon\Mod{2}$, then there exists a unique $f_{l}\in \fraH$ such that $f_{l}(\kappa_{\theta}) = e^{il\theta}$, which satisfies $\rho(\kappa_{\theta})f_{l} = e^{il\theta}f_{l}$. 
Iwasawa decomposition gives an explicit description of $f_{l}$: 
$$
f_{l}\left( \pmat{u}{}{}{u} \pmat{y^{1/2}}{xy^{-1/2}}{}{y^{-1/2}} \kappa_{\theta}\right) = u^{s_{1} + s_{2}} y^{s} e^{il\theta}.
$$
By the direct computation, we can show that this function satisfies the following relations:
\begin{proposition}
\begin{align*}
Hf_{l} &= lf_{l} \\
Rf_{l} &= \left( s + \frac{l}{2}\right) f_{l+2} \\
Lf_{l} &= \left( s - \frac{l}{2}\right) f_{l-2} \\
\Delta f_{l} &= \lambda f_{l} \\
Zf_{l} &= \mu f_{l}
\end{align*}
where $\lambda = s(1-s), \mu = s_{1} + s_{2}, s = \frac{1}{2}(s_{1} - s_{2} +1)$. 
\end{proposition}
Now, as you expect, these representations give examples of the previous representations with two parameters $\lambda, \mu$ and $K$-type. 
The above $f_{l}$'s generate the space of $K$-finite vectors. 
\begin{theorem}
Let $s_{1}, s_{2}, s, \lambda, \mu, \epsilon$ are given as above, and let $\fraH$ be the $(\frag, K)$-module of $K$-finite vectors in $H(s_{1}, s_{2}, \epsilon)$, where $\Delta, Z$ acts as $\lambda, \mu$, respectively. 
\begin{enumerate}
\item If $s$ is not of the form $\frac{k}{2}$ for $k\equiv \epsilon\Mod{2}$, then $\fraH$ is irreducible. 
\item If $s\geq \frac{1}{2}$ and $s = \frac{k}{2}$ for some integer $k\geq 1$ with $k\equiv \epsilon\Mod{2}$, then $\fraH$ has two irreducible invariant subspaces $\fraH_{+}, \fraH_{-}$, with the set of $K$-types as $\Sigma^{+}(k), \Sigma^{-}(k)$, respectively. The quotient $\fraH/(\fraH_{+}\oplus \fraH_{-})$ is irreducible with a set of $K$-type $\Sigma_{0}(k)$ for $k\neq 1$, where zero for $k=1$. 
\item If $s\leq \frac{1}{2}$ and $s = 1-\frac{k}{2}$ for some integer $k\geq 1$ with $k\equiv \epsilon\Mod{2}$, then $\fraH$ has an invariant subspace $\fraH_{0}$ which is irreducible and whose set of $K$-types is $\Sigma^{0}(j)$. 
The quotient $\fraH/\fraH_{0}$ decomposes into two irreducible invariant subspaces $\fraH_{+}$ and $\fraH^{-}$, with the set of $K$-types $\Sigma^{+}(k), \Sigma^{-}(k)$ respectively. 
\end{enumerate}
\end{theorem}
In other words, this gives a classification of $(\frag, K)$-module for $\GL(2, \Rr)^{+}$. 
\begin{theorem}[Classification of $(\frag, K)$-module for $\GL(2, \Rr)^{+}$]
Let $\lambda, \mu$ be given complex numbers and $\epsilon \in \{0, 1\}$. 
\begin{enumerate}
\item If $\lambda$ is not of the form $\frac{k}{2}\left( 1- \frac{k}{2}\right)$ for $k\equiv \epsilon \Mod{2}$, then there exists a unique irreducible admissible $(\frag, K)$-module of parity $\epsilon$ on which $\Delta$ and $Z$ act by scalars $\lambda$ and $\mu$, and we have $\Sigma = \{k\,:\, k\equiv \epsilon\Mod{2}\}$ in this case. 
\item If $\lambda = \frac{k}{2}\left( 1 - \frac{k}{2} \right)$ for some $k\geq 1$, $k\equiv \epsilon\Mod{2}$, then there exists three irreducible admissible  $(\frag, K)$-modules of parity $\epsilon$ on which $\Delta$ and $Z$ act by scalars $\lambda$ and $\mu$, except that if $k = 1$, there are only two. 
The set of $K$-types are $\Sigma ^{\pm}(k)$ and (if $k>1$) $\Sigma^{0}(k)$. 
\end{enumerate}
\end{theorem}
When $\lambda$ is not of the form $\frac{k}{2}\left(1-\frac{k}{2}\right)$, then the equivalence class of irreducible admissible $(\frag, K)$-modules of $\GL(2, \Rr)^{+}$ with given $\lambda, \mu$ are denoted by $\calP_{\mu}(\lambda, \epsilon)$. When $\mu = 0$, we denote it as $\calP(\lambda, \epsilon)$ and it is called \emph{principal series}. (By tensoring with a suitable power of determinant, we can assume $\mu = 0$ easily.) Later, we wiil check that the representation is unitarizable if and only if $\lambda\in \Rr$ and $\lambda \geq 1/4$, so we will concentrate on this case. 

The finite dimensional representation with a set of $K$-types $\Sigma^{0}(k)$ can be realized as a space of polynomials: consider the space of homogeneous polynomials of degree $k-2$ in two variables $x_{1}, x_{2}$, and let $G = \GL(2, \Rr)^{+}$ acts on the space by 
$$
\pi(g)f(x_{1}, x_{2}) = \det(g)^{(\mu-k-2)/2} f((x_{1}, x_{2})g),
$$
which is a degree $k-1$ irreducible admissible representation where $Z$ acts as the scalar $\mu$. 
This will not appear again since it is \emph{not unitarizable}. (We will prove that the only finite dimensional unitarizable representation is 1-dimensional, which factors through the determinant map.)

If $k>1$, we have irreducible admissible representations with set of $K$-types as $\Sigma^{\pm}(k)$, and equivalence class of these representations will be denoted as $\calD^{\pm}_{\mu}(k)$ and called the \emph{discrete series}. When $k = 1$, the representations $\calD_{\mu}^{\pm}(1)$ are called \emph{limit of discrete series}. 


To classify $(\frag, K)$-modules of $\GL(2, \Rr)$, we need some modification. We can check that the representation of $\rmO(2)$ has a symmetric property: the set of $K$-types is symmetric so that $k\in\Sigma$ if and only if $-k\in \Sigma$. Hence $\calD_{\mu}^{\pm}(k)$ cannot be extended to $\GL(2, \Rr)$, but $\calD_{\mu}^{+}(k)\oplus \calD_{\mu}^{-}(k)$ can be. We will denote the latter one by $\calD_{\mu}(k)$, with $(\frag, \rmO(k))$-module structure. 

For the construction of principal series reresentation of $\GL(2, \Rr)$, we define $\chi: B(\Rr)\to \Cc^{\times}$ as 
$$
\chi\pmat{y_{1}}{x}{}{y_{2}} = \chi_{1}(y_{1})\chi_{2}(y_{2})
$$
for $\chi_{i}(y) = \sgn(y)^{\epsilon_{i}}|y|^{s_{i}}$, where $\epsilon_{i} \in \{0, 1\}$ and $\epsilon_1 + \epsilon_2 \equiv \epsilon \Mod{2}$. 
Then we denote $\Ind_{B(\Rr)}^{\GL(2, \Rr)} (\chi)$ as $H(\chi_1, \chi_2)$, and we will denote by $\pi(\chi_1, \chi_2)$ the underlying $(\frag, \rO(2))$-module of $K$-finite vectors. 
Note that $H(\chi_1, \chi_2) \simeq H(s_1, s_2, \epsilon)$ since each function in $H(\chi_1, \chi_2)$ is determined by its restriction to $\GL(2, \Rr)^{+}$. 
So there are two extensions of the $\GL(2, \Rr)^{+}$-module structure on $H(s_1, s_2, \epsilon)$ to a $\GL(2, \Rr)$-module structure (corresponds to the choice of $(\epsilon_1, \epsilon_2)$), and the same is true for the corresponding $(\frag, K)$-modules. 
\begin{theorem}[Classification of $(\frag, K)$-module for $\GL(2, \Rr)$]
\begin{enumerate}
\item The finite \\dimensional representations have a form of $\Sym^{n}\rho_{0} \otimes (\chi\circ \det)$, where $\rho_{0}$ is the standard representation and $\chi:\Rr^{\times}\to \Cc$ a character. 
\item If $\chi_{1}, \chi_{2}$ are characters of $\Rr^{\times}$ such that $\chi_{1}\chi_{2}^{-1} \neq \sgn(\cdot)^{\epsilon}|\cdot|^{k-1}$, where $\epsilon\equiv k\Mod{2}$, then $\pi(\chi_{1}, \chi_{2})$ is an irreducible admissible $(\frag, \rmO(2))$-module.
\item If $\mu \in \Cc$  and $k\geq 1$ an integer, then we have discrete series $\calD_{\mu}(k)$ ($k\geq 2$) and limits of discrete series $\calD_{\mu}(1)$. 
\end{enumerate}
\end{theorem}

\subsection{Unitaricity and intertwining integrals}
Now we will see which representations in the above list are unitarizable. For some special case (so-called complementary series), we will show that the representation is unitary by using the intertwining integral, which is an hidden explicit isomorphism between two isomorphic $(\frag, K)$-modules. 


The following theorem tells us that induced representation of unitary representation is again unitary in some special case. 
\begin{theorem}
Let $G$ be a unimodular locally compact group, $P$ be a closed subgroup, and $K$ be a compact subgroup such that $PK = G$, so that $P\bs G$ is compact. If $(\sigma, V)$ is a unitary representation of $P$ with an inner product $\langle, \rangle$, then the induced representation $\Ind_{P}^{G}(\sigma)$ is also unitary with respect to the inner product 
$$
\dbra{f_{1}}{f_{2}} = \int_{K} \langle f_{1}(\kappa), f_{2}(\kappa)\rangle d\kappa. 
$$ 
\end{theorem}
\begin{proof}
It is easy to check that the function $g\mapsto \bra{f_{1}(g)}{f_{2}(g)}$ is in $C(P\bs G, \delta)$, i.e. satisfies $f(pg) =\delta(p)f(g)$ for all $p\in P$ and $g\in G$. 
One can prove that the linear functional $I:C(P\bs G, \delta)\to \Cc$ defined as $I(f) = \int_{K} f(\kappa)d\kappa$ is $G$-invariant under the right regular representation, by showing that the map $\Lambda:C_{c}(G)\to C(P\bs G, \delta), \phi\mapsto (g\mapsto \int_{P} \phi(pg)dp)$, is surjective and $I(\Lambda f) = \int_{G}f(g)dg$. 
For details, see Lemma 2.6.1 in \cite{bu}. 
\end{proof}
Using this, we can prove that there are some class of representations that are induced from unitary representation, so is unitarizable. 
\begin{theorem}
Let $\mu$ be a pure imaginary number, $\lambda\geq \frac{1}{4}$  be a real numbers, $\epsilon\in \{0, 1\}$, and assume that $\lambda$ is not of the form $\frac{k}{2}\left(1-\frac{k}{2}\right)$ for any integer $k\equiv \epsilon\Mod{2}$. Then $\calP_{\mu}(\lambda, \epsilon)$ contains a unitary representation of $\GL(2, \Rr)^{+}$. 
\end{theorem}
\begin{proof}
With the assumption, we can easily check that $s_{1}, s_{2}$ satisfying $\mu = s_{1} + s_{2}, s = \frac{1}{2}(s_{1}-  s_{2} + 1), \lambda = s(1-s)$ are all pure imaginary. Then the character $\chi:B(\Rr)^{+}\to \Cc^{\times}$ defined as
$$
\chi\pmat{y_{1}}{x}{}{y_{2}} = \sgn(y_{1})^{\epsilon}|y_{1}|^{s_{1}}|y_{2}|^{s_{2}}
$$
is unitary and the induced representation that is contained in the class $\calP_{\mu}(\lambda, \epsilon)$ is also unitary by the previous theorem. 
\end{proof}

If $(\pi, \fraH)$ is a unitary representation of $G$ and  $X\in \frag$, then the action of $X$ on $\fraH^{\infty}$ is skew-symmetric, i.e. $\langle Xv, w\rangle = -\langle v, Xw\rangle$ for all $v, w\in \fraH^{\infty}$.  
Especially, we have $\langle Rv, w\rangle = -\langle v, Lw\rangle$. 
The following theorem give some necessary conditions for unitaricity. 
\begin{theorem}
Let $(\pi, \fraH)$ be a unitary representation of $G = \GL(2, \Rr)^{+}$. 
\begin{enumerate}
\item If $Z$ and $\Delta$ in $U\frag$ acts by scalars $\mu$ and $\lambda$, then $\mu\in i\Rr$ and $\lambda\in \Rr$. 
\item Assume that $(\pi, \fraH)$ is in the class of $\calP_{\mu}(\lambda, \epsilon)$, where $\lambda$ is not of the form $\frac{k}{2}\left(1-\frac{k}{2}\right)$ for integer $k\equiv \epsilon\Mod{2}$. If $\epsilon = 0$, then $\lambda > 0$, and if $\epsilon = 1$, then $\lambda > \frac{1}{4}$. 
\end{enumerate}
\end{theorem}
\begin{proof}
1 follows from the fact that $Z\in \frag$, so action is skew-symmetric, and the action of $\Delta$ is symmetric. For 2, we know that $\fraH(k)\neq 0$ for all $k\equiv\epsilon\Mod{2}$. From $-4\Delta - H^{2} + 2H = 4RL$, $Hf_{k} = kf_{k}$ (where $0\neq f_{k}\in \fraH(k)$), and $\langle RL f_{\epsilon}, f_{\epsilon}\rangle = \langle Lf_{\epsilon}, Lf_{\epsilon}\rangle  >0$, we get $-4\lambda - \epsilon^{2} + 2\epsilon <0$ which gives the results. 
\end{proof}

From the above theorems, we know unitarizability of $\calP_{\mu}(\lambda, \epsilon)$ except for $\epsilon = 0$ and $0 < \lambda < \frac{1}{4}$. We will also show that these representations are also unitary, but induced from nonunitary representations of Borel subgroup. Such representations are called \emph{complementary series}, and the corresponding eigenvalues are \emph{exceptional eigenvalues}. 

To construct such representation, we will use intertwining integral. We know that $H(s_{1}, s_{2}, \epsilon)$ and $H(s_{2}, s_{1}, \epsilon)$ are isomorphic, when they are irreducible,  as $(\frag, K)$-module since they have the same $\lambda$ and $\mu$. 
Also, when they are not irreducible (when $\lambda = \frac{k}{2}\left(1-\frac{k}{2}\right)$) they are not isomorphic, but their composition factors are isomorphic. 
We will construct an intertwining map between those to representations as an integral. 

For $s\in \Cc$, the operators $M(s)$ are defined by 
$$
(M(s)f)(g) = \int_{-\infty}^{\infty} f\left(\pmat{0}{-1}{1}{0}\pmat{1}{x}{0}{1}g\right)dx.
$$
The next proposition shows that this is the desired intertwining map, when the integral converges. 
\begin{proposition}
Let $f\in H^{\infty}(s_{1}, s_{2}, \epsilon)$ and suppose $\Re s >\frac{1}{2}$ where $s = \frac{1}{2}(s_{1} - s_{2} + 1)$, so that $\Re s_{1} > \Re s_{2}$. Then the integral $M(s)f$ is convergent and define an intertwining map 
$$
M(s):H^{\infty}(s_{1}, s_{2}, \epsilon) \to H^{\infty}(s_{2}, s_{1}, \epsilon).
$$
Also, it sends a $K$-finite vector to a $K$-finite vector, which therefore induces a homomorphism of $(\frag, K)$-modules $H(s_{1}, s_{2}, \epsilon)_{\fin} \to H(s_{2}, s_{1}, \epsilon)_{\fin}$. 
\end{proposition}
\begin{proof}
It is almost direct to check that the map is indeed an intertwining map, if we know that the intertwining map is convergent. 
For the convergence, we only need to check convergence for $g = 1$ (since $M(s)$ is an intertwining map). The identity  
\begin{align*}
\pmat{}{-1}{1}{}\pmat{1}{x}{}{1} &= \pmat{\Delta_{x}^{-1}}{-x\Delta_{x}^{-1}}{}{\Delta_{x}} \kappa_{\theta(x)} \\
\Delta_{x} &= \sqrt{1+x^{2}}, \quad \theta(x) = \arctan\left(-\frac{1}{x}\right). 
\end{align*}
gives
$$
(M(s)f)(1) = \int_{-\infty}^{\infty} (1+x^{2})^{-s}f(\kappa_{\theta(x)})dx, 
$$
and by the boundedness of $f$ on $K$, the integral converges if 
$$
\int_{-\infty}^{\infty} \frac{1}{|(1+x^{2})^{s}|} dx
$$
converges, which is true for $\Re s>\frac{1}{2}$. 
To check that $M(s)f\in H^{\infty}(s_{2}, s_{1}, \epsilon)$, it is enough to check the following equations
\begin{align*}
(M(s)f)\left(\pmat{1}{\xi}{}{1}g\right) &= (M(s)f)(g) \\
(M(s)f)\left(\pmat{y_{1}}{}{}{y_{2}}g\right) &= |y_{1}|^{s_{2} + \frac{1}{2}} |y_{2}|^{s_{1} - \frac{1}{2}} (M(s)f)(g)
\end{align*}
for $\xi\in \Rr$ and $y_{1}, y_{2}>0$, which can be checked by direct computation (with some substitutions). 
Smoothness and $K$-finiteness are also can be easily checked from
$$
(M(s)f)(\kappa_{t}) = \int_{-\infty}^{\infty} (1+x^{2})^{-s}f(\kappa_{\theta(x)+t})dx.
$$
\end{proof}

We can also compute the effect of $M(s)$ on a $K$-finite vector. 
\begin{proposition}
If $\Re s>\frac{1}{2}$, we have
$$M(s)f_{k, s} = (-1)^{k} \sqrt{\pi} \frac{\Gamma(s)\Gamma\left( s- \frac{1}{2}\right)}{\Gamma\left(s+\frac{k}{2}\right)\Gamma\left(s-\frac{k}{2}\right)} f_{k, 1-s}.$$
\end{proposition}
\begin{proof}
It is enough to show for $g = 1$, which is equivalent to 
$$
\int_{-\infty}^{\infty} (1+x^{2})^{-s}\exp(ik\theta(x)) dx = (-1)^{k} \sqrt{\pi} \frac{\Gamma(s)\Gamma\left( s-\frac{1}{2}\right)}{\Gamma\left(s+\frac{k}{2}\right)\Gamma\left(s-\frac{k}{2}\right)}.
$$
Under the substitution $y = \frac{x-i}{x+i}$, the integral equals
$$
2i(-i)^{k} 4^{-s} \int_{C} (1-y)^{2s-2}(-y)^{\frac{k}{2}-s}dy
$$
where $C$ is a contour consisting of  unit circle centered at the origin and moves counterclockwise. 
For the convergence, we may assume $\Re(2s - 1), \Re(\frac{k}{2}-s) >0$, and use analytic continuation on $k$. 
If we deform the contour $C$ so that it proceeds directly from 1 to 0 along real axis, circles the origin in the counterclockwise direction, then returns to 1 along the real line, then the integral became
\begin{align*}
2i&(-i)^{k}4^{-s}[e^{-i\pi(s-k/2)} - e^{i\pi (s-k/2)}] \int_{0}^{1}(1-y)^{2s-2}y^{k/2-s}dy \\
&=(-1)^{k}\sqrt{\pi} \frac{\Gamma(s)\Gamma\left( s- \frac{1}{2}\right)}{\Gamma\left(s+\frac{k}{2}\right)\Gamma\left(s-\frac{k}{2}\right)},
\end{align*}
which follows from the Beta function identity and some other formulas of Gamma function. 
\end{proof}

Now we can prove that the complementary series is unitary. 
\begin{theorem}
Let $\mu\in i\Rr$ and $0<\lambda < \frac{1}{4}$. Then $\calP_{\mu}(\lambda, 0)$ contains a representative that is a unitary representation. 
\end{theorem}
\begin{proof}
For $s_{1}, s_{2}\in \Cc$, we have a Hermitian pairing 
\begin{align*}
H^{\infty}(s_{1}, s_{2}, \epsilon) &\times H^{\infty}(-\ol{s_{1}}, -\ol{s_{2}}, \epsilon) \to \Cc \\
(f, f')&\mapsto \int_{K} f(\kappa) \ol{f'(\kappa)} d\kappa
\end{align*}
which is $G$-invariant. 

Now assume that $s_{1} = -\ol{s_{2}}$ and $s_{2} = -\ol{s_{1}}$, so that $\mu = s_{1} + s_{2} \in i\Rr$ and $s = \frac{1}{2}(s_{1} - s_{2} + 1)\in \Rr$. Since $H^{\infty}(s_{2}, s_{1}, \epsilon) = H^{\infty} (-\ol{s_{1}}, -\ol{s_{2}}, \epsilon)$, we can define Hermitian pairing on $H^{\infty}(s_{1}, s_{2}, \epsilon)$ by 
$$
\langle f, f'\rangle = \int_{K} f(\kappa)\ol{i^{\epsilon}(M(s)f')(\kappa)}d\kappa, 
$$
which is $G$-invariant. We only need to show that this pairing is positive definite, and it follows from the following computation
$$
\langle f_{k, s}, f_{k, s}\rangle = (-1)^{\frac{k}{2}} \sqrt{\pi} \frac{\Gamma(s)\Gamma\left(s-\frac{1}{2}\right)}{\Gamma\left(s+\frac{k}{2}\right)\Gamma\left(s-\frac{k}{2}\right)}
$$
which is positive for $\frac{1}{2} <s < 1$ and even $k$. 
\end{proof}

In contrast, finite dimensional representations are not unitary in general. In fact, the easiest ones are the only one which are unitary. 
\begin{proposition}
The only irreducible finite dimensional unitary representation of $\GL(n, \Rr)^{+}$ are 1-dimensional character $g\mapsto \det(g)^{r}$ where $r\in i\Rr$. 
\end{proposition}
\begin{proof}
Finite dimensional unitary representation of $\GL(n, \Rr)^{+}$ can be regarded as a homomorphism $\pi:\GL(n, \Rr)^{+}\to U(m)$ where $m$ is the dimension of the representation. Since $U(m)$ is compact, image of $\pi$ is also compact. It is known that $\SL(n, \Rr)$  is simple for odd $n$ and $\mathrm{PSL}(n, \Rr) = \SL(n, \Rr)/\{\pm I\}$ is simple for even $n$, so the only compact homomorphic image of $\SL(n, \Rr)$ is the trivial group. 
Hence $\SL(n, \Rr)\subset \ker\pi$ and the representation factors through the determinant map. Now we know that the only unitary representation of $\Rr_{+}^{\times}$ are of the form $t\mapsto t^{r}$ for $r\in i\Rr$. 
\end{proof}
The only thing remain that we have to figure out is unitarizability of discrete series. We will prove that there is a unitary representation in the infinitesimal equivalence class $\calD^{\pm}(k)$ for $k>1$, by constructing such representation on a space of holomorphic functions on $\mathcal{H}$ which has bounded $L^{2}$-norm. 
We know that $\mu\in i\Rr$ if the representation is unitary, and we may assume $\mu = 0$ as before. 

\begin{theorem}
\begin{enumerate}
\item Let $\fraH$ be the space of holomorphic functions $f$ on the upper half plane $\mathcal{H}$ which satisfies
$$
\int_{\mathcal{H}} |f(z)|^{2} y^{k} \frac{dxdy}{y^{2}}<\infty.
$$
Define the $G = \GL(2, \Rr)^{+}$ action on $\fraH$ by 
$$
(\pi^{\pm}(g)f)(z)= (ad-bc)^{k/2} (\mp bz+d)^{-k} f\left( \frac{az\mp c}{\mp bz + d}\right), \quad g = \pmat{a}{b}{c}{d}.
$$
Then $(\pi^{\pm}, \fraH)$ are admissible unitary representations in $\calD^{\pm}(k)$. 
\item Let $Z$ be the center of $G = \GL(2, \Rr)^{+}$. Then the right regular representation of $G$ on $L^{2}(G/Z)$ contains an irreducible admissible representation in the class $\calD^{\pm}(k)$. 
\end{enumerate}
\end{theorem}
\begin{proof}
The automorphism $\iota:G\to G$ defined as 
$$
\iota\left(\pmat{a}{b}{c}{d}\right)= \pmat{a}{-b}{-c}{d}
$$
relates $\pi^{+}$ and $\pi^{-}$ by $\pi^{+}(g)= \pi^{-}(\iota(g))$. Thus it is sufficient to show that $(\pi^{-}, \fraH)$ is an irreducible admissible representation in $\calD^{-}(k)$. 

Define the representation $(\pi, \fraH)$ by 
$$
\pi(g) f= f|_{k}g^{-1}
$$
where $|_{k}$ is the weight $k$ slash operator, i.e.
$$
(f|_{k}g)(z) = \det(g)^{k/2} (cz+d)^{-k} f\left( \frac{az+b}{cz+d}\right)
$$
for $g\in \GL(2, \Cc)$. Then $\pi\simeq \pi^{-}$ since $\pi^{-}(g)f = \pi(w_{0}gw_{0}^{-1})f$ for $w_{0} = \smat{0}{-1}{1}{0}$. So it is enough to show that $(\pi, \fraH)$ is an irreducible admissible representation in $\calD^{-}(k)$. 

Let $s_{1} = -s_{2} = (k-1)/2$, so that $s = k/2$ and $\mu = 0$, and let $\epsilon\in \{0, 1\}$ with $\epsilon\equiv k\Mod{2}$. We can define a bilinear pairing $\dbra{\,}{}: H^{\infty}(s_{1}, s_{2}, \epsilon) \times H^{\infty} (-s_{1}, -s_{2}, \epsilon) \to \Cc$ by 
$$
\dbra{f_{1}}{f_{2}} = \int_{K} f_{1}(\kappa)f_{2}(\kappa) d\kappa
$$
which is $G$-equivariant. Now we define a map $\sigma:H^{\infty}(-s_{1}, -s_{2}, \epsilon)\to C^{\infty}(G)$ by 
$$
(\sigma f)(g) = \dbra{\rho(g)f_{k, s}}{f}, 
$$
then the function 
$$
(\Sigma f)(z) = y^{-k/2}(\sigma f)\left(\pmat{y^{1/2}}{xy^{-1/2}}{}{y^{1/2}}\right)
$$
is an holomorphic function on $\mathcal{H}$ that satisfies $(\Sigma f|_{k}g)(i) = (\sigma f)(g)$. 
(Holomorphicity follows from $L(\sigma f) =0$.)

We will now prove that 
$$
\Sigma f_{l, 1-s} = c(l) (z-i)^{-(i+k)/2}(z+i)^{(l-k)/2}
$$
for some constant $c(l)$ which is zero for $l>-k$. We have
$$
(\sigma f_{l, 1-s})(\kappa_{\theta}g) = \dbra{\rho(g)f_{k, s}}{\rho(\kappa_{\theta}^{-1})f_{l, 1-s}} = e^{-il\theta}(\sigma f_{l, 1-s})(g),
$$
and this implies that the function $\phi = \Sigma f_{l, 1-s}$ satisfies
$$
\phi|_{k}\kappa_{\theta} = e^{-il\theta}\phi. 
$$
If $C = -\frac{1+i}{2} \smat{i}{1}{i}{-1}$ (Cayley transform), $\psi:= \phi_{k}|C^{-1}$ is a function on the unit disk with
$$
\psi\Big|_{k} \pmat{e^{i\theta}}{}{}{e^{-i\theta}} = e^{-il\theta}\psi
$$
and if we consider the Taylor expansion of $\psi$, we get $\psi(w) = cw^{(-l-k)/2}$ for some constant $c$, which implies $\phi= c(l) (z-i)^{-(i+k)/2}(z+i)^{(l-k)/2}$ where $c(l) =0$ for $l>-k$. 
From this, the kernel of the map $\Sigma:H(-s_{1}, -s_{2}, \epsilon) \to C^{\omega}(\calH)$ contains the (reducible) invariant subspace $\langle f_{l, 1-s}\,:\, l\geq 2-k\rangle$, and we can check that $\Sigma f_{l, 1-s}$ are all square-integrable for $l\leq -k$ by using the explicit description, so the image lies in $\fraH$. Also, $\Sigma f_{l, 1-s}$ span $\fraH$ for $l\leq -k$ because as a function on the unit disk (via Cayley transform), power series expansion of a holomorphic function on the unit disk can be regarded as a Fourier expansion in terms of $\Sigma f_{l, 1-s}$.
This completes the proof of 1. 

For 2, note that the correspondence between $\sigma f$ and $\Sigma f$ is an isometry, and this gives a realization of $\calD^{-}(k)$ in the left regular representation of $G$ on $L^{2}(G/Z)$, and it can be transferred to the right regular representation by composing with $g\mapsto g^{-1}$. 
\end{proof}

The limits of the discrete series representation $\calD^{\pm}(1)$ also can be realized in a space of holomorphic functions on $\calH$ with the norm 
$$
|f|^{2} = \sup_{y>0}  \int_{-\infty}^{\infty} |f(x+iy)|^{2} dx,
$$
but we don't need this since they are subrepresentations of $H(0, 0, 1)$, which is already unitary. 

Until now, we studied which $(\frag, K)$-module arises from irreducible admissible unitary representation of $\GL(2, \Rr)^{+}$. The following theorem tells us that this actually classifies all the irreducible unitary representations. 
\begin{theorem}
\label{unigk}
Irreducible admissible unitary representation of $\GL(2, \Rr)^{+}$ is determined by the corresponding $(\frag, K)$-module. 
\end{theorem}
\begin{proof}
Let $(\pi, \fraH)$ and $(\pi', \fraH')$ be irreducible admissible unitary representations of $G = \GL(2, \Rr)^{+}$ such that the spaces $V = \fraH_{\fin}$ and $V' = \fraH_{\fin}'$ are isomorphic as $(\frag, K)$-modules, and let $\phi:V\to V'$ be an isomorphism. 
Decompose $V$ and $V'$ as $V = \oplus_{k} V(k), V' = \oplus_{k} V'(k)$ and choose $k$ so that $V(k)\neq 0$. 
Then we can find $0\neq x\in V(k)$ which satisfies $|x| = 1$, and by normalizing $\phi$ we can also assume that $|\phi(x)| = 1$. (Note that all the spaces $V(k)$ are at most 1-dimension.) 
Then 
$$
|Rx|^{2} = \langle Rx, Rx\rangle = -\langle LRx, x\rangle = \left(\lambda + \frac{k}{2}\left(1+\frac{k}{2}\right)\right) \langle x, x\rangle = \lambda + \frac{k}{2}\left(1+\frac{k}{2}\right), 
$$
and we get the same result for $|R\phi(x)|$. By repeating this, we can prove that $|R^{n}x| = |R^{n}\phi(x)|$ and $|L^{n}x| = |L^{n}\phi(x)|$ for all $n\geq 1$, which proves that $\phi$ is an isometry. Since $\fraH$ and $\fraH'$ are Hilbert space completions of $V$ and $V'$, we can extend $\phi$ to an isometry $\phi:\fraH\to \fraH'$. 

Now we have to show that $\phi$ is an intertwining operator. For $f\in V = \fraH_{\fin}$ and $X\in \frag$, we have
\begin{align*}
\phi(\pi(e^{X})f) = \sum_{n\geq 0} \frac{1}{n!} \phi(X^{n}f) = \sum_{n\geq 0} \frac{1}{n!}X^{n}\phi(f) = \pi'(e^{X})\phi(f)
\end{align*}
and the result follows from the fact that $V\subset \fraH$ is dense and $G$ is generated by elements of the form $e^{X}$. 
\end{proof}

By combining all of the results, we get the following classification. 

\begin{theorem}[Unitary representations of $\GL(2, \Rr)^{+}$]
The following is a complete list of the isomorphism classes of irreducible admissible unitary representations of $\GL(2, \Rr)^{+}$:
\begin{enumerate}
\item 1-dimensional representation $g\mapsto \det(g)^{\mu}$ for $\mu\in i\Rr$. 
\item The principal series $\calP_{\mu}(\lambda, \epsilon)$ with $\mu\in i\Rr, \epsilon \in \{0, 1\}$ and $\lambda\in \Rr$ with $\lambda \geq \frac{1}{4}$. 
\item The complementary series $\calP_{\mu}(\lambda, 0)$ with $\mu\in i\Rr$ and $0<\lambda <\frac{1}{4}$. 
\item The holomorphic discrete series and limits of discrete series $\calD_{\mu}^{\pm}(k)$ with $\mu\in i\Rr$. 
\end{enumerate}
\end{theorem}


\subsection{Whittaker models}
Now we know all the representations of $\GL(2, \Rr)$. However, if someone give an arbitrary abstract representation, then it is not easy to study it directly. To resolve such a problem, we may realize the abstract representation as a space of certain functions with an explicit and easy action (right translation). 
This is a main philosophy of \emph{Whittaker models}, and we will show that it is possible to realize almost all representations as a space of such functions. 

Let $W:\GL(2, \Rr)^{+}\to \Cc$ be a smooth function that satisfies 
$$
W\left(\pmat{1}{x}{}{1}g \right) = \psi(x) W(g)
$$
for a fixed nontrivial unitary additive character of $\Rr$, which has a form of $\Psi(x) = \Psi_{a}(x) = e^{iax}$ where  $0\neq a\in\Rr$. We say that $W$ is of moderate growth if, when we express the function $W$ in terms of $u, x, y, \theta$ via Iwasawa decomposition, it is bounded by a polynomial in $y$ as $y\to \infty$. We say that $W$ is rapidly decreasing if $y^{N}W\to 0$ as $y\to \infty$ for any $N>0$. We say that $W$ is analytic if it is locally given by a convergent power series. 
The function $W$ satisfying the functional equation and of moderate growth is called \emph{Whittaker function}.

The following proposition shows uniqueness of such function with fixed eigenvalues of $\Delta$ and $Z$. 
\begin{proposition}
Let $\mu, \lambda\in \Cc$ and $k\in \Zz$. Let $\calW(\lambda, \mu, k)$ be the space of Whittaker functions with prescribed eigenvalues $\lambda, \mu$ of $\Delta, Z$  and weight $k$, i.e. the space of functions $W:\GL(2, \Rr)^{+}\to \Cc$ satisfying 
\begin{align*}
W\left( \pmat{1}{x}{}{1}g \kappa_{\theta}\right) &= \psi(x)e^{ik\theta}W(g) \\
\Delta W &= \lambda W \\
ZW &= \mu W
\end{align*}
and is of moderate growth. Then $\calW(\lambda, \mu, k)$ is one-dimensional, and a function in this space is actually rapidly decreasing and analytic. 
Also, the operators $R$ and $L$ map $\calW(\lambda, \mu,k)$ to $\calW(\lambda, \mu, k+2)$ and $\calW(\lambda, \mu, k-2)$, respectively. 
\end{proposition}
\begin{proof}
We will assume that $\psi(x) =\psi_{1/2}(x) = e^{ix/2}$. The condition $ZW = \mu W$ implies 
$$
W\left(\pmat{u}{}{}{u}g\right) = u^{\mu} W(g)
$$
and we get
\begin{align*}
W(g) = u^{\mu} e^{i(x/2+k\theta)}w(y), \quad w(y) = W\pmat{y^{1/2}}{}{}{y^{-1/2}}\end{align*}
where $g = \smat{u}{}{}{u} \smat{y^{1/2}}{xy^{-1/2}}{}{y^{-1/2}} \kappa_{\theta}$. 
Then the condition $\Delta W = \lambda W$ is equivalent to the 2nd order differential equation
$$
w'' + \left(-\frac{1}{4} + \frac{k}{2y} + \frac{\lambda}{y^{2}}\right)w = 0.
$$
It is known that there exists two linearly independent solutions of this equation, $W_{\frac{k}{2}, s-\frac{1}{2}}(y)$ and $W_{-\frac{k}{2}, s-\frac{1}{2}}(-y)$, which are asymptotically $e^{-y/2}y^{k/2}$ and $e^{y/2}(-y)^{-k/2}$.  (Here $s = \frac{1}{2} + (-\lambda + \frac{1}{4})^{1/2}$, and such functions are \emph{classical Whittaker functions}.) 
Thus the assumption of moderate growth excludes the second solution and $\calW(\lambda, \mu, k)$ is 1-dimensional space spanned by the function
$$
W_{k, \lambda, \mu}(g) = u^{\mu}e^{i(x/2 + k\theta)} W_{\frac{k}{2}, s-\frac{1}{2}}(y),
$$
which is known to be rapidly decreasing and analytic. 
The statement about $R$ and $L$ action also follows from analytic properties of classical Whittaker functions. 
\end{proof}

From this, we can prove uniqueness of Whittaker model.
\begin{theorem}
\label{archwit}
Let $(\pi, V)$ be an irreducible admissible $(\frag, K)$-module for $G = \GL(2, \Rr)^{+}$ or $\GL(2, \Rr)$. Then there exists at most one space $\calW(\pi, \psi)$ of smooth $K$-finite Whittaker functions $W$ which is isomorphic to $(\pi, V)$ as a $(\frag, K)$-module. Every function in $\calW(\pi, \psi)$ is rapidly decreasing and analytic. 
\end{theorem}
\begin{proof}
Let $\mu, \lambda$ be scalars corresponds to action of $Z$ and $\Delta$. Decompose $V$ as $\oplus_{k} V(k)$. If $V(k)\neq 0$, then its image under the isomorphism with $\calW(\pi, \psi)$ is $\calW(\lambda, \mu, k)$, and the previous theorem implies uniqueness and analytic properties. 
\end{proof}
Such uniqueness is important, and we also have uniqueness theorem for non-archimedean local fields (we will prove this in Chapter 3 using the theory of Jacquet functor). By combining uniqueness result for archimedean and non-archimedean local fields, we get the global result, which is called \emph{multiplicity one}. 

\subsection{Classical Automorphic Forms and Spectral Problem}
In this section, we will see how the representation theory relates to classical modular forms, Maass forms and spectral problems. 

First, the elements $R, L, H, \Delta \in U\frag_{\Cc}$ coincide with the classical Maass operators.
Recall that we have (weight $k$) Maass differential operators
\begin{align*}
R_{k} &= (z-\ol{z}) \parz + \frac{k}{2} \\
L_{k} &= -(z-\ol{z}) \parzb - \frac{k}{2}
\end{align*}
and the (weight $k$) Laplacian
$$
\Delta_{k} = -y^{2}\left( \frac{\partial^{2}}{\partial x^{2}} + \frac{\partial^{2}}{\partial y^{2}} \right) iky \frac{\partial}{\partial x}
$$
which acts on the space of smooth functions on $\calH$, the complex upper half plane. 
Since $\calH \simeq \SL(2, \Rr) / \SO(2)$, we can lift such function as a smooth function on $\SL(2, \Rr)$, so on $G = \GL(2, \Rr)^{+}$ by letting it translation invariant under $Z(\Rr)^{+} = \{ \smat{a}{}{}{a}\,:\, a>0\}$. 
This gives a 1-1 correspondence between space of functions on $\calH$ and on $G$. More precisely:
\begin{proposition}
Let $\Gamma$ be a discrete cofinite subgroup of $G$ and let $\chi:\Gamma \to \Cc^{\times}$ be a character. 
Let $L^{2}(\Gamma \bs \calH, \chi, k)$ be a space of functions $f(z)$ on $\calH$ satisfying
\begin{align*}
\chi(\gamma)f(z) = &\left(\frac{c\ol{z} + d}{|cz+d|}\right)^{k} f\left( \frac{az+b}{cz+d}\right)=:(f||_{k}\gamma)(z), \quad \gamma = \pmat{a}{b}{c}{d} \in \Gamma 
\end{align*}
and
\begin{align*}
\int_{\Gamma\bs \calH} |f(z)|^{2} \frac{dxdy}{y^{2}} <\infty.
\end{align*}
Similarly, let $L^{2}(\Gamma \bs G, \chi, k)$ be a space of functions $F(g)$ on $G$ satisfying 
$$
F(\gamma g u \kappa_{\theta}) = \chi(\gamma)e^{ik\theta}F(g), \quad \gamma\in \Gamma, u\in Z^{+}, g\in G, \kappa_{\theta}\in \SO(2)
$$
and
$$
\int_{G/Z^{+}} |F(g)|^{2} dg <\infty.
$$
There is a Hilbert space isomorphism 
$$
\sigma_k : L^{2}(\Gamma \bs \calH, \chi, k) \to L^{2}(\Gamma \bs G, \chi, k)
$$
given by 
$$
(\sigma_k f)(g) = (f||_{k}g)(i).
$$
\end{proposition}
\begin{proof}
Proof follows from direct computations. The inverse map is given by 
$$
f(z) = F\left(\pmat{y}{x}{}{1}\right)
$$
for given $F\in L^{2}(\Gamma \bs G, \chi, k)$. 
\end{proof}
The main point is that under this isomorphism, Maass differential operators and the (weight $k$) Laplacian operator correspond to the elements $R, L, \Delta \in U\frag_{\Cc}$ we defined. 
\begin{proposition}
Let $R, L, H, \Delta$ be elements in $U\frag_{\Cc}$ we defined before. Then it acts as differential operators on $C^{\infty}(G)$. 
We have 
\begin{align*}
dR &= e^{2i\theta} \left( iy \frac{\partial}{\partial x} + y\frac{\partial}{\partial y} + \frac{1}{2i} \frac{\partial}{\partial \theta}\right) \\
dL &= e^{-2i\theta} \left(-iy \frac{\partial}{\partial x} + y \frac{\partial}{\partial y} - \frac{1}{2i}\frac{\partial}{\partial \theta}\right) \\
dH & = -i\frac{\partial}{\partial\theta} \\
\Delta &= -y^{2}\left( \frac{\partial^{2}}{\partial x^{2}} + \frac{\partial^{2}}{\partial y^{2}}\right) + y\frac{\partial^{2}}{\partial x \partial \theta} 
\end{align*} 
where $x, y, \theta$ are parameters in the Iwasawa decomposition of $g\in G$. 
Also, we have
$$
\sigma_{k+2} \circ R_k = R\circ \sigma_k, \quad \sigma_{k-2} \circ L_k = L \circ \sigma_k, \quad \sigma_k \circ \Delta_k = \Delta \circ \sigma_k. 
$$
\end{proposition}

We know that there are three types of automorphic forms on $\Gamma \bs \calH$:
\begin{enumerate}
\item Holomorphic modular forms: For a given character $\chi:\Gamma \to \Cc^{\times}$ and $k\geq 1$ with $\chi(-I) = (-1)^{k}$, a weight $k$ holomorphic modular form on $\Gamma$ is a holomorphic function $f:\calH \to \Cc$ satisfying $f(\gamma z) = \chi(\gamma)(cz+d)^{k}f(z)$ for all $\gamma\in \Gamma, z\in \calH$, and holomorphic at the cusps of $\Gamma$. 
\item Maass forms: Also a function on $\calH$, but smooth, not holomorphic. $f:\calH \to \Cc$ satisfies $(f||_{k}\gamma)(z) = \chi(\gamma)f(z)$ for $\gamma\in \Gamma$, and an eigenfunction of the Laplacian operator $\Delta_k$. 
\item The constant function. $f(z) = 1$ for all $z\in \calH$ is clearly invariant under (any) discrete subgroup $\Gamma \subset G$. 
More generally, $f(z) = y^{s}$ is also  an automorphic form for any $s\in \Cc$.
\end{enumerate}
Why are there precisely these types of automorphic forms on $\Gamma \bs \calH$ and no others? 
You may see that this list of automorphic forms are very similar to the classification of $(\frag, K)$-modules of $\GL(2, \Rr)^{+}$ (and $\GL(2, \Rr)$). 
In fact, this gives an answer to the above question. 
We can consider such an automorphic form $f(z)$ on $\calH$ as  a function $F(g)$ on $G$ (by the above map $\sigma_k$), and we can consider a $(\frag, K)$-submodule generated by the single element $F(g)$. 
This is an irreducible admissible $(\frag, K)$-module (admissibility is a result of Harish-Chandra, see Theorem \ref{autoadm}), and the previous classification gives us three types of automorphic forms. 

Another important question is the spectral problem. We can formulate it as follows: 
\begin{enumerate}
\item Determine the spectrum of the symmetric unbounded operator $\Delta_k$ on $L^{2}(\Gamma \bs \calH, \chi, k)$. 
\item Determine the decomposition of the Hilbert space $L^{2}(\Gamma \bs G, \chi)$ into irreducible subspaces. 
\end{enumerate}
We don't know the complete answer yet, but we  understand some of them. First, one can prove that such decomposition \emph{exists}. 
\begin{theorem}
\label{l2decomp}
$L^{2}(\Gamma \bs G, \chi)$ decomposes into a Hilbert space direct sum of  irreducible representations, and $L^{2}(\Gamma\bs\calH, \chi, k)$ decomposes into a Hilbert space direct sum of eigenspaces for $\Delta_k$. 
\end{theorem}
\begin{proof}
First statement uses Zorn's lemma. If we define $\Sigma$ to be the set of all sets $S$ of irreducible invariant subspaces of $L^{2}(\Gamma \bs G, \chi)$ such that the elements of $S$ are mutually orthogonal, then there exists a maximal element $S$ in $\Sigma$. 
If we put $\fraH$ as the orthogonal complement of the closure of the direct sum of the elements of $S$, then one can show that $\fraH = 0$. (For details, see Theorem 2.3.3 in \cite{bu}.)

For the second statement, it is equivalent to showing that $L^{2}(\Gamma \bs G, \chi, k)$ decomposes into direct sum of eigenspaces of $\Delta$. 
In Proposition \ref{commchar}, we showed that $C_{c}^{\infty}(K\bs G /K, \sigma)$ is a commutative ring, where $\sigma(\kappa_{\theta}) = e^{ik\theta}$. 
For each character $\xi$ of $C_{c}^{\infty}(K\bs G/K, \sigma)$, let $H(\xi) := \{f\in L^{2}(\Gamma \bs G, \chi, k)\,:\, \pi(\phi)f = \xi(\phi)f, \phi\in C_{c}^{\infty}(K\bs G/K, \sigma)\}$. 
Here $C_{c}^{\infty}(K\bs G/K, \sigma)\subset C_{c}^{\infty}(G)$ acts as
$$
\pi(\phi)f = \int\dpl{G} \phi(g)\pi(g)f dg. 
$$
One can show that 
$$
L^{2}(\Gamma \bs G, \chi, k) = \bigoplus_{\xi} H(\xi), 
$$
where the direct sum is a Hilbert space direct sum and $\xi$ ranges through all distinct characters of $C_{c}^{\infty}(K\bs G/K, \sigma)$ with $H(\xi)\neq 0$. 
Each of $H(\xi)$ is finite dimensional. (Most of the result follows from the spectral theorem for self-adjoint compact operators, applied to $\pi(\phi)$. See Theorem 2.3.4 of \cite{bu} for details.) 
Since $\Delta$ commutes with $\pi(\phi)$ (recall that $H$ lies in the center of $U\frag_{\Cc}$, and it is both invariant under left and right regular representations - see Theorem \ref{center}), the spaces $H(\xi)$ are $\Delta$-invariant, so these decomposes as a direct sum of $\Delta$-eigenspaces since $\Delta$ is self-adjoint. Hence $L^{2}(\Gamma \bs G, \chi, k)$ also decomposes as $\Delta_k$-eigenspaces. 
\end{proof}
Now the previous classification of irreducible admissible unitary representations of $\GL(2, \Rr)^{+}$ gives the decomposition of $L^{2}(\Gamma \bs G, \chi)$. 
For each irreducible subspace $H$ of it, $\Delta$ acts as a scalar $\lambda = \lambda(H)$ on $H$ and it depends only on the isomorphism class of $H$. According to the value of $\lambda$, the different types of irreducible admissible unitary representations occur as constituents of the decomposition with some multiplicity. 
\begin{theorem}
The right regular representation of $G$ on $L^{2}(\Gamma \bs G, \chi)$ decomposes as following:
\begin{align*}
L^{2}(\Gamma \bs G, \chi) = \Cc.1 &\bigoplus \left( \bigoplus_{\substack{\lambda \neq \frac{k}{2}\left(1-\frac{k}{2}\right), k\equiv \epsilon \Mod{2}\\ \lambda \geq \frac{\epsilon}{4}}} m(\lambda, \epsilon) \calP(\lambda, \epsilon)\right) \\&\bigoplus \left(\bigoplus_{\substack{k\geq 1 \\ k\equiv \epsilon \Mod{2}}}d(k, \chi) (\calD^{+}(k) \oplus \calD^{-}(k))\right)
\end{align*}
where $m(\lambda, \epsilon)$ is a multiplicity of $\calP(\lambda, \epsilon)$ in the decomposition, which is equal to the multiplicity of the eigenvalue $\lambda$ in $L^{2}(\Gamma\bs\calH, \chi, k)$ for $k\equiv \epsilon\Mod{2}$, and $d(k, \chi) = \dim M_{k}(\Gamma, \chi)$, the dimension of the space of weight $k$ holomorphic modular forms on $\Gamma$ with character $\chi$. 
\end{theorem}
\begin{proof}
The only point worth to mention is the connection between discrete series representations and holomorphic modular forms. The multiplicity of $\calD^{+}(k)$ equals the dimension of the $\frac{k}{2}\left( 1- \frac{k}{2}\right)$-eigenspace in $L^{2}(\Gamma\bs G, \chi, k)$, or in $L^{2}(\Gamma \bs \calH, \chi, k)$. This eigenspace is isomorphic to the space of modular forms $M_{k}(\Gamma, \chi)$: let $H$ be an irreducible subspace that is isomorphic to $\calD^{\pm}(k)$. 
Then $H(k-2) = 0$ implies that $L_{k}f = 0$ for any $f\in L^{2}(\Gamma \bs \calH, \chi, k)$. This is equivalent to $y^{-k/2}f(z)$ to be a holomorphic modular form in $M_{k}(\Gamma, \chi)$. 
\end{proof}
It is not hard to compute $d(k, \chi)$ (using Riemann-Roch theorem or other tools), but it is extremely hard to compute $m(\lambda, \epsilon)$ and we conjecture that all of them are one, but until now, we don't know any single exact value of it. (There are some known upper bounds.) 
It is known that  if $\Gamma$ is cocompact (i.e. $\Gamma \bs \calH$ is compact), it is known that the spectrum of $\Delta_k$ on $L^{2}(\Gamma\bs\calH, \chi, k)$ is discrete and the eigenvalues $ \lambda_1 < \lambda_2 < \cdots$ tend to infinity. 


\newpage

\section{Non-archimedean theory}
Now we will get into the representation theory of $\GL(2, F)$ over non-archimedean local fields. Archimedean and non-archimedean cases are very similar, but also very different. 
Their topologies are completely different from archimedean case, which make the situations easier or harder. However, their representations are very similar. For example, we can construct most of the representation from principal series representations, which are induced representations of characters of Borel subgroup, as in the archimedean case. 

There are some other representations that do not come from principal series representations, which are called supercuspidal representations. Such representations are also interesting, and we will present some methods to construct such representations (Weil representations). 

\subsection{Smooth and admissible representation}
In this section, we will fix some notations as follows:
\begin{itemize}
\item $F$: a non-archimedean local field
\item $\calO$: a ring of integers
\item $\frap$: the unique maximal ideal of $\calO$
\item $\varpi$: a uniformizer, i.e. generator of $\frap$
\item $k = \calO / \frap$: a residue field
\item $q$: cardinality of $k$
\item $v:F\to \Zz\cup\{\infty\}$: normalized valuation of $F$
\item $dx$: nomalized additive Haar measure
\item $d^{\times}x$: normalized multiplicative Haar measure
\end{itemize}
The biggest difference between archimedean and non-archimedean local fields is the topology. Every group over non-archimedean local fields that we will see will be totally disconnected locally compact spaces. 
Such groups always have a basis of open subgroups at the identity, which can be chosen as normal subgroups when $G$ is compact. 
For example, in case of $G = \GL(n, F)$, the subgroups $K(\varpi^{n})$ ($n\geq0$) of elements in $\GL(n, \calO)$ congruent to identity modulo $\varpi^{n}$ forms such a basis, and these are even normal in a compact subgroup $\GL(n, \calO)$. 


As in the archimedean case, we will concentrate on representations that  we can handle, which are \emph{smooth} and \emph{admissible} representations. 
\begin{definition}
Let $G$ be a totally disconnected locally compact group and $(\pi, V)$ be a representation of $G$. We say that $\pi$ is \emph{smooth} if $\Stab(v) = \{g\in G\,:\,\pi(g)v= v\}$ is open for all $v\in V$. 
If $\pi$ is smooth and $V^{U} = \{v\in V\,:\, \pi(g)v= v\,\forall g\in U\}$ is finite dimensional for any open subgroup $U\subset G$, then $V$ is called \emph{admissible}.
\end{definition} 
 One can check that complex representation of $G$ is smooth if and only if the map $\pi:G\times V\to V$ is continuous, where $V$ is given by usual complex topology. 


Admissible representations are important because they satisfy important properties that also holds for representations of finite groups. Also, most of the properties can be proved by using the corresponding result of representations of finite groups. 
For example, the following theorem shows that any smooth representation of totally disconnected locally compact group is semisimple, and each isotypic part of the decomposition is finite dimensional if and only if the representation is admissible. 
\begin{proposition}
Let $(\pi, V)$ be a smooth representation of $G$ and $K$ be a compact open subgroup of $G$. Then $V$ is semisimple, i.e.
$$
V = \bigoplus_{\rho\in \wh{K}} V(\rho)\qquad(\text{algebraic direct sum}).
$$
$\pi$ is admissible if and only if $V(\rho)$ is finite dimensional for all $\rho$. 
\end{proposition}
\begin{proof}
We show first that $V\subset \sum_{\rho\in \wh{K}} V(\rho)$. 
For $v\in V$, it is fixed by a compact open subgroup $K_{0}$ of $K$, which can be assumed to be normal. Then 
$$
v\in V^{K_{0}} = \bigoplus_{\rho\in \wh{\Gamma}} V(\rho) \subseteq \sum_{\rho\in \wh{K}} V(\rho)
$$
where $\Gamma = K/K_{0}$, which is finite. 

To show that the sum is direct, let's assume that it is not, so $\sum_{\rho\in S} c_{\rho}v_{\rho} = 0$ for some finite subset $S\subset \wh{K}$, $v_{\rho}\in V(\rho)$ and $c_{\rho}\in \Cc$ that are not all zero. If we put $K_{0} = \cap_{\rho\in S}\ker (\rho)$, then we obtain a contradiction to the directness of the summation for $\Gamma = K/K_{0}$. 

For the last statement, $V(\rho)\subset V^{\ker(\rho)}$ implies that $V(\rho)$ is finite dimensional if $\pi$ is admissible since $\ker(\rho)$ is an open subgroup. 
Conversely, if $\pi$ is not admissible, then $V^{K_{0}}$ is infinite dimensional for some open normal subgroup $K_{0}$ of $K$. From $V^{K_{0}} = \oplus_{\rho\in \wh{K/K_{0}}} V(\rho)$, since $K/K_{0}$ is a finite group, $V(\rho)$ is infinite dimensional for some $\rho$. 
\end{proof}

We call that a linear functional $\wh{v}:V\to \Cc$ is \emph{smooth} if there exists an open neighborhood $U$ of identity such that $\langle \pi(g)v, \wh{v}\rangle = \langle v, \wh{v}\rangle$ for all $g\in U$ and $v\in V$. 
We will denote the space of smooth linear functionals as $\wh{V}$. For any representation $(\pi, V)$, we define its \emph{contragredient representation} $(\wh{\pi}, \wh{V})$ by 
$$
\bra{v}{\wh{\pi}(g)\wh{v}} = \bra{\pi(g^{-1})v}{\wh{v}}.
$$
By smoothness, we can check that $\wh{V}$ also decomposes as
$$
\wh{V} = \bigoplus_{\rho} V(\rho)^{*}
$$
so the contragredient of an admissible representation is admissible. 

As in the archimedean case, representation $\pi$ of $G$ on the space $V$ induces an action of Hecke algebra $\calH = C_{c}^{\infty}(G)$ of compactly supported smooth functions, i.e. locally constant functions, where the multiplication is given by the convolution 
$$
(\phi_{1} * \phi_{2})(g) = \int_{G} \phi_{1}(gh^{-1})\phi_{2}(h) dh.
$$
The action of $\calH$ is given by 
$$
\pi(\phi)v = \int_{G} \phi(g)\pi(g)vdg
$$
which satisfies $\pi(\phi_{1}*\phi_{2}) = \pi(\phi_{1})\circ \pi(\phi_{2})$. Note that the above integration is actually a finite sum. 
There's no identity in the algebra $\calH$. However, for any compact open subgroup of $G$, the subalgebra of $K_{0}$-biinvariant functions $\calH_{K_{0}} = C_{c}^{\infty}(K_{0}\backslash G/K_{0})$ has an identity element 
$$
\epsilon_{K_{0}} = \frac{1}{|K_{0}|} \chf_{K_{0}}. 
$$
The following proposition shows that irreducibility of the representation is equivalent to irreducibility of correponding Hecke algebra representation. 

\begin{proposition}
\label{simple}
Let $(\pi, V)$ be a smooth representation of $G$. TFAE:
\begin{enumerate}
\item $\pi$ is irreducible. 
\item $V$ is a simple $\calH$-module. 
\item $V^{K_{0}}$ is either zero or simple $\calH_{K_{0}}$-module for all open subgroup $K_{0}$ of $G$. 
\end{enumerate}
\end{proposition}
\begin{proof}
We will show that $G$-invariance of subspace is equivalent to $\calH$-invariance, which proves $1\Leftrightarrow 2$. Clearly,  $G$-invariant space is also $\calH$-invariant. Conversely, let $W\subset V$ be a $\calH$-invariant subspace. Assume that $W$ is not $G$-invariant, so that $\pi(g)w\neq w$ for some $g\in G$ and $w\in W$. 
Now $w$ is fixed by some neighborhood $N$ of the identity, so let $\phi = \frac{1}{|N|} \chf_{gN}$ then we have $w = \pi(\phi)w = \phi(g)w$, a contradiction. 

$3\Rightarrow 2$ is also simple: assume that $V$ is not simple and let $W \subset V$ be a proper $\calH$-submodule. From $V = \cup_{K_{0}} V^{K_{0}}$, we can find $K_{0}$ small enough so that $W^{K_{0}}$ is a nonzero proper subspace of $V^{K_{0}}$.

For $2\Rightarrow3$, let $W_{0}\subset V^{K_{0}}$ be a nonzero proper $\calH_{K_{0}}$-submodule. We will show that $\pi(\calH)W_{0} \cap V^{K_{0}} = W_{0}$, which implies that $\pi(\calH)W_{0}$ is a nonzero proper $\calH$-submodule of $V$. 
Assume that $w = \sum_{i} \pi(\phi_{i})w_{i} \in \pi(\calH)W_{0}\cap V^{K_{0}}$, where $w_{i}\in W_{0}$. 
Since $w_{i}\in V^{K_{0}}$ and $w\in V^{K_{0}}$, we have $\pi(\epsilon_{K_{0}})w_{i} = w_{i}$ and $\pi(\epsilon_{K_{0}})w = w$, which shows that $w = \sum_{i} \pi(\epsilon_{K_{0}} * \phi_{i} * \epsilon_{K_{0}}) w_{i}$. 
However, $\epsilon_{K_{0}} * \phi_{i} * \epsilon_{K_{0}} \in \calH_{K_{0}}$ and since $W_{0}$ is $\calH_{K_{0}}$-stable, we get $w\in W_{0}$. 
\end{proof}

Another important feature is that irreducible admissible representations are determined by their characters. 
For a representation of a finite group $G$, we defined its character as a trace of the representation, i.e. the function $\chi:G\to \Cc$ defined as $\chi(g) = \Tr(\pi(g))$. 
We can also define character of admissible representations as a distribution on $\calH = C^{\infty}_{c}(G)$. The key property of trace is that the trace of $L:V\to V$ is same as the trace of restriction $L|_{W}$ on any invariant subspace $W\subseteq V$. From this, we can define the character $\chi:\calH\to \Cc$ as follows: for any $f\in \calH$, there exists an open compact subgroup $K_{0}$ such that $f\in\calH_{K_{0}}$.  
Then $V^{K_{0}}$ is invariant under $\pi(f)$, which is a finite dimensional subspace, so the trace of the map is well-defined and we let $\chi(f) = \Tr(\pi(f))$. 
\begin{theorem}
\label{char}
Let $(\pi_{1}, V_{1})$ and $(\pi_{2}, V_{2})$ be irreducible admissible representations of the totally disconnected locally compact group $G$. If characters of $\pi_{1}$ and $\pi_{2}$ agree, then the two representations are equivalent. 
\end{theorem}
\begin{proof}
It is known that for any $k$-algebra $R$, structure of simple $R$-module is completely determined by traces of endomorphisms induced by multiplication of elements in $R$. Hence the assumption implies that $V_{1}^{K_{1}}\simeq V_{2}^{K_{1}}$ as $\calH_{K_{1}}$-modules for any open compact subgroup $K_{1}$ of $G$. 

Let $K_{0}$ be a small open compact subgroup so that $V_{1}^{K_{0}}$ and $V_{2}^{K_{0}}$ are nonzero. 
By hypothesis, we have an  $\calH_{K_{0}}$-module isomorphism $\sigma_{K_{0}}:V_{1}^{K_{0}}\to V_{2}^{K_{0}}$, which is unique up to constant by Schur's lemma.  
Then for any open subgroup $K_{1} \subset K_{0}$, we can extend $\sigma_{K_{0}}$ uniquely to a $\calH_{K_{1}}$-module isomorphism $\sigma_{K_{1}}:V_{1}^{K_{1}}\to V_{2}^{K_{1}}$.
Indeed, the existence is in our hypothesis and from $V_{i}^{K_{0}} = \pi_{i}(\epsilon_{K_{0}}) V_{i}^{K_{1}}$ we have
\begin{align*}
\sigma_{K_{1}}(V_{1}^{K_{0}}) = \sigma_{K_{1}}(\pi_{1}(\epsilon_{K_{0}})V_{1}^{K_{1}}) = \pi_{2}(\epsilon_{K_{0}}) (\sigma_{K_{1}}(V_{1}^{K_{1}})) = \pi_{2}(\epsilon_{K_{0}}) V_{2}^{K_{1}} = V_{2}^{K_{0}},
\end{align*}
so $\sigma_{K_{1}}|_{V_{1}^{K_{0}}} : V_{1}^{K_{0}} \to V_{2}^{K_{0}}$ is an $\calH_{K_{0}}$-module isomorphism, and uniqueness implies that the restriction of $\sigma_{K_{1}}$ and $\sigma_{K_{0}}$ agrees up to scalar, so we can assume that they coincides on $V_{1}^{K_{0}}$ by normalizing. 
Now we can repeat this for an open compact basis of identities $\{K_{n}\}_{n\geq 0}$, and we get a map $\sigma:V_{1}\to V_{2}$ which is an $\calH$-module isomorphism. 

To show that $\sigma$ is an intertwining operator, let $g\in G$ and $v\in V_{1}$. Choose an open compact subgroup $K_{1}$ such that $v\in V^{K_{1}}$, and let $\phi = \frac{1}{|K_{1}|} \chf_{gK_{1}}$. 
Then $\pi_{1}(\phi)v = \pi_{1}(g)v$ and $\pi_{2}(\phi)\sigma(v)= \pi_{2}(g)\sigma(v)$, and we get
$$
\sigma(\pi_{1}(g)v) = \sigma(\pi_{1}(\phi)v) = \pi_{2}(\phi)\sigma(v) = \pi_{2}(g)\sigma(v). 
$$
This shows that $\sigma$ is an intertwining operator between $V_{1}$ and $V_{2}$. 
\end{proof}

Using the theorem, we can prove that contragredient representation of $\GL(2, F)$ is isomorphic to other representations on the original space with different actions by comparing characters. 
\begin{theorem}
\label{contra}
Let $G = \GL(n, F)$ with $F$ non-archimedeal local field, and let $(\pi, V)$ be an irreducible admissible representation of $G$. 
\begin{enumerate}
\item Let $(\pi_{1}, V)$ be a representation defined as $\pi_{1}(g) = \pi(\pre{T}{}g^{-1})$. Then $\wh{\pi}\simeq \pi_{1}$. 
\item For $n = 2$, let $\omega$ be the central quasi-character of $\pi$. Define $(\pi_{2}, V)$ on the same space as $\pi_{2}(g) = \omega(\det(g))^{-1}\pi(g)$. Then $\wh{\pi} \simeq \pi_{2}$. 
\end{enumerate}
\end{theorem}
Here the central character $\omega:F^{\times} \to \Cc^{\times}$ is a character corresponds to the action of $\pi$ restricted to $Z(F)$. Note that the center acts as a scalar by Schur's lemma. 
\begin{proof}
For $\phi\in C^{\infty}(G)$, let $\phi', \phi''\in C^{\infty}(G)$ as $\phi'(g) = \phi(g^{-1})$, $\phi''(g) = \phi(\pre{T}{}g^{-1})$. We know that character is conjugation invariant, and it is known that conjugation invariant distribution on $\GL(n, F)$ is also transpose invariant. (This is a nontrivial result proved by Bernstein-Zelevinski. You can found a proof in p. 449 of \cite{bu}.)
Hence we have
$$
\chi_{\pi_{1}}(\phi) = \chi_{\pi}(\phi'') = \chi_{\pi}(\phi') = \chi_{\wh{\pi}}(\phi)
$$
where the last equality follows from the fact that $\pi(\phi)$ and $\wh{\pi}(\phi')$ are adjoints of each other, so have equal trace. 

For 2, the following identity
$$
\pre{T}{}g^{-1} = \pmat{\det(g)}{}{}{\det(g)}^{-1} w^{-1}gw, \quad w = \pmat{}{-1}{1}{}
$$
shows that $\pi(w)$ is an intertwining operator from $(\pi_{1}, V)$ to $(\pi_{2}, V)$. 
\end{proof}

By the previous theorem, we can directly check that irreducibility of admissible representation is preserved by taking dual. 
\begin{proposition}
Let $\pi$ be an admissible representation of $\GL(n, F)$. Then $\pi$ is irreducible if and only if $\wh{\pi}$ is irreducible. 
\end{proposition}
\begin{proof}
$\pi$-invariant subspace is also $\pi_{1}$-invariant. 
\end{proof}

There's one more thing worth to mention about totally disconnected locally compact groups. We use the following \emph{no small subgroup argument} several times, which is very useful and important. 

\begin{proposition}[No small subgroups argument]
\label{nss}
Let $G$ be totally disconnected locally compact group, so that it has a basis of open neighborhoods of the identity consisting of open and compact subgroups. For any homomorphism $\phi:G \to \GL(n, \Cc)$, the kernel $\ker \phi$ contains an open subgroup. 
\end{proposition}
\begin{proof}
It is enough to show that there exists an open neighborhood $N$ of the identity of $\GL(n, \Cc)$ that does not contain any nontrivial open subgroups. Then we can take the compact open subgroup that is contained in $\phi^{-1}(N)$. 
To show the existence of such $N$, let $\mathfrak{g}= \mathfrak{gl}(n, \mathbb{C})$ be its Lie algebra and let $\exp : \mathfrak{g} \to G'$ be the exponential map. Since $\exp$ is a local homeomorphism, we can find an open neighborhood $U \subseteq \mathfrak{g}$ of the identity such that $\exp:U\to \exp(U)$ is a homeomorphism. 
Fix an inner product on $\mathfrak{g}$ and we can assume that $U$ is of the form $\{v\in \mathfrak{g}\,:\, |v| <\epsilon$ for some $\epsilon>0$. 
Let $V = \frac{1}{2} U = \{v\in \mathfrak{g}\,:\, 2v\in U\} = \{v\in \mathfrak{g}\,:\, |v| <\epsilon/2\}$. 
We will show that $\exp(V)$ contains no nontrivial subgroups. 
Suppose that $H$ is a nontrivial subgroup contained in  $\exp(V)$ and choose $1\neq g\in H$ and $v\in V$ so that $g = \exp(v)$.  Since $g^{2}\in H$, $g^{2} = \exp(w)$ for some $w\in V$ and $\exp(2v) = g^{2} = \exp(w)$ implies that $w = 2v$ since $\exp$ is a homeomorphism on $V$. 
Now iterate this and we have $2^{n}v\in V$ for all $n$, and this implies $v = 0$ since $|2^{n}v| = 2^{n}|v|$. This gives a contradiction since $1\neq g = \exp(v) = \exp(0) = 1$. 
\end{proof}

\subsection{Distributions}

In this section, we will briefly introduce properties about distributions that will be used in the later chapters. For a totally disconnected locally compact space $X$, we define a distribution on $X$ as a linear functional on $C_{c}^{\infty}(X)$. Note that there's no restriction that the functional is continuous. 
We denote $\fraD(X)$ for the space of distributions on $X$. 
We have an exact sequence:

\begin{proposition}
Let $X$ be a totally disconnected locally compact space, and let $C\subseteq X$ be a closed subset. Then we have exact sequences
$$
0 \to C_{c}^{\infty}(X\bs C) \to C_{c}^{\infty}(X) \to C_{c}^{\infty}(C) \to 0
$$
and 
$$
0 \to \fraD(C) \to \fraD(X) \to \fraD(X\bs C) \to 0. 
$$
\end{proposition}
\begin{proof}
The only nontrivial part for the first exact sequence is the surjectivity of $C_{c}^{\infty}(X) \to C_{c}^{\infty}(C)$. Let $f\in C_{c}^{\infty}(C)$. Since $f$ is locally constant and compactly supported, there exists disjoint open and compact sets $U_{i} \subseteq C$ and $a_{i}\in \Cc$ such that $f(x) = a_i$ if $x\in U_i$ and $f(x) = 0$ off $\cup U_i$. 
Let $V_i$ be open and compact subsets of $X$ such that $U_i = V_i \cap C$. By replacing $V_i$ by $V_i \bs \cup_{j<i} V_j$, we can assume that $V_i$ are disjoint. Then we can extend the function $f$ to $X$ by letting $f(x) = a_i$ if $x\in V_i$ and $f(x) = 0$ off $\cup V_i$. 
Exactness of the second one follows by dualizing the first one. 
\end{proof}


We can also define actions of $G$ on $G, C_{c}^{\infty}(G)$, and $\fraD(G)$ by left and right translations. More precisely, we have
\begin{align*}
(\rho(g) f)(x) = f(xg), &\quad (\lambda(g)f)(x) = f(g^{-1}x), \\
(\rho(g)T)(f) = T(\rho(g^{-1})f), &\quad (\lambda(g)T)(f) = T(\lambda(g^{-1})f)
\end{align*}
for $T\in \fraD(G)$ and $f\in C_{c}^{\infty}(G)$. 
The following proposition shows that a distribution which is left $G$-invariant up to some character of $G$  is unique up to constant. The proof is not so hard, but this proposition will be used a lot later. 
\begin{proposition}
\label{distuniq}
Let $G$ be a locally compact totally disconnected group, and let $\xi$ be a character of $G$. Suppose that $T$ is a distribution on $G$ that satisfies $\lambda(h)T = \xi(h)^{-1}T$ for $h\in G$. Then there exists a constant $c$ such that
$$
T(f) = c\int_{G} \xi(h)f(h)dh
$$
where $dh = d_{L}h$ is the left Haar measure. 
\end{proposition}
\begin{proof}
We define another distribution by $f\mapsto T(\xi^{-1}f)$. (Note that $\xi^{-1}f$ is locally constant since $\xi^{-1}$ is locally constant by no small subgroups argument (Proposition \ref{nss}). Replacing $T$ by this, we may assume that $\xi = 1$ so that $\lambda(h)T = T$ for all $h$. 

Let $K$ be an open compact subgroup of $G$. If $f\in C_{c}^{\infty}(G)$, let $S(f) = \{h\in G\,:\, \lambda(h)f = f\}$. 
Then $S(f)$ is a compact open subgroup of $G$ and for any open subgroup $K_0$ of $S(f)$, we have 
$$
f = |K_{0}|\sum_{i=1}^{r} a_{i}\lambda(h_{i})\epsilon_{K_{0}}
$$
where $h_{1}, \dots, h_{r}$ be representatives of cosets in $K_{0}\bs G$ on which $f$ does not vanish and $a_{i} = f(h_{i})$. 
Then 
$$
T(f) = |K_{0}| \left[ \sum_{i=1}^{r}a_{i}\right] T(\epsilon_{K_{0}}).
$$
Apply this for $f = \epsilon_K$ so that $S(f) = K$ and $K_{0}$ can be any open subgroup of $K$. 
Since $[K:K_{0}] = |K|/|K_{0}|$, we have $T(\epsilon_{K_{0}}) = T(\epsilon_K) = c$. 
For general $f$, we may assume $K_{0} \subseteq K$ and this implies the desired result. 
\end{proof}

Finally, we introduce the concept of cosmooth modules and there relation with sheaves on $X$. 
\begin{definition}
Let $X$ be a locally compact totally disconnected space and let $M$ be a $C_{c}^{\infty}(X)$-module. 
We call $M$ cosmooth if for every $x\in M$, there exists an open compact subset $U$ of $X$ such that $\chf_{U} \cdot x = x$. 
\end{definition}
The following proposition shows an equivalence of category of sheaves of $C^{\infty}$-modules and the category of cosmooth modules over $C_{c}^{\infty}(X)$. 
\begin{proposition}
Let $M$ be a cosmooth $C_{c}^{\infty}(X)$-module. There exists a sheaf $\calM$ of $C^{\infty}$-modules on $X$ associated to $M$ such that $\calM(U) = \chf_{U}\cdot M$ for open compact sets $U$, and a restriction map $\rho_{U, V} : \calM(U) \to \calM(V)$ by $\rho_{U, V}(m) := \chf_{V}\cdot m$. 

Conversely, let $\calF$ be a sheaf of $C^{\infty}(X)$-modules on $X$. Then there exists an $C_{c}^{\infty}(X)$-module $\calF_{c}$ with embeddings $i_U : \calF(U) \hookrightarrow \calF_{c}$ for each open compact subsets $U$ such that if $U\supset V$, then we have $i_{V, U} : \calF(V)\hookrightarrow \calF(U)$ which satisfies $\rho_{U, V} \circ i_{V, U} = \mathrm{id}_{\calF(V)}$. 
Also, for $f\in \calF(V)$, $\rho_{U, U\bs V}(f) = 0$. These two constructions are inverse each other. 
\end{proposition}

The following theorem of Bernstein-Zelevinsky will be used in the proof of the uniqueness of local Whittaker models, i.e. local multiplicity one theorem. 
\begin{proposition}[Bernstein-Zelevinsky]
\label{bz}
Let $X, Y$ be totally disconnected locally compact spaces, and let $p:X\to Y$ be a continuous map. 
Let $\calF$ be a sheaf on $X$. 
Suppose that  $G$ is a group acting on $X$ and on its sheaf $\calF$. Assume that the action  satisfies $p(g\cdot x) = p(x)$ for $g\in G, x\in X$. 
Let $\chi$ be a character of $G$. 
\begin{enumerate}
\item Let $y\in Y$, and let $Z = p^{-1}(y)$. 
Let $\calF_{c}(\chi)$ (resp. $(\calF_{Z})_{c}(\chi)$) be the submodule of $\calF_{c}$ (resp. $(\calF_{Z})_{c}$) generated by elements of the form $g\cdot f - \chi(g)^{-1}f$ for $f\in \calF_{c}$ (resp. $f\in (\calF_{Z})_{c})$. 
Then $M =\calF_{c}/ \calF_{c}(\chi)$ is a cosmooth $C_{Y}^{\infty}$-module; let $\calG$ be the corresponding sheaf on $Y$. If $y\in Y$, then the stalk $\calG_{y}$ is isomorphic to $(\calF_{Z})_{c} / (\calF_{Z})_{c}(\chi)$. 
\item Assume that there are no nonzero distributions $D$ in $\fraD(p^{-1}(y), \calF_{p^{-1}(y)})$ that satisfy $g\cdot D = \chi(g)D$ for all $g\in G$, for any $y\in Y$. 
Then there are non nonzero distributions in $\fraD(X, \calF)$ satisfying the same equation. 
\end{enumerate}
\end{proposition}
Here $\fraD(X, \calF)$ is a space of $\calF$-valued distribution, which is a space of linear functional on $\calF_{c}$. In the proof of local multiplicity one theorem, this will help us to prove certain distribution is zero by only proving it fiberwise. 



\subsection{Whittaker functionals and Jacquet functor}
Like archimedean cases, non-archimedean theory also has a notion of Whittaker models. 
Let $\psi$ be a nontrivial additive character of $F$ and $\psi_{N}$ be a character of $N(F)$, the group of upper triangular unipotent matrices in $\GL(n, F)$, by
$$
\psi_{N}(u) = \psi\left( \sum_{i=1}^{n-1} u_{i, i+1}\right), \quad u = (u_{ij})\in N(F). 
$$

\begin{definition}
Let $(\pi, V)$ be a smooth representation of $\GL(n, F)$. 
A Whittaker functional on $V$ is a linear functional (non necessarily smooth) $\lambda: V\to \Cc$ such that $\lambda(\pi(u)x) = \psi_{N}(u)\lambda(x)$ for all $u\in N(F), x\in V$.
\end{definition}

The following theorem claims that local Whittaker functional is unique (up to constant), which is referred as \emph{local multiplicity one} theorem. 
\begin{theorem}[Uniqueness of Whittaker functional]
\label{nonarchmultone}
Let $(\pi, V)$ be an irreducible admissible representation of $\GL(n, F)$. Then the dimension of the space of Whittaker functionals on $V$ is at most one. 
\end{theorem}

For the proof, we need a lemma, which claims that a distribution that transforms like Whittaker functional under left and right translations (we will call such distribution as Whittaker distribution, only in this note) is invariant under certain involution on the space of distributions. 
Let $\Delta\in\fraD(\GL(n, F))$ be a distribution. $D$ is called a \emph{Whittaker distribution} if 
$$
\lambda(u) \Delta = \psi_{N}(u)^{-1}\Delta, \quad \rho(u)\Delta = \psi_{N}(u)\Delta
$$
for all $u\in N(F)$. 
We define an involution $\iota:\GL(n, F) \to \GL(n, F)$ by $\iota(g) = w^{0}\pre{T}{}gw^{0}$, where 
$$
w^{0} =\begin{pmatrix} & & 1 \\ & \iddots & \\1 & & \end{pmatrix}
$$
This also induces an action on $C^{\infty}_{c}(\GL(n, F))$ and $\fraD(\GL(n, F))$. Note that $\iota(N(F)) = N(F)$. 
\begin{theorem}
Let $\Delta\in \fraD(\GL(n, F))$ be a Whittaker distribution. Then $\Delta$ is stable under $\iota$. 
\end{theorem}

\begin{proof}
By replacing $\Delta$ by $\Delta - \pre{\iota}{}\Delta$, we can assume that $\pre{\iota}\Delta = -\Delta$, too. Now we want to show that a Whittaker distribution satisfying the above condition is zero. 

The above conditions (transformations laws) on $\Delta$ can be written in a more simpler way. Let $G$ be a semidirect product of the group $N(F)\times N(F)$ and an order 2 cyclic group generated by $\calI$ satisfying $\calI^{2} = 1$ and $\calI(u_{1}, u_{2})\calI^{-1} = (\pre{\iota}{}u_{2}^{-1}, \pre{\iota}{}u_{1}^{-1})$ for $(u_{1}, u_{2})\in N(F)\times N(F)$. 
Let $\chi$ be a character of $G$ defined as $\chi(u_{1}, u_{2})= \psi_{N}(u_{1})^{-1}\psi_{N}(u_{2})$ and $\chi(\calI) = -1$. 
Let $\sigma$ be the action of $G$ on $\GL(n, F), C_{c}^{\infty}(\GL(n, F))$, and $\fraD(\GL(n, F))$ by $\sigma(u_{1}, u_{2}) = \lambda(u_{1})\rho(u_{2}), \sigma(\calI) =\iota$. Then the conditions on $\Delta$ can be summarized as $\sigma(g) \Delta = \chi(g)\Delta$. 

To show that such distribution is zero, we will use the Bruhat decomposition and the corresponding exact sequence of distributions. We have an exact sequence
$$
0 \to \fraD(B(F)) \to \fraD(\GL(2, F)) \to \fraD(X) \to 0
$$
where $X = B(F)w^{0}B(F)$. 

We first show that the image in $\fraD(X)$ of $\Delta$ is zero. 
For a continuous mapping $p:X\to Y$ with $Y = F^{\times} \oplus F^{\times}$ given by $\smat{a}{b}{c}{d} \mapsto (c, (ad-bc)/c)$, the fibers of this map are $\sigma$-invariant and they are the double cosets 
$$
N(F) \pmat{}{b_{0}}{c_{0}}{} N(F)
$$
which is homeomorphic to $N(F)\times N(F)$ under the map $(u_{1}, u_{2})\mapsto u_{1} \smat{}{b_{0}}{c_{0}}{} u_{2}^{-1}$. By the theorem of Bernstein-Zelevinski (Proposition \ref{bz}), we only need to show that there are no nonzero distributions $\Delta$ on the single double coset that satisfies $\sigma(g)\Delta = \chi(g)\Delta$. 
By Proposition \ref{distuniq}, there exists $c\in \Cc$ such that 
$$
\Delta(f) = c \int\limits_{N(F)\times N(F)} \psi_{N}(u_{1})\psi_{N}(u_{1}) f\left( u_{1} \pmat{}{b_{0}}{c_{0}}{} u_{2}\right) du_{1}du_{2}.
$$
This distribution is invariant under $\iota$, since $\iota\left( u_{1} \smat{}{b_{0}}{c_{0}}{} u_{2}\right) = u_{2}\smat{}{b_{0}}{c_{0}}{} u_{1}$. Thus $\pre{\iota}{}\Delta = \Delta = -\pre{\iota}{}\Delta$ and so $\Delta = 0$. 
By exactness, $\Delta \in \fraD(B(F))$. We can use the similar argument to show $\Delta = 0$. 
Let $Y_{1} =F^{\times} \oplus F^{\times}$ and let $p:B(F) \to Y_{1}, \smat{a}{b}{}{d} \mapsto (a, d)$. 
Then each fibers are homeomorphic to $N(F)$ via $u\mapsto u\delta = u\smat{a}{}{}{d}$. If we apply the Proposition \ref{distuniq} for left and right translations, we get
$$
\Delta(\phi) = c_{1}\int\limits_{N(F)} \phi(u\delta)\psi_{N}(u)du = c_{2} \int\limits_{N(F)} \phi(u\delta) \psi_{N}(\delta^{-1}u\delta) du
$$
for some $c_{1}, c_{2}\in \Cc$, where $du$ is the right Haar measure of $N(F)$. If $a\neq d$, then $c_{1} =c_{2} = 0$: otherwise, we may choose $u$ such that $c_{1}\psi_{N}(u) \neq c_{2}\psi_{N}(\delta^{-1}u\delta)$, and then taking a test function $\phi$ that is the characteristic function of a small neighborhood of $u$ gives a contradiction. 
If $a = d$, then $\pre{\iota}\Delta = \Delta$ so $\Delta = 0$. 
\end{proof}



\begin{proof}[proof of Theorem \ref{nonarchmultone} when $n = 2$]
The representation $\pi’(g) = \pi(\iota(g^{-1}))$ is isomorphic to $\pi_1$, so to $\hat{\pi}$. Hence we have a pairing $V \times V \to \Cc$ s.t. $\langle \pi(g)v,w \rangle = \langle v,\pi(\iota(g))w\rangle$. By Riesz representation theorem, any linear functional $\Lambda$ corresponds to a vector $[\Lambda]$ by $\langle v, [\Lambda]\rangle = \Lambda(v)$. 
We can also define another linear functional $\Lambda * \phi$ for any $\phi \in \calH$ by
$$
(\Lambda * \phi)(\xi) = \Lambda(\pi(\phi)\xi) = \int_{G} \Lambda(\pi(g)\xi)\phi(g)dg
$$ 
which satisfies the associativity $\Lambda * (\phi_1 * \phi_2) = (\Lambda * \phi_1) * \phi_2$. 
It satisfies following transformation laws:
\begin{align*}
\pi(g) [\Lambda*\phi] &= [\Lambda * \rho(\iota(g^{-1}))\phi] \\
[L* \phi] &= \pi(\pre{\iota}{}\phi)[L]\\
[\Lambda * \lambda(u)\phi] &= \psi_N(u)[\Lambda * \phi]
\end{align*}
(Second one holds for smooth $L$, and the third one holds for Whittaker functionals.) 
 Now if $\Lambda_1,\Lambda_2$ are Whittaker functionals, define a distribution $\Delta(\phi) = \Lambda_2([\Lambda_1 * \phi])$. 
 The above transformation properties imply that this is a Whittaker distribution, so it is invariant under the involution by the previous theorem. Using that, we can show $\Lambda_1 * \phi = 0 \Rightarrow \Lambda_2 * \phi = 0$. One can show that for any given nonzero linear functional $\Lambda$ on $V$, any vector in V has a form of $[\Lambda * \phi]$ for some $\phi\in \calH$. 
Then  we can define a map $T:[\Lambda_1 * \phi] \mapsto  [\Lambda_2 * \phi]$, which is an intertwining map by the above transformation law. By Schur’s lemma, $T = cI$ for some $c$, and this gives $\Lambda_2 = c\Lambda_1$. 
\end{proof}

Since we just proved uniqueness, we wonder about existence. We will prove that any irreducible representation of $\GL(2, F)$ of $\dim >1$ has a Whittaker model. For this, we need a concept of (twisted) Jacquet functor.
\begin{definition}[Jacquet functor]
Let $(\pi, V)$ be a smooth representation of $B(F)$. Let $V_N$ be a subspace of $V$ generated by elements of the form $\pi(u)v - v$ for $u\in N(F)$ and $v\in V$. 
Then $V_N$ is $T(F)$-invariant, and we get a $T(F)$-module $J(V):= V/V_N$. The smooth representation $(\pi_N, J(V))$ of $T(F)$ obtained in this way is called Jacquet module of $V$. 
$J$ is a functor from the category of $B(F)$-modules to the category of $T(F)$-modules. 
\end{definition}
We can also define the twisted version of the Jacquet functor. 
\begin{definition}[Twisted Jacquet functor]
Fix a nontrivial additive character $\psi$ of $F$. 
Let $V_{N, \psi}$ be the subspace generated by elements of the form $\pi(u)v - \psi_{N}(u)v$ for $u\in N(F)$ and $v\in V$. Then $J_{\psi}(V) := V/V_{N, \psi}$ is a $Z(F)$-module, and $J_{\psi}$ is a functor from the category of $B(F)$-modules to the category of $Z(F)$-modules. 
\end{definition}
First important property of these functors is exactness. 
\begin{proposition}
The functor $J$ and $J_{\psi}$ are exact. 
\end{proposition}
\begin{proof}
Let $0\to V'\to V\to V'' \to 0$ be a short exact sequence of $B(F)$-modules. 
Then we can prove that the induced sequence $0\to V_{N}' \to V_N \to V_{N}'' \to 0$ is also exact. Here we use the following characterization: $x\in V_N$ iff 
$$
\int_{\frap^{-n}} \pi\pmat{1}{x}{}{1} v dx = 0
$$
for sufficiently large $n$. Now we get the result by the snake lemma. 
Proof for $J_{\psi}$ is similar. 
\end{proof}

Another important property of the twisted Jacquet module is that it is directly related to the space of Whittaker functionals. This is almost direct from the definition. 
\begin{proposition}
The space of Whittaker functionals on $V$ is isomorphic to the dual space of $J_{\psi}(V)$. Therefore, if $(\pi, V)$ is an irreducible admissible representation, $\dim J_{\psi}(V) \leq 1$.
\end{proposition}

Now we will prove our main result - existence of a Whittaker functional. 
For a $B(F)$-module $(\pi, V)$, we can associate sheaf of $C^{\infty}_{c}(F)$-module by defining the $(C_{c}^{\infty}(F), \cdot)$-action as
$$
\phi \cdot v = \rho(\wh{\phi})v = \int_F \wh{\phi}(x) \pi \pmat{1}{-x}{}{1} v dx. 
$$
(Here we fix a nonzero additive character $\psi$.) One can check that $V$ is a cosmooth $C_{c}^{\infty}(F)$-module under this action, so we have a sheaf $\calS(V)$ associated to $V$. Using this, we can prove our theorem. 
\begin{theorem}[Existence of Whittaker functional]
\label{dim1}
Any irreducible representation of $\GL(2)$ of dimension greater that 1 has a nonzero Whittaker functional. If there's no nonzero Whittaker functional, then it factors through the determinant map. 
\end{theorem}
Note that this is not true for $\GL(n)$, but it is still true for \emph{generic} representations in the sense of Gelfand-Kirillov dimension. 


\begin{proof}
Assume that $(\pi, V)$ has no nonzero Whittaker functional. Fix an additive char $\psi$ of $F$ and corresponding self-dual Haar measure. Let $\psi_a:F\to \Cc^{\times}$ be a character $\psi_a(x)=\psi(ax)$. Then the stalk of the above sheaf is given by
 $$
\calS(V)_{a} \simeq \begin{cases} J(V) & a = 0 \\ J_{\psi_{a}}(V) \simeq J_{\psi}(V) & a\neq 0 \end{cases}
$$
so $\calS(V)_a = 0$ for $a\neq 0$ and $\calS(V)$  is a skyscraper sheaf at $a=0$. Then $V\to S(V)_0=J(V)$ is an isomorphism, so that $V_N=0$ and $N(F)$ acts trivially. Then all conjugates of $N(F)$ also acts trivially, so does $\SL(2, F)$ (they generate $\SL(2,F)$) and factors through determinant map. 
\end{proof}

Like archimedean case, we can also think the Whittaker functional as a \emph{Whittaker model}, which gives a concrete model of a given representation. This is almost same as the archimedean case. 
In non-archimedean case, we also have \emph{Kirillov model}, which is a model given by functions on $F^{\times}$. 
\begin{definition}[Whittaker model and Kirillov model]
Assume that $(\pi, V)$ is an infinite dimensional irreducible admissible representation, so that it has a nonzero Whittaker functional.  The space $\calW$ of the Whittaker model consists of tunctions $W_{v}:\GL(2, F) \to \Cc$ for $v\in V$ of the form $W_{v}(g) = \Lambda(\pi(g)v)$. From $W_{\pi(g)v}(h) = W_{v}(hg)$,  $\calW$ is closed under the right translation by $\GL(2, F)$, and the resulting representation is isomorphic to $(\pi, V)$. 

The Kirillov model of $(\pi, V)$ is the space of functions $\phi_v:F^{\times} \to \Cc$ for $v\in V$ defined by 
$$
\phi_v(a) = W_{v} \pmat{a}{}{}{1}. 
$$
\end{definition}
Note that if Whittaker functional is nonzero, then the Kirillov model is also nonzero. (For the proof, see Proposition 4.4.6 and 4.4.7 in \cite{bu}.) It is not easy to describe the action of $\GL(2, F)$ on $\calK$, but the action of $B(F)$ is rather easy: 
$$
\pi\pmat{a}{}{}{1} \phi(x) = \phi(ax), \quad \pi\pmat{1}{b}{}{1}\phi(x) = \psi(bx)\phi(x). 
$$
With the action of center by central character, this completely determines the action of $B(F)$. We will investigate $\calK$ more explicitly in the later chapter. 

Another important property of Jacquet functor is that it sends an admissible representation to an admissible representation. Proof uses Iwahori subgroups and the Iwahori factorization. See page 466--469 of \cite{bu} for the proof. 
\begin{theorem}[Harish-Chandra]
\label{jacadm}
If $(\pi, V)$ is an admissible representation of $\GL(2, F)$, then the corresponding representation $(\pi_N, J(V))$ of $T(F)$ is also admissible.
\end{theorem}




\subsection{Classification}
Now we introduce the classification of irreducible admissible representations of $\GL(2, F)$. Definition of each terms will be defined in following sections. 
\begin{theorem}
Irreducible admissible representation of $\GL(2, F)$ is isomorphic to one of the following:
\begin{enumerate}
\item Principal series representations $\pi(\chi_{1}, \chi_{2})$, where $\chi_{1}, \chi_{2}:F^{\times} \to \Cc^{\times}$ are quasi-characters of $F^{\times}$ satisfying $\chi_{1}\chi_{2}^{-1} \neq |\cdot |^{\pm 1}$. 
\item Special representations (or twisted Steinberg representations) $\sigma(\chi_{1}, \chi_{2})$. 
\item 1-dimensional representations $g\mapsto \chi(\det(g))$ for some $\chi:F^{\times}\to \Cc^{\times}$. 
\item Supercuspidal representations.
\end{enumerate}
\end{theorem}
Classification of representation of $\GL(2, \Ff_{q})$ (over a finite field) is almost same. For details, see chapter 4.1 of \cite{bu}. For the later chapters, we will study about these representations. 

Later, we will see that \emph{global} automorphic representations decomposes as a product of local representations. For almost all place $v$, the $v$-part of the representation will be \emph{spherical} principal series representation, which corresponds to certain characters $\chi_1, \chi_2$. 
The classification of unitarizable principal series representation is somehow similar to the archimedean theory.


\subsection{Principal series representations}
In short, \emph{principal series representations} are representations induced by characters of Borel subgroup (same as archimedean case). 
Usually, they are irreducible, but there are some special cases that the induced representations are not irreducible. In that case, it has an infinite dimensional irreducible subrepresentation (or quotient) with 1-dimensional complement. Such infinite dimensional representation is called (twisted) Steinberg representations. 

The definition of induced representation is slightly different from that for finite groups. We need some extra factors for latter purpose. We did the same thing for archimedean case (see Section 2.4). 
\begin{definition}[Induced representation]
Let $G$ be a totally disconnected locally compact group, and let $H$ be a closed subgroup. 
Let $(\pi, V)$ be a smooth representation of $H$. The induced representation $\Ind_{H}^{G}\pi$ is the space of functions $f:G\to V$ such that 
\begin{enumerate}
\item We have 
$$
f(hg) = \delta_G(h)^{-1/2} \delta_H(h)^{1/2}\pi(h)f(g)
$$
for all $h\in H, g\in G$, where $\delta_G$ and $\delta_H$ are the modular quasi-characters of $G$ and $H$. 
\item There exists an open subgroup $K_0$ of $G$ such that $f(gk) = f(g)$ for all $g\in G$ and $k\in K_0$. 
\end{enumerate}
Then $G$ acts on this space by the right translation, and this gives induced representation of $G$. 

Similarly, we can define compact induction, as a space of functions that satisfies above conditions and compactly supported modulo $H$ (image of the support of $f$ in $H\bs G$ is compact). It is denoted by $\cInd_{H}^{G}\pi$. The are same if $H\bs G$ is compact. 
\end{definition}
By definition, (compact) induced representations are also smooth. As before, we have Frobenius reciprocity, which we need extra factor again. 

\begin{proposition}[Frobenius reciprocity]
Let $G$ be a totally disconnected locally compact group and $H$ a closed subgroup. 
Let $(\pi, V)$ and $\sigma, W)$ be smooth representations of $H$ and $G$, respectively. 
Then there is a natural isomorphism $\Hom_G(\sigma, \Ind_H^{G}\pi) \simeq \Hom_H(\sigma|_{H}, \pi \otimes (\delta_{G}^{-1}\delta_{H})^{1/2})$
\end{proposition}
\begin{proof}
For $\Phi\in \Hom_G(\sigma, \Ind_{H}^{G}\pi)$, define $\phi\in \Hom_H(\sigma|_{H}, \pi\otimes (\delta_{G}^{-1}\delta_{H})^{1/2})$ by $\phi(w) = \Phi(w)(1)$. 
Conversely, if $\phi\in \Hom_H(\sigma|_{H}, \pi\otimes (\delta_{G}^{-1}\delta_{H})^{1/2})$ is given, we can define $\Phi\in \Hom_G(\sigma, \Ind_{H}^{G}\pi)$ as $\Phi(w)(g) = \phi(\sigma(g)w)$. 
It is easy to check that these maps give isomorphisms between two spaces. 
\end{proof}

Now we can define principal series representation. (Compare this with the archimedean case in Chapter 2.4.)
\begin{definition}
Let $G = \GL(2, F)$ and $H = B(F)$. 
Let $\chi_1, \chi_2$ be quasi-characters of $F^{\times}$. 
Then we define a quasi-character $\chi$ of $B(F)$ by 
$$
\chi\pmat{y_1}{*}{}{y_2} = \chi_1(y_1)\chi_2(y_2).
$$
Let $\calB(\chi_1, \chi_2) = \Ind_{B(F)}^{\GL(2, F)} \chi$. 
So this is a space of smooth functions $f:G\to \Cc$ which satisfies 
$$
f(bg) = \left| \frac{b_{1}}{b_{2}}\right|^{1/2} \chi_{1}(b_{1})\chi_{2}(b_{2})f(g)
$$
for all $b = \smat{b_{1}}{*}{}{b_{2}}\in B(F)$ and $g\in \GL(2, F)$. 
If $\calB(\chi_1, \chi_2)$ is irreducible, then we call it principal series representation. We denote its isomorphism class as $\pi(\chi_1, \chi_2)$. 
\end{definition}
We can also consider the compact induction $\cInd_{B(F)}^{\GL(2, F)}\chi$. In this case, they agree since $B(F)$ is cocompact by the following theorem:
\begin{proposition}[Iwasawa decomposition]
Let $G = \GL(n, F)$ and $B(F)B$ be the Borel subgroup of $G$. 
Let $K = \GL(n, \calO_{F})$ be the maximal compact subgroup of $G$. 
Then $G = B(F)K$ and $B(F)\bs G$ is compact. 
\end{proposition}
\begin{proof}
Use induction on $n$. One can find $k_1\in K$ such that 
$$
gk_{1} =\pmat{g_{n-1}}{*}{0}{x_{n}}
$$
for some $g_{n-1}\in \GL(n-1, F)$. By induction hypothesis, there exists $k'\in \GL(n-1, \calO_F)$ such that $g_{n-1}k'$ is upper triangular. Then $k = k_1 \smat{k'}{}{}{1}$ makes $gk \in B(F)$. 
Then the map $K \to B(F)\bs G$ is continuous and surjective, so the coset $B(F)\bs G$ is compact. 
\end{proof}

Now we can ask some basic questions about principal series representations: 
\begin{itemize}
\item When $\calB(\chi_{1}, \chi_2)$ is irreducible?
\item What is a contragredient representation of given principal series representation?
\item When two principal series representations are isomorphic?
\item What is a Jacquet module of it?
\end{itemize}

Note that $\Hom_G(V, V) =1$ \emph{does not} imply that $V$ is irreducible, since the representation may not be unitary. So we need other approach to prove irreducibility. 
We will use Jacquet functor that we defined in the previous section. 

First, we will prove uniqueness of Whittaker functional, and use it to prove irreducibility. The argument is similar to the proof Theorem \ref{nonarchmultone}. 
(We can't use the uniqueness result in the previous section since we don't know whether the representation is irreducible or not.) 

\begin{theorem}
\label{psrepwh}
Principal series representations admit at most one Whittaker functional. 
\end{theorem}
\begin{proof}
Define $P:C^{\infty}_{c}(\GL(2,F))\to V$ by convolutioning with $\delta^{1/2}\chi$ over $B(F)$, where $\delta = \delta_{B(F)}$,
$$
(P\phi)(g) = \int\dpl{B(F)} \phi(b^{-1}g) (\delta^{1/2}\chi)(b)db.
$$
(Here $db = d_{L}b$.) One can show that $P$ is surjective by choosing $\phi = \frac{1}{|K\cap B(F)|}\chf_{K}f$ for $f\in V$. 
Also, it satisfies $P(\lambda(b)^{-1}\phi) = (\delta^{-1/2}\chi)(b)P(\phi)$ and $P(\rho(g)\phi) = \pi(g)P(\phi)$ for all $b\in B(F)$ and $g\in \GL(2, F)$. 

Now let $\Lambda: V\to \Cc$ be a Whittaker functional. Define $\Delta\in \fraD(\GL(2, F))$ as $\Delta(\phi) = \Lambda(P\phi)$. 
Then it satisfies $\lambda(b) \Delta = (\delta^{-1/2}\chi)(b)\Delta$ and $\rho(n)\Delta = \psi_{N}(n)^{-1}\Delta$ for all $b\in B(F)$ and $n\in N(F)$. 
Now we use Bruhat decomposition again: we have an exact sequence 
$$
0\to \fraD(B(F)) \to \fraD(\GL(2, F)) \to \fraD(X) \to 0
$$
where $X = \GL(2, F) -  B(F) = B(F)w_{0}N(F)$ where $w_{0} = \smat{}{-1}{1}{}$. 
Let $\Delta_1\in \fraD(X)$ be a distribution satisfies the above transformation laws. 
By the Proposition \ref{distuniq}, there exists $c\in \Cc$ such that 
$$
\Delta_{1}(\phi) = c\int\dpl{B(F)}\int\dpl{N(F)} \phi(bw_{0}n^{-1})\psi_{N}(n)(\delta^{1/2}\chi^{-1})(b)db\,dn.
$$
Also, let $\Delta_2\in \fraD(B(F))$ be a distribution satisfies the above transformation laws. 
By the Proposition \ref{distuniq} again, there exists $c\in \Cc$ such that
$$
\Delta_2(\phi) = c\int\dpl{B(F)} \phi(b) (\delta^{1/2}\chi^{-1})(b)db
$$
for all $\phi\in C_{c}^{\infty}(B(F))$. 
Then $\rho(n)\Delta_2 = \Delta_2$ for $n\in N(F)$, so together with $\rho(n)\Delta_2 = \psi_{N}(n)^{-1}\Delta_2$ we get $\Delta_2 =0$. 
By combining these two results, we can show that the space of $\Delta\in \fraD(\GL(2, F))$ satisfying the above equations is one dimensional. 
\end{proof}


Let's answer the second question first. It is relatively easy to answer the second question, and we will use this for the first question. 
\begin{theorem}
contragredient representation of $\calB(\chi_1, \chi_2)$ is $\calB(\chi_1^{-1},\chi_2^{-1})$. \end{theorem}
\begin{proof}
Let $(\pi, V) = \calB(\chi_1, \chi_2)$ and $(\pi', V') = \calB(\chi_1^{-1}, \chi_{2}^{-1})$. 
Then we can define a non-degenerate pairing $\langle\,,\,\rangle : V\times V' \to \Cc$ by 
$$
\bra{f}{f'} = \int\dpl{K} f(k)f'(k)dk
$$
which is $G$-invariant. 
\end{proof}

Now we return to the first question. The following lemma proves that if $\calB(\chi_1, \chi_2)$ has a 1-dimensional subrepresentation (or quotient), then $\chi_1, \chi_2$ should satisfy some relation. 
\begin{lemma}
If $\calB(\chi_1, \chi_2)$ has an 1-dimensional invariant subspace, then $\chi_{1}\chi_{2}^{-1} = |\cdot |^{-1}$. Similarly, if $\calB(\chi_1, \chi_2)$ admits a one-dimensional quotient representation, then $\chi_{1}\chi_{2}^{-1} = |\cdot |$. 
\end{lemma}
\begin{proof}
For a fixed vector $f$, $\pi$ acts as a character, which factors through commutator subgroup $\SL(2,F)$, hence $\pi(g) = (\rho\circ\det)(g)$ for some quasi-character $\rho$. Then $f(bg) = (\delta^{1/2} \chi)(b) \rho(\det(g)) f(1)$, and taking $b = g^{-1} =\smat{y}{}{}{y^{-1}}$ gives the result. Second one follows from taking dual of first one. 
\end{proof}

Now we can examine when $\calB(\chi_1, \chi_2)$ is irreducible. 
\begin{theorem}
$\calB(\chi_1, \chi_2)$ is irreducible if and only if $\chi_1^{-1}\chi_2 \neq  |\cdot|^{\pm 1}$. 
\end{theorem}
\begin{proof}
We use exactness of twisted Jacquet module and its relation to Whittaker models. 
Assume that $V$ is not irreducible, so it has a nontrivial proper invariant subspace $0\subsetneq V'\subsetneq V$. 
Let $V'' = V/V'$ and let $\pi', \pi''$ be the corresponding representations. 
Then we get an exact sequence
$$
0\to J_{\psi}(V') \to J_{\psi}(V) \to J_{\psi}(V'') \to 0. 
$$
Since $\dim J_\psi(V) \leq 1$, at least one of $J_\psi(V')$ or $J_{\psi}(V'')$ is zero. 
If $J_{\psi}(V') = 0$, then $\pi'$ factors through $\det : \GL(2, F) \to F^{\times}$ by Theorem \ref{dim1}. 
One can prove that admissible representation of $F^{\times}$ contains a 1-dimensional invariant subspace, so we get $\chi_1 \chi_2^{-1} = |\cdot|^{-1}$ by the previous lemma. In this case, the function $f(g) = \chi(\det(g))$ spans an invariant 1-dimensional subspace when $\chi_{1}(y) = \chi(y)|y|^{-1/2}$ and $\chi_{2}(y) = \chi(y)|y|^{1/2}$. 
We can prove another case $J_\psi(V'') = 0$ by dualizing. 
\end{proof}
When it is irreducible, we denote the isomorphism class as $\pi(\chi_1, \chi_2)$. If it is reducible, we showed that it has two composition factors in its Jordan-H\"older series, a 1-dimensional factor and an infinite dimensionalfactor. 
In either case, the infinite dimensional factor is irreducible and we denote its isomorphism class as $\sigma(\chi_1, \chi_2)$. 
(Irreducibility of $\sigma(\chi_1, \chi_2)$ can be proved by using $J_{\psi}$ again. If it is not irreducible, we may assume $\chi_{1}^{-1}\chi_{2} = |\cdot |^{-1}$ and $\sigma(\chi_1, \chi_2)$ has a 1-dimensional subrepresentation. 
From this, we get a 2-dimensional subrepresentation of $\calB(\chi_1, \chi_2)$. 
However, one can show that every finite dimensional representation of $\GL(2, F)$ factors through the determinant by no small subgroup argument, and $\GL(2, F) = B(F) \SL(2, F)$ proves that such representation should be 1-dimensional. In other words, the only finite dimensional representation of $\calB(\chi_1, \chi_2)$ is 1-dimensional.)
Such representation is called a \emph{special} or \emph{Steinberg} representation. 
The 1-dimensional factor is denoted $\pi(\chi_1, \chi_2)$ again. There are also other kinds of representations - supercuspidal representations - which we will see later. 

Now, let's answer the next question. When two principal series representations  are isomorphic?
\begin{theorem}
If $\calB(\chi_1, \chi_2)\simeq \calB(\mu_1, \mu_2)$, then $\chi_1 = \mu_1$ and $\chi_2 = \mu_2$, or $\chi_1 = \mu_2$ and $\chi_2 = \mu_1$. 
\end{theorem}
\begin{proof}
By Frobenius reciprocity, intertwining map corresponds to a linear functional $\Lambda:V\to \Cc$ with $B(F)$-module structure on $\Cc$ by means of the quasi-character $\delta^{1/2}\mu$. 
Then $\Delta = \Lambda \circ P$ (here $P:C_{c}^{\infty}(\GL(2, F))\to V$ is the convolutioning map we defined in Theorem \ref{psrepwh})  is a nonzero distribution and satisfies $\lambda(b)\Delta = (\delta^{-1/2}\chi)(b)\Delta$ and $\rho(b)\Delta = (\delta^{-1/2}\mu^{-1})(b)\Delta$ for $b\in B(F)$. 
From the exact sequence of distribution, there exists a nonzero distribution that satisfies same equations in either $\fraD(B(F))$ or $\fraD(\GL(2, F) - B(F))$. 

First, assume that  $\Delta\in \fraD(\GL(2, F) - B(F))$ is a such nonzero distribution. By Proposition \ref{distuniq}, we have
$$
\Delta(\phi) = \int\dpl{B(F)}\int\dpl{N(F)} \phi(bw_{0}n^{-1}) (\delta^{1/2}\chi^{-1})(b) db\,dn
$$
after adjusting by a nonzero constant. If we apply $\rho(t)$ action and using a change of variable, we have
\begin{align*}
(\delta^{-1/2}\mu^{-1})(t)\Delta(\phi) &= (\rho(t)\Delta)(\phi) \\
&=\int\dpl{N(F)}\int\dpl{B(F)} \phi(bw_{0}n^{-1}t^{-1})(\delta^{1/2}\chi^{-1})(b)db\,dn \\
&=\delta(t)^{-1}(\delta^{1/2}\chi^{-1})(w_{0}tw_{0}^{-1})\Delta(\phi).
\end{align*}
where the last equality follows from the change of variables $n\mapsto t^{-1}nt, b\mapsto bw_{0}tw_{0}^{-1}$. From $\delta(t) = \delta(w_{0}tw_{0}^{-1})^{-1}$, we have $\mu(t) = \chi(w_{0}tw_{0}^{-1})$ and so $\chi_1 = \mu_2$ and $\chi_2 = \mu_1$. 

Another case is similar. By using the integral form of the distribution with change of variables, we can show that for $b\in B(F)$ we have $(\delta^{-1/2}\mu^{-1})(b)\Delta = \rho(b)\Delta = (\delta^{-1/2}\chi^{-1})(b)\Delta$, so $\chi_1 = \mu_1$ and $\chi_2 = \mu_2$. 
\end{proof}

We can even write the isomorphism $\calB(\chi_1, \chi_2) \to \calB(\chi_2, \chi_1)$ explicitly via \emph{intertwining integral}. 
We will define such map $M:\calB(\chi_1, \chi_2) \to \calB(\chi_2, \chi_1)$ as an integral
$$
Mf(g) = \int\dpl{F} f\left( \pmat{}{-1}{1}{}{}\pmat{1}{x}{}{1} g\right) dx.
$$
Formally, this gives an intertwining map since
$$
Mf\left(\pmat{1}{x}{}{1}g\right) = Mf(g)
$$
holds and 
\begin{align*}
Mf\left( \pmat{y_{1}}{}{}{y_{2}}g \right) &= \int\dpl{F} f\left( \pmat{}{-1}{1}{} \pmat{1}{x}{}{1}\pmat{y_{1}}{}{}{y_{2}}g\right) dx \\
&= \int\dpl{F} f\left( \pmat{y_{2}}{}{}{y_{1}} \pmat{}{-1}{1}{} \pmat{1}{y_{2}y_{1}^{-1}x}{}{1}g\right)dx \\
&=\left| \frac{y_{2}}{y_{1}}\right|^{1/2} \chi_{1}(y_{2})\chi_{2}(y_{1}) |y_{1}y_{2}^{-1}| Mf(g) \\
&=\left|\frac{y_{1}}{y_{2}}\right|^{1/2} \chi_{1}(y_{2})\chi_{2}(y_{1}) Mf(g)
\end{align*}
where the third equality follows from the substitution $x\mapsto y_{1}y_{2}^{-1}x$. Also, if $f$ is locally constant then $Mf$ is also locally constant. At last, $M$ is a nonzero map since the function
$$
f(g) = \begin{cases} \left|\frac{y_{1}}{y_{2}}\right|^{1/2} \chi_{1}(y_{1})\chi_{2}(y_{2}) & \substack{g = \smat{y_{1}}{z}{}{y_{2}} \smat{}{-1}{1}{} \smat{1}{x}{}{1}, \\ y_{1}, y_{2}\in F^{\times},\, z\in F,\, x\in \calO_F }\\
0 & \text{otherwise} \end{cases}
$$
is in $\calB(\chi_1, \chi_2)$ and satisfies $Mf(1) = 1$. 

However, the integral may not converges. The integral converges when $\chi_1, \chi_2$ satisfies certain relations: if we fix two unitary characters $\xi_1, \xi_2$ and take $\chi_i(y) = |y|^{s_i}\xi_i(y)$ for $i = 1,2$, then the integral converges when $\Re(s_1 - s_2) >0$. 
\begin{proposition}
If $\Re(s_1 - s_2) >0$, the integral is absolutely convergent and defines a nonzero intertwining map. 
\end{proposition}
\begin{proof}
We have
\begin{align*}
f\left(\pmat{}{-1}{1}{}\pmat{1}{x}{}{1}g\right) &= f\left( \pmat{x^{-1}}{-1}{}{x} \pmat{1}{}{x^{-1}}{1} g\right) \\
&= |x|^{-1}(\chi_{1}^{-1}\chi_{2})(x) f\left( \pmat{1}{}{x^{-1}}{1}g\right).
\end{align*}
Since $f$ is smooth, there exists $N$ such that if $|x|>q^N$, then $$f\left( \pmat{1}{}{x^{-1}}{1}g\right) = f(g).$$ 
Hence the absolute convergence of the integral is equivalent to the convergence of 
$$
\int\dpl{|x|>q^N} |x|^{-1}|(\chi_{1}^{-1}\chi_{2})(x)| dx = \int\dpl{|x|>q^N} |x|^{-s_{1} + s_{2} - 1} dx
$$
and this converges if $\Re(s_1 - s_2) >0$. 
\end{proof}
By interchanging roles of $\chi_1$ and $\chi_2$, we also get a map for $\Re(s_1 - s_2) <0$. 
So we still have a remaining case of $\Re(s_1 - s_2) =0$, which is the most interesting case since it is conjectured that the only when $\Re(s_1) = \Re(s_2) = 0$ can $\calB(\chi_1, \chi_2)$ occur as a constituent in an automorphic cuspidal representation. 
To extend the map for this case, we will analytically continue the map with respect to the complex parameters $s_1, s_2$. There will be a pole when $\chi_1 = \chi_2$, but we don't need to worry about this case since we automatically have $\calB(\chi_1, \chi_2) \simeq \calB(\chi_2, \chi_1)$. 

Let $V_0$ be the space of functions on $K = \GL(2, \calO_F)$ that satisfies
$$
f\left( \pmat{y_{1}}{x}{}{y_{2}}k\right) = \xi_1(y_1)\xi_2(y_2)f(k)
$$
for all $y_1, y_2\in \calO^{\times}_F, x\in \calO_F, k\in K$. For each $f_{0}\in V_{0}$ and $s_1, s_2\in\Cc$, there exists a unique extension $f_{s_{1}, s_{2}}$ of $f_{0}$ to $V_{s_{1}, s_{2}} = \calB(\chi_{1}, \chi_{2})$. We will refer to $(s_{1}, s_{2}) \mapsto f_{s_{1}, s_{2}}$ as a flat section of the family $(\pi_{s_{1}, s_{2}}, V_{s_{1}, s_{2}})$. 

\begin{proposition}
Fix $f_{0}\in V_{0}$ and let $(s_{1}, s_{2})\mapsto f_{s_{1}, s_{2}}$ be the corresponding flat section. 
For fixed $g\in GL(2, F)$, the integral $Mf_{s_{1}, s_{2}}(g)$, originally defined for $\Re(s_{1} - s_{2})>0$, has analytic continuation to all $s_{1}, s_{2}$ where $\chi_1 \neq\chi_2$, and defines a nonzero intertwining operator $V_{s_{1}, s_{2}}\to V_{s_{2}, s_{1}}' = \calB(\chi_2, \chi_1)$. 
\end{proposition}
\begin{proof}
Fix $\xi_1, \xi_2, f_{0}\in V_{0}$ and $g$. Since $f$ is smooth, there exists $N$ such that 
\begin{align*}
Mf_{s_{1}, s_{2}}(g) &= \int\dpl{|x|\leq q^{N}} f_{s_{1}, s_{2}}\left( \pmat{}{-1}{1}{} \pmat{1}{x}{}{1} g\right) dx \\
&+ \int\dpl{|x|\geq q^{N+1}} |x|^{-s_{1} + s_{2} -1} (\xi_{1}^{-1}\xi_{2})(x) dx \cdot f_{s_{1},s_{2}}(g). 
\end{align*}
The first integral converges absolutely (since the domain is compact), so the analytic continuation is clear. 
For the second integral, if $\xi_{1}\xi_{2}^{-1}$ is ramified then 
$$
\int\dpl{|x| = q^{m}} (\xi_{1}^{-1}\xi_{2})(x)dx = 0
$$ 
for all $m\in \Zz$ (consider the change of variable $x\mapsto \alpha x$ with $\alpha\in \calO_{F}^{\times}$, $(\xi_{1}^{-1}\xi_{2})(\alpha)\neq 1$) and so the integral vanishes. 
If $\xi_{1}\xi_{2}^{-1}$ is unramified then there exists $\alpha\in \Cc$ such that $(\xi_{1}\xi_{2}^{-1})(y) = \alpha^{v(y)}$ where $v:F^{\times} \to \Zz$ is the valuation map. 
Then the integral became
\begin{align*}
&\sum_{m = N+1}^{\infty}\int\dpl{|x|= q^{m}} |x|^{-s_{1} + s_{2}} (\xi_{1}^{-1}\xi_{2})(x) \frac{dx}{|x|} f_{s_{1}, s_{2}}(g) \\
&= |\calO_{F}^{\times}| f_{s_{1}, s_{2}}(g) \sum_{m \geq N+1} (\alpha q^{-s_{1} + s_{2}})^{m}. 
\end{align*}
The latter sum equals constant times $(\alpha q^{-s_{1} + s_{2}})^{N+1} (1-\alpha q^{-s_{1}+s_{2}})^{-1}$ for $\Re(s_{1} - s_{2}) >0$, and it has analytic continuation for all $s_{1}, s_{2}$, except where $\alpha q^{-s_{1} +s_{2}} = 1 \Leftrightarrow \chi_{1} = \chi_{2}$. 
This also proves that the analytically continued integral remains an intertwining operator and nonzero. 
\end{proof}

The composition $M'\circ M : \calB(\chi_1, \chi_2)\to \calB(\chi_2, \chi_1)\to \calB(\chi_1, \chi_2)$ is a scalar by Schur's lemma because $\calB(\chi_1, \chi_2)$ is irreducible for most $\chi_1, \chi_2$. 
We can compute this scalar, and it is given by product of two gamma factors. (For the definition of the Tate gamma factor, see Chapter 4.1.)

\begin{proposition}
\label{intcomp}
The scalar $M'\circ M: \calB(\chi_1, \chi_2) \to \calB(\chi_1, \chi_2)$ is given by 
$$
\gamma(1-s_1 + s_2, \xi_{1}^{-1}\xi_{2}, \psi) \gamma(1+s_{1} - s_{2}, \xi_{1}\xi_{2}^{-1}, \psi).
$$
\end{proposition}
\begin{proof}
The proof is based on the uniqueness of Whittaker models. We have a Whittaker functional $\Lambda:\calB(\chi_1, \chi_2)\to \Cc$ defined by 
$$
\Lambda (f) = \int\dpl{F} f\left( w_{0}\pmat{1}{x}{}{1}\right)\psi(-x) dx.
$$
This is absolutely convergent if $\Re(s_1 - s_2) >0$, and we also have an analytic continuation for all $s_1, s_2$. 
Similarly, we have a Whittaker functional $\Lambda' : \calB(\chi_2, \chi_1)\to \Cc$. By uniqueness, there exists $\lambda\in \Cc$ such that $(\Lambda' \circ M)(f) = \lambda\Lambda(f)$. 
We can compute the constant $\lambda$ by inserting suitable $f$. 
This gives us $\lambda = \xi_{1}\xi_{2}^{-1}(-1) \gamma(1-s_1 + s_2, \xi_1^{-1}\xi_2, \psi)$, which implies the desired result. For detail, see Proposition 4.5.9 and 4.5.10 in \cite{bu}. 
\end{proof}
This computation is useful for the functional equations of Eisenstein series. Also, Kazhdan-Petterson, Bank showed that the image of intertwining integral is irreducible by using this. 
Also, this helped Harish-Chandra to compute the Plancherel measure of $\GL(2, F)$. 


At last, we can compute Jacquet module of principal series representations. More precisely, we can compute the action of $T(F)$ on $J(\calB(\chi_1, \chi_2))$ explicitly. 
First, we have a classification of 2-dimensional smooth representations of $F^{\times}$. 

\begin{proposition}
There are two kinds of 2-dimensional smooth representations of $F^{\times}$:
$$
t\mapsto \pmat{\xi(t)}{}{}{\xi'(t)}
$$
for some quasi-characters $\xi, \xi':F^{\times} \to \Cc^{\times}$, or 
$$
t\mapsto \xi(t) \pmat{1}{v(t)}{}{1}
$$
where $v:F^{\times}\to \Zz\subset \Cc$ is the valuation map. 
\end{proposition}
\begin{proof}
One can always find 1-dimensional invariant subspace, so there exists a nonzero vector fixed by the action. Hence there exists a character $\xi:F^{\times} \to \Cc^{\times}$ such that $\rho(t)x = \xi(t)x$ for all $t\in F^{\times}$. 
Since $V/\Cc x$ is 1-dimensional again, $F^{\times}$ acts by quasi-character $\xi'$ on this space. 
If we choose $y\in V - \Cc x$, then we have
$$
\rho(t)y = \xi'(t)y + \lambda(t)x
$$
for some $\lambda:F^{\times} \to \Cc$. From $\rho(tu) = \rho(t)\rho(u)$, we get
$$
\lambda(tu) = \xi'(u)\lambda(t) + \lambda(u)\xi(t).
$$
When $\xi\neq \xi'$, one can show that $\lambda(t) = C(\xi(t) - \xi'(t))$ for some $C\in \Cc$, and then $\rho(t)$ has the above form of diagonal matrix with respect to the basis $\{x, y - Cx\}$. 
If $\xi = \xi'$, $t\mapsto \lambda(t)/\xi(t)$ is a homomorphism from $F^{\times}$ to $\Cc$, and $\calO_F^{\times}$ lies in the kernel since the image is a compact subgroup of $\Cc$, which is trivial. 
Hence $\lambda(t) = c\xi(t)v(t)$ for some $c\in \Cc$, and $\rho$ is isomorphic to the first one if $c =0$, or the second one if $c\neq 0$. 
\end{proof}

\begin{theorem}[Jacquet module of $\calB(\chi_1, \chi_2)$]
\label{psjac}
Let $\chi_1, \chi_2$ be quasi-characters of $F^{\times}$, and let $\chi, \chi'$ be quasi-characters of $T(F)$ defined by 
$$
\chi\pmat{t_1}{}{}{t_2} = \chi_1(t_1)\chi_2(t_2), \quad \chi'\pmat{t_1}{}{}{t_2} = \chi_2(t_1)\chi_1(t_2). 
$$
Then the representation of $T(F)$ on $J(\calB(\chi_1, \chi_2))$ is isomorphic to the following 2-dimensional representation:
\begin{align*}
t\mapsto \begin{cases} \pmat{\delta^{1/2}\chi(t)}{}{}{\delta^{1/2}\chi'(t)} & \chi_1 \neq \chi_2 \\ 
\delta^{1/2}\chi(t) \pmat{1}{v(t_1/t_2)}{}{1} & \chi_1 = \chi_2 \end{cases}
\end{align*}
\end{theorem}
\begin{proof}
First, one can show that $\dim J(\calB(\chi_1, \chi_2)) = 2$ for any $\chi_1, \chi_2$. 
To show this, we can construct two explicit linearly independent linear functionals on $J(V)$, which shows $\dim J(V) \geq 2$. The opposite direction uses the Bruhat decomposition, the exact sequence of distributions, and the Proposition \ref{distuniq}. 

Now consider $T_{1}(F) = \left\{ \smat{a}{}{}{1}\,:\, a\in F^{\times}\right\}$. Since $T_1(F)\simeq F^{\times}$, the action is isomorphic to one of the representations in the previous proposition. 
Since $T(F) = T_{1}(F)Z(F)$ and $Z(F)$ acts as a scalar, $(T(F), J(V))$ is isomorphic to one of the following representations
$$
t\mapsto \pmat{\delta^{1/2}\xi(t)}{}{}{\delta^{1/2}\xi'(t)}\quad\text{or}\quad \delta^{1/2}\xi(t)\pmat{1}{v(t_1/t_2)}{}{1}. 
$$
To distinguish two cases, we use
\begin{align*}
\Hom_{T(F)}(J(V), \delta^{1/2}\eta) \simeq \Hom_{B(F)}(V, \delta^{1/2}\eta) \simeq \Hom_{\GL(2, F)}(V, \calB(\eta_1, \eta_2))
\end{align*}
where $\eta_1, \eta_2$ are quasi-characters of $F^{\times}$ and $\eta\smat{t_1}{}{}{t_2} = \eta_1(t_1)\eta_2(t_2)$ is a quasi-character of $T(F)$, trivially extended to $B(F)$. 
If $\chi_1 \neq \chi_2$, then the $\Hom$ is nonzero iff  $\eta = \chi$ or $\eta = \chi'$, and this is the case when $(\pi_N, J(V))$ has a form of diagonal matrices with $\xi = \chi$ and $\xi' = \chi'$. 
If $\chi_1 = \chi_2$, then the $\Hom$ is nonzero iff $\eta_1 =\eta_2 = \chi_1 = \chi_2$, in which chase it is 1-dimensional by Schur's lemma because $\calB(\chi_1, \chi_2)$ is irreducible. This is possible only if $(\pi_N, J(V))$ is of the second form with $\xi = \chi$. 
\end{proof}




\subsection{Supercuspidal and Weil representations}
We just saw principal series representations and special representations. One can show that if $(\pi, V)$  is an irreducible admissible representation of $\GL(2, F)$, then $\dim J(V) \leq 2$. (See Proposition \ref{jacdim2}.) 
We also know that $\dim J(\calB(\chi_1, \chi_2)) = 2$, and it is known that the converse is true: any irreducible admissible representation $(\pi, V)$ with $\dim J(V) = 2$ is a principal series representation. 
Also, using the exactness of Jacquet functor we can prove that $\dim J(V) = 1$ for 1-dimensional representions or special representations (twisted Steinberg representations).  
So we have only one more case left: when $J(V) = 0$. 
\begin{definition}
Let $(\pi, V)$ a representation of $\GL(2, F)$. If $J(V) = 0$, then $\pi$ is called a supercuspidal representation. 
\end{definition}

Such representations don't come from principal series representations, and they are interesting itself. 
One way to construct such representation is using representation of $\GL(2, \Ff_{q})$. 
We also have a similar classification of irreducible representations of the finite group $\GL(2, \Ff_{q})$, 
and there are so-called cuspidal representations of $\GL(2, \Ff_{q})$, which are representations that there's no nonzero linear functional $l:V\to \Cc$ invariant under $N(\Ff_{q})$.  (For the detailed explanation about representations of $\GL(2, \Ff_{q})$, see the chapter 4.1 of \cite{bu}.) 
One can get supercuspidal representations of $\GL(2, F)$ by using cuspidal representations of $\GL(2, \Ff_{q})$. 
\begin{theorem}
Let $(\pi_0, V_0)$ be a cuspidal representation of $\GL(2, \Ff_{q})$ where $\Ff_{q}= \calO_{F}/ \frap$. Lift $\pi_{0}$ to a representation of $K = \GL(2, \calO_{F})$ under the projection map $K \to GL(2, \Ff_{q})$. 
The central character $\omega_0$ of $\pi_{0}$ is lifted to a character of $\calO_{F}$, and we extend to a unitary character of $F^{\times}$. 
Then we get a representation of $KZ(F)$. Now let $\pi = \cInd_{KZ(F)}^{\GL(2, F)} (\pi_{0})$ be a compact induction of $\pi_0$ to $\GL(2, F)$. 
Then $\pi$ is a unitarizable irreducible admissible supercuspidal representation. 
\end{theorem}
\begin{proof}
The proof uses Mackey's theory, which gives an explicit description of decomposition of the space $\Hom_{G}(V_{1}, V_{2})$ where both $V_{1}, V_{2}$ are induced representations form some closed subgroups of $G$. For the detailed proof, see the Theorem 4.8.1 of \cite{bu}. 
\end{proof}

There's direct way to construct such representations by means of Weil representations. Here we assume that the characteristic of $F$ is not 2. 

\begin{definition}[Heisenberg group]
Let $F$ be a local field of characteristic not 2, and let $\psi:F\to \Cc$ be a nontrivial additive character. Let $V$ be a vector space over $F$ with a nondegenerate symmetric bilinear form $B:V\times V\to \Cc$. 
We define Heisenberg group $H$ by giving a group structure on a set $V\times V\times F$ by 
$$
(v_{1}^{*}, v_{1}, x_{1}) (v_{2}^{*}, v_{2}, x_{2}) = (v_{1}^{*} + v_{2}^{*}, v_{1} + v_{2}, x_{1} + x_{2} + B(v_{1}^{*}, v_{2}) - B(v_{1}, v_{2}^{*}))
$$
for $v_1, v_2\in V, v_1^{*}, v_2^{*}\in V^{*} \simeq V, x_1, x_2\in F$. 
Here we identify $V^{*}$ with $V$ by $v\mapsto (v'\mapsto \psi(-2B(v, v')))$. 
Also, define a group $A(V) = V\times V\times \Tt$ with multiplication $(v_1^{*}, v_1, t_1)(v_2^{*}, v_2, t_2) = (v_1^{*} + v_2^{*}, v_1 + v_2, t_1 t_2 \psi(-2B(v_1, v_2^{*})))$. 
We have a homomorphism $\tau:H\to A(V)$ by $\tau(v^{*}, v, x) = (v^{*}, v, \psi(x)\psi(-B(v, v^{*})))$. 
Also, we have an action of $A(V)$ on $L^{2}(V)$ given by 
$$
(\rho(v^{*}, v, t)\Phi)(u) = t\bra{u}{v^{*}} \Phi(u+v).
$$
Let $\pi= \rho\circ \tau$ be the corresponding representation of $H$ on $L^2(V)$. 

We have actions of $\SL(2, F)$ and $\rO(V)$ on $H$ as 
\begin{align*}
\pre{g}{(v_1, v_2, x)} &= (av_1 + bv_2, cv_1 + dv_2, x), \quad g = \smat{a}{b}{c}{d}\in \SL(2, F) \\
\pre{k}{(v_1, v_2, x)} &= (k(v_1), k(v_2), x), \quad k\in \rO(V)
\end{align*}
We define the Fourier transform as 
$$
\wh{\Phi}(v) = \int\dpl{V} \Phi(u) \psi(2B(u, v))du
$$
for $\Phi\in L^{2}(V)$, where $du$ is the self-dual Haar measure on $V$.
\end{definition}
\begin{theorem}
There exists a unitary projective representation $\omega_1$ of $\SL(2, F)$ on $L^2(V)$ such that $\omega_1(g)\pi(h)\omega_1(g)^{-1} = \pi(\pre{g}{}h)$ for $g\in\SL(2, F)$ and $h\in H$. 
There exists a representation $\omega_2$ of $\rmO(V)$ on $L^{2}(V)$ such that $\omega_2(k)\pi(h)\omega_2(k)^{-1} = \pi(\pre{k}{}h)$ for $k\in \rmO(V)$ and $h\in H$. 
The Schwartz space $\calS(V)$ is invariant under both these representations. 
We have
\begin{align*}
\left(\omega_1\pmat{1}{x}{}{1}\Phi\right)(v) &=\psi(xB(v, v))\Phi(v)  \\
\left(\omega_1\pmat{a}{}{}{a^{-1}}\Phi\right)(v) &= |a|^{d/2}\Phi(av) \\
\omega_1(w_1)\Phi &= \wh{\Phi}, \quad w_1 = \pmat{}{1}{-1}{} \\
(\omega_2(k)\Phi)(v) &= \Phi(k^{-1}v)
\end{align*}
\end{theorem}
To get a true representation of $\SL(2, F)$ (and $\GL(2, F)$), we need to \emph{lift} the projective representation $\omega_1$. 
We can interpret projective representations as a cohomology class in $\rH^{2}(G, \Cc^{\times})$ (or $\rH^{2}(G, \Tt)$ if the representation is unitary), and we can show that when $\dim V$ is even, the the corresponding cohomology class vanishes so the projective representations can be lifted to a true representation. 
\begin{definition}
Let $V$ be a (possibly infinite dimensional) Hilbert space. 
A projective representation of $G$ is a homomorphism $\omega : G\to \PGL(V)$ where $\PGL(V) = \GL(V)/Z(\GL(V))$. By definition, there exists a lift $\omega' : G\to \GL(V)$ and $c:G\times G\to \Cc^{\times}$ such that $\omega'(g_{1}g_{2}) = c(g_{1}, g_{2})\omega'(g_{1})\omega'(g_{2})$ for all $g_{1}, g_{2}\in G$. 
Such $c$ defines a cohomology class in $\rH^{2}(G, \Cc^{\times})$. 
Also, we can do the same thing for unitary representations by replacing $\PGL(V), \GL(V), \Cc^{\times}$ to $\PU(V), \rU(V), \Tt = \{z\in \Cc\,:\, |z| = 1\}$, where $U(V)$ is a unitary group that preserves inner product on $V$. 
\end{definition}

\begin{theorem}
Let $V$ be a $F$-vector space of even dimension. Then the cohomology class in $\rH^{2}(\SL(2, F), \Tt)$ attached to $\omega_1$ is trivial. 
Hence there exists a lift of $\omega_1$ to the true representation of $\SL(2, F)$.  
\end{theorem}
The proof uses quaternion algebra and Hilbert symbol. 
Note that this is false for even dimension, and the corresponding cohomology class of $\rH^{2}(\SL(2, F), \Tt)$ defines an important central extension of $\SL(2, F)$ called the \emph{metaplectic group}. 

Using this, we can construct a supercuspidal representation of $\GL(2, F)$. 
Let $\omega = \omega_{0}\boxtimes \omega_{2}$ be the representation of $\SL(2, F)\times \rO(V)$, and let $\omega_\infty$ be the restriction of $\omega$ to $C_{c}^{\infty}(V)$. 
Howe conjectured that there's certain duality between representations of $\SL(2, F)$ and $\rO(V)$. 
\begin{definition}
Let $\pi_1, \pi_2$ be irreducible admissible representations of $\SL(2, F)$ and $\rO(V)$. 
We say that $\pi_1$ and $\pi_2$ correspond if there exits a nonzero $\SL(2, F)\times \rO(V)$ intertwining operator $\omega_\infty \to \pi_1\boxtimes \pi_2$. 
\end{definition}
\begin{theorem}[Howe duality, Waldspurger]
For each irreducible admissible represenattion of $\SL(2, F)$, there exists at most one irreducible admissible representation of $\rO(V)$ that $\pi_1$ and $\pi_2$ correspond, and vice versa. 
Such correspondence is called theta correspondence. 
\end{theorem}
We are interested in $\GL(2, F)$ rather than $\SL(2, F)$, and it is possible to modify Howe duality as a correspondence between representations of $\GL(2, F)$ and $\GO(V)$, the group of automorphisms of $V$ that preserves $\beta$ up to constant. 

When $\dim V = 2$. In this case, the quadratic space $(V, \beta)$ can be identified with $(E, N)$, where $E$ is a 2-dimensional commutative semisimple algebra over $F$ and $N:E\to F$ is the norm map. 
We have two possible cases: when $\beta$ splits ($E = F\oplus F$ and $N(x, y) = xy$) or not ($E = F(\sqrt{D})$ is a quadratic extension and $N(a+b\sqrt{D}) = a^{2} - b^{2}D$). 
We can embed $E^{\times}$ into $\GO(V) = \GO(E)$ by $x\mapsto (a\mapsto xa)$, and we also have a nontrival involution $\sigma:E\to E$. Those two generates $\GO(V)$ subject to the relation $\sigma^{2} =1$ and $\sigma x \sigma^{-1} = \ol{x}$. 

Now let $\xi:E^{\times} \to \Cc^{\times}$ be a quasicharacter. It is known that $\xi$ can't be extended to the quasicharacter of $\GO(V)$ if and only if $\xi$ does not factor through the norm map $E^{\times} \to F^{\times}$. (This follows from Hilbert's theorem 90.) 
In this case, we get an induced representation of $\GO(V)$, and there exists a corresponding representation of $\GL(2, F)$ under the theta correspondence. This gives a supercuspidal representation. For non-split case, such representation can be described as follows:
\begin{theorem}
Let $E/F$ be a quadratic extension of non-archimedean local fields, and let $\xi$ be a quasicharacter of $E^{\times}$ that does not factor through the norm map $N:E^{\times} \to F^{\times}$. 
Let $U_{\xi, \psi}$ be the space of compactly supported smooth functions on $E$ such that 
$$
\Phi(yv) = \xi(y)^{-1}\Phi(v), \quad \forall y\in E_{1}^{\times} = \ker N
$$
and let $\chi: F^{\times} \to \{\pm 1\}$ be the quadratic character attached to the extension $E/F$. 
Let $\GL(2, F)_{+}$ denote the subgroup of $\GL(2, F)$ consisting of elements whose determinants are norms from $E$. 
Then there exists an irreducible admissible representation $\omega_{\xi, \psi}$ of $\GL(2, F)_{+}$ on $U_{\xi, \psi}$ such that 
$$
\left(\omega_{\xi, \psi}\pmat{a}{}{}{1} \Phi\right)(v) = |a|^{1/2} \xi(b) \Phi(bv)
$$
if $b\in E^{\times}, N(b) = a\in F^{\times}$, 
\begin{align*}
\left(\omega_{\xi, \psi}\pmat{1}{x}{}{1} \Phi\right)(v) &= \psi(xN(v)) \Phi(v)
\left(\omega_{\xi, \psi}\pmat{a}{}{}{a^{-1}} \Phi\right)(v) &= |a|\chi(a)\Phi(av)
\end{align*}
and
$$
\omega_{\xi, \psi}(w_{1})\Phi = \gamma(N)\wh{\Phi}
$$
where the Fourier transform
$$
\wh{\Phi}(v) = \int\dpl{E} \Phi(u) \Tr(u\ol{v}) du.
$$
The representation $\omega_{\xi} = \Ind_{\GL(2, F)_{+}}^{\GL(2, F)}(\omega_{\xi, \psi})$  is irreducible and supercuspidal. 
\end{theorem}
\begin{proof}
For supercuspidality, we can show that restriction of $\omega_{\xi}$ to $B_{1}(F)$ is isomorphic to $\cInd_{N(F)}^{B_{1}(F)}(\psi_{N})$, which is a $B_{1}(F)$-representation on the space $C_{c}^{\infty}(F^{\times})$. Since $V_N = C_{c}^{\infty}(F^{\times})$ and it is irreducible, we get $V = V_N$ and $\omega_{\xi}$ is a supercuspidal representation. 
For details and proof of smoothness, admissibility and irreducibility, see 542p of \cite{bu}. 
\end{proof}
When $E = F\oplus F$, similar construction gives Whttaker models for principal series representations. 
\begin{definition}
For $\Phi\in\calS(E) = \calS(F\oplus F)$, let $W_{\Phi}:\GL(2, F)\to \Cc$ be a function
$$
W_{\Phi}(g) = \int\dpl{F^{\times}} \chi(t) (\omega_0(g)\Phi)(t, t^{-1}) d^{\times}t.
$$
Here the Weil representation $\omega_0$ of $\SL(2, F)$ is extended to $\GL(2, F)$ by 
$$
\left(\omega_{0}\pmat{y}{}{}{1}\Phi\right)(v_1, v_2) = \sqrt{|y|}\chi_1(y)\Phi(yv_1, v_2).
$$
Let $\calW = \{W_{\Phi}\,:\, \Phi\in\calS(E)\}$. 
\end{definition}
\begin{proposition}[Jacquet-Langlands]
Assume that $\Re(s_1 - s_2 + 1)>0$ and that $\calB(\chi_1, \chi_2)$ is irreducible. 
Then the space $\calW$ of functions $W_{\Phi}$ comprises the Whittaker model of $\calB(\chi_1, \chi_2)$.
If $\rho$ denotes the action of $\GL(2, F)$ on $\calW$ by right translation, then for $g\in \GL(2, F)$ 
$$
\rho(g)W_{\Phi} = W_{\omega_0(g)\Phi}. 
$$
\end{proposition}

\subsection{Spherical representation and Unitarizability}

In Chapter 3, we will show that every automorphic representation $\pi$ of $\GL(2, \Aa)$ (we will define this in Chapter 3) decomposes as a restricted product of local factors, $\pi = \otimes_{v}\pi_v$ where almost all $\pi_v$ are \emph{spherical} (or \emph{unramified}) representations. 
Spherical representations is defined in the following way. 
\begin{definition}
Let $K = \GL(2, \calO_F)$ be the maximal compact subgroup of $\GL(2, F)$. 
An irreducible admissible representation $(\pi, V)$ of $\GL(2, F)$ is called spherical if it contains a $K$-fixed vector, i.e. $V^{K} = \{v\in V\,:\, \pi(k)v = v\,\forall k\in K\}\neq 0$. 
Such nonzero vector is called a spherical vector.  
\end{definition}
We can show that dual of spherical representations are also spherical. 
\begin{proposition}
If $(\pi, V)$ is a spherical representation of $\GL(2, F)$, then the contradgradient representation $(\wh{\pi}, \wh{V})$ is also spherical. 
\end{proposition}
\begin{proof}
By Theorem 3.2, it is enough to show that $\pi_{1}(g) = \pi(\pre{T}{g}^{-1})$ is spherical. Since $K$ is invariant under transpose, the spherical vector for $\pi$ is also spherical vector for $\pi_1$. 
\end{proof}
One of our aim in this section is to understand the structure of the \emph{spherical Hecke algebra} $\calH_{K} = (C_{c}^{\infty}(K\bs G / K), *)$. 
First, we show that this is commutative, and the proof is almost same as archimedean case (Theorem \ref{archec}), which uses Cartan decomposition and Gelfand's trick. 

\begin{theorem}
\label{nonarchsphcom}
The spherical Hecke algebra $\calH_K$ is commutative. 
\end{theorem}
\begin{proof}
We use $p$-adic version of Cartan decomposition theorem: a complete set of double coset representatives for $K\bs \GL(2, F)/K$ consists of diagonal matrices
$$
\pmat{\varpi^{n_1}}{}{}{\varpi^{n_2}}
$$
where $n_1 \geq n_2$ are integers. Proof is almost same as archimedean case. 
Now define $\iota:\calH_K\to \calH_K$ by $\pre{\iota}{\phi}(g) = \phi(\pre{T}{g})$. 
Then $\pre{\iota}{(\phi_1 * \phi_2)} = \pre{\iota}{\phi_2} * \pre{\iota}{\phi_1}$ holds by direct computation. 
By the Cartan decomposition, double cosets are invariant under transpose and so $\iota$ is the identity map. This implies $\phi_1 * \phi_2 = \phi_2 * \phi_1$. 
\end{proof}

\begin{theorem}
For an irreducible admissible representation $(\pi, V)$ of $\GL(2, F)$,  $\dim V^{K} \leq 1$, and the space of $K$-fixed linear functionals on $V$ is also at most 1-dimensional. 
\end{theorem}
\begin{proof}
Assume that $V^{K} \neq 0$. By Proposition \ref{simple}, $V^{K}$ is a finite dimensionalsimple $\calH_K$-module, so is 1-dimensional since $\calH_K$ is commutative. 
The second assertion follows from the first assertion, since such $L$ would be a $K$-fixed vector in the contragredient representation. 
\end{proof}

Now let $(\pi, V)$ be an irreducible admissible spherical representation and let $v_K\in V$ be a spherical vector. 
Then $\pi(\phi)$ is also spherical for $\phi\in \calH_K$, so there exists $\xi(\phi)\in \Cc$ such that $\pi(\phi)v_K = \xi(\phi)v_K$. 
Such $\xi$ defines a character of $\calH_{K}$, and we cal $\xi$ the \emph{character of $\calH_K$ associated with the spherical representation $(\pi, V)$}. 
We proved that irreducible admissible representations are determined by their characters in Theorem \ref{char}. 
For spherical representation, stronger result holds - $\xi$ determines the representation. 
\begin{theorem}
\label{sphchar}
Let $(\pi_1, V_1)$ and $(\pi_2, V_2)$ be irreducible admissible spherical representations. 
Suppose that the characters of $\calH_K$ associated with $\pi_1, \pi_2$ are equal, i.e. $\xi_1 = \xi_2$. Then $\pi_1\simeq \pi_2$. 
\end{theorem}
\begin{proof}
By Proposition \ref{simple}, it is sufficient to show that $V_{1}^{K_1}\simeq V_{2}^{K_1}$ as $\calH_{K_{1}}$-modules for any open subgroup $K_1\subseteq K$. 
It is known that such $\calH_{K_{1}}$-module structures are determined by matrix coefficients, i.e. a function $c:\calH_{K_1} \to \Cc$ of the form $c_i(\phi) = L_i(\pi(\phi)x)$ for a linear functional $L:V_i\to \Cc$ and $x\in V_i$. 
So it is enough to show that 
$$
\bra{\pi_1(\phi)v_1}{\wh{v_1}} = \bra{\pi_2(\phi)v_2}{\wh{v_2}}
$$
for any $\phi\in \calH$, where $v_i, \wh{v_i}$ are normalized spherical vectors of $V_i, \wh{V_i}$ so that $\bra{v_i}{\wh{v_i}} = 1$ for $i= 1,2$.
 If we define $P: \calH\to \calH$ as $P(\phi) = \epsilon_K * \phi * \epsilon_K$, then we have $P^{2} = P$, $\Img(P) = \calH_K$, and $\calH = \Img(P) \oplus \ker(P) = \calH_K \oplus \ker(P)$. 
So it is enough to show for $\phi\in \calH_K$ and $\phi\in \ker(P)$ separately. 
We can easily check that for $\phi \in \calH_K$, both sides are same as $\xi(\phi)$, and for $\phi\in \ker(P)$, both sides vanishes (here we use $\pi_i(\epsilon_K) v_i = v_i$ and $\wh{\pi_i}(\epsilon_K) = \wh{v_i}$. 
Now restrict the equation for $\phi\in \calH_{K_1}$ for $K_1\subseteq K$ and we get $V_1^{K_1}\simeq V_2^{K_1}$ as $\calH_{K_1}$-modules. 
\end{proof}

We can do more. We can study the precise structure of $\calH_K$ in terms of simple generators and relations. 
For $k\geq 0$, let $T(\frap^{k})$ be the characteristic function of the set of all $g\in \mathrm{Mat}_{2}(\calO_F)$ such that the ideal generated by $\det(g)$ in $\calO$ is $\frap^k$. 
Also, let $R(\frap)\in \calH_K$ be the characteristic function of $K \smat{\varpi}{}{}{\varpi} K = K\smat{\varpi}{}{}{\varpi}$. ($R(\frap)$ is invertible)
Then we have a nontrivial and simple relation, which might be familiar to you. 
\begin{proposition}
\label{heckerel}
For $k\geq 1$, $T(\frap)*T(\frap^k) = T(\frap^{k+1}) + qR(\frap)*T(\frap^{k-1})$. 
\end{proposition}
\begin{proof}
Since $T(\frap)*T(\frap^k), T(\frap^{k+1}), R(\frap)*T(\frap^{k-1})$ are all supported on the double cosets whose determinants generate the ideal $\frap^{k+1}$, so it is sufficient to verify it for the matrices 
$$
\pmat{\varpi^{k+1-r}}{}{}{\varpi^{r}}, \quad r\in \Zz
$$
with $0\leq r\leq k+1$. Using
$$
K \pmat{\varpi}{}{}{1}K = \pmat{1}{}{}{\varpi}K \cup \bigcup_{b\Mod{\frap}} \pmat{\varpi}{b}{}{1} K
$$
we can prove
$$
(T(\frap) * T(\frap^{k}))(g) = T(\frap^{k}) \left( \pmat{1}{}{}{\varpi}^{-1}g\right) + \sum_{b\Mod{\frap}} T(\frap^{k}) \left( \pmat{\varpi}{b}{}{1}^{-1} g\right)
$$
and we get the result by comparing both sides for the above matrices. 
\end{proof}
\begin{proposition}
\label{heckegen}
$\calH_K$ is generated by $T(\frap), R(\frap)$ and $R(\frap)^{-1}$. 
\end{proposition}
\begin{proof}
By Cartan decomposition, a basis of $\calH_K$ are characteristic functions of the double cosets
$$
K\pmat{\varpi^n}{}{}{\varpi^m} K, \quad n\geq m. 
$$
which equals $R(\frap)^m$ times the characteristic function of
$$
K\pmat{\varpi^{n-m}}{}{}{1} K. 
$$
or $T(\frap^{n-m}) - R(\frap)*T(\frap^{n-m-2})$. Since $T(\frap^{k})$ can be generated by $T(\frap)$ and $R(\frap)$, we are done. 
\end{proof}

Now we have a natural question - which representations are spherical? 
First simple but nontrivial examples are principal series representations with unramified chracters. 
\begin{proposition}
Let $\chi_1, \chi_2$ be unramified quasicharacters of $F^{\times}$ such that $\chi_{1}\chi_{2}^{-1}\neq |\cdot|^{\pm 1}$, so that $\calB(\chi_1, \chi_2)$ is irreducible. 
Then it is spherical. 
\end{proposition}
\begin{proof}
Let $\chi\smat{b_{1}}{*}{}{b_{2}} = \chi_{1}(b_{1})\chi_{2}(b_{2})$ be a quasicharacter of $B(F)$. 
Let $\phi_{K, \chi}:\GL(2, F)\to \Cc$ be a function defined as $\phi_{K, \chi}(bk) = (\delta^{1/2}\chi)(b)$. (Recall that any elements in $\GL(2, F)$ can be written as a form of $bk$ for $b\in B(F)$ and $k\in K$ by Iwasawa decomposition.)
Well-definedness follows from unramifiedness of $\chi_1, \chi_2$. 
Also, it is clear that $\phi_{K, \chi}$ is a spherical vector. 
\end{proof}
Is there any other spherical representations? Obviously, there are simpler ones: 1-dimensional representations, which are just unramified quasicharacters of $F^{\times}$. 
We will show that these are all, i.e. there are no other spherical representations. 
For this, we need to know how $\calH_K$ acts on the spherical vector in the spherical principal series representation. 
\begin{proposition}
Let $\phi_K$ be a spherical vector in $\pi(\chi_1, \chi_2)$ wehre $\chi_1, \chi_2$ are unramified quasicharacters of $F^{\times}$. 
Let $\alpha_i = \chi_i(\varpi)$ for $i = 1, 2$. 
Then $T(\frap)\phi_K = \lambda \phi_K$ and $R(\frap)\phi_K = \mu \phi_K$, where
$$
\lambda = q^{1/2}(\alpha_1 + \alpha_2), \quad \mu = \alpha_1 \alpha_2.
$$
\end{proposition}
\begin{proof}
We use the previous decomposition of $K\smat{\varpi}{}{}{1}K$. Since $\phi_K(1) = 1$, 
\begin{align*}
\lambda &= (T(\frap)\phi_K)(1) = \int\dpl{K\smat{\varpi}{}{}{1}K} \phi_{K}(g)dg \\
&= \sum_{\gamma\in K\smat{\varpi}{}{}{1}K/K} \int_K \phi_K(\gamma k)dk \\
&= (\delta^{1/2}\chi)\pmat{1}{}{}{\varpi} + q(\delta^{1/2}\chi)\pmat{\varpi}{}{}{1} = q^{1/2}(\alpha_1 + \alpha_2). 
\end{align*}
For $R(\frap)$, it is much easier:
$$
\mu = (R(\frap)\phi_K)(1) = \int\dpl{K\smat{\varpi}{}{}{\varpi}K} \phi_K(g)dg = (\delta^{1/2}\chi)\pmat{\varpi}{}{}{\varpi} = \alpha_1 \alpha_2. 
$$
\end{proof}
Using this with Theorem \ref{sphchar}, we can prove that the only spherical representations are principal series representations and 1-dimensional representations.
\begin{theorem}
\label{sphps}
Let $(\pi, V)$ be an irreducible admissible spherical representation of $\GL(2, F)$. 
Then either $V$ is 1-dimensional that has a form of $\pi(g) = \chi(\det(g))$ for some unramified quasicharacter $\chi$ of $F^{\times}$, or a spherical principal series representation. 
\end{theorem} 
\begin{proof}
Let $\xi$ be the character of $\calH_K$ associated with the spherical representation $(\pi, V)$ and let $\lambda, \mu$ be the eigenvalues of $T(\frap)$ and $R(\frap)$. 
Since $R(\frap)$ is invertible, $\mu\neq 0$. 
Let $\alpha_1, \alpha_2$ be the roots of the quadratic polynomial $X^{2} - q^{-1/2}\lambda X + \mu = 0$, and let $\chi_1,\chi_2$ be the unramified quasicharacters of $F^{\times}$ with $\chi_i(\varpi) = \alpha_i$. 
Then $T(\frap)$ and $R(\frap)$ have the same eigenvalues $\lambda$ and $\mu$ on $V^{K}$, so the character $\xi$ and the character associated with $\calB(\chi_1, \chi_2)$ coincides. 
So if $\calB(\chi_1, \chi_2)$ is irreducible, $(\pi, V)\simeq \calB(\chi_1, \chi_2)$ by the Theorem \ref{sphchar}. 
If not, we may assume $\alpha_1 \alpha_2^{-1} = q$ so that $\calB(\chi_1, \chi_2)$ has a 1-dimensional subspace. 
By Theorem \ref{sphchar} again, $(\pi, V)$ is isomorphic to this 1-dimensional subrepresentation, and we can check that $\pi(g) = \chi(\det(g))$ where $\chi(y) = |y| \chi_1(y)$. 
\end{proof}

We can also compute the action of intertwining operator on the spherical vector. 
\begin{proposition}
Let $(\pi, V) = \calB(\chi_1, \chi_2), (\pi', V') = \calB(\chi_2, \chi_1)$, and let $M:V\to V'$ be the intertwining map. 
Then we have
$$
M\phi_{K, \chi} = \frac{1-q^{-1}\alpha_{1}\alpha_{2}^{-1}}{1- \alpha_{1}\alpha_{2}^{-1}} \phi_{K, \chi'}. 
$$
\end{proposition}
\begin{proof}
It is clear that $M\phi_{K, \chi}$ is a spherical vector in $V'$, so $M\phi_{K, \chi} = c\phi_{K, \chi'}$ for some constant $c = M\phi_{K, \chi}(1) \in \Cc$. 
For the computation of $c$, we have
\begin{align*}
\int\dpl{F} \phi_{K, \chi}\left(\pmat{}{-1}{1}{} \pmat{1}{x}{}{1}\right) dx &= 1 + \sum_{m\geq 1} |\frap^{-m} - \frap^{-(m-1)}| q^{-m}\alpha_{1}^{m}\alpha_{2}^{-m} \\
&= 1 + \sum_{m\geq 1}(1-q^{-1})(\alpha_{1}\alpha_{1}^{-1})^{m} \\
&= \frac{1-q^{-1}\alpha_{1}\alpha_{2}^{-1}}{1- \alpha_{1}\alpha_{2}^{-1}}.
\end{align*}
\end{proof}

There are two special functions on $\GL(2, F)$ associated with a spherical representation, called \emph{spherical Whittaker function} and \emph{spherical function}. 
Let's study spherical Whittaker model first. 
\begin{definition}
Let $(\pi, V) = \calB(\chi_1, \chi_2)$ be a spherical principal series representation where $\chi_1, \chi_2$ are unramified quasicharacters. 
Define a Whittaker functional $\Lambda$ on $\calB(\chi_1, \chi_2)$ by 
$$
\Lambda(f) = \int\dpl{F} f\left( \pmat{}{-1}{1}{} \pmat{1}{x}{}{1}\right) \psi(-x) dx
$$
where $\psi$ is a nonzero additive character with conductor $\calO_F$. 
We define the spherical Whittaker function $W_{0}:\GL(2, F)\to \Cc$ as $W_{0}(g) = \Lambda(\pi(g)\phi_K)$. 
\end{definition}
The integral absolutely converges if $\chi$ is \emph{dominant}, i.e. $|\alpha_{1}| < |\alpha_{2}|$. 
For general $\chi$, we can define it as a limit
$$
\Lambda(f) = \lim_{k\to\infty} \int\dpl{\frap^{-k}} f\left( w_{0}\pmat{1}{x}{}{1}\right) \psi(-x)dx.
$$
which makes sense and defines a Whittaker functional. 

We can compute the spherical Whittaker function explicitly. 
Note that we have
$$
W_{0}\left( \pmat{1}{x}{}{1} \pmat{z}{}{}{z} gk\right) = \psi(x)\omega(z)W_{0}(g)
$$
for $x\in F, z\in F^{\times}$ and $k\in K$, where $\omega = \chi_1 \chi_2$ is the central quasicharacter of $\calB(\chi_1, \chi_2)$. 
So it is sufficient to compute $W_{0}(g)$ as $g$ runs through a set of coset representatives for $N(F)Z(F)\bs \GL(2, F)/ K$, and by Iwasawa decomposition, it is sufficient to compute 
$$
W_{0}\pmat{\varpi^m}{}{}{1}
$$
for $m\in \Zz$. 
\begin{theorem}
\label{explicitsph}
Let $a_{m} = \smat{\varpi^m}{}{}{1}$. Then we have the following explicit formula
$$
W_{0}(a_m) = \begin{cases} (1-q^{-1}\alpha_1 \alpha_2^{-1}) q^{-m/2} \frac{\alpha_{1}^{m+1} -\alpha_{2}^{m+1}}{\alpha_{1} - \alpha_{2}} & m \geq 0 \\ 0 & m < 0. \end{cases}
$$
\end{theorem}
\begin{proof}
For $m<0$, we have
$$
W_{0}\left( a_m \pmat{1}{x}{}{1} \right) = W_{0}\left( \pmat{1}{\varpi^{m}x}{}{1}a_m \right) = \psi(\varpi^{m}x)W_{0}(a_{m})
$$
and by choosing $x\in \calO_F$ with $\psi(\varpi^{m}x)\neq 1$, we get $W_{0}(a_{m}) = 0$. 
For $m\geq 0$, we use a special basis $\{\phi_0, \phi_1\}$ of $V^{K_{0}(\frap)} \simeq J(V)$, which is called Casselman basis. These are vectors such that $L_{0}(\phi_{0}) = L_{1}(\phi_{1}) = 1$ and $L_{1}(\phi_{0}) = L_{0}(\phi_{1}) = 0$, where $L_{0}, L_{1}$ are linear functionals on $V$ defined as $L_{1}(\phi) = \phi(1)$ and $L_{0}(\phi) = (M\phi)(1)$, which can be regarded as a functional on $J(V)$ or $V^{K_{0}(\frap)}$. One can check that these are nonzero and linearly independent, so form a basis of $J(V)^{*} \simeq (V^{K_{0}(\frap)})^{*}$. 
Using this, one can prove that
$$
W_{0}(a_{m}) = C q^{-m/2}\alpha_{1}^{m} + M\phi_{K, \chi}(1)  q^{-m/2}\alpha_2^m,
$$
for some $C\in \Cc$. Also, $(1-q^{-1}\alpha_{1}\alpha_{2}^{-1})^{-1}W_{0}(g)$ is invariant under the interchange of $\alpha_1$ and $\alpha_2$ since it is a normalized spherical vector and $\calB(\chi_1, \chi_2)\simeq \calB(\chi_2, \chi_1)$. 
Using this, we can compute $C$ and we obtain the formula. 
\end{proof}
Spherical function is defined as $\sigma(g) = \bra{\pi(g)v}{\wh{v}}$ where $v\in V, \wh{v}\in \wh{V}$ are normalized spherical vectors so that $\bra{v}{\wh{v}} = 1$. 
We can also compute this function explicitly. 
Note that $\sigma$ is $K$-biinvariant and $\sigma(\smat{z}{}{}{z}g) = \omega(z)\sigma(g)$, so we only need to compute its values on a coset representatives for $KZ(F) \bs \GL(2, F) / K$. 
By Cartan decomposition, it is sufficient to compute $\sigma(a_{m})$ for $m\geq 0$. 
\begin{theorem}[Macdonald formula] For $m\geq 0$, 
$$
\sigma(a_m) = \frac{1}{1+q^{-1}} q^{-m/2} \left[ \alpha_1^m \frac{1 - q^{-1}\alpha_{2}\alpha_{1}^{-1}}{1-\alpha_{2}\alpha_{1}^{-1}} + \alpha_{2}^{m} \frac{1-q^{-1}\alpha_{1}\alpha_{2}^{-1}}{1-\alpha_{1}\alpha_{2}^{-1}}\right].
$$
\end{theorem}

We can ask a different kind of question. When principal series representations are unitarizable? 
If the representation is induced from unitary data, then it is also unitary. 
\begin{proposition}
If $\chi_1, \chi_2$ are unitary characters of $F^{\times}$, then $\calB(\chi_1, \chi_2)$ is unitarizable. 
\end{proposition}
\begin{proof}
For $f_1, f_2\in V$,
$$
\bra{f_1}{f_2} = \int\dpl{K} f_1(k)\ol{f_2(k)} dk
$$
defines a positive-definite $\GL(2, F)$-invariant Hermitian pairing. 
\end{proof}
However, there may exist other principal series representations that are unitary but not induced from unitary characters.
\begin{proposition}
If $\calB(\chi_1, \chi_2)$ is unitarizable, then either $\chi_1, \chi_2$ are unitary, or $\chi_1 = \ol{\chi_2}^{-1}$. 
\end{proposition}
\begin{proof}
If $\langle \,,\,\rangle$ is the paring, then $(f_1, f_2) := \bra{f_1}{\ol{f_2}}$ is a nondegenerate $\GL(2, F)$-invariant bilinear pairing
$$
\calB(\chi_1, \chi_2) \times \calB(\ol{\chi}_1, \ol{\chi}_{2})\to \Cc,
$$
and so $\calB(\ol{\chi}_{1},  \ol{\chi}_2) \simeq \wh{\calB(\chi_{1}, \chi_2)} \simeq \calB(\chi_1^{-1}, \chi_2^{-1})$. 
\end{proof}
So we want to know when $\calB(\chi, \ol{\chi}^{-1})$ is unitary. 
If we write $\chi(y) = \chi_{0}(y) |y|^{s}$ with a unitary character $\chi_{0}$ and $s\in \Rr$, 
Then $\calB(\chi, \ol{\chi}^{-1})\simeq \chi_{0}\otimes \calB(\chi_s, \chi_s^{-1})$ where $\chi_s(y) = |y|^s$. 
So we are reduced to determine when $\calB(\chi_s, \chi_s^{-1})$ is unitarizable. 
\begin{proposition}
Suppose that $s\neq \pm\frac{1}{2}$ is a real number, so that $\calB(\chi_s, \chi_s^{-1})$ is irreducible. 
Then $\calB(\chi_s, \chi_s^{-1})$ is unitarizable if and only if $-\frac{1}{2} < s < \frac{1}{2}$. 
\end{proposition}
\begin{proof}
We may assume $s<0$. Since $V = \calB(\chi_s, \chi_s^{-1})$ is irreducible, there can be at most one nondegenerate $\GL(2, F)$-invariant sesquilinear pairing on $V$ (up to a constant multiple). 
If we put $M_{s}: \calB(\chi_s, \chi_s^{-1})\to \calB(\chi_s^{-1}, \chi_s)$, then we get a nondegenerate sesquilinear pairing
$$
\bra{f_1}{f_2} = \int\dpl{K} (M_s f_1)(k) \ol{f_{2}(k)} dk. 
$$
Define an Iwahori fixed vector 
$$
f_{0}(g) = \begin{cases} \delta^{s+1/2}(b) & g = bk, b\in B(F), k_{0} \in K_{0}(\frap) \\ 0 & g \in B(F)w_{0}K_{0}(\frap) \end{cases}
$$
where $K_{0}(\frap) = \{ \smat{a}{b}{c}{d}\in \GL(2 ,\calO_{K})\,:\, c\in \frap\}$. 
Then we get
$$
\bra{f_0}{f_0} = \frac{1-q^{-1}}{1+q} \frac{q^{-2s}}{1-q^{-2s}}.
$$
Here the integral converges for $s>0$ and this equation gives an analytic continutation for $s<0$, and this expression is negative for $s<0$. 
For the standard spherical vector $\phi_K$, we have
$$
\bra{\phi_K}{\phi_K} = \frac{1-q^{-1-2s}}{1-q^{-2s}}
$$
which is negative if $-\frac{1}{2} < s < 0$ but positive $s<-\frac{1}{2}$. 
So it can't be unitarizable for $s< -\frac{1}{2}$ (since it is not positive definite) and it remain to show that it is unitarizable for $-\frac{1}{2} < s< 0$. 
To prove this, we consider a new intertwining operator $M_{s}^{*} = (1-q^{-2s})M_{s}$. 
The original $M_{s}$ is not defined at $s = 0$ (it has a pole), but $M_{s}^{*}$ is even defined at $s = 0$ and it varies continuously. 
We know that $\calB(\chi_s, \chi_s^{-1})$ is unitary for $s = 0$, and the new pairing $\bra{f_1}{f_2}^{*} = (1-q^{-2s})\bra{f_1}{f_2}$ become positive definite. 
Now let $\rho$ be an irreducible admissible representation of $K$. 
If it is not positive definite for some $-\frac{1}{2} <s < 0$, then there exists $s$ such that $M_{s}^{*}$ has zero eigenvalue, which means that $M_{s}^{*}$ is not invertible. This contradicts to the fact that $M_{s}$ is nonzero, so invertible for $-\frac{1}{2} <s < 0$. 
Hence $\langle \,,\,\rangle^{*}$ defines a positive definite Hermitian form on $V(\rho)$, so on $V = \oplus_{\rho \in \wh{K}} V(\rho)$. 
\end{proof}
These representations are called \emph{complementary series} representation (recall that there is also complementary series representation for $\GL(2, \Rr)$). In summary, we have the following result. 
\begin{theorem}
$\calB(\chi_1, \chi_2)$ is unitary if and only if either $\chi_1, \chi_2$ are unitary, or else there exists a unitary character $\chi_0$ and a real number $-\frac{1}{2} < s < \frac{1}{2}$ such that $\chi_1(y) = \chi_0(y)|y|^{s}$ and $\chi_2(y) = \chi_0(y) |y|^{-s}$. 
\end{theorem}



\subsection{Local zeta functions and local functional equations}
In this section, we will define local zeta functions and prove local functional equations, which will be used to define and prove global functional equations of global automorphic $L$-functions in Chapter 4. 
We will use some notations from Tate's thesis, so you may have to read Chapter 4.1 first. 
To define local zeta functions, 


First, we will show that Jacquet module controls the asymptotics of the functions in the Kirillov model of $V$. 
This will allow us to define local zeta functions (we will define it as an integral of Kirillov model over $F^{\times}$, and the following results control the convergence).

\begin{proposition}
\label{jacdim2}
Let $(\pi, V)$ be an irreducible admissible representation of $\GL(2, F)$. Then $\dim J(V) \leq 2$, and if it is nonzero, then $\pi$ is isomorphic to a subrepresentation of $\calB(\chi_1, \chi_2)$. 
\end{proposition}
\begin{proof}
Since it is trivial if $J(V) = 0$, assume that $J(V) \neq 0$. 
By Theorem \ref{jacadm}, $J(V)$ is admissible $T(F)$-module, so is its contragredient. 
Since $T(F)$ is abelian, there exists 1-dimensional $T(F)$-invariant subspace of $J(V)^{*}$, which means that there exists a quasicharacter $\chi$ of $T(F)$ and a nonzero linear functional $L:J(V)\to \Cc$ such that $L(\pi_{N}(t)v) = (\delta^{1/2}\chi)(t)L(v)$ for $v\in J(V), t\in T(F)$. 
If we consider $L$ as a linear functional on $V$ that is trivial on $V_N$, we have $L(\pi(b)v) = (\delta^{1/2}\chi)(b)L(v)$ for $v\in V, b\in B$. (Here we extend $\chi$ to $B(F)$ that is trivial on $N(F)$. 
By Frobeinus reciprocity, this corresponds to a nonzero intertwining map $V \to \calB(\chi_1, \chi_2)$, which is injective because of irreducibility of $V$. 
Since Jacquet functor is exact, we have $J(V)\hookrightarrow J(\calB(\chi_1, \chi_2))$ and the result follows from $\dim J(\calB(\chi_1, \chi_2)) = 2$. 
\end{proof}

\begin{proposition}
Let $(\pi, V)$ be an infinite dimensional  irreducible representation of $\GL(2, F)$, and let's identify it with its Kirillov model. 
Then $\phi\in V$ is locally constant and $\phi(y)$ vanishes for sufficiently large $|y|$. 
Also, if $\phi\in V_N$ then $\phi(y) = 0$ for sufficiently small $|y|$, so that $\phi:F^{\times} \to \Cc$ is a compactly supported smooth function. 
\end{proposition}
\begin{proof}
Recall that the action of $B_{1}(F)$ on the Kirillov model is given as
$$
\pi\pmat{a}{}{}{1} \phi(x) = \phi(ax), \qquad \pi\pmat{1}{b}{}{1}\phi(x) = \psi(bx)\phi(x). 
$$
Since $\pi$ is a smooth representation, $\phi$ is fixed by an open subgroup of $T_1(F)$, and the first equation implies that $\phi$ is locally constant. 
Also, $\phi$ is fixed by $N(\frap^k)$ for some $k$, which gives $\phi(y) = \psi(xy) \phi(y)$ for all $x\in \frap^k$. If $|y|$ is sufficiently large, then $\psi(xy) \neq 1$ and so $\phi(y) = 0$. 
For the last claim, a function $\phi' = \pi\smat{1}{x}{}{1}\phi - \phi$ satisfies $\phi'(y) = (\psi(xy) -1 )\phi(y)$, and if $|y|$ is sufficiently small then $\psi(xy) = 1$, hence $\phi' = 0$. 
\end{proof}
Now we can completely understand what is $V_N \subseteq V$ as a Kirillov model. 
\begin{theorem}
\label{kicpt}
Let $(\pi, V)$ be an irreducible smooth representation of $\GL(2, F)$. 
Assume that $V$ is infinite dimensional, so that it has a Kirillov model; identify $(\pi, V)$ with its Kirillov model. 
Then $V_N = C_{c}^{\infty}(F^{\times})$. 
\end{theorem}
\begin{proof}
By the previous proposition, we have $V_N\subseteq C_{c}^{\infty}(F^{\times})$. 
Also, $V_N\neq 0$ since $\dim V = \infty$ and $\dim J(V) <\infty$. So it is enough to show that $C_{c}^{\infty}(F^{\times})$ is an irreducible $B_1(F)$-module. 

Let $U$ be a nonzero invariant subspace of $C_c^{\infty}(F^{\times})$. For any $a\in F^{\times}$, we will show that $U$ contains a characteristic function of any sufficiently small neighborhood of $a$, which proves $U = C_{c}^{\infty}(F^{\times})$. 
Let $0\neq \phi\in U$ and we may assume $\phi(a)\neq 0$. Let $W$ be an open neighborhood of $a$ such that $\phi|_{W} = \phi(a)$. 
Choose $f\in C_{c}^{\infty}(F)$ such that $\wh{f} = \frac{1}{f(a)} \chf_{W}$, then 
$$
\phi_1(y) := \int\dpl{F} f(x) \pi\pmat{1}{x}{}{1}\phi(y) dx = \int\dpl{F}f(x)\psi(xy)\phi(y)dx = \wh{f}(y)\phi(y) =  \chf_{W}(y)
$$
is in $C_{c}^{\infty}(F^{\times})$ since it is a finite sum of elements of $C_{c}^{\infty}(F^{\times})$. 
\end{proof}

Now we will see that we can control the asymptotic of functions in the Kirillov model of representations of $\GL(2, F)$ near 0. 
The previous theorem tells us that if $(\pi, V)$ is a supercuspidal representation, then the functions in the Kirillov model vanishes near 0. 
We will also study the other two cases - principal series representations and special representations. 
\begin{proposition}
Let $(\pi, V)$ be an irreducible representation identified with its Kirillov model. 
Let $\chi$ be a quasicharacter of $T(F)$ and $\chi_1, \chi_2$ be the corresponding quasicharacters of $F^{\times}$. 
Assume that $\phi\in V$ satisfies $\pi_N(t)\ol{\phi} = (\delta^{1/2}\chi)(t)\ol{\phi}$ for all $t\in T(F)$, where $\ol{\phi}$ is the image of $\phi$ in $J(V)$. 
Then there exists a constant $C$ such that
$$
\phi(t) = C|t|^{1/2}\chi_1(t)
$$
for sufficiently small $|t|$. 
\end{proposition}
\begin{proof}
Let $t_0 \in \varpi\calO_F^{\times}$. 
By assumption, 
$$
\pi\pmat{t_0}{}{}{1}\phi - (\delta^{1/2}\chi)\pmat{t_0}{}{}{1} \phi
$$
is in $V_N$, so it vanishes near zero. 
So there exists a constant $\epsilon(t_0)>0$ such that 
$$
\phi(tu) - |t|^{1/2}\chi_1(t)\phi(u) = 0
$$
for $t = t_0$ and $|u|\leq \epsilon(t_0)$. 
By smoothness of $\pi$ and $\chi$, it also holds when $t$ is near $t_0$. 
By compactness of $\varpi \calO_F^{\times}$, there exists uniform $\epsilon >0$ such that the above equation is true for $t\in \varpi \calO_{F}^{\times}$ and $|u|\leq \epsilon$. 
By factoring $0\neq t\in \frap$ as a product of elements of $\varpi \calO_{F}^{\times}$, we get the same equation for $0\neq t\in \frap$ and $|u|\leq \epsilon$, which proves the claim. 
\end{proof}
\begin{theorem}
\label{pskr}
Let $(\pi, V) = \pi(\chi_1, \chi_2)$ be an irreducible principal series representation. 
\begin{enumerate}
\item If $\chi_1 \neq \chi_2$, the space of Kirillov model of $V$ consists of the functions $\phi$ on $F^{\times}$ that are smooth, vanish for large $t$ and 
$$
\phi(t) = C_{1}|t|^{1/2}\chi_{1}(t) + C_{2}|t|^{1/2}\chi_2(t)
$$
for small $t$, where $C_1, C_2$ are constants. 
\item If $\chi_1 = \chi_2$, the space of Kirillov model of $V$ consists of the functions $\phi$ on $F^{\times}$ that are smooth, vanish for large $t$ and 
$$
\phi(t) = C_{1}|t|^{1/2} \chi_{1}(t) + C_{2}v(t)|t|^{1/2}\chi_{1}(t)
$$
for small $t$, where $C_1, C_2$ are constants and $v:F^{\times} \to \Cc$ is the valuation map. 
\end{enumerate}
\end{theorem}
\begin{proof}
This follows from Theorem \ref{psjac} and the previous proposition. For details, see p.515 of \cite{bu}. 
\end{proof}
\begin{theorem}
\label{spkr}
Let $(\pi, V) = \sigma(\chi_1, \chi_2)$ be a special representation, where $(\chi_1\chi_2^{-1})(t) = |t|^{-1}$.  Then the space of Kirillov model of $V$ consists of the functions $\phi$ on $F^{\times}$ that are smooth, vanish for large $t$ and $$\phi(t) = C|t|^{1/2}\chi_{2}(t)$$ for small $t$, where $C$ is a constant. 
\end{theorem}
\begin{proof}
The proof is almost same as the proof of Theorem \ref{pskr}. Note that $T(F)$ acts as $\delta^{1/2}\chi$ on the Jacquet module of $\sigma(\chi_1, \chi_2)$. 
\end{proof}

Now we can define local $L$-function $L(s, \pi, \xi)$ for given $(\pi, V)$ and a quasicharacter $\xi:F^\times \to\Cc^{\times}$, and local zeta functions $Z(s, \phi, \xi)$ for $\phi\in V$ (identified with the Kirillov model). 
The above theorems about asymptotes of Kirillov models will allow us to define local zeta functions. 
\begin{definition}
Let $(\pi, V)$ be an irreducible admissible representation of $GL(2, F)$ that admits a Whittaker model.
Define the local $L$-function $L(s, \pi)$ associated with $(\pi, V)$ as
$$
L(s, \pi) = \begin{cases} (1-\alpha_1 q^{-s})^{-1}(1-\alpha_2 q^{-s})^{-1} & \text{spherical principal series}, \alpha_i = \chi_i(\varpi) \\
(1-\alpha_{2} q^{-s})^{-1} & \text{special representation}, (\chi_{1}\chi_{2}^{-1})(y) = |y|^{-1} \\
1 &  \text{otherwise.}\end{cases}
$$
Also, for a quasicharacter $\xi:F^{\times} \to \Cc^{\times}$, define $L(s, \pi, \xi):= L(s, \pi\otimes \xi)$. 
\end{definition}

\begin{proposition}
Let $(\pi, V)$ be an irreducible admissible representation of $\GL(2, F)$ that admits a Whittaker model. 
If $\phi$ is an element of the space of the Kirillov model of $\pi$, consider the integral 
$$
Z(s, \phi, \xi) = \int\dpl{F^{\times}} \phi(y)\xi(y)|y|^{s-1/2} d^{\times}y, 
$$
where $d^{\times}y$ denotes the normalized Haar measure on $F^{\times}$. 
This integral is convergent for sufficiently large $\Re s$ and has meromorphic continuation to all $s$. 
More precisely, $Z(s, \phi, \xi) = p_{\phi}(s^{-1})L(s, \pi, \xi)$, where $p_{\phi}$ is a rational function. 
Moreover, $\phi$ can be chosen so that $p_{\phi} = 1$. 
\end{proposition}
\begin{proof}
We will only show the case where $(\pi, V)$ is a spherical principal series representation and $\xi$ is unramified. 
By Theorem \ref{pskr}, we can assume that $\phi(y) = 0$ for $|y| > q^N$ and  $\phi(y) = C_{1}|y|^{1/2}\chi_1(y) + C_{2}|y|^{1/2}\chi_2(y)$ for $|y|\leq q^{-N'}$. 
Then the integral can be written as
\begin{align*}
Z(s, \phi, \xi) &= \sum_{m\in \Zz} \int_{|y| = q^{m}} \phi(y)\xi(y) |y|^{s-1/2} d^{\times}y \\
&= \sum_{m=-N}^{N' - 1} q^{-m(s-1/2)} \int\dpl{|y| = q^{-m}} \phi(y)\xi(y)d^{\times}y \\
&+ \sum_{m\geq N'} q^{-m(s-1/2)}\int\dpl{|y| = q^{-m}} (C_{1}|y|^{1/2}\chi_1(y) + C_{2}|y|^{1/2} \chi_2(y)) \xi(y) d^{\times}y \\
&=r_{\phi}(q^{-s}) + \sum_{m\geq N'} [C_{1} (\alpha_1 q^{-s})^{m}  + C_{2}(\alpha_{2} q^{-s})^{m}]\int\dpl{|y| = q^{-m}}\xi(y)d^{\times}y \\
&= r_{\phi}(q^{-s}) + \sum_{m\geq N'} [C_{1} (\alpha_1 \xi(\varpi) q^{-s})^{m}  + C_{2}(\alpha_{2} \xi(\varpi)q^{-s})^{m}] \\
&= r_{\phi}(q^{-s}) + \frac{C_1 \alpha_{1}^{N'} q^{-N's}}{1-\alpha_{1}\xi(\varpi)q^{-s}} + \frac{C_2 \alpha_{2}^{N'} q^{-N's}}{1-\alpha_{2}\xi(\varpi)q^{-s}} \\
&=p_{\phi}(q^{-s}) L(s, \pi, \xi)
\end{align*}
where $p_{\phi}$ is a rational function, $C' = \int_{|y| = 1} \xi(y)d^{\times} y$ and $L(s, \pi, \xi) = L(s, \pi\otimes \xi)$ with $\pi\otimes \xi \simeq \pi(\xi \chi_1, \xi \chi_2)$. 
Now, define $\phi :F^{\times} \to \Cc$ as
$$
\phi(y) = \begin{cases} \frac{\alpha_{1}}{\alpha_{1} - \alpha_{2}} |y|^{1/2} (\xi\chi_{1})(y) - \frac{\alpha_{2}}{\alpha_{1} - \alpha_{2}} |y|^{1/2} (\xi\chi_{2})(y) & y\in \calO_F \\ 0 & \text{otherwise}\end{cases}
$$
Then this function is in the Kirillov model of $\pi(\xi \chi_1, \xi\chi_2)$, and we can check $Z(s, \phi, \xi) = L(s, \pi, \xi)$ by the above computation. 
\end{proof}

The next theorem gives us local functional equations of local zeta integrals.
\begin{theorem}[Local functional equation]
\label{localfe}
Let $(\pi, V)$ be an irreducible admissible representation of $\GL(2, F)$ with central quasicharacter $\omega$ that admits a Whittaker model, and let $\xi$ be a quasicharacter of $F^{\times}$. 
Identify $V$ with the space of Kirillov model. 
There exists a meromorphic function $\gamma(s, \pi, \xi, \psi)$ such that 
$$
Z(1-s, \pi(w_1)\phi, \omega^{-1}\xi^{-1}) = \gamma(s, \pi, \xi, \psi) Z(s, \phi, \xi), \quad w_1 = \pmat{}{1}{-1}{}
$$
for all $\phi\in V$. 
\end{theorem}
\begin{proof}
For fixed $s$, define two linear functionals $L_1, L_2$ on $V$ by 
$$
L_1(\phi) = Z(s, \phi, \xi), \qquad L_2(\phi) = Z(1-s, \pi(w_1)\phi, \omega^{-1}\xi^{-1}).
$$
We can check that both linear functionals satisfy 
$$
L\left( \pi\pmat{y}{}{}{1} \phi\right) = \xi(y)^{-1}|y|^{-s+1/2} L(\phi)
$$
by using change of variables and analytic continuations. 
Using Proposition \ref{distuniq}, one can show that $L_{1}$ and $L_{2}$ are linearly dependent when restricted to $V_N$, so $c_1 L_1 + c_2 L_2$ factors through $J(V)$ for some $c_1, c_2\in \Cc$ that not both zero. 
This implies that $c_1 L_1 + c_2 L_2 = 0$ for all but two possible choices of $s$ in $\Cc$ modulo $2\pi i / \log (q)$, and meromorphic continuation proves that they are proportional for all $s$, i.e. there exists a meromorphic function $\gamma(s, \pi, \xi, \psi)$ satisfies $L_{1} = \gamma L_{2}$. 
\end{proof}

We call the meromorphic function $\gamma(s, \pi, \xi, \psi)$ (which does not depend on the choice of $\phi$) as a gamma factor. 
The next proposition shows that the gamma factors $\gamma(s, \pi, \xi, \psi)$ determine the representation $\pi$. 
\begin{proposition}
Let $\pi_1, \pi_2$ be irreducible admissible representations of $\GL(2, F)$. 
Suppose that $\pi_1, \pi_2$ have the same central quasicharacter $\omega$ and that $\gamma(s, \pi_1, \xi, \psi) = \gamma(s, \pi_2, \xi, \psi)$ for all quasicharacters $\xi$ of $F^{\times}$. 
Then $\pi_1\simeq \pi_2$. 
\end{proposition}
\begin{proof}
Let's identify $\pi_1, \pi_2$ with their Kirillov models, so $V_1, V_2$ are subspaces of $C^{\infty}(F^{\times})$ with $\GL(2, F)$-actions. 
Let $V_0 = V_1 \cap V_2$. 
It is enough to show that $\pi_1(w_1)\phi = \pi_2(w_1)\phi$ for $\phi\in V_0$, because this implies $V_1, V_2$ are nonzero irreducible $\GL(2, F)$-spaces so we get $V_1= V_0 = V_2$. (Note that $B(F)$ and $w_1$ generate $\GL(2, F)$.) 
Also, if we put $\phi_i = \pi_i(w_1)\phi$, then it is sufficient to show that $\phi_1(1) = \phi_2(1)$ by considering the action of $\smat{a}{}{}{1}$ for $a\in F^{\times}$. 
To show this, we use Fourier inversion formula: if $M$ is compact abelian group with normalized Haar measure (so that $|M| = 1$), and if $F$ is a continuous function on $M$, then 
$$
F(1) = \sum_{\chi\in \hat{M}} \,\,\int\dpl{M}F(m)\chi(m)dm.
$$
Now for $n\in \Zz$, let
$$
F_{\xi}(n) = \int\dpl{|y| = q^{-n}} (\phi_1(y) - \phi_2(y))\xi(y) d^{\times}y.
$$
Then $F_{\xi}(0)$ depends only on the restriction of $\xi$ on $\calO_{F}^{\times}$, and $F_{\xi}(0) = 0$ for all but finitely many characters $\xi$ of $\calO_{F}^{\times}$ and 
$$
\phi_{1}(1) - \phi_{2}(1) = \sum_{\xi\in \wh{\calO_{F}^{\times}}} F_{\xi}(0)
$$
by Fourier inversion formula applied to $M =\calO_{F}^{\times}$ and $F(y) = \phi_{1}(y) - \phi_{2}(y)$. 
By hypothesis and the functional equations of local zeta functions, we have $Z(s, \phi_1, \xi) = Z(s, \phi_2, \xi)$ for all characters $\xi$ of $F^{\times}$. 
Also, since $\phi_i(y) = 0$ for sufficiently large $|y|$, $F_{\xi}(n) =0$ for sufficiently small $n$. If we put $x = q^{-s +1/2}$, then 
$$
\sum_{n\in \Zz} F_{\xi}(n) x^{n} = Z(s, \phi_1, \xi) - Z(s, \phi_2, \xi) = 0
$$
for sufficiently small $x$ (i.e. sufficiently large $\Re s$), so $F_{\xi}(n) = 0$ for all $n$, and in particular, $F_{\xi}(0) = 0$. 
\end{proof}

There's a simple relation between this (local) gamma factor for  $\GL(2, F)$ and $\GL(1, F)$, i.e. Tate gamma factors.
\begin{theorem}[Jacquet-Langlands]
Let $\chi_1, \chi_2$ be quasicharacters of $F^{\times}$ such that $(\pi, V) = \calB(\chi_1, \chi_2)$ is irreducible. Then 
$$
\gamma(s, \pi, \xi, \psi) = \gamma(s, \xi\chi_1, \psi)\gamma(s, \xi\chi_2, \psi)
$$
where the Tate gamma factors $\gamma(s, \xi\chi_i, \psi)$ are defined in Theorem \ref{tate}. 
\end{theorem}
\begin{proof}
The proof uses Whittaker model of $\calB(\chi_1, \chi_2)$ constructed in section 3.6 using the Weil representation. 
Indeed, we defined another (and simpler) Whittaker model in the proof of Proposition \ref{intcomp}. 
However, the one constructed by means of Weil representation is much more helpful to prove this. 
It allows us to express local zeta integral of the principal series representation as a product of two local zeta integrals corresponds to two quasicharacters $\xi\chi_1$ and $\xi\chi_2$ naturally, and the result directly follows from this. For details, see p. 548 of \cite{bu}.
\end{proof}

\newpage

\section{Global theory}

Using local theories, now we can define $\GL(2)$-automorphic forms. We will \emph{glue} local theories and interpret things in ad\'elic language. Also, we will see how to interpret the classical modular forms and Maass forms in this way. Before we start, we will study $\GL(1)$-theory first, which is developed by Tate in his celebrated thesis. His thesis shows how powerful ad\'elic languages are, and why this is the right way to study global things. 

After that, we define the notion of automorphic forms and representations for $\GL(2)$, and define $L$-functions attached to automorphic representations for $\GL(2)$, by generalizing Tate's idea. 
Here we need Flath's decomposition theorem and multiplicity one theorem. 

\subsection{Tate's thesis}

Tate's thesis is a theory of $\GL(1)$-automorphic forms over a global field. In 1950s, Riemann proved that his famous Riemann zeta function has an analytic continuation and a functional equation, by using the theta function. Tate \emph{re}-proved this fact, but in a completely different way.
Tate's idea is the following:
\begin{enumerate}
\item Develop Fourier theory on ad\'eles $\Aa$, including Fourier transform and Fourier inversion formula.
\item Define ad\'elic version of Hecke $L$-functions and local \& global zeta integrals. Prove functional equation for these zeta integrals. 
\item Show that the local zeta integrals coincides with the local $L$-functions for all but finitely many places. 
\item Derive analytic continuation and functional equation for Hecke $L$-functions  from corresponding local statements. Also, Euler product becomes simply a factorization of global integral according to the product structure of $\Aa^{\times}$. 
\end{enumerate}
This gives a natural way to get the global result from local results, and we will develop $\GL(2)$-theory via similar way. 


Let $F$ be a global field (number field or function field over a finite field), and let $\Aa = \Aa_{F}$ be its ad\'ele ring, i.e. the restricted product
$$
\Aa = \sideset{}{'}\prod_{v} F_{v} = \{ (a_{v}) \in \prod_{v} F_{v}\,:\, a_{v} \in \calO_{v}\text{ for all but finitely many }v\}
$$
where $v$ runs over the set of places of $F$ and $F_{v}$ is a completion of $F$ with respect to $v$. For non-archimedean $v$, $\calO_{v}$ is a ring of integer of $F_{v}$. 
This is a locally compact abelian group and we have Haar measure on it, which is both left and right invariant. 
$F$ can be embedded into $\Aa$ diagonally, and the quotient $\Aa/F$ is compact. It is known that we can always find a nontrivial additive character $\psi = \prod_{v}\psi_{v}$ on $\Aa$ that is trivial on $F$. 
Also, any continuous character of $\Aa$ has the form $\psi_{a}(x) = \psi(ax)$ for some $a\in \Aa$, and $a\mapsto \psi_{a}$ gives an isomorphism $\Aa \simeq \Aa^{*}$. 
Let $\Aa_{\fin} = \sideset{}{'}\prod_{v<\infty} F_{v}$ be a ring of finite ad\'eles, which is a restricted product of $F_{v}$'s for non-archimedean $v$. 

Now we want to define Fourier transform  as 
$$
\wh{f}(x) = \int_{\Aa}f(y)\psi(xy)dy,
$$
where $dy$ is a Haar measure on $\Aa$. However, there are two problems with this definition. 

First, the integral does not converge for some $f\in L^{2}(\Aa)$. To fix this, we consider smaller but dense subspace, which is the space of Schwartz functions. 

\begin{definition}[Schwartz function on $\Aa$]
Let $F$ be a local field. If $F = \Rr$, then a $\Cc$-valued function $f:\Rr^{n} \to \Cc$ is a Schwartz function if 
$$
|f|_{\alpha, \beta} = \sup_{x\in \Rr^{n}} |x_{1}^{\alpha_{1}}\cdots x_{n}^{\alpha_{n}}| \left|\frac{\partial^{\beta_{1} + \cdots + \beta_{n}} f}{\partial x_{1}^{\beta_{1}} \cdots \partial x_{n}^{\beta_{n}}}(x)\right|
$$
is bounded for all $\alpha_{i}, \beta_{i}\in \Zz_{\geq 0}$. 
We topologize the space of Schwartz functions $\calS(\Rr^{n})$ by giving it the smallest topology in which all the seminorms $|\,|_{\alpha, \beta}$ are continuous. 
If $F = \Cc$, we regard $\Cc^{n} = \Rr^{2n}$ and define Schwartz space of $\Cc$ similarly. 

For non-archimedean $F$, define the Schwartz space $\calS(F^{n})$ of Schwartz functions on $F^{n}$ as the space of compactly supported smooth (i.e. locally constant) functions, and give the weakest topology in which every linear functional is continuous. 
In fact, we can ignore the topology for non-archimedean case. 

Now define $\calS(\Aa)$ as the space of all finite linear combinations of the form 
$$
\Phi(x) = \prod_{v} \Phi_{v}(x_{v}), \quad x = (x_{v}) \in \Aa
$$
where each $\Phi_{v}\in \calS(F_{v})$ and $\Phi_{v} = \chf_{\calO_{v}}$ for all but finitely many $v$. 
\end{definition}
For Schwartz functions, the integral absolutely converges and the problem is resolved. 
However, there's one more problem. 
We want that the Fourier inversion formula holds, so that $\doublehat{f}(x) = f(-x)$ for all $x\in\Aa$. 
To do this, Haar measures on $\Aa$ and its dual $\Aa^{*}$ should be compatible in some sense. 
We saw that $\Aa^{*} \simeq \Aa$ by fixing a nontrivial additive character $\psi:\Aa/F \to \Cc^{\times}$, and this gives a unique normalization of the Haar measure for which the Fourier inversion formula holds. 
Such measure is called \emph{self-dual} Haar measure. 
If $dx_{v}$'s are local self-dual measures for each place $v$, then $dx = \prod_{v} dx_{v}$ is a self-dual measure of $\Aa$, and same thing holds for $d^{\times}x = \prod_{v} d^{\times}x_{v}$ on $\Aa^{\times}$. 

We can also think Dirichlet characters in ad\'elic setting. Such characters are called Hecke characters.
\begin{definition}[Hecke character]
A Hecke character $\chi$ is a continuous character of $\Aa^{\times}/F^{\times}$. 
We can write it as $\chi = \prod_{v}\chi_{v}$ where $\chi_{v} =  \chi \circ i_{v}: F_{v}^{\times} \to \Cc^{\times}$ where $i_{v} : F_{v}^{\times} \hookrightarrow \Aa^{\times}$. 
\end{definition}
First, for all but finitely many $v$, local components $\chi_{v}$ are \emph{unramified}:
\begin{proposition}
Let $F$ be a global field and $\Aa = \Aa_F$. Let $\chi:\Aa^{\times}\to \Cc^{\times}$ be a continuous character. 
Then there exists a finite set $S$ of places, including all archimedean ones, such that $\chi_{v}|_{\calO_{v}^{\times}} = 1$ if $v\not\in S$. Such $\chi_{v}$ is called unramified.
\end{proposition}
\begin{proof}
By no small subgroup argument (Proposition \ref{nss}), $\ker (\chi|_{\Aa_{\fin}^{\times}})$ contains an open neighborhood of the identity. 
\end{proof}

Now the following proposition shows that finite order Hecke characters and Dirichlet characters are just same things, at least for $F = \Qq$. 
\begin{proposition}
\label{dirad}
\begin{enumerate}
\item Let $F = \Qq$ and $\chi:\Aa^{\times}/F^{\times} \to \Cc^{\times}$ be a character. There exists a unique character $\chi_{1}$ of finite order of $\Aa^{\times}/F^{\times}$ and a unique purely imaginary number $\lambda$ such that $\chi(x) = \chi_{1}(x) |x|^{\lambda}$. 
\item Let $F = \Qq$ and $\chi$ be a character of finite order of $\Aa^{\times}/F^{\times}$. 
There exists an integer $N$ whose prime divisors are precisely the primes $p_{v}$ such that $v$ is a non-archimedean place of $\Qq$ and $\chi_{v}$ is ramified, and a primitive Dirichlet character $\chi_{0}$ modulo $N$ such that if $v$ is a non-archimdean place such that $p_{v}\nmid N$, then $\chi_{0}(p_{v}) = \chi(p_{v})$. 
This correspondence $\chi\mapsto \chi_{0}$ is a bijection between the characters of finite order of $\Aa^{\times}/F^{\times}$ and the primitive Dirichlet characters. 
\end{enumerate}
\end{proposition}
\begin{proof}
Let $N$ be a positive integer, and let 
\begin{align*}
S_{0}(N) &= \{v\,:\, v\text{ is non-archimedean},\, p_{v}|N\} \\
S_{1}(N) &= \{v\,:\, v\text{ is non-archimedean},\, p_{v}\nmid N\}.
\end{align*}
For each $v\in S_{0}(N)$, let 
$$
U_{v}(N) = \{x\in\calO_{v}\,:\, x\equiv 1\Mod{N}\}
$$
and
$$
U_{\fin}(N) = \prod_{v\in S_{0}(N)} U_{v}(N) \times \prod_{v\in S_{1}(N)} \calO_{v}^{\times}, \quad U(N) = \Rr_{+}^{\times} \times U_{\fin}(N).
$$
Then $U_{\fin}(N)$ form a basis of neighborhoods of the identity in $\Aa_{\fin}^{\times}$, so there exists $N$ such that $\chi|_{U_{\fin}(N)}=1$ by no small subgroup argument. 
The restriction $\chi|_{\Rr_{+}^{\times}}$ is of the form $|x|^{\lambda}$ for some unique $\lambda\in i\Rr$, so $\chi_{1}(x):= \chi(x)|x|^{-\lambda}$ is trivial on $U(N)$. 
If we put 
$$
V(N) = \Rr_{+}^{\times} \times \prod_{v\in S_{0}(N)} U_{v}(N) \times\resp_{v\in S_{1}(N)} \Qq_{v}^{\times},
$$
then this is an open subgroup and $\Aa^{\times} = \Qq^{\times}V(N)$ by the approximation theorem. 
Hence $\Aa^{\times}/\Qq^{\times} \simeq V(N) /(\Qq^{\times}\cap V(N))$ and it is enough to show that $\chi_{1}|_{V(N)}$ has finite order. Since $\chi_{1}$ is trivial on $U(N)(\Qq^{\times}\cap V(N))$, it is enough to show $[V(N):U(N)(\Qq^{\times} \cap V(N))] <\infty$. 
In fact, we have
$$
V(N)/U(N)(\Qq^{\times}\cap V(N)) \simeq I_{N}/P_{N} \simeq (\Zz/N\Zz)^{\times}
$$
where $I_{N}$ is a group of all fractional ideals of $\Qq$ prime to $N$, and $P_{N}$ is the subgroup of principal fractional ideals $\alpha\Zz$ with $\alpha\in \Qq^{\times} \cap V(N)$. 

2  follows from composing with the above isomorphism. Note that we have to take minimal $N$ to make the corresponding Dirichlet character primitive. 
\end{proof}
This also holds for general global fields. 
This proposition will be used later to show that the classicial Dirichlet $L$-function (or Hecke $L$-function for general number fields) is same as the ad\'elic version of it. 

\begin{definition}
Let $S$ be a finite set of places containing archimedean places so that $\chi_{v}$ is unramified for all $v\not\in S$. For $v\not\in S$, we define the local $L$-function $L_{v}(s, \chi_{v})$ as 
$$
L_{v}(s, \chi_v) = (1-\chi(\frap_{v})q_{v}^{-s})^{-1}
$$
and the partial $L$-function as
$$
L_{S}(s, \chi) = \prod_{v\not\in S} L_{v}(s, \chi_{v}). 
$$
\end{definition}
If $\chi(x) = \chi_{1}(x)|x|^{\lambda}$, then $L_{S}(s, \chi) = L_{S}(s + \lambda, \chi_{1})$ and so we can assume that $\chi$ is of finite order. We will define $L_{v}(s, \chi_{v})$ for $v\in S$ later. 


\begin{proposition}[Poisson summation formula]
\begin{enumerate}
\item The volume of $\Aa/F$ is 1 with respect to the self-dual Haar measure on $\Aa$. 
\item Let $\Phi$ be a Schwartz function on $\Aa$ and let 
$$
\wh{\Phi}(x) = \int_{\Aa} \Phi(y) \psi(xy)dy
$$
be its Fourier transform. Then 
$$
\sum_{\alpha\in F} \Phi(\alpha t) = \frac{1}{|t|} \sum_{\alpha\in F} \wh{\Phi}\left(\frac{\alpha}{t}\right)
$$
for any $t\in \Aa^{\times}$. 
\end{enumerate}
\end{proposition}
\begin{proof}
For $t\in \Aa^{\times}$, define 
$$
F(x) = \sum_{\alpha\in F} \Phi((x+\alpha) t). 
$$
This is a continuous function on the compact abelian group $\Aa/F$ and has a Fourier expansion
$$
F(x) = \sum_{\beta\in F} c_{\beta} \psi(-\beta x).
$$
By orthogonality of characters, coefficients can be computed by 
\begin{align*}
c_{\beta} &= \frac{1}{V} \int_{\Aa/F} F(x) \psi(\beta x)dx \\
&= \frac{1}{V} \int_{\Aa/F} \sum_{\alpha\in F} \Phi((x+\alpha)t) \psi(\beta(x+\alpha)) dx \\
&= \frac{1}{V} \int_{\Aa} \Phi(xt)\psi(\beta x)dx \\
&=\frac{1}{V|t|} \int_{\Aa} \Phi(x)\psi(\beta x/t) dx = \frac{1}{V|t|} \wh{\Phi}\left(\frac{\beta}{t}\right).
\end{align*}
Here $V$ is the volume of $\Aa/F$ and we use the substitution $x \to x/t$ in the last equality. 
Now put $x =0$ and we get 
$$
\sum_{\alpha\in F} \Phi(\alpha t) = F(0) = \sum_{\beta\in F} c_{\beta}= \frac{1}{V|t|} \sum_{\beta\in F}\wh{\Phi}\left(\frac{\beta}{t}\right).
$$
If we apply this twice and put $t = 1$, then 
$$
\sum_{\alpha\in F} \Phi(\alpha) = \frac{1}{V^{2}} \sum_{\alpha\in F}\Phi(-\alpha)
$$
and this implies $V = 1$. 
\end{proof}



\begin{definition}[Zeta integral]
For $\Phi\in \calS(\Aa)$ and a Hecke character $\chi:\Aa^{\times}/F^{\times} \to \Cc^{\times}$, define the zeta integral as
$$
\zeta(s, \chi, \Phi) = \int_{\Aa^{\times}} \Phi(x) \chi(x) |x|^{s} d^{\times}x. 
$$
If $\Phi = \prod_{v}\Phi_{v}$, then this integral factorizes formally as
$$
\zeta(s, \chi, \Phi) = \prod_{v} \zeta_{v}(s, \chi_{v}, \Phi_{v})
$$
where 
$$
\zeta_{v}(s, \chi_{v}, \Phi_{v}) = \int_{F_{v}^{\times}} \Phi_{v}(x)\chi_{v}(x)|x|_{v}^{s}d^{\times}x.
$$
The last integrals $\zeta_{v}(s, \chi_{v}, \Phi_{v})$ are called local zeta integrals. 
\end{definition}
We will show that the above factorization of zeta integral makes sense, i.e. it converges for $\Re s >1$. 
Also, we will show functional equations of local zeta integrals, which automatically gives the functional equation for the global zeta integral. 
\begin{proposition}
\begin{enumerate}
\item The local integrals are convergent if $\Re s >0$. 
\item There exists a finite set $S$ of places containing the archimedean ones such that 
$$
\zeta_{v}(s, \chi_{v}, \Phi_{v}) = (1-\chi(\mathfrak{p}_{v}) q_{v}^{-s})^{-1}
$$
for all $v\not\in S$. 
Indeed, it is sufficient to choose $S$ so that if $v\not\in S$, then $\chi_{v}$ is unramified and $\Phi_{v}$ is the characteristic function of $\calO_{v}$. 
\item The global integral integral is absolutely convergent for $\Re s >1$, in which case the decomposition is valid. 
\end{enumerate}
\end{proposition}
\begin{proof}
First, we will show that the integral absolutely converges for $\Re s > 0$. 
Since $\chi_{v}$ is unitary, the integral is bounded by 
$$
\int_{F_{v}^{\times}} |\Phi_{v}(x)| |x|_{v}^{s} d^{\times}x_{v}  = \int_{|x|_{v} \leq 1}|\Phi_{v}(x)| |x|_{v}^{s} d^{\times}x_{v} + \int_{|x|_{v} > 1} |\Phi_{v}(x)| |x|_{v}^{s} d^{\times}x_{v}
$$
For the above two integrals on RHS, second integral absolutely converges because of rapid decay of $\Phi_{v}(x)$ (the function is in the Schwartz space). For the first integral corresponds to the region $|x|_{v} \leq 1$, $|\Phi_{v}(x)|$ is bounded because $|x|_{v} \leq 1$ is compact. So we can ignore $\Phi_{v}(x)$ and the remaining term decomposes as
$$
\sum_{k\geq 0} \int_{\ord_{v}(x) = k} |x|_{v}^{s} d^{\times}x_{v} = \sum_{k\geq 0} q_{v}^{-ks}
$$
when $v$ is non-archimedean, and the summation converges for $\Re s >0$. Real and complex case comes from the convergence of the integrals
$$
\int_{-1}^{1} |t|^{s} \frac{dt}{t}, \quad \frac{1}{2\pi} \int_{0}^{2\pi} \int_{0}^{1} r^{2s} \frac{dr}{r} d\theta
$$
for $\Re s >0$. 

Now let $S$ be a finite set of primes so that for all $v\not\in S$, $v$ is non-archimedean, $\chi_{v}$ is unramified and $\Phi_{v} = \chf_{\calO_{v}}$. Then 
\begin{align*}
\zeta_{v}(s, \chi_{v}, \Phi_{v}) &= \int_{F_{v}^{\times}} \chf_{\calO_{v}}(x) \chi_{v}(x) |x|_{v}^{s} d^{\times}x_{v} \\
&= \sum_{k\geq 0} \int_{\ord_{v}(x) = k} \chi_{v}(\frap_{v})^{k} q_{v}^{-ks} d^{\times}x_{v} \\
&= \sum_{k\geq 0} (\chi_{v}(\frap_{v})q_{v}^{-s})^{k} = (1-\chi_{v}(\frap_{v})q_{v}^{-s})^{-1}
\end{align*}
Now the product 
$$
\prod_{v} \zeta_{v}(s, \chi_{v}, \Phi_{v})
$$
absolutely converges for $\Re s > 1$ because the local zeta integrals agrees with the local factors of corresponding Hecke $L$-function. 
\end{proof}


\begin{theorem}[Tate]
\label{tate}
\begin{enumerate}
\item The local integral has meromorphic continuation to all $s\in\Cc$, with no poles in the region $\Re s > 0$. 
\item There exists a meromorphic function $\gamma_{v}(s, \chi_{v}, \psi_{v})$, independent of the test function $\Phi_{v}$ such that 
$$
\zeta_{v}(1-s, \chi_{v}^{-1}, \wh{\Phi}_{v}) = \gamma_{v}(s, \chi_{v}, \psi_{v}) \zeta_{v}(s, \chi_{v}, \Phi_{v}).
$$
\item Let $s_{0}\in \Cc$ be given. We can choose $\Phi_{v}$ so that $\zeta_{v}(s, \chi_{v}, \Phi_{v})$ has neither zero nor pole at $s = s_{0}$. If $v$ is non-archimedean, we may even choose the local data so that $\zeta_{v}(s, \chi_{v}, \Phi_{v})$ is identically equal to 1. 
\end{enumerate}
\end{theorem}
\begin{proof}
First, we show that $\gamma$ doesn't depend on the test function when $0<\Re s < 1$. In other words, we must prove that 
$$
\zeta_{v}(1-s, \chi_{v}^{-1}, \wh{\Phi}_{v}) \zeta_{v}(s, \chi_{v}, \Phi_{v}') = \zeta_{v}(1-s, \chi_{v}^{-1}, \wh{\Phi_{v}'}) \zeta_{v}(s, \chi_{v}, \Phi_{v}).
$$
The LHS equals to 
\begin{align*}
&\int_{F_{v}^{\times}}\left[ \int_{F_{v}} \Phi_{v}(y) \psi_v(xy)dy\right]\chi_{v}(x)^{-1}|x|_{v}^{1-s} d^{\times}x \int_{F_{v}^{\times}} \Phi_{v}'(z)\chi_{v}(z) |z|_{v}^{s} d^{\times}z \\
&= \int_{F_{v}^{\times}}\int_{F_{v}}\int_{F_{v}^{\times}} \Phi_{v}(y)\Phi'_{v}(z)\psi_v(xy)\chi_{v}(x)^{-1}\chi_{v}(z) |x|_{v}^{1-s}|z|_{v}^{s} d^{\times}x dyd^{\times}z \\
&= \frac{1}{m_{v}} \int_{F_{v}^{\times}}\int_{F_{v}^{\times}}\int_{F_{v}^{\times}} \Phi_{v}(y)\Phi'_{v}(z) \chi_{v}(yz) |yz|_{v}^{s} \psi_v(x) \chi_{v}(x)^{-1} |x|_{v}^{1-s}d^{\times}x d^{\times}y d^{\times} z
\end{align*}
where the last equality follows from the substitution $x\to y^{-1}x$ for $y\neq 0$. Here $m_{v}$ is a constant that satisfies $d^{\times}y = m_{v}dy / |y|_{v}$ between nomarlized additive Haar measure and normalized multiplicative Haar measure.  Clearly, this is symmetric in $\Phi_{v}$ and $\Phi_{v}'$, so the equation holds. 

To extend this for all $s$, choose $\Phi_{v}$ so that $\wh{\Phi}_{v}$  vanishes in a neighborhood of zero. Then $\zeta_{v}(1-s,\chi_{v}^{-1}, \wh{\Phi}_{v})$ is convergent for all $s$ and we already know that $\zeta_{v}(s, \chi_{v}, \Phi_{v})$ is convergent and holomorphic for $\Re s >0$. 
This implies that their quotient $\gamma_{v}(s, \chi_{v}, \psi_{v})$ has a meromorphic continuation on $\Re s> 0$, and we get a similar result for $\Re s <1$ by choosing nice $\Phi_{v}$ that vanishes near zero. This also proves 1 together with the previous proposition. 

For 3, choose $\Phi_{v}$ as compactly supported near $x = 1$, so that integral converges for any $s\in \Cc$ and $\chi_{v}(x) |x|^{s_{0}}_{v}$ has positive real part on the support of $\Phi_{v}$, so is nonzero. For non-archimedean $v$, choose $\Phi_{v}$ so that the support of $\Phi_{v}$ is in $\calO_{v}^{\times}$, then $\zeta_{v}(s, \chi_{v}, \Phi_{v})$ is independent of $s$ and we can normalize it as 1. 
\end{proof}

\begin{proposition}
The function $\zeta(s, \chi, \Phi)$ has meromorphic continuation and its entire unless the restriction of $\chi$ to the subgroup $\Aa_{1}^{\times} = \{x\in \Aa\,:\, |x| = 1\}$ is trivial. In this case, $\chi(x) = |x|^{\lambda}$ for some $\lambda\in i\Rr$ and $\zeta(s, \chi, \Phi)$ can have pole at $s = 1-\lambda$ and $\lambda$. Also, it satisfies the functional equation 
$$
\zeta(s, \chi, \Phi) = \zeta(1-s, \chi^{-1}, \wh{\Phi}).
$$
\end{proposition}
\begin{proof}
The main idea is to split the integral into two parts, $|x| <1$ and $|x|>1$, and using the Poisson summation formula to unfold and refold integral. Let
\begin{align*}
\zeta_{1}(s, \chi, \Phi) &:= \int\displaylimits_{\substack{\Aa^{\times} \\ |x| > 1}} \Phi(x)\chi(x)|x|^{s} d^{\times} x \\
\zeta_{0}(s, \chi, \Phi) & := \int\displaylimits_{\substack{\Aa^{\times} \\ |x| < 1 }} \Phi(x) \chi(x) |x|^{s} d^{\times}x
\end{align*}
We already know that the first integral converges for all $s$, since it converges for $\Re s > 1$ by the previous results and decreasing $\Re s$ only improves the convergence in the region $|x| >1$. For the second one, we can unfold the integral as
\begin{align*}
\zeta_{0}(s, \chi, \Phi) &= \sum_{\alpha\in F^{\times}} \int\dpl{\substack{\Aa^{\times}/F^{\times} \\ |x| < 1}} \Phi(\alpha x)\chi(\alpha x) |\alpha x|^{s} d^{\times}x \\
&= \int\dpl{\substack{\Aa^{\times}/F^{\times} \\ |x| < 1}} \left[\sum_{\alpha\in F^{\times}} \Phi(\alpha x)\right] \chi(x) |x|^{s} d^{\times}x \\
&= \int\dpl{\substack{\Aa^{\times}/F^{\times} \\ |x| < 1}} \left[\sum_{\alpha\in F} \Phi(\alpha x)\right] \chi(x) |x|^{s} d^{\times}x - \Phi(0) \int\dpl{\substack{\Aa^{\times}/F^{\times} \\ |x|<1}} \chi(x)|x|^{s}d^{\times}x
\end{align*}
since $\chi(\alpha) = 1 = |\alpha|$ for all $\alpha\in F$. The last integral can be written as
$$
\int\limits_{0}^{1}\int\dpl{\substack {\Aa^{\times}/F^{\times} \\ |x| = t}} \chi(x)|x|^{s}d^{\times}x \frac{dt}{t}
$$
If $\chi|_{\Aa^{\times}_{1}}$ is nontrival, then this integral is zero since there exists $y_{0}\in \Aa^{\times}_{1}$ with $\chi(y_{0}) \neq 1$ and
$$
\int\limits_{\substack{\Aa^{\times}/F^{\times} \\ |x| = t}} \chi(x)|x|^{s} d^{\times}x = \int\limits_{\substack{\Aa^{\times}/F^{\times} \\ |x| = t}} \chi(xy_{0})|xy_{0}|^{s} d^{\times}x = \chi(y_{0})\int\limits_{\substack{\Aa^{\times}/F^{\times} \\ |x| = t}} \chi(x)|x|^{s} d^{\times}x
$$
impiles that the integral vanishes. If  it is trivial, then $\chi(x) = |x|^{\lambda}$ for some $\lambda\in i\Rr$ (since $\chi$ is unitary) and the integral became
$$
\rho_{F} \int_{0}^{1} t^{s+\lambda} \frac{dt}{t} = \frac{\rho_{F}}{s+\lambda}
$$
where $\rho_{F}$ is the volume of $\Aa^{\times}_{1} /F^{\times}$. By the Poisson summation formula, we have
\begin{align*}
\zeta_{0}(s, \chi, \Phi) = \begin{cases} \int\dpl{\substack{\Aa^{\times}/F^{\times} \\ |x| < 1}} \left[ \sum_{\alpha\in F} \wh{\Phi}\left(\frac{\alpha}{x}\right)\right] \chi(x) |x|^{s-1}d^{\times}x & \chi|_{\Aa^{\times}_{1}} \neq 1 \\ \int\dpl{\substack{\Aa^{\times}/F^{\times} \\ |x| < 1}} \left[ \sum_{\alpha\in F} \wh{\Phi}\left(\frac{\alpha}{x}\right)\right] \chi(x) |x|^{s-1}d^{\times}x  - \frac{\rho_{F}\Phi(0)}{s + \lambda}& \chi|_{\Aa^{\times}_{1}} = 1\end{cases}
\end{align*}
Now apply the change of variable $x\to x^{-1}$ and we get 
\begin{align*}
\zeta(s, \chi, \Phi) = \begin{cases}\zeta_{1}(s, \chi, \Phi) + \zeta_{1}(1-s, \chi^{-1}, \wh{\Phi}) & \chi|_{\Aa^{\times}_{1}} \neq 1 \\
 \zeta_{1}(s, \chi, \Phi) + \zeta_{1}(1-s, \chi^{-1}, \wh{\Phi}) - \rho_{F} \left( \frac{\Phi(0)}{s+\lambda} + \frac{\wh{\Phi}(0)}{1-s-\lambda}\right)& \chi|_{\Aa^{\times}_{1}} = 1\end{cases}
\end{align*}
Essential boundedness follows from 
$$
|\zeta_{1}(s, \chi, \Phi)| \leq \int\dpl{\substack{\Aa^{\times} \\ |x| >1 }} |\Phi(x)||x|^{\Re (s)} d^{\times} x. 
$$
\end{proof}

By combining all the previous results, we get the analytic continuation and functional equation of a Hecke $L$-function. 
\begin{theorem}[Hecke, Tate]
Let  $\chi$ be a Hecke character of $\Aa^{\times}/F^{\times}$. Let $S$ be a finite set of places of $F$ such that for all $v\not\in S$, $\chi_{v}$ is unramified and the conductor of $\psi_{v}$ is $\calO_{v}$. 
Then $L_{S}(s, \chi)$ has meromorphic continuation to all $s$, entire unless there exists a complex number $\lambda$ such that $\chi(x) = |x|^{\lambda}$, in which case the poles are at $s =- \lambda$ and $s = 1-\lambda$. We have the functional equation
$$
L_{S}(s, \chi) = \left( \prod_{v\in S} \gamma_{v}(s, \chi_{v}, \psi_{v})\right) L_{S}(1-s, \chi^{-1}). 
$$
\end{theorem}
\begin{proof}
We proved that $L_{v}(s, \chi_{v}) = \zeta_{v}(s, \chi_{v}, \Phi_{v})$ holds except for finitely many places (especially, for $v\not\in S$). Then the functional equation of the local and global zeta integrals give the result.  
\end{proof}

We will complete the $L$-function by defining $L_{v}(s, \chi_{v})$ for $v\in S$, too. 
In this case, the functional equation will contain some extra factors called $\epsilon$ factors. 
If $v\in S$ is non-archimedean (so that $\chi_{v}$ is ramified), then define $L_{v}(s, \chi_{v}) = 1$. 
If $v$ is real, $\chi_{v}(x) = (x/|x|_{v})^{\epsilon}$ for some $\epsilon \in \{0, 1\}$, and we define $L_{v}(s, \chi_{v}) = \pi^{-(s+\epsilon)/2}\Gamma((s+\epsilon)/2)$. 
Finally, for complex $v$, 
$$
\chi_{v}(x) = |x|_{v}^{\nu} \left( \frac{x}{\sqrt{|x|_{v}}}\right)^{k}
$$
for some $\nu\in i\Rr$ and $k\in \Zz$ (since it is unitary). (Here $|x|_{v}$ is the square of the usual complex norm.) Then we put
$$
L_{v}(s, \chi_{v}) = \frac{1}{2}(2\pi)^{-s-\nu-|k|/2} \Gamma\left( s + \nu + \frac{|k|}{2}\right).
$$

We say that a nonzero function $e:\Cc\to \Cc$ is of exponential type if $e(s) = ab^{s}$ for some $a\in \Cc$ and $b\in\Rr$. We will show that we can choose appropriate $\Phi_{v}$ so that the quotient of local zeta integrals by local $L$-functions became a function of exponential type. 


\begin{proposition}
Let $v$ be any place of $F$ and $s_{0}\in \Cc$. Then $L_{v}(s, \chi_{v})$ has a pole at $s = s_{0}$ iff $\zeta_{v}(s, \chi_{v}, \Phi_{v})$ has a pole there for some $\Phi_{v}\in \calS(F_{v})$. 
In particular, $\zeta_{v}(s, \chi_{v}, \Phi_{v}) / L_{v}(s, \chi_{v})$ is holomorphic for all values of $s$. 
If $v$ is non-archimedean, then $\zeta_{v}(s, \chi_{v},\Phi_{v})$ is rational function in $q_{v}^{-s}$. 
There exists a choice of $\Phi_{v}$ such that $\zeta_{v}(s, \chi_{v} ,\Phi_{v})/L_{v}(s, \chi_{v})$ is a function of exponential type. 
\end{proposition}
\begin{proof}
For a non-archimedean $v$, we can write the local zeta integral as
$$
\zeta_{v}(s, \chi_{v}, \Phi_{v}) = \sum_{k\in \Zz} q_{v}^{ks} \int\limits_{|x|_{v} = q_{v}^{k}} \Phi_{v}(x)\chi_{v}(x)d^{\times}x.
$$
Since $\Phi_{v}(x)$ is compactly supported, the contribution is zero for large $k$. Also, if $-k$ is large, 
then $\Phi_{v}(x) = \Phi_{v}(0)$ (since the function is locally constant) and the contribution equals
$$
\Phi_{v}(0)\int\limits_{|x|_{v} = q_{v}^{k}} \chi_{v}(x)d^{\times}x.
$$
If $\chi_{v}$ is ramified, then this is zero and the sum is only a finite sum, i.e. $\zeta_{v}(s, \chi_{v}, \Phi_{v})$ is entire and rational in $q_{v}^{-s}$. 
If $\chi_{v}$ is unramified, then it became a geometric series for small $k$, and we can explicitly describe pole of the function. 

If $v$ is real, the integral can be decomposed as integral over $|x|\leq 1$ and $|x|>1$ as before, and we only need to analyze the first part since the second part converges absolutely for all $s$ by rapid decay of $\Phi_{v}$. 
We may split $\Phi_{v}$ into even and odd parts and handle these two cases separately. 
The integral vanishes unless the parity of $\Phi_{v}$ and $\chi_{v}$ matches. 
If $\chi_{v} = 1$ and $\Phi_{v}$ is even, then the Taylor expansion of $\Phi_{v}(x)$ has only even terms and the possible poles of the integral 
$$
\int\limits_{|x|\leq 1} \Phi_{v}(x) \chi_{v}(x)|x|_{v}^{s} d^{\times}x = 2\int_{0}^{1} \Phi_{v}(x) x^{s} d^{\times} x
$$
are at $s = 0, -2, -4, \dots$, which agrees with the poles of $L_{v}(s, \chi_{v})$. Similarly thing holds for odd cases, too. 

If $v$ is complex,  we use polar coordinate and 
$$
\zeta_{v}(s,\chi_{v}, \Phi_{v}) = \int_{0}^{\infty} r^{2\nu + 2s} \phi(r) \frac{dr}{r}
$$
where $$\phi(r) = \frac{1}{2\pi} \int_{0}^{2\pi} \Phi_{v}(re^{ik\theta}) e^{ik\theta}d\theta.$$
We consider the Taylor expansion of $\Phi_{v}$ and $\phi(r)$, which gives
$$
\phi(r) = \sum_{m-n = k} a(n, m) r^{n+m}
$$
where $a(n, m)$ is a Taylor coefficient of $\Phi_{v}(x)$ of $x^{n}\overline{x}^{m}$. 
Then we get the result about poles by the same argument as real case. 

For the suitable choice of $\Phi_{v}$, choose
\begin{align*}
\Phi_{v}(x) = \begin{cases} \chf_{\calO_{v}}(x) & v\text{ non-archimedean, $\chi_{v}$ unramified} \\
\chi_{v}(x)^{-1} \chf_{\calO_{v}^{\times}}(x) & v\text{ non-archimedean, $\chi_{v}$ ramified} \\
x^{\epsilon}e^{-\pi x^{2}} & v \text{ real} \\
\overline{x}^{k} e^{-2\pi |x|_{v}} & v\text{ complex}, k>0 \\
x^{-k} e^{-2\pi |x|_{v}} & v \text{ complex}, k<0
\end{cases}
\end{align*}
then we get $L_{v}(s, \chi_{v}) = \zeta_{v}(s, \Phi_{v}, \chi_{v})$ for all $v$. 
\end{proof}



The following proposition describes so-called (local) $\epsilon$ factor (or root numbers), which is an extra factor for the completed $L$-function. 
\begin{proposition}
Let $v$ be any place of $F$, and define 
$$
\epsilon_{v}(s, \chi_{v}, \psi_{v}) = \frac{\gamma_{v}(s, \chi_{v}, \psi_{v})L_{v}(s, \chi_{v})}{L_{v}(1-s, \chi_{v}^{-1})}.
$$
Then $\epsilon_{v}(s, \chi_{v}, \psi_{v})$ is a function of exponential type. 
If  $v$ is non-archimedean, $\chi_{v}$ is unramified and the conductor of $\psi_{v}$ is $\calO_{v}$, then $\epsilon_{v}(s, \chi_{v}, \psi_{v}) = 1$. 
\end{proposition}

\begin{proof}
By definition, we have
$$
\epsilon_{v}(s, \chi_{v}, \psi_{v}) = \frac{\zeta_{v}(1-s, \chi_{v}^{-1}, \widehat{\Phi}_{v})}{L_{v}(1-s, \chi_{v}^{-1})} \cdot \frac{L_{v}(s, \chi_{v})}{\zeta_{v}(s, \chi_{v}, \Phi_{v})} 
$$
and the result follows from the previous proposition. 
\end{proof}

Now we define
\begin{align*}
L(s, \chi) &= \prod_{v} L_{v}(s, \chi_{v}) \\
\epsilon(s, \chi) &= \prod_{v} \epsilon_{v}(s, \chi_{v}, \psi_{v})
\end{align*}
where the product is over all places $v$ of $F$. 
The next theorem show that the completed $L$-function $L(s, \chi)$ has meromorphic continuation with the functional equation that contains $\epsilon(s, \chi)$. Also, we show that $\epsilon(s, \chi)$ doesn't depend on the choice of $\psi$, although the local factors do. 


\begin{theorem}
The factor $\epsilon(s, \chi)$ is independent of the choice of $\psi$. 
The $L$-function $L(s, \chi)$ has analytic continuation to all $s$ except $s = 0$ or $1$, where it can have simple poles. We have the functional equation 
$$
L(s, \chi) = \epsilon(s, \chi) L(1-s, \chi^{-1})
$$
and $L(s, \chi)$ is essentially bounded in vertical strips. 
\end{theorem}
\begin{proof}
The functional equation follows from previous propositions:
\begin{align*}
L(s, \chi) &=\left( \prod_{v\in S}L_{v}(s, \chi_{v})\right) L_{S}(s, \chi) \\
&= \left(\prod_{v\in S} L_{v}(s, \chi_{v}) \gamma_{v}(s, \chi_{v}, \psi_{v}) L_{v}(s, \chi_{v})\right) L_{S}(1-s, \chi^{-1}) \\
&= \left( \prod_{v\in S} \frac{L_{v}(s, \chi_{v}) \gamma_{v}(s, \chi_{v}, \psi_{v})}{L_{v}(1-s, \chi_{v}^{-1})}\right) L(1-s, \chi^{-1}) \\
&= \epsilon(s, \chi) L(1-s, \chi^{-1})
\end{align*}
and it is evident from the functional equation that $\epsilon(s, \chi)$ is independent of the choice of $\psi$. Essential boundedness follows from 
$$
L(s, \chi) = \zeta(s, \Phi, \chi) \prod_{v} \frac{L_{v}(s, \chi)}{\zeta_{v}(s, \Phi_{v}, \chi_{v})}
$$
and the fact that we can choose $\Phi = \prod_{v}\Phi_{v}$ so that the quotient $L_{v} / \zeta_{v}$ is exponential type for all $v$ (and identically 1 for almost all $v$), and $\zeta(s, \Phi, \chi)$ is essentially bounded. 
\end{proof}


















\subsection{Definition of automorphic forms and representations}

Now we will develop similar theory for $\GL(2)$. Before doing this, we first define automorphic forms of $\GL(2, \Rr)^{+}$, which is a generalization of both modular forms and Maass forms, and then define automorphic representations which use ad\'elic language. 


Let $G = \GL(2, \Rr)^{+}$ and let $\Gamma$ be a discrete subgroup that contains $-I$ and cofinite, i.e. $\Gamma\backslash \calH$ has a finite volume. But we \emph{do not} assume that $\Gamma\backslash \calH$ is compact, so that we will allow discrete subgroups like $\Gamma_{0}(N)$ (congruence subgroups) that has \emph{cusps}. Let $K = \SO(2)$, $\chi$ be a character of $\Gamma$, and let $\omega$ be a character of the center $Z(\Rr)$ of $G$. 
Here we assume that all the characters are unitary. 



\begin{definition}[Automorphic forms on $\GL(2, \Rr)^{+}$]
Let $\calA(\Gamma\backslash G, \chi, \omega)$ be the space of smooth functions $F:G\to \Cc$ such that 
\begin{enumerate}
\item (automorphic) $$F(\gamma gz) = \chi(\gamma)\omega(z)F(g), \quad \gamma\in \Gamma, \, g\in G, \, z\in Z(\Rr)$$
\item (finiteness) $F$ is $K$-finite; its right translates by elements of $K$ span a finite dimensional space. Also, $F$ is $\calZ$-finite; it lies in a finite dimensional vector space that is invariant by action of the center of the universal enveloping algebra $U\frag_{\Cc}$.  
\item (growth condition) there exists a constant $C$ and $N$ such that $$|F(g)|  < C||g||^{N}, \quad g\in G,$$where $||g||$ is the length of the vector $(g, \det g^{-1})$ in $\Rr^{5}$. 
\end{enumerate}
We call elements of $\calA(\Gamma\backslash G, \chi, \omega)$ automorphic forms. 
\end{definition}

We also define \emph{cusp forms}. Assume that $a = \infty$ is a cusp of $\Gamma$ so that $\Gamma$ contains an element of the form $\tau_{r} = \smat{1}{r}{0}{1}$. We say that $F$ is cuspidal at $\infty$ if either $\chi(\tau_{r})\neq 1$ or 
$$
\int_{0}^{r} F\left( \pmat{1}{x}{0}{1}g \right) dx = 0. 
$$
If $a$ is an arbitrary cusp, we can find $\xi\in \SL(2, \Rr)$ such that $\xi(\infty) = a$. 
Then $F'(g) = F(\xi g)$ is an element in $L^{2}(\Gamma'\backslash G, \chi', \omega)$ with $\Gamma' = \xi^{-1}\Gamma\xi$ and $\chi'(\gamma) = \chi(\xi\gamma\xi^{-1})$. We say that $F$ is cuspidal at $a$ if $F'$ is cuspidal at $\infty$. 

\begin{definition}
Let $\calA_{0}(\Gamma\backslash G, \chi, \omega)$ be a subspace of automorphic functions which are cuspidal at every cusp. We call elements of $\calA_{0}(\Gamma\backslash G, \chi, \omega)$ cusp forms. 
\end{definition}


\begin{theorem}
\label{autoadm}
The spaces $\calA(\Gamma\backslash G, \chi, \omega)$ and $\calA_{0}(\Gamma \backslash G, \chi, \omega)$ are stable under the action of $U\frag_{\Cc}$. If $f\in \calA(\Gamma\backslash G, \chi, \omega)$ ,then $U\frag_{\Cc}f$ is an admissible $(\frag, K)$-module. 
For any $f\in  \calA(\Gamma\backslash G, \chi, \omega)$ and $D\in U\frag_{\Cc}$, $Df$ also satisfies the growth estimate with the same $N$ as $f$. 
\end{theorem}

\begin{proof}
This is a theorem of Harish-Chandra. More precisely, he proved the same theorem for $G = \SL(2, \Rr)$ and $K = \SO(2)$. Now we can reduce the original statement to the $\SL(2, \Rr)$ case. 
Indeed, for $f\in \calA(\Gamma \bs G, \chi, \omega)$, $|f|$ is constant on the cosets of $Z(\Rr)$ and it is easy to see that in each coset of $Z(\Rr)$, the element $g$ with the minimal height $||g||$ is actually in $\SL(2, \Rr)$. 
For the proof of Harish-Chandra's theorem, see Theorem 2.9.2 of \cite{bu}.
\end{proof}

For example, modular forms and Maass forms are automorphic forms with some additional conditions, such as being holomorphic  or being an eigenvector of Laplacian operator. 
Also, we know that modular forms can be regarded as a Maass form: if $f(z)$ is a weight $k$ modular form, then $z\mapsto y^{k/2}f(z)$ is a Maass form with the eigenvalue $\lambda = -\frac{k}{2}\left(1-\frac{k}{2}\right)$. 

Now we relate these classical automorphic forms to $(\frag, K)$-modules in the previous theorem. Let $f$ be a Maass form of weight $k$. Define a function $F:G\to \Cc$ as
$$
F(g) = (f||_{k}g)(i). 
$$
One can check that $F\in C^{\infty}(\Gamma\backslash G, \chi, \omega)$, where $\omega$ is the character of $Z(\Rr) = \Rr^{\times}$ that is trivial on the connected component of the identity and agrees with $\chi$ on $-I$. Since $f$ is an eigenfunction of $\Delta_{k}$, $F$ is an eigenfunction of $\Delta$ and it is $Z(\Rr)^{+}$-finite. 
Also, the function $F$ satisfies the equation $F(g\kappa_{\theta}) = e^{ik\theta} F(g)$, which implies that $F$ is also $K$-finite. Growth condition of $f$ is equivalent to that of $F$, so that $F$ generates an admissible $(\frag, K)$-module. The effects of $R_{k}, L_{k}$ on $f$ are transferred into the effects of $R, L$ on $F$. 


We are also interested in the decomposition of (right) regular representation of $L^{2}(\Gamma \bs G, \chi, \omega)$ and $L_{0}^{2}(\Gamma \bs G, \chi, \omega)$. 
As before, we have the corresponding action of Hecke algebra $C_{c}^{\infty}(G)$ given by 
$$
(\rho(\phi)f)(g) = \int\dpl{G} f(gh)\phi(h)dh
$$
for $f\in L^{2}(\Gamma \bs G, \chi, \omega)$. 
$\rho(\phi)$ is an operator on $L^{2}(\Gamma \bs G,\chi, \omega)$ leaving $L^{2}_{0}(\Gamma \bs G, \chi, \omega)$ invariant, and $\rho$ is a unitary representation on both spaces. 
Also, we can rewrite the above equation as an integral over $Z(\Rr)\bs G$:
$$
(\rho(\phi)f)(g) = \int\dpl{Z(\Rr)\bs G} f(gh)\phi_{\omega}(h) dh
$$
where 
$$
\phi_{\omega}(g) = \int\dpl{\Rr^{\times}} \phi\left( \pmat{z}{}{}{z} g\right) \omega(z)\dd z.
$$
Our aim is to prove that $\rho(\phi)$ is a compact operator, so that the space $L^{2}_{0}(\Gamma \bs G, \chi, \omega)$ decomposes into a Hilbert space direct sum of irreducible invariant subspaces. 
To prove that, we need a notion of Siegel sets. 
For $c, d>0$, we denote by $\calF_{c, d}$ the Siegel set of $z = x+iy\in \calH$ such that $0\leq x\leq d$ and $y\geq c$. 
Also, we denote by $\calF_{d}^{\infty}$ the set of $z$ such that $0\leq x\leq d$. 
\begin{proposition}
\label{siegel}
\begin{enumerate}
\item Let $a_{1}, \dots, a_{h}\in \Rr\cup \{\infty\}$ be the cusps of $\Gamma$, and let $\xi_{i}\in \SL(2, \Rr)$ be chosen such that $\xi_{i}(a_{i}) = \infty$. 
Then we can choose $c, d>0$ so that the set 
$$
\bigcup_{i = 1}^{h} \xi_{i}^{-1}\calF_{c, d}
$$
contains a fundamental domain for $\Gamma$. 
\item Suppose that $\infty$ is a cusp of $\Gamma$. Then if $d$ is large enough, $\calF_{d}^{\infty}$ contains a fundamental domain for $\Gamma$. 
\end{enumerate}
\end{proposition}
\begin{proof}
One can find $c, d>0$ such that $\xi_{i}^{-1}\calF_{c, d}$ contains a neighborhood of the cusp $a_i$ in the fundamental domain $F$ of $\Gamma$, and so $F - \bigcup_{i} \xi_{i}^{-1}\calF_{c, d}$ is relatively compact in $\calH$. 
Then we can increase $d$ and decrease $c$ so that $F = \bigcup_{i} \xi_{i}^{-1} \calF_{c, d}$. 
For 2, we can also find sufficiently large $d'$ so that $\calF_{d'}^{\infty}$ contains each $\xi_{i}^{-1}\calF_{c, d}$. 
\end{proof}
We can lift these Siegel sets to subsets of $G$ under the map $G\to \calH, \smat{a}{b}{c}{d}\mapsto \frac{ai+b}{ci+d}$. 
Let $\calG_{c, d}, \calG_{d}^{\infty}$ be the preimages of $\calF_{c, d}, \calF_{d}^{\infty}$ in $G$. Also, we denote $\ol{\calG_{c, d}}, \ol{\calG_{d}^{\infty}}$. Then the previous proposition also holds for fundamental domains of $\Gamma \bs G$ and $\Gamma \bs G/Z(\Rr)$. 
\begin{proposition}[Gelfand, Graev, Piatetski-Shapiro]
\label{compact}
\begin{enumerate}
\item There exists a constant $C$ depending on $\phi$ such that 
$$
\sup_{g\in G} |\rho(\phi)f(g)| \leq C||f||_{2}
$$
for all $f\in L_{0}^{2}(\Gamma\backslash G, \chi, \omega)$, where 
$$
||f||_{2} = \left(\int_{G/Z(\Rr)} |f(g)|^{2} dg \right)^{1/2}.
$$
\item The restriction of the operator $\rho(\phi)$ to $L_{0}^{2}(\Gamma\backslash G, \chi, \omega)$ is a compact operator. 
\end{enumerate}
\end{proposition}
\begin{proof}
Recall that $\Gamma$ has cusps. We can assume that $\infty$ is a cusp of $\Gamma$ and that $\Gamma$ contains the group 
$$
\Gamma_{\infty} = \left\{ \pmat{1}{n}{}{1} \,:\, n\in \Zz\right\}.
$$
By the Proposition \ref{siegel}, it is enough to show that 
$$
\sup_{g\in \calG_{c, d}}|\rho(\phi)f(g)| \leq C_{0}||f||_{2}
$$
for some $C_{0} >0$. We have
\begin{align*}
(\rho(\phi)f)(g) &= \int_{Z(\Rr)\bs G} f(gh)\phi_{\omega}(h) dh \\
&= \int_{Z(\Rr)\bs G} f(h) \phi_{\omega}(g^{-1}h)dh \\
&= \int\limits_{\Gamma_{\infty} Z(\Rr)\bs G} \sum_{\gamma\in \Gamma_{\infty}} f(\gamma h)\phi_{\omega}(g^{-1}\gamma h) \\
&= \int\limits_{\Gamma_{\infty} Z(\Rr)\bs G} K(g, h) f(h) dh
\end{align*}
where 
$$
K(g, h) = \sum_{\gamma\in \Gamma_{\infty}} \chi(\gamma) \phi_{\omega}(g^{-1}\gamma h). 
$$
Now we will assume $\chi$ is trivial. The nontrivial case is almost same as the following proof.
Since $f$ is cuspidal, we have 
$$
\int\limits_{\Gamma_{\infty} Z(\Rr)\bs G} K_{0}(g, h) f(h) dh = 0
$$
where we define
$$
K_{0}(g, h) = \int_{-\infty}^{\infty} \phi_{\omega} \left( g^{-1} \pmat{1}{x}{}{1} h\right) dx.
$$
So we may write 
$$
(\rho(\phi)f)(g) = \int\limits_{\Gamma_{\infty} Z(\Rr)\bs G} K'(g, h) f(h) dh
$$
where $K'(g, h) = K(g, h) - K_{0}(g, h)$. Now we will estimate this function to get the desired result. 
By the Poisson summation formula, we have
$$
K'(g, h) = \sum_{n\neq 0} \wh{\Phi}_{g, h}(n)
$$
where 
$$
\Phi_{g, h}(t) = \phi_{\omega} \left( g^{-1} \pmat{1}{t}{}{1} h\right). 
$$
Using Iwasasa decomposition, we can write $g, h$ as 
$$
g = \pmat{\eta}{}{}{\eta}\pmat{y}{x}{}{1} \kappa_{\theta}, \quad h = \pmat{\zeta}{}{}{\zeta}\pmat{v}{u}{}{1} \kappa_{\sigma}
$$
where $y, u, \eta, \zeta > 0$. By the change of variable, we can show that the absolute value of $|\wh{\Phi}_{g, h}(n)|$ is  independent of $x, u$ and $|\wh{\Phi}_{g, h} (n)| = |y|\,|\wh{F}_{\theta, \sigma, y^{-1}v}(yn)|$, where 
$$
F_{\theta, \sigma, v}(t) = \phi_{\omega} \left( \kappa_{\theta}^{-1} \pmat{1}{t}{}{1} \pmat{v}{}{}{1} \kappa_{\sigma}\right). 
$$
(Note that the central character $\omega$ is unitary.)
By Fourier theory, since $F$ is smooth, $\wh{F}$ decays faster than any polynomial, i.e. for any $N$ we have a constant $C = C_{\theta, \sigma, v}$ (vary continuously in $\theta, \sigma, v$) such that 
$$
|\wh{F}_{\theta, \sigma, v}(y)| \leq C_{\theta, \sigma, v}|y|^{-N}.
$$
Since $\phi_{w}$ is compactly supported modulo $Z(\Rr)$, there exists $B>1$ such that $F_{\theta, \sigma, v}(t) = 0$ unless $B^{-1} \leq v \leq B$.  So we have 
$$
|\wh{\Phi}_{g, h}(n)| \leq C_{1} |y|^{1-N}|n|^{-N}
$$
where $C_{1} = \max_{(\theta, \sigma, v)\in [0, 2\pi]\times [0, 2\pi] \times [yB^{-1}, yB]} C_{\theta, \sigma, v} < \infty$. 
Also, $\Phi_{g, h}(n) = 0$ unless $B^{-1} \leq y^{-1}v \leq B$, so by summing up we get
$$
|K'(g, h)| \leq C_{2} |y|^{1-N}
$$
where $N\geq 2$ and $C_{2}$ is a constant depending on $\phi$ and $N$. 

To estimate $(\rho(\phi)f)(g)$ for $g\in \calG_{c, d}$, since the kernel $K'$ vanishes unless $B^{-1}\leq y^{-1}v$, we have 
$$
|(\rho(\phi)f)(g)| \leq C_{2}y^{1-N} \int\limits_{\overline{\calG}_{B^{-1}c, d}}|f(h)| dh. 
$$
Now $\overline{\calG}_{B^{-1}c, d}$ can be covered by a finite number of copies of a fundamental domain, so it is dominated by $L^{1}$ norm of $f$, so is by $L^{2}$ norm (the fundamental domain has finite volume). 
This proves 1 and also shows that $\rho(\phi)f(g)$ decays rapidly. 

To prove compactness of $\rho(\phi)$, we use Arz\'ela-Ascoli theorem. Let $Y$ be a compactification of $\Gamma Z(\Rr) \bs G$ by adjoining cusps and let $\Sigma$ be the image of the unit ball in $L^{2}_{0}(\Gamma\bs G, \chi, \omega)$ under $\rho(\phi)$; we extend each $\rho(\phi)f$ to $Y$ by making it vanish at the cusps. This $\Sigma$ is bounded because of the inequality we just proved, and equicontinuity follows from that derivatives are bounded uniformly for all $f$ with $||f||_{2} \leq 1$. 
This follows from $(X\rho(\phi)f)(g) = \rho(\phi_{X})f(g)$ where 
$$
\phi_{X}(g) = \frac{d}{dt} \phi(\exp (-tX)g)|_{t=0}. 
$$
Hence $\Sigma$ is compact in $L^{\infty}$ topology and hence also in $L^{2}$ topology. 
\end{proof}
\begin{theorem}
\label{cuspdecom}
The space $L_{0}^{2}(\Gamma\backslash G, \chi, \omega)$ decomposes into a Hilbert space direct sum of subspaces that are invariant and irreducible under the right regular representation $\rho$. Let $H$ be a such a subspace. 
Then $K$-finite vectors $H_{\fin}$ in $H$ are dense, and every $K$-finite vectors form an irreducible admissible $(\frag, K)$-module contained in $\calA_{0}(\Gamma\backslash G, \chi, \omega)$. 
\end{theorem}
\begin{proof}
The proof that $L_{0}^{2}(\Gamma \bs G, \chi, \omega)$ decomposes into a Hilbert space direct sum of irreducible invariant subspaces is almost same as the proof of Theorem \ref{l2decomp}. Here we use the previous proposition and the spectral theorem for self-adjoint compact operators. 
Let $H$ be an irreducible invariant subspace of $L_{0}^{2}(\Gamma \bs G, \chi, \omega)$. 
Then $H = \oplus_{k} H_{k}$ where $H_{k} = \{f\in H\,:\, \rho(\kappa_{\theta})f = e^{ik\theta}f\}$. 
By Theorem \ref{mult1char}, $\dim H_{k} \leq 1$ for all $k$, and this implies that $H_{\fin}$ is a $(\frag, K)$-module. 

Finally, to show $H_{\fin} \subseteq \calA_{0}(\Gamma\bs G, \chi, \omega)$, it is enough to show that $H_{k} \subseteq \calA_{0}(\Gamma\bs G, \chi, \omega)$. 
Let $0\neq f\in H_k$. 
One can prove that there exists $\phi\in C_{c}^{\infty}(G)$ such that $\phi(\kappa_{\theta}g) = \phi(g\kappa_{\theta}) = e^{-ik\theta}\phi(g)$ and $\rho(\phi)f \neq 0$. 
It is easy to check that $\rho(\phi)f\in H_k$, so from $\dim H_k \leq 1$, we may assume that $\rho(\phi) f = f$. 
By the way, convolutioning $f$ with a compactly supported smooth function $\phi\in C_{c}^{\infty}(G)$ makes $\rho(\phi)f = f$ as a smooth and rapidly decreasing function (this follows from the equation $X\rho(\phi)f = \rho(\phi_X)f$ and the estimation of $K'(g, h)$), so $f\in \calA_{0}(\Gamma \bs G, \chi, \omega)$. 
\end{proof}


We can also consider the subspace $\calA_{0}(\Gamma, \chi, \omega, \lambda, \rho)$ of automorphic forms which are cuspidal, $\lambda$-eigenspace of $\Delta$, and $\sigma$-isotypic. 
The following theorem shows that this space is finite dimensional, and all the irreducible admissible unitary representations of $\GL(2, \Rr)$ occur in $L^{2}_{0}$ with finite multiplicity. 

\begin{theorem}
\label{autofin}
\begin{enumerate}
\item Let $(\pi, V)$ be an irreducible admissible unitary representation of $\GL(2, \Rr)$. Then the multiplicity of $\pi$ in the decomposition of $L^{2}_{0}(\Gamma \backslash G, \chi, \omega)$ is finite. 
\item Let $\lambda\in \Cc$ and let $\sigma$ be a character of $K$. Then $\calA_{0}(\Gamma, \chi, \omega, \lambda, \rho)$ is finite dimensional. 
\end{enumerate}
\end{theorem}
\begin{proof}
We prove 1. 
Choose $\sigma \in \wh{K}$ such that $V(\sigma)\neq 0$, and let $0\neq \xi\in V(\sigma)$. 
One can choose $\phi\in C_{c}^{\infty}(G)$ such that $\pi(\phi)\xi = \xi$ and $\pi(\phi)$ is self-adjoint and compact. 
If $T:V\to L_{0}^{2}(\Gamma \bs G, \chi, \omega)$ is an intertwining map, then $\rho(\phi)T\xi = T\xi$, so $T\xi$ lies in the 1-eigenspace of the compact operator $\rho(\phi)$, which is finite dimensional by the spectral theorem. 
Since $\pi$ is irreducible, $T$ is determined by the image of any single nonzero vector, and it follows that $\Hom_{\GL(2, \Rr)}(V, L_{0}^{2}(\Gamma \bs G, \chi, \omega))$ is finite dimensional. 

For the second part, we assume that $\Delta$ acts as a constant $\lambda$ and also $\omega$ is fixed, so the action of  $Z = \smat{1}{0}{0}{1}\in \calZ (U\frag)$ is also given by a constant $\mu$ determined by $\omega$. By Theorem \ref{unigk}, there are only finite number of isomorphism classes of unitary irreducible admissible representations with given $\omega$ and $\lambda$ (in fact, only one or two). 
Let $\Sigma$ be the set of these isomorphism classes. 
By the previous Theorem \ref{cuspdecom}, we know that $L_{0}^{2}(\Gamma \bs G, \chi, \omega)$ decomposes into a direct sum of irreducible admissible representations and that the $K$-finite vectors in each of these are elements of $\calA_{0}(\Gamma \bs G, \chi, \omega)$, so $\calA_{0}(\Gamma, \chi, \omega, \lambda, \rho)$ is the direct sum of the $\rho$-isotypic components of the irreducible subspaces of $L_{0}^{2}(\Gamma \bs G, \chi, \omega)$ that are isomorphic of an element of $\Sigma$. 
Now the finite dimensionality follows from the finiteness of $\Sigma$ and the finite multiplicity of each representations in $\Sigma$. 
\end{proof}

Now, we are going to transfer everything in terms of ad\'elic setting. In particular, we will define automorphic forms and representations of $\GL(n, \Aa)$. 
This is a modern point of view and this is also a natural way to study in philosophy of local-global principle - think about Tate's thesis. 
(Actually, Tate's thesis is exactly about the theory of $\GL(1)$ automorphic forms.) 

Before we define them, we will prove the ad\'elic version of the Theorem \ref{cuspdecom}, which can be prove by using the ad\'elic version of the Proposition \ref{compact}. 
Ideas are almost same and we only need to define everything properly in terms of ad\'eles. 

\begin{definition}
\label{l2def}
\begin{enumerate}
\item
Let $\omega$ be a unitary Hecke character (unitary character of  $\Aa^{\times}/F^{\times}$). 
Let $L^{2}(\GL(n, F) \bs \GL(n, \Aa), \omega)$ be the space of all function $\phi$ on $\GL(n, \Aa)$ that are measurable with respect to Haar measure and that satisfy
\begin{enumerate}
\item For any $z\in \Aa^{\times}$ and $\gamma \in \GL(n, F)$, 
$$
\phi\left(\gamma g\begin{pmatrix} z & & \\ & \ddots & \\ & & z\end{pmatrix}\right) = \omega(z)\phi(g)
$$
\item Square integrable modulo the center:
$$
\int\dpl{Z(\Aa)\GL(n, F) \bs \GL(n, A)} |\phi(g)|^{2} dg < \infty. 
$$ 
\end{enumerate}
Also, $\phi$ is cuspidal if it satisfies 
$$
\int\dpl{\Mat_{r\times s}(F) \bs \Mat_{r\times s}(\Aa)} \phi\left(\pmat{I_{r}}{X}{}{I_{s}}g\right)dX = 0
$$
for any $r, s>0$ with $r + s = n$ and for a.e. $g$. 
We let $\GL(n, \Aa)$ act on $L^{2}(\GL(n, F)\bs \GL(n, \Aa), \omega)$ by right translation. 
\item Let $C_{c}^{\infty}(\GL(n, \Aa))$ be a space of functions that are finite linear combinations of functions $\phi(g) = \prod_{v} \phi_{v}(g_{v})$, where $\phi_v \in C_{c}^{\infty}(\GL(n, F_{v}))$ for each $v$ and $\phi_v = \chf_{\calO_{v}}$ for almost all $v$. 
Then we have $C_{c}^{\infty}(\GL(n, \Aa))$-action on $L^{2}(\GL(n, F)\bs \GL(n, \Aa), \omega)$ given by 
$$
(\rho(\phi)f)(g) = \int\dpl{\GL(n, \Aa)} \phi(h)f(gh)dh = \int\dpl{Z(\Aa)\bs \GL(n, \Aa)} \phi_{\omega}(h) f(gh)dh, 
$$
where
$$
\phi_{\omega}(g) = \int\dpl{\Aa^{\times}} \phi \left(\begin{pmatrix}z & & \\ & \ddots & \\ & & z \end{pmatrix}\right)\omega(z)\dd z.
$$
\end{enumerate}
\end{definition}

To prove the decomposition theorem of $L^{2}_{0}(\GL(2, F) \bs GL(2, \Aa), \omega)$, we need some well-known properties of $\GL(n, \Aa)$. 
We use the following propositions in the proof of the Proposition \ref{compactad}, but we will not prove these theorems. See Theorem 3.3.1, Proposition 3.3.1, and Proposition 3.3.2 in \cite{bu}. 



\begin{theorem}[Strong approximation]
Let $F$ be a number field. 
\begin{enumerate}
\item $\SL(n, F_{\infty})\SL(n, F)$ is dense in $\SL(n, \Aa)$. 
\item Let $K_{0}$ be a open compact subgroup of $\GL(n, \Aa_{\fin})$. Assume that the image of $K_{0}$ in $\Aa_{\fin}^{\times}$ under the determinant map is $\prod_{v\not\in S_{\infty}} \calO_{v}^{\times}$. Then the cardinality of 
$$
\GL(n, F)\GL(n, F_{\infty})\bs \GL(n, \Aa)/K_{0}
$$
is equal to the class number of $F$. 
\end{enumerate}
\end{theorem}

\begin{proposition}
Let $\Aa = \Aa_{\Qq}$ be the ad\'ele ring of $\Qq$. The inclusion $\SL(2, \Rr)\to \GL(2, \Aa)$ induces a homeomorphism 
$$
\Gamma_{0}(N) \bs \SL(2, \Rr) \simeq Z(\Aa) \GL(2, \Qq) \bs \GL(2, \Aa) /K_{0}(N).
$$
As a corollary, $Z(\Aa)\GL(2, \Qq) \bs \GL(2, \Aa)$ has finite measure. 
\end{proposition}
Note that this holds for any $n \geq 1$ and a number field $F/\Qq$. 
Now we can prove our main proposition for the decomposition theorem. 


\begin{proposition}[Gelfand-Graev-Piatetski-Shapiro]
\label{compactad}
Let $\phi\in C_{c}^{\infty}(\GL(n, \Aa))$. 
\begin{enumerate}
\item There exists $C>0$ (depending on $\phi$) such that 
$$
\sup_{g\in \GL(2, \Aa)} |\rho(\phi)f(g)| \leq C||f||_{2}
$$
for all $f\in L_{0}^{2}(\GL(2, F) \bs \GL(2,\Aa), \omega)$. 
\item The operator $\rho(\phi)$ is compact on $L_{0}^{2}(\GL(n, F)\bs \GL(n, \Aa), \omega)$. 
\end{enumerate}
\end{proposition}
\begin{proof}
We will only prove when $n = 2$ and $F = \Qq$. 
We can define Siegel sets $\calG_{c, d}\subset \GL(2, \Aa)$ as the set of ad\'eles of the form $(g_{v})$ where $g_{v} \in K_v$ for all non-archimedean $v$ and 
$$
g_{\infty} = \pmat{z}{}{}{z} \pmat{y}{x}{}{1} \kappa_{\infty}, \qquad z\in \Rr^{\times}, y\geq c, 0\leq x\leq d, \kappa_{\infty} \in K_{\infty}. 
$$
Then if $c\leq \sqrt{3}/2$ and $d\geq 1$ we have $\GL(2, \Aa) = \GL(2, \Qq) \calG_{c, d}$. 
This follows from the strong approximation theorem and the fact that $\calF_{c, d}$ contains a fundamental domain for $\SL(2, \Zz)$. 

To prove the inequality, we use the same trick as before. We can express $\rho(\phi)f$ as 
$$
(\rho(\phi)f)(g) = \int\dpl{N(F)Z(\Aa)\bs \GL(2, \Aa)} K'(g, h)f(h)dh, 
$$
where
\begin{align*}
K'(g, h) & = K(g, h) - K_{0}(g, h), \\
K(g, h) & = \sum_{\gamma \in N(F)} \phi_{\omega}(g^{-1}\gamma h), \\
K_{0}(g, h) = \int\dpl{\Aa/F} &\phi_{\omega} \left( g^{-1} \pmat{1}{x}{}{1} h\right)dx. 
\end{align*}
Let $g\in \calG_{c, d}$. We can write it as
$$
g = \pmat{\eta}{}{}{\eta} \pmat{y}{x}{}{1} \kappa_{g}
$$
where $\eta\in \Rr^{\times}, 0\leq x\leq d, y\geq c$ and $\kappa_{g} \in K$. 
Also, an arbitrary element $h\in \GL(2, \Aa)$ can be written as
$$
h = \pmat{\zeta}{}{}{\zeta}\pmat{v}{u}{}{1} \kappa_{h}
$$
where $\zeta v\in \Aa^{\times}, u\in \Aa$, and $\kappa_{h} \in K$. 
By the Poisson summation formula, 
$$
K'(g, h) = \sum_{\alpha\in F^{\times}} \wh{\Phi}_{g, h}(\alpha)
$$
where $\Phi_{g, h}:\Aa\to \Cc$ is the compactly supported continuous function 
$$
\Phi_{g, h}(x) =\phi_{\omega} \left( g^{-1} \pmat{1}{x}{}{1} h\right).
$$
By substitution, we can also check that
$$
\wh{\Phi}_{g, h}(\alpha) = \psi(\alpha(x-u))\omega(\zeta^{-1}\eta) |y|\wh{F}_{\kappa_{g}, \kappa_{h}, y^{-1}v}(\alpha y),
$$
where 
$$
F_{\kappa_{g}, \kappa_{h}, y}(t) = \phi_{\omega} \left( \kappa_{g}^{-1} \pmat{1}{t}{}{1} \pmat{y}{}{}{1} \kappa_{h}\right). 
$$
Since $K\supp(\phi)K \cap B(\Rr)$ is compact, there exists a compact subset $\Omega \subset \Aa^{\times}$ such that if $F_{\kappa_g, \kappa_h, y}(t) \neq 0$ for any $t$, then $y\in \Omega$. We have
$$
|K'(g, h)| \leq |y| \sum_{\alpha\in F^{\times}} |\wh{F}_{\kappa_g, \kappa_h, y^{-1}v}(\alpha y)| 
$$
and $F_{\kappa_g, \kappa_h, y^{-1}v}(y)$ vanishing identically unless $(\kappa_g, \kappa_h, y^{-1}v) \in K\times K\times \Omega$, which is a compact set. 
Therefore for any given $N>0$, there exists a constant $C_N > 0$ such that $|K'(g, h)| \leq C_N |y|^{-N}$ and $K'(g, h) = 0$ unless $y^{-1}v \in \Omega$. Thus
$$
|(\rho(\phi)f)(g)| \leq C_{N}|y|^{-N} \int\dpl{\Aa/F} \int\dpl{y^{-1}v\in \Omega}\int\dpl{K} \Bigg| f\left( \pmat{v}{u}{}{1} \kappa_{h}\right)\Bigg| |v|^{-1} d\kappa_{h}d^{\times}v du. 
$$
(Here $|v|^{-1}$ comes from $d_{L}b = |v_{2}/v_{1}| d_{R}b = |v_{2}/v_{1}|d^{\times}v_1 d^{\times}v_2 du$ for $b = \smat{v_1}{u}{}{v_2}$.) 
Since $\Aa/F \times y\Omega \times K$ is compact, it can be covered by a finite number of copies of a fundamental domain of $\GL(2, F)Z(\Aa)$. Since it has finite measure, RHS is dominated by $||f||_{1}$, so $||f||_{2}$. 
Compactness of $\rho(\phi)$ follows from Arzela-Ascoli theorem as before, by showing that the image of unit ball in $L^{2}_{0}(\GL(2, F) \bs \GL(2, \Aa), \omega)$ is equicontinuous. 
By no small subgroup argument, there exists an open subgroup of $\GL(2, \Aa_{\fin})$ under which $\phi$ is right invariant, and any element of the image of $\rho(\phi)$ will be right invariant under this same subgroup. So we only need to show that $(\rho(\phi)f)(g)$ are equicontinuous as functions of $g_{\infty}$, and this follows from a uniform bound for $X\rho(\phi)f$ as before. 
\end{proof}

This with the spectral theorem prove the following decomposition theorem. 

\begin{theorem}
\label{cuspdecomad}
The space $L_{0}^{2} (\GL(n, F) \bs \GL(n, \Aa), \omega)$ decomposes into a Hilbert direct sum of irreducible invariant subspaces. 
\end{theorem}


Now we define automorphic forms and representations of $\GL(n, \Aa)$. 
Automorphic forms of $\GL(n, \Aa)$ are functions on $\GL(n, \Aa)$ that satisfies the transformation law, the $K$-finiteness and $\calZ$-finiteness condition, and the growth condition. 
\begin{definition}[Automorphic forms on $\GL(n, \Aa)$]
An automorphic forms on $\GL(n, \Aa)$ with central quasi-character $\omega$ are functions that satisfies 
\begin{enumerate}
\item (automorphic) 
$$
\phi\left( \gamma g\begin{pmatrix} z & & \\ & \ddots & \\  & & z\end{pmatrix} \right) = \omega(z)\phi(g)
$$
for all $g\in \GL(n, \Aa), z\in \Aa^{\times}$ and $\gamma\in \GL(n, F)$. 
\item (finiteness) $\phi$ is $K$-finite; its right translates by elements of $K$ span a finite dimensional space. Also, $\phi$ is $\calZ$-finite; it lies in a finite dimensional vector space that is invariant by action of centers of universal enveloping algebras $U(\mathfrak{gl}(n, F_{v})_{\Cc}) = U\mathfrak{gl}(n, \Cc)$ for each archimedean place $v$. 
\item (growth condition) there exists a constant $C$ and $N$ such that $f(g) \leq C||g||^{N}$, where $||g|| := \prod_{v} \max_{1\leq i, j\leq n}\{ |g_{ij}|_{v}, |\det(g)^{-1}|_{v}\}$. 
\end{enumerate}
We denote by $\calA(\GL(n, F)\bs \GL(n, \Aa), \omega)$ the space of automorphic forms with central quasi-character $\omega$ and by $\calA_{0}(\GL(n, F)\bs \GL(n, \Aa), \omega)$ the space of cusp forms, which are further assumed to satisfy exactly same integral equation in Definition \ref{l2def}. 
\end{definition}

Note that $\GL(n, \Aa)$ does not act on the space $\calA(\GL(n, F)\bs \GL(n, \Aa), \omega)$ since $K$-finiteness is not preserved by right translation by elements of $\GL(n, F_{v})$ for archimedean $v$. 
However, it is still a representation of $\GL(n, \Aa_{\fin})$ and also $(\frag_{\infty}, K_{\infty})$-module, where $\frag_{\infty} = \prod_{v\in S_{\infty}}\mathfrak{gl}(n, F_{v})$ and $K_{\infty} = \prod_{v\in S_{\infty}} K_{v}$. 
This motivates the following definition. 


\begin{definition}[Automorphic representation of $\GL(n, \Aa)$]
Automorphic representation of $\GL(n, \Aa)$ is a representation of $\GL(n, \Aa_{\fin})$ and a commuting $(\frag_{\infty}, K_{\infty})$-module structure which can be realized as a subquotient of the representation $\calA(\GL(n, F)\bs \GL(n, \Aa), \omega)$. 
\end{definition}
In Section 4.6, we will explain how to attach an automorphic representation by using the classical automorphic forms, for example, modular forms. 

We can also define the notion of an \emph{admissible} representation of $\GL(n, \Aa)$. 
\begin{definition}[Admissible representation of $\GL(n, \Aa)$]
Let $V$ be a complex vector space with $(\frag_{\infty}, K_{\infty})$-module and $\GL(n, \Aa_{\fin})$-module structure where two actions commute. 
Let's denote both actions by $\pi$. 
Then $V$ is admissible if every vector in $V$ is $K$-finite and the $\rho$-isotypic part $V(\rho)$ is finite dimensional for any irreducible finite dimensional representation $\rho$ of $K$. 
\end{definition}
In Section 4.6,  we will see that this is equivalent to a representation of the global Hecke algebra.
The following theorem shows that any irreducible subrepresentation of $L_{0}^{2}$ induces an irreducible admissible representation of $\GL(n, \Aa)$. 
\begin{theorem}
\label{l2autoad}
Let $(\pi, V)$ be an irreducible constituent of the decomposition of $L^{2}_{0}(\GL(n, F)\bs \GL(n, \Aa), \omega)$.
Then $\pi$ induces an irreducible admissible representation of $\GL(n, \Aa)$ on the space of $K$-finite vectors in $V$. 
\end{theorem}
\begin{proof}
We only prove for $n = 2$ and $F = \Qq$. We will reduce this to the Theorem \ref{autofin}. 
We need to show that $\dim V(\rho) < \infty$ for any irreducible finite dimensional representation $\rho$ of $K$. 
One can show that $\rho$ decomposes as a restricted tensor product of local factors, i.e. $\rho = \otimes_{v} \rho_{v}$ where $(\rho_{v}, V_{v})$ are finite dimensional representations of $K_v$ and $\rho_v$ is trivial for almost all $v$. (Look up the next section for the definition of restricted product of representations and the proof of this property.)
%%%%%%%%%%%%%%
\begin{comment}
Indeed, no small subgroup argument implies that $\ker \rho$ contains an open subgroup of $K_{\fin}$, so ti contains $K_v$ for all but finitely many $v$. 
Then $\rho$ factors through the projection 
$$
K \twoheadrightarrow K/ \left[ \prod_{v\not\in S}K_v\right] \simeq \prod_{v\in S} K_v
$$
for some finite set of places $S$ containing $S_{\infty}$. 
Now $\rho$ is a representation of a finite direct product of compact groups, and it factors as $\otimes_{v\in S} \rho_v$ where $\rho_v$ is an irreducible representation of $K_v$. Then the original $\rho$ is the restricted tensor product of $\rho_v$'s where $(\rho_v, V_v)$ are trivial for $v\not\in S$. 
\end{comment}
%%%%%%%%%%%%%%%
From this, there exists an open subgroup $K_{0, \fin}\subseteq K_{\fin}$ such that every vector in $V(\rho)$ is invariant under $K_{0, \fin}$ so that $V(\rho) = V^{K_{0,\fin}}(\rho)$. 

Now, we will show that $V^{K_{0, \fin}}(\rho_\infty)$ has a finite dimension. Since $V(\rho) = V^{K_{0, \fin}}(\rho) \subseteq V^{K_{0, \fin}}(\rho_{\infty})$, this shows  $\dim V(\rho) <\infty$. 
In fact, we will prove that $V^{K_{0, \fin}}(\rho_{\infty})$ is isomorphic  to  a finite product of spaces of automorphic forms on $\GL(2, \Rr)^{+}$, each of which is finite dimensional by Theorem \ref{autofin}. By strong approximation theorem, $\GL(2, \Aa) =\GL(2,\Qq)\GL(2, \Rr)^{+}K_\fin$ and since $[K_{\fin}:K_{0, \fin}] < \infty$, the set of double cosets
$$
\GL(2, \Qq) \GL(2, \Rr)^{+} \bs \GL(2, \Aa)/K_{0, \fin}
$$
is finite. 
Let $g_1, \dots, g_n$ be a set of representatives - we may assume $g_{i, \infty} = 1$ for all $i$. For $\phi\in V^{K_{0, \fin}}$, we can associate $h$ functions $\Phi_i$ on $\GL(2, \Rr)^{+}$  defined by $\Phi_{i}(g_\infty) = \phi(g_{\infty}g_{i})$. 
Any $g\in \GL(2, \Aa)$ can be written as $g = \gamma g_{\infty}g_{i}k_{0}$ for $\gamma\in \GL(2, \Qq), g_{\infty} \in \GL(2, \Rr)^{+}$ and $k_{0} \in K_{0, \fin}$, so $\phi(g) = \Phi_{i}(g_{\infty})$. 
This means that $\phi$ is uniquely determined by $\Phi_i$'s, so it is sufficient to show that each of these lies in a finite dimensional vector spaces. 

Let $\Gamma$ be the projection onto $\GL(2, \Rr)^{+}$ of $\GL(2, \Qq) \cap (\GL(2, \Rr)^{+}K_{0, \fin})$, which became a finite index subgroup of $\SL(2, \Zz)$ (in fact, this is a congruence subgroup). 
Then it is easy to check that $\Phi_i \in \calA(\Gamma \bs \GL(2, \Rr)^{+}, 1, \omega_{\infty})$ where $\omega = \prod_{v} \omega_{v}$. (Moderate growth of $\Phi_i$ follows from that of $\phi$.) 
Since $\rho_{\infty}$-isotypic subspace of this space is finite dimensional by Theorem \ref{autofin}, so is $V^{K_{0, \fin}}(\rho_\infty)$. 
\end{proof}










\subsection{Flath's decomposition theorem}

We can ask a simple but hard question: how to construct a nontrivial example of (irreducible admissible) representation of $\GL(n, \Aa)$, and how can we study? 
There's one possible and natural way to do it by \emph{glueing} local representations. To be more specific, first we define restricted tensor product of representations. 
\begin{definition}[restricted tensor product]
Let $\Sigma$ be the index set and  $\{V_{v}\}_{v\in \Sigma}$ be a family of infinite number of vector spaces. 
For almost all $v$ let there be a given nonzero $x_{v}^{\circ} \in V_{v}$. 
Let $\Omega$ be a set of all finite subsets $S$ of $\Sigma$ having the property that $x_{v}^{\circ}$ is defined for $v\not\in S$. 
We order $\Omega$ by inclusion, so that for all $S, S'\in \Omega$ and $S\subseteq S'$, there exists an injective map $\lambda_{S, S'} : \bigotimes_{v\in S} V_{v}  \to \bigotimes_{v\in S'} V_{v}$ given by $x\mapsto x\otimes (\otimes_{v\in S'\bs S}\,x_{v}^{\circ})$. 
Then this form a direct family, and we define the restricted tensor product as a direct limit
$$
\bigotimes_{v\in \Sigma} V_{v} = \lim_{\longrightarrow} \bigotimes_{v\in S} V_{v}.
$$
\end{definition}
For each $S\in \Omega$, we have natural injective maps $\lambda_{S} : \bigotimes_{v\in S} V_{v} \to \bigotimes_{v\in \Sigma} V_{v}$ and we denote the image of $x = \otimes_{v\in s} \, x_{v}$ under this map as $\otimes_{v\in \Sigma}\, x_{v}$ where $x_{v} = x_{v}^{\circ}$ for $v\not\in S$. 
Such element is called a pure tensor. 
We can consider $\bigotimes_{v\in \Sigma}V_{v}$ as a vector space spanned by pure tensors. 
One can check that changing finite number of $x_{v}^{\circ}$ does not change the restricted tensor product. 

Using this,  we can also define a restricted tensor product of representations. 

\begin{definition}
Let $\Sigma$ be an index set and for each $v\in \Sigma$, suppose that a group $G_{v}$ and a subgroup $K_{v}$ is given. 
For each $v\in \Sigma$, let $(\rho_{v}, V_{v})$ be a representation of $G_{v}$. 
Assume that there are nonzero $K_{v}$-fixed vectors $\xi_{v}^{\circ}\in V_{v}$ for almost all $v$.
Then we can define a representation $(\otimes_{v}\rho_{V}, \otimes_{v} V_{v})$ by 
$$
(\otimes_{v}\rho_{v})(g_{v})\xi_{v} = \otimes_{v}\rho_{v}(g_{v})\xi_{v}. 
$$
\end{definition}
In this note, we assume that $\Sigma$ is a set of places of some global field $F$, $K_{v}$ be maximal compact group of $\GL(n, F_{v})$ and $G_{v}= \GL(n, F_{v})$ or $G_{v} = K_{v}$. 
The following lemma shows that a finite dimensional representation of the maximal compact subgroup $K$ can be decomposed into local representations.
\begin{lemma}
\label{cptprod}
Let $\rho$ be an irreducible finite dimensionalrepresentation of $K$. Then there exists finite dimensional representations $\rho_{v}$ or $K_{v}$ such that $(\rho_{v}, V_{v})$ is 1-dimensional for almost all $v$, and nonzero vectors for such $\xi_{v}^{\circ}$ such that the restricted tensor product $\otimes_{v}\rho_{v}$ with respect to $\{\xi_{v}^{\circ}\}$ is isomorphic to $(\rho, V)$. 
\end{lemma}
\begin{proof}
By \emph{no small subgroup argument}, $\ker \rho$ contains an open subgroup of $K_{\fin}$ and so there exists a finite set of places $S$ containing $S_{\infty}$ such that $\rho(K_{v}) = 1$ for all $v\not\in S$. 
Then $\rho$ factors through the projection 
$$
K \twoheadrightarrow K / (\prod_{v\not\in S} K_{v}) \simeq \prod_{v\in S} K_{v}
$$
where the last group is a finite direct product of compact groups. 
Any irreducible representation of such group has a form $\otimes_{v\in S} \rho_{v}$ where $\rho_{v}$ is an irreducible representation of $K_{v}$. Then $\rho$ is isomorphic to the restricted tensor product with $(\rho_{v}, V_{v})$ trivial (one-dimensional) for $v\not\in S$.  
\end{proof}


%%%%%%%%%%%%%
\begin{comment}
\begin{definition}[Admissible representation of $\GL(n, \Aa)$]
A representation $(\pi, V)$ of $\GL(n, \Aa)$ is admissible if $V$ is both $(\frag_{\infty}, K_{\infty})$-module and $\GL(n, \Aa_{\fin})$-module with commuting actions, where every vector in $V$ is $K$-finite and $\rho$-isotypic part $V(\rho)$ of $V$ is finite dimensional for any irreducible finite dimensionalrepresentation of $K$. 
\end{definition}
Note that this is almost same as the local definitions that we defined in the previous chapters. 
\end{comment}
%%%%%%%%%%%%%%%


Now Flath's decomposition theorem (tensor product theorem) says that any irreducible admissible representation of $\GL(n, \Aa)$ decomposes as a restricted product of local irreducible representations, where almost all of them are spherical. 

\begin{theorem}[Flath, Tensor product theorem]
Let $(V, \pi)$ be an irreducible admissible representation of $\GL(n, \Aa)$. Then there exists for each archimedean place $v$ of $F$ an irreducible admissible $(\frag_{v}, K_{v})$-module, and for each non-archimedean place $v$ there exists an irreducible admissible representation $(\pi_{v}, V_{v})$ of $\GL(n, F_{v})$ such that for almost all $v$, $V_{v}$ contains a nonzero $K_{v}$-fixed vector $\xi_{v}^{0}$ such that $\pi$ is the restricted tensor product of the representations $\pi_{v}$. 
\end{theorem}
We will not give a complete proof here. The proof uses properties of so-called \emph{idempotented algebras}. One can use theory of idempotented algebras and apply it to Hecke algebras. 
For a non-archimedean place $v$, $\calH_{v} = C_{c}^{\infty}(G_{v})$ is a convolution algebra where $\chf_{K_{v}}$ became a spherical idempotent element of $\calH_{v}$ when the Haar measure is normalized so that the volume of $\GL(n, \calO_{v})$ is 1. (These are commutative by the Cartan decomposition theorem). 
For an archimedean place $v$, define $\calH_{v}$ as a convolution algebra of compactly supported distributions on $G_{v}$ that are supported in $K_{v}$ and $K_{v}$-finite under the left and right translations. 
(The support of $\calH_{v} = \calH_{G_{v}}$ contains $K_{v}$. $\calH_{v}$ itself contains both $\calH_{K_{v}}$ and $U\frag_{\Cc}$, and every element of $\calH_{G}$ has the form $f * D$ with $f\in \calH_{K_{v}}$ and $D\in U\frag_{\Cc}$.)
Then $(\pi, V)$, which is $\calH_{G}$-module, became a restricted tensor product of $\calH_{v}$-modules (with respect to spherical Hecke algebras $\chf_{K_{v}}\calH_{v} \chf_{K_{v}}$) which are just representations $(\pi_{v}, V_{v})$ of $G_{v}$).  


By using the decomposition theorem, we can define the contragredient representation of an irreducible admissible representation $(\pi, V)$ of $\GL(2, \Aa)$ as $\wh{\pi} = \otimes_{v} \wh{\pi}_v$. 
We already defined contragredient representation of $\GL(2, F_{v})$ for non-archimedean $v$ in Section 3.1, and we can also define the contragredient representation of given $(\frag, K)$-module as a $(\frag, K)$-module $\wh{V} = \bigoplus_{\rho\in \wh{K}}^{\alg}V(\rho)$ by 
\begin{align*}
\bra{v}{\wh{\pi}(k)\Lambda} &= \bra{\pi(k^{-1})v}{\Lambda} \\
\bra{v}{\wh{\pi}(X)\Lambda} &= -\bra{\pi(X)v}{\Lambda} 
\end{align*}
for $k\in K$ and $X\in \frag$. 
Then we can show the global analogue of the Theorem \ref{contra}, which almost directly follows from the local results. 
Note that archimedean analogue of the theorem is also true, but we will not prove here. One can prove it by using the classification of irreducible admissible $(\frag, K)$-modules of $\GL(2, \Rr)$. 
\begin{proposition}
Let $(\pi, V)$ be an automorphic cuspidal representation of $\GL(n, \Aa)$ with central quasicharacter $\omega$. 
Then $(\wh{\pi}, \wh{V})$ is also an automorphic cuspidal representation. 
If $V\subset \calA_{0}(\GL(n, F)\bs\GL(n, \Aa), \omega)$, then a subspace of $\calA_{0}(\GL(n, F)\bs\GL(n, \Aa), \omega^{-1})$ affording a representation isomorphic to $\wh{\pi}$ consists of all functions of the form $g\mapsto \phi(\pre{T}{g}^{-1})$ where $\phi\in V$. 
Also, we have $\wh{\pi} \simeq \omega^{-1}\otimes \pi$ for $n = 2$ and $F = \Qq$. 
\end{proposition}





\subsection{Whittaker models and multiplicity one} 

In this section, we will prove that irreducible representation of $\GL(n, \Aa)$ is determined by all but finitely many local components. More precisely:
\begin{theorem}[Strong multiplicity one]
\label{multone}
Let $(\pi, V), (\pi', V')$ be two irreducible admissible subrepresentations of $\calA_{0}(\GL(n, F)\bs \GL(n, \Aa), \omega)$. 
Assume that $\pi_{v} \simeq \pi_{v}'$ for all archimedean $v$ and all but finitely many non-archimedean $v$. 
Then $V = V'$. 
\end{theorem}
We will only show this for $n = 2$. 
To prove this, we will construct global version of Whittaker model (by glueing local Whittaker models) and prove existence and uniqueness. 

First, we will prove the result for function field, so that we can ignore archimedean places. 
Let $\psi$ be a nontrivial additive character of $\Aa/F$ and let $(\pi, v)$ be an irreducible admissible representation of $\GL(2, \Aa)$. 
\begin{definition}
Global Whittaker functional of an irreducible admissible representation $(\pi, V)$ of $\GL(2, \Aa)$ is a functional $\Lambda : V\to \Cc$ satisfying 
$$
\Lambda \left( \pi \pmat{1}{x}{}{1}v \right) = \psi(x) \Lambda(v), \quad x\in \Aa, v\in V.
$$
\end{definition}
The following theorem proves that Whittaker functional is unique and always decomposes as local Whittaker functionals. 
\begin{theorem}
\label{ffwituniq}
Let $F$ be a function field, $\Aa = \Aa_{F}$ be its ad\'ele ring and let $\pi = \otimes_{v} \pi_{v}$ be an irreducible admissible representation of $\GL(2, \Aa)$ with $K_{v} = \GL(2, \calO_{v})$-fixed vectors $\xi_{v}^{\circ}\in V_{v}$ for almost all $v$. 
If $\Lambda$ is a nonzero Whittaker functional on $V$, then for each place $v$ of $F$ there exists a Whittaker functional $\Lambda_{v}$ on $V_{v}$ such that $\Lambda_{v}(\xi_{v}^{\circ}) = 1$ for almost all $v$, and 
$$
\Lambda ( \otimes_{v}\xi_{v}) = \prod_{v} \Lambda_{v}(\xi_{v})
$$
and the dimension of Whittaker functionals on $V$ is at most one. 
\end{theorem}
\begin{proof}
Since $\Lambda$ is nonzero, there exists a nonzero pure tensor $\xi^{\circ} = \otimes_{v} \xi_{v}^{\circ}$ such that $\Lambda(\xi^{\circ})= 1$. (Note that changing finite number of $\xi_{v}^{\circ}$ does not change the restricted tensor product.) 
For each place $w$ of $F$, let $i_{w} : V_{w} \to V, \xi_{w} \mapsto \xi_{w} \otimes (\bigotimes_{v\neq w} \xi_{v}^{\circ})$ and $\Lambda_{w} = \Lambda \circ i_{w}$. 
Then $\Lambda_{w}(\xi_{w}^{\circ}) =1$ and $\Lambda_{w}$ is a Whittaker functional on $V_{w}$. 
To prove the equation, we can use induction on the cardinality of the finite set $S$ such that $\xi_{v} =\xi_{v}^{\circ}$ for all $v\not\in S$. 
Base case is trivial because both sides are 1, and to add a single place $w$ to $S$, let's assume that $\xi_{v} = \xi_{v}^{\circ}$ for all $v\not\in S\cup \{w\}$. 
Then 
$$
x_{w} \mapsto \Lambda \left( x_{w} \otimes \left( \bigotimes_{v\neq w} \xi_{v}\right)\right)
$$
is a Whittaker functional on $V_{w}$, so the uniqueness implies that there exists $c\in \Cc$ such that 
$$
\Lambda \left( x_{w} \otimes \left( \bigotimes_{v\neq w} \xi_{v}\right)\right) = c\Lambda_{w} (x_{w}).
$$
Now evaluate at $x_{w} = \xi_{w}^{\circ}$ and it gives a result for $S \cup \{w\}$. 
Uniqueness follows from the uniqueness of local Whittaker functionals. 
\end{proof}

%%%%%%%%%%%%%%%
\begin{comment}
We can think Whittaker models as functions on $\GL(2, F)$ that satisfy the similar equation. 
\begin{definition}
Let $F$ be a non-archimedean local field, $\psi$ a nontrivial additive character of $F$, and let $(\pi, V)$ be an irreducible admissible representation of $\GL(2, F)$. 
A space of functions $W:\GL(2, F) \to \Cc$ satisfying 
$$
W\left( \pmat{1}{x}{}{1} g\right) = \psi(x) W(g), \quad x\in F, g\in\GL(2, F)
$$
is called a Whittaker model for $(\pi, V)$, and denoted as $\calW_{\pi}$. 
\end{definition}

There's no big difference between Whittaker functionals and Whittaker models. 

\begin{theorem}[Local multiplicity one, Whittaker model version]
The space of Whittaker functional is isomorphic to the space of Whittaker models. In particular, $\dim \calW_{\pi} \leq 1$. 
\end{theorem}
\begin{proof}
For a given nonzero $\Lambda$, define $W_{\xi}: \GL(2, F) \to \Cc$ as 
$
W_{\xi}(g) = \Lambda(\pi(g)\xi)
$
for $\xi\in V$. Then $W_{\pi(g)\xi} = \rho(g)W_{\xi}$ (where $\rho$ is the action of $\GL(2, F)$ by right translation), so the space $\calW = \{W_{\xi}\,:\, \xi\in V\}$ is closed under right translation and isomorphic to $\pi$. 
Conversely, if $\calW$ is given with an isomorphism $\xi \mapsto W_{\xi}$ between $\pi$ and $\calW$, then $\Lambda:\xi \mapsto W_{\xi}(1)$ define a nonzero Whittaker functional. 
The last statement follows from the local multiplicity one theorem (Theorem \ref{nonarchmultone}). 
\end{proof}
\end{comment}
%%%%%%%%%%%%%%%%%

To prove global uniqueness for number fields, we have to prove uniqueness theorem for archimedean places, too. We already proved for $(\frag, K)$-modules of $\GL(2, \Rr)$ and the same statement is also  true for $\GL(2, \Cc)$. 
To unite both archimedean and non-archimedean cases, we will consider $C_{c}^{\infty}(G)$ as $\calH_{G}$-module for $G = \GL(2, F)$, where $F$ is a local field. (We've defined $\calH_G = \calH_{G_{v}}$ in the previous section.)
For such $G$, let $V$ be a simple admissible $\calH_{G}$-module and denote the action by $\pi:\calH_{G}\to \End(V)$. 

\begin{definition}
Let $(\pi, V)$ be a simple admissible $\calH_{G}$-module and $\psi$ be a fixed nontrivial character of $F$. 
Whittaker model of $(\pi, V)$ with respect to $\psi$, denoted as $\calW$, is a space of smooth functions $W: G\to \Cc$ that satisfy
$$
W\left(\pmat{1}{x}{}{1}g\right) = \psi(x)W(g), \quad x\in F, g\in G
$$
and satisfy a growth condition, i.e. the function $W(\smat{y}{}{}{1}g)$ is bounded by a polynomial in $|y|$ as $|y| \to \infty$. Also, we assume that there exists an $G$-equivariant  isomorphism $V\to \calW, \xi\mapsto W_{\xi}$ so that $W_{\pi(\phi)\xi} = \rho(\phi)W_{\xi}$. 
\end{definition}

In terms of this formulation, we can describe the uniqueness of local Whittaker models as follows. 
\begin{proposition}
Whittaker model of simple admissible $\calH_{G}$-module is unique up to isomorphism. 
\end{proposition}

Global Whittaker model is almost the same as the local definition. 
Let $F$ be a global field, let $\Sigma$ be the set of all places of $F$, and let $\calH = \calH_{\GL(2, \Aa)} = \bigotimes_{v\in \Sigma} \calH_{v}$ be the restricted tensor product of the $\calH_v$ with respect to the spherical idempotents (see the previous section). 
Let $\psi$ be a nontrivial additive character on $\Aa/F$. 
\begin{definition}
Let $(\pi, V)$ be an irreducible admissible representation of $\GL(2, \Aa)$. 
Whittaker model of $\pi$ with respect to a nontrivial character $\psi$  is a space $\calW$ of smooth $K$-finite functions on $\GL(2, \Aa)$ satisfying 
$$
W\left( \pmat{1}{x}{}{1} g\right) = \psi(x)W(g)
$$
for all $x\in \Aa, g\in G$ and are of moderate growth, i.e. $W\left( \smat{y}{}{}{1} g\right)$ is bounded by a polynomial in $|y|$ for large $y$. We assume that there is an $\calH$-module isomorphism $V \to \calW, \xi\mapsto W_{\xi}$ such that 
$$
W_{\pi(\phi)\xi} = \rho(\phi) W_{\xi}. 
$$
\end{definition}
To prove the uniqueness theorem, we need one more proposition. 
\begin{proposition}
Let $F, \psi, G, \calH_{G}$ be as above and $(\pi, V)$ be a simple admissible $\calH_{G}$-module. 
Let $\calW = \calW_{\pi}$ be a Whittaker model of $(\pi, V)$ with respect to $\psi$, and let $\xi\mapsto W_{\xi}$ be a $G$-equivariant isomorphism $V\simeq\calW$. 
Then there exists $\xi\in V$ such that $W_{\xi}(1) \neq 0$. 
If $V$ is non-archimedean and $\pi$ is spherical, and if the conductor of $\psi$ is the ring of integers of $F$, then we may take $\xi$ to be $\GL(2, \calO)$-invariant. 
\end{proposition}
\begin{proof}
Let $F$ be a non-archimedean and $0\neq W_{\xi}\in \calW$, so $W_{\xi}(g_0)\neq 0$ for some $g_0\in G$. For given $\calH_G$-action, one can prove that there exists a representation $\pi$ of $\GL(2, F)$ such that the corresponding $\calH_G$-action coincides with the previous one. If we denote the right translation action by $\rho$, then $W_{\pi(g)\xi} = \rho(g)W_{\xi}$ for $g\in \GL(2, F)$, so that $W_{\pi(g_{0})\xi} (1) = W_{\xi}(g_0)\neq 0$.
If $\pi$ is spherical and the conductor of $\psi_v$ is $\calO_v$, then by Theorem \ref{sphps}, $\pi$ is a spherical principal series representation. Then the nonvanishing of $W_{0}(1)$ follows from the explicit formula Theorem \ref{explicitsph}. 

If $F$ is archimedean, we have to be more careful.  
Since $V$ is an admissible $\calH_G$-module, it is also a $(\frag, K)$-module, and we can make use of the action $\pi : K\to \End(V)$. Since $K$ intersects every connected component of $G$, by applying $\pi(k)$ for suitable $k$ we may assume $W_{\xi}$ does not vanish identically on the connected component of the identity of $G$. 
Since $W_{\xi}$ is analytic (this is a solution of certain 2nd order differential equation), so $DW_{\xi}(1)\neq 0$ for some $D\in U\frag$. 
This equals $W_{D\xi}(1)$, so we may take $W = W_{D\xi}$. 
\end{proof}


\begin{theorem}[Global uniqueness]
Let $(\pi, V)$ be an irreducible admissible representation of $\GL(2, \Aa)$. 
Then $(\pi, V)$ has a Whittaker model $\calW$ with respect to $\psi$ if and only if each $(\pi_v, V_v)$ has a Whittaker model $\calW_v$ with respect to $\psi_v$. 
If this is the case, then $\calW$ is unique and consists of all finite linear combinations of functions of the form $W(g) = \prod_{v} W_{v}(g_{v})$, where $W_{v}\in \calW_{v}$ and $W_{v} = W_{v}^{\circ}$ for almost all $v$, where $W_{v}^{\circ}$ is the normalized spherical Whittaker function in $\calW_v$. 
\end{theorem}
\begin{proof}
First, assume that each $\pi_v$ has a Whittaker model $\calW_v$. 
By the previous proposition, if $\pi_v$ is a spherical representation and the conductor of $\psi_v$ is $\calO_v$, there exists $W_v^{\circ} \in \calW_v$ such that $W_{v}^{\circ}(k_v) = 1$ for all $k_v\in \GL(2, \calO_v)$, which is a spherical element of $\calW_v$. 
Then we can define a global Whittaker model $\calW$ as the space of all finite linear combinations of functions of the form $W_{\xi}$ where 
$$
W_{\xi}(g) = \prod_{v} W_{v, \xi_v} (g_v), \qquad g = (g_v) \in \GL(2, \Aa)
$$
where $\xi = \otimes_v \xi_v$ is a pure tensor in $V = \otimes_v V_v$ (so that $\xi_v = \xi_v^{\circ}$ for almost all $v$) and $W_{v, \xi_v^{\circ}} = W_{v}^{\circ}$ for almost all $v$. 
Then the product is well-defined because $W_{v, \xi_v}(g_v) =1$ for almost all $v$ and $\calW$ affords an irreducible admissible representation of $\GL(2, \Aa)$. 
Rapid decay of $W_{\xi}$ follows from the local results - see Theorem \ref{archwit} and \ref{pskr}.

Uniqueness proof is similar to the proof of Theorem \ref{ffwituniq}. We will show that if $\pi$ has a Whittaker model $\calW$, then it is same as the one just described. Let $\xi\mapsto W_{\xi}$ be an isomorphism $V\simeq \calW$. 

First, one can show that there exists $\xi\in V$ such that $W_{\xi}(1) \neq 0$. The argument is almost same as the proof of the previous proposition. 
So we may assume $\xi$ is a pure tensor and $W_{\xi}(1) = 1$. Let $\xi = \otimes_v \xi_v^{\circ}$ where $\xi_v^{\circ}$ is $\GL(2, \calO_v)$-fixed for almost all $v$. 

Now for each $v$, if $\xi_v\in V_v$ and $g_v\in \GL(2, F_v)$ we define 
$$
W_{v, \xi_v}(g_v) = W_{i_v(\xi_v)}(g_v)
$$
where $i_v:V_v \to V$ is the map in the proof of Theorem \ref{ffwituniq}. 
Then the space of functions $W_{v, \xi_v}$ form a Whittaker model $\calW_v$ for $\pi_v$ and $W_{v, \xi_{v}^{\circ}}(1) = 1$. 
Let $S$ be a finite set of places and let $\Aa_{S} = \prod_{v\in S} F_{v} \subset \Aa$. By induction on the size of $S$, we can prove that 
$$
W_{\xi}(g) = \prod_{v\in S} W_{v, \xi_v}(g_v)
$$
for all $g = (g_v) \in \GL(2, \Aa_S)$. Now for an arbitrary $g = (g_v) \in \GL(2, \Aa)$ and an arbitrary pure tensor $\xi = \otimes_v \xi_v \in V$, there exists a finite set $S$ of places such that if $v\not\in S$, then $v$ is non-archimedean, $\xi_v$ is $K_v$-fixed,  and $g_v\in K_v$. 
Let $h \in \GL(2, \Aa_\fin)$ be the adele such that $h_v = g_v$ for $v\in S$ and $h_v = 1$ for $v\not\in S$. 
Then 
$$
W_{\xi}(g) = W_{\xi}(h) = \prod_{v\in S} W_{v, \xi_v}(h_v) = \prod_{v} W_{v, \xi_v}(g_v), 
$$
so we get the exactly same equation as before. This proves the global uniqueness. 
\end{proof}

When $(\pi, V)$ is an automorphic cuspidal representation of $\GL(n, \Aa)$, then we can explicitly construct Whittaker models in terms of integral and prove existence. 
This is not true in general - for example, there's no Whittaker models for $\mathrm{Sp}(2n)$. 
\begin{theorem}[Global existence]
Let $(\pi, V)$ be an automorphic cuspidal representation of $\GL(2, \Aa)$, so $V\subset \calA_{0}(\GL(2, F)\bs \GL(2, \Aa), \omega)$, where $\omega$ is a character of $\Aa^{\times}/F^{\times}$. 
If $\phi\in V$ and $g\in \GL(2, \Aa)$, let 
$$
W_{\phi}(g) = \int\dpl{\Aa/F}\phi\left( \pmat{1}{x}{}{1}g\right) \psi(-x)dx. 
$$
Then the space $\calW$ of functions $W_{\phi}$ is a Whittaker model for $\pi$. 
We have the Fourier expansion 
$$
\phi(g) = \sum_{\alpha\in F^{\times}} W_{\phi} \left( \pmat{\alpha}{}{}{1}g\right).
$$
\end{theorem}
\begin{proof}
It is not hard to see that $W_{\phi}$ satisfy the transformation law and of moderate growth. 
Also, $(\frag_\infty, K_\infty)$ and $\GL(2, \Aa_{\fin})$ act on $\calW$ by right translation, and the action is compatible with the action on $V$. 
So we only need to show that the map $\phi\mapsto W_{\phi}$ is injective, and this will follow from the Fourier expansion. 

To prove the Fourier expansion, let $f:\Aa \to \Cc$ be a function 
$$
f(x) = \phi\left( \pmat{1}{x}{}{1} g\right). 
$$
Then $f$ is continuous (since $\phi$ is) and $f(x+a) = f(x)$ for all $a\in F$. 
So it can be regarded as a function on the compact group $\Aa/F$, and therefore it has a Fourier expansion as
$$
\phi\left( \pmat{1}{x}{}{1} g\right) = \sum_{\alpha\in F} C(\alpha) \psi(\alpha x)
$$
with 
$$
C(\alpha) = \int\dpl{\Aa/F} \phi\left( \pmat{1}{x}{}{1} g\right) \psi(-\alpha x)dx. 
$$
We have $C(0) = 0$ since $\phi$ is cuspidal, and for $\alpha \neq 0$, because $\phi$ is automorphic
\begin{align*}
C(\alpha) &= \int\dpl{\Aa/F} \phi\left( \pmat{\alpha}{}{}{1} \pmat{1}{x}{}{1} g\right) \psi(-\alpha x) dx \\
&= \int\dpl{\Aa/F} \phi\left( \pmat{1}{\alpha x}{}{1} \pmat{\alpha}{}{}{1} g\right) \psi(-\alpha x) dx
\end{align*}
and the change of variables $x\mapsto \alpha^{-1} x$ gives 
$$
C(\alpha) = W_{\phi}\left(\pmat{\alpha}{}{}{1} g\right). 
$$
Now put $x = 0$ and we obtain the equation. 
\end{proof}

Using the existence and uniqueness of global Whittaker model, we can prove the multiplicity one theorem for $n = 2$. 
\begin{proof}[proof of Theorem \ref{multone} when $n=2$]
First, we prove weaker form of the result (weak multiplicity one theorem), which is that if two automorphic representation $(\pi, V)$ and $(\pi, V')$ satisfies $\pi_{v}\simeq \pi'_{v}$ for \emph{all} $v$, then $V = V'$. 
By the existence theorem, if $\calW$ is the Whittaker model of $\pi$ then $V$ consists of the space of all functions $\phi$ of the form 
$$
\phi(g) = \sum_{\alpha\in F^{\times}} W\left( \pmat{\alpha}{}{}{1}g \right), \quad W\in \cal W
$$
and by the same reasoning, $V'$ consists of the same space. So we get $V = V'$. 

For the original statement, let $(\pi, V)$ and $(\pi', V')$ be cuspidal automorphic representations such that such that $\pi \simeq \otimes_{v}\pi_{v}, \pi'\simeq \otimes_{v}\pi_{v}'$ and $\pi_{v}\simeq \pi_{v}'$ for all $v\not\in S$, where $S$ is a finite set of non-archimedean places. 
Let $\calW_{v}$ and $\calW_{v}'$ be the Whittaker models of $\pi_{v}$ and $\pi_{v}'$. 
For each $v$, we choose a nonzero $W_{v}\in \calW_{v}$ so that 
\begin{itemize}
\item $W_{v}(k_{v}) = 1$ for all $k_{v}\in K_{v}$, for all but finitely many $v$, 
\item $W_{v}\smat{y}{}{}{1}$ on $F_{v}^{\times}$ is compactly supported  for $v\in S$. 
\end{itemize}
(First choice is possible because $\pi_v$ is spherical all but finitely many $v$. The second assertion follows from Theorem \ref{kicpt} - for any $\sigma \in C_{c}^{\infty}(F_{v}^{\times})$, there exists $W_v \in \calW_v$ such that $\sigma(y) = W_{v} \smat{y}{}{}{1}$.)  
Then chose $W_{v}'\in \calW_{v}'$ as follows. 
For $v\not\in S$, $\calW_{v}' = \calW_{v}$ so choose $W_{v}' = W_{v}$. 
If $v\in S$, then by the Theorem \ref{kicpt} we may arrange that 
$$
W_{v}\pmat{y}{}{}{1} = W_{v}'\pmat{y}{}{}{1}
$$
for all $y\in F_{v}^{\times}$. 
With this choices, define $\phi\in V$ by 
$$
\phi(g) = \sum_{\alpha\in F^{\times}} W\left(\pmat{\alpha}{}{}{1}g \right), 
$$
where $g = (g_{v})\in \GL(2, \Aa), W(g) = \prod_{v} W_{v}(g_{v})$, and similarly $\phi'\in V'$ from $W_{v}'$. 
Then it is enough to show that $\phi = \phi'$. 
By definition, $\phi(g) = \phi'(g)$ for all $g = \smat{y}{}{}{1}$ where $y\in \Aa^{\times}$. 
Also, $W_{v}= W_{v}'$ if $v$ is archimedean and $W, W'$ are right invariant uner some open subgroup $K_{0}$ of $\GL(2, \Aa_{\fin})$, and they are automorphic. So $\phi(g) = \phi'(g)$ for $g = \gamma\smat{y}{}{}{1} g_{\infty}k_{0}$, where $\gamma\in \GL(2, F), y\in \Aa^{\times}, g_{\infty}\in \GL(2, F_{\infty}), k_{0}\in K_{0}$, and the strong approximation concludes that $\phi = \phi'$. 
Since $W$ is nonzero, $\phi$ is nonzero and we can express $W$ in terms of $\phi$. So $V\cap V'\neq\emptyset$ and this proves $V = V'$.  
\end{proof}


\subsection{Automorphic $L$-functions for $\GL(2)$}

The multiplicity one theorem is important as itself, but it is also important because we can construct automorphic $L$-functions, which are $L$-functions attached to cuspidal automorphic representations. 
This is a generalization of Tate's thesis for  $\GL(2)$ automorphic forms. 

In the previous section, we defined local $L$-functions $L(s, \pi, \xi)$ for an irreducible admissible representation of $\GL(2, F_v)$ and a quasicharacter $\xi:F_v^{\times} \to \Cc^{\times}$. 
Using this, we can define partial $L$-function as follows. 

%%%%%%%%%%%%%%%%%%%%%%%%%%%
\begin{comment}
First, we define local $L$-functions for spherical principal series representations. Recall that if $\pi = \otimes_{v} \pi_{v}$ is a cuspidal automorphic representation of $\GL(2)$, then $\pi_{v}$ is spherical for all but finitely many $v$, so is spherical principal series representation since 1-dimensional spherical representations can't admit Whittaker models. 

\begin{definition}[Local $L$-function]
Let $F$ be a non-archimedean local field with a ring of integer $\calO = \calO_{F}$ and a uniformizer $\varpi$. Let $q$ be the cardinality of the residue field $\calO/(\varpi)$ and let $\xi$ be a unramified character of $F^{\times}$. Let $\pi = \pi(\chi_{1}, \chi_{2})$ be a spherical principal series representation of $\GL(2, F)$ associated to unramified characters $\chi_{1}, \chi_{2}$.  
Define the local $L$-function as
$$
L(s, \pi, \xi) = (1-\alpha_{1}\xi(\varpi)q^{-s})^{-1}(1-\alpha_{2}\xi(\varpi)q^{-s})^{-1}
$$
where $\alpha_{i} = \chi_{i}(\varpi)$. 
\end{definition}
By multiply those local $L$-functions, we can define (incomplete) global $L$-function. 
\end{comment}
%%%%%%%%%%%%%%%%%%%

\begin{definition}[Partial $L$-function]
Let $\pi = \otimes_{v} \pi_{v}$ be an automorphic cuspidal representation of $\GL(2, \Aa)$. 
Let $\xi = \prod_{v}\xi_{v}$ be a Hecke character. 
Let $S$ be a finite set of places contains archimedean places such that if $v\not\in S$, then $\pi_{v}$ is spherical principal series representation and $\xi_{v}$ is unramified. 
Then define the partial $L$-function as
$$
L_{S}(s, \pi, \xi) = \prod_{v\not\in S} L_{v}(s, \pi_{v}, \xi_{v}). 
$$
\end{definition}
Our aim is to prove the functional equation of $L_{S}(s, \pi, \xi)$, by using the existence and uniqueness of the global Whittaker model.  
As we did in the Tate's thesis, we first define a global zeta integral. 

\begin{definition}[Zeta integral]
Let $\phi\in V$ and let $\xi$ be a unitary Hecke character. Define the global zeta integral as
$$
Z(s, \phi, \xi) = \int\dpl{\Aa^{\times}/F^{\times}} \phi\pmat{y}{}{}{1} |y|^{s-1/2} \xi(y) d^{\times}y
$$
By the Fourier expansion, this can be written as
$$
Z(s, \phi, \xi) = \int\dpl{\Aa^{\times}} W_{\phi}\pmat{y}{}{}{1} |y|^{s-1/2} \xi(y) d^{\times}y. 
$$
Also, we define the local zeta integral as 
$$
Z_{v}(s, W_{v}, \xi_{v}) = \int\dpl{F_{v}^{\times}} W_{v}\pmat{y_{v}}{}{}{1} |y_{v}|_{v}^{s-1/2} \xi_{v}(y_{v})d^{\times}y_{v}. 
$$
Note that if $\phi$ corresponds to a pure tensor in $\otimes_{v}\pi_{v}$, then 
$$
Z(s, \phi, \xi) =\prod_{v} Z_{v}(s, W_{v}, \xi_{v}). 
$$
\end{definition}
By rapid decay of $\phi\in V$, we can show that the global zeta integral converges for \emph{any} $s$. 
Indeed, $\phi\smat{y}{}{}{1}$ is rapidly decreasing as $|y| \to \infty$, i.e. for any $N>0$ there exists a constant $C_N>0$ such that $|\phi\smat{y}{}{}{1}| < C_{N}|y|^{-N}$ for sufficiently large $|y|$. 
Also, since $\phi$ is automorphic, 
$$
\phi\pmat{y}{}{}{1} = \omega(y) \left( \pi \pmat{}{1}{1}{} \phi\right) \pmat{y^{-1}}{}{}{1}
$$
so $\phi\smat{y}{}{}{1}$ also rapidly decreases as $|y|\to 0$. 
This implies that the global zeta integral (of the first form, integral over $\Aa^{\times} / F^{\times}$) absolutely converges for any $s$. (Recall that local zeta integrals converge for sufficiently large $\Re s$.) However, the second form (integral over $\Aa^{\times}$) does not absolutely converges for all $s$, but for sufficiently large $\Re s$. 

By simple transformation, we can prove a functional equation of global zeta integral. 
\begin{proposition}[Global functional equation of zeta integral]
Let $(\pi, V)$ be an automorphic representation of $\GL(2, \Aa)$ and let $\phi \in V$. 
Let $\xi$ be a quasicharacter of $\Aa^{\times}/F^{\times}$. 
Then 
$$
Z(s, \phi, \xi) = Z(1-s, \pi(w_1)\phi, \xi^{-1} \omega^{-1})
$$
for all $s\in \Cc$, where $w_1 = \smat{}{1}{-1}{}$. 
\end{proposition}
\begin{proof}
Since $\phi$ is automorphic, we have
\begin{align*}
Z(s, \phi, \xi) &= \int\dpl{\Aa^{\times}/F^{\times}} \left( w_1 \pmat{y}{}{}{1} \right) |y|^{s-1/2} \xi(y)d^{\times} y \\
&= \int\dpl{\Aa^{\times}/F^{\times}} \phi\left( \pmat{1}{}{}{y} w_{1}\right) |y|^{s-1/2} \xi(y) d^{\times} y.
\end{align*}
If we substitute $y^{-1}$ for $y$, we obtain
$$
\int\dpl{\Aa^{\times}/F^{\times}} (\pi(w_1)\phi)\pmat{y}{}{}{1} |y|^{-s+1/2} (\xi \omega)^{-1}(y)d^{\times}y = Z(1-s, \pi(w_1)\phi, \xi^{-1}\omega^{-1}). 
$$
\end{proof}
 
Following proposition shows that the local zeta integral coincides with local $L$-functions for \emph{unramified} places. 
\begin{proposition}
Let $v$ be an unramified place, so that 
\begin{itemize}
\item $v$ is non-archimedean, 
\item $\pi_{v}$ is a spherical principal series, 
\item the conductor of $\psi_{v}$ is $\calO_{v}$, 
\item the vector $\phi_{v}$ is the spherical vector in the representation, 
\item $W_{v}(1) = 1$, 
\item $\xi_{v}$ is trivial on $\calO_{v}^{\times}$. 
\end{itemize}
(As before, this is true for all but finitely many $v$). Then for sufficiently large $\Re s$, we have $$Z_{v}(s, W_{v}, \xi_{v}) = L_{v}(s, \pi_{v}, \xi_{v}).$$ 
\end{proposition}
\begin{proof}
The proof uses explicit formula of $W_{v}$ in Theorem \ref{explicitsph} in terms of the Satake parameters $\alpha_{1}, \alpha_{2}$. Recall that 
$$
W_{v}\pmat{y}{}{}{1} = \begin{cases} q^{-m/2} \frac{\alpha_{1}^{m+1} - \alpha_{2}^{m+1}}{\alpha_{1} - \alpha_{2}} & m\geq 0 \\ 0 & \text{otherwise.} \end{cases}
$$
where $m = \ord_{v}(y)$ and $q = q_{v} = |\calO_{v}/(\varpi_{v})|$. (Here the formula is slightly different because we use the different normalization $W_{v}(1) = 1$.)
We can break the integral up into a sum over $m=0$ to $\infty$ to obtain
\begin{align*}
&\sum_{m=0}^{\infty} q^{-m/2} \frac{\alpha_{1}^{m+1} - \alpha_{2}^{m+1}}{\alpha_{1} - \alpha_{2}} q^{m/2 - ms}\xi_{v}(\varpi)^{m} \\
&= \frac{1}{\alpha_{1} - \alpha_{2}} \left( \frac{\alpha_{1}}{1-\alpha_{1}\xi_{v}(\varpi_{v})q^{-s}} - \frac{\alpha_{2}}{1-\alpha_{2}\xi_{v}(\varpi_{v})q^{-s}}\right)\\
&= L_{v}(s, \pi_{v}, \xi_{v}).
\end{align*}
\end{proof}


%%%%%%%%%%%%%%
\begin{comment}
\begin{theorem}[Local functional equation]
The local zeta integral $Z_{v}(s, W_{v}, \xi_{v})$ converges for sufficiently large $\Re s$, and has meromorphic continuation to all $s$. 
Also, there exists a meromorphic function $\gamma_{v}(s, \pi_{v}, \xi_{v}, \psi_{v})$ such that 
$$
Z_{v}(1-s, \pi_{v}(w_{1})W_{v}, \xi_{v}^{-1}\omega_{v}^{-1}) = \gamma_{v}(s, \pi_{v},\xi_{v}, \psi_{v}) Z_{v}(s, W_{v}, \xi_{v}). 
$$
\end{theorem}
\begin{proof}
non-archimedean case is proved in the previous chapter (Theorem \ref{localfe}). Archimedean case is proved in Jacquet-Langlands' book, which uses Weil representation. 
\end{proof}
\end{comment}
%%%%%%%%%%%%%%%%%

Now, we are ready to prove the global functional equation of $L$-function. 
\begin{theorem}[Global functional equation of $L$-function]
Let $\pi$ be an automorphic cuspidal represenation of $\GL(2, \Aa)$ and let $\xi$ be a Hecke character. 
Let $S$ be a finite set of places of $F$ containing all the archimedean ones such that if $v\not\in S$, then $\pi_{v}$ is spherical, $\xi_{v}$ is unramified and $\psi_{v}$ has conductor $\calO_{v}$. 
Then we have a functional equation
$$
L_{S}(s, \pi, \xi) = \left( \prod_{v\in S} \gamma_{v}(s, \pi_{v}, \xi_{v}, \psi_{v})\right) L_{S}(1-s, \wh{\pi}, \xi^{-1})
$$
where $\wh{\pi}$ is the contragredient representation of $\pi$. 
\end{theorem}
\begin{proof}
Choose a pure tensor $\phi = \otimes_{v}\phi_{v}\in V$ such that $\phi_{v}$ is spherical for $v\not\in S$ and normalized so that if $W_{v}$ is a local Whittaker function corresponding to $\phi_{v}$, then $W_{v}(1) =1$ for $v\not\in S$. 
We will evaluate 
$$
\left( \prod_{v\in S} Z_{v}(s, W_{v}, \xi_{v})^{-1} \right) Z(s, \phi, \xi)
$$ 
in two different ways. 
First, it is easy to check that this equals the LHS of the above functional equation for large $\Re s$. 
Now, take $-\Re s$ to be large and positive. 
Local functional equation allow us to write above equation as
\begin{align*}
&\left(\prod_{v\in S} Z_{v}(s, W_{v}, \xi_{v})^{-1}Z_{v}(1-s, \pi(w_{1})W_{v}, \xi_{v}^{-1}\omega_{v}^{-1})\right) \\
&\times \prod_{v\not\in S} Z_{v}(1-s, \pi(w_{1})W_{v}, \xi_{v}^{-1}\omega_{v}^{-1}). 
\end{align*}
Thus we will obtain the RHS if we show 
$$
Z_{v}(s, \pi(w_{1})W_{v}, \xi_{v}^{-1}\omega_{v}^{-1}) = L_{v}(s, \wh{\pi}_{v}, \xi_{v}^{-1})
$$
for $v\not\in S$. 
Since $v$ is unramified, $W_{v}$ is the spherical vector and 
\begin{align*}
Z_{v}(s, \pi(w_{1})W_{v}, \xi_{v}^{-1}\omega_{v}^{-1}) &= Z_{v}(s, W_{v}, \xi_{v}^{-1}\omega_{v}^{-1})\\
&= L_{v}(s, \pi_{v}, \omega_{v}^{-1}\xi_{v}^{-1}) \\
&= L_{v}(s, \wh{\pi}_{v}, \xi_{v}^{-1}).
\end{align*}
The last equality follows from $\wh{\pi}_{v} \simeq \omega_{v}^{-1} \otimes  \pi_{v}$, or by direct calculation (if $\alpha_1, \alpha_2$ are Satake parameters of $\pi_v$, then $\alpha_1^{-1}, \alpha_{2}^{-1}$ are Satake parameters of $\wh{\pi}_v$). 
\end{proof}
To complete the $L$-function, we want to define local $L$-functions for ramified places $v$. 
For any place $v$, we should have $Z_{v}(s, W_{v}, \xi_{v}) / L(s, \pi_v, \xi_v)$ holomorphic for all $W_v$, and if we define 
$$
\epsilon_{v}(s, \pi_v, \xi_v, \psi_v) = \frac{\gamma_v(s, \pi_v, \chi_v, \psi_v)L_v(s, \pi_v, \chi_v)}{L_v(1-s, \wh{\pi}_v, \xi_{v}^{-1})}
$$
then $\epsilon_v(s, \pi_v, \xi_v, \psi_v)$ is a function of exponential type. 

When $\pi_v = \pi(\chi_1, \chi_2)$ is a principal series representation, we define 
$$
L_{v}(s, \pi_v, \xi_v) := L(s, \xi_v \chi_1)L(s, \xi_v\chi_2)
$$
Where the $L$-factors on the RHS are as defined in section 4.1. In this case, we have
$$
\epsilon_v(s, \pi_V, \xi_v, \psi_v) = \epsilon(s, \xi_v \chi_1, \psi_v) \epsilon_v(s, \xi_v \chi_2, \psi_v).
$$
If $v$ is non-archimedean and $\pi_v= \sigma_v(\chi_1, \chi_2)$ is a special representation (so that $\chi_1\chi_2^{-1}(x) = |x|$), then we have
$$
L_{v}(s, \pi_v, \xi_v) = L(s, \xi_v \chi_1)
$$
and
$$
\epsilon(s, \pi_v, \xi_V, \psi_v) = \epsilon(s, \xi_v \chi_1,\psi_v)\epsilon(s, \xi_v \chi_2, \psi_v) \frac{L(1-s, \xi_{v}^{-1}\chi_1^{-1})}{L(s, \chi_1\chi_2)}.
$$
For other cases (such as  supercuspidal representation on non-archimedean place), we put $L_{v}(s, \pi_v, \xi_v) = 1$. In these cases, we have $\epsilon_{v}(s, \pi_v, \xi_v, \chi_v) = \gamma_v(s, \pi_v, \xi_v, \psi_v)$. 





\subsection{Adelization of classical automorphic forms}

In this last section, we will study how to get automorphic representations from classical automorphic forms such as  modular forms and Maass forms. 
For a given modular form, we can lift a function as a function on the ad\'ele group $\GL(2, \Aa)$, then consider a $(\frag, K)$-module generated by the function. 
This is an irreducible admissible representation of $\GL(2, \Aa)$ by Theorem \ref{autoadm}. We will describe this procedure more rigorously and prove that the associated representation is automorphic when the original modular form is a Hecke eigenfunction. 

Let $f:\calH \to \Cc$ be a modular form or Maass form of weight $k$ on $\Gamma_0(N)$ with a character $\chi:\Gamma_0(N) \to \Cc^{\times}$. 
We already saw that the function $F:\GL(2, \Rr)^{+}\to \Cc$ defined as $F(g) = (f||_{k}g)(i)$ is of moderate growth, an eigenfunction of $\Delta$, and 
$$
F(\gamma g \kappa_\theta) = \chi(d) e^{ik\theta}F(g)
$$
for $\gamma = \smat{a}{b}{c}{d} \in \Gamma_{0}(N), \kappa_{\theta} = \smat{\cos \theta}{\sin \theta}{-\sin \theta}{\cos\theta} \in \SO(2)$. 
Here $\chi$ is a Dirichlet character modulo $N$ (not necessarily primitive). 

To ad\'elize $F$ as a function $\phi$ on $\GL(2, \Aa)$, we also need to ad\'elize $\chi$ as a character of $K_{0}(N) = \{g  = (g_{v}) \in K_{\fin}\,:\, c_{v}\in N\calO_{v}\}$. 
In Proposition \ref{dirad}, we saw that there's 1-1 correspondence between Dirichlet characters and characters of $\Aa^{\times}/\Qq^{\times}$. 
So we have a character $\omega = \prod_{v} \omega_{v}$ of $\Aa^{\times}/\Qq^{\times}$ corresponds to $\chi$, so that $\chi(p) = \omega_{v}(\varpi_{v})$ for $p\nmid N$ and $v = p$. 
Also, $\omega_{v}$ is unramified for $v\nmid N$ and $\omega_{v}$ is trivial on the subgroup of $\calO_{v}^{\times}$ consisting of elements $\equiv 1\Mod{N}$. 
$\omega_{\infty}$ is trivial on $\Rr_{+}^{\times}$. 

Now we define a character $\lambda$ of $K_{0}(N)$ by 
$$
\lambda\pmat{a}{b}{c}{d} = \prod_{v\in S_{\fin}(N)} \omega_{v}(d_{v})
$$
where $S_{\fin}(N)$ is a set of non-archimedean places dividing $N$. By strong approximation theorem, we can write any $g\in \GL(2, \Aa)$ by $g = \gamma g_{\infty} k_0$ with $\gamma \in \GL(2, \Qq), g_{\infty} \in \GL(2, \Rr)^{+}, k_{0}\in K_{0}(N)$. 
Then we define 
$$
\phi(g) = \phi_{F}(g) = F(g_{\infty}) \lambda(k_{0})
$$
as associated function on $\GL(2, \Aa)$. This function is well-defined: this follows from the equation
$$
\chi(d) = \prod_{v\in S_{\fin}(N)} \omega_{v}^{-1}(d_{v})
$$
for $d\in \Zz$ coprime to $N$. This $\phi$ is an automorphic form with a central quasicharacter $\omega$, which can be shown by using 
$$
\Aa^{\times}= \Qq^{\times} \Rr_{+}^{\times} \prod_{p<\infty} \Zz_{p}^{\times}. 
$$

We can also extend classical Hecke operators to ad\'elic setting. For each prime $p\nmid N$, the corresponding ad\'elized Hecke operator $\Tt_{p}$ will be the operator in the local spherical Hecke algebra $\calH_{p}=\calH_{K_p}$, which is defined in Section 3.8. 
 
First, we define Hecke operator for automorphic forms on $\GL(2, \Rr)^{+}$. 
Let $\Sigma = \{ p\,:\, p|N\}$ and let $\Zz_{\Sigma}$ be the localization of $\Zz$ at the prime in $\Sigma$, so that $r/s\in \Zz_{\Sigma}$ iff $N\nmid s$. 
We can trivially extend the Dirichlet character $\chi$ to $\Zz_{\Sigma}$. If we put $G_{0}(N) := \{\smat{a}{b}{c}{d}\in \GL(2, \Zz_{\Sigma})\,:\, c\in N\Zz_{\Sigma}\}$, then we have a right action of $G_{0}(N)$ on functions on $\GL(2, \Rr)^{+}$ by 
$$
(F|_{\chi}\alpha)(g) = \chi(d)^{-1}F(\alpha g), \qquad \alpha = \pmat{a}{b}{c}{d} \in G_{0}(N). 
$$
Then for $\xi \in G_{0}(N)$, we can define the Hecke operator $T_{\xi}$ as 
$$
T_{\xi}F = \sum_{i=1}^{h} F|_{\chi}\xi_i
$$
where $\{\xi_{1}, \dots, \xi_{h}\}$ is a complete set of coset representatives for $\Gamma_{0}(N) \bs \Gamma_{0}(N) \xi \Gamma_{0}(N)$. Especially we put $T_{p} := T_{\xi_p}$ for $\xi_{p} = \smat{p}{}{}{1}$.  
Now, the following theorem shows that every Hecke eigenform gives rise to an automorphic representation. 
\begin{theorem}
Suppose that $F$ is an eigenfunction of all the Hecke operators $T_{p}$ when $p\nmid N$. Then $\phi$ lies in an irreducible subspace of $L_{0}^{2}(\GL(2, F) \bs \GL(2, \Aa), \omega)$. Hence, by Theorem \ref{l2autoad}, the space generated by $\phi$ induces an irreducible automorphic representation. 
\end{theorem}
\begin{proof}
By Theorem \ref{cuspdecomad}, $L^{2}_{0}$ decomposes as Hilbert space direct sum of irreducible invariant subspaces. Now choose $(\pi, V) \subseteq L_{0}^{2}$ such that the projection of $\phi$ to $V$ is nonzero. We will show that $\pi$ is uniquely determined by eigenvalues of $T_{p}$ with $p\nmid N$ and $\omega$. 
This will show that $\phi \in \pi$. Note that $\phi$ is $K_{v} = \GL(2, \Zz_{p})$-fixed for $p\nmid N$ since $\lambda$ is trivial on that group. 

For $p\nmid N$, let $G_{p} = \GL(2, \Qq_{p})$, $\calH_{p} =C_{c}^{\infty}(G_p)$ be the Hecke algebra and $\calH_{p}^{\circ} = C_{c}^{\infty}(K_{p} \bs G_{p} / K_p)$ be the spherical Hecke algebra. 
We studied the structure of spherical Hecke algebra in Section 3.7 - $\calH_p^{\circ}$ is commutative and generated by three elements $T(\frap), R(\frap),$ and $R(\frap)^{-1}$. (See Theorem \ref{nonarchsphcom}, Proposition \ref{heckerel} and Proposition \ref{heckegen}.) 
We will use new notations $\Tt_{p} := T(\frap)$ and $\Rr_{p} := R(\frap)$ in here to avoid confusion between classical and ad\'elized Hecke operators. 
So we have
$$
\Tt_{p} := \chf_{K_{p} \smat{\varpi_p}{}{}{1} K_{p}}, \quad \Rr_{p} := \chf_{K_{p} \smat{\varpi_{p}}{}{}{\varpi_{p}}}, \quad \Rr_{p} := \chf_{K_{p} \smat{\varpi_{p}}{}{}{\varpi_p}^{-1}}
$$
where $\varpi_{p}$ is the idele whose $p$th component is $p$ and all of whose other components are 1. 
As before, we can decompose the double coset as
$$
K_{p} \pmat{\varpi_p}{}{}{1} K_{p} = \bigcup_{i=1}^{p+1} i_{p}(\xi_{i})K_p
$$ 
where $i_{p}: \GL(2, \Qq) \to \GL(2, \Aa)$ is the map induced by the composition $\Qq\hookrightarrow \Qq_p \hookrightarrow \Aa$ and 
$$
\xi_{i} = \pmat{p}{i}{}{1} \quad (1\leq i\leq p), \quad \xi_{p+1} = \pmat{1}{}{}{p}.
$$
Also, we have an action $\rho$ of $\calH_{p}$ on automorphic forms given by 
$$
(\rho(\sigma)\phi)(g) = \int\dpl{\GL(2, \Aa)} \sigma(h)\phi(gh)dh, \quad \sigma\in \calH_p
$$
and we will denote $\rho(\Tt_{p})\phi$ as $\Tt_{p}(\phi)$. We will evaluate $(\Tt_{p}\phi)(g)$ when $g = \gamma g_{\infty} k_{0}$.  
Since $\phi$ is right $K_{p}$-invariant, we have
$$
(\Tt_{p}\phi)(g) = \sum_{i=1}^{p+1} \phi(gi_{p}(\xi_{i})).
$$
For each $1\leq i\leq p+1$ and $k_{0} \in K_{0}(N)$, there exists $1\leq j\leq p+1$ and $k_{0}'\in K_{0}(N)$ such that $k_{0}i_{p}(\xi_{i}) = i_{p}(\xi_{j})k_{0}'$. If we write $\xi_{j} = \xi_{j, \infty}\xi_{j, \fin}$, then 
$$
gi_{p}(\xi_{i}) = (\gamma\xi_{j}) (\xi_{j, \infty}^{-1}g_{\infty}) (\xi_{j, \fin}^{-1}i_{p}(\xi_{j})k_{0}')
$$
where each component lies in $\GL(2, \Qq), \GL(2, \Rr)^{+}$ and $K_{0}(N)$. (Note that $\xi_{j}$ is considered as an element of $\GL(2, \Aa)$ via diagonal map $\GL(2, \Qq)\hookrightarrow \GL(2, \Aa)$, and the $p$th component of $\xi_{j, \fin}^{-1}i_{p}(\xi_j)$ is 1.)
Then 
$$\phi(g i_{p}(\xi_{i})) = F(\xi_{j, \infty}^{-1}g_{\infty}) \lambda(\xi_{j, \fin}^{-1} i_{p}(\xi_j) k_{0}').$$
We know that $\lambda$ is determined by places $v|N$ and for each $v$, the $v$th component of $i_{p}(\xi_{j})k_{0}'$ is the same as the $v$th component of $k_{0}'$ or $k_{0}$. 
Thus we get
$$
\phi(gi_{p}(\xi_{i})) = (F|_{\chi}\xi_{j, \infty}^{-1})(g_{\infty})\lambda(k_{0})
$$
and so if $F$ is an eigenfunction of the classical Hecke operator $T_{p} = T_{\smat{p}{}{}{1}}$, then $\phi$ is an eigenfunction of $\Tt_p$ with the same eigenvalue. 
Similarly, we have
$$
(\Rr_{p}\phi)(g) = \int\dpl{K_{p}\smat{\varpi_{p}}{}{}{\varpi_{p}}} \phi(gh)dh = \phi(g) \omega(\varpi_{p}) = \chi(p)\phi(g)
$$
so $\phi$ is an eigenfunction of $\Rr_{p}$ with eigenvalue $\chi(p)$. 
To summarize, $\phi$ is an eigenvector of $\calH_{p}$, and the eigenvalues are determined by $\chi$ and the eigenvalues of the classical Hecke operators on $F$. 

The projection of $L_{0}^{2}$ onto the invariant subspace $V$ is $\GL(2, \Aa)$-equivariant, so the image of $\phi$ in $V$ is an eigenvector of $\calH_p$ with the same eigenvalues. 
Now Theorem \ref{sphchar} tells us that this determines the irreducible constituent $\pi_{p}$ of $\pi$, and so $\pi$ is itself determined by the global multiplicity one theorem. 
\end{proof}

We can also obtain theorems for classical automorphic forms by ad\'elizing it and use tools that we proved for automorphic representations. 
For example, the global multiplicity one theorem implies that if we know almost all Fourier coefficients of a modular form, then it is uniquely determined. This theorem also holds for Maass wave forms. 
\begin{theorem}[Multiplicity one for modular forms]
Let $$f(z)= \sum_{n\geq 1} a_{n}q^{n}, \quad g(z) = \sum_{n\geq 1} b_{n}q^{n}$$ be holomorphic cusp forms of weight $k$ on $\SL(2, \Zz)$ which are normalized $(a_{1} = b_{1} = 1)$ Hecke eigenforms. 
Assume that $a_{p} = b_{p}$ for all but finitely many $p$. 
Then $f(z) = g(z)$. 
\end{theorem}
\begin{proof}
By ad\'elizing it, we obtain two automorphic representations $\pi_{f}, \pi_{g}$ of $\GL(2, \Aa)$. 
Since these are Hecke eigenforms, we have $T_{p}f = a_{p}f$ and $T_{p}g = b_{p}f$ for all prime $p$, and $a_{p}$, $b_{p}$ are also eigenvalues of $\Tt_{p} \in \calH_{p}$ for each components of representations $\pi_{f, p}$, $\pi_{g, p}$. Since $\Rr_{p}$ acts trivially, Theorem \ref{sphchar} implies $\pi_{f, p} \simeq \pi_{g, p}$ for any $p$ with $a_{p}= b_{p}$. Now the multiplicity one gives us $\pi_{f} = \pi_{g}$, which is $a_{p}= b_{p}$ for all $p$. So we get $f = g$. 
\end{proof}
Note that this is not true for general congruence groups $\Gamma_{0}(N)$ with characters - we also need to assume that $f, g$ are \emph{newforms}. 



\begin{thebibliography}{5}

\bibitem{bu} Bump, Daniel. \emph{Automorphic forms and representations}. Vol. 55. Cambridge University Press, 1998.


\end{thebibliography}
\end{document}