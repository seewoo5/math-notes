\newpage
\section{Introduction}

Modular forms and Maass wave forms are certain functions defined on the complex upper half plane that satisfies $\SL(2, \Zz)$-transformations laws (or more generally, transform under congruence subgroups $\Gamma_{0}(N)$). 
There are a lot of applications of modular forms in number theory, such as sum of squares and the irrationality of $\zeta(3)$, and the Wiles' famous proof of Fermat's Last Theorem. 
There are also applications in other subjects, such as combinatorics (partition numbers), physics, representation theory (monstrous moonshine), knot theory, etc. 
 
In this note, we will study how to interpret such functions (so-called classical automorphic forms) as a representation of ad\'ele groups $\GL(2, \Aa)$ (here $\Aa$ is a ring of ad\'eles of global fields such as $\Qq$), and study representation theory of it. 
This can be a starting point of the \emph{Langlands' Program}, which connects representation of Galois groups, algebraic geometry, and automorphic forms (representations). 

To study such representations, we first study local representations. 
There are two kinds of local representations - archimedean and non-archimedean. 
For the archimedean cases, we study representation theory of $\GL(2, \Rr)$  via so-called ($\frag, K$)-modules. $(\frag, K)$-module is a vector space with compatible $\frag_{\Cc} = \mathfrak{gl}(2, \Cc)$ and $K = \rO(2)$-actions. It is easier to study $(\frag, K)$-modules than studying the representation of $\GL(2, \Rr)$ directly since $(\frag, K)$-modules are more \emph{algebraic}. We will classify $(\frag, K)$-modules for $\GL(2, \Rr)$ and also study which of them are unitarizable, since we are interested in the representation that lives in $L^{2}$ space. Also, we will see how these representations are related to classical automorphic forms (such as modular forms and Maass wave forms). 

We also have non-archimedean representations - which are representation of $p$-adic groups $\GL(2, \Qq_{p})$ for a prime $p$. They are very different from archimedean cases because of their topology. This makes the situation easier or harder, but anyway, we will also classify all the representations of such groups and study their unitarizability. 

When we finish the local theories, we can \emph{glue} these representations to obtain the representation of the ad\'ele group $\GL(2, \Aa)$. (In fact, this is not a true representation of $\GL(2, \Aa)$, but a representation of $(\frag_{\infty}, K_{\infty}) \times \GL(2, \Aa_{\fin})$.) 
While we are studying such representations (local or global), we will only concentrate on some \emph{nice} representations (\emph{admissible} representations) that are close to the representation of finite groups. 
\emph{Automorphic} representations are some nice representations that also satisfies some analytic conditions on growth. 
Later, we will see that Flath's decomposition theorem tells us that it is enough to study such glued representations to study automorphic representations. 

Before we get into the representation theory of $\GL(2, \Aa)$, we will study $\GL(1, \Aa)$ first, which are  completed by Tate in his celebrated thesis. He find a natural way to prove the analytic continuation and the functional equation of Hecke's $L$-function using local-global principle, and such idea will be used to define $L$-functions attached to automorphic representations of $\GL(2,\Aa)$. 

It may be hard to study an abstract representation of a given group (such as $\GL(2, \Rr), \GL(2, \Qq_{p})$ or $\GL(2, \Aa)$). 
Whittaker model (or Whittaker functional) help us to study such representations as a very concrete representation that functions on the group lives (and the group acts as a right translation). 
Most case, such Whittaker model exist and unique, and such results are called (local or global) multiplicity one theorem. 
In the last section, we will see how the multiplicity one theorem is related to the classical modular forms. 

%%%%%%%%%%%%%%%%%%%%%%%%%%%%