\section{General formulation of Murmuration}
\label{sec:general}

In his letter to Sutherland and Zubrilina, Sarnak proposed a general framework of murmuration \cite{sarnak} for \emph{families} of $L$-functions.
We explain his definition and explain its connection with Katz--Sarnak philosophy \cite{katz1999zeroes,katz2023random}.
Also, we revisit the previous murmuration results \cite{zubrilina2025murmurations,lee2025murmurations,sawin2025murmurations} in this framework.
See also Lowry-Duda's survey note \cite{lowry2025murmurations}.

\subsection{Murmuration for families}

Let $\scF$ be a family of $L$-functions in a suitable sense (e.g. See \cite{sarnak2016families}).
For a smooth nonnegative function $\Phi : (0, \infty) \to \bR$ with compact support and $f : \scF \to \bC$, consider the $\Phi$-weighted average of $f$:
\begin{equation}
    \bE_{\pi \in \scF}[f; \Phi, N] := \frac{\sum_{\pi \in \scF} \Phi\left(\frac{N_\pi}{N}\right)f(\pi)}{\sum_{\pi \in \scF} \Phi\left(\frac{N_\pi}{N}\right)} = \frac{A_{\scF}(f; \Phi, N)}{A_{\scF}(1; \Phi, N)}
\end{equation}
where
\begin{equation}
    A_{\scF}(f; \Phi, N) := \sum_{\pi \in \scF} \Phi\left(\frac{N_\pi}{N}\right)f(\pi).
\end{equation}
Here $N_\pi$ is the ``conductor'' of $\pi$ (e.g. conductor of an elliptic curve or analytic conductor of an automorphic form).
When we order the family by the conductor, we say that $\scF$ has \emph{conductor dimension $\delta$} if
\begin{equation}
    \# \{\pi \in \scF : N_\pi \le N\} \sim \alpha N^\delta
\end{equation}
as $N \to \infty$ for some $\alpha > 0$ and $\delta = \delta(\scF) > 0$.
For such family, we have
\[
A_\scF(1; \Phi, N) \sim \alpha \delta N^\delta \int_0^\infty \Phi(x) x^\delta \frac{\dd x}{x}.
\]

Most of the known murmuration results consider the function
\begin{equation}
    f(\pi) = a_\pi(p) := \sqrt{p} \lambda_\pi(p)
\end{equation}
for a given prime $p$, where $\lambda_\pi(p)$ is the normalized trace of Frobenius at $p$ so that the Ramanujan--Petersson conjecture says $|\lambda_\pi(p)| \le n$ for $\GL_n$ automorphic forms $\pi$.
Furthermore, if $\scF$ is self-dual, then $a_\pi(p)$ are real and the global root number $w_\pi$ is either $1$ or $-1$.
Then we can separate by root number and consider the averages
\begin{equation}
    \bE_{\pi \in \scF^w} [a_\pi(p); \Phi, N]
\end{equation}
for $w \in \{\pm 1\}$ and $\scF^w = \{\pi \in \scF : w_\pi = w\}$.

When $\pi$ is self-dual, functional equation relates $L(s, \pi)$ and $L(1-s, \pi)$ and the completed $L$-function $\Lambda(s, \pi \times \pi)$ of $\pi \times \pi$ factors as
\[
\Lambda(s, \pi \times \pi) = \Lambda(s, \pi, \Sym^2) \Lambda(s, \pi, \wedge^2).
\]
$\pi$ is said to be \emph{orthogonal} if the first factor $\Lambda(s, \pi, \Sym^2)$ has a pole at $s = 1$, and \emph{symplectic} if the second factor $\Lambda(s, \pi, \wedge^2)$ has a pole at $s = 1$.
The symplectic case occur only if $n$ is even, and root number of orthogonal $\pi$ is always $1$.


\begin{definition}
A continuous function $M_\Phi : (0, \infty) \to \bR$ is a \emph{murmuration function for $\scF$ with weight $\Phi$} if there is $0 \le \gamma < 1$ such that for $P \sim N$
\begin{equation}
    \label{eqn:sarnak_murmuration}
    \bE_{\pi \in \scF} [a_\pi(p);\Phi, N] = M_\Phi\left(\frac{p}{N}\right) + R(p, N)
\end{equation}
where the local oscillating term $R(p, N)$ satisfies
\begin{equation}
    \label{eqn:sarnak_local_avg}
    \bE_{P - H \le p \le P + H} [R(p, N)] = o(1)
\end{equation}
for $N^\gamma \le H = o(N)$.
\end{definition}
\eqref{eqn:sarnak_local_avg} is the local averaging over $p$ of length at least $N^\gamma$ and less than $N$.
The smaller $\gamma$ we can take, the more visible $M_\Phi$ is.
In particular, $\gamma = 0$ means that no local averaging is needed.
He conjectured that if
\begin{equation}
    \label{eqn:sarnak_local_thres}
    \delta + \gamma > 1
\end{equation}
then the local oscillating term $R(p, N)$ will vanish as $N \to \infty$.
In particular, we may not need local averaging if $\delta > 1$.

The function $M_\Phi$ is linear in $\Phi$, and it supposed to have a form of
\begin{equation}
    \label{eqn:sarnak_zubrilina_density}
    M_\Phi(y) = \int_0^{\infty} \Phi(u) M\left(\frac{y}{u}\right) u^\delta \frac{\dd u}{u}
\end{equation}
for some universal $M : (0, \infty) \to \bR$.
If such function exists, we call it as \emph{Zubrilina murmuration density for $\scF$}, denoted as $Z_{\scF}$.
It might be a distribution on $C_c^\infty((0, \infty))$ rather than a function.



\subsection{Katz--Sarnak philosophy}


Katz and Sarnak \cite{katz1999zeroes,katz2023random} studied statistics of zeros of $L$-functions via random matrix models.
In particular, they considered \emph{one-level density} of low-lying zeros: for an even function $\phi$ with rapid decay as $|x| \to \infty$, the one-level density of a family $\scF$ is
\begin{equation}
    \OLD(\scF; \phi) = \lim_{N \to \infty} \bE_{\pi \in \scF(N)} \left[\sum_{\gamma_\pi} \phi\left(\frac{\gamma_\pi \log N}{2 \pi}\right)\right]
\end{equation}
where $\scF(N) := \{\pi \in \scF: N_\pi = N\}$ and $\gamma_\pi$ runs through the ordinates of nontrivial zeros of $L(s, \pi)$ on the critical line, i.e. $L\left(\frac{1}{2} + i\gamma_\pi, \pi\right) = 0$.
The factor $\frac{\log N}{2\pi}$ guarantees that the nontrivial zeros have unit spacing on average.
Katz--Sarnak philosophy claims that there is a measure $W_\scF$ coming from matrices related to the ``type'' of $\scF$ such that
\begin{equation}
    \OLD(\scF; \phi) = \int_{\bR} \what{\phi}(x) \what{W_\scF}(x) \dd x
\end{equation}
for any nice text function $\phi$.

One such example is the following theorem on the family of Hecke eigenforms by Iwaniec, Luo, and Sarnak \cite{iwaniec2000low}.
\begin{theorem}[Iwaniec--Luo--Sarnak \cite{iwaniec2000low}]
    Assume GRH. Let $\phi$ be an even Schwartz function with $\supp(\what{\phi}) \subset (-2, 2)$.
    Let $H_k^\pm$ be a set of Hecke eigenforms of weight $k$ and root number $\epsilon = \pm 1$.
    Then
    \begin{equation}
        \OLD(H_k^\pm; \phi) = \int_{\bR} \what{\phi}(x) \what{W_{\SO(\pm)}}(x) \dd x
    \end{equation}
    where
    \begin{equation}
        \label{eqn:WSO}
        W_{\SO(+)}(x) = 1 + \frac{\sin(2\pi x)}{2 \pi x}, \quad W_{\SO(-)}(x) = 1 - \frac{\sin(2\pi x)}{2 \pi x} + \delta_0(x).
    \end{equation}
\end{theorem}
There is an explicit formula relates the summation over zeros of $L$-functions and over primes, which is given by \cite[Section 4]{iwaniec2000low}
\begin{align*}
    \sum_{\gamma_\pi} \phi\left(\frac{\gamma_\pi \log N}{2\pi}\right) &= C - 2 \sum_p \sum_{\nu \ge 1} \left(\sum_{j} \alpha_j(p)^\nu\right) \what{\phi}\left(\frac{\nu \log p}{\log N}\right) \frac{\log p}{p^{\nu/2} \log N}
\end{align*}
and since $\what{\phi}$ is compactly supported, the main contribution comes from $\nu = 1$ summand, so we are mostly interested in 
\begin{equation}
    \sum_{p} \frac{\lambda_\pi(p)}{p^{1/2}} \what{\phi}\left(\frac{\log p}{\log N}\right) \frac{\log p}{\log N} 
\end{equation}
The Fourier transforms of \eqref{eqn:WSO} are
\begin{equation}
    \what{W_{\SO(+)}}(y) = \delta_0(y) + \frac{2 - \1_{[-1, 1]}(y)}{2}, \quad \what{W_{\SO(-)}}(y) = \delta_0(y) + \frac{\1_{[-1, 1]}(y)}{2}
\end{equation}
and there are obvious discontinuities at $y = \pm 1$.


\subsection{Revisiting the murmuration theorems}

Let's see how the previous works \cite{zubrilina2025murmurations,lee2025murmurations,sawin2025murmurations} fit into the above framework.

\subsubsection{Dirichlet characters}

In case of Dirichlet characters \cite{lee2025murmurations}, the family of even/odd \emph{complex} Dirichlet characters are not self-dual, so it does not fit into Sarnak's framework to be precise.
Still, we can observe and prove murmuration phenomena for such family, where the density function is much simpler compare to other families.
Also, for each prime $P$, there are $\phi(P) = P-1$ primitive characters modulo $P$, so there are about $\frac{CN^2}{\log N}$ of primitive Dirichlet characters of prime conductors $\le N$ for some constant $C > 0$, which means that the conductor dimension of the family is $2-\epsilon$\footnote{The $\log N$ factor disappears if we consider all conductors that are not necessarily prime, and the conductor dimension becomes $2$. See \cite[Section 6]{lee2025murmurations} for the version including composite conductors.}.
In particular, \eqref{eqn:sarnak_local_thres} suggests that we don't need any further local averaging for the family, which is indeed the case of Theorem \ref{thm:lop_dirichlet}.
Note that normalization used in Theorem \ref{thm:lop_dirichlet} is also different from the one Sarnak suggested.

For the quadratic characters, local averaging is included in Theorem \ref{thm:lop_dirichlet_quad} over $X^{\gamma}$-many primes with $\frac{3}{4} < \gamma < 1$.
Since the number of square-free odd numbers up to $X$ is $\Theta(X)$, the conductor dimension of the family $\scF = \{\chi_{8d} : d\text{ odd and square-free}\}$ is $1$ so we may only need local averaging over $X^{\epsilon}$-many primes for any $\epsilon > 0$ from \eqref{eqn:sarnak_local_thres}.

\subsubsection{Modular forms}

It is known that the dimension of the space of newforms of fixed weight $k$ and level $\Gamma_0(N)$ is asymptotically $\frac{45(k-1)N}{\pi^2}$ \cite[Theorem 8]{martin2005dimensions}.
It implies that the number of newforms of fixed weight $k$ and level $\Gamma_0(M)$ for square-free $M \le N$ is asymptotically $c_k N^2$ for a constant $c_k > 0$, so the conductor dimension of the family is $2 > 1$.
From Sarnak's suggestion \eqref{eqn:sarnak_local_thres}, we may not need any local averaging, and this is indeed the case of Theorem \ref{thm:zubrilina_modform} and \ref{thm:zubrilina_geom}.

Also, she studied several properties of her density function $\cM_k$.
In particular, for a compactly supported smooth function $\Phi$ on $(0, \infty)$, she proved that the function
\begin{equation}
    \cM_{\Phi}^k(y) := \frac{\int_0^\infty \cM_k\left(\frac{y}{u}\right) \Phi(u) u^2 \frac{\dd u}{u}}{\int_0^\infty \Phi(u) u^2 \frac{\dd u}{u}}
\end{equation}
is continuous in $y$, $\lim_{y \to 0^+} \cM_{\Phi}^k(y) = 0$ and
\[
\cM_k(y) = \frac{1}{2} + o_k(1)
\]
as $y \to \infty$.

\subsubsection{Elliptic curves (ordered by heights)}

The conductor dimension of a family of elliptic curves ordered by heights is $\frac{5}{6}$, so we may need further local averaging over $X^{1/6 + \epsilon}$ many primes.
Sawin and Sutherland introduced local averaging in Theorem \ref{thm:sawin-sutherland}, but it is slightly different from Sarnak's suggestion, since they take local average over $\Theta(N_E)$-many primes, not $O(H_E^{1/6 + \epsilon})$.
\ber Cowan \cite{cowan2024murmurations}\er