\section{Other known cases}

After the success of Zubrilina, a lot of people are interested in murmuration density for different objects in number theory.
We list the known works here.

\subsection{Flying Dirichlet characters}

Lee, Oliver, and Pozdnyakov computed murmuration density for Dirichlet characters \cite{lee2025murmurations}\footnote{which can be thought as  automorphic forms on $\GL_1$ over $\bQ$.}.
For complex characters, the corresponding murmuration densities are given by the following theorem.

\begin{theorem}[{Lee--Oliver--Pozdnyakov \cite[Theorem 1.1]{lee2025murmurations}}]
    Let $\cD_{+}(N)$ (resp. $\cD_{-}(N)$) denote the set of primitive even (resp. odd) Dirichlet characters modulo $N$.
    For $x \in \bR_{> 0}$, let $\lceil x \rceil^{\frp}$ be the smallest prime $\ge x$.
    For $c > 1$, $\delta > 0$, and $y > 0$, define
    \begin{align*}
        P_{\pm}(y, X, c) &:= \frac{\log X}{X} \sum_{\substack{N \in [X, cX] \\ N \text{ prime}}} \sum_{\chi \in \cD_{\pm}(N)} \frac{\chi(\lceil yX \rceil^{\frp})}{\tau(\chi)} \\
        P_{\pm}(y, X, \delta) &:= \frac{\log X}{X^\delta} \sum_{\substack{N \in [X, X + X^\delta] \\ N \text{ prime}}} \sum_{\chi \in \cD_{\pm}(N)} \frac{\chi(\lceil yX \rceil^{\frp})}{\tau(\chi)}.
    \end{align*}
    Then
    \begin{equation}
        \label{eqn:lee_1}
        \lim_{X \to \infty} P_{\pm} (y, X, c) = \begin{cases}
            \int_{1}^{c} \cos\left(\frac{2 \pi y}{x}\right) \dd x & \text{if } + \\
            -i \int_{1}^{c} \sin \left(\frac{2 \pi y}{x}\right) \dd x & \text{if } -
        \end{cases}
    \end{equation}
    and assumming RH, if $\frac{1}{2} < \delta < 1$ we have
    \begin{equation}
        \label{eqn:lee_2}
        \lim_{X \to \infty} P_{\pm} (y, X, \delta) = \begin{cases}
            \cos (2 \pi y) & \text{if } + \\
            -i \sin (2 \pi y) & \text{if } -
        \end{cases}
    \end{equation}
\end{theorem}

The proof is much simpler than the case of modular forms.
The main ingredient of the proof is the following formulas \cite[Lemma 2.6]{lee2025murmurations}: for two distinct primes $p$ and $N$,
\begin{align}
    \sum_{\chi \in \cD_{+}(N)} \frac{\chi(p)}{\tau(\chi)} &= \left(\frac{N-1}{N}\right) \cos \left(\frac{2 \pi p}{N}\right) + \frac{1}{N} \\
    \sum_{\chi \in \cD_{-}(N)} \frac{\chi(p)}{\tau(\chi)} &= -i\left(\frac{N-1}{N}\right) \sin \left(\frac{2 \pi p}{N}\right)
\end{align}
Combined with the prime number theorem (which gives equidistribution results of primes in $[X, cX]$ normalized by $X$), we get \eqref{eqn:lee_1}.
For \eqref{eqn:lee_2}, we need to estimate the number of primes in the short interval $[X, X + X^\delta]$, which is the part that requires RH.

They also proved similar results for real Dirichlet characters, but the proof is more complicated.
Let $\scG$ be the set of odd square-free integers and let $\chi_{d} = \left(\frac{d}{\cdot}\right)$.
For a compactly supported smooth function $\Phi \ge 0$ on $\bR$, define
\begin{equation}
    M_{\Phi}(y, X, \delta) = \frac{\log X}{X^{1 + \delta}} \sum_{\substack{p \in [yX, yX + X^\delta] \\ p \text{ prime}}} \sum_{d \in \scG} \Phi\left(\frac{d}{X}\right) \chi_{8d}(p) \sqrt{p}.
\end{equation}

\begin{theorem}[{Lee--Oliver--Pozdnyakov \cite[Theorem 1.2]{lee2025murmurations}}]
    Fix $y > 0$ and assume $\frac{3}{4} < \delta < 1$.
    Assumming GRH, we have
    \begin{equation}
        M_{\Phi} (y, \delta) := \lim_{X \to \infty} M_{\Phi}(y, X, \delta) = \frac{1}{2} \sum_{\substack{a \ge 1 \\ a \text{ odd}}} \frac{\mu(a)}{a^2} \sum_{m \ge 1} (-1)^{m} \widetilde{\Phi} \left(\frac{m^2}{2 a^2 y}\right),
    \end{equation}
    where
    \begin{equation}
        \widetilde{\Phi}(\xi) = \int_{-\infty}^{\infty} (\cos(2 \pi \xi x) + \sin(2 \pi \xi x)) \Phi(x) \dd x.
    \end{equation}
\end{theorem}
The proof is more involved, which is based on the Polya--Vinogradov inequality 
\[
\left|\sum_{\substack{p \in [yX, yX + X^\delta] \\ p\text{ prime}}} \chi_d(p)\right| \ll (yX)^{\frac{1}{2} + \epsilon}
\]
(for non-principal $\chi_d$ with $\frac{1}{2} < \delta < 1$, which uses GRH \cite{granville2007large})
and a summation formula
\[
\frac{1}{X} \sum_{\substack{d \in \bZ \\ d \text{ odd}}} \left(\sum_{\substack{a^2 | |d| \\ a \le A}} \mu(a)\right) \Phi\left(\frac{d}{X}\right) \left(\frac{d}{p}\right) \sqrt{p} = \frac{1}{2} \left(\frac{2}{p}\right) \sum_{\substack{0 < a \le A \\ (a, 2p) = 1}} \frac{\mu(a)}{a^2} \sum_{k \in \bZ} (-1)^k \left(\frac{k}{p}\right) \widetilde{\Phi} \left(\frac{k X}{2 a^2 p}\right)
\]
which can be proved by using Poisson summation formula.


\subsection{Flying Hecke characters of imaginary quadratic fields}

\cite{wang2025murmurations}

\subsection{Flying modular forms (in weight direction)}

Recall that Zubrilina computed murmuration density for a \emph{fixed weight $k$ and varying level $N$}.
In \cite{bober2023murmurations}, Bober, Booker, Lee\footnote{Min Lee, not Kyu Hwan Lee}, and Lowry-Duda considered the opposite case, where they fix the level $N = 1$ and vary the weight $k$.
In this case, the considered family of Hecke newforms whose \emph{analytic conductor}
\[
\cN(k) := \left(\frac{\exp \psi(k/2)}{2 \pi}\right)^2 \approx \left(\frac{k-1}{4 \pi}\right)^2 + O(1)
\]
are in certain range, where $\psi(x) = \Gamma'(x)/\Gamma(x)$ is the digamma function.

\begin{theorem}[{Bober--Booker--Lee--Lowry-Duda \cite[Theorem 1.1]{bober2023murmurations}}]
    Fix $\epsilon \in (0, \frac{1}{12})$, $\delta \in \{0, 1\}$, and a compact interval $E \subset \bR_{> 0}$ with $|E| > 0$.
    Let $K, H > 0$ with $K^{\frac{5}{6} + \epsilon} < H < K^{1 - \epsilon}$, and let $N = \cN(K)$.
    Then as $K \to \infty$, we have

    \begin{equation}
        \frac{
            \sum_{\substack{p \text{ prime} \\ p / N \in E}}
            \log p
            \sum_{\substack{k \equiv 2 \delta \Mod{4} \\ |k - K| \le H}}
            \sum_{f \in H_k(1)} \lambda_f(p)
        }
        {
            \sum_{\substack{p \text{ prime} \\ p / N \in E}} 
            \log p 
            \sum_{\substack{k \equiv 2 \delta \Mod{4} \\ |k - K| \le H}} 
            \sum_{f \in H_k(1)} 1
        } = \frac{(-1)^\delta}{\sqrt{N}} \left(\frac{\nu(E)}{|E|} + o_{E, \epsilon}(1)\right)
    \end{equation}
    where
    \begin{align}
        \nu(E) &= \frac{1}{\zeta(2)} \sum_{\substack{a, q \in \bZ_{>0} \\ (a, q) = 1 \\ q^2 / a^2 \in E}} \frac{\mu(q)^2}{\varphi(q)^2 \sigma(q)} \left(\frac{a}{q}\right)^{-3} \\
        &= \frac{1}{2} \sum_{t \in \bZ} \prod_{p \nmid t} \frac{p^2 - p - 1}{p^2 - p} \int_E \cos\left(\frac{2 \pi t}{\sqrt{y}}\right) \dd y
    \end{align}
    where the summation $\sum^{*}$ indicates that the terms occuring at the endpoints of $E$ are halved. 
\end{theorem}

The main tool for the proof is the (original) Eichler--Selberg trace formula that does not include Atkin--Lehner operators (e.g. \cite[Theorem 2.1]{child2022twist}).
Then apply class number formula to replace class numbers with the special values of Dirichlet $L$-functions at $s = 1$, which can be estimated under GRH.

\subsection{Flying Maass forms}

Booker, Lee, Lowry-Duda, Seymour-Howell, and Zubrilina computed murmuration densities for weight 0 and level 1 Maass forms \cite{booker2024murmurations}.
They considered a family of Maass forms where the spectral parameter ($R$ with $\lambda = \frac{1}{4} + R^2$) goes ot $\infty$, which is equivalent to the \emph{analytic conductor} $\cN(R)$ going to $\infty$.

\begin{theorem}[{Booker--Lee--Lowry-Duda--Seymour-Howell--Zubrilina \cite[Theorem 1.1]{booker2024murmurations}}]
    Let $E \subset \bR_{>0}$ be a fixed compact interval with $|E| > 0$.
    Let $R, H > 0$ with $R^{\frac{5}{6} + \delta} < H < R^{1 - \delta}$ for some $\delta > 0$ and $N = \cN(R)$.
    Assumming GRH for $L$-functions of Dirichlet characters and Maass forms, as $R \to \infty$ we have 
    \begin{equation}
        \frac{\sum_{\substack{p\text{ prime} \\ p / N \in E}} \log p \sum_{|r(f) - R| \le H} \epsilon(f) a_{f}(p)}{\sum_{\substack{p\text{ prime} \\ p / N \in E}} \log p \sum_{|r(f) - R| \le H} 1} \to \frac{1}{\sqrt{N}|E|}  \sideset{}{^{*}}\sum_{\frac{q^2}{a^2} \in E}  \frac{\mu(q)^2}{\varphi(q)^2 \sigma(q)} \left(\frac{a}{q}\right)^{-3}
    \end{equation}
    where the summation $\sum^{*}$ indicates that the terms occuring at the endpoints of $E$ are halved. 
\end{theorem}

Proof uses an explicit Selberg trace formula due to Str\"ombergsson in his unpublished work \cite{strombergsson}, which requires an analytic test function and cannot be compactly supported, where GRH is needed to control the cutoff error term.
The remaining proof is similar to the weight aspect case of the modular forms \cite{bober2023murmurations}.

\subsection{General formulation}

In fact, all the above works fit into the general framework suggested by Sarnak, in his letter to Sutherland and Zubrilina \cite{sarnak}.

\subsection{Elliptic curves?}

There \emph{are} some works on murmuration density of elliptic curves.

In the last slide of Sutherland's lecture \cite{sutherland} at Tate conference (\emph{The legacy of John Tate, and beyond} at Harvard university), he mentioned an unpublished work with Sawin on murmuration density of elliptic curves, proved by Voronoi summation formula.
He considered it as \emph{a} murmuration theorem, and might not be \emph{the} murmuration theorem since the density formula is too complicated.

\begin{theorem}[Sutherland--Sawin, unpublished]
    Let $\scE(X) := \{y^2 = x^3 + ax + b: a, b \in \bZ, p^4 \mid a \Rightarrow p^6 \nmid b, \max\{4|a|^3, 27 b^2\} \le X\}$ be the set of naive isomorphism classes of elliptic curves over $\bQ$ ordered by height.
    For any smooth function $W : \bR_{>0} \to \bR$ with compact support, the limit
    \begin{equation}
        \lim_{X \to \infty} \frac{1}{|\scE(X)|} \sum_{E \in \scE(X)} \frac{\varepsilon(E)}{N_E} \sum_{n \ge 1} W\left(\frac{n}{N_E}\right) a_n(E)
    \end{equation}
    exists and is equal to
    \begin{equation}
        \int_{0}^{\infty} 2 \pi W(u) \sum_{n \ge 1} \frac{\prod_{p | n} \ell_{p^{\nu_p(n)}}}{\sqrt{n}} \sqrt{u} J_1 (4 \pi \sqrt{un}) \dd u
    \end{equation}
    where $\ell_{p^\nu} = \frac{p^9 - p^8}{p^{10} - 1} \Tr(T_p | S_{\nu + 2}(1))$.
\end{theorem}