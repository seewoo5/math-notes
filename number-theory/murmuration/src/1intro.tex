\section{Introduction}
\label{sec:intro}

\subsection{What is the role of Machine Learning?}

Although this note focus on the recent results on computing murmuration densities, I make a brief comment on the relation between machine learning and murmuration, which I found that existing literatures are often misleading on distinguishing the machine learning part and the murmuration part.
I read few articles on internet which basically say that ``AI found new mathematics,'' which is half true and half false.

One of the main motivation of the paper \cite{he2024murmurations} is to study  elliptic curves via machine learning.
Especially, they were interested in predicting the rank of elliptic curves (which is widely known to be hard to compute in general) by means of machine learning, where the coefficients $a_p(E)$ of Hasse--Weil $L$-functions are used as features.
Surprisingly, they found that simple logictic regression model can already distinguish between rank $0$ and $1$ elliptic curves with high accuracy of $>90\%$.
Along the line, they (more precisely, He, Lee, and Oliver) were curious about what was actually going on, and Pozdnyakov (who was an undergraduate student of Lee at that time) figured out the murmuration pattern.
This somehow gives an explanation of the high accuracy of the model, since the mumuration patterns for rank $0$ and $1$ elliptic curves are noticably different.
But the correct way to say is that the machine learning experiments \emph{motivated} them to study what models were doing, which is essentially the work of humans, not the ML models.
You can find more story in the Quanta Magazine article \cite{chiou2024elliptic}.

% and they conjectured that the murmuration density is related to the rank of elliptic curves.

\subsection{Sato--Tate conjecture and Murmuration}

Another confusing part (at least for me) is the difference between murmuration and Sato--Tate conjecture.