\documentclass{article}
\usepackage{amsfonts, amssymb, amsmath, amsthm}
\usepackage{tikz}
\usepackage{hyperref}
\usepackage{enumitem}
\usepackage{mathtools}
\usepackage{tabu}
\usepackage{comment}
\usepackage{mathrsfs}
\usetikzlibrary{positioning}
\title{Hilbert's theorem 90}
\author{Seewoo Lee}


\newtheorem{theorem}{Theorem}
\newtheorem{lemma}{Lemma}
\newcommand{\Gal}{\mathrm{Gal}}
\newtheorem{claim}{Claim}
\newcommand{\Tr}{\mathrm{Tr}}
\newcommand{\Nm}{\mathrm{N}}
\newcommand{\Cl}{\mathrm{Cl}}
\newcommand{\sgn}{\mathrm{sgn}}
\newtheorem{corollary}{Corollary}
\newcommand{\cha}{\mathrm{char}\,}
\newcommand{\rH}{\mathrm{H}}
\newcommand{\Mod}[1]{\,(\text{mod }#1)}
\newtheorem{proposition}{Proposition}
\newcommand\smallO{
  \mathchoice
    {{\scriptstyle\mathcal{O}}}% \displaystyle
    {{\scriptstyle\mathcal{O}}}% \textstyle
    {{\scriptscriptstyle\mathcal{O}}}% \scriptstyle
    {\scalebox{.7}{$\scriptscriptstyle\mathcal{O}$}}%\scriptscriptstyle
  }

\newcommand{\quadres}[2]{\left(\frac{#1}{#2}\right)}

\begin{document}
\maketitle

In this note, we introduce Hilbert's theorem 90 and its applications. 
\section{Hilbert's theorem 90}

Basically, Hilbert's theorem 90 is a vanishing theorem of some first Galois cohomology. 
Let $E/F$ be a (finite) Galois extension. We can naturally view $E^{\times}$ as a $G = \Gal(E/F)$-module. 
With the $G$-module structure, Hilbert's theorem 90 claims that first group cohomology of $G$ with coefficient $E^{\times}$ vanishes.  

\begin{theorem}[Hilbert]
Let $E/F$ be a Galois extension with a Galois group $G$. Then $\rH^{1}(G,E^{\times})=0$. 
\end{theorem}
\begin{proof}We need a following lemma:
\begin{lemma}
Let $E$ be a field and $\sigma_{1}, \dots, \sigma_{n}$ be a distinct automorphisms of $E$. Then they are $E$-linearly independent. 
\end{lemma}
\begin{proof}
Assume that they are not linearly independent. Let $r$ be a number of nonzero coefficients among $c_{i}$'s and assume that $r$ is minimal among such $r$'s. Also, we may assume that  $c_{1}, \dots, c_{r}\neq 0$ and $c_{r+1} = \cdots = c_{n} = 0$. 
Then $r>1$ since $c_{1}\sigma_{1} = 0$ implies $c_{1} = c_{1}\sigma_{1}(1) = 0$. 
Now choose $a\in E$ such that $\sigma_{1}(a)\neq \sigma_{r}(a)$. From
\begin{align*}
c_{1}\sigma_{1}(ax)+c_{2}\sigma_{2}(ax) + \cdots + c_{r}\sigma_{r}(ax) &= 0 \\
c_{1}\sigma_{r}(a)\sigma_{1}(x) + c_{2}\sigma_{r}(a)\sigma_{2}(x) + \cdots + c_{r}\sigma_{r}(a)\sigma_{r}(x) &= 0
\end{align*}
we have
$$
c_{1}(\sigma_{1}(a) - \sigma_{r}(a))\sigma_{1}(x) + \cdots + c_{r-1}(\sigma_{r-1}(a) - \sigma_{r}(a))\sigma_{r-1}(x) = 0.
$$
By the way, we have $c_{1}(\sigma_{1}(a) - \sigma_{r}(a))\neq 0$, and this contradicts to the minimality of $r$. 
\end{proof}
We have to show that if $\alpha:G\to E^{\times}$ is a 1-cocycle, then it is a 1-coboundary, i.e. $\alpha = d\beta \Leftrightarrow \alpha_{\sigma} = \sigma(\beta)/\beta$ for any $\sigma\in G$. 
Note that $\alpha$ is a 1-cocycle if and only if $\alpha_{\sigma\tau} = \alpha_{\sigma}\sigma(\alpha_{\tau})$ for all $\sigma, \tau\in G$. 
For given 1-cocycle $\alpha$, consider the map 
$$
\sum_{\sigma\in G} \alpha_{\sigma} \sigma:E\to E. 
$$
By the previous lemma, the above map is nonzero and we can find $\theta\in E$ such that $\gamma:= \sum_{\sigma\in G}\alpha_{\sigma}\sigma(\theta)\neq 0$. 
Then we have
\begin{align*}
\sigma\gamma  = \sum_{\sigma\in G}\sigma(\alpha_{\tau}) \sigma\tau(\theta) 
= \sum_{\tau\in G}\alpha_{\sigma}^{-1}\alpha_{\sigma\tau}\sigma\tau(\theta) 
= \alpha_{\sigma}^{-1}\sum_{\tau\in G} \alpha_{\sigma\tau}\sigma\tau(\theta) 
= \alpha_{\sigma}^{-1}\gamma
\end{align*}
which implies that $\alpha_{\sigma} = \sigma(\beta)/\beta$ for $\beta = \gamma^{-1}$. 
\end{proof}
This is a \emph{modern} version of Hilbert's theorem 90. The original version is about when $E/F$ is a cyclic extension. 

\begin{corollary}
Let $E/F$ be a finite cyclic extension and let $\sigma$ be a generator of the Galois group $G = \Gal(E/F)$. For $\alpha\in E$, if $\Nm_{E/F}(\alpha) = 1$, then $\alpha = \beta/\sigma(\beta)$ for some $\beta\in E$. 
\end{corollary}
\begin{proof}
Let $n = [E:F]$. $\Nm_{E/F} (\alpha) = 1$ is equivalent to $\alpha \sigma(\alpha)\cdots \sigma^{n-1}(\alpha) = 1$. From this assumption, we can define a 1-cocycle $\alpha:G\to E^{\times}$ such that $\alpha_{\sigma} = \alpha$. Then Hilbert's theorem 90 implies that $\alpha$ is a 1-coboundary, so we can find $\beta$ such that $\alpha = \alpha_{\sigma} = \beta / \sigma(\beta)$. 
\end{proof}
This is somehow multiplicative version of Hilbert's theorem 90. There's also additive version for the trace map. 
\begin{theorem}[Hilbert's theorem 90, Additive form]
Let $E/F$ be a cyclic extension of degree $n$ with Galois group $G$. Let $G = \langle \sigma\rangle$. Then for $\alpha\in E$, $\Tr_{E/F}(\alpha) = 0$ if and only if $\alpha = \beta -\sigma(\beta)$ for some $\alpha\in E$. 
\end{theorem}
\begin{proof}
By the lemma again, we can prove that there exists $\theta\in E$ such that
$$
\Tr_{E/F}(\theta) = \theta + \sigma(\theta) + \sigma^{2}(\theta) + \cdots + \sigma^{n-1}(\theta) \neq 0. 
$$
Now put
$$
\beta := \frac{1}{\Tr_{E/F}(\theta)} (\alpha \sigma(\theta) + (\alpha+\sigma(\alpha))\sigma^{2}(\theta) + \cdots + (\alpha+\sigma(\alpha) + \cdots +\sigma^{n-2}(\alpha))\sigma^{n-1}(\theta))
$$
then one can check $\alpha = \beta - \sigma(\beta)$ holds. 
\end{proof}
More generally, we have $\rH^{r}(G, E) = 0$ for any finite Galois extension $E/F$, by using the normal basis theorem and a vanishing property of cohomology of  induced modules. 



\section{Pythagorean triples}
Using Hilbert's theorem 90, we can find all Pythagorean triples, i.e. rational points on the circle $x^{2} + y^{2} =1$. 
\begin{corollary}
Let $a, b$ be rational numbers s.t. $a^{2}+ b^{2} =1 $. Then there exists $c, d\in \mathbb{Z}$ s.t. 
$$
 (a, b) = \left( \frac{c^{2}-d^{2}}{c^{2}+d^{2}}, \frac{2cd}{c^{2}+d^{2}}\right).
$$
In other words, every rational point on the circle $x^{2} + y^{2} = 1$ has above form. 
\end{corollary}
\begin{proof}
This directly follows from Hilbert's theorem 90 by applying to the extension $\mathbb{Q}(i)/\mathbb{Q}$. 
In fact, if $a^{2} + b^{2} = 1$, then $\alpha = a+bi\in \mathbb{Q}(i)$ has a norm 1, so there exists $c+di\in \mathbb{Q}(i)$ s.t. 
$$
\alpha = a+bi = \frac{c+di}{\sigma(c+di)} = \frac{c+di}{c-di} = \frac{c^{2}-d^{2}}{c^{2} + d^{2}} + \frac{2cd}{c^{2} + d^{2}}i
$$
and we obtain the theorem by multiplying common denominator of $c$ and $d$. (Here $\sigma : \mathbb{Q}(i)\to \mathbb{Q}(i)$ is the nontrivial element in $\Gal(\mathbb{Q}(i)/\mathbb{Q})$.) 
\end{proof}
More generally, by considering the extension $\mathbb{Q}(\sqrt{-D})/\mathbb{Q}$ for square-free integer $D>0$, we can prove the following:
\begin{corollary}
Let $a, b$ be rational numbers s.t. $a^{2} +Db^{2} = 1$. 
Then there exists $c, d\in\mathbb{Z}$ s.t. 
$$
(a, b) = \left(\frac{c^{2} - Dd^{2}}{c^{2} + Dd^{2}}, \frac{2cd}{c^{2} + Dd^{2}}\right).
$$
\end{corollary}


\section{Kummer extension}
Using the Hilbert's theorem 90, we can prove that any degree $n$ cyclic extension can be obtained by adjoining certain $n$-th root of element, if the base field contains a primitive $n$-th root of unity. 
\begin{theorem}
Let $F$ be a field and let $n\geq 1$ be a natural number with $(\cha p, n) = 1$. Assume that $F$ contains a primitive $n$-th root of unity, $\zeta_{n}$. If $E/F$ is a cyclic extension of degree $n$, then there exists $a\in F$ s.t. $E = F(\sqrt[n]{a})$. 
\end{theorem}
\begin{proof}
Let $E/F$ be a cyclic extension of degree $n$ and let $G = \Gal(E/F) = \langle \sigma\rangle$. 
Since $\Nm_{E/F}(\zeta_{n}^{-1}) = (\zeta_{n}^{-1})^{n} = 1$, by Hilbert's theorem 90, there exists $\alpha\in E^{\times}$ s.t. $\zeta_{n}^{-1} = \alpha/\sigma(\alpha)\Leftrightarrow \sigma(\alpha) = \zeta_{n}\alpha$. 
Then $\sigma^{j}(\alpha) = \zeta_{n}^{j}\alpha$ for any $0\leq j\leq n-1$, so $\alpha$ has $n$ distinct conjugates and $[F(\alpha):F]\geq n$. 
However, from $[E:F] = n$ and $F(\alpha)\subseteq E$, we have $E = F(\alpha)$. 
Moreover, $\sigma(\alpha^{n}) = \sigma(\alpha)^{n}= \zeta_{n}^{n}\alpha^{n} = \alpha^{n}$, so $\alpha^{n}\in F$ and $E = F(\sqrt[n]{a})$ if we put $a = \alpha^{n}$. 
\end{proof}

Kummer theory studies about such kind of extension. It states that any abelian extension of $F$ of exponent dividing $n$ is formed by extraction of roots of elements in $F$. Moreover, there exists a one-to-one correspondence  abelian extensions of $F$ of exponent $n$ and subgroups of $F^{\times}/(F^{\times})^{n}$. 

\section{Artin-Schreier extension}
Using the additive form of Hilbert's theorem 90, we can prove that degree $p$ extension of a characteristic $p$ field can be obtained by adjoining a root of certain  polynomial. This can be considered  additive analogue of Kummer extension. 

\begin{theorem}[Artin-Schreier extension]
Let $F$ be a field of characteristic $p > 0$.
\begin{enumerate}
\item For any $a\in F$, the polynomial  $x^{p} - x - a\in F[x]$ is completely reducible (every root of the polynomial is in $F$) or irreducible. 
\item Conversely, if $E/F$ is a cyclic extension of degree $p$, $E$ is a splitting field of $x^{p} - x - a$ for some $a\in E$. 
\end{enumerate} 
\end{theorem}
\begin{proof}
1. First, we can observe that if $\alpha$ is a root of the polynomial $f(x) = x^{p} - x - a$, then $\alpha + j$ is also root of the polynomial for any $0\leq j\leq p-1$, since $(\alpha+j)^{p} - (\alpha + j)  - a= \alpha^{p} - j^{p} - \alpha - j - a= \alpha^{p} - \alpha - a = 0$. 
Hence if $f(x)$ has a root in $F$, then every root of $f(x)$ is in $F$. 

Now assume that $f(x)$ doesn't have a root in $F$. We claim that $f(x)$ is irreducible over $F$. 
Suppose that $f(x)$ is not irreducible, so that $f(x) = g(x) h(x)$ for some  non-constant polynomials $g(x), h(x)\in F[x]$. 
If $\alpha\in \overline{F}$ is a root of $f(x)$, then as we mentioned above, $\alpha + j$ is a root for any $0\leq j\leq p-1$. Thus $f(x) = \prod_{0\leq j\leq p-1} (x-\alpha - j)$ and we have
$$
g(x) = \prod_{j\in S}(x-\alpha-j), \quad h(x) = \prod_{j\not\in S} (x-\alpha - j)
$$
for some subset $S\subsetneq \{0, 1, \dots, p-1\}$. If $d = |S|$, then the $(d-1)$-th coefficient of $g(x)$ is $-\sum_{j\in S} (\alpha + j) = -d\alpha - \sum_{j\in S} j \in F$, which implies $d\alpha \in F$. Since $0<d<p$, we have $\alpha\in F$,  which gives a contradiction. 
Hence $f(x)$ is irreducible over $F$. 

2. Let $E/F$ be a cyclic extension of degree $p$ and let $G = \Gal(E/F) = \langle \sigma\rangle$. Since $\Tr_{E/F}(-1) = -p = 0$, by additive form of Hilbert's theorem 90,  there exists $\alpha\in E$ s.t. $\alpha - \sigma(\alpha) = -1$, i.e. $\sigma(\alpha) = \alpha + 1$. Then $\sigma^{j}(\alpha) = \alpha + j$ for all $0\leq j\leq p-1$, so $\alpha$ has $p$ distinct conjugates and $[F(\alpha):F]\geq p$. 
However, from $[E:F] = p$ and $F(\alpha) = E$, this gives $E = F(\alpha)$. Now we have
$$
\sigma(\alpha^{p} - \alpha) = \sigma(\alpha)^{p} - \sigma(\alpha) = (\alpha +1)^{p} - (\alpha + 1) = \alpha^{p} -\alpha, 
$$
so $\alpha^{p} -\alpha \in F$ and $\alpha$ is a root of the polynomial $f(x) = x^{p} - x - a\in F[x]$, where $a = \alpha^{p}- \alpha$. 
\end{proof}

In general, it is hard to find an irreducible polynomial over a finite field of given degree. However, the above theorem shows that for any prime $p$, the polynomial $x^{p} - x -1$ is irreducible polynomial over $\mathbb{F}_{p}$.




\section{Function fields}
We can obtain the following interesting theorem for rational functions:
\begin{theorem}
Let $f(x)\in \mathbb{C}(x)$ be a rational function which satisfies 
$$
f(x) f(\zeta x) f(\zeta^{2}x) \cdots f(\zeta^{n-1}x) = 1
$$
for $\zeta = \zeta_{n} = e^{2\pi i/n}$, $n$-th root of unity. Then there exists $g(x)\in \mathbb{C}(x)$ s.t. 
$$
f(x) = \frac{g(x)}{g(\zeta x)}.
$$
\end{theorem}
For example, $f(x) = \zeta$ clearly satisfies the condition, and we have $\zeta = g(x)/g(\zeta x)$ for $g(x) = 1/x$.
\begin{proof} 
Actually, this is a direct consequence of Hilbert's theorem 90. Let $E = \mathbb{C}(x)$ and $F = \mathbb{C}(x^{n})$ be a subfield. 
Then $E/F$ is a Galois extension since $E$ is a splitting field of the polynomial $y^{n} - x^{n} \in F[y] = \mathbb{C}(x^{n})[y]$. 
It's Galois group is $G = \Gal(E/F) \simeq\mathbb{Z}/n\mathbb{Z}$, where the generator of the group is given by $\sigma:E\to E$, $\sigma(f(x)) = f(\zeta x)$. 

Now the condition on $f(x)$ is equivalent to $\Nm_{E/F}f(x) = 1$. So by Hilbert's theorem 90, there exists $g(x)\in E$ s.t. $f(x) = g(x)/\sigma(g(x)) = g(x)/g(\zeta x)$. 
\end{proof}
As one can see in the proof, the theorem also holds if we replace $\mathbb{C}$ by any field $k$ with $(\cha k, n) = 1$. 
\end{document}