\section{Holonomy bound(s)}

In previous talks, we already covered a version of the main theorem of \cite{calegari2024linear}.
In fact, there are three different (but closely related) holonomy bounds in \cite{calegari2024linear}, which are Theorem 6.0.2, 7.0.1, and 7.1.6, where all three give the desired contradiction (i.e. small enough upper bound).
In this seminar, we focus on Theorem 7.0.1, whose special case (i.e. when $\mathbf{e} = 0$ \cite[Theorem 2.5.1]{calegari2024linear}) is covered previously.
Recall that the previous bound was of the form
\begin{equation}
\label{eqn:holonomybounde0}
    m \le \frac{\iint_{\bT^2} \log |\varphi(z) - \varphi(w)| \dd \mu(w) \dd \mu(z)}{\log |\varphi'(0)| - \tau(\mathbf{b})}
\end{equation}
where the array $\mathbf{b} = (b_{i, j})$ records the denominator types of the power series
$$
    f_i(x) = \sum_{n \ge 0} a_{i, n} \frac{x^n}{[1, \dots, b_{i, 1} n] [1, \dots, b_{i, 2}n] \cdots [1, \dots, b_{i, r}n]}
$$
with $a_{i,n} \in \bZ$ and $[1, 2, \dots, bn] := \mathrm{lcm}(1, 2, \dots, \lfloor bn \rfloor)$ and $\tau(\mathbf{b})$ is a nonnegative number associated to the array $\mathbf{b}$ (which has a value in $[0, b_{m, 1} + \cdots + b_{m, r}]$).
However, some of the 14 power series have slightly more general denominator types of
$$
    f_i(x) = \sum_{n \ge 0} a_{i, n} \frac{x^n}{n^{e_i}[1, \dots, b_{i, 1} n] [1, \dots, b_{i, 2}n] \cdots [1, \dots, b_{i, r}n]}
$$
where we have an extra $n^{e_i}$ term in the denominator with $e_i \in \bZ_{\ge 0}$ comes from integration.
It is possible to replace $n^{e_i}$ with $[1, \dots, n]^{e_i}$, but we will see that we cannot obtain a good enough holonomy bound with such replacements.


Here is the full version of the theorem that we need:
\begin{theorem}[{\cite[Theorem 7.0.1]{calegari2024linear}}]
\label{thm:bound}
Consider two positive integers $m, r$, a nonnegative integer vector $\mathbf{e} = (e_1, \dots, e_m)$, and an $m \times r$ rectangular array of nonnegative real numbers
$$
    \mathbf{b} := (b_{i,j})_{1 \le i \le m, 1 \le j \le r},
$$
all of whose columns have the form
$$
0 = b_{1, j} = b_{2, j} = \dots = b_{u_j, j} < b_{u_j + 1, j} = \cdots = b_{m, j} =: b_j, \quad \forall 1, \dots, r
$$
for some $u_j \in \{0, 1, \dots, m\}$ depending on the column.
Let
$$
    \sigma_i := b_{i, 1} + \cdots + b_{i, r}, \quad i = 1, \dots, m
$$
be the $i$-th row sum, and define
$$
    \tau^{\flat}(\mathbf{b}) := \frac{1}{m^2} \sum_{i=1}^{m}(2i - 1)\sigma_i = \sigma_m - \frac{1}{m^2} \sum_{j=1}^{r} u_j^2 b_j \in [0, \sigma_m]
$$
and
$$
    \tau^{\sharp}(\mathbf{e}) := \frac{2}{m^2} \min_{\xi \in [0, m]} \left\{ \xi \sum_{i=1}^{m} e_i + \left(\max_{1 \le i \le m} e_i\right) I_\xi^{m}(\xi)\right\}
$$
where $I_{u}^{v}(w)$ a function defined as
\begin{align*}
    I_{u}^{v}(w) &:= \int_{\min\{u, 1\}}^{1} \max\{t - w, 0\} \dd t + \int_{\max\{u, 1\}}^{v} \sum_{h=1}^{\left\lfloor \frac{t - 1}{ \max \{1, w\}} \right\rfloor} \frac{1}{h} \dd t \\
    &+ \int_{\max\{u, 1\}}^{v} \max\left\{ \frac{t}{\left\lfloor \frac{t + \max\{0, w-1\}}{\max\{1, w\}} \right\rfloor} - w, 0\right\} \dd t
\end{align*}
for $0 \le \max\{u, 1\} \le v$ and $w \le v$.
Finally, let
$$
\tau(\mathbf{b}; \mathbf{e}) := \tau^{\flat}(\mathbf{b}) + \tau^{\sharp}(\mathbf{e}).
$$
Let $\varphi: (\overline{\bD}, 0) \to (\bC, 0)$ be a holomorphic function with
\begin{equation}
\label{eqn:cond}
    \log |\varphi'(0)| > \max\{\sigma_m, \tau(\mathbf{b}; \mathbf{e})\}.
\end{equation}
Suppose there exists an $m$-tuple $f_1, \dots, f_m \in \bQ\llb x\rrb$ of $\bQ(x)$-linearly independent formal functions with denominator types of the form
$$
    f_i(x) = a_{i, 0} + \sum_{n=1}^{\infty} a_{i, n} \frac{x^n}{n^{e_i} [1, \dots, b_{i, 1}n][1, \dots, b_{i, 2}n] \cdots[1, \dots, b_{i, r}n]}, \quad a_{i, n} \in \bZ
$$
such that $f_i(\varphi(z))$ defines a meromorphic function on $|z| < 1$ for all $i = 1, \dots, m$.
Then
\begin{equation}
\label{eqn:holonomybound}
    m \le \frac{\iint_{\bT^2} \log |\varphi(z) - \varphi(w)| \dd \mu(w) \dd \mu(z)}{\log |\varphi'(0)| - \tau(\mathbf{b};\mathbf{e})}.
\end{equation}
In particular, the formal functions $f_1, \dots, f_m$ are holonomic.
Moreover, if the functions $f_i$ are \emph{a priori} assumed to be holonomic, the condition \eqref{eqn:cond} can be relaxed to $\log |\varphi'(0)| > \tau(\mathbf{b}; \mathbf{e})$.
\end{theorem}

Let me give a (very) brief sketch of the proof.
Recall that we considered the Euclidean lattice
\begin{align*}
    E_D &:= \bigoplus_{h=0}^{r} \bigoplus_{i=u_h + 1}^{u_{h+1}} \frac{1}{[1, \dots, y_{h+1}D] \cdots [1, \dots,  y_{r}D]} \cdot \bZ\left[\frac{1}{x}\right]_{\le D} \\
    &=\bigoplus_{i=1}^{m} \frac{1}{[1, \dots, y_1 D] \cdots [1, \dots, y_{r}D]} \cdot \bZ \left[\frac{1}{x} \right]_{\le D}
\end{align*}
which is of rank $m(D+1)$ as a free $\bZ$-module (with a suitable Euclidean norm inherited from a Hermitian metric of a line bundle on $\bP^1$) and $F_\bQ = x^{-D} \bQ \llb x \rrb$, with a map
$$
    \psi_D: E_{D, \bQ} \to F_\bQ, \quad (Q_i)_{1 \le i \le m} \mapsto \sum_{i=1}^{m} Q_i f_i
$$
which is injective due to the linear independence of $f_i$'s.
Now we applied Bost's slope inequality \eqref{eqn:bostineq} and arithmetic Hilbert--Samuel formula to get the bound of the form \eqref{eqn:holonomybounde0}.
Here $y_1, \dots, y_r$ are auxiliary parameters that are optimized later and give the expression of $\tau^\flat(\mathbf{b})$, while estimating $h_{\mathrm{fin}}(\psi_D^{(n)})$.
When $\mathbf{e} \ne \mathbf{0}$, we modify $E_D$ as
\begin{align*}
    E_D := \bigoplus_{h=0}^{r} \bigoplus_{i=u_h + 1}^{u_{h+1}} \frac{[1, \dots, \xi D]^{e_i}}{[1, \dots, y_{h+1}D] \cdots [1, \dots,  y_{r}D]} \cdot \bZ\left[\frac{1}{x}\right]_{\le D}
\end{align*}
with an additional auxiliary parameter $\xi \in [0, m]$, and this introduces additional terms in the estimation of $h_{\mathrm{fin}}(\psi_D^{(n)})$ which accounts for the term $\tau^\sharp(\mathbf{e})$.

Note that both $\tau^\flat(\mathbf{b})$ and $\tau^\sharp(\mathbf{e})$, hence $\tau(\mathbf{b}; \mathbf{e})$ can be computed explicitly.
Especially, we can easily plot the graph of $\xi \mapsto \xi \sum_{1 \le i \le m} e_i + (\max_{1 \le i \le m} e_i) I_\xi^m(\xi)$ and see where it attains its minimum value.
(The function is complicated but elementary, and we can compute the exact minimum value.)