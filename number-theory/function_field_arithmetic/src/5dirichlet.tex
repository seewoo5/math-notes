\section{How many primes in arithmetic progressions?}
\label{sec:dirichlet}

In Section \ref{sec:reciprocity}, we saw that the polynomial $T^2 + 1$ has a root in $\bF_p$ if and only if $p \equiv 1 \pmod{4}$\footnote{or $p = 2$.}.
Especially, we also saw that almost half of primes $p < 100$ satisfy $p \equiv 1 \pmod{4}$, and the other half satisfy $p \equiv 3 \pmod{4}$.
One might ask if the ``density'' of primes actually converges to $1/2$ as we consider more and more primes.

In this section, we will show that this is indeed true.
More generally, we will study Dirichlet's theorem on primes in arithmetic progressions, which states that this is a general phenomenon for any arithmetic progression, and the densities are uniform across all congruence classes.
Of course, we will also study the polynomial analogue of Dirichlet's theorem on irreducible polynomials.

\subsection{Dirichlet's theorem on primes}
\label{sec:dirichlet-primes}

Before we talk about denisties of primes in arithmetic progressions $4k + 1$ and $4k + 3$, let's first prove that there are infinitely many primes in each of these two congruence classes, by modifying the proof of Euclid's theorem on the infinitude of primes.
\begin{theorem}
    \label{thm:1m4-prime-infinite}
    There are infinitely many primes of the form $4k + 1$.
\end{theorem}
\begin{proof}
    Assume that there are only finitely many primes of the form $4k + 1$, and let $p_1, p_2, \dots, p_r$ be the set of such primes.
    Consider $N = 4 (p_1 p_2 \cdots p_r)^2 + 1$, and let $p$ be a prime dividing $N$.
    Since $N$ is odd, we have $p > 2$.
    Also, since $p$ divides $N$, $p$ has to be different from $p_1, \dots, p_r$.
    However, $(2 p_1 p_2 \cdots p_r)^2 \equiv -1 \pmod{p}$ implies that $-1$ is a quadratic residue modulo $p$, hence $p \equiv 1 \pmod{4}$ by Theorem \ref{thm:m1sq} and we get a contradiction.
\end{proof}

\begin{exercise}
    Prove that there are infinitely many primes of the form $4k + 3$.\footnote{Hint: Assume there are finitely many, $p_1, \dots, p_r$. Now consider $N = 4p_1 p_2 \dots p_r + 3$. Show that there is at least one prime $p$ dividing $N$ such that $p \equiv 3 \pmod{4}$.}
\end{exercise}

One may ask if such an argument can be used to prove that there are infinitely many primes in any arithmetic progression $ak + b$ for $a > 0$ and $\gcd(a, b) = 1$, by choosing the polynomial defining $N$ carefully.
Unfortunately, this is not the case, and it turns out that such \emph{Euclidean proof} only exists in very special cases.
\begin{theorem}[{Schur \cite{schur1912uber}, Murty \cite{murty1988primes}}]
    \emph{Euclidean proof} exists for the arithmetic progression $ak + b$ if and only if $b^2 \equiv 1 \pmod{a}$.
\end{theorem}
 
However, one can find that the primes of the form $ak + b$ for different $a, b$ (with $\gcd(a, b) = 1$) seems to be infinitely many, and also seems to be evenly distributed across all congruence classes.
\begin{exercise}\sage
    Count the number of primes $p < 10^6$ in the arithmetic progressions $8k + b$ for $b = 1, 3, 5, 7$.
\end{exercise}

Dirichlet \cite{dirichlet1837beweis} proved that this is indeed true for any arithmetic progression $ak + b$ with $\gcd(a, b) = 1$.
He actually proved that they are evenly distributed across all congruence classes modulo $a$ in the following sense: for a subset $S$ of positive integers, the Dirichlet density of $S$ is defined as the limit (if exists)
\begin{equation}
    \delta(S) := \lim_{s \to 1^+} \frac{\sum_{p \in S} p^{-s}}{\sum_{p} p^{-s}}.
    \label{eqn:dirichlet-density}
\end{equation}
Note that the denominator $\sum_p p^{-s}$ goes to $\infty$ as $s \to 1^+$, which can be shown by Euler factorization:
\[
\log \zeta(s) = \log \prod_p (1 - p^{-s})^{-1} = \sum_p \log(1 - p^{-s})^{-1} = \sum_p \frac{1}{p^s} + \sum_{p, k \ge 2} \frac{1}{k p^{ks}} = \sum_p \frac{1}{p^s} + R(s)
\]
where $R(s)$ is bounded as $s \to 1^+$, and the claim follows from $\lim_{s \to 1^+} \zeta(s) = \infty$.
This might be different from what you'd expect as a density, which is usually called \emph{natural density}:
\begin{equation}
    \delta_{\mathrm{nat}}(S) := \lim_{n \to \infty} \frac{\#\{p \in S : p \le n\}}{\#\{p : p \le n\}}.
\end{equation}

Dirichlet proved the following theorem:
\begin{theorem}[{Dirichlet \cite{dirichlet1837beweis}}]
    \label{thm:dirichlet}
    For each $a, b$ with $\gcd(a, b) = 1$, there are infinitely many primes of the form $ak + b$.
    Moreover, the Dirichlet density of such primes is $\frac{1}{\phi(a)}$, where $\phi$ is the Euler $\phi$-function.\footnote{In fact, natural density also exists and is equal to $\frac{1}{\phi(a)}$.}
\end{theorem}
The proof is often considered as a starting point of modern analytic number theory.
It can be found in many places, and we will only give a sketch.
For the historical point of view, see the article by Avigad and Morris \cite{avigad2014concept}.

The main idea is to use \emph{Dirichlet characters} and associated \emph{Dirichlet L-functions}.
As a motivating example, let's consider the case when $a = 4$ and $b = 1$, which is already proved in Theorem \ref{thm:1m4-prime-infinite}.
In this case, the Dirichlet's theorem states that
\begin{equation}
    \label{eqn:dirichlet-1m4}
    \delta(\{p : p \equiv 1 \Mod{4}\}) = \lim_{s \to 1^+} \frac{\sum_{p \equiv 1 \pmod{4}} p^{-s}}{\sum_{p} p^{-s}} = \frac{1}{2}.
\end{equation}
From the previous discussion, one may expect to use the Dirichlet series
\[
L_1(s) = \sum_{n \equiv 1 \pmod{4}} \frac{1}{n^s} = \frac{1}{1^s} + \frac{1}{5^s} + \frac{1}{9^s} + \frac{1}{13^s} + \cdots.
\]
However, this series is not good enough, since it does not admit Euler factorization.
Similarly, the series
\[
L_3(s) = \sum_{n \equiv 3 \pmod{4}} \frac{1}{n^s} = \frac{1}{3^s} + \frac{1}{7^s} + \frac{1}{11^s} + \frac{1}{15^s} + \cdots
\]
is not the right choice.
However, their sum and differences are nice: we have
\begin{align}
    L_1(s) + L_3(s) &= \sum_{n \equiv 1 \pmod{4}} \frac{1}{n^s} = \prod_{p \ne 2} (1 - p^{-s})^{-1} = (1 - 2^{-s})\zeta(s) \label{eqn:L1p3} \\
    L_1(s) - L_3(s) &= \frac{1}{1^s} - \frac{1}{3^s} + \frac{1}{5^s} - \frac{1}{7^s} + \cdots = \prod_{p \equiv 1 \pmod{4}} (1 - p^{-s})^{-1} \prod_{p \equiv 3 \pmod{4}} (1 + p^{-s})^{-1} \label{eqn:L1m3}
\end{align}
Especially, taking logarithm of \eqref{eqn:L1m3} gives
\[
\log (L_1(s) - L_3(s)) = \sum_{p \equiv 1 \pmod{4}} \frac{1}{p^s} - \sum_{p \equiv 3 \pmod{4}} \frac{1}{p^s} + R(s)
\]
where $R(s)$ is bounded as $s \to 1^+$.
\emph{Assuming $\lim_{s \to 1^+} L_1(s) - L_3(s) \ne 0$}, we can conclude that the Dirichlet density of primes $p \equiv 1 \pmod{4}$ and primes $p \equiv 3 \pmod{4}$ are the same as $1/2$.

The series $L_1(s) - L_3(s)$ is a special case of \emph{Dirichlet L-functions}, associated to a \emph{Dirichlet character}: for the function $\chi: \bZ \to \bC$ defined by
\begin{equation}
    \chi(n) = \chi_{-4}(n) := \begin{cases}
        1 & n \equiv 1 \pmod{4} \\
        -1 & n \equiv 3 \pmod{4} \\
        0 & \text{otherwise}
    \end{cases}
\end{equation}
we have $L_1(s) - L_3(s) = \sum_{n \ge 1} \chi(n) n^{-s}$.
More generally,
\begin{definition}
    A \emph{Dirichlet character modulo $a$} is a function $\chi: \bZ \to \bC$ such that:
    \begin{enumerate}
        \item $\chi(n + a) = \chi(n)$ for all $n \in \bZ$.
        \item $\chi(mn) = \chi(m)\chi(n)$ for all $m, n \in \bZ$.
        \item $\chi(n) = 0$ if and only if $\gcd(n, a) > 1$.
    \end{enumerate}
\end{definition}
Note that this induces a group homomorphism $(\bZ / a\bZ)^\times \to \bC^\times$, and conversely, any group homomorphism $\chi: (\bZ / a\bZ)^\times \to \bC^\times$ can be extended to a Dirichlet character modulo $a$.

For general modulus, we consider all Dirichlet characters modulo $a$ and the associated Dirichlet L-functions, and consider the linear combination
\[
\sum_{\chi} \overline{\chi(b)} \log L(s, \chi) = \phi(a) \sum_{p \equiv b \pmod{a}} \frac{1}{p^s} + R(s)
\]
where $R(s)$ is bounded as $s \to 1^+$.
Equality follows from the \emph{orthogonality} of Dirichlet characters (see Theorem \ref{thm:dirichlet-orthogonality-poly} for the polynomial case).
Then Theorem \ref{thm:dirichlet} follows once we prove $L(1, \chi) \ne 0$ for all nontrivial $\chi$, which is the main step of Dirichlet's proof.

\begin{exercise}
    In this exercise, we will prove 
    \begin{equation}
        L(1, \chi_{-4}) = 1 - \frac{1}{3} + \frac{1}{5} - \frac{1}{7} + \cdots = \frac{\pi}{4} \ne 0.
    \end{equation}
    \begin{enumerate}
        \item Without computing the actual value, show that the series converges to a positive number.
        \item Consider the polynomial
        \[
        f_n(x) = x - \frac{x^3}{3} + \frac{x^5}{5} - \frac{x^7}{7} + \cdots + (-1)^n \frac{x^{2n + 1}}{2n + 1}.
        \]
        Show that $\lim_{n \to \infty} f_n(1) = L(1, \chi_{-4})$ and
        \[
        f_n'(x) = \frac{1 - (-x^2)^{n+1}}{1 + x^2}.
        \]
        \item By integrating $f_n'(x)$ from $0$ to $1$, show that
        \[
        f_n(1) = \int_0^1 f_n'(t) \dd t = \int_0^1 \frac{\dd t}{1 + t^2} - \int_0^1 \frac{(-t^2)^{n+1}}{1 + t^2} \dd t = \frac{\pi}{4} - \int_0^1 \frac{(-t^2)^{n+1}}{1 + t^2} \dd t.
        \]
        Now use dominated convergence theorem to conclude the proof.
    \end{enumerate}
\end{exercise}

\subsection{Dirichlet's theorem on irreducible polynomials}
\label{sec:dirichlet-polynomials}

We will follow the same idea as in the integer case to prove the polynomial analogue of Dirichlet's theorem.
The notable difference is that proving $L(1, \chi) \ne 0$ for nontrivial $\chi$ is much easier, since the $L$-functions $L(s, \chi)$ are simply polynomials in $p^{-s}$.
We can define Dirichlet character as you expect.
\begin{definition}
    \label{def:dirichlet-character-poly}
    Let $h \in A$ be a (monic) polynomial of degree $d \geq 1$.
    A \emph{Dirichlet character modulo $h$} is a function $\chi: A \to \bC$ such that:
    \begin{enumerate}
        \item $\chi(a + bh) = \chi(a)$ for all $a \in A$ and $b \in A$.
        \item $\chi(ab) = \chi(a)\chi(b)$ for all $a, b \in A$.
        \item $\chi(a) \ne 0$ if and only if $\gcd(a, h) = 1$.
    \end{enumerate}
\end{definition}

As you expect, this induces a group homomorphism $(A / hA)^\times \to \bC^\times$, and conversely, any group homomorphism $\chi: (A / hA)^\times \to \bC^\times$ can be extended to a Dirichlet character modulo $h$.
The trivial Dirichlet character modulo $h$ is defined by $\chi(a) = 1$ for all $a \in A$ with $\gcd(a, h) = 1$.
We will denote the set of all Dirichlet characters modulo $h$ by $X_h$.

\begin{example}
    \label{ex:dirichlet-character1}
    Let $p = 3$ and $h(T) = T$.
    Then we have the following nontrivial Dirichlet characters modulo $h$:    
    \[
    \chi(f) := \begin{cases}
        1 & \text{if } f \equiv 1 \pmod{T} \\
        -1 & \text{if } f \equiv 2 \pmod{T} \\
        0 & \text{otherwise}
    \end{cases}
    \]
\end{example}

\begin{example}
    \label{ex:dirichlet-character2}
    Let $p = 2$ and $h(T) = T^2 + T + 1$.
    Then the following defines a nontrivial Dirichlet character modulo $h$:
    \[
    \chi(f) = \begin{cases}
        1 & \text{if } f \equiv 1 \pmod{h} \\
        \zeta_3 & \text{if } f \equiv T \pmod{h} \\
        \zeta_3^2 & \text{if } f \equiv T + 1 \pmod{h} \\
        0 & \text{otherwise}
    \end{cases}
    \]
    where $\zeta_3 = e^{2\pi i / 3}$ is a third root of unity.
\end{example}

\begin{exercise}
    For the above example, how many Dirichlet characters modulo $h$ are there?
\end{exercise}

\begin{theorem}[Orthogonality]
    \label{thm:dirichlet-orthogonality-poly}
    Let $\chi, \psi$ be Dirichlet characters modulo $h$ and $a, b \in A$ be such that $\gcd(a, h) = \gcd(b, h) = 1$.
    Then we have
    \begin{enumerate}
        \item
        \begin{equation}
            \sum_{a \in A / hA} \chi(a) \overline{\psi(a)} = \begin{cases}
                \phi(h) & \text{if } \chi = \psi \\
                0 & \text{if } \chi \ne \psi
            \end{cases}
        \label{eqn:dirichlet-orthogonality1}
        \end{equation}
        \item
        \begin{equation}
            \sum_{\chi \in X_h} \chi(a) \overline{\chi(b)} = \begin{cases}
                \phi(h) & \text{if } a \equiv b \pmod{h} \\
                0 & \text{otherwise}
            \end{cases}
        \label{eqn:dirichlet-orthogonality2}
        \end{equation}
    \end{enumerate}
    Here $\phi(h) = \#(A / hA)^\times$ is the Euler $\phi$-function defined previously.
\end{theorem}
\begin{proof}
    We will only prove \eqref{eqn:dirichlet-orthogonality1}, and leave the proof of \eqref{eqn:dirichlet-orthogonality2} as an exercise.
    Note that we only need to consider $a \in (A / hA)^\times$ since $\chi(a) = 0$ and $\psi(a) = 0$ if $\gcd(a, h) \ne 1$.
    If $\chi = \psi$, then we have $\chi(a)\overline{\chi(a)} = 1$ for all $a \in (A / hA)^\times$, and hence the sum becomes $\#(A / hA)^\times = \phi(h)$.
    If $\chi \ne \psi$, there exists $a_0 \in (A / hA)^\times$ such that $\chi(a_0) \ne \psi(a_0)$, i.e. $\chi(a_0) \overline{\psi(a_0)} = \chi(a_0) \psi(a_0)^{-1} \ne 1$.
    Since the map $a \mapsto a a_0$ is a bijection on $(A / hA)^\times$, we have
    \[
    \sum_{a \in (A / hA)^\times} \chi(a) \overline{\psi(a)} = \sum_{a \in (A / hA)^\times} \chi(a a_0) \overline{\psi(a a_0)} = \chi(a_0)\overline{\psi(a_0)}\sum_{a \in (A / hA)^\times} \chi(a) \overline{\psi(a)}
    \]
    and $\chi(a_0) \overline{\psi(a_0)} \ne 1$ implies that the sum is zero.
\end{proof}

\begin{exercise}
    Prove \eqref{eqn:dirichlet-orthogonality2}.\footnote{Hint: When $a \not\equiv b \pmod{h}$, there exists $\chi$ such that $\chi(a) \ne \chi(b)$.}
\end{exercise}

\begin{definition}
    For a Dirichlet character $\chi$ modulo $h$, we define the \emph{Dirichlet L-function} of $\chi$ as
    \begin{equation}
        L(s, \chi) = \sum_{\substack{0 \ne f \in A \\ f\text{ monic}}} \frac{\chi(f)}{|f|^s}
        \label{eqn:dirichlet-l-function}
    \end{equation}
\end{definition}

Using orthogonality, we can show that Dirichlet L-functions for nontrivial characters are simply polynomials in $p^{-s}$.
\begin{proposition}
    \label{prop:dirichlet-l-function-poly}
    Let $\chi$ be a nontrivial Dirichlet character modulo $h$.
    Then the Dirichlet L-function $L(s, \chi)$ is a polynomial in $p^{-s}$ of degree at most $\deg(h) - 1$.
\end{proposition}
\begin{proof}
    Define
    \begin{equation}
        A(n, \chi) = \sum_{\substack{\deg f = n \\ f\text{ monic}}} \chi(f)
    \end{equation}
    so that
    \begin{equation}
        L(s, \chi) = \sum_{n = 0}^{\infty} A(n, \chi) p^{-ns}
    \end{equation}
    Then our goal reduces to showing that $A(n, \chi) = 0$ for all $n \geq \deg(h)$.
    If $\deg(f) = n$, it can be unique written as $f = g h + r$ where $\deg(r) < \deg(h)$ or $r = 0$, and $\deg(g) = n - \deg(h)$.
    For each $r$, we have $p^{n - \deg(h)}$ possibilities for $g$, and $\chi(f) = \chi(r)$.
    Therefore, we have
    \[
    A(n, \chi) = p^{n - \deg (h)}\sum_{\deg(r) < \deg(h)} \chi(r) = p^{n - \deg (h)} \sum_{r \in A / hA} \chi(r) = 0
    \]
    where the summation is zero by the orthogonality of Dirichlet characters \eqref{eqn:dirichlet-orthogonality1} ($\chi$ is nontrivial).
\end{proof}

\begin{exercise}
    Is it always true that $A(\deg(h) - 1, \chi) \ne 0$, i.e. the degree of $L(s, \chi)$ is exactly $\deg(h) - 1$?
    If not, give a counterexample.
\end{exercise}

Since $\chi$ is multiplicative, Dirichlet $L$-function admits Euler product formula:
\begin{proposition}
    \label{prop:dirichlet-l-function-poly-euler}
    Let $\chi$ be a Dirichlet character modulo $h$.
    Then we have
    \begin{equation}
        L(s, \chi) = \prod_{P \nmid h} (1 - \chi(P) |P|^{-s})^{-1}
    \end{equation}
    where $P$ runs over all irreducible monic polynomials not dividing $h$.
\end{proposition}

\begin{exercise}
    Prove Proposition \ref{prop:dirichlet-l-function-poly-euler}.
\end{exercise}

\begin{example}
    \label{ex:dirichlet-l-function}
    Consider the Example \ref{ex:dirichlet-character1} with $p = 3$ and $h(T) = T$.
    By Proposition \ref{prop:dirichlet-l-function-poly}, $L(s, \chi)$ is a polynomial of degree at most $0$, i.e. a constant.
    One can directly check that $A(0, \chi) = 1$\footnote{2 is not monic!}, hence $L(s, \chi) = 1$.
\end{example}

\begin{example}
    Consider the Example \ref{ex:dirichlet-character2} with $p = 2$ and $h(T) = T^2 + T + 1$.
    By Proposition \ref{prop:dirichlet-l-function-poly}, $L(s, \chi)$ is a polynomial of degree at most $1$.
    One can directly check that $A(0, \chi) = 1$ and $A(1, \chi) = \zeta_3 + \zeta_3^2 = -1$, hence $L(s, \chi) = 1 - 2^{-s}$.
\end{example}

When $\chi = \chi_0$ is the trivial Dirichlet character modulo $h$, the Euler factorization of $L(s, \chi_0)$ is almost the same as \eqref{eqn:euler-factorization-poly} except that we exclude the factor corresponding to $P$ with $P \mid h$, hence
\begin{equation}
    L(s, \chi_0) = \prod_{P \mid h} (1 - |P|^{-s}) \cdot \zeta_A(s) = \frac{\prod_{P \mid h} (1 - |P|^{-s})}{1 - p^{1 - s}}.
    \label{eqn:dirichlet-l-function-chi0-euler}
\end{equation}

Our main goal is to prove $L(1, \chi) \ne 0$ for all nontrivial Dirichlet characters $\chi$, as in the integer case.
We need the following lemma.
\begin{lemma}
    Let $\chi$ vary over all Dirichlet characters modulo $h$.
    Then, for each irreducible polynomial $P$ not dividing $h$, there exist positive integers $f_P$ and $g_P$ such that $f_P g_P = \phi(h)$ and
    \begin{equation}
        \prod_{\chi \in X_h} L(s, \chi) = \prod_{P \nmid h} (1 - |P|^{-f_P s})^{-g_P}.
        \label{eqn:dirichlet-l-function-prod-euler-factor}
    \end{equation}
\end{lemma}
\begin{proof}
    For each $P \nmid h$, the map $\chi \mapsto \chi(P)$ gives a group homomorphism $X_h \to \bC^\times$.
    Its image is a finite order subgroup of $\bC^\times$, and hence it is cyclic.
    Let $f_P$ be the order of the image, and $g_P$ be the order of the kernel, so that $f_P g_P = \phi(h)$.
    Then the $P$-th Euler factor of $\prod_\chi L(s, \chi)$ is given by
    \[
    \prod_{j=0}^{f_P - 1} (1 - \zeta_{f_P}^j |P|^{-s})^{-g_P} = (1 - |P|^{-f_P s})^{-g_P}.
    \]
    Here $\zeta_{f_P}$ is a primitive $f_P$-th root of unity.
    Multiplying over all irreducible polynomials $P \nmid h$ gives us the desired formula.
\end{proof}

Using this, we can prove the nonvanishing result.
\begin{lemma}
    Suppose $\chi$ is a complex Dirichlet character modulo $h$, i.e. $\overline{\chi} \ne \chi$.
    Then $L(1, \chi) \ne 0$.
\end{lemma}
\begin{proof}
    Assume $L(1, \chi) = 0$.
    Consider \eqref{eqn:dirichlet-l-function-prod-euler-factor} for $h$.
    Among the factors of the LHS of \eqref{eqn:dirichlet-l-function-prod-euler-factor}, $L(s, \chi_0)$ has a simple pole at $s = 1$ and $L(s, \chi)$ has a zero at $s = 1$, so is $L(s, \overline{\chi})$.
    Since $L(s, \chi)$ is regular (i.e. no pole) at $s = 1$ for all nontrivial character $\chi$, the LHS a zero at $s = 1$.
    However, the RHS is positive for all $s > 1$; each Euler factor has positive coefficients as
    \[
    (1 - |P|^{-f_P s})^{-g_P} = \sum_{k \ge 0} \binom{g_P + k - 1}{g_P - 1} |P|^{-f_P k s}.
    \]
    By taking the limit as $s \to 1$, we get a contradiction.
\end{proof}

We need to be more careful about the case when $\chi$ is a real Dirichlet character modulo $h$, i.e. $\overline{\chi} = \chi$ (a quadratic character).
\begin{lemma}
    \label{lem:dirichlet-nonvanishing-real}
    Suppose $\chi$ is a nontrivial real Dirichlet character modulo $h$.
    Then $L(1, \chi) \ne 0$.
\end{lemma}
\begin{proof}
    The main idea is to consider
    \begin{equation}
    \label{eqn:G}
    G(s) = \frac{L(s, \chi_0) L(s, \chi)}{L(2s, \chi_0)} = \sum_{f} \frac{a(f)}{|f|^{s}}
    \end{equation}
    where $\chi_0$ is the trivial Dirichlet character modulo $h$.
    The $P$-th Euler factor of $G(s)$ for $P \nmid h$ is
    \[
    \begin{cases}
        \frac{1 + |P|^{-s}}{1 - |P|^{-s}} = 1 + 2 \sum_{k \ge 1} |P|^{-s} & \text{if }\chi(P) = 1 \\
        1 & \text{if }\chi(P) = -1
    \end{cases}
    \]
    so $G(s)$ has nonnegative coefficients, i.e. $a(f) \ge 0$ for all $f$.
    One can rewrite \eqref{eqn:G} as
    \begin{equation}
        \frac{(1 - pu^2) L^\ast(u, \chi)}{1 - pu} = \sum_{n \ge 0} A(n) u^n,
        \label{eqn:G2}
    \end{equation}
    where $u = p^{-s}$, $L^\ast(u, \chi) = L(s, \chi)$, and $A(n) = \sum_{\deg f = n} a(f)$.
    By Proposition \ref{prop:dirichlet-l-function-poly}, $L^\ast(u, \chi)$ is a polynomial in $u$ (of degree at most 1).
    Now, assume that $L(1, \chi) = 0$, i.e. $L^\ast(p^{-1}, \chi) = 0$.
    Then $1 - pu$ divides $L^\ast(u, \chi)$ and the LHS of \eqref{eqn:G2} is a polynomial.
    Thus RHS is also a polynomial, where all the coefficients are nonnegative since $a(f)$'s are.
    However, the LHS has a zero $u = p^{1/2}$, which is implossible for polynomials with nonnegative coefficients.
\end{proof}

\begin{exercise}
    Let $\chi$ be a nontrivial real Dirichlet character modulo $h \in A$.
    Prove that there exists $c \in \bF_p$ such that $\chi(T + c) = 1$.\footnote{Hint: This is equivalent to $L(1, \chi) \ne 0$. In other words, proving this without using the nonvanishing result would give a new proof of Lemma \ref{lem:dirichlet-nonvanishing-real}.}
\end{exercise}

\begin{theorem}[Dirichlet's theorem for polynomials]
    \label{thm:dirichlet-poly}
    Let $h \in A$ be a monic polynomial of degree $d \ge 1$.
    Let $a \in A$ be a polynomial such that $\gcd(a, h) = 1$.
    Then the set of irreducible monic polynomials $f \in A$ such that $f \equiv a \pmod{h}$ has Dirichlet density $\frac{1}{\phi(h)} > 0$.
    Especially, there are infinitely many such polynomials.
\end{theorem}
\begin{proof}
    From Proposition \ref{prop:dirichlet-l-function-poly-euler} and $-\log(1 - x) = \sum_{k \ge 1} \frac{x^k}{k}$, we have
    \begin{equation}
        \log L(s, \chi) = - \sum_{P} \log\left(1 - \frac{\chi(P)}{|P|^{s}}\right) = \sum_{P} \frac{\chi(P)}{|P|^s} + \sum_{P, k \ge 2} \frac{\chi(P)^k}{k |P|^{ks}} = \sum_{P} \frac{\chi(P)}{|P|^s} + R(s, \chi)
    \end{equation}
    where $R(s, \chi)$ is bounded as $s \to 1$.
    Now, multiplying $\overline{\chi(a)}$ and summing over $\chi$ gives
    \begin{equation}
        \sum_\chi \overline{\chi(a)} \log L(s, \chi) = \phi(h) \sum_{P \equiv a \pmod{h}} \frac{1}{|P|^s} + R(s)
    \end{equation}
    where $R(s) = \sum_\chi \overline{\chi(a)} R(s, \chi)$ is still remain bounded as $s \to 1$.
    Dividing by $\sum_{P} 1 / |P|^{s}$ gives
    \begin{equation}
        \frac{\log L(s, \chi_0)}{\sum_{P} 1 / |P|^s} + \sum_{\chi \ne \chi_0} \overline{\chi(a)} \cdot \frac{\log L(s, \chi_0)}{\sum_{P} 1 / |P|^s} = \phi(h) \frac{\sum_{P \equiv a \pmod{h}} 1 / |P|^s}{\sum_{P} 1 / |P|^s} + \frac{R(s)}{\sum_{P} 1 / |P|^s}
    \end{equation}
    Using \eqref{eqn:dirichlet-l-function-chi0-euler} and $L(1, \chi) \ne 0$ for $\chi \ne \chi_0$, one can show that the first term of the LHS of the above equation converges to 1, while the other terms correspond to the nontrivial characters converges to 0, as $s \to 1$.
    This gives the desired result.
\end{proof}

\begin{corollary}
    Let $h \in A$ be a monic polynomial of degree $d \ge 1$.
    For each $a \in A$ such that $\gcd(a, h) = 1$ and $n \ge 0$, consider the set of irreducible monic polynomials
    \begin{equation}
        S_n(a, h) := \{P \in A : P \equiv a\Mod{m}, \,\deg(P) = n\}.
    \end{equation}
    Then we have
    \begin{equation}
        S_n(a, h) = \frac{1}{\phi(h)} \frac{p^n}{n} + O\left(\frac{p^{n/2}}{n}\right).
        \label{eqn:dirichlet-poly-count-deg}
    \end{equation}
\end{corollary}
If we denote as $S_n$ for the set of irreducible monic polynomials of degree $n$, then \eqref{eqn:dirichlet-poly-count-deg} implies
\[
\lim_{n \to \infty} \frac{\# S_n(a, h)}{\# S_n} = \frac{1}{\phi(h)},
\]
i.e. the \emph{natural density} of $S_n(a, h)$ is $\frac{1}{\phi(h)}$.
\begin{proof}
    Proof can be found in \cite[p. 40, Theorem 4.8]{rosen2013number}.
    Note that it uses Riemann hypothesis for $L(s, \chi)$, which will be covered later.
\end{proof}

\begin{exercise}[{\cite[p. 43, Exercise 6]{rosen2013number}}]
    Let $f_0(T) \in A$ be a polynomial of degree $d$ with a non-zero constant term.
    Show that there are infinitely many irreducible monic polynomials $f(T) \in A$ whose first $m + 1$ terms coincide with $f_0(T)$.
    What is the Dirichlet density of such irreducible polynomials?
\end{exercise}


\subsection{More precise formula and Chebyshev's bias}
\label{sec:dirichlet-more-precise}

Now we know that Dirichlet's theorem is true for polynomials, i.e.the irreducible polynomials are equidistributed in the congruence classes modulo $h$.
In the previous section, we were able to prove an exact formula for the number of irreducible polynomials \eqref{eqn:counting_irred_poly_2}.
Then it is natural to ask if we can also prove an exact formula for the number of irreducible polynomials in a given congruence class modulo $h$.
We mainly work with the case when $p = 3$ and $h(T) = T$, but the same argument may generalizes to any prime $p$ and any polynomial $h$.

For each $n$, let $a_{n,1}$ and $a_{n,2}$ be the number of irreducible monic polynomials of degree $n$ in $\bF_3[T]$ that are congruent to $1$ and $2$ modulo $T$, respectively.
Our main goal is express the Dirichlet $L$-functions $L(s, \chi)$ and $L(s, \chi_0)$ in terms of $a_{n,1}$ and $a_{n,2}$, which eventually gives us the exact formulas for $a_{n,1}$ and $a_{n,2}$.
Here $\chi_0$ is the trivial Dirichlet character modulo $T$, which is defined by $\chi_0(f) = 1$ for all $f \in A$ with $\gcd(f, T) = 1$.
\begin{lemma}
    Let $\chi$ be the Dirichlet character modulo $T$ defined in Example \ref{ex:dirichlet-character1}.
    Then we have
    \begin{equation}
        1 = L(s, \chi) = \prod_{n \ge 1} (1 - 3^{-ns})^{-a_{n,1}} (1 + 3^{-ns})^{-a_{n,2}}
        \label{eqn:dirichlet-l-function-chi}
    \end{equation}
\end{lemma}
\begin{proof}
    The first equality is shown in Example \ref{ex:dirichlet-l-function}.
    The second equality follows from the Euler factorization of $L(s, \chi)$.
\end{proof}

\begin{lemma}
    Let $\chi_0$ be the trivial Dirichlet character modulo $T$.
    Then we have
    \begin{equation}
        \frac{1 - 3^{-s}}{1 - 3^{1 - s}} = L(s, \chi_0) = \prod_{n \ge 1} (1 - 3^{-ns})^{-a_{n,1} -a_{n,2}}.
        \label{eqn:dirichlet-l-function-chi0}
    \end{equation}
\end{lemma}
\begin{proof}
    Both of the equalities follow from Euler factorization of $L(s, \chi_0)$.
    On the one hand, the Euler factorization of $L(s, \chi_0)$ is the same as \eqref{eqn:euler-factorization-poly} except that we exclude the factor corresponding to $T$, which is $(1 - 3^{-s})^{-1}$. This gives us the first equality.
    On the other hand, grouping the factors by degree gives us the second equality.
\end{proof}

Now, we can divide \eqref{eqn:dirichlet-l-function-chi0} by \eqref{eqn:dirichlet-l-function-chi} to get
\begin{align*}
    \frac{1 - u}{1 - 3u} &= \prod_{n \ge 1} \left(\frac{1 + u^n}{1 - u^n}\right)^{a_{n,2}}.
\end{align*}
for $u = 3^{-s}$.
By taking logarithmic derivative, we have
\begin{align*}
    -\frac{1}{1 - u} + \frac{3}{1 - 3u} &= -(1 + u + u^2 + \cdots) + (3 + 3^2 u + 3^3 u^2 + \cdots) \\
    &= \sum_{n \ge 1} a_{n,2} \left(\frac{nu^{n-1}}{1 + u^n} + \frac{nu^{n-1}}{1 - u^n}\right) \\
    &= \sum_{n \ge 1} 2 n a_{n,2} u^{n-1} (1 + u^{2n} + u^{4n} + \cdots)
\end{align*}
and hence
\[
\sum_{n \ge 1} (3^n - 1) u^n = \sum_{n \ge 1} 2 n a_{n,2} (u^n + u^{3n} + u^{5n} + \cdots) = \sum_{n \ge 1} \left(\sum_{\substack{d \mid n \\ n/d \text{ odd}}} 2 d a_{d,2}\right) u^n.
\]
By comparing the coefficients, we have
\begin{equation}
    \frac{3^n - 1}{2}  = \sum_{\substack{d \mid n \\ n/d \text{ odd}}} d a_{d,2}.
\end{equation}
By applying the Möbius inversion formula, we can express $a_{n,2}$ in terms of $n$:
\begin{theorem}
    We have
    \begin{align}
        a_{n,1} &= \frac{1}{2n} \left(\sum_{2 \nmid d \mid n} \mu(d) \cdot 3^{\frac{n}{d}} + 2 \sum_{2 \mid d \mid n} \mu(d) \cdot 3^{\frac{n}{d}} + \epsilon_n \right) \label{eqn:anp_formula}\\
        a_{n,2} &= \frac{1}{2n} \left(\sum_{2 \nmid d \mid n} \mu(d) \cdot 3^{\frac{n}{d}} - \epsilon_n\right) \label{eqn:anm_formula}
    \end{align}
    where
    \begin{equation}
        \epsilon_n = \begin{cases}
            1 & n = 2^k \text{ for some } k \ge 0 \\
            0 & \text{otherwise}
        \end{cases}
    \end{equation}
\end{theorem}
\begin{proof}
    Equation \ref{eqn:anm_formula} follows from the above discussion, and \eqref{eqn:anp_formula} follows from the fact that $a_{n,1} + a_{n,2} = a_n$ where the formula for $a_n$ is given in \eqref{eqn:counting_irred_poly_2}.
\end{proof}
Not only this explains the exact number of irreducible polynomials in each congruence class modulo $T$, but also it shows that one of them is larger than the other, even if they are asymptotically equal as $n \to \infty$.
\begin{corollary}
    We have $a_{n,2} \ge a_{n,1}$ for all $n \ge 1$, and the equality holds if and only if $n$ is odd.
\end{corollary}
\begin{proof}
    It is clear that $a_{n,1} = a_{n,2}$ when $n$ is odd.
    For $n$ even, we have
    \begin{align*}
    a_{n,2} - a_{n,1} &= \frac{1}{n} \left(- \sum_{2 \mid d \mid n} \mu(d)\cdot 3^{\frac{n}{d}} - \epsilon_n\right) = \frac{1}{n} \left(-\sum_{d \mid \frac{n}{2}} \mu(2d) \cdot 3^{\frac{n}{2d}} - \epsilon_n\right) \\
    &\ge \frac{1}{n} \left(3^{\frac{n}{2}} - 3^{\frac{n}{2} - 1} - \cdots - 3 - 1\right) > 0.
    \end{align*}
\end{proof}

In other words, although $a_{n,1}$ and $a_{n,2}$ are asymptotically same as $n \to \infty$, one of them is always larger than or equal to the other.
Such a \emph{bias} is oberved by Chebyshev for the primes congruent to $1$ and $3$ modulo $4$ \cite{chebyshev1853lettre}.
Let $\pi(x;m,a)$ be the number of prime numbers $p \le x$ such that $p \equiv a \pmod{m}$.
Chebyshev found that $\pi(x;4, 3) > \pi(x;4,1)$ occurs much more often than $\pi(x;4,1) > \pi(x;4,3)$.
It might be even plausible to conjecture that the inequality is true for sufficiently large $x$.
However, this turned out to be false:
\begin{theorem}[{Littlewood \cite{littlewood1914distribution}}]
    There are infinitely many $x$ such that $\pi(x;4,1) > \pi(x;4,3)$.
    More precisely, we have
    \[
    \pi(x;4,1) - \pi(x;4,3) > \frac{1}{2} \frac{\sqrt{x}}{\log x} \log \log \log x
    \]
    for infinitely many $x$.
\end{theorem}

The second thing we can guess is that the inequality $\pi(x;4,3) > \pi(x;4,1)$ might be true for almost all $x$, i.e. the density of the set $\{x : \pi(x;4,3) > \pi(x;4,1)\}$ is $1$.
However, this is also false; in fact, Rubinstein and Sarnak \cite{rubinstein1994chebyshev} showed that the \emph{natural density} of the set $\{x : \pi(x;4,3) > \pi(x;4,1)\}$ does not exist.
Instead, they showed that \emph{logarithmic density} exists and very close to $1$.
\begin{theorem}[{Rubinstein--Sarnak \cite{rubinstein1994chebyshev}}]
    \[
    \lim_{X \to \infty} \frac{1}{\log X} \sum_{\substack{x \le X \\ \pi(x;4,3) > \pi(x;4,1)}} \frac{1}{x} = 0.9959\dots.
    \]
\end{theorem}

In the case of polynomials, Cha \cite{cha2008chebyshev} proved that Chebyshev's bias still holds for odd characteristics, under certain assumption on the zeros of $L$-functions (\emph{Grand Simplicity Hypothesis, GSH}).
For more story about Chebyshev's bias, see the excellent article of Granville and Martin \cite{granville2006prime}.

\begin{exercise}\sage
    Reproduce Chebyshev's observation by plotting the graphs of $\pi(x;4,1)$ and $\pi(x;4,3)$.
\end{exercise}

\begin{exercise}
    Adapt the same argument to the Example \ref{ex:dirichlet-character2}, where $p = 2$ and $h(T) = T^2 + T + 1$.
    For each $a = 1, T, T + 1$, find formulas for the number of irreducible polynomials of degree $n$ in $\bF_2[T]$ that are congruent to $a$ modulo $h$.
    Which is the largest and which is the smallest?\footnote{Hint: You may need to consider three $L$-functions $L(s, \chi)$, $L(s, \chi^2)$, and $L(s, \chi^3)$ where $\chi^3$ is the trivial Dirichlet character modulo $h$.}
\end{exercise}