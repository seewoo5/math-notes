\appendix
\section{Appendix}


\subsection{Dictionary between integers and polynomials}
\label{subsec:dictionary}

Here we summarize the dictionary between the integers and the polynomials over finite fields.
Here all $g, g_1, g_2, \dots$ on the $A$-side are all irreducible polynomials over $\bF_p$.
Also, we omit all the technical assumptions for each theorem, which can be found in the main text.

\begin{table}[h]
% \scriptsize
    \begin{center}
        \begin{tabular}{c|c|c}
            \toprule
            & $\bZ$ & $A = \bF_p[T]$ \\
            \midrule
            indecomposable & prime & irreducible \\
            number of units &$2 = \#(\bZ^\times)$ & $p - 1 = \# (\bF_p[T]^\times) = \# (\bF_p^\times)$ \\
            normalized & $\bN$ & monic polynomials \\
            absolute value & $|n| = \# (\bZ / n\bZ)$ & $|f| = \# (A / f A) = p^{\deg (f)}$ \\
            Euler $\varphi$ function & $\varphi(n) = \# (\bZ / n \bZ)^\times$ & $\varphi(f) = \# (A / fA)^\times$ \\
            Fermat's little theorem & $a^{p-1} \equiv 1 \pmod{p}$ & $f^{|g| - 1} \equiv 1 \pmod{g}$ \\
            Euler's theorem & $a^{\varphi(n)} \equiv 1 \pmod{n}$ & $f^{\varphi(f_0)} \equiv 1 \pmod{f_0}$ \\
            Wilson's theorem & $(p-1)! \equiv -1 \pmod{p}$ & $\prod_{f \in (A / g A)^\times} f \equiv -1 \pmod{g}$ \\
            Quadratic reciprocity & $\left(\frac{p}{q}\right)\left(\frac{q}{p}\right) = (-1)^{\frac{p-1}{2}\frac{q-1}{2}}$ & $\left(\frac{g_1}{g_2}\right)\left(\frac{g_2}{g_1}\right) = (-1)^{\frac{|g_1| - 1}{2}\frac{|g_2| - 1}{2}}$\\
            Riemann Zeta function & $\zeta(s) = \sum_{n \ge 1}\frac{1}{n^s}$ & $\zeta_{A}(s) = \sum_{\substack{f \ne 0, \text{ monic}}} \frac{1}{|f|^s}$ \\
            Euler factorization & $\zeta(s) = \prod_{p} (1 - p^{-s})^{-1}$ & $\zeta_A(s) = \prod_{g} (1 - |g|^{-s})^{-1}$ \\
            M\"obius function & $\mu(p_1 \cdots p_k) = (-1)^k$ & $\mu(g_1 \cdots g_k) = (-1)^k$ \\
            Sum of arithmetic functions & $\sum_{n \le x} a_n$ & $\sum_{f, \deg(f) = n} a_f$ \\
            Dirichlet series & $L(s) = \sum_{n \ge 1} \frac{a_n}{n^s}$ & $L(s) = \sum_{0 \ne f,\text{ monic}} \frac{a_f}{|f|^s}$ \\
            \bottomrule
        \end{tabular}
        \caption{Integers and Polynomials.}
        \label{tab:dictionary}
    \end{center}
\end{table}

% \newpage

% \subsection{Handbook of Galois theory for finite fields}
% \label{subsec:handbook_galois_ff}

% \begin{theorem}
%     Let $p$ be a prime number, and let $n$ be a positive integer.
%     Up to isomorphism, there is a unique finite field of order $p^n$, which we denote by $\bF_{p^n}$.
%     It can be constructed as a splitting field of the polynomial $T^{p^n} - T$ over $\bF_p$.
% \end{theorem}

% \begin{exercise}
%     \begin{enumerate}
%         \item Let $p = 3$. Prove that both $g_1(T) = T^2 + 1$ and $g_2(T) = T^2 + T + 2$ are irreducible polynomials over $\bF_3$.
%         \item Find all isomorphisms between the finite fields $\bF_3[T] / (g_1(T))$ and $\bF_3[T] / (g_2(T))$.
%         In other words, find all polynomials $h(T) \in \bF_3[T] / (g_2(T))$ such that $g_1(h(T))$ is a multiple of $g_2(T)$.
%         Then the map $T \mapsto h(T)$ gives an isomorphism between the two finite fields.\footnote{Hint: There are exactly two such polynomials (modulo $g_2(T)$).}
%     \end{enumerate}
% \end{exercise}

% \begin{theorem}
%     Let $S_{p}(d)$ be the set of all monic irreducible polynomials of degree $d$ over $\bF_p$.
%     Then the following holds:
%     \begin{equation}
%         T^{p^n} - T = \prod_{d \mid n} \prod_{f \in S_p(d)} f(T).
%     \end{equation}
%     In other words, the polynomial $T^{p^n} - T$ is equal to the product of all monic irreducible polynomials of degree $d$ over $\bF_p$, where $d$ divides $n$.
% \end{theorem}

% \begin{exercise}
%     Prove that the product of all monic irreducible polynomials of degree $3$ over $\bF_p$ is
%     \[
%         T^{(p^2 + p)(p - 1)} + \cdots + T^{2(p-1)} + T^{p-1} + 1.
%     \]
%     Can you generalize this to other degrees?
% \end{exercise}

% \begin{theorem}
%     The Galois group $\Gal(\bF_{p^n} / \bF_p)$ is cyclic of order $n$, generated by the Frobenius automorphism $\sigma: x \mapsto x^p$.
% \end{theorem}



% \newpage

% \subsection{Handbook of Sage functions}
% \label{subsec:handbook_sage}

% Here we give some useful Sage functions for working with finite fields and polynomials.
% First, finite fields can be defined as follows:
% \begin{minted}[fontsize=\footnotesize,framesep=2mm,bgcolor=lightgray!20]{python}
% k = GF(3)
% \end{minted}
% This defines the finite field $\bF_3$.
% If you want to define a finite field of order $p^n$, you can write as
% \begin{minted}[fontsize=\footnotesize,framesep=2mm,bgcolor=lightgray!20]{python}
% k.<a> = GF(3^2, 'a')
% \end{minted}
% where \texttt{'a'} is the name of the generator of the field.

