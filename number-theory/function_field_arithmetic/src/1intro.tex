\section{Introduction}
\label{sec:intro}

The goal of this note is to introduce the arithmetic of function fields, which is the analogue of number theory for polynomials.
Especially, our main goal is to study various evidences of the following claim:

\begin{myquote}
A theorem that holds for integers is also true for polynomials (over finite fields), and latter is often easier to prove.
\end{myquote}
For example, we will see a proof of Fermat's Last Theorem for polynomials, which only requires few pages to prove.

Dictionary between the integers and the polynomials over finite fields can be found in Table \ref{tab:dictionary} of Appendix.

\subsection{Prerequisites}
We assume that the readers are familiar with undergraduate level algebra (groups, rings, fields, etc.), number theory (congruences, prime numbers, etc.), and a bit of complex analysis.
Some of the theory of finite fields will be reviewed in Appendix \ref{subsec:handbook_galois_ff}.


\subsection{Notations}

Let $p$ be a prime number. We denote by $\bF_p$ the finite field of order $p$, which is the field with $p$ elements.
We denote the polynomial ring $\bF_p[T]$ by $A$.
For each nonzero polynomial $f \in A$, we denote it's norm by $|f| = p^{\deg (f)}$, where $\deg (f)$ is the degree of $f$, and we set $|0| = 0$.

\subsection{SageMath}

There are some codes in this note, which are mostly written in \href{https://www.sagemath.org/}{Sage}.
Sage is a free \href{https://github.com/sagemath/sage}{open-source} mathematics software system, which is built on top of many existing open-source packages and wrappedn in a Python interface.
You can run them online in \href{https://sagecell.sagemath.org/}{SageMathCell}, or install it on your computer.
Especially, a lot of number-theoretic functions are implemented in Sage, so it is much easier to experiment with it than writing your own code from scratch.
For example, to check if a large number is prime, you can simply run
\begin{minted}[fontsize=\footnotesize,framesep=2mm,bgcolor=lightgray!20]{python}
is_prime(10 ^ 9 + 7)
\end{minted}
Some of the exercies in this note are designed to be solved in Sage (or other programming languages that you are familiar with), and those exercises are marked with a Sage logo \sage.
Several useful Sage functions are listed in Appendix \ref{subsec:handbook_sage}.


\subsection*{Acknowledgements}


\begin{exercise}
    Prove that $\bZ$ is not a polynomial ring over a field. In other words, show that there is no field $k$ such that $\bZ \cong k[T]$ as rings.
\end{exercise}

\begin{exercise}
    Think about your favorite theorems in number theory, and try to find their polynomial analogues. Some of them may appear in this note, but some of them may not.
\end{exercise}


\newpage