\section{Introduction}
\label{sec:intro}

The goal of this note is to introduce the arithmetic of function fields, which is the analogue of number theory for polynomials.
Especially, our main goal is to study various evidences of the following claim:

\begin{myquote}
A theorem that holds for integers is also true for polynomials (over finite fields), and latter is often easier to prove.
\end{myquote}
For example, we will see a proof of Fermat's Last Theorem for polynomials, which only requires few pages to prove.

Dictionary between the integers and the polynomials over finite fields can be found in Table \ref{tab:dictionary} of Appendix.

\subsection*{Notations}

Let $p$ be a prime number. We denote by $\bF_p$ the finite field of order $p$, which is the field with $p$ elements.
We denote the polynomial ring $\bF_p[T]$ by $A$.
For each nonzero polynomial $f \in A$, we denote it's norm by $|f| = p^{\deg (f)}$, where $\deg (f)$ is the degree of $f$, and we set $|0| = 0$.


\subsection*{Exercises}
\begin{enumerate}
    \item Prove that $\bZ$ is not a polynomial ring over a field. In other words, show that there is no field $k$ such that $\bZ \cong k[T]$ as rings.
    \item Think about your favorite theorems in number theory, and try to find their polynomial analogues. Some of them may appear in this note, but some of them may not.
\end{enumerate}