\section{Reciprocity Laws}
\label{sec:reciprocity}

Quadratic reciprocity is one of the most important theorem in number theory, which is the origin of all modern number theory including algebraic number theory and Langlands program.
In this section, we will see the analogue of quadratic reciprocity for function fields.
Especially, we will present the higher reciprocity laws, which is much more complicated for integer case.

\subsection{Quadratic Reciprocity for Integers}
\label{subsec:quadratic-reciprocity}

CRT essentially tells us how to solve \emph{linear} congruences, i.e. equations of the form $ax + b \equiv 0 \pmod{n}$.
Now, let's go one step further and consider \emph{quadratic} congruences, i.e. equations of the form $a x^2 + bx + c\equiv 0 \pmod{n}$.
How can we solve such equations?
Recall that we solve usual quadratic eqautions by ``completing the square'', and we can do the same for congruences.\footnote{when $n$ is odd and $a$ is invertible modulo $n$.}
Hence we can reduce to a simpler question:

\begin{myquote}
For given integer $a$ and $n$, when does the congruence $x^2 \equiv a \pmod{n}$ have a solution?
\end{myquote}

Although the question seems pretty simple, the answer is quite deep and also interesting.
To get some intuition, let's first consider the case when $n = p$ is a prime number and $a = -1$.
In other words, we want to know when the congruence $x^2 \equiv -1 \pmod{p}$ has a solution (this is equivalent to one of the exercises in Section \ref{sec:basicnt}).
Let's just try some small primes.
Here's a Sage code to check the congruence $x^2 \equiv -1 \pmod{p}$ for primes $p < 100$:

\begin{minted}[fontsize=\footnotesize,framesep=2mm,bgcolor=lightgray!20]{python}
with_sol = []
without_sol = []
for p in primes(100):
    for a in range(p):
        if (a ^ 2 + 1) % p == 0:
            with_sol.append(p)
            break
        if a == p - 1:
            without_sol.append(p)
print("Primes with solution", with_sol)
print("Primes without solution", without_sol)
\end{minted}

Here's the output of the code you should get:
\begin{minted}[fontsize=\footnotesize,framesep=2mm,bgcolor=lightgray!20]{text}
Primes with solution [2, 5, 13, 17, 29, 37, 41, 53, 61, 73, 89, 97]
Primes without solution [3, 7, 11, 19, 23, 31, 43, 47, 59, 67, 71, 79, 83]
\end{minted}

Stare the lists carefully for a while. What do you notice?

\newpage
The answer is:
\begin{theorem}
    \label{thm:m1sq}
    Let $p$ be a prime number. Then the congruence $x^2 \equiv -1 \pmod{p}$ has a solution if and only if $p = 2$ or $p \equiv 1 \pmod{4}$.
\end{theorem}
\begin{proof}
    The case $p = 2$ is trivial, so we assume $p$ is odd.
    Assume thet the equation has a solution $x$.
    Then $x^4 \equiv (-1)^2 \equiv 1 \pmod{p}$, so $x$ is an element of order $4$ in $(\bZ / p\bZ)^\times$. 
    By Lagrange's theorem, the order of $x$ must divide $p - 1$, so $4 \mid (p - 1) \Leftrightarrow p \equiv 1 \pmod{4}$.

    Conversely, assume $p \equiv 1 \pmod{4}$ and let $g \in (\bZ / p \bZ)^\times$ be a generator of the group.
    Then $g^{(p - 1) / 4}$ is an element of order $4$, so we can set $x = g^{(p - 1) / 4}$ and $(x^2)^2 \equiv x^4 \equiv 1 \pmod{p}$.
    Since $x^2 \not\equiv 1 \pmod{p}$, we have $x^2 \equiv -1 \pmod{p}$.
\end{proof}

It is amazing that the solvability of the congruence equation is simply given by a congruence condition on the prime $p$.

Let's try other $a$, say $a = 3$.
Here's a Sage code to check the congruence $x^2 \equiv 3 \pmod{p}$ for primes $p < 100$:
\begin{minted}[fontsize=\footnotesize,framesep=2mm,bgcolor=lightgray!20]{python}
with_sol = []
without_sol = []
for p in primes(100):
    for a in range(p):
        if (a ^ 2 - 3) % p == 0:
            with_sol.append(p)
            break
        if a == p - 1:
            without_sol.append(p)
print("Primes with solution", with_sol)
print("Primes without solution", without_sol)
\end{minted}
Here's the output of the code you should get:
\begin{minted}[fontsize=\footnotesize,framesep=2mm,bgcolor=lightgray!20]{text}
Primes with solution [2, 3, 11, 13, 23, 37, 47, 59, 61, 71, 73, 83, 97]
Primes without solution [5, 7, 17, 19, 29, 31, 41, 43, 53, 67, 79, 89]
\end{minted}
Again, the question is: what do you notice?
The answer is slightly more complicated than the previous case, and you need to observe the numbers modulo $12$.
\begin{theorem}
    \label{thm:m3sq}
    Let $p$ be a prime number. Then the congruence $x^2 \equiv 3 \pmod{p}$ has a solution if and only if $p = 2, 3$ or $p \equiv 1, 11 \pmod{12}$.
\end{theorem}

To answer the question for general $a$, we need to introduce the Legendre symbol.
\begin{definition}[Legendre symbol]
    Let $p$ be a prime and $a$ be an integer.
    The \emph{Legendre symbol} $\left(\frac{a}{p}\right)$ is defined as follows:
    \begin{equation}
        \left(\frac{a}{p}\right) = \begin{cases}
            0 & a \equiv 0 \pmod{p} \\
            1 & \text{if } x^2 \equiv a \pmod{p} \text{ has a solution} \\
            -1 & \text{otherwise.}
        \end{cases}
        \label{eqn:legendre-symbol}
    \end{equation}
\end{definition}

Do not confuse it with fraction.
By definition, it tells you whether the congruence $x^2 \equiv a \pmod{p}$ has a solution or not.
But it does not tell you how to compute the symbol.
The following theorem gives a way to compute the Legendre symbol for odd primes.
\begin{theorem}[Euler's criterion]
    \label{thm:euler-criterion}
    Let $p$ be an odd prime and $a$ be an integer.
    Then
    \begin{equation}
        \left(\frac{a}{p}\right) = a^{\frac{p - 1}{2}} \pmod{p}.
        \label{eqn:euler-criterion}
    \end{equation}
\end{theorem}
\begin{proof}
    If $a \equiv 0 \pmod{p}$, then the result is trivial.
    Recall that the group $\bF_p^\times$ is cyclic of order $p - 1$.
    Let $g$ be a generator of the group.
    Then every nonzero element $a \in \bF_p^\times$ can be written as $a \equiv g^k \pmod{p}$ for some integer $k$.
    The equation is solvable if and only if $k$ is even modulo $p - 1$.
\end{proof}

\begin{theorem}[Quadratic reciprocity]
    \label{thm:quadratic-reciprocity}
    Let $p$ and $q$ be distinct odd prime numbers.
    Then
    \begin{equation}
        \left(\frac{p}{q}\right) \left(\frac{q}{p}\right) = (-1)^{\frac{(p-1)(q-1)}{4}}.
        \label{eqn:quadratic-reciprocity}
    \end{equation}
    Especially, the congruence $x^2 \equiv p \pmod{q}$ has a solution if and only if the congruence $x^2 \equiv q \pmod{p}$ has a solution, except when $p \equiv q \equiv 3 \pmod{4}$. 
\end{theorem}
There are tons of proofs of this theorem, and we will not give a proof here.
You can refer to an archive of proofs by Lemmermeyer \cite{lemmermeyer_qr}, or a book by Baumgart \cite{baumgart2015quadratic}.

We also need the following supplementary results to compute the Legendre symbol completely.
\begin{theorem}
    \label{thm:quadratic-reciprocity-supplementary}
    For odd prime $p$, we have
    \begin{align*}
        \left(\frac{-1}{p}\right) &= (-1)^{\frac{p - 1}{2}} = \begin{cases} 1 & p \equiv 1 \pmod{4} \\ -1 & p \equiv 3 \pmod{4} \end{cases} \\
        \left(\frac{2}{p}\right) &= (-1)^{\frac{p^2 - 1}{8}} = \begin{cases}
            1 & p \equiv 1, 7 \pmod{8} \\
            -1 & p \equiv 3, 5 \pmod{8}
        \end{cases}
    \end{align*}
\end{theorem}
\begin{proof}
    We already proved the first in Theorem \ref{thm:m1sq}.
    For the second here is a proof brought from \cite[Theorem]{serre2012course}.
    Let $\alpha \in \bF_{p^2}$ be a primitive $8$-th root of unity (note that $p^2 - 1$ is divisible by $8$ for any odd $p$).
    then $\alpha^4 = -1$ and $y = \alpha + \alpha^{-1}$ satisfies $y^2 = 2 + \alpha^2 + \alpha^{-2} = 2$.
    By binomial expansion, we have $y^p = \alpha^p + \alpha^{-p}$ modulo $p$, hence $y^p = y$ for $p \equiv \pm 1\pmod{8}$ and this gives $\left(\frac{2}{p}\right) = 1$ for such $p$.
    If $p \equiv \pm 3 \pmod{8}$, then $y^p = -y \Leftrightarrow y^{p-1} = -1$, hence $\left(\frac{2}{p}\right) = y^{p-1} = -1$.\footnote{It may look random to consider such $y$, but this can be thought as an analogue of the identity of complex numbers $\sqrt{2} = e^{2\pi i / 8} + e^{-2 \pi i / 8}$ in finite fields.}
\end{proof}

We can use quadratic reciprocity to prove Theorem \ref{thm:m3sq}.
\begin{proof}[Proof of Theorem \ref{thm:m3sq}]
    By Theorem \ref{thm:quadratic-reciprocity}, we have
    \[
    \left(\frac{3}{p}\right) \left(\frac{p}{3}\right) = (-1)^{\frac{(p-1)(3-1)}{4}} = (-1)^{\frac{(p-1)}{2}} \Leftrightarrow \left(\frac{3}{p}\right) = (-1)^{\frac{(p-1)}{2}} \left(\frac{p}{3}\right).
    \]
    We can compute each factor on the right-hand side seperately, which are
    \begin{align*}
        \left(\frac{p}{3}\right) &= \begin{cases} 1 & p \equiv 1 \pmod{3} \\ -1 & p \equiv 2 \pmod{3} \end{cases}, \\
        (-1)^{\frac{(p-1)}{2}} &= \begin{cases} 1 & p \equiv 1 \pmod{4} \\ -1 & p \equiv 3 \pmod{4} \end{cases}.
    \end{align*}
    By combining these two congruences, we can see that
    \[
    \left(\frac{3}{p}\right) = \begin{cases}
        1 & p \equiv 1, 11 \pmod{12} \\
        -1 & p \equiv 5, 7 \pmod{12}
    \end{cases}
    \]
    for $p \neq 2, 3$.
\end{proof}

\begin{exercise}
    When $p \equiv 1 \pmod{4}$, we know that the congruence $x^2 \equiv -1 \pmod{p}$ has a solution by Theorem \ref{thm:m1sq}.
    Show that $x = \left(\frac{p-1}{2}\right)!$ becomes a solution.\footnote{Hint: Wilson's theorem.}
\end{exercise}

\begin{exercise}
    Prove Theorem \ref{thm:m3sq} without using quadratic reciprocity, by adapting the proof of Theorem \ref{thm:quadratic-reciprocity-supplementary}.
    Here's a sketch of a proof:
    \begin{enumerate}
        \item Show that, for any prime $p > 3$, there is a $12$-th root of unity $\beta$ in $\bF_{p^2}$.
        \item Show that $z = \beta + \beta^{-1}$ satisfies $z^2 = 3$.
        \item Show that $z^p = z$ if and only if $p \equiv 1, 11 \pmod{12}$, and $z^p = -z$ if and only if $p \equiv 5, 7 \pmod{12}$. Conclude that $\left(\frac{3}{p}\right) = 1$ if and only if $p \equiv 1, 11 \pmod{12}$.
    \end{enumerate}
    It is natural to ask if one can generalize this proof to other primes.
    This essentially gives a proof of quadratic reciprocity, and the correct analogue of $y$ or $z$ would be the \emph{Gauss sum}.
\end{exercise}

\begin{exercise}
    Characterize the prime $p$ such that the congruence $x^2 \equiv 5 \pmod{p}$ has a solution.\footnote{Hint: Use quadratic reciprocity. Answer will be given modulo $5$.}
\end{exercise}

\begin{exercise}
    For a prime $p$ and integers $a, b, c$, give a necessary and sufficient condition for the congruence $ax^2 + bx + c \equiv 0 \pmod{p}$ to have a solution.
\end{exercise}

\begin{exercise}
    \begin{enumerate}
        \item Find all prime $p < 100$ where the congruence $x^3 - 3x + 1 \equiv 0 \pmod{p}$ has a solution. Can you find a pattern? If you are brave enough, try to prove it.\footnote{Hint: modulo 9. To prove it, one can use the fact that the roots of $x^3 - 3x - 1$ are $2\cos (2\pi / 9), 2 \cos(4\pi/9), 2\cos(8\pi / 9)$, and try to express them in terms of $9$-th roots of unity.}
        \item Find all prime $p < 100$ where the congruence $x^3 - x - 1 \equiv 0 \pmod{p}$ has a solution. Can you find a pattern?
        \item Consider a power series
        \[
        A(T) = T \prod_{n \ge 1} (1 - T^n) (1 - T^{23 n}) = \sum_{n\ge 1} a_n T^n \in \bZ\llb T \rrb.
        \]
        Compute the first $100$ coefficients of $A(T)$, and check if there's any relation between the $p$-th coefficients $a_p$ and the congruence $x^3 - x - 1 \equiv 0 \pmod{p}$.\footnote{Asnwer: it has a solution in $\bF_p$ if and only if $a_p \in \{0, 2\}$, \emph{when $p \ne 23$}. Buzzword: Artin representation and modular form of weight 1, Nonabelian class field theory.}
    \end{enumerate}    
\end{exercise}


\subsection{Quadratic Reciprocity for Polynomials}
\label{subsec:quadratic-reciprocity-polynomials}

Let's move on to the polynomial case.
We are going to ask the same question as before:
\begin{myquote}
For given polynomial $f \in A$ and irreducible polynomial $P$, when does the congruence $x^2 \equiv f \pmod{P}$ have a solution?
\end{myquote}
We are going to adapt the exact same strategy as before.
\textbf{For this section, we assume that $p$ is an odd prime.}
Characteristic 2 case is slightly more complicated, and we will not discuss it here.
You can refer to Keith Conrad's note \cite{conrad_rec_2} for more details for $p = 2$.

Define the \emph{Legendre symbol} for polynomials as follows:
\begin{definition}[Legendre symbol for polynomials]
    Let $P \in A$ be an irreducible polynomial and $f \in A$ be a polynomial.
    The \emph{Legendre symbol} $\left(\frac{f}{P}\right)$ is defined as follows:
    \begin{equation}
        \left(\frac{f}{P}\right) = \begin{cases}
            0 & f \equiv 0 \pmod{P} \\
            1 & \text{if } x^2 \equiv f \pmod{P} \text{ has a solution in } A / P A, \\
            -1 & \text{otherwise.}
        \end{cases}
        \label{eqn:legendre-symbol-poly}
    \end{equation}
\end{definition}

We have Euler's criterion for polynomials as well.
\begin{theorem}[Euler's criterion for polynomials]
    \label{thm:euler-criterion-poly}
    Let $P \in A$ be an irreducible polynomial and $f \in A$ be a polynomial.
    Then
    \begin{equation}
        \left(\frac{f}{P}\right) = f^{\frac{|P| - 1}{2}} \pmod{P}.
        \label{eqn:euler-criterion-poly}
    \end{equation}
\end{theorem}
\begin{proof}
    
\end{proof}

\begin{theorem}[Quadratic reciprocity for polynomials]
    \label{thm:quadratic-reciprocity-poly}
    Let $P, Q \in A$ be distinct irreducible polynomials.
    Then
    \begin{equation}
        \left(\frac{P}{Q}\right) \left(\frac{Q}{P}\right) = (-1)^{\frac{|P| - 1}{2} \cdot \frac{|Q| - 1}{2}}.
        \label{eqn:quadratic-reciprocity-poly}
    \end{equation}
\end{theorem}
\begin{proof}
    This proof is due to Carlitz, which can be found in Rosen's book \cite[p. 25, Theorem 3.3]{rosen2013number} and Conrad's note \cite[Section 3]{conrad_rec_odd}.
    Let $k / \bF_p$ be a (finite) field contains all roots of $P$ and $Q$.
    Let $d_P = \deg P$ and $d_Q = \deg Q$.
    We will first show that \eqref{eqn:quadratic-reciprocity-poly} is equivalent to:
    \begin{equation}
        \left(\frac{P}{Q}\right) = (-1)^{\frac{(p-1) d_P d_Q}{2}} \left(\frac{Q}{P}\right).
        \label{eqn:quadratic-reciprocity-poly-cor}
    \end{equation} 
    From $|P| = p^{d_P}$,
    \begin{align*}
        \frac{|P| - 1}{2} = \frac{p^{d_P} - 1}{2} = \frac{p-1}{2} \cdot (p^{d_P-1} + p^{d_P-2} + \cdots + 1) \equiv \frac{(p-1) d_P}{2} \pmod{2},
    \end{align*}
    since $p$ is odd.
    Similar congruence holds for $Q$, hence
    \[
        \frac{|P| - 1}{2} \cdot \frac{|Q| - 1}{2} \equiv \frac{(p-1) d_P}{2} \cdot \frac{(p-1) d_Q}{2} \equiv \frac{p-1}{2} \cdot d_P d_Q \pmod{2}.
    \]
    Note that $n^2 \equiv n \pmod{2}$ for any integer $n$.
    
    Now, let $\alpha, \beta \in k$ be roots of $P$ and $Q$, respectively.
    Then all the zeros of $P$ and $Q$ are given by the $p$-powers of $\alpha$ and $\beta$ (images of Frobenius map): they factor as
    \begin{align*}
        P(T) = \prod_{i = 0}^{d_P - 1} (T - \alpha^{p^i}), \quad Q(T) = \prod_{j = 0}^{d_Q - 1} (T - \beta^{p^j}).
    \end{align*}
    By Euler's criterion for polynomials and properties of finite fields, we have
    \begin{align*}
        \left(\frac{P}{Q}\right) &\equiv P(T)^{\frac{|Q| - 1}{2}} \pmod{Q(T)} \\
        &\equiv P(T)^{\frac{p - 1}{2}(1 + p + \cdots + p^{d_Q - 1})} \pmod{Q(T)} \\
        &\equiv (P(T)P(T)^p \cdots P(T)^{p^{d_Q - 1}})^{\frac{p-1}{2}} \pmod{Q(T)} \\
        &\equiv (P(T) P(T^p) \cdots P(T^{p^{d_Q - 1}}))^{\frac{p-1}{2}} \pmod{Q(T)}
        %  \\
        % &\equiv (P(\beta) P(\beta^p) \cdots P(\beta^{p^{d_Q - 1}}))^{\frac{p-1}{2}} \pmod{Q(T)} \\
        % &\equiv \prod_{i=0}^{d_P - 1} \prod_{j=0}^{d_Q - 1} (\beta^{p^j} - \alpha^{p^i}) \pmod{Q(T)}
    \end{align*}
    If we set $T = \beta$ for the last congruence, it becomes
    \[
    \prod_{i=0}^{d_P - 1} \prod_{j=0}^{d_Q - 1} (\beta^{p^j} - \alpha^{p^i})^{\frac{p-1}{2}} \in \{\pm 1\}
    \]
    Now we can swap $\alpha$ and $\beta$, which gives $\left(\frac{Q}{P}\right)$ up to a sign
    \[
    (-1)^{\frac{(p-1) d_P d_Q}{2}}
    \]
    which is exactly matches with the exponent of $(-1)$ in \eqref{eqn:quadratic-reciprocity-poly-cor}.
\end{proof}
Note that the proof is very specific for polynomials over finite fields, and it does not work for integer case.
There are \emph{some} proofs of similar taste, which writes the product $\left(\frac{p}{q}\right)\left(\frac{q}{p}\right)$ as other function that is symmetric in $p$ and $q$.

\begin{exercise}
    For an irreducible polynomial $P \in A$ and polynomials $f, g, h \in A$, give a necessary and sufficient condition for the congruence $f x^2 + gx + h \equiv 0 \pmod{P}$ to have a solution (in $A$).
\end{exercise}


\newpage
