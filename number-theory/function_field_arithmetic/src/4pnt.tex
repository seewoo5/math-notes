\section{How many primes?}
\label{sec:how-many-primes}

In this section, we will study several questions on counting numbers and polynomials, e.g. number of prime numbers / irreducible polynomials up to certain bounds.
Over integers, many of such problems are wide open.
However, the polynomial analogue of these problems are often much easier to prove.
Especially, both of the sides uses certain \emph{zeta functions} or \emph{L-functions} to count certain objects, which becomes much simpler in the polynomial case.

\subsection{Prime number theorem and Riemann zeta function}
\label{subsec:prime-number-theorem}

We start with the most important fact about prime numbers, proven by Euclid few hundred years ago.
\begin{theorem}[Euclid]
    \label{thm:euclid}
    There are infinitely many prime numbers.
\end{theorem}
\begin{proof}
    Suppose that there are finitely many prime numbers $p_1, p_2, \dots, p_n$.
    Consider the number $N = p_1 p_2 \cdots p_n + 1$.
    Then $N$ is not divisible by any of the prime numbers $p_i$, so it must be either prime or divisible by some other prime number.
    This contradicts our assumption that there are only finitely many prime numbers.
\end{proof}

Once we know that there are infinitely many prime numbers, we can ask the following question:

\begin{myquote}
    For given real number $x > 0$, how many prime numbers are there less than or equal to $x$?
\end{myquote}

We follow the standard notation and denote the number of prime numbers less than or equal to $x$ by $\pi(x)$.
Hadamard and de la Vall\'ee-Poussin proved independently in 1896 that
\begin{theorem}[Prime number theorem]
    \label{thm:prime-number-theorem}
    We have
    \[
        \pi(x) \sim \frac{x}{\log x},
    \]
    where $\sim$ means that the ratio of the two sides tends to $1$ as $x$ tends to infinity.
\end{theorem}
In other words, the density of prime numbers among integers is approximately $\frac{1}{\log x}$, which tends to $0$ as $x$ tends to infinity.

\begin{exercise}\sage
    Plot the graphs of the functions $\pi(x)$ and $\frac{x}{\log x}$ for $x \in [1, 10^6]$.\footnote{Hint: Sage has \texttt{prime\_pi}.}
\end{exercise}


Although there are some elementary proofs by Selberg \cite{selberg1949elementary} and Erd\"os \cite{erdos1949new}, the original proofs and the most of other proofs (e.g. Newman's proof \cite{newman1980simple,zagier1997newman}) uses \emph{Riemann zeta function}.
In \cite{riemann1859ueber}, Riemann defined the zeta function:

\begin{definition}[Riemann zeta function]
    \label{def:riemann-zeta}
    The \emph{Riemann zeta function} is defined by
    \[
        \zeta(s) = \sum_{n=1}^\infty \frac{1}{n^s}
    \]
    for complex numbers $s$ with $\Re(s) > 1$.
\end{definition}
\begin{exercise}
    Why it converges for $\Re(s) > 1$?
\end{exercise}

By writing $\zeta(s)$ as an integral, Riemann showed that it can be analytically (more precisely, meromorphically) continued to the whole complex plane, except for a simple pole at $s = 1$.
\begin{theorem}
    \begin{enumerate}
        \item $\zeta(s)$ can be analytically continued to a meromorphic function on the whole complex plane, with a simple pole at $s = 1$ with residue $1$.
        \item Define the \emph{completed zeta function} as
        \begin{equation}
            \xi(s) = \pi^{-\frac{s}{2}} \Gamma\left(\frac{s}{2}\right) \zeta(s),
        \end{equation}
        where $\Gamma(s)$ is the gamma function.
        Then $\xi(1 - s) = \xi(s)$ for all $s \in \bC \backslash \{0, 1\}$.
    \end{enumerate}
\end{theorem}
\begin{proof}
    See Apostol \cite{apostol2013introduction} or Stein--Shakarchi \cite{stein2010complex}.
\end{proof}
This allow us to state the infamous \emph{Riemann Hypothesis}:
\begin{conjecture}[Riemann Hypothesis]
    \label{conj:riemann-hypothesis}
    All \emph{non-trivial} zeros of $\zeta(s)$ lie on the critical line $\Re(s) = \frac{1}{2}$.
\end{conjecture}
Here the \emph{trivial} zeros are the negative even integers, i.e. $s = -2, -4, -6, \dots$.

But why Riemann zeta function is related to prime numbers?
The following theorem by Euler is the key.
\begin{theorem}[Euler factorization]
    \label{thm:euler-factorization}
    The Riemann zeta function can be expressed as an infinite product over all prime numbers $p$:
    \begin{equation}
        \zeta(s) = \prod_{p} \frac{1}{1 - p^{-s}},
        \label{eqn:euler-factorization}
    \end{equation}
    for $\Re(s) > 1$.
\end{theorem}
\begin{proof}
    This essentially follows from the fundamental theorem of arithmetic (Theorem \ref{thm:fta}).
    More precisely, using geometric series we can rewrite the right hand side as
    \[
    \prod_{p} \frac{1}{1 - p^{-s}} = \prod_{p} \left(1 + \frac{1}{p^s} + \frac{1}{p^{2s}} + \cdots \right)
    \]
    and Theorem \ref{thm:fta} implies that each term $\frac{1}{n^s}$ in the left hand side in \eqref{eqn:euler-factorization} appears exactly once in the right hand side.
\end{proof}

Let's return to the prime number theorem.
The main idea of the proof(s) is in two steps:
\begin{enumerate}
    \item Reduce the problem to the following claim: $\zeta(s)$ is non-vanishing for $\Re(s) = 1$ and $s \ne 1$.
    \item Prove the above claim.
\end{enumerate}

Here we give a sketch of the proof of 2.
By taking logarithmic derivative of \eqref{eqn:euler-factorization}, we get
\[
\log \zeta(s) = - \sum_p \log(1 - p^{-s}) = \sum_p \sum_{k \ge 1} \frac{p^{-ks}}{k}.
\]
If $s = x + iy$, we get
\[
|\zeta(x + iy)| = \exp\left(\sum_{p} \sum_{k \ge 1} \frac{\cos (ky \log p)}{k p^{kx}}\right)
\]
and
\begin{equation}
\label{eqn:riemann-zeta-abs}
|\zeta(x)^3 \zeta(x + iy)^4 \zeta(x + 2iy)| = \exp\left(\sum_{p} \sum_{k \ge 1} \frac{3 + 4 \cos(k y \log p) + \cos(2 k y \log p)}{k p^{kx}}\right)
\end{equation}
for all $x > 1$.
Now, one can prove the inequality
\begin{equation}
    \label{eqn:cosineq}
    3 + 4 \cos \theta + \cos(2\theta) \ge 0
\end{equation}
for any $\theta \in \bR$, which shows that the right hand side of \eqref{eqn:riemann-zeta-abs} is at least $1$.
Assume that $\zeta(1 + iy) = 0$ for some $y$. Then one can derive a contradiction from \eqref{eqn:riemann-zeta-abs} by considering the limit as $x \to 1^+$, where the left hand side tends to $0$ while the right hand is bounded below by $1$.

\begin{exercise}
    Prove \eqref{eqn:cosineq}.
\end{exercise}

So we can approximate $\pi(x)$ by $\frac{x}{\log x}$, but how good is this approximation?
More precisely, what is the growth of the error term?
First of all, there's a better approximation than $\frac{x}{\log x}$ suggested by Dirichlet, which is \emph{logarithmic integral}:
\begin{equation}
    \Li(x) := \int_2^x \frac{\dd t}{\log t}.
\end{equation}

In terms of growth, $\frac{x}{\log x}$ and $\Li(x)$ are asymptotically equivalent (see Exercise \ref{ex:Li} below).
However, $\Li(x)$ gives much better approximation than $\frac{x}{\log x}$.
In \cite{de1899fonction}, de la Vall\'ee-Poussin proved that the error term $\pi(x) - \Li(x)$ is bounded by 
\[
|\pi(x) - \Li(x)| = O(x e^{-c \sqrt{\log x}}) 
\]
for some constant $c > 0$.
Koch \cite{von1901distribution} showed that, \emph{assuming the Riemann Hypothesis}, we obtain the improved bound
\begin{equation}
    |\pi(x) - \Li(x)| = O\left(\sqrt{x} \log x\right)
    \label{eqn:koch}
\end{equation}
which in fact turned out to be equivalent to the Riemann Hypothesis.

\begin{exercise}
    \label{ex:Li}
    Prove that $\Li(x) \sim \frac{x}{\log x}$ as $x \to \infty$.
    More precisely, show that
    \begin{equation}
        \Li(x) \sim \frac{x}{\log x} \sum_{k \ge 0} \frac{k!}{(\log x)^k} = \frac{x}{\log x} + \frac{x}{(\log x)^2} + \frac{2x}{(\log x)^3} + \cdots.
    \end{equation}
\end{exercise}

\begin{exercise}\sage
    Plot the graphs of the functions $\pi(x)$, $\Li(x)$, and $\frac{x}{\log x}$ for $x \in [1, 10^6]$, and observe that $\Li(x)$ is much closer to $\pi(x)$ than $\frac{x}{\log x}$.
    Also, compare $\pi(x)$ and $\Li(x)$ - do you have any conjecture on the difference?\footnote{You may find that $\pi(x) < \Li(x)$ when $x$ is not too small, and it is tempting to conjecture that $\pi(x) < \Li(x)$ for sufficiently large $x$. However, this is not true. In fact, Littlewood \cite{littlewood1914distribution} proved that the sign of $\pi(x) - \Li(x)$ changes infinitely many times, although the smallest number $x > 10$ such that $\pi(x) < \Li(x)$ (known as Skewes's number) is probably very huge.}
\end{exercise}


\subsection{Counting irreducible polynomials with zeta function}
\label{subsec:counting-irred-poly-zeta}

How about polynomials?
First of all, there are infinitely many irreducible polynomials:
\begin{theorem}
    \label{thm:euclid-poly}
    For each prime $p$, there are infinitely many irreducible polynomials over $\bF_p$.
\end{theorem}

\begin{exercise}
    Prove Theorem \ref{thm:euclid-poly}. You can follow the argument of Theorem \ref{thm:euclid} above, or give more interesting proof.
\end{exercise}

Now, we can ask more interesting question, i.e. number of irreducible polynomials of certain degree.
\begin{myquote}
    For a given prime $p$ and integer $n$, how many irreducible monic polynomials of degree $n$ in $A = \bF_p[T]$?
\end{myquote}
We will fix $p$, and denote the number of such polynomials as $a_n$.
When $n$ is small, we can get nice formulas.
\begin{proposition}
    \label{prop:monic_irred_n2}
    \begin{equation}
        a_2 = \frac{p^2 - p}{2}.
        \label{eqn:monic_irred_n2}
    \end{equation}
\end{proposition}
\begin{proof}
    One can count number of \emph{reducible} monic polynomials, and subtract from the total number of monic quadratic polynomials, which is simply $p^2$ (you have $p$ choices for each $c_0$ and $c_1$ in $f(T) = T^2 + c_1 T + c_0$).
    If $f(T)$ is not irreducible, then it should factor as $f(T) = (T - \alpha)(T - \beta)$ for some $\alpha, \beta \in \bF_p$.
    Now, there are $p$ cases where $\alpha = \beta$, and $\binom{p}{2}$ cases where $\alpha \ne \beta$.
    This gives
    \[
    a_2 = p^2 - p - \binom{p}{2} = \frac{p^2 - p}{2}.
    \]
\end{proof}
\begin{exercise}
    Prove
    \begin{equation}
        a_3 = \frac{p^3 - p}{3}.
        \label{eqn:monic_irred_n3}
    \end{equation}
    You may need to exclude the polynomials $f(T)$ which factor as 1) products of three linear polynomials, or 2) products of linear and irreducible quadratic polynomial.
\end{exercise}

It seems that we may even able to find a nice formula for $a_n$.
To do this, we will define a \emph{zeta function} for polynomials, which is similar to the Riemann zeta function but much simpler.
\begin{definition}[Zeta function for $A$]
    We define \emph{zeta function} as
    \begin{equation}
        \zeta_A(s) = \sum_{\substack{0 \ne f\,\text{monic}}} \frac{1}{|f|^s}
        \label{eqn:zetaA}
    \end{equation}
    which converges for $\Re(s) > 1$.
\end{definition}

We have a simple formula for $\zeta_A$:
\begin{proposition}
    \label{prop:zetaA}
    We have
    \begin{equation}
        \zeta_A(s) = \frac{1}{1 - p^{1 - s}}.
        \label{eqn:zetaA_formula}
    \end{equation}
\end{proposition}
\begin{proof}
    We can group the sum in \eqref{eqn:zetaA} by degree, since $|f|$ only depends on degree of $f$.
    Since there are $p^d$-many monic polynomials of degree $d$, we have
    \begin{align*}
        \zeta_A(s) = \sum_{d \ge 1} \frac{p^d}{p^{ds}} = \sum_{d \ge 0} p^{d(1 - s)} = \frac{1}{1 - p^{1 - s}}.
    \end{align*}
\end{proof}
Especially, it has a simple pole at $s = 1$ with residue $1 / \log p$, and analytic continuation is clear from the formula.
Also, there are no zeros of $\zeta_A(s)$, so Riemann Hypothesis for polynomials is trivially true.
Note that the notation $\zeta_A$ somehow emphasizes $A$, since we will eventually study more general ``zeta functions'' for other polynomial rings (``extensions'' of $A$), where the Riemann Hypothesis is more interesting (but still true).

What about Euler factorization?
Since $A$ is also UFD, the same argument as in Theorem \ref{thm:euler-factorization} gives us
\begin{theorem}[Euler factorization for polynomials]
    \label{thm:euler-factorization-poly}
    We have
    \begin{equation}
        \zeta_A(s) = \prod_{P\text{ monic irred}} (1 - |P|^{-s})^{-1}.
        \label{eqn:euler-factorization-poly}
    \end{equation}
\end{theorem}

As in Proposition \ref{prop:zetaA}, we can group the product in \eqref{eqn:euler-factorization-poly} by degree, which gives
\begin{equation}
    \zeta_A(s) = \prod_{d \ge 1} (1 - p^{-ds})^{-a_d}.
    \label{eqn:euler-factorization-poly2}
\end{equation}
By comparing \eqref{eqn:zetaA_formula} and \eqref{eqn:euler-factorization-poly2}, we can deduce the following formula for $a_n$:
\begin{theorem}
    We have
    \begin{equation}
        p^n = \sum_{d \mid n} d a_d.
        \label{eqn:counting_irred_poly}
    \end{equation}
\end{theorem}
\begin{proof}
    Let $u = p^{-s}$, so
    \begin{equation}
        \frac{1}{1 - pu} = \prod_{d \ge 1} (1 - u^d)^{-a_d}.
    \end{equation}
    By taking logarithmic derivative, we have
    \begin{equation}
        \frac{p}{1 - pu} = \sum_{d \ge 1} a_d \frac{d u^{d-1}}{1 - u^d} \Leftrightarrow \frac{pu}{1 - pu} = \sum_{d \ge 1} a_d \frac{d u^d}{1 - u^d}.
    \end{equation}
    Using geometric series, we can rewrite the equations as
    \begin{equation}
        \sum_{n \ge 1} p^n u^n = \sum_{d \ge 1} d a_d \sum_{k \ge 1} u^{dk} = \sum_{n \ge 1} \left(\sum_{d \mid n} d a_d\right) u^n.
    \end{equation}
    Now, we can compare the coefficients of $u^n$ on both sides and get \eqref{eqn:counting_irred_poly}.
\end{proof}

We can \emph{invert} this formula to get a formula for $a_n$.
Such a process is called \emph{M\"obius inversion} - before we state the general formula, we will first give small examples that may leads to the general formula (in other words, \emph{you can invent M\"obius inversion formula yourself}).
Let $b_n := n a_n$.
For $n = 1, 2, 3, 4, 6, 12$, \eqref{eqn:counting_irred_poly} gives us the following equations:
\begin{align*}
    p &= b_1,\\
    p^2 &= b_1 + b_2,\\
    p^3 &= b_1 + b_3,\\
    p^4 &= b_1 + b_2 + b_4,\\
    p^6 &= b_1 + b_2 + b_3 + b_6, \\
    p^{12} &= b_1 + b_2 + b_3 + b_4 + b_6 + b_{12}.
\end{align*}
From these equations, we can compute $b_n$ recursively:
\begin{align*}
    b_1 &= p,\\
    b_2 &= p^2 - p,\\
    b_3 &= p^3 - p,\\
    b_4 &= p^4 - p^2,\\
    b_6 &= p^6 - p^3 - p^2 + p,\\
    b_{12} &= p^{12} - p^6 - p^4 + p^2.
\end{align*}
Let's see how we actually computed $b_6$ and $b_{12}$. $b_{6}$ can be written as
\begin{align*}
    b_6 &= p^6 - (b_1 + b_2 + b_3)\\
        &= p^6 - (b_1 + b_3) - (b_1 + b_2) + b_1\\
        &= p^6 - p^3 - p^2 + p.
\end{align*}
Similarly, we can compute $b_{12}$ as
\begin{align*}
    b_{12} &= p^{12} - (b_1 + b_2 + b_3 + b_4 + b_6)\\
        &= p^{12} - (b_1 + b_2 + b_3 + b_6) - (b_1 + b_2 + b_4) - (b_1 + b_2)\\
        &= p^{12} - p^6 - p^4 + p^2.
\end{align*}
This may reminds you the principle of inclusion-exclusion, and indeed, we can write the general formula as follows: for $n = p_1^{e_1} p_2^{e_2} \cdots p_k^{e_k}$,
\begin{align*}
    b_{n} &= p^n - \sum_{d \mid n, d < n} b_d\\
        &= p^n - (b_1 + \cdots + b_{n / p_1}) - (b_1 + \cdots + b_{n / p_2}) - \cdots - (b_1 + \cdots + b_{n / p_k}) \\
        &\quad + (b_1 + \cdots + b_{n / p_1 p_2}) + (b_1 + \cdots + b_{n / p_1 p_3}) + \cdots\\
        &\cdots\\
        &= p^n - p^{n / p_1} - p^{n / p_2} - \cdots - p^{n / p_k} + p^{n / (p_1 p_2)} + p^{n / (p_1 p_3)} + \cdots\\
\end{align*}

This gives us the following formula for $a_n$:
\begin{corollary}
    \label{cor:counting_irred_poly}
    For a given prime $p$ and integer $n$, we have
    \begin{equation}
        a_n = \frac{1}{n} \sum_{d \mid n} \mu(d) p^{n/d},
        \label{eqn:counting_irred_poly_2}
    \end{equation}
    where $\mu$ is the M\"obius function, defined as
    \begin{equation}
        \mu(n) = \begin{cases}
            1 & \text{if } n = 1,\\
            (-1)^k & \text{if } n = p_1 p_2 \cdots p_k \text{ with } k \text{ distinct primes} \\
            0 & \text{otherwise.}
        \end{cases}
    \label{eqn:mobius}
    \end{equation}
\end{corollary}

\begin{exercise}
    Complete the proof of Corollary \ref{cor:counting_irred_poly}.
\end{exercise}

Note that \eqref{eqn:counting_irred_poly_2} gives an \emph{exact} formula for $a_n$.
This also gives an analogue of the prime number theorem for polynomials: the dominating term of $a_n$ is
\[
\frac{p^n}{n} = \frac{p^n}{\log_p p^n},
\]
where $p^n$ is equal to the number of monic polynomials of degree $n$ over $\bF_p$.
Moreover, the error term is bounded by $O\left(\frac{p^{n/2}}{n}\right)$, so we get
\begin{equation}
    a_n = \frac{p^n}{\log_p p^n} + O\left(\frac{\sqrt{p^n}}{\log_p p^n}\right).
    \label{eqn:pi-f}
\end{equation}
which has a similar form as Koch's bound \eqref{eqn:koch} (or even better!).

\begin{exercise}\sage
    Let $p = 3$. For each $1 \le n \le 10$, find the number of irreducible monic polynomials of degree $n$ over $\bF_3$, by using the factorization of $T^{3^n} - T$.
    Check that the results match with the formula \eqref{eqn:counting_irred_poly_2}.
\end{exercise}

\begin{exercise}
    Prove the following non-asymptotic version of \eqref{eqn:pi-f}:
    \[
    \left|a_n - \frac{p^n}{n} \right| \le \frac{p^{n/2}}{n}
    \]
\end{exercise}

\begin{exercise}
    For each $p$ and $d$, prove that the polynomial
    \[
    \left(\prod_{0 \le \deg(f) < d} f\right) + 1
    \]
    is equal to a constant multiple of the product of all monic irreducible polynomials of degree $d$ over $\bF_p$.
    What is the constant?
\end{exercise}

\begin{exercise}
    Prove that the polynomial $T^p - T - 1$ is irreducible over $\bF_p$ for all prime $p$.\footnote{Such a polynomial is called \emph{Artin--Schreier polynomial}.}
\end{exercise}

\begin{exercise}\sage
    For a given prime $p$ and an integer $n$, design an algorithm to find an irreducible polynomial of degree $n$ over $\bF_p$? (This is not an easy question.)
\end{exercise}

\begin{exercise}
    What would be a good analogue of the logarithmic integral $\Li(x)$ for polynomials?
\end{exercise}


\subsection{Counting other things}

The power of the above method is that we can count other objects by replacing the zeta function with \emph{Dirichlet series}
\begin{equation}
    L(s, a) = \sum_{n \ge 1} \frac{a_n}{n^s}
    \label{eqn:dirichlet_series}
\end{equation}
or
\begin{equation}
    L(s, a) = \sum_{\substack{0 \ne f \in A \\ f \text{ monic}}} \frac{a_f}{|f|^s},
    \label{eqn:dirichlet_series_poly}
\end{equation}
for given sequence $(a_n)_{n \ge 1}$ or sequence of polynomials $(a_f)_{f \in A}$.
When these sequences are ``interesting'', we call the Dirichlet series \emph{L-function}.
One condition for being ``interesting'' is that the Dirichlet series has an Euler factorization, i.e. it can be written as a product over primes (which might be more complicated than \eqref{eqn:euler-factorization} or \eqref{eqn:euler-factorization-poly}).

For example, we will coun the number of \emph{square-free} monic polynomials of given degree $n$ over $\bF_p$.
Note that $f \in A$ is square-free if and only if $f$ is not divisible by $P^2$ for any monic irreducible polynomial $P$.
In other words, it has no repeated factors in its factorization.
To count such polynomials, one can consider the indicator function
\begin{equation}
    \delta(f) = \begin{cases}
        1 & \text{if } f \text{ is square-free},\\
        0 & \text{otherwise}
    \end{cases}
    \label{eqn:square_free_indicator}
\end{equation}
and the corresponding Dirichlet series
\begin{equation}
    L(s, \delta) = \sum_{\substack{0 \ne f \in A \\ f \text{ monic}}} \frac{\delta(f)}{|f|^s} = \sum_{n \ge 0} \frac{c_n}{p^{ns}},
    \label{eqn:square_free_dirichlet_series}
\end{equation}
where $c_n = \sum_{\deg f = n} \delta(f)$ is the number of square-free monic polynomials of degree $n$ over $\bF_p$ (here $c_0 = 1$).
Now, only the polynomials with no repeated factors contribute to the sum, so they are product of distinct monic irreducible polynomials.
This gives us the Euler factorization
\begin{equation}
    L(s, \delta) = \prod_{P \text{ monic irred}} (1 + |P|^{-s})
    \label{eqn:square_free_euler_factorization}
\end{equation}
From $1 + |P|^{-s} = (1 - |P|^{-2s}) / (1 - |P|^{-s})$, comparing with the Euler factorization \eqref{eqn:euler-factorization-poly} we can rewrite \eqref{eqn:square_free_euler_factorization} as
\begin{equation}
    L(s, \delta) = \frac{\zeta_A(s)}{\zeta_A(2s)} = \frac{1 - p^{1-2s}}{1 - p^{1 - s}}.
    \label{eqn:square_free_dirichlet_series_2}
\end{equation}
Substituting with $u = p^{-s}$ gives us
\[
    \frac{1 - pu^2}{1 - pu} = \sum_{n \ge 0} c_n u^{n}.
\]
and expanding the left hand side as a geometric series gives us

\begin{theorem}
    \label{thm:square-free-monic-poly}
    Let $c_n$ be the number of square-free monic polynomials of degree $n$ over $\bF_p$.
    Then $c_1 = p$ and $c_n = p^n(1 - p^{-1})$ for $n \ge 2$.
\end{theorem}
Since there are $p^n$ monic polynomials of degree $n$, the Theorem shos that the proportion of square-free monic polynomials of degree $n$ is $(1 - p^{-1})$ for $n \ge 2$.

We can ask the same question for integers.
The analogous $L$-series would be
\begin{equation}
    L(s, \delta) = \sum_{n \ge 1} \frac{\delta(n)}{n^s} = \prod_{p \text{ prime}} (1 + p^{-s}) = \frac{\zeta(s)}{\zeta(2s)},
    \label{eqn:square_free_dirichlet_series_int}
\end{equation}
where $\delta(n)$ is the indicator function for square-free integers, i.e.
\begin{equation}
    \delta(n) = \begin{cases}
        1 & \text{if } n \text{ is square-free},\\
        0 & \text{otherwise.}
    \end{cases}
    \label{eqn:square_free_indicator_int}
\end{equation}
Then we are interested in the number of square-free integers $n \le x$, i.e. the sum
\begin{equation}
\sum_{n \le x} \delta(n).
\label{eqn:square_free_int_sum}
\end{equation}
The main idea is that the above sum \eqref{eqn:square_free_int_sum} can be approximated by the limit of $L(s, \delta)$ as $s \to 0$ (so you need to know analytic continuation of it), and use analytic properties of $L(s, \delta)$ to get the asymptotic formula (such an argument is often called as \emph{Tauberian arguments or theorems}).
Here we only give the result.
\begin{theorem}
    \label{thm:square-free-int}
    The number of square-free integers $n \le x$ is asymptotically equal to
    \begin{equation}
        \sum_{n \le x} \delta(n) \sim \frac{1}{\zeta(2)} x = \frac{6}{\pi^2} x.
        \label{eqn:square_free_int_asymptotic}
    \end{equation}
\end{theorem}
Note that Theorem \ref{thm:square-free-int} and \ref{thm:square-free-monic-poly} are analogous to each other; the proportion of square-free integers (resp. polynomials) is $\frac{1}{\zeta(2)}$ (resp. $\frac{1}{\zeta_A(2)} = 1 - p^{-1}$).

\begin{exercise}
    Prove that $\delta(n) = \mu(n)^2$ for all integers $n \ge 1$, where $\mu$ is the M\"obius function defined in \eqref{eqn:mobius}.
\end{exercise}

\begin{exercise}
    Compute the number of \emph{cube-free} polynomials of degree $n$ over $\bF_p$, i.e. the number of monic polynomials $f \in A$ such that $f$ is not divisible by $P^3$ for any monic irreducible polynomial $P$.
    Can you generalize this to higher powers?
\end{exercise}



\subsection{Sum of arithmetic functions}

More generally, we can adapt the above method to compute the sum of \emph{arithmetic functions}, i.e. functions defined on integers or polynomials.

For example, we consider the divisor counting function
\begin{equation}
    d(n) := \sum_{d \mid n} 1
    \label{eqn:divisor_counting_function}
\end{equation}
for integer $n$, or
\begin{equation}
    d(f) := \sum_{\substack{h \mid f \\ h\text{ monic}}} 1
    \label{eqn:divisor_counting_function_poly}
\end{equation}
for (monic) polynomial $f \in A$.
You may found that the function $d(n)$ behaves quite irregularly, e.g. $d(p) = 2$ for all prime $p$, while $d(n)$ can be large for composite $n$.
But the sum or average of $d(n)$ (or arithmetic functions in general) behaves much better.
For example, we have
\begin{theorem}[Dirichlet]
    \begin{equation}
        \sum_{n \le x} d(n) = x \log x  + (2\gamma - 1)x + O(\sqrt{x} \log x),
        \label{eqn:divisor_counting_function_asymptotic}
    \end{equation}
    where $\gamma = 0.577\dots$ is the \emph{Euler-Mascheroni constant}.
\end{theorem}
\begin{proof}
    See \cite[Theorem 3.3]{apostol2013introduction}.
\end{proof}
As you expect, it is much easier to compute the sum of $d(f)$ for (monic) polynomials $f \in A$ of given degree $n$.
In fact, we can obtain an exact formula for the sum of $d(f)$:
\begin{theorem}
    \label{thm:divisor_counting_function_poly}
    For a given prime $p$ and integer $n$, we have
    \begin{equation}
        \sum_{\deg(f) = n} d(f) = (n + 1) p^n.
        \label{eqn:divisor_counting_function_poly_formula}
    \end{equation}
\end{theorem}
\begin{proof}
    Proof also uses $L$-series.
    In fact, we will prove that the $L$-series of $d(f)$ is simply given by
    \begin{equation}
        \sum_{\substack{0 \ne f \in A \\ f\text{ monic}}} \frac{d(f)}{|f|^s} = \zeta_A(s)^2
        \label{eqn:divisor_counting_function_poly_L_series}
    \end{equation}
    Let's assume that the above formula is true for a moment.
    Denote the sum of $d(f)$ for monic polynomials of degree $n$ as $D_n$.
    Then we have ($u = p^{-s}$)
    \[
    \sum_{n \ge 0} D_n u^n = \sum_{\substack{0 \ne f \in A \\ f\text{ monic}}} \frac{d(f)}{|f|^s} = \zeta_A(s)^2 = \left(\frac{1}{1 - p^{1 - s}}\right)^2 = \frac{1}{(1 - pu)^2}.
    \]
    One can compute the right hand side by differentiating the geometric series $\frac{1}{1 - pu} = 1 + pu + p^2 u^2 + \cdots$, which gives the desired formula.

    To prove \eqref{eqn:divisor_counting_function_poly_L_series}, we can simply take square of $\zeta_A(s)$:
    \[
    \zeta_A(s)^2 = \left(\sum_{f} \frac{1}{|f|^s}\right)^2 = \left(\sum_{f_1} \frac{1}{|f_1|^s}\right) \left(\sum_{f_2} \frac{1}{|f_2|^s}\right) = \sum_{f} \left(\sum_{h_1 h_2 = f} 1\right) \frac{1}{|f|^s} = \sum_{f} \frac{d(f)}{|f|^s}.
    \]
\end{proof}
Note that one can write \eqref{eqn:divisor_counting_function_poly_formula} as
\[
\sum_{\deg(f) = n} d(f) = n p^n + p^n = p^n \log_p p^n + p^n,
\]
which is analogous to the main terms $x \log x + (2\gamma - 1)x$ in \eqref{eqn:divisor_counting_function_asymptotic}.

Why do we care about such sums?
One nice example is the following theorem by Littlewood.
\begin{theorem}[Littlewood]
    Define the \emph{Mertens function} as
    \begin{equation}
        M(x) = \sum_{n \le x} \mu(n).
        \label{eqn:mertens_function}
    \end{equation}
    Then the Riemann Hypothesis is equivalent to the following bound:
    \begin{equation}
        |M(x)| = O(x^{\frac{1}{2} + \epsilon})
        \label{eqn:mertens_bound}
    \end{equation}
    for any $\epsilon > 0$.
\end{theorem}

\begin{exercise}
    For integer $n$, \emph{Dedekind's $\psi$-function} is defined as
    \begin{equation}
        \psi(n) = n \prod_{p \mid n} \left(1 + \frac{1}{p}\right).
        \label{eqn:dedekind_psi}
    \end{equation}
    \begin{enumerate}
        \item Prove that $\psi(n)$ is multiplicative, i.e. if $\gcd(m, n) = 1$, then $\psi(mn) = \psi(m) \psi(n)$.
        \item Prove that $\psi(n) = n \sum_{d \mid n} \frac{|\mu(d)|}{d}$, where $\mu$ is the M\"obius function defined in \eqref{eqn:mobius}.
        \item Prove that $\psi(n)$ is equal to the number of cyclic subgroups of order $n$ in $(\bZ / n \bZ)^2$.
        \item Prove
        \[
        \sum_{n \ge 1} \frac{\psi(n)}{n^s} = \frac{\zeta(s) \zeta(s - 1)}{\zeta(2s)}
        \]
        for $\Re(s) > 2$.\footnote{Hint: use Euler factorization.}
    \end{enumerate}
    It is known that the sum of $\psi(n)$ has an asymptotic formula:
    \begin{equation}
        \sum_{n \le x} \psi(n) \sim \frac{\zeta(2)}{2\zeta(4)} x^2 + O(x \log x).
        \label{eqn:dedekind_psi_asymptotic}
    \end{equation}
    (See Apostol \cite[p. 71, Exercise 11]{apostol2013introduction}.)
\end{exercise}

\begin{exercise}
    For $f \in A$, define $\psi(f)$ as
    \begin{equation}
        \psi(f) := |f| \prod_{P \mid f} (1 + |P|^{-1}),
        \label{eqn:dedekind_psi_poly}
    \end{equation}
    where the product is over all monic irreducible polynomials $P$ dividing $f$.
    \begin{enumerate}
        \item Prove all the analogous statements in the previous exercise.
        \item Compute the sum
        \[
        \sum_{f \in A, \deg(f) = n} \psi(f).
        \]
        Compare the result with \eqref{eqn:dedekind_psi_asymptotic}.
    \end{enumerate}
\end{exercise}


\newpage
