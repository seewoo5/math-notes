\section{Basic number theory and their analogues for polynomials}
\label{sec:basicnt}

In this section, we will introduce polynomial analogues of the theorems in number theory, including
\begin{itemize}
    \item Fundamental Theorem of Arithmetic,
    \item Chinese Remainder Theorem,
    \item Fermat's Little Theorem and Euler's Theorem,
    \item Wilson's Theorem,
\end{itemize}


\subsection{Fundamental Theorem of Arithmetic}
\label{subsec:basicnt_fta}

The Fundamental Theorem of Arithmetic states that every integer greater than 1 can be uniquely expressed as a product of prime numbers, up to the order of the factors.
More fancier way to say this is that

\begin{theorem}
    \label{thm:fta}
    $\mathbb{Z}$ is a unique factorization domain (UFD).
\end{theorem}
The standard proof is based on the following implication:
\begin{theorem}
    If $R$ is a Euclidean domain (ED), then $R$ is a principal ideal domain (PID), and hence a UFD.
\end{theorem}
Recall that $R$ is a Euclidean domain if there exists a function $f : R \backslash \{0\} \to \mathbb{Z}_{\geq 0}$ such that for any $a, b \in R$ with $b \neq 0$, there exist $q, r \in R$ such that $a = bq + r$ and either $r = 0$ or $f(r) < f(b)$.
Intuitively, $R$ is a Euclidean domain if we can perform the division with remainder, and the function $f$ is a measure of the size of the elements in $R$.
For any (not necessarily finite) field $k$, we can also divide a polynomial by another polynomial over $k$, where $\deg : R \backslash \{0\} \to \mathbb{Z}_{\geq 0}$ works as the function $f$.
This shows that:
\begin{theorem}
    \label{thm:polynomial_fta}
    The polynomial ring $k[T]$ is a ED, where $k$ is any field.
    Hence it is a PID, and hence a UFD.
\end{theorem}

\begin{exercise}
    For each prime $p \le 20$, determine if the polynomial $T^2 + 1$ is irreducible over $\bF_p$ or not. Can you find a pattern?\footnote{Answer will be given in the next section.}
\end{exercise}

\subsection{Chinese Remainder Theorem}
\label{subsec:basicnt_crt}

The Chinese Remainder Theorem (CRT) states that if $n_1, n_2, \ldots, n_k$ are pairwise coprime integers, then the system of congruences
\[
    x \equiv a_1 \pmod{n_1}, \quad x \equiv a_2 \pmod{n_2}, \quad \ldots, \quad x \equiv a_k \pmod{n_k}
\]
has a unique solution modulo $N = n_1 n_2 \cdots n_k$.
\begin{theorem}[Chinese Remainder Theorem]
    \label{thm:crt}
    Let $n_1, n_2, \ldots, n_k$ be pairwise coprime integers and $a_1, a_2, \ldots, a_k$ be integers.
    Then the system of congruences
    \[
        x \equiv a_1 \pmod{n_1}, \quad x \equiv a_2 \pmod{n_2}, \quad \ldots, \quad x \equiv a_k \pmod{n_k}
    \]
    has a unique solution modulo $N = n_1 n_2 \cdots n_k$.
    More precisely, the unique solution is given by
    \[
        x \equiv \sum_{i=1}^k a_i N_i y_i \pmod{N},
    \]
    where $N_i = N / n_i$ and $y_i$ is the multiplicative inverse of $N_i$ modulo $n_i$, i.e., $N_i y_i \equiv 1 \pmod{n_i}$.
\end{theorem}

\begin{proof}
    The proof is essentially hidden in the following isomorphism:
    \[
        \bZ / N\bZ \cong \bZ / n_1\bZ \times \bZ / n_2\bZ \times \cdots \times \bZ / n_k\bZ.
    \]
    The natural map from the left hand side to the right hand side is given by reducing modulo $n_i$ for each $i$, and such a map is injective because $n_1, n_2, \ldots, n_k$ are pairwise coprime.
    Since the size of the both sides are equal, the map is also surjective, hence an isomorphism.
    Finding the solution to the system of congruences is equivalent to finding an element in $\bZ / N\bZ$ that maps to $(a_1, a_2, \ldots, a_k)$ under the isomorphism.
    It is enough to find solution for the equations
    \[
        x_i \equiv 0 \pmod{n_1}, \quad x_i \equiv 0 \pmod{n_2}, \quad \cdots \quad x_i \equiv 1 \pmod{n_i}, \quad \cdots, \quad x_i \equiv 0 \pmod{n_k}
    \]
    for each $i$, then the solution to the original system of congruences is given by the linear combination of the solutions to these equations.
    Such $x_i$ has to be a multiple of $N_i$, and if we write $x_i = N_i y_i$, then we have $N_i y_i \equiv 1 \pmod{n_i}$, which means $y_i$ is the multiplicative inverse of $N_i$ modulo $n_i$.
    Such $y_i$ exists and can be found by the Euclidean Algorithm, since $N_i$ and $n_i$ are coprime.
\end{proof}

For example, consider a system of congruences
\[
\begin{cases}
    x \equiv 2 \pmod{3} \\
    x \equiv 3 \pmod{5}
\end{cases}
\]
From Euclidean Algorithm, we can write $1 = 2 \cdot 3 - 1 \cdot 5$, which means $5^{-1} \equiv (-1) \pmod{3}$ and $3^{-1} \equiv 2 \pmod{5}$.
Then the formula in Theorem \ref{thm:crt} gives us $x \equiv 2 \cdot 5 \cdot (-1) + 3 \cdot 3 \cdot 2 \pmod{15}$, which simplifies to $x \equiv -10 + 18 \equiv 8 \pmod{15}$.

What is a polynomial analogue of the Chinese Remainder Theorem?
We simply replace integers with polynomials over a finite field, and we get the following theorem.
\begin{theorem}[Chinese Remainder Theorem for Polynomials]
    \label{thm:polynomial_crt}
    Let $g_1(T), g_2(T), \ldots, g_k(T) \in \bF_p[T]$ be pairwise coprime polynomials and $a_1(T), a_2(T), \ldots, a_k(T) \in \bF_p[T]$ be polynomials.
    Then the system of congruences
    \[
        f \equiv a_1 \pmod{g_1}, \quad f \equiv a_2 \pmod{g_2}, \quad \ldots, \quad f \equiv a_k \pmod{g_k}
    \]
    has a unique solution modulo $G = g_1 g_2 \cdots g_k$.
    More precisely, the unique solution is given by
    \[
        f \equiv \sum_{i=1}^k a_i G_i b_i \pmod{G},
    \]
    where $G_i = G / g_i$ and $b_i$ is the multiplicative inverse of $G_i$ modulo $g_i$, i.e., $G_i b_i \equiv 1 \pmod{g_i}$.
\end{theorem}

Using the exact same method, we can solve a system of congruences.
For example, consider
\[
\begin{cases}
    f \equiv 1 \pmod{T} \\
    f \equiv T \pmod{T^2 + 1}
\end{cases}
\]
We can write $1 = (-T) \cdot T + 1 \cdot (T^2 + 1)$ (hence $T$ and $T^2 + 1$ are coprime), which means $(T^2 + 1)^{-1} \equiv 1 \pmod{T}$ and $T^{-1} \equiv -T \pmod{T^2 + 1}$.
Then the formula in Theorem \ref{thm:polynomial_crt} gives us $f \equiv 1 \cdot (T^2 + 1) \cdot 1 + T \cdot T \cdot (-T) \pmod{(T^2 + 1)T}$, which simplifies to $f \equiv T^2 + 1 - T^3 \equiv T^2 + T + 1 \pmod{(T^2 + 1)T}$.

\begin{exercise}
    \begin{enumerate}
        \item Find a polynomial $f \in \bF_3[T]$ such that\footnote{Answer: $f = T^4 + T^2 + T + 2$}
        \[
        \begin{cases}
            h \equiv T + 2 \pmod{T^2 + 1} \\
            h \equiv T^2 + T \pmod{T^3 - T^2 +  T + 2}
        \end{cases} 
        \]
        \item Find a polynomial $f \in \bF_{7}[T]$ such that\footnote{Answer: $f = T^5 + 3T^2 + 2T + 1$}
        \[
        \begin{cases}
            f \equiv 5T^2 + 3T + 6 \pmod{T^3 + 2T^2 + 3T + 2} \\
            f \equiv T^3 + 1 \pmod{T^4 + 3T^3 + 2T^2 + 1} \\
            f \equiv T^4 - T + 2 \pmod{T^2 + 3T + 1}
        \end{cases} 
        \]
        You can try to write a code to find such a polynomial.
    \end{enumerate}
\end{exercise}

\subsection{Fermat's Little Theorem and Euler's Theorem}
\label{subsec:basicnt_flittlet}

Ferma's \emph{Little} (not last!) Theorem states the following:
\begin{theorem}[Fermat's Little Theorem]
    \label{thm:fermat_little}
    Let $p$ be a prime number and $a$ an integer not divisible by $p$.
    Then
    \[
        a^{p - 1} \equiv 1 \pmod{p}.
    \]
\end{theorem}

Here is a proof using group theory.

\begin{proof}
    Consider the group $G = (\bZ / p\bZ)^\times$.
    The order of $G$ is $p - 1$, and the order of the subgroup generated by $a$ is a divisor of $p - 1$.
    Thus, by Lagrange's theorem, we have $a^{p - 1} \equiv 1 \pmod{p}$, as desired.
\end{proof}

Euler's theorem is a generalization of Fermat's Little Theorem, which considers general moduli.
To state Euler's theorem, we need to define the \emph{Euler's totient function} $\varphi(n)$, which counts the number of integers from $1$ to $n$ that are coprime to $n$.

\begin{theorem}[Euler's Theorem]
    \label{thm:euler}
    Let $n$ be a positive integer and $a$ an integer coprime to $n$.
    Then
    \[
        a^{\varphi(n)} \equiv 1 \pmod{n}.
    \]
\end{theorem}
\begin{proof}
    Proof is similar to the proof of Theorem \ref{thm:fermat_little}, where we consider the group $G = (\bZ / n\bZ)^\times$ which has order $\varphi(n)$.
\end{proof}

One may ask how to compute $\varphi(n)$.
By CRT again (but for unit groups), we have an isomorphism
\[
    (\bZ / n\bZ)^\times \cong (\bZ / p_1^{k_1}\bZ)^\times \times (\bZ / p_2^{k_2}\bZ)^\times \times \cdots \times (\bZ / p_m^{k_m}\bZ)^\times,
\]
where $n = p_1^{k_1} p_2^{k_2} \cdots p_m^{k_m}$ is the prime factorization of $n$.
Thus, we can compute $\varphi(n)$ as
\[
    \varphi(n) = \varphi(p_1^{k_1}) \varphi(p_2^{k_2}) \cdots \varphi(p_m^{k_m}),
\]
hence we only need to compute $\varphi(p^k)$ for a prime $p$ and a positive integer $k$.
This counts the number of integers from $1$ to $p^k$ that are coprime to $p^k$, or equivalently multiples of $p$, which is $p^k - p^{k - 1} = p^{k - 1}(p - 1)$.
This gives us the formula
\begin{theorem}
    \label{thm:phi_prime_power}
    For an integer $n = p_1^{k_1} p_2^{k_2} \cdots p_m^{k_m}$, we have
    \begin{equation}
    \label{eqn:phi_formula}
    \varphi(n) = n \prod_{i=1}^m \left(1 - \frac{1}{p_i}\right).
    \end{equation}
\end{theorem}

What is a polynomial analogue of Fermat's Little Theorem and Euler's Theorem?
Based on Table \ref{tab:dictionary}, prime will be replaced by irreducible polynomial.
Unfortunately, it does not make sense to take exponentiation of a polynomial by another polynomial.
However, since $\varphi(n)$ was defined as a size of the group $(\bZ / n\bZ)^\times$, we can define a polynomial analogue of $\varphi(g)$ as the size of the group $(A / gA)^\times$.
Especially, if $g$ is irreducible, then $(A / gA)^\times = (A / gA) \backslash \{0\}$, which has size $|g| - 1$.
The exact same argument as the proof of Theorem \ref{thm:fermat_little} and Theorem \ref{thm:euler} works for polynomials, and we get the following theorems.


\begin{theorem}[Fermat's Little Theorem for Polynomials]
    \label{thm:polynomial_fermat_little}
    Let $f, g \in A$ be polynomials, where $g$ is irreducible and $f$ is not divisible by $g$.
    Then
    \[
        f^{|g| - 1} \equiv 1 \pmod{g}.
    \]
\end{theorem}


\begin{theorem}[Euler's Theorem for Polynomials]
    \label{thm:polynomial_euler}
    Let $f, g \in A$ be coprime polynomials.
    Then
    \[
        f^{\varphi(g)} \equiv 1 \pmod{g}.
    \]
\end{theorem}

The formula \eqref{eqn:phi_formula} also generalizes to polynomials.
\begin{theorem}
    \label{thm:polynomial_phi_prime_power}
    Let $g(T) = p_1^{k_1} p_2^{k_2} \cdots p_m^{k_m}$ be a polynomial in $A$, where $p_1, p_2, \ldots, p_m$ are distinct irreducible polynomials in $A$.
    Then
    \[
        \varphi(g) = |g| \prod_{i=1}^m \left(1 - \frac{1}{|p_i|}\right).
    \]
\end{theorem}

\begin{exercise}
    There are several different proofs of Theorem \ref{thm:fermat_little}, e.g. see \href{https://en.wikipedia.org/wiki/Proofs_of_Fermat%27s_little_theorem}{wikipedia page}.
    Choose your favorite argument (other than the one used in Theorem \ref{thm:fermat_little}) and try to generalize it to prove \ref{thm:polynomial_fermat_little}.\footnote{For example, there's a proof using induction on $a$. Can you generalize it to polynomials?}
\end{exercise}

\begin{exercise}
    Prove the original version of Fermat's little theorem from the polynomial version.\footnote{Hint: for a prime $p$ and an integer $a$ not divisible by $p$, consider $f(T) = T + a$ and $g(T) = T$.}
\end{exercise}



\subsection{Wilson's Theorem}
\label{subsec:basicnt_wilson}

Another interesting theorem on prime numbers is Wilson's theorem:
\begin{theorem}[Wilson's Theorem]
    Let $p$ be a prime number. Then
    \[
        (p - 1)! \equiv -1 \pmod{p}.
    \]
\end{theorem}
\begin{proof}
    Consider the group $G = (\bZ / p\bZ)^\times$.
    The left hand side is the product of all elements in $G$.
    Now, we can pair up each element $a \in G$ with its inverse $a^{-1}$, except for the case when $a = a^{-1}$, which happens if and only if $a^2 \equiv 1 \pmod{p} \Leftrightarrow a \equiv \pm 1 \pmod{p}$.
    Thus the product of all elements in $G$ is $\equiv 1 \cdot (-1) \equiv -1 \pmod{p}$, as desired.
\end{proof}

What is a polynomial analogue of Wilson's theorem?
Note that the left hand side of Wilson's theorem is the product of all nonzero elements in $\bF_p$, so it might be reasonable to define ``factorial'' $(g - 1)!$ of an irreducible polynomial $g(T) \in \bF_p[T]$ as the product of all elements in $(A / gA)^\times$.

\begin{theorem}[Wilson's Theorem for Polynomials]
    \label{thm:polynomial_wilson}
    Let $p$ be a prime number and $A = \bF_p[T]$ the polynomial ring over the finite field $\bF_p$.
    Let $g(T) \in A$ be an irreducible polynomial of degree $d$.
    Then
    \[
        \prod_{0 \le \deg(f) \le d} f \equiv -1 \pmod{g}.
    \]
\end{theorem}

Note that the left hand side only depends on the degree of $g(T)$.
Especially, (LHS) + $1$ is divisible by any irreducible polynomial $g(T)$ of degree $d$.

\begin{exercise}
    Prove Theorem \ref{thm:polynomial_wilson}.
\end{exercise}

\begin{exercise}
    Prove the original version of Wilson's theorem from the polynomial version.\footnote{Hint: consider $g(T) = T$.} 
\end{exercise}

\begin{exercise}
    It is known that the following converse of Wilson's theorem holds: if a natural number $n$ satisfies $(n - 1)! \equiv -1 \pmod{n}$, then $n$ is a prime number.
    We can consider a polynomial analogue of this converse: if a polynomial $g(T) \in \bF_p[T]$ of degree $d$ satisfies
    \[
        \prod_{\substack{0 \le \deg(f) \le d \\ \gcd(f, g) = 1}} f \equiv -1 \pmod{g},
    \]
    then $g(T)$ is irreducible over $\bF_p$.
    Here the left hand side is the product of all elements in $(A / gA)^\times$.
    Prove or disprove this claim.
\end{exercise}

\newpage
