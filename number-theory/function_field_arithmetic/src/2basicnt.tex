\section{Basic number theory and their analogues for polynomials}
\label{sec:basicnt}

In this section, we will introduce polynomial analogues of the theorems in number theory, including
\begin{itemize}
    \item Fundamental Theorem of Arithmetic,
    \item Chinese Remainder Theorem,
    \item Fermat's Little Theorem and Euler's Theorem,
    \item Wilson's Theorem,
\end{itemize}


\subsection{Fundamental Theorem of Arithmetic}
\label{subsec:basicnt_fta}

The Fundamental Theorem of Arithmetic states that every integer greater than 1 can be uniquely expressed as a product of prime numbers, up to the order of the factors.
More fancier way to say this is that

\begin{theorem}
    $\mathbb{Z}$ is a unique factorization domain (UFD).
\end{theorem}
The standard proof is based on the following implication:
\begin{theorem}
    If $R$ is a Euclidean domain (ED), then $R$ is a principal ideal domain (PID), and hence a unique factorization domain (UFD).
\end{theorem}
Recall that $R$ is a Euclidean domain if there exists a function $f : R \backslash \{0\} \to \mathbb{Z}_{\geq 0}$ such that for any $a, b \in R$ with $b \neq 0$, there exist $q, r \in R$ such that $a = bq + r$ and either $r = 0$ or $f(r) < f(b)$.


\subsection{Chinese Remainder Theorem}
\label{subsec:basicnt_crt}


\subsection{Fermat's Little Theorem and Euler's Theorem}
\label{subsec:basicnt_flittlet}

Ferma's \emph{Little} (not last!) Theorem states the following:
\begin{theorem}[Fermat's Little Theorem]
    \label{thm:fermat_little}
    Let $p$ be a prime number and $a$ an integer not divisible by $p$.
    Then
    \[
        a^{p - 1} \equiv 1 \pmod{p}.
    \]
\end{theorem}

Here is a proof using group theory.

\begin{proof}
    Consider the group $G = (\bZ / p\bZ)^\times$.
    The order of $G$ is $p - 1$, and the order of the subgroup generated by $a$ is a divisor of $p - 1$.
    Thus, by Lagrange's theorem, we have $a^{p - 1} \equiv 1 \pmod{p}$, as desired.
\end{proof}

Euler's theorem is a generalization of Fermat's Little Theorem, which considers general moduli.
To state Euler's theorem, we need to define the \emph{Euler's totient function} $\varphi(n)$, which counts the number of integers from $1$ to $n$ that are coprime to $n$.

\begin{theorem}[Euler's Theorem]
    \label{thm:euler}
    Let $n$ be a positive integer and $a$ an integer coprime to $n$.
    Then
    \[
        a^{\varphi(n)} \equiv 1 \pmod{n}.
    \]
\end{theorem}
\begin{proof}
    Proof is similar to the proof of Theorem \ref{thm:fermat_little}, where we consider the group $G = (\bZ / n\bZ)^\times$ which has order $\varphi(n)$.
\end{proof}

What is a polynomial analogue of Fermat's Little Theorem and Euler's Theorem?
Based on Table \ref{tab:dictionary}, prime will be replaced by irreducible polynomial.
Unfortunately, it does not make sense to take exponentiation of a polynomial.


\begin{theorem}[Fermat's Little Theorem for Polynomials]
    \label{thm:polynomial_fermat_little}
    Let $f, g \in A$ be polynomials, where $g$ is irreducible and $f$ is not divisible by $g$.
    Then
    \[
        f^{|g| - 1} \equiv 1 \pmod{g}.
    \]
\end{theorem}


\begin{theorem}[Euler's Theorem for Polynomials]
    \label{thm:polynomial_euler}
    Let $f, g \in A$ be coprime polynomials.
    Then
    \[
        f^{\varphi(g)} \equiv 1 \pmod{g}.
    \]
\end{theorem}

\subsection{Wilson's Theorem}
\label{subsec:basicnt_wilson}

Another interesting theorem on prime numbers is Wilson's theorem:
\begin{theorem}[Wilson's Theorem]
    Let $p$ be a prime number. Then
    \[
        (p - 1)! \equiv -1 \pmod{p}.
    \]
\end{theorem}
\begin{proof}
    Consider the group $G = (\bZ / p\bZ)^\times$.
    The left hand side is the product of all elements in $G$.
    Now, we can pair up each element $a \in G$ with its inverse $a^{-1}$, except for the case when $a = a^{-1}$, which happens if and only if $a^2 \equiv 1 \pmod{p} \Leftrightarrow a \equiv \pm 1 \pmod{p}$.
    Thus the product of all elements in $G$ is $\equiv 1 \cdot (-1) \equiv -1 \pmod{p}$, as desired.
\end{proof}

What is a polynomial analogue of Wilson's theorem?
Note that the left hand side of Wilson's theorem is the product of all nonzero elements in $\bF_p$, so it might be reasonable to define ``factorial'' $(g - 1)!$ of an irreducible polynomial $g(T) \in \bF_p[T]$ as the product of all nonzero elements in $A / gA$.

\begin{theorem}[Wilson's Theorem for Polynomials]
    \label{thm:polynomial_wilson}1
    Let $p$ be a prime number and $A = \bF_p[T]$ the polynomial ring over the finite field $\bF_p$.
    Let $g(T) \in A$ be an irreducible polynomial of degree $d$.
    Then
    \[
        \prod_{0 \le \deg(f) \le d} f \equiv -1 \pmod{g}.
    \]
\end{theorem}

Note that the left hand side only depends on the degree of $g(T)$.
Especially, (LHS) + $1$ is divisible by any irreducible polynomial $g(T)$ of degree $d$.

\subsection*{Exercises}

\begin{enumerate}
    \item Let $p = 3$ and $f(T) = T^2 + 1$, $g(T) = T^3 - T + 1$.
    \begin{enumerate}
        \item Prove that $f(T)$ and $g(T)$ are irreducible in $\bF_3[T]$.
        \item Prove that $f(T)$ and $g(T)$ are coprime in $\bF_3[T]$, by finding polynomials $a(T)$ and $b(T)$ in $\bF_3[T]$ such that $1 = a(T)f(T) + b(T)g(T)$.
        \item Find all polynomials $h(T)$ in $\bF_3[T]$ such that 
        \[
        \begin{cases}
            h(T) \equiv 1 \pmod{f(T)} \\
            h(T) \equiv T \pmod{g(T)}
        \end{cases} 
        \]
    \end{enumerate}
    \item There are several different proofs of Theorem \ref{thm:fermat_little}, e.g. see \href{https://en.wikipedia.org/wiki/Proofs_of_Fermat%27s_little_theorem}{wikipedia page}.
    Choose your favorite argument and try to generalize it to prove \ref{thm:polynomial_fermat_little}.\footnote{For example, there's a proof using induction on $a$. Can you generalize it to polynomials?}
    \item Prove Theorem \ref{thm:polynomial_wilson}.
    \item Prove the original version of Fermat's little theorem from the polynomial version.\footnote{Hint: for a prime $p$ and an integer $a$ not divisible by $p$, consider $f(T) = T + a$ and $g(T) = T$.}
    \item Prove the original version of Wilson's theorem from the polynomial version.\footnote{Hint: consider $g(T) = T$.} 
\end{enumerate}