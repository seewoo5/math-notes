\section{It is as easy as ABC}
\label{sec:abc}

Infamous Fermat's Last Theorem (FLT) states that there are no positive integer solutions to the equation
\[
x^n + y^n = z^n
\]
for any integer \( n \ge 2 \). The theorem was famously proven by Andrew Wiles in 1994, using sophisticated techniques from algebraic geometry and number theory.
See the great expository book by Cornell, Silverman, and Stevens \cite{cornell2013modular} for a detailed account of the proof.

As you expect, the goal of this section is to prove a polynomial analogue of FLT.
Indeed, we will show a stronger result, named \emph{Mason--Stothers theorem} or \emph{Polynomial ABC} \cite{mason1984diophantine,stothers1981polynomial}.
The proof given in this section is based on the elementary argument by Snyder \cite{snyder2000alternate}, but we will also give a \emph{geometric} proof in the next section.

\subsection{Polynomial FLT}
\label{sec:abc-flt}

Let \( k \) be a field and \( k[T] \) the polynomial ring over \( k \).
We have the following version of FLT:

\begin{theorem}[Polynomial FLT]
\label{thm:abc-flt}
Let \( k \) be a field and \( n \geq 3 \) an integer.
Let \( f, g, h \in k[T] \) be mutually coprime polynomials such that not all derivatives \( f', g', h' \) are zero.
Then \( f^n + g^n \ne h^n \).
\end{theorem}
Note that we have solutions for $n \le 2$, e.g. $(1 - T^2)^2 + (2T)^2 = (1 + T^2)^2$.
Also, when $k$ is of characteristic $0$, $f' = 0$ is equivalent to $f$ being a constant polynomial.
However, there is a nonconstant solution when $k$ is of characteristic $p$, e.g. $1^p + T^p = (1 + T)^p$ for $n = p$.

The main idea of a proof is to use ``derivatives'' of polynomials.
\begin{proof}
    Let's assume that \( f(T)^n + g(T)^n = h(T)^n \).
    Differentiating both sides gives
    \[
    f(T)^{n-1} f'(T) + g(T)^{n-1} g'(T) - h(T)^{n-1} h'(T) = 0.
    \]
    By multiplying $h(T)$ on the both sides, we can rewrite this as
    \begin{align*}
        f^{n-1} f' h + g^{n-1} g' h - h^n h' &= f^{n-1} f' h + g^{n-1} g' h - (f^n + g^n) h' = 0\\
        \Leftrightarrow f^{n-1} (f'h - fh') &= -g^{n-1} (g'h - gh').
    \end{align*}
    Since $f$ and $g$ are coprime, this implies that $f^{n-1}$ divides $g'h - gh'$.
    Similarly, we can show that $g^{n-1}$ divides $h'f - hf'$ and $h^{n-1}$ divides $f'g - fg'$.
    Also, we have \( gh' - g'h \ne 0 \); otherwise, \( gh' = g'h \) and coprimality of \( g, h \) implies that \( g \mid g' \), so \( g' = 0 \) from \( \deg (g') < \deg g \).
    Without loss of generality, assume that $f$ has the largest degree among the three polynomials.
    Then
    \[
    2 \deg f \le (n-1) \deg f = \deg f^{n-1} \le \deg (g'h - gh') \le \deg g' + \deg h - 1 \le 2 \deg f - 1,
    \]
    which is a contradiction.
\end{proof}

Compared to the Wiles and Taylor's proof of original FLT \cite{wiles1995modular,taylor1995ring}, the proof of Theorem \ref{thm:abc-flt} is very elementary and short.
However, the above argument cannot be applied to the original FLT, since we don't have a good notion of derivatives for integers.

\begin{exercise}
    \begin{enumerate}
        \item Show that the only map \( D: \bZ \to \bZ \) such that \( D(m + n) = D(m) + D(n) \) (additive) and \( D(mn) = mD(n) + nD(m) \) (Leibniz rule) is the zero map.
        \item If we give up the additivity condition, then show that there is a nonzero map \( D: \bZ \to \bZ \) such that \( D(mn) = mD(n) + nD(m) \). Such a map is called an \emph{arithmetic derivative}, which is used in \cite{pasten2024derivation} to given an alternative proof of infinitude of primes.
    \end{enumerate}
\end{exercise}

\subsection{Polynomial ABC}
\label{sec:abc-abc}

In fact, we can prove more general result called \emph{Mason--Stothers theorem} or \emph{Polynomial ABC} \cite{mason1984diophantine,stothers1981polynomial,snyder2000alternate}.
Before stating the theorem, we introduce the original ABC conjecture for integers first.
For \( n \in \bZ \), we define the \emph{radical} of \( n \) as the product of distinct prime factors of \( n \): \(\rad(n) = \prod_{p \mid n} p\).
Oesterl\'e and Masser conjectured the following:
\begin{conjecture}[ABC conjecture]
\label{conj:abc}
For any \( \epsilon > 0 \), there exists a constant \( K = K_\epsilon > 0 \) such that
\begin{equation}
\label{eqn:abc}
c < K \cdot \rad(abc)^{1 + \epsilon}
\end{equation}
for any coprime integers \( a, b, c \) satisfying \( a + b = c \).
\end{conjecture}
Compared to $n$, $\rad(n)$ is small when $n$ can be divisible by high powers of primes.
In other words, to violate \eqref{eqn:abc}, we need to have $a, b, c$ to be divisible by many distinct primes so that $\rad(abc)$ is small.
It means that the conjecture is saying that the sum of two coprime integers that are divisible by high powers of primes is not divisible by high powers anymore.
If we define the \emph{quality} of numbers \( a, b, c \) as
\[
q(a, b, c) := \frac{\log c}{\log \rad(abc)},
\]
then the current record for the highest quality is achieved by
\[
(a, b, c) = (2, 3^{10} \cdot 109, 23^5), \quad q(a, b, c) \approx 1.6299.
\]
There are numerous applications of the conjecture, which can be found in the \href{https://en.wikipedia.org/wiki/Abc_conjecture}{Wikipedia page}.
Here we introduce few of them, where some of them are proven (by different methods) or follows from the conjecture.
The simplest example is the following weak version of FLT:
\begin{theorem}[Weak FLT]
    Assume Conjecture \ref{conj:abc} holds.
    If \( n \ge 4 \), then there are finitely many coprime integers \( x, y, z \) such that \( x^n + y^n = z^n \).
\end{theorem}
\begin{proof}
    Take $\epsilon = 1/4$ and \( (a, b, c) = (x^n, y^n, z^n) \) in Conjecture \ref{conj:abc}.
    Then there exists a constant \( K > 0 \) such that
    \[
    z^n < K \cdot \rad(x^n y^n z^n)^{5 / 4} = K \cdot \rad(xyz)^{5 / 4} \le K \cdot (xyz)^{5 / 4} < K \cdot z^{15 / 4}
    \]
    and there are only finitely many \( z \) satisfying this inequality.
\end{proof}

Another example is the Roth's theorem on Diophantine approximation.
\begin{theorem}[Roth \cite{roth1955rational}]
    Let $\alpha$ be an irrational algebraic number.
    Then the inequality
    \[
    \left| \alpha - \frac{p}{q} \right| < \frac{1}{q^{2 + \epsilon}}
    \]
    can have only finitely many solutions in coprime integers \( p, q \) for any \( \epsilon > 0 \).
\end{theorem}
\begin{proof}
    Bombieri \cite{bombieri1994roth} gave a proof of Roth's theorem assuming Conjecture \ref{conj:abc}.
    In fact, he proved a stronger version of a Roth's theorem, which can be stated as an inequality on heights of algebraic numbers.
    See Frankenhuysen's note \cite{frankenhuysen} for a detailed account of the proof.
\end{proof}

The last example is the weak version of Hall's conjecture, which is about the difference of a cube and a square.
\begin{theorem}[Weak Hall's conjecture \cite{hall1971diophantine}]
    \label{thm:abc-hall}
    Assume Conjecture \ref{conj:abc} holds.
    For any \( \epsilon > 0 \), there exists a constant \( K = K_\epsilon > 0 \) such that \( | y^2 - x^3 | > K \cdot x^{1 / 2 - \epsilon} \) for any  integers \( x, y \) such that \( y^2 - x^3 \ne 0 \).
\end{theorem}

\begin{exercise}
    Prove Theorem \ref{thm:abc-hall}.
    Answer can be found in \cite{schmidt2006diophantine}.
\end{exercise}

See Granville and Tucker's survey \cite{granville2002s} for more applications of the conjecture.

\begin{exercise}
    For any $K > 0$, prove that there exist infinitely many coprime integers \( a, b, c \) such that \( a + b = c \) and \( c > K \cdot \rad(abc) \).
    In other words, the condition \( \epsilon > 0 \) in Conjecture \ref{conj:abc} is necessary.\footnote{Hint: Consider \( (a, b, c) = (1, 2^{p(p-1)n} - 1, 2^{p(p-1)n})\) for large prime \( p \) and integer \( n \). Prove that \( b \) is divisible by \( p^2 \).}
\end{exercise}

Let's move on to the polynomial analogue of Conjecture \ref{conj:abc}.
As you expect, the radical of a polynomial \( f \in k[T] \) is defined as the product of distinct irreducible monic factors of \( f \).
The polynomial version of the ABC conjecture is now a theorem by Mason and Stothers \cite{mason1984diophantine,stothers1981polynomial}.

\begin{theorem}[Mason--Stothers theorem]
    Let \( k \) be a field and \( f, g, h \in k[T] \) be mutually coprime polynomials such that \( f + g + h = 0 \) and not all of \( f', g', h' \) are zero.
    Then
    \begin{equation}
        \label{eqn:abc-poly}
        \max \{ \deg f, \deg g, \deg h \} \le \deg \rad(f g h) - 1.
    \end{equation}
\end{theorem}

We follow the proof of Snyder \cite{snyder2000alternate}, which is similar to the proof of Theorem \ref{thm:abc-flt}.
The main step of the proof is the following lemma:
\begin{lemma}
    For any nonzero \( f \in k[T] \), \( f / \rad(f) \) divides \( f' \).
\end{lemma}
\begin{proof}
    This follows from factorization of \( f \) into irreducible monic factors.
    If we write \( f = \prod_{i=1}^m f_i^{e_i} \) where \( f_i \) are distinct irreducible monic polynomials and \( e_i \ge 1 \), then
    \begin{align*}
        f' &= \sum_{i=1}^m e_i f_i^{e_i - 1} f_i' \prod_{j \ne i} f_j^{e_j} \\
        \frac{f}{\rad(f)} &= \prod_{i=1}^m f_i^{e_i - 1}
    \end{align*}
    so \( f / \rad(f) \) divides \( f' \).
\end{proof}
\begin{proof}[Proof of Mason--Stothers theorem]
    For two polynomials \( f, g \in k[T] \), we define the \emph{Wronskian} \( W(f, g) \) as \( W(f, g) = f g' - g f' \).
    Then \( f + g + h = 0 \) implies that all three Wronskians \( W(f, g), W(g, h), W(h, f) \) are the same:
    \begin{align*}
        W(f, g) = f g' - g f' = (- g - h) g' - g (-g' -h') = gh' - g'h = W(g, h).
    \end{align*}
    Let $W$ be the common Wronskian of \( f, g, h \).
    If \( W = 0\), then \( fg' = f'g \) and \( \gcd(f, g) = 1 \) gives \( f \mid f' \) and \(f' = 0\), and similarly \( g' = 0 \) and \( h' = 0 \), which contradicts the assumption that not all of \( f', g', h' \) are zero.
    Hence \( W \ne 0 \).
    By the previous Lemma, \( f / \rad(f), g / \rad(g), h / \rad(h) \) divide \( W \), so does \( fgh / \rad(f g h) \) since \( f, g, h \) are mutually coprime.
    Thus
    \begin{align*}
        \deg f + \deg g + \deg h - \deg \rad(fgh) &= \deg \left( \frac{fgh}{\rad(fgh)} \right) \\
        &\le \deg W \\
        &= \deg (f g' - g f') \\
        &\le \deg f + \deg g - 1
    \end{align*}
    which implies \( \deg h \le \deg \rad(f g h) - 1 \).
    Similar inequality holds for \( \deg f \) and \( \deg g \) and we get the desired inequality.
\end{proof}

This theorem has numerous applications.
For example, one can prove non-solvability of the Fermat--Catalan equations, which is a generalization of the Fermat equation.
\begin{theorem}
    Let \( k \) be a field and \( p, q, r \ge 0 \) be integers satisfying
    \begin{equation}
        \label{eqn:abc-pqr}
        \frac{1}{p} + \frac{1}{q} + \frac{1}{r} \le 1
    \end{equation}
    and not divisible by the characteristic of \( k \).
    Then there are no nonconstant and mutually coprime polynomials \( f, g, h \in k[T] \) such that \( f^p + g^q + h^r = 0 \).
\end{theorem}
\begin{proof}
    Apply the Mason--Stothers theorem to the polynomials \( f^p, g^q, h^r \) to show that \( f' = g' = h' = 0 \).
    When \( k \) has a positive characteristic, use the fact that \( f' = 0 \) implies that \( f(T) = f_0(T^\ell) \) for some \( f_0 \in k[T] \) and \( \ell = \mathrm{char}(k) \) and apply infinite descent.
\end{proof}

Note that the finiteness result of the previous theorem for integers and fixed \(p, q, r\) satisfying \( 1 / p + 1 / q + 1 / r < 1 \) is a consequence of the original ABC conjecture, and also proved by Darmon and Granville \cite{darmon1995equations} using Falting's theorem.

We also have a version of the weak Hall's conjecture for polynomials, which is a theorem of Davenport \cite{davenport1965f3}.
\begin{theorem}[Davenport \cite{davenport1965f3}]
    Let \( k \) be a field and \( f, g \in k[T] \) be mutually coprime polynomials with nonzero derivatives such that \( f^3 - g^2 \ne 0 \).
    Then \( \deg f + 2 \le 2 \deg (f^3 - g^2) \).
\end{theorem}

\begin{exercise}
    Prove Davenport's theorem using the Mason--Stothers theorem.
\end{exercise}

\begin{exercise}
    Search for the result that follows from ABC conjecture for integers, and prove polynomial analogue of the result (if exists) using Mason--Stothers theorem.
\end{exercise}

\newpage