\section{Automorphic Forms on $\widetilde{\SL}_2$}


Let $k$ be a global field.
The ad\'ele group $\SL_2(\bA)$ has a unique non-trivial two-fold covering $\widetilde{\SL}_2(\bA)$:
\[
1 \to \{\pm 1\} \to \widetilde{\SL}_2(\bA) \to \SL_2(\bA) \to 1.
\]
There is a unique embedding of $\SL_2(k)$ into $\widetilde{\SL}_2(\bA)$ such that the following diagram commutes.

\begin{center}
\begin{tikzcd}
 & \widetilde{\SL}_2(\bA) \arrow[d] \\
\SL_2(k) \arrow[r] \arrow[ru] & \SL_2(\bA)
\end{tikzcd}
\end{center}

This means covering splits over $\SL_2(k)$.
Similarly, there is an embedding of $N(\bA)$ into $\widetilde{\SL}_2(\bA)$, where $N$ is the upper unipotent subgroup of $\SL_2$.

Let $A_0$ denote the space of genuine cuspidal functions on $\widetilde{\SL}_2(\bA)$.
In particular, if $f \in A_0$, then

\begin{enumerate}
    \item $f(\xi \gamma g) = \xi f(g)$ \qquad ($\xi \in \{ \pm 1 \}, \gamma \in \SL_2(k), g \in \widetilde{\SL}_2(\bA)$)
\end{enumerate}

Under right translation, $A_0$ decomposes discretely into a countable number of irreducible subspaces.
An irreducible representation of $\widetilde{\SL}_2(\bA)$ which occurs in $A_0$ is called a genuine automorphic cuspidal
representation.
Let $A_{00}$ denote the subspace of forms in $A_0$ orthogonal to the Weil representations of $\widetilde{\SL}_2(\bA)$.

\begin{theorem}[Multiplicity One \cite{waldspurger80shimura}]
The multiplicity of an irreducible genuine automorphic cuspidal representation in $A_{00}$ is one.
\end{theorem}
\begin{remark*}
If a is a genuine irreducible automorphic cuspidal representation lying in a Weil representation of $\widetilde{\SL}_2(\bA)$, then multiplicity one is obvious.
\end{remark*}

If $\psi$ is a character of $k \backslash \bA$, and $f \in A_{00}$, the $\psi$-Fourier coefficient of $f$ is defined to be
\[
f_\psi(g) = \int_{k \backslash \bA} f\left(\begin{pmatrix} 1 & n \\ & 1\end{pmatrix}g\right) \psi(n) \dd n \qquad (g \in \widetilde{\SL}_2(\bA))
\]
The multiplicity result follows from the uniqueness of Whittaker models for $\widetilde{\SL}_2(\bA)$, and the following result of Waldspurger.

\begin{theorem}[\cite{waldspurger81demientier,waldspurger84shimura}]
\label{thm:1.2}
Let $(\sigma, V)$ be a genuine irreducible automorphic cuspidal representation of $\widetilde{\SL}_2(\bA)$.
If $v \mapsto \varphi(v)$ ($v \in V, \varphi(v) \in A_{00}$)
is an embedding of $(\sigma, V)$ into $A_{00}$, then the vanishing of the $\psi$-Fourier coefficient $\varphi(v)_\psi$ depends only on $(\sigma, V)$ as an abstract representation, and not on the embedding $\varphi$.
\end{theorem}


\begin{proof}[Proof of the multiplicity one]
Suppose $v \mapsto \varphi'(v)$ and $v \mapsto \varphi''(v)$
($v \in v$) are two distinct embeddings of an irreducible genuine automorphic cuspidal representation $(\sigma, V)$ into $A_0$.
We may select a character $\psi$ of $k \backslash \bA$ so that the $\psi$-Fourier coefficient $\varphi'(v)_\psi$ does
not vanish for some $v \in V$.
Let us consider the $\psi$-Fourier coefficient
$\varphi''(v)_\psi$.
If $\varphi''(v)_\psi$ vanishes, then Theorem \ref{thm:1.2} says $\varphi'(v)_\psi$ must also vanish, a contradiction.
If $\varphi''(v)$ does not vanish, then the uniqueness of Whittaker models for $\widetilde{\SL}_2(\bA)$ tells us that $\varphi''(v) = c\varphi'(v)$ for some constant c.
Since $\varphi'$ and $\varphi''$ are assumed to be distinct
embeddings of $(\sigma, V)$ into $A_{00}$, the map $w \mapsto \varphi''(w) - c\varphi'(w)$ is a non-trivial embedding of 
 $(\sigma, V)$ into $A_{00}$.
The $\psi$-Fourier coefficient of $\varphi''(v) - c\varphi'(v)$ vanishes.
This again contradicts Theorem \ref{thm:1.2}; therefore $(\sigma, V)$ must occur in $A_{00}$ with multiplicity one.
\end{proof}

Two irreducible genuine automorphic cuspidal representations of $\widetilde{\SL}_2(\bA)$, $\sigma = \otimes_v \sigma_v$ and $\sigma' = \otimes_v \sigma_{v}'$, are said to be nearly equivalent if $\sigma_v \simeq \sigma_v'$ for almost all places $v$.
Let $\ell(\sigma)$ denote the set of irreducible genuine automorphic cuspidal representations nearly equivalent to $\sigma$.
$\ell(\sigma)$, of course, just measures departure from strong multiplicity one.
In order to determine the set $\ell(\sigma)$, Waldspurger
has defined an involution $\sigma \mapsto \sigma^W$ whenever $\sigma$ is a discrete series representation of $\widetilde{\SL}_2(k_v)$.
If $\sigma = \otimes_v \sigma_v \subset A_{00}$, define
\[
\Sigma = \{ v: \sigma_v\text{ is a discrete series representation}\}.
\]
If $M \subseteq \Sigma$, and $|M|$ is even, put
\[
\sigma^M = \otimes_v \sigma_v^M \text{ where } \sigma_v^M = \begin{cases} \sigma_v & \text{if }v\in M \\ \sigma_V^W & \text{if }v \not\in M.\end{cases}
\]

The relationship of the $\sigma^M$'s and $\ell(\sigma)$ is given in the following theorem.

\begin{theorem}[\cite{waldspurger91quaternion}]
\label{thm:1.3}
Any representation in $\ell(\sigma)$ is of the form $\sigma^M$ for some $M \subseteq \Sigma$.
\end{theorem}

\begin{corollary}
$|\ell(\sigma)| = 2^{|\Sigma| - 1}$.
\end{corollary}

\begin{remark*}
Recall that $\left(\begin{smallmatrix}
    -1 & 0 \\ 0 & -1
\end{smallmatrix}\right)$ lies in the center of $\widetilde{\SL}_2(k_v)$.
Waldspurger has shown that $\sigma_v^M\left(\begin{smallmatrix}
    -1 & 0 \\ 0 & -1
\end{smallmatrix}\right) = - \sigma_v\left(\begin{smallmatrix}
    -1 & 0 \\ 0 & -1
\end{smallmatrix}\right)$
Since $\left(\begin{smallmatrix}
    -1 & 0 \\ 0 & -1
\end{smallmatrix}\right) \in \SL_2(k)$, it follows that if $M \subseteq \Sigma$ has an odd number of elements, $\sigma^M$ cannot be an automorphic representation.
\end{remark*}
