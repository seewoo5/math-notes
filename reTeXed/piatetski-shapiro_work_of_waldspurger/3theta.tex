\section{The $\theta$-correspondence}
\label{sec:3}

Let $k$ be a global field.
We shall use the same notion globally as was previously introduced locally.
The global Weil (oscillator) representation $\omega_\psi$ acts on $\cS(X_1(\bA))$.
It is easy to see that it is the tensor product of the local Weil representations.
Let $X = X_1 \oplus X_2$ be the standard polarization of $X$, and identify $X_1$ wiht $M$.
For $\phi \in \cS(X_1(\bA))$,
\[
\vartheta_\psi^\phi(g, h) = \sum_{x \in X_1(k)} \omega_\psi(g, h) \phi(x)\qquad (g \in G(\bA), h \in \widetilde{\SL}_2(\bA)).
\]
Here, $G$ is either $\PGL_2$ or $\rP D^\times$.
It is well known that $\vartheta_\psi^\phi$ is an automorphic function on $G(\bA) \times \widetilde{\SL}_2(\bA)$ of moderate growth.

The theta function $\vartheta_\psi^\phi$'s can be used to define a correspondence between the automorphic representation of $G(\bA)$ and those of $\widetilde{\SL}_2(\bA)$.
To describe this correspondence, let $\pi$ be an irreducible automorphic cuspidal representation of $G(\bA)$.
If $f \in \pi \subset A_0$, put
\[
\varphi(h) := \int_{G(k) \backslash G(\bA)} \vartheta_\psi^\phi(g, h) f(g) \dd g.
\]
In the case $G = \rP D^\times$, we assume that $\int_{G(k)\backslash G(\bA)} f(g) \dd g = 0$.
The fact that $\vartheta_\psi^\phi$ is a function of moderate grwoth on $(G(k) \times \SL_2(k)) \backslash (G(\bA) \times \widetilde{SL}_2(\bA))$ means that the integral is well-defined, and that $\varphi$ is a function on $\SL_2(k) \backslash \widetilde{\SL}_2(\bA)$.

\begin{claim}
$\varphi$ is a cusp form.
\end{claim}

\begin{proof}
It is enough to show that $\int_{k\backslash \bA} \varphi\left(\begin{smallmatrix}
    1 & z \\ 0 & 1
\end{smallmatrix}\right)\dd z = 0$.
\begin{align*}
\int_{k \backslash \bA} \varphi \begin{pmatrix}
    1 & z \\ 0 & 1
\end{pmatrix} \dd z &= \int_{k \backslash \bA} \int_{G(k) \backslash G(\bA)}  \sum_{x \in X(k)} \omega_\psi\left(g \begin{pmatrix}
    1 & z \\ 0 & 1
\end{pmatrix}\right)\phi(x) f(g) \dd g \dd z \\
&= \int_{G(k) \backslash G(\bA)} \sum_{x \in X(k)} \omega_\psi(g) \phi(x) f(g) \int_{k \backslash \bA} \psi(z q(x)) \dd z \dd g.
\end{align*}
The inner integral $\int_{k \backslash \bA} \psi(z q(x)) \dd z$ is zero unless $q(x) = 0$.
If $G = \rP D^\times$, then $q(x) = 0$ if and only if $x = 0$, and the integral becomes
\[
\int_{k \backslash \bA} \varphi \begin{pmatrix}
    1 & z \\ 0 & 1
\end{pmatrix} \dd z = \int_{G(k) \backslash G(\bA)} \phi(0) f(g) \dd g = 0.
\]
If $G = \PGL_2$, then $q(x)=0$ means either $x = 0$ or $x$ is a non-zero nilpotent element of $M_2(k)$. The integral in this situation is
\begin{align*}
    \int_{k \backslash \bA} \varphi \begin{pmatrix}
        1 & z \\ 0 & 1
    \end{pmatrix} \dd z &= \int_{G(k) \backslash G(\bA)} \phi(0) f(g) \dd g \\
    &+ \int_{G(k) \backslash G(\bA)} \sum_{N(k) \backslash G(k)} \phi \left(g^{-1} \gamma^{-1} \begin{pmatrix}
        0 & 1 \\ 0 & 0
    \end{pmatrix} \gamma g\right) f(g) \dd g \\
    &= 0 + \int_{N(\bA) \backslash G(\bA)} \phi(g^{-1} xg) f(\gamma^{-1} g) \dd g \\
    &= \int_{N(\bA) \backslash G(\bA)} \omega_\psi(g) \phi(x) \int_{N(k) \backslash N(\bA)} f(ng) \dd n \dd g = 0.
\end{align*}
Here $N$ is centralizer in $G$ of $\left(\begin{smallmatrix}
    0 & 1 \\ 0& 0
\end{smallmatrix}\right)$, and $\int_{N(k) \backslash N(\bA)} f(ng) \dd n = 0$, $\int_{N(\bA) \backslash G(\bA)} f(g) \dd g = 0$  since $f$ is a cusp form.
\end{proof}

Let $\theta(\pi, \psi)$ denote the representation of $\widetilde{\SL}_2(\bA)$ spanned by the $\varphi$'s ($\phi \in \cS(X_1(\bA)), f \in \pi$).
$\theta(\pi, \psi)$ is a genuine automorphic cuspidal representation of $\widetilde{\SL}_2(\bA)$.

\begin{theorem}[\cite{waldspurger80shimura}]
\label{thm:3.1}
The $\theta$-correspondence $\pi \mapsto \theta(\pi, \psi)$ is compatible with the local correspondences introduced in \S \ref{sec:2}.
\end{theorem}

\begin{proof}
Let $\pi$ be an irreducible automorphic cuspidal representation of $G(\bA)$. For $f \in \pi$, and $\phi \in \cS(X_1(\bA))$, let $\varphi$ again be the cusp form
\[
\varphi(h) = \int_{G(k) \backslash G(\bA)} \vartheta_{\psi}^\phi(g, h) f(g) \dd g.
\]
If $a \in k^\times$, then a calculation similar to the one used to show $\varphi$ is a cusp form shows
\begin{align}
    \varphi_a(1) &:= \int_{k \backslash \bA} \varphi \begin{pmatrix}
        1 & z \\ 0& 1
    \end{pmatrix} \overline{\psi(az)} \dd z \nonumber \\ 
    &= \int_{T^a(\bA) \backslash G(\bA)} \omega_\psi(g) \phi(x_a) \int_{T^a(k) \backslash T^a(\bA)} f(tg) \dd t \dd g. \label{eqn:F}
\end{align}
Here, $x_a$ is any element in $X$, such that $q(x_a) = a$ (if $G = \rP D^\times$, we assume $a$ is representable by $q$), and $T^a$ is the stabilizer of $x_a$.
$T^a$ is a torus in $G$.
Put
\[
U(f, g) := \int_{T^a(k) \backslash T^a(\bA)} f(tg) \dd t\qquad (g \in G(\bA), f \in \pi).
\]
The function $U(f, -)$ satisfies the property $U(f, tg) = U(f, g)$ for $t \in T^a(\bA)$, and the linear function $\ell: f \mapsto U(f, 1)$ is a linear functional on $(f\in \pi)$ for which $\ell(\pi(t)f) = \ell(f)$ ($t \in T^a(\bA)$).
Locally, such a linear functional is unique, hence $\ell$ is globally unique and
\[
U(f, -) = \otimes_v U_v(-),
\]
where $U_v$ is a function on $G_v$ such that $U_v(t_v g_v) = U_v(g_v)$ ($t_v \in T^a(k_v), g_v \in G(k_v)$).
Under right translations by $G_v$ on $T^a_v \backslash G_v$, $U_v$ generates a representation equivalent to $\pi_v$. In analogy with the global formula
\[
\varphi_a(h) = \int_{T^a(\bA) \backslash G(\bA)} \omega_\psi(g, h) \phi(x_a) U(f, g) \dd g,
\]
if $U$ is an element in the space generated by $U_v$, and if
\[
W_{\psi^a}(h) := \int_{T^a \backslash G_v} \omega_{\phi, v}(g, h) \phi(x_a) U(g) \dd g
\]
then $W_{\psi^a}\left(\left(\begin{smallmatrix}
    1 & z \\ 0 & 1
\end{smallmatrix}\right)h\right) = \psi_v(za) W_{\psi_a}(h)$.
\end{proof}

\begin{theorem}[\cite{waldspurger80shimura}]
\label{thm:3.2}
The $\theta$-correspondence  is a 1-1 correspondence between certain automorphic cuspidal irreducible representations of $G(\bA)$ and certain genuine automorphic cuspidal irreducible representations of $\widetilde{\SL}_2(\bA)$.
\end{theorem}


\begin{theorem}[\cite{waldspurger80shimura,hps83sp4}]
\label{thm:3.3} 
Let $G = \PGL_2$. Suppose $\sigma \subset A_{00}$, and $\pi$ is an automorphic cuspidal representation of $\PGL_2(\bA)$. Then
\begin{enumerate}
    \item $\theta(\sigma, \psi^{-1}) \neq 0$ if and only if $\sigma$ possesses a nonvanishing $\psi$-Fourier coefficient.
    \item $\theta(\pi, \psi) \neq 0$ if and only if $L(\pi, \frac{1}{2}) \neq 0$.
\end{enumerate}
\end{theorem}

\begin{proof}
In order to prove this theroem, we must use a polarization for which the usual subgroups of $\PGL_2(\bA)$ and $\widetilde{\SL}_2(\bA)$ lie inside $P$.
As before, let $M$ be the elements of $M_2(k)$ of trace zero, and let $q(m) = -\det(m)$.
Let $Y$ be a 2-dimensional symplectic vector spcaes over $k$ with form $\langle\,,\,\rangle$ and symplectic basis $y_1, y_2$.
Let $e_1, e_2, e_3$ be a basis of $M$ such that $q$ has the matrix $\left(\begin{smallmatrix}
    0 & 0 & 1 \\0 &1 & 0 \\1 & 0 & 0
\end{smallmatrix}\right)$.
Put $X_1 = e_1 \otimes Y +e_2 \otimes ky_1, X_2 = e_3 \otimes Y + e_2 \otimes ky_2$.
Suppose $\sigma$ is an irreducible genuine automorphic representation of $\widetilde{\SL}_2(\bA)$ lying in $A_{00}$.
If $\varphi \in \sigma$, let $f(g) = \int_{\SL_2(k) \backslash \SL_2(\bA)} \vartheta_\psi^\phi(g, h) \varphi(h) \dd h$.
We can identify $X_1$ with $Y \oplus k$, and we can choose $\phi$ in the form $\phi = \phi_1 \phi_2$, where $\phi_1 \in \cS(Y(\bA)), \phi_2 \in \cS(\bA)$.
In this situation,
\[
\vartheta_\psi^\phi(1, h) = F_1(h) F_2(h)
\]
where
\begin{align*}
    F_1(h) &= \sum_{Y(k)} \phi_1(yh) = \phi_1(0) + \sum_{\gamma \in B(k) \backslash \SL_2(k)} \phi_1(y_2 \gamma) \\
    F_2(h) &= \sum_{t \in k} \omega_{\psi}'(h) \phi_2(t)
\end{align*}
In the formula for $F_2$, $\omega_\psi'$ is the 1-dimensional Weil representation.
\begin{align*}
    f(1) &= \int_{\SL_(k) \backslash \SL_2(\bA)} \phi_1(0) F_2(h) \varphi(h) \dd h + \int_{N(k) \backslash \SL_2(\bA)} \phi_1(y_2 h) F_2(h) \varphi(h) \dd h.
\end{align*}
Since $\sigma \in A_{00}$, and $F_2$ lies in the space of the Weil representation of $\widetilde{\SL}_2(\bA)$, the first integral is zero.
It follows that $\theta(\sigma, \psi^{-1}) \neq 0$ if and only if the second integral does not vanish identically.
\begin{align*}
    f(1) &= \int_{N(k) \backslash \SL_2(\bA)} \phi_1(y_2 h) F_2(h) \varphi(h) \dd h \\
    &= \sum_{t \in k} \int_{N(k) \backslash \SL_2(\bA)} \phi_1(y_2 h) \omega_\psi'(h) \phi_2(t) \varphi(h) \dd h.
\end{align*}
Since $\phi_1(y_2 nh) = \phi_1(y_2 h)$ and $\omega_\psi'(nh) \phi_2(t) = \psi(t^2 n) \phi_2(t)$ ($n = \left(\begin{smallmatrix}
    1 & n \\ 0 & 1
\end{smallmatrix}\right) \in N(\bA)$) it follows that
\[
f(1) = \sum_{t \in k} \int_{N(\bA) \backslash \SL_2(\bA)} \phi_1(y_2 h) \omega_\psi'(h) \phi_2(t) \varphi_{\psi^{t^2}}(h) \dd h.
\]
Thus, if $\theta(\sigma, \psi^{-1}) \neq 0$, then there exists a $t$ for which $\varphi_{t^2}$ is non-zero.
This means $\sigma$ possesses a non-zero $\psi$-Fourier coefficient.
Conversely, now suppose $\sigma$ possesses a non-vanishing $\psi$-Fourier coefficient.
Let
\begin{align*}
    f_t(1) &= \int_{N(\bA) \backslash \SL_2(\bA)} \phi_1(y_2 h) \omega_\psi'(h) \phi_2(t) \varphi_{\psi^{t^2}}(h) \dd h \\
    &= \int_{N(\bA) \backslash \SL_2(\bA)} \omega_\psi(h) \phi(y_2, t)  \varphi_{\psi^{t^2}}(h) \dd h.
\end{align*}
The latter formula allows us to define $f_t(1)$ for arbitrary $\phi$.
In this situation, we still have $f(1) = \sum_{t \in k^\times} f_t(1)$. ($f_0(1) = 0$, since $f$ is a cusp form).
Let $N$ be the unipotent subgroup of $\PGL_2$.
For $n \in N$
\[
\omega_\psi(n) \phi(y_2, t) = \psi(tn) \phi(y_2, t).
\]
It follows form this formula that $f_t(1)$ is a Fourier coefficient of $f$.
Therefore, if $\varphi_\psi \neq 0$, then $f_t(1) \neq 0$ and so $\theta(\sigma, \psi^{-1}) \neq 0$.
To prove the second part of the theorem, we use the standard polarization.
If $\sigma =\theta(\pi, \psi) \neq 0$, then $\theta(\sigma, \psi^{-1})$ equals $\pi$.
This means by part 1 that $\sigma$ possesses a non-zero $\psi$-Fourier coefficient.
If $T$ is the split torus in $\PGL_2$, then formula \eqref{eqn:F} in the proof of Theorem \ref{thm:3.1} tells us that
\[
\int_{T(k) \backslash T(\bA)} f(t) \dd t \neq 0.
\]
From the Jacquet-Langlands theory of L-functions, it is known that for an appropriate choice of $f$,
\[
L(\pi, s) = \int_{T(k) \backslash T(\bA)} f(t) |t|^{s - \frac{1}{2}} \dd t.
\]
In particular,
\[
L\left(\pi, \frac{1}{2}\right) = \int_{T(k)\backslash T(\bA)} f(t) \dd t \neq 0.
\]
Conversely, if $L(\frac{1}{2}, \pi) \neq 0$, then it is clear that $\int_{T(k) \backslash T(\bA)} f(t) \dd t \neq 0$, and hence that $\theta(\pi, \psi) \neq 0$.
\end{proof}