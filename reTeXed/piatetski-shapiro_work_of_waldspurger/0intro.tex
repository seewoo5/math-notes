\section{Introduction}
\label{sec:0}

The work of Waldspurger \cite{waldspurger78theta,waldspurger80shimura,waldspurger81demientier,waldspurger84shimura,waldspurger91quaternion} is devoted to a very deep study of the automorphic forms on $\widetilde{\SL}_2$.
The main tool for such a study is the correspondence between automorphic forms on $\SL_2$ and automorphic forms on $\PGL_2$.
This correspondence was first discovered by Shintani and Niwa using the Weil representation.
An earlier approach to this correspondence, based on L-functions, was suggested by Shimura \cite{shimura73half}.
Indeed, Shimura's work seemed to stimulate Shintani's and Niwa's work on the subject.

R. Howe has outlined a general theory of duality correspondence based on the use of the Weil representation. 
He has introduced the general notion of a dual reductive pair, and has defined both a local and global duality correspondence.
R. Howe has obtained many deep results in the general situation: but many important problems remain \cite{howe79theta}.

A systematic study of the duality correspondence for the simplest dual reductive pair $(\widetilde{\SL}_2, \PGL_2)$ from the point of view of representation theory has been carried out by Rallis and Schiffmann \cite{rs77metaplectic}.
In his work, Waldspurger refers in many places to Rallis and Schiffmann, and, in a way, Waldspurger's work is a continuation of that of Rallis and Schiffmann.
However, I would like to emphasize that Waldspurger's work contains many fundamental new ideas especially in the global case.

Flicker has studied a correspondence between the automorphic forms of $\GL_2$ and those of $\widetilde{\GL}_2$ using the trace formula \cite{flicker80covering}.
He has in fact obtained a complete description of this correspondence.
Since $\widetilde{\SL}_2$ is a subgroup of $\widetilde{\GL}_2$ there is a close connection between the automorphic forms of these two groups.
Waldspurger has used Flicker's results in a substantial way to obtain his own results.
However, let me say that Waldspurger's results for $\widetilde{\SL}_2$ are quite surprising and were not predicted from the results for $\widetilde{\GL}_2$.
It remains a mystery to me why the automorphic forms on $\widetilde{\SL}_2$ and $\widetilde{\GL}_2$ behave so differently.
For example, strong multiplicity one is true for $\widetilde{\GL}_2$ but not for $\widetilde{\SL}_2$,
Also, the descent (correspondence) of automorphic forms from $\GL_2$ to $\widetilde{\GL}_2$ has only a local obstruction, while the correspondence from $\PGL_2$ to $\widetilde{\SL}_2$ has a global obstruction, but no local obstruction.

Let me also mention work \cite{gps81shimura,gps83metaplectic} which deals with L-functions for $\widetilde{\GL}_2$.
This work can be considered as an ad\'elization of Shimura's
work.
It establishes an injection of the automorphic representations of $\widetilde{\GL}_2$ into those of $\GL_2$.

In this talk, I would like to explain Waldspurger's work in the framework of representation theory.
I will explain all of Waldspurger's work except \cite{waldspurger81demientier}, which deals with the Fourier coefficients of automorphic forms of half-integral weight.
This latter work, which is based on the material explained here, is very important for number theory, but lies outside the framework of this talk.
Despite the fact that I have omitted many local proofs, I hope this talk will be useful to the mathematical community. 
A beautiful exposition of Waldspurger's work from the classical point of view has been given in a talk by Marie-France Vigneras \cite{vigneras79center}.