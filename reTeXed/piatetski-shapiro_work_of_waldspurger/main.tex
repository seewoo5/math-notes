% --- LaTeX Lecture Notes Template - S. Venkatraman ---

% --- Set document class and font size ---

\documentclass[letterpaper, 12pt]{article}

% --- Package imports ---

% Extended set of colors
\usepackage[dvipsnames]{xcolor}

\usepackage{
  amsmath, amsthm, amssymb, mathtools, dsfont, units,          % Math typesetting
  graphicx, wrapfig, subfig, float,                            % Figures and graphics formatting
  listings, color, inconsolata, pythonhighlight,               % Code formatting
  fancyhdr, sectsty, hyperref, enumerate, enumitem, framed }   % Headers/footers, section fonts, links, lists

% lipsum is just for generating placeholder text and can be removed
\usepackage{hyperref, lipsum} 


% --- Fonts ---

\usepackage{newpxtext, newpxmath, inconsolata}
\usepackage{amsfonts}

\usepackage{tikz}
\usepackage{tikz-cd}
\usepackage{enumitem}
\usepackage[title]{appendix}


% --- Page layout settings ---

% Set page margins
\usepackage[left=1.35in, right=1.35in, top=1.0in, bottom=.9in, headsep=.2in, footskip=0.35in]{geometry}

% Anchor footnotes to the bottom of the page
\usepackage[bottom]{footmisc}

% Set line spacing
\renewcommand{\baselinestretch}{1.2}

% Set spacing between paragraphs
\setlength{\parskip}{1.3mm}

% Allow multi-line equations to break onto the next page
\allowdisplaybreaks

% --- Page formatting settings ---

% Set image captions to be italicized
\usepackage[font={it,footnotesize}]{caption}

% Set link colors for labeled items (blue), citations (red), URLs (orange)
\hypersetup{colorlinks=true, linkcolor=RoyalBlue, citecolor=RedOrange, urlcolor=ForestGreen}

% Set font size for section titles (\large) and subtitles (\normalsize) 
\usepackage{titlesec}
\titleformat{\section}{\large\bfseries}{{\fontsize{19}{19}\selectfont\textreferencemark}\;\; }{0em}{}
\titleformat{\subsection}{\normalsize\bfseries\selectfont}{\thesubsection\;\;\;}{0em}{}

% Enumerated/bulleted lists: make numbers/bullets flush left
%\setlist[enumerate]{wide=2pt, leftmargin=16pt, labelwidth=0pt}
\setlist[itemize]{wide=0pt, leftmargin=16pt, labelwidth=10pt, align=left}

% --- Table of contents settings ---

\usepackage[subfigure]{tocloft}

% Reduce spacing between sections in table of contents
\setlength{\cftbeforesecskip}{.9ex}

% Remove indentation for sections
\cftsetindents{section}{0em}{0em}

% Set font size (\large) for table of contents title
\renewcommand{\cfttoctitlefont}{\large\bfseries}

% Remove numbers/bullets from section titles in table of contents
\makeatletter
\renewcommand{\cftsecpresnum}{\begin{lrbox}{\@tempboxa}}
\renewcommand{\cftsecaftersnum}{\end{lrbox}}
\makeatother

% --- Set path for images ---

\graphicspath{{Images/}{../Images/}}

% --- Math/Statistics commands ---

% Add a reference number to a single line of a multi-line equation
% Usage: "\numberthis\label{labelNameHere}" in an align or gather environment
\newcommand\numberthis{\addtocounter{equation}{1}\tag{\theequation}}

% Shortcut for bold text in math mode, e.g. $\b{X}$
\let\b\mathbf

% Shortcut for bold Greek letters, e.g. $\bg{\beta}$
\let\bg\boldsymbol

% Shortcut for calligraphic script, e.g. %\mc{M}$
\let\mc\mathcal

% \mathscr{(letter here)} is sometimes used to denote vector spaces
\usepackage[mathscr]{euscript}

% Convergence: right arrow with optional text on top
% E.g. $\converge[p]$ for converges in probability
\newcommand{\converge}[1][]{\xrightarrow{#1}}

% Weak convergence: harpoon symbol with optional text on top
% E.g. $\wconverge[n\to\infty]$
\newcommand{\wconverge}[1][]{\stackrel{#1}{\rightharpoonup}}

% Equality: equals sign with optional text on top
% E.g. $X \equals[d] Y$ for equality in distribution
\newcommand{\equals}[1][]{\stackrel{\smash{#1}}{=}}

% Normal distribution: arguments are the mean and variance
% E.g. $\normal{\mu}{\sigma}$
\newcommand{\normal}[2]{\mathcal{N}\left(#1,#2\right)}

% Uniform distribution: arguments are the left and right endpoints
% E.g. $\unif{0}{1}$
\newcommand{\unif}[2]{\text{Uniform}(#1,#2)}

% Independent and identically distributed random variables
% E.g. $ X_1,...,X_n \iid \normal{0}{1}$
\newcommand{\iid}{\stackrel{\smash{\text{iid}}}{\sim}}

% Sequences (this shortcut is mostly to reduce finger strain for small hands)
% E.g. to write $\{A_n\}_{n\geq 1}$, do $\bk{A_n}{n\geq 1}$
\newcommand{\bk}[2]{\{#1\}_{#2}}

\setcounter{section}{-1}

\newcommand{\SL}{\mathrm{SL}}
\newcommand{\Sp}{\mathrm{Sp}}
\newcommand{\Mp}{\mathrm{Mp}}
\newcommand{\GL}{\mathrm{GL}}
\newcommand{\SO}{\mathrm{SO}}
\newcommand{\SU}{\mathrm{SU}}
\newcommand{\PGL}{\mathrm{PGL}}
\newcommand{\PSL}{\mathrm{PSL}}
\newcommand{\rM}{\mathrm{M}}
\newcommand{\rN}{\mathrm{N}}
\newcommand{\rO}{\mathrm{O}}
\newcommand{\rP}{\mathrm{P}}
\newcommand{\JL}{\mathrm{JL}}
\newcommand{\stab}{\mathrm{Stab}}

\newcommand{\dd}{\mathrm{d}}

\newcommand{\bA}{\mathbb{A}}
\newcommand{\bR}{\mathbb{R}}
\newcommand{\bZ}{\mathbb{Z}}
\newcommand{\bC}{\mathbb{C}}
\newcommand{\bQ}{\mathbb{Q}}
\newcommand{\cS}{\mathcal{S}}
\newcommand{\cO}{\mathcal{O}}

% Math mode symbols for common sets and spaces. Example usage: $\R$
\newcommand{\R}{\mathbb{R}}	% Real numbers
\newcommand{\C}{\mathbb{C}}	% Complex numbers
\newcommand{\Q}{\mathbb{Q}}	% Rational numbers
\newcommand{\Z}{\mathbb{Z}}	% Integers
\newcommand{\N}{\mathbb{N}}	% Natural numbers
\newcommand{\F}{\mathcal{F}}	% Calligraphic F for a sigma algebra
\newcommand{\El}{\mathcal{L}}	% Calligraphic L, e.g. for L^p spaces

% Math mode symbols for probability
\newcommand{\pr}{\mathbb{P}}	% Probability measure
\newcommand{\E}{\mathbb{E}}	% Expectation, e.g. $\E(X)$
\newcommand{\var}{\text{Var}}	% Variance, e.g. $\var(X)$
\newcommand{\cov}{\text{Cov}}	% Covariance, e.g. $\cov(X,Y)$
\newcommand{\corr}{\text{Corr}}	% Correlation, e.g. $\corr(X,Y)$
\newcommand{\B}{\mathcal{B}}	% Borel sigma-algebra

% Other miscellaneous symbols
\newcommand{\tth}{\text{th}}	% Non-italicized 'th', e.g. $n^\tth$
\newcommand{\Oh}{\mathcal{O}}	% Big-O notation, e.g. $\O(n)$
\newcommand{\1}{\mathds{1}}	% Indicator function, e.g. $\1_A$

% Additional commands for math mode
\DeclareMathOperator*{\argmax}{argmax}		% Argmax, e.g. $\argmax_{x\in[0,1]} f(x)$
\DeclareMathOperator*{\argmin}{argmin}		% Argmin, e.g. $\argmin_{x\in[0,1]} f(x)$
\DeclareMathOperator*{\spann}{Span}		% Span, e.g. $\spann\{X_1,...,X_n\}$
\DeclareMathOperator*{\bias}{Bias}		% Bias, e.g. $\bias(\hat\theta)$
\DeclareMathOperator*{\ran}{ran}			% Range of an operator, e.g. $\ran(T) 
\DeclareMathOperator*{\dv}{d\!}			% Non-italicized 'with respect to', e.g. $\int f(x) \dv x$
\DeclareMathOperator*{\diag}{diag}		% Diagonal of a matrix, e.g. $\diag(M)$
\DeclareMathOperator*{\trace}{Tr}		% Trace of a matrix, e.g. $\trace(M)$
\DeclareMathOperator*{\supp}{supp}		% Support of a function, e.g., $\supp(f)$

% Numbered theorem, lemma, etc. settings - e.g., a definition, lemma, and theorem appearing in that 
% order in Lecture 2 will be numbered Definition 2.1, Lemma 2.2, Theorem 2.3. 
% Example usage: \begin{theorem}[Name of theorem] Theorem statement \end{theorem}
\theoremstyle{definition}
\newtheorem{theorem}{Theorem}[section]
\newtheorem{proposition}[theorem]{Proposition}
\newtheorem{lemma}[theorem]{Lemma}
\newtheorem{corollary}[theorem]{Corollary}
\newtheorem{definition}[theorem]{Definition}
\newtheorem{example}[theorem]{Example}
\newtheorem{remark}[theorem]{Remark}

% Un-numbered theorem, lemma, etc. settings
% Example usage: \begin{lemma*}[Name of lemma] Lemma statement \end{lemma*}
\newtheorem*{theorem*}{Theorem}
\newtheorem*{proposition*}{Proposition}
\newtheorem*{lemma*}{Lemma}
\newtheorem*{corollary*}{Corollary}
\newtheorem*{definition*}{Definition}
\newtheorem*{example*}{Example}
\newtheorem*{remark*}{Remark}
\newtheorem*{claim}{Claim}

% --- Left/right header text (to appear on every page) ---

% Do not include a line under header or above footer
\pagestyle{fancy}
\renewcommand{\footrulewidth}{0pt}
\renewcommand{\headrulewidth}{0pt}

% Right header text: Lecture number and title
\renewcommand{\sectionmark}[1]{\markright{#1} }
\fancyhead[R]{\small\textit{\nouppercase{\rightmark}}}

% Left header text: Short course title, hyperlinked to table of contents
\fancyhead[L]{\hyperref[sec:contents]{\small Work of Waldspurger}}

% --- Document starts here ---

\begin{document}

% --- Main title and subtitle ---

\title{Work of Waldspurger \\[1em]
\normalsize Re-\TeX ed by Seewoo Lee\footnote{seewoo5@berkeley.edu}}

% --- Author and date of last update ---

\author{Ilya Piatetski-Shapiro \\ Yale University \\ and \\ University of Tel Aviv}
\date{\normalsize\vspace{-1ex} Last updated: \today}

% --- Add title and table of contents ---

\maketitle
\tableofcontents\label{sec:contents}

% --- Main content: import lectures as subfiles ---

% \input{Lectures/Lecture1}

\section{Introduction}

In the years following Apery's discovery of his irrationality proofs for $\zeta(2), \zeta(3)$ (see \cite{van1979proof}), it has become clear that these proofs do not only have significance as irrationality proofs, but the numbers that occur in them serve as interesting examples for several phenomena in algebraic geometry and modular form theory.
See \cite{gessel1982some,beukers1982irrationality,beukers1985some,beukers1987another} for congruences of the Ap\'ery numbers and \cite{beukers1984family,stienstra1985picard} for geometrical and modular interpretations.\footnote{The original citations were in a different order, but it seems that this is the correct order.}
Furthermore, it turns out that Apery's proofs themselves are in fact simple consequences of elementary complex analysis on spaces of certain modular forms.
In the present paper we describe this analysis together with some generalisations in Theorems 1 to 5. 
For example, we prove that $8\zeta(3) - 5 \sqrt{5} L(3) \not \in \bQ(\sqrt{5})$, where $L(3) = \sum_{n=1}^{\infty} \left(\frac{n}{5}\right) n^{-3}$.
Although the use of modular forms in irrationality proofs looks promising at first sight, the yield of new irrationality results thus far is disappointingly low. 
However, in methods such as this it is easy to overlook some simple tricks that may give new interesting results.

The first section of this paper describes the general framework of the proofs.
This section may seem vague at first sight, but in combination with the proof of Theorem 1 we hope that things will be clear.
We have given the proof of Theorem 1 as extensively as possible in order to set it as an example for the other proofs, where we omit some minor details now and then. 
\section{Automorphic Forms on $\widetilde{\SL}_2$}
\label{sec:1}

Let $k$ be a global field.
The ad\'ele group $\SL_2(\bA)$ has a unique non-trivial two-fold covering $\widetilde{\SL}_2(\bA)$:
\[
1 \to \{\pm 1\} \to \widetilde{\SL}_2(\bA) \to \SL_2(\bA) \to 1.
\]
There is a unique embedding of $\SL_2(k)$ into $\widetilde{\SL}_2(\bA)$ such that the following diagram commutes.

\begin{center}
\begin{tikzcd}
 & \widetilde{\SL}_2(\bA) \arrow[d] \\
\SL_2(k) \arrow[r] \arrow[ru] & \SL_2(\bA)
\end{tikzcd}
\end{center}

This means covering splits over $\SL_2(k)$.
Similarly, there is an embedding of $N(\bA)$ into $\widetilde{\SL}_2(\bA)$, where $N$ is the upper unipotent subgroup of $\SL_2$.

Let $A_0$ denote the space of genuine cuspidal functions on $\widetilde{\SL}_2(\bA)$.
In particular, if $f \in A_0$, then

\begin{enumerate}[label=\roman*)]
    \item $f(\xi \gamma g) = \xi f(g)$ \qquad ($\xi \in \{ \pm 1 \}, \gamma \in \SL_2(k), g \in \widetilde{\SL}_2(\bA)$)
    \item $\int_{k\backslash \bA} f\left(\left(\begin{smallmatrix}
        1 & n \\ 0 & 1
    \end{smallmatrix}\right)g\right) \dd n = 0$.
\end{enumerate}

Under right translation, $A_0$ decomposes discretely into a countable number of irreducible subspaces.
An irreducible representation of $\widetilde{\SL}_2(\bA)$ which occurs in $A_0$ is called a genuine automorphic cuspidal
representation.
Let $A_{00}$ denote the subspace of forms in $A_0$ orthogonal to the Weil representations of $\widetilde{\SL}_2(\bA)$.

\begin{theorem}[Multiplicity One \cite{waldspurger80shimura}]
The multiplicity of an irreducible genuine automorphic cuspidal representation in $A_{00}$ is one.
\end{theorem}
\begin{remark*}
If $\sigma$ is a genuine irreducible automorphic cuspidal representation lying in a Weil representation of $\widetilde{\SL}_2(\bA)$, then multiplicity one is obvious.
\end{remark*}

If $\psi$ is a character of $k \backslash \bA$, and $f \in A_{00}$, the $\psi$-Fourier coefficient of $f$ is defined to be
\[
f_\psi(g) = \int_{k \backslash \bA} f\left(\begin{pmatrix} 1 & n \\ & 1\end{pmatrix}g\right) \psi(n) \dd n \qquad (g \in \widetilde{\SL}_2(\bA))
\]
The multiplicity result follows from the uniqueness of Whittaker models for $\widetilde{\SL}_2(\bA)$, and the following result of Waldspurger.

\begin{theorem}[\cite{waldspurger81demientier,waldspurger84shimura}]
\label{thm:1.2}
Let $(\sigma, V)$ be a genuine irreducible automorphic cuspidal representation of $\widetilde{\SL}_2(\bA)$.
If $v \mapsto \varphi(v)$ ($v \in V, \varphi(v) \in A_{00}$)
is an embedding of $(\sigma, V)$ into $A_{00}$, then the vanishing of the $\psi$-Fourier coefficient $\varphi(v)_\psi$ depends only on $(\sigma, V)$ as an abstract representation, and not on the embedding $\varphi$.
\end{theorem}


\begin{proof}[Proof of the multiplicity one]
Suppose $v \mapsto \varphi'(v)$ and $v \mapsto \varphi''(v)$
($v \in v$) are two distinct embeddings of an irreducible genuine automorphic cuspidal representation $(\sigma, V)$ into $A_0$.
We may select a character $\psi$ of $k \backslash \bA$ so that the $\psi$-Fourier coefficient $\varphi'(v)_\psi$ does
not vanish for some $v \in V$.
Let us consider the $\psi$-Fourier coefficient
$\varphi''(v)_\psi$.
If $\varphi''(v)_\psi$ vanishes, then Theorem \ref{thm:1.2} says $\varphi'(v)_\psi$ must also vanish, a contradiction.
If $\varphi''(v)$ does not vanish, then the uniqueness of Whittaker models for $\widetilde{\SL}_2(\bA)$ tells us that $\varphi''(v)_\psi = c\varphi'(v)_\psi$ for some constant $c$.
Since $\varphi'$ and $\varphi''$ are assumed to be distinct
embeddings of $(\sigma, V)$ into $A_{00}$, the map $w \mapsto \varphi''(w) - c\varphi'(w)$ is a non-trivial embedding of 
 $(\sigma, V)$ into $A_{00}$.
The $\psi$-Fourier coefficient of $\varphi''(v) - c\varphi'(v)$ vanishes.
This again contradicts Theorem \ref{thm:1.2}; therefore $(\sigma, V)$ must occur in $A_{00}$ with multiplicity one.
\end{proof}

Two irreducible genuine automorphic cuspidal representations of $\widetilde{\SL}_2(\bA)$, $\sigma = \otimes_v \sigma_v$ and $\sigma' = \otimes_v \sigma_{v}'$, are said to be nearly equivalent if $\sigma_v \simeq \sigma_v'$ for almost all places $v$.
Let $\ell(\sigma)$ denote the set of irreducible genuine automorphic cuspidal representations nearly equivalent to $\sigma$.
$\ell(\sigma)$, of course, just measures departure from strong multiplicity one.
In order to determine the set $\ell(\sigma)$, Waldspurger
has defined an involution $\sigma \mapsto \sigma^W$ whenever $\sigma$ is a discrete series representation of $\widetilde{\SL}_2(k_v)$.
If $\sigma = \otimes_v \sigma_v \subset A_{00}$, define
\[
\Sigma = \{ v: \sigma_v\text{ is a discrete series representation}\}.
\]
If $M \subseteq \Sigma$, and $|M|$ is even, put
\[
\sigma^M = \otimes_v \sigma_v^M \quad \text{where}\quad \sigma_v^M = \begin{cases} \sigma_v & \text{if }v\not\in M \\ \sigma_V^W & \text{if }v \in M.\end{cases}
\]

The relationship of the $\sigma^M$'s and $\ell(\sigma)$ is given in the following theorem.

\begin{theorem}[\cite{waldspurger91quaternion}]
\label{thm:1.3}
Any representation in $\ell(\sigma)$ is of the form $\sigma^M$ for some $M \subseteq \Sigma$.
\end{theorem}

\begin{corollary}
$|\ell(\sigma)| = 2^{|\Sigma| - 1}$.
\end{corollary}

\begin{remark*}
Recall that $\left(\begin{smallmatrix}
    -1 & 0 \\ 0 & -1
\end{smallmatrix}\right)$ lies in the center of $\widetilde{\SL}_2(k_v)$.
Waldspurger has shown that $\sigma_v^M\left(\begin{smallmatrix}
    -1 & 0 \\ 0 & -1
\end{smallmatrix}\right) = - \sigma_v\left(\begin{smallmatrix}
    -1 & 0 \\ 0 & -1
\end{smallmatrix}\right)$.
Since $\left(\begin{smallmatrix}
    -1 & 0 \\ 0 & -1
\end{smallmatrix}\right) \in \SL_2(k)$, it follows that if $M \subseteq \Sigma$ has an odd number of elements, $\sigma^M$ cannot be an automorphic representation.
\end{remark*}

\section{The Oscillator Representation Over a Local Field}
\label{sec:2}

Let $k$ be a local field, and let $X$ be a $2n$-dimensional vector space over $k$ with a symplectic form $\langle\,,\,\rangle$.
If $X = X_1 \oplus X_2$ is a polarization of $X$, let $P$ be the subgroup of $\Sp(X)$ which preserves $X_2$.
If $\psi$ is a non-trivial character of $k$, let $\omega_\psi$ be the oscillator representation of $\Mp_{2n}(k) = \Mp(X)$, the double cover of $\Sp_{2n}(k) = \Sp(X)$, acts on the Schwartz-Bruhat space $\cS(X_1)$.

Let us now consider the 3-dimensional vector space $M = \{m \in \rM_2(k): \trace(m) = 0\}$.
$\PGL_2$ acts on $M$ by conjugation:
\[
m \mapsto g^{-1} m g \qquad(g \in \PGL_2, m \in M).
\]
This conjugation action preserves the symmetric form $q(x) = -\det(x)$.
Let $Y$ be a 2-dimensional vector space over $k$ with a symplectic form $\langle\,,\,\rangle$.
Define a symplectic vector space $X$ by $X = M \otimes_k Y$,
$\langle m_1 \otimes y_1, m_2 \otimes y_2\rangle = (m_1, m_2) \langle y_1, y_2 \rangle$.
Since $\PGL_2$ and $\SL_2$ preserve
the forms $(\,,\,)$ and $\langle\,,\,\rangle$ respectively, there is a natural embedding of $\PGL_2 \times \SL_2$ into $\Sp(X) = \Sp_6$.
Our aim is to use the oscillator representation of $\Mp_6$ to define a correspondence between certain irreducible representations of $\PGL_2$ and certain irreducible representations of $\widetilde{\SL}_2$.
Waldspurger has given a different definition of the correspondence based on explicit integral formulas.
These integral formulas, though complicated and defined only for the case $\PGL_2$, $\widetilde{\SL}_2$, yield much more information about the correspondence.

Let $T$ be a subgroup of $G = \PGL_2$ and let $N$ a subgroup of $H = \widetilde{\SL}_2$.
Let $\alpha$ and $\beta$ be characters of $T$ and $N$ respectively.
Let $X = X_1 \oplus X_2$ be a polarization of $X$ such that $T \times N \subset P$.
Let us suppose that $x_1 \in X_1$ is a vector such that the linear functional
\[
\phi \mapsto \phi(x_1) \qquad (\phi \in \cS(X_1))
\]
transforms under $T\times N$ by $\alpha \times \beta$, i.e.,
\[
\omega_\psi(t, n)\phi(x_1) = \alpha(t) \beta(n) \phi(x_1).
\]
Let $(\pi, V)$ be an irreducible admissible representation of $\PGL_2$ and let us assume that $\ell$ is a linear functional on $V$ such that $\ell(\pi(t)v) = \alpha(t)^{-1}\ell(v)$ ($t \in T$).
If the integral
\[
F(h) = \int_{T\backslash G} \omega_\psi(g, h) \phi(x_1) \ell(\pi(g)v)  \dd g \qquad (h \in H)
\]
converges, then $F(nh) = \beta(n)F(h)$ ($n \in N$).
Let $W$ be the space of all the functions $F$ obtained in this fashion by varying $\phi$ and $v$.
$\widetilde{\SL}_2$ acts on $W$ by right translation. We shall denote this representation by $\theta(\pi, \psi)$.
Conversely, given an irreducible admissible genuine representation $\sigma$ of $\widetilde{\SL}_2$ it is possible to define a representation
$\theta(\sigma, \psi)$ of $\PGL_2$, which may be a zero representation.

In order to explain Waldspurger's integral formulas for the correspondence, we have to consider two polarizations of $X$. For the first polarization, let $y_1, y_2 \in Y$ be a symplectic basis, i. e., $\langle y_1, y_2 \rangle = 1$, and put $X_1 = M \otimes y_1, X_2 = M \otimes y_2$.
Let $m_1$ be an element of $M$ such that $\det m_1 \neq 0$ and let $T = \stab(m_1)$.
$T$ is a torus in $G$.
Let $N$ be the unipotent subgroup of $\SL_2$ which preserves $Y_2$.
Let $\alpha$ be the trivial character, and $\beta$ the character $\beta\left(\begin{smallmatrix} 1 & n \\ 0 & 1\end{smallmatrix}\right) = \psi(q(m_1)n)$.
We shall now describe the second polarization which has the property that the unipotent subgroups of $\PGL_2$ and $\widetilde{\SL}_2$ both lie in $P$.
Let $e_1, e_2, e_3$ be a basis of $M$ such that the matrix of the symmetric form is
\[
\begin{pmatrix}
0 & 0 & 1 \\ 0 & 1 & 0 \\ 1 & 0 & 0
\end{pmatrix}.
\]
Define $X_1 = e_1 \otimes Y + e_2 \otimes ky_1$ and $X_2 = e_3 \otimes Y + e_2 \otimes ky_2$.
It is clear that the unipotent subgroup of $G = \PGL_2$ which preserves $e_3$ also preserves $X_2$.
We shall denote this subgroup by $T$. Similarly, the unipotent subgroup $N$ of $\SL_2$ which preserves $y_2$ preserves $X_2$.
Let $x_1 = e_1 \otimes y_2 + \lambda e_2 \otimes y_1$ and define $\alpha$ and $\beta$ by
\begin{align*}
    \alpha\begin{pmatrix}
        1 & t \\ 0 & 1
    \end{pmatrix} &= \psi(-\lambda t) \\
    \beta\begin{pmatrix}
        1 & n \\ 0 & 1
    \end{pmatrix} &= \psi(\lambda^2 n).
\end{align*}
Waldspurger has proved the following theorems:
\begin{theorem}[\cite{waldspurger80shimura}]
\label{thm:2.1}
Let $T$ and $N$ be as above.
If $(\pi, V)$ (respectively $(\sigma, V)$) is an irreducible admissible representation of $\PGL_2$ (respectively $\widetilde{\SL}_2$), then the representation of $\widetilde{\SL}_2$ (respectively $\PGL_2$) obtained from the above integral formulas is irreducible admissible and depends only on the additive character $\psi$.
It is independent of the choice of the subgroups $T$ and $N$ and the characters $\alpha$ and $\beta$.
\end{theorem}
\begin{theorem}[\cite{waldspurger91quaternion}]
\label{thm:2.2}
Let $\xi \in k^\times$, and let $\chi_\xi$ be the quadratic character of $k^\times$ associated to $k(\sqrt{\xi})$.
If and $\theta(\sigma, \psi)$ and $\theta(\sigma, \psi^\xi)$ are both nonzero representations of $\PGL_2$, then $\theta(\sigma, \psi^\xi) = \theta(\sigma, \psi) \otimes \chi_\xi$.
\end{theorem}
\begin{remark*}
$\theta(\pi, \psi)$ is non-zero for any irreducible admissible representation $\pi$ of $\PGL_2$ and any $\psi$.
It follows from this that any irreducible admissible representation of $\PGL_2$ admits a linear functional which is invariant with respect to the split torus.
$\theta(\sigma, \psi)$ is nonzero if and only if $\sigma$ admits a linear functional which transforms under $N$ by $\psi^{-1}$.
\end{remark*}

Let us now make a few remarks about a similar construction for the quaternion algebra $D$ over $k$.
Let $M'$ be the elements of trace zero in $D$, and let $q$ be the symmetric form on $M'$ given by $q(m) = -\rN_D(x)$.
$\rP \dd^\times$ acts on $M'$ by conjugation, and this action preserves the form $q$.
We can introduce a symplectic space $X' = M'\otimes_k Y$ and as above, we have an embedding $\rP \dd^\times \times \SL_2 \hookrightarrow \Sp_6$.
In an analogous fashion, we can also introduce integral formulas to describe a correspondence between some of the irreducible admissible representations of $\rP \dd^\times$ and some of the irreducible genuine admissible representations of $\widetilde{\SL}_2$.
The analogues of Theorems \ref{thm:2.1} and \ref{thm:2.2} are also true for the quaternion algebra.
If $\sigma$ (respectively $\pi$) is an irreducible admissible representation of $\widetilde{\SL}_2$ (respectively $\rP \dd^\times$), we shall denote the corresponding representation of $\rP \dd^\times$ (respectively $\widetilde{\SL}_2$) by $\theta'(\sigma, \psi)$ (respectively $\theta'(\pi,\psi)$).

From the explicit integral formulas, it is easy to show that
$\theta'(\pi, \psi)$ does not admit a linear functional which transforms under $N$ by $\psi^{-1}$.
This together with the remark after Theorem \ref{thm:2.2} implies that the representations $\theta(\sigma, \psi)$ and $\theta'(\sigma, \psi)$ cannot both be non-zero representations.
However, Waldspurger has the following result.
\begin{theorem}[\cite{waldspurger91quaternion}]
\label{thm:2.3}
One of the representations $\theta(\sigma, \psi)$ and $\theta'(\sigma, \psi)$ is always non-zero.
\end{theorem}

\begin{claim}
$\theta'(\pi', \psi')$ is non-zero if and only if $\pi'$ is a spherical representation, i.e., $\pi'$ possesses a $T$-invariant vector for some $T \subset \rP \dd^\times$.
\end{claim}

\begin{proof}(Waldspurger)
Consider for the moment, an irreducible admissible representation $\pi$ of $\PGL_2$.
If $\chi$ is a quadratic character of $k^\times$, we define the Waldspurger symbol as follows.
Let $\varepsilon(\pi, s, \psi)$ be the $\varepsilon$-factor introduced in \cite{jl70gl2}.
It is easy to check that $\varepsilon(\pi, \frac{1}{2}, \psi) = \pm 1$ does not depend on $\psi$.
Let $\varepsilon(\pi, \frac{1}{2})$ denote $\varepsilon(\pi, \frac{1}{2}, \psi)$.
We then define $\left(\frac{\chi}{\pi}\right)$ by
\[
\varepsilon\left(\pi \otimes \chi, \frac{1}{2}\right) = \left( \frac{\chi}{\pi} \right) \chi(-1) \varepsilon\left( \pi, \frac{1}{2}\right).
\]
$\left(\frac{\chi}{\pi}\right) = 1$, and if $\chi$ is the trivial character, then $\left(\frac{\chi}{\pi}\right) = 1$.
It is easy to see that if $\pi$ is an irreducible principal series representation, then $\left(\frac{\chi}{\pi}\right) =1$ for all $\chi$.
On the other hand, if $\pi$ is a discrete series representation, then there exists a $\chi$ such that $\left(\frac{\chi}{\pi}\right) = -1$ \cite{waldspurger91quaternion}.
Let us now return to the proof of the claim.
Let $\pi$ be the discrete series representation of $\PGL_2$ associated to $\pi'$ under the Jacquet-Langlands map.
Let $\chi$ be a quadratic character of $k^\times$ such
that $\left(\frac{\chi}{\pi}\right) = -1$, and denote by $K = k(\sqrt{\xi})$ the field corresponding to $\chi$.
Put
\[
\sigma_1 = \theta(\pi \otimes \chi, \psi^{\xi}),\qquad \sigma = \theta(\pi, \psi).
\]
Waldspurger has proved that 
\[
\sigma_1\begin{pmatrix}
    -1 & 0 \\ 0 & -1
\end{pmatrix} = \left(\frac{\chi}{\pi}\right) \sigma\begin{pmatrix}
    -1 & 0 \\ 0 & -1
\end{pmatrix}.
\]
Since $\left(\frac{\chi}{\pi}\right) = -1$, $\sigma_1 \neq \sigma$.
This means that $\sigma$ does not admit a $\psi^\xi$-linear functional, for if it did, $\theta(\sigma, (\psi^{\xi})^{-1}) \neq 0$, so $\theta(\sigma_1, (\psi^{\xi})^{-1}) = \theta(\sigma, \psi^{-1}) \otimes \chi_\xi$ which would imply $\sigma_1 = \sigma$, a contradiction.
Now, Theorem \ref{thm:2.3} tells us that $\theta'(\sigma, (\psi^{\xi})^{-1}) \neq 0$ and so $\theta(\pi', \psi^{\xi}) \neq 0$ which means $\pi'$ is spherical.
\end{proof}

The next theorem defines Waldspurger's involution.

\begin{theorem}[\cite{waldspurger91quaternion}]
\label{thm:2.4}
Let $\sigma$ be an irreducible representation of $\widetilde{\SL}_2$, and let $\psi$ be a character of $k$ such that $\theta(\sigma, \psi) \neq 0$.
The composition of 3 maps
\[
    \sigma \mapsto \theta(\sigma, \psi) = \pi \xmapsto{\JL} \pi' \mapsto \theta(\pi', \psi^{-1})
\]
(where $\JL$ means the Jacquet-Langlands map) is independent of $\psi$ and defines an involution.
\end{theorem}

Finally, it is not difficult to prove
\begin{theorem}
\label{thm:2.5}
If $\pi = \theta(\sigma, \psi) \neq 0$, then
\[
\theta(\pi \otimes \chi_\xi, \psi^\xi) = \begin{cases} \sigma & \text{if } \left(\frac{\chi_\xi}{\pi}\right) = 1 \\
\sigma^W & \text{if } \left(\frac{\chi_\xi}{\pi}\right) = -1\end{cases}
\]
where $\chi_\xi$ is the character associated to $k(\sqrt{\xi})$.
\end{theorem}
\section{The $\theta$-correspondence}
\label{sec:3}

Let $k$ be a global field.
We shall use the same notion globally as was previously introduced locally.
The global Weil (oscillator) representation $\omega_\psi$ acts on $\cS(X_1(\bA))$.
It is easy to see that it is the tensor product of the local Weil representations.
Let $X = X_1 \oplus X_2$ be the standard polarization of $X$, and identify $X_1$ wiht $M$.
For $\phi \in \cS(X_1(\bA))$,
\[
\vartheta_\psi^\phi(g, h) = \sum_{x \in X_1(k)} \omega_\psi(g, h) \phi(x)\qquad (g \in G(\bA), h \in \widetilde{\SL}_2(\bA)).
\]
Here, $G$ is either $\PGL_2$ or $\rP D^\times$.
It is well known that $\vartheta_\psi^\phi$ is an automorphic function on $G(\bA) \times \widetilde{\SL}_2(\bA)$ of moderate growth.

The theta function $\vartheta_\psi^\phi$'s can be used to define a correspondence between the automorphic representation of $G(\bA)$ and those of $\widetilde{\SL}_2(\bA)$.
To describe this correspondence, let $\pi$ be an irreducible automorphic cuspidal representation of $G(\bA)$.
If $f \in \pi \subset A_0$, put
\[
\varphi(h) := \int_{G(k) \backslash G(\bA)} \vartheta_\psi^\phi(g, h) f(g) \dd g.
\]
In the case $G = \rP D^\times$, we assume that $\int_{G(k)\backslash G(\bA)} f(g) \dd g = 0$.
The fact that $\vartheta_\psi^\phi$ is a function of moderate grwoth on $(G(k) \times \SL_2(k)) \backslash (G(\bA) \times \widetilde{SL}_2(\bA))$ means that the integral is well-defined, and that $\varphi$ is a function on $\SL_2(k) \backslash \widetilde{\SL}_2(\bA)$.

\begin{claim}
$\varphi$ is a cusp form.
\end{claim}

\begin{proof}
It is enough to show that $\int_{k\backslash \bA} \varphi\left(\begin{smallmatrix}
    1 & z \\ 0 & 1
\end{smallmatrix}\right)\dd z = 0$.
\begin{align*}
\int_{k \backslash \bA} \varphi \begin{pmatrix}
    1 & z \\ 0 & 1
\end{pmatrix} \dd z &= \int_{k \backslash \bA} \int_{G(k) \backslash G(\bA)}  \sum_{x \in X(k)} \omega_\psi\left(g \begin{pmatrix}
    1 & z \\ 0 & 1
\end{pmatrix}\right)\phi(x) f(g) \dd g \dd z \\
&= \int_{G(k) \backslash G(\bA)} \sum_{x \in X(k)} \omega_\psi(g) \phi(x) f(g) \int_{k \backslash \bA} \psi(z q(x)) \dd z \dd g.
\end{align*}
The inner integral $\int_{k \backslash \bA} \psi(z q(x)) \dd z$ is zero unless $q(x) = 0$.
If $G = \rP D^\times$, then $q(x) = 0$ if and only if $x = 0$, and the integral becomes
\[
\int_{k \backslash \bA} \varphi \begin{pmatrix}
    1 & z \\ 0 & 1
\end{pmatrix} \dd z = \int_{G(k) \backslash G(\bA)} \phi(0) f(g) \dd g = 0.
\]
If $G = \PGL_2$, then $q(x)=0$ means either $x = 0$ or $x$ is a non-zero nilpotent element of $M_2(k)$. The integral in this situation is
\begin{align*}
    \int_{k \backslash \bA} \varphi \begin{pmatrix}
        1 & z \\ 0 & 1
    \end{pmatrix} \dd z &= \int_{G(k) \backslash G(\bA)} \phi(0) f(g) \dd g \\
    &+ \int_{G(k) \backslash G(\bA)} \sum_{N(k) \backslash G(k)} \phi \left(g^{-1} \gamma^{-1} \begin{pmatrix}
        0 & 1 \\ 0 & 0
    \end{pmatrix} \gamma g\right) f(g) \dd g \\
    &= 0 + \int_{N(\bA) \backslash G(\bA)} \phi(g^{-1} xg) f(\gamma^{-1} g) \dd g \\
    &= \int_{N(\bA) \backslash G(\bA)} \omega_\psi(g) \phi(x) \int_{N(k) \backslash N(\bA)} f(ng) \dd n \dd g = 0.
\end{align*}
Here $N$ is centralizer in $G$ of $\left(\begin{smallmatrix}
    0 & 1 \\ 0& 0
\end{smallmatrix}\right)$, and $\int_{N(k) \backslash N(\bA)} f(ng) \dd n = 0$, $\int_{N(\bA) \backslash G(\bA)} f(g) \dd g = 0$  since $f$ is a cusp form.
\end{proof}

Let $\theta(\pi, \psi)$ denote the representation of $\widetilde{\SL}_2(\bA)$ spanned by the $\varphi$'s ($\phi \in \cS(X_1(\bA)), f \in \pi$).
$\theta(\pi, \psi)$ is a genuine automorphic cuspidal representation of $\widetilde{\SL}_2(\bA)$.

\begin{theorem}[\cite{waldspurger80shimura}]
\label{thm:3.1}
The $\theta$-correspondence $\pi \mapsto \theta(\pi, \psi)$ is compatible with the local correspondences introduced in \S \ref{sec:2}.
\end{theorem}

\begin{proof}
Let $\pi$ be an irreducible automorphic cuspidal representation of $G(\bA)$. For $f \in \pi$, and $\phi \in \cS(X_1(\bA))$, let $\varphi$ again be the cusp form
\[
\varphi(h) = \int_{G(k) \backslash G(\bA)} \vartheta_{\psi}^\phi(g, h) f(g) \dd g.
\]
If $a \in k^\times$, then a calculation similar to the one used to show $\varphi$ is a cusp form shows
\begin{align}
    \varphi_a(1) &:= \int_{k \backslash \bA} \varphi \begin{pmatrix}
        1 & z \\ 0& 1
    \end{pmatrix} \overline{\psi(az)} \dd z \nonumber \\ 
    &= \int_{T^a(\bA) \backslash G(\bA)} \omega_\psi(g) \phi(x_a) \int_{T^a(k) \backslash T^a(\bA)} f(tg) \dd t \dd g. \label{eqn:F}
\end{align}
Here, $x_a$ is any element in $X$, such that $q(x_a) = a$ (if $G = \rP D^\times$, we assume $a$ is representable by $q$), and $T^a$ is the stabilizer of $x_a$.
$T^a$ is a torus in $G$.
Put
\[
U(f, g) := \int_{T^a(k) \backslash T^a(\bA)} f(tg) \dd t\qquad (g \in G(\bA), f \in \pi).
\]
The function $U(f, -)$ satisfies the property $U(f, tg) = U(f, g)$ for $t \in T^a(\bA)$, and the linear function $\ell: f \mapsto U(f, 1)$ is a linear functional on $(f\in \pi)$ for which $\ell(\pi(t)f) = \ell(f)$ ($t \in T^a(\bA)$).
Locally, such a linear functional is unique, hence $\ell$ is globally unique and
\[
U(f, -) = \otimes_v U_v(-),
\]
where $U_v$ is a function on $G_v$ such that $U_v(t_v g_v) = U_v(g_v)$ ($t_v \in T^a(k_v), g_v \in G(k_v)$).
Under right translations by $G_v$ on $T^a_v \backslash G_v$, $U_v$ generates a representation equivalent to $\pi_v$. In analogy with the global formula
\[
\varphi_a(h) = \int_{T^a(\bA) \backslash G(\bA)} \omega_\psi(g, h) \phi(x_a) U(f, g) \dd g,
\]
if $U$ is an element in the space generated by $U_v$, and if
\[
W_{\psi^a}(h) := \int_{T^a \backslash G_v} \omega_{\phi, v}(g, h) \phi(x_a) U(g) \dd g
\]
then $W_{\psi^a}\left(\left(\begin{smallmatrix}
    1 & z \\ 0 & 1
\end{smallmatrix}\right)h\right) = \psi_v(za) W_{\psi_a}(h)$.
\end{proof}

\begin{theorem}[\cite{waldspurger80shimura}]
\label{thm:3.2}
The $\theta$-correspondence  is a 1-1 correspondence between certain automorphic cuspidal irreducible representations of $G(\bA)$ and certain genuine automorphic cuspidal irreducible representations of $\widetilde{\SL}_2(\bA)$.
\end{theorem}


\begin{theorem}[\cite{waldspurger80shimura,hps83sp4}]
\label{thm:3.3} 
Let $G = \PGL_2$. Suppose $\sigma \subset A_{00}$, and $\pi$ is an automorphic cuspidal representation of $\PGL_2(\bA)$. Then
\begin{enumerate}
    \item $\theta(\sigma, \psi^{-1}) \neq 0$ if and only if $\sigma$ possesses a nonvanishing $\psi$-Fourier coefficient.
    \item $\theta(\pi, \psi) \neq 0$ if and only if $L(\pi, \frac{1}{2}) \neq 0$.
\end{enumerate}
\end{theorem}

\begin{proof}
In order to prove this theroem, we must use a polarization for which the usual subgroups of $\PGL_2(\bA)$ and $\widetilde{\SL}_2(\bA)$ lie inside $P$.
As before, let $M$ be the elements of $M_2(k)$ of trace zero, and let $q(m) = -\det(m)$.
Let $Y$ be a 2-dimensional symplectic vector spcaes over $k$ with form $\langle\,,\,\rangle$ and symplectic basis $y_1, y_2$.
Let $e_1, e_2, e_3$ be a basis of $M$ such that $q$ has the matrix $\left(\begin{smallmatrix}
    0 & 0 & 1 \\0 &1 & 0 \\1 & 0 & 0
\end{smallmatrix}\right)$.
Put $X_1 = e_1 \otimes Y +e_2 \otimes ky_1, X_2 = e_3 \otimes Y + e_2 \otimes ky_2$.
Suppose $\sigma$ is an irreducible genuine automorphic representation of $\widetilde{\SL}_2(\bA)$ lying in $A_{00}$.
If $\varphi \in \sigma$, let $f(g) = \int_{\SL_2(k) \backslash \SL_2(\bA)} \vartheta_\psi^\phi(g, h) \varphi(h) \dd h$.
We can identify $X_1$ with $Y \oplus k$, and we can choose $\phi$ in the form $\phi = \phi_1 \phi_2$, where $\phi_1 \in \cS(Y(\bA)), \phi_2 \in \cS(\bA)$.
In this situation,
\[
\vartheta_\psi^\phi(1, h) = F_1(h) F_2(h)
\]
where
\begin{align*}
    F_1(h) &= \sum_{Y(k)} \phi_1(yh) = \phi_1(0) + \sum_{\gamma \in B(k) \backslash \SL_2(k)} \phi_1(y_2 \gamma) \\
    F_2(h) &= \sum_{t \in k} \omega_{\psi}'(h) \phi_2(t)
\end{align*}
In the formula for $F_2$, $\omega_\psi'$ is the 1-dimensional Weil representation.
\begin{align*}
    f(1) &= \int_{\SL_(k) \backslash \SL_2(\bA)} \phi_1(0) F_2(h) \varphi(h) \dd h + \int_{N(k) \backslash \SL_2(\bA)} \phi_1(y_2 h) F_2(h) \varphi(h) \dd h.
\end{align*}
Since $\sigma \in A_{00}$, and $F_2$ lies in the space of the Weil representation of $\widetilde{\SL}_2(\bA)$, the first integral is zero.
It follows that $\theta(\sigma, \psi^{-1}) \neq 0$ if and only if the second integral does not vanish identically.
\begin{align*}
    f(1) &= \int_{N(k) \backslash \SL_2(\bA)} \phi_1(y_2 h) F_2(h) \varphi(h) \dd h \\
    &= \sum_{t \in k} \int_{N(k) \backslash \SL_2(\bA)} \phi_1(y_2 h) \omega_\psi'(h) \phi_2(t) \varphi(h) \dd h.
\end{align*}
Since $\phi_1(y_2 nh) = \phi_1(y_2 h)$ and $\omega_\psi'(nh) \phi_2(t) = \psi(t^2 n) \phi_2(t)$ ($n = \left(\begin{smallmatrix}
    1 & n \\ 0 & 1
\end{smallmatrix}\right) \in N(\bA)$) it follows that
\[
f(1) = \sum_{t \in k} \int_{N(\bA) \backslash \SL_2(\bA)} \phi_1(y_2 h) \omega_\psi'(h) \phi_2(t) \varphi_{\psi^{t^2}}(h) \dd h.
\]
Thus, if $\theta(\sigma, \psi^{-1}) \neq 0$, then there exists a $t$ for which $\varphi_{t^2}$ is non-zero.
This means $\sigma$ possesses a non-zero $\psi$-Fourier coefficient.
Conversely, now suppose $\sigma$ possesses a non-vanishing $\psi$-Fourier coefficient.
Let
\begin{align*}
    f_t(1) &= \int_{N(\bA) \backslash \SL_2(\bA)} \phi_1(y_2 h) \omega_\psi'(h) \phi_2(t) \varphi_{\psi^{t^2}}(h) \dd h \\
    &= \int_{N(\bA) \backslash \SL_2(\bA)} \omega_\psi(h) \phi(y_2, t)  \varphi_{\psi^{t^2}}(h) \dd h.
\end{align*}
The latter formula allows us to define $f_t(1)$ for arbitrary $\phi$.
In this situation, we still have $f(1) = \sum_{t \in k^\times} f_t(1)$. ($f_0(1) = 0$, since $f$ is a cusp form).
Let $N$ be the unipotent subgroup of $\PGL_2$.
For $n \in N$
\[
\omega_\psi(n) \phi(y_2, t) = \psi(tn) \phi(y_2, t).
\]
It follows form this formula that $f_t(1)$ is a Fourier coefficient of $f$.
Therefore, if $\varphi_\psi \neq 0$, then $f_t(1) \neq 0$ and so $\theta(\sigma, \psi^{-1}) \neq 0$.
To prove the second part of the theorem, we use the standard polarization.
If $\sigma =\theta(\pi, \psi) \neq 0$, then $\theta(\sigma, \psi^{-1})$ equals $\pi$.
This means by part 1 that $\sigma$ possesses a non-zero $\psi$-Fourier coefficient.
If $T$ is the split torus in $\PGL_2$, then formula \eqref{eqn:F} in the proof of Theorem \ref{thm:3.1} tells us that
\[
\int_{T(k) \backslash T(\bA)} f(t) \dd t \neq 0.
\]
From the Jacquet-Langlands theory of L-functions, it is known that for an appropriate choice of $f$,
\[
L(\pi, s) = \int_{T(k) \backslash T(\bA)} f(t) |t|^{s - \frac{1}{2}} \dd t.
\]
In particular,
\[
L\left(\pi, \frac{1}{2}\right) = \int_{T(k)\backslash T(\bA)} f(t) \dd t \neq 0.
\]
Conversely, if $L(\frac{1}{2}, \pi) \neq 0$, then it is clear that $\int_{T(k) \backslash T(\bA)} f(t) \dd t \neq 0$, and hence that $\theta(\pi, \psi) \neq 0$.
\end{proof}
\section{Non-vanishing of a Fourier Coefficient}
\label{sec:4}

In this section we will prove Theorem \ref{thm:1.2} that the non-vanishing of the $\psi$-Fourier coefficient of $\sigma$ depends on $\sigma$ only as an abstract representation.
Let $\sigma \subset A_{00}$ and let $\psi$ be a non-trivial character of $k \backslash \bA$.
There exists a $\xi \in k^\times$ such that $\theta(\sigma, \psi^\xi) \neq 0$.
Define $W(\sigma, \psi)$ to be $\theta(\sigma, \psi^\xi) \otimes \chi_\xi$.
By Theorem \ref{thm:2.2}, $W(\sigma, \psi)$ depends only on $\psi$.
Define $L_\psi(\sigma, s)$ to be $L(W(\sigma, \psi), s)$.

\begin{theorem}
\label{thm:4.1}
Let $\sigma = \otimes_v \sigma_v \subseteq A_{00}$.
$\sigma$ admits a non-zero $\psi$-Fourier coefficient if and only if
\begin{enumerate}[label=\roman*)]
    \item at each place $v$, there is a linear functional $\ell_v$ on the space $W$ of $\sigma_v$ such that
    \[
        \ell_v\left(\sigma\begin{pmatrix}
            1 & t \\ 0 & 1
        \end{pmatrix}w\right) = \psi_v(t) \sigma_v(w) \qquad (w \in W),
    \]
    \item $L_\psi(\sigma, \frac{1}{2}) \neq 0$.
\end{enumerate}
\end{theorem}

\begin{proof}
If $\sigma$ admits a non-zero $\psi$-Fourier coefficient, it is clear that i) is satisfied.
ii) follows from Theorem \ref{thm:3.3}.
In order to prove the converse statement, Waldspurger developed a remarkable method, based on the generalization of the Siegel-Weil formula.
We shall now describe Waldspurger's generalization of the Siegel-Weil formula.
The Siegel-Weil formula for the simplest dual reductive pair $\Sp_{2n}$ and $\rO_{m}$ expresses the integral $\int_{\rO_{m}(k) \backslash \rO_{m}(\bA)} \vartheta_\psi^\phi(g, h) \dd g$
in terms of an Eisenstein series on $\Sp_{2n}$, when $m$ is sufficiently large compared to $n$.
Waldspurger's generalization of the Siegel-Weil formula considers the case when $m$ is small.
Let $T$ be an anisotropic form of $\SO_2$; thus, $T$ is isomorphic to the norm one elements of some quadratic extension $K$ of $k$.
Let $\chi$ be the id\'ele class character associated to $K$. 
Let $X$ be the 2-dimensional space on which $T$ acts, and let $Y$ be 2-dimensional with a symplectic form $\langle \,,\,\rangle$.
Put $Z = X \otimes_k Y$, $\langle x_1 \otimes y_1, x_2 \otimes y_2 \rangle = (x_1, x_2) \langle y_1, y_2 \rangle$.
As usual, $\SO_2 \times \SL_2 \hookrightarrow \Sp_4$.
For $h_v \in \SL_2(k_v)$, we have an Iwasawa decomposition $h_v = \left(\begin{smallmatrix}
    \alpha & * \\ 0 & \alpha^{-1}
\end{smallmatrix}\right) u$ with $u \in \SL_2(\cO_v)$ if $k_v$ is non-archimedean, and $u \in \SO_2(\bR)$ or $u \in \SU_2(\bC)$ in the archimedean case.
Define $A_v(h_v)$ to be $|\alpha|$, and if $h \in \SL_2(\bA)$ put
\[
A(h) = \prod_v A_v(h_v).
\]
We define Eisenstein series by
\[
E^\phi(h, s) = L\left(\chi, s + \frac{1}{2}\right) \sum_{\gamma \in B(k) \backslash \SL_2(k)} A(\gamma h)^{s - \frac{1}{2}}\omega_\psi(1, \gamma h)\phi(0)
\]
where $\phi$ is a Schwartz-Bruhat function on $X(\bA)$.
Using the standard theory of Eisenstein series, it is easy to show that this Eisenstein series converges absolutely in some half-plane and admits a meromorphic continuation to the entire plane.
We have the following Sieqel-Weil-Waldspurger identity
\[
E^\phi\left(h, \frac{1}{2}\right) = c \int_{T(k) \backslash T(\bA)} \vartheta_\psi^\phi(g, h) \dd g
\]
where $c$ is a constant depending only on $K/k$. This identity can be proved by Poisson summation.
We now return to the proof of Theorem \ref{thm:4.1}.
According to Theorem \ref{thm:3.3}, it is sufficient to prove for the dual reductive pair $\PGL_2, \widetilde{\SL}_2$ such that
\[
\zeta = \int_{\SL_2(k) \backslash \SL_2(\bA)} \varphi(h) \vartheta_\psi^\phi(g, h) \dd h
\]
is non-zero for some choice of $\phi, \varphi$, and $g$.
Suppose $\zeta \equiv 0$.
Since $\varphi \neq 0$, there is $a \in k^\times$ such that $\varphi_{\psi^a} \neq 0$.
If $a \in (k^\times)^2$, then since $\varphi_{\psi}$ and $\varphi_{\psi^{\lambda^2}}$ is related in an elementary fashion, our statement is true.
Thus, we may assume that $a \not\in (k^\times)^2$.
Let $x_a$ be an element of $X$ so that $q(x_a) = a$, and decompose $X$ into the line $(x_a) = kx_a$ generated by $x_a$ and the orthogonal complement $X_a'$.
We may take a $\phi$ of the form $\phi(\lambda x_a + x') = \phi_1(\lambda x_a) \phi_2(x')$ ($x' \in X_a'$).
For $g \in T = \stab (x_a)$, we have
\[
0 \equiv \zeta = \int_{\SL_2(k) \backslash \SL_2(\bA)} \varphi(h) \vartheta_\psi^{\phi_1}(h) \vartheta_{\psi}^{\phi_2}(g, h) \dd h.
\]
Let $K = k(\sqrt{a})$, and let $T$ be the anisotropic torus of the norm-one elements in $K^\times$.
We can integrate with respect to $g\in T(k) \backslash T(\bA)$.
Since $T(k) \backslash T(\bA)$ is compact, we can change the order of integration to obtain
\[
0 \equiv \int_{\SL_2(k) \backslash \SL_2(\bA)} \varphi(h) \vartheta_\psi^{\phi_1}(h) E^{\phi_2}\left(h, \frac{1}{2}\right) \dd h.
\]
Let
\[
\zeta(s) = \int_{\SL_2(k) \backslash \SL_2(\bA)} \varphi(h) \vartheta_\psi^{\phi_1}(h) E^{\phi_2}(h, s) \dd h.
\]
For $\Re(s)$ sufficiently large, the Eisenstein series converges absolutely, and hence we can write
\begin{align*}
    \zeta(s) &= \int_{B(k) \backslash \SL_2(\bA)} \varphi(h) \vartheta_\psi^{\phi_1}(h) L\left(\chi, s + \frac{1}{2}\right) A(h)^{s - \frac{1}{2}} \omega_\psi(1, h) \phi_2(0) \dd h \\
    &= \int_{N(k) \backslash \SL_2(\bA)} \varphi(h) \omega_\psi'(h) \phi_1(x_a) L\left(\chi, s + \frac{1}{2}\right)A(h)^{s - \frac{1}{2}} \omega_\psi(1, h) \phi_2(0) \dd h \\
    &= L\left(\chi, s + \frac{1}{2}\right) \int_{N(\bA) \backslash \SL_2(\bA)}\varphi_{\psi^{0}}(h) \omega_\psi'(h)\phi_1(x_a) \omega_\psi(1, h) \phi_2(0) A(h)^{s- \frac{1}{2}} \dd h \\
    &= L \left(\chi, s + \frac{1}{2}\right) \int_{N(\bA) \backslash \SL_2(\bA)} \varphi_{\psi^{0}}(h) \omega_\psi(1, h) \phi(x_a) A(h)^{s - \frac{1}{2}} \dd h.
\end{align*}
Each function in the integral factorizes as a product of local factors so
\[
\zeta(s) = L\left(\chi, s + \frac{1}{2}\right) \prod_v \int_{N(k_v) \backslash \SL_2(k_v)} \ell_v(\sigma(h_v) w) \omega_\psi(h_v) \phi_v(x_a) A(h_v)^{s - \frac{1}{2}} \dd h_v
\]
By \S \ref{sec:2}, we know that the local integral $\int_{N(k_v) \backslash \SL_2(k_v)} \ell_v(\sigma(h_v)w) \omega_\psi(h_v)\phi_v(x_a) A(h_v)^{s -\frac{1}{2}} \dd h_v$ does not vanish identically if and only if $\theta(\sigma_v, \psi_v^{-1}) \neq 0$, which in turn is equivalent to the existence of linear functional $\ell_v$, which transforms under $N(k_v)$ by $\psi_v^a$.
We have
\[
\frac{\zeta(s)}{L_\psi(\sigma, s)} = \prod_v R_v(s),
\]
where $R_v(s) \equiv 1$ for almost all $v$, and $R_v(\frac{1}{2}) \neq 0$ for all $v$.
Since $L_\psi(\sigma, \frac{1}{2}) \neq 0$, we obtain $0 \neq \zeta(\frac{1}{2}) = \zeta$, a contradiction.
Thus $\zeta \neq 0$.
Thus, there exists a $\phi = \prod_v \phi_v$ and a $w = \otimes_v w_v$ such that for all $v$, we have
\[
\int_{N(k_v) \backslash \SL_2(k_v)}  \ell_v(\sigma(h_v) w_v) \omega_{\psi_v}(h_v) \phi_v(x_a) \dd h_v \neq 0.
\]
\end{proof}

\begin{proof}[Proof of Theorem \ref{thm:1.3}]
Let $\sigma_1 = \otimes_v \sigma_{1, v}$, $\sigma_2 = \otimes_v \sigma_{2, v} \subset A_{00}$, and assume that they are nearly equivalent.
In this situation, $\pi_1 = W(\sigma, \psi)$ and $\pi_2 = W(\sigma_2, \psi)$ will have the same local components at almost all places.
By the strong multiplicity theorems for $\PGL_2$, it follows that $\pi_1 \simeq \pi_2$.
This means that $\sigma_{1, v} \simeq \sigma_{2, v}$ at all places $v$ for which $\sigma_{1, v}$ is not a discrete series representation.
Furthermore, at the places $v$ for which $\sigma_{1, v}$ is in the discrete series, it follows from Theorem \ref{thm:2.5} and the local Wa1dspurger involution, that either $\sigma_{2, v} = \sigma_{1, v}$ or $\sigma_{2, v} = \sigma_{1, v}^W$.
Since
\[
\sigma_{1, v}^W \begin{pmatrix}
    -1 & 0 \\ 0 & -1
\end{pmatrix} = -\sigma_{1, v}\begin{pmatrix}
    -1 & 0 \\ 0 & -1
\end{pmatrix},
\]
and $\left(\begin{smallmatrix}
    -1 & 0\\0&-1
\end{smallmatrix}\right) \in \SL_2(k)$, the number of places for which $\sigma_{2, v} = \sigma_{1, v}^{W}$ is even.
From this, we conclude that the representations in $A_{00}$ nearly equivalent to $\sigma_1$ must be of the form of $\sigma_1^M$ (see \S \ref{sec:1} for the definition of $\sigma_1^M$).
To complete the proof of Theorem \ref{thm:1.3}, we must show that every $\sigma^M$ (with $|M|$ even) lies in $A_{00}$.
To do this, Waldspurger used a result of Flicker \cite{flicker80covering} which we shall now describe.
Flicker established a correspondence between the representations of $\widetilde{\GL}_2$ and $\GL_2$ $(\rho \mapsto \pi)$.
A representation $\pi = \otimes_v \pi_v$ of $\GL_2$ lies in the image of the Flicker correspondence if and only if at each place $v$ for which $\pi_v$ is a principal series representation $\pi_v = \pi_v(\mu_v^1, \mu_v^2)$ with $\mu_v^1(-1) = \mu_v^2(-1) = 1$.
It is known \cite{waldspurger91quaternion,gps81shimura} that $\pi$ is in the Wa1dspurger correspondence if and only if there is an id\'ele character $\omega$ such that $\pi \otimes \omega$ is in the Flicker correspondence.
Waldspurger used this fact to prove that $\sigma^M$ is automorphic.
\end{proof}
\appendix

\section{Supplementary Figures}

\emph{This section is added by Seewoo Lee, which does not exist in the original document.} Here we give illustrative figures of the ``formulas'' used in the previous section, following more standard notations used in the Rubik's Cube community.

We denote the faces of the cubes as \rr{U}\rr{D}\rr{R}\rr{L}\rr{F}\rr{B}, which stands for Up, Down, Right, Left, Front, and Back. These faces correspond to the previous section's notations $abcdef$ as follows:

\begin{table}[h!]
    \center
    \begin{tabular}{c|c|c|c|c|c}
    \toprule
    $a$ & $b$ & $c$ & $d$ & $e$ & $f$ \\ \midrule
    \rr{U} & \rr{F} & \rr{L} & \rr{R} & \rr{B} & \rr{D} \\ \midrule
    \fcolorbox{black}{white}{\rule{0pt}{6pt}\rule{6pt}{0pt}} & \fcolorbox{black}{orange}{\rule{0pt}{6pt}\rule{6pt}{0pt}} & \fcolorbox{black}{blue}{\rule{0pt}{6pt}\rule{6pt}{0pt}} & \fcolorbox{black}{green}{\rule{0pt}{6pt}\rule{6pt}{0pt}} & \fcolorbox{black}{red}{\rule{0pt}{6pt}\rule{6pt}{0pt}} & \fcolorbox{black}{yellow}{\rule{0pt}{6pt}\rule{6pt}{0pt}} \\ 
    \bottomrule
    \end{tabular}
\end{table}

Also, the counter-clockwise rotations of these faces are denoted as \rr{Up}, \rr{Dp}, \rr{Rp}, \rr{Lp}, \rr{Fp}, and \rr{Bp} respectively.
Note that the action of $\Rub$ on the cube is \emph{left} action, hence you need to read the moves from right to left.
However, the usual notation in the Rubik's Cube community is \emph{right} action, and the sequence of moves is read from left to right.
For example, the element $abc$ with the previous notation corresponds to the move sequence \rr{L}\rr{F}\rr{U} in the new notation, which means that you first do \rr{L} ($c$), then \rr{F} ($b$), and finally \rr{U} ($a$).

\subsection{Edge permutation}

The first step is to put the edges in the correct positions. The previous formula $(a^2 b)^5$ is $($\rr{F}\rr{U}${}^{2})^{5}$ in the new notation, which means that we do the following 5 times:

\begin{figure}[hbt]
    \centering%
    \RubikCubeSolvedWY%
    \ShowCube{2cm}{0.4}{\DrawRubikCubeRU}%
    \Rubik{F}%
    \RubikRotation{F}%
    \ShowCube{2cm}{0.4}{\DrawRubikCubeRU}%
    \Rubik{U}\Rubik{U}%
    \RubikRotation{U2}%
    \ShowCube{2cm}{0.4}{\DrawRubikCubeRU}%
    \caption{\rr{F}\rr{U}${}^{2}$\ on a solved cube.}
    \label{fig:U2F}
\end{figure}


As described in the footnote 9, \rr{F}\rr{U}${}^{2}$ acts as a composition of 2-cycle (swapping white-blue and white-green edges) and 5-cycle on the edges, and doing this 5 times will only swap the white-blue and white-green edges, not flipping any edges of the cube.

\begin{figure}[hbt]
    \centering%
    \RubikCubeSolvedWY%
    \ShowCube{2cm}{0.4}{\DrawRubikCubeRU}%
    $\bigg(\Rubik{F}\Rubik{U}\Rubik{U}\bigg)^{5}$
    \RubikRotation{F}%
    \RubikRotation{U2}%
    \RubikRotation{F}%
    \RubikRotation{U2}%
    \RubikRotation{F}%
    \RubikRotation{U2}%
    \RubikRotation{F}%
    \RubikRotation{U2}%
    \RubikRotation{F}%
    \RubikRotation{U2}%
    \ShowCube{2cm}{0.4}{\DrawRubikCubeRU}%
    \caption{$($\rr{F}\rr{U}${}^{2})^{5}$ on a solved cube. White-green and white-blue edges are swapped without flip.}
    \label{fig:swap-edge1}
\end{figure}

Using conjugation, we can swap any two edges in the cube, by placing these two edges on the top-left and top-right positions by some move $g$, applying the above formula, and then returning the cube to its original position by $g^{-1}$.
For example, you can swap white-green and orange-green edges with $g=\text{\rr{F}}^{2}\text{\rr{Lp}}$ as in Figure \ref{fig:swap-edge2}.
Repeating this process, you can place all the edges in their correct positions.

\begin{figure}[hbt]
    \centering%
    \RubikCubeSolvedWY%
    \ShowCube{2cm}{0.4}{\DrawRubikCubeRU}%
    \Rubik{F}\Rubik{F}\Rubik{Lp}%
    \RubikRotation{F2}%
    \RubikRotation{Lp}%
    \ShowCube{2cm}{0.4}{\DrawRubikCubeRU}
    $\bigg($\Rubik{F}\Rubik{U}\Rubik{U}$\bigg)^{5}$%
    \RubikRotation{F}%
    \RubikRotation{U2}%
    \RubikRotation{F}%
    \RubikRotation{U2}%
    \RubikRotation{F}%
    \RubikRotation{U2}%
    \RubikRotation{F}%
    \RubikRotation{U2}%
    \RubikRotation{F}%
    \RubikRotation{U2}%
    \ShowCube{2cm}{0.4}{\DrawRubikCubeRU}%
    \Rubik{L}\Rubik{F}\Rubik{F}%
    \RubikRotation{L}%
    \RubikRotation{F}%
    \RubikRotation{F}%
    \ShowCube{2cm}{0.4}{\DrawRubikCubeRU}%
    \caption{$g($\rr{F}\rr{U}${}^{2})^{5}g^{-1}$ with $g = \text{\rr{F}}^{2}\text{\rr{Lp}}$ on a solved cube. White-green and orange-green edges are swapped without flip.}
    \label{fig:swap-edge2}
\end{figure}


\subsection{Edge orientation}

Now, one need to flip the edges correctly.
As mentioned in the previous section, the formula $d^2 f b d^{-1}$, which is $\text{\rr{Rp}\rr{F}\rr{D}\rr{R}}{}^{2}$ in the new notation, flips the white-green edge $y_{\text{\rr{U}\rr{R}}}$ while fixing the white-blue edge $y_{\text{\rr{U}\rr{L}}}$.
Hence
\[
h = (\text{\rr{Rp}\rr{F}\rr{D}\rr{R}}^{2}) (\text{\rr{F}\rr{U}}^{2})^{5} (\text{\rr{Rp}\rr{F}\rr{D}\rr{R}}^{2})^{-1}(\text{\rr{F}\rr{U}}^{2})^{5}
\]
flips $y_{\text{\rr{U}\rr{R}}}$ and $y_{\text{\rr{U}\rr{L}}}$, leaving the other edges unchanged (Figure \ref{fig:flip-edge}).

\begin{figure}[hbt]
    \centering%
    \RubikCubeSolvedWY%
    % \ShowCube{2cm}{0.4}{\DrawRubikCubeSF}%
    \RubikRotation{Rp,F,D,R2}%
    \RubikRotation{F,U2,F,U2,F,U2,F,U2,F,U2}%
    \RubikRotation{R2,Dp,Fp,R}%
    \RubikRotation{F,U2,F,U2,F,U2,F,U2,F,U2}%
    \ShowCube{2cm}{0.4}{\DrawRubikCubeSF}%
    \caption{$h = (\text{\rr{Rp}\rr{F}\rr{D}\rr{R}}^{2}) (\text{\rr{F}\rr{U}}^{2})^{5} (\text{\rr{Rp}\rr{F}\rr{D}\rr{R}}^{2})^{-1}(\text{\rr{F}\rr{U}}^{2})^{5}$ on a solved cube. White-green and white-blue edges are flipped.}
    \label{fig:flip-edge}
\end{figure}

Again, using conjugation, you can flip any two edges in the cube.

\subsection{Corner permutation}

For the corner permutation, we use $(\text{\rr{U}\rr{F}\rr{Up}\rr{Fp}})^{3}$ which swaps two pairs of corners, $x_{\text{\rr{U}\rr{F}\rr{L}}} \leftrightarrow x_{\text{\rr{D}\rr{L}\rr{F}}}$ and $x_{\text{\rr{U}\rr{R}\rr{F}}} \leftrightarrow x_{\text{\rr{R}\rr{U}\rr{B}}}$, while leaving the other edges and corners unchanged (Figure \ref{fig:swap-corners}).
Lemma \ref{lem:AltX} uses this move to show that any element of $\Alt_\bfX$ can be reached with the legal moves.

\begin{figure}[hbt]
    \centering%
    \RubikCubeSolvedWY%
    \ShowCube{2cm}{0.4}{\DrawRubikCubeRU}%
    $\bigg(\Rubik{U}\Rubik{F}\Rubik{Up}\Rubik{Fp}\bigg)^{3}$
    \RubikRotation{U,F,Up,Fp,U,F,Up,Fp,U,F,Up,Fp}%
    \ShowCube{2cm}{0.4}{\DrawRubikCubeRU}%
    \caption{$(\text{\rr{U}\rr{F}\rr{Up}\rr{Fp}})^{3}$ on a solved cube. White-green and white-blue edges are flipped.}
    \label{fig:swap-corners}
\end{figure}

\subsection{Corner orientation}

Finally, we need to orient the corners.
The element $ede^{-1}d^{-1}e$, which is $\text{\rr{B}\rr{Rp}\rr{Bp}\rr{R}\rr{B}}$ fixes three corners $x_{\text{\rr{U}\rr{F}\rr{L}}}$, $x_{\text{\rr{D}\rr{L}\rr{F}}}$, and $x_{\text{\rr{U}\rr{R}\rr{F}}}$, while rotating the corner $x_{\text{\rr{U}\rr{R}\rr{B}}}$ (Figure \ref{fig:orient-corner}).
Hence the element
\[
    (\text{\rr{B}\rr{Rp}\rr{Bp}\rr{R}\rr{B}})^{-1} (\text{\rr{U}\rr{F}\rr{Up}\rr{Fp}})^{3} (\text{\rr{B}\rr{Rp}\rr{Bp}\rr{R}\rr{B}}) (\text{\rr{U}\rr{F}\rr{Up}\rr{Fp}})^{3}
\]
rotates exactly two corners $x_{\text{\rr{U}\rr{R}\rr{F}}}$ and $x_{\text{\rr{R}\rr{U}\rr{B}}}$ in opposite directions, fixing other pieces (Figure \ref{fig:orient-corner2}).

\begin{figure}[hbt]
    \centering%
    \RubikCubeSolvedWY%
    \RubikRotation{B,Rp,Bp,R,B}%
    \ShowCube{2cm}{0.4}{\DrawRubikCubeSF}%
    \caption{$\text{\rr{B}\rr{Rp}\rr{Bp}\rr{R}\rr{B}}$ on a solved cube. White-green and white-blue edges are flipped.}
    \label{fig:orient-corner}
\end{figure}

\begin{figure}[hbt]
    \centering%
    \RubikCubeSolvedWY%
    % \ShowCube{2cm}{0.4}{\DrawRubikCubeRU}%
    \RubikRotation{Bp,Rp,B,R,Bp}%
    \RubikRotation{U,F,Up,Fp,U,F,Up,Fp,U,F,Up,Fp}%
    \RubikRotation{B,Rp,Bp,R,B}%
    \RubikRotation{U,F,Up,Fp,U,F,Up,Fp,U,F,Up,Fp}%
    \ShowCube{2cm}{0.4}{\DrawRubikCubeSF}%
    \caption{$(\text{\rr{B}\rr{Rp}\rr{Bp}\rr{R}\rr{B}})^{-1} (\text{\rr{U}\rr{F}\rr{Up}\rr{Fp}})^{3} (\text{\rr{B}\rr{Rp}\rr{Bp}\rr{R}\rr{B}}) (\text{\rr{U}\rr{F}\rr{Up}\rr{Fp}})^{3}$ on a solved cube. Two corners $x_{\text{\rr{U}\rr{R}\rr{F}}}$ and $x_{\text{\rr{R}\rr{U}\rr{B}}}$ are rotated, whilte other pieces are unchanged.}
    \label{fig:orient-corner2}
\end{figure}

\subsection{Putting it all together}

Following the above steps, we can solve the Rubik's Cube in the following order:
\begin{enumerate}
    \item Place the edges in their correct positions using $(\text{\rr{F}\rr{U}}^{2})^{5}$ and its conjugates.
    \item Flip the edges correctly using $h = (\text{\rr{Rp}\rr{F}\rr{D}\rr{R}}^{2}) (\text{\rr{F}\rr{U}}^{2})^{5} (\text{\rr{Rp}\rr{F}\rr{D}\rr{R}}^{2})^{-1}(\text{\rr{F}\rr{U}}^{2})^{5}$ and its conjugates.
    \item Place the corners in their correct positions using $(\text{\rr{U}\rr{F}\rr{Up}\rr{Fp}})^{3}$ and its conjugates.
    \item Rotate the corners correctly using $(\text{\rr{B}\rr{Rp}\rr{Bp}\rr{R}\rr{B}})^{-1} (\text{\rr{U}\rr{F}\rr{Up}\rr{Fp}})^{3} (\text{\rr{B}\rr{Rp}\rr{Bp}\rr{R}\rr{B}}) (\text{\rr{U}\rr{F}\rr{Up}\rr{Fp}})^{3}$ and its conjugates.
\end{enumerate}

% --- Bibliography ---

% Start a bibliography with one item.
% Citation example: "\cite{williams}".

\bibliographystyle{acm} % We choose the "plain" reference style
\bibliography{refs} % Entries are in the refs.bib file


% \begin{thebibliography}{1}

% \bibitem{williams}
%    Williams, David.
%    \textit{Probability with Martingales}.
%    Cambridge University Press, 1991.
%    Print.

% % Uncomment the following lines to include a webpage
% % \bibitem{webpage1}
% %   LastName, FirstName. ``Webpage Title''.
% %   WebsiteName, OrganizationName.
% %   Online; accessed Month Date, Year.\\
% %   \texttt{www.URLhere.com}

% \end{thebibliography}

% --- Document ends here ---

\end{document}