\section{The Rubik's Group}

Let $\bfE$ denote the set of all possible states of the cube. This set is the product of the set $\bfE_\bfX$ of corner states and the set $\bfE_\bfY$ of edge states. Since there are 8 corners that can be permuted freely, and each corner, once its position is chosen, can be placed in 3 different orientations (the outer faces must be visible), we have $|\bfE_\bfX| = 8! \times 3^8$.
Similarly, the 12 edges can be freely permuted, and each can be flipped once its position is fixed; therefore, $|\bfE_\bfY| = 12! \times 2^{12}$, and hence $|\bfE| = 12! \times 8! \times 3^8 \times 2^{12} = 2^{29} \times 3^{15} \times 5^{3} \times 7^{2} \times 11$.

Now, there is a group $\bfG$ that acts naturally on $\bfE$; it is the group of all Rubik's Cube configurations (i.e., all possible scramblings), described more explicitly below (we allow ourselves to disassemble the Rubik's Cube and reassemble it, with the colored faces on the outside). There is a natural bijection between $\bfG$ and $\bfE$, consisting of letting an element $g \in \bfG$ act on the initial state of the Rubik's Cube\footnote{In fact, we could have started from any state $e$, and obtained a bijection $g \mapsto g \cdot e$ from $G$ to $E$; in summary, one can go from any state of the cube to any other by letting $G$ act, and this through the action of a unique element of $G$. We say that $E$ is a principal homogeneous space under the action of $G$.
A similar situation occurs when $E$ is an affine space and $G$ is the associated vector space: the choice of an origin $O$ in $E$ defines a bijection $v \mapsto O + v$ from $G$ to $E$, and one can go from any point in $E$ to any other point by translating by a vector from $G$, and in a unique way.
Likewise, the set of bases of a vector space of dimension $n$ over a field $K$ is a principal homogeneous space under the action of the group $\mathrm{GL}_n(K)$.}. However, it is important to distinguish\footnote{This amounts to distinguishing between the pieces that make up the cube and their positions: the group of moves acts on the positions, and an element $g \in G$ sends the piece $x$ located at position $p$ to position $g(p)$, independently of the initial position of $x$ in the cube's initial state.} between $\bfG$ and $\bfE$ in order to understand in what sense the Rubik's Cube forms a group.

Let $\Rub$ denote the Rubik's group, which is the subgroup of $\bfG$ generated by the 6 rotations of the layers (thus, it is the subgroup of scramblings that can be achieved without taking the cube apart). The statement we aim to prove can then be expressed as the following, which is a purely group-theoretic result.

\begin{theorem}
    \label{thm:index}
    The index of the subgroup $\Rub$ of $\bfG$ is 12.
\end{theorem}

This result is a consequence of a more precise description (see Theorem 5) of $\Rub$ as a subgroup of $\bfG$. Since the size of $\bfG$ is known, we can deduce that of $\Rub$, which is nothing other than the number of cube states that can be reached through a sequence of legal moves (given the size of this number, it is difficult to hope to solve the Rubik's Cube by relying on pure chance).

\begin{corollary}
    $|\Rub| = \frac{1}{12} \cdot 12! \cdot 8! \cdot 2^{12} \cdot 3^8 = 43\,252\,003\,274\,489\,856\,000$.
\end{corollary}
