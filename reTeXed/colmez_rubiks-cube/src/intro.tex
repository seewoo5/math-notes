\section*{Introduction}

The Rubik's cube is made up of 27 small cubes, of which 7 are fixed (the central cube and those at the center of the faces), and 20 are movable (the 8 corners and the 12 edges; we denote by $\bfX$ and $\bfY$ the sets of corners and edges, respectively).
An ingenious mechanism allows each of the outer layers to rotate, thereby scrambling the movable cubes; this is physically visible since the outer faces of the movable cubes are colored (an outer face remains on the outside while rotating the layers). Solving the Rubik's Cube means returning it to the \emph{initial} state, where each face is a single color.
We will explain why, if you disassemble a Rubik's Cube and reassemble it randomly, you have a $\frac{1}{12}$ chance of being able to solve it. This will require transforming the Rubik's Cube into a group\footnote{It is one of the rare groups you can walk down the street with; you can do the same with Artin's braid group, but it tends to get tangled easily.}.
