\section{The Rubik's Group as a Subgroup of the Scrambling Group}

By combining the three group homomorphisms defined above, we obtain a group homomorphism:
$$
\rt : \bfG \to (\bZ / 3 \bZ) \times (\bZ / 2 \bZ) \times \{\pm 1\}, \quad\text{with } \rt(g) = (\rt_\bfX \circ \pi_\bfX(g), \rt_\bfY \circ \pi_\bfY(g), \varepsilon(g)).
$$
This morphism is clearly surjective; its kernel $\bfH$ therefore has index 12 in G, and Theorem \ref{thm:index} is thus a consequence of the following result:

\begin{theorem}
    \label{thm:RubH}
    We have $\Rub = \bfH$. In other words, an element $g$ of $G$ belongs to the Rubik's group $\Rub$ if and only if:
    $\pi_\bfX(g)$ and $\pi_\bfY(g)$ have total rotation zero, and $g$ induces an even permutation on the cube's positions.
\end{theorem}

The proof of this result consists of two parts:
the first (Proposition \ref{prop:RubSubH}), rather pleasant, is to verify that every element of $\Rub$ satisfies the conditions above,
and the second (Proposition \ref{prop:HSubRub}), a bit more tedious, requires showing that every element of $\bfG$ satisfying the theorem's conditions can be written as a product of layer rotations of the cube; this amounts to describing a solution algorithm\footnote{The resulting algorithm is not very efficient: it has been verified, with the help of a computer, that it is always possible to solve the Rubik's Cube under 25 rotations. Its value is more theoretical; it serves to illustrate the effect of conjugation on the action of a group on a set.} for the Rubik's Cube.

\begin{proposition}
    \label{prop:RubSubH}
    The group $\Rub$ is a subgroup of $\bfH$.
\end{proposition}
\begin{proof}
    Since $\bfH$ is the intersection of the kernels of
    $\rt_\bfX \circ \pi_\bfX$, $\rt_\bfY \circ \pi_\bfY$, and $\varepsilon$,
    and since $\Rub$ is generated by the layer rotations, it suffices to prove that these layer rotations belong to each of those kernels.
    Let $g$ be a layer rotation.

    \begin{itemize}
        \item From Proposition 4, the kernel of $\rt_\bfY$ is also the set of elements of $\bfG_\bfY$ that induce a permutation of signature 1 on the set $F$ of edge faces.
        But $g$ induces a product of two 4-cycles on these 24 faces, so its signature is 1.
        We deduce that $g$ belongs to the kernel of $\rt_\bfY \circ \pi_\bfY$.
        \item We may define the distinguished faces as those on the top and bottom of the cube;
        then horizontal layer rotations contribute zero rotation at each corner, so the total corner rotation is zero.
        If a vertical slice is rotated, the four corners not on that slice have zero rotation,
        and the four involved corners have rotations of 1, 2, 1, and 2, whose sum is indeed 0 in $\mathbb{Z}/3\mathbb{Z}$.
        Thus, in all cases, $g$ belongs to the kernel of $\rt_\bfX \circ \pi_\bfX$.
        \item $g$ induces a 4-cycle on the corners and a 4-cycle on the edges; hence $\varepsilon(g) = 1$, which shows that $g$ lies in the kernel of $\varepsilon$.
    \end{itemize}
    This completes the proof of $\Rub \subseteq \bfH$.
\end{proof}
