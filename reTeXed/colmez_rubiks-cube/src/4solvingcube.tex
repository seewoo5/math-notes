\section{Solving Rubik's Cube}

The algorithm described below\footnote{It is easier to follow with a Rubik's Cube in hand, but with a bit of determination, paper and pencil can suffice (though it's a pity to miss out on the existence of a physical version of the Rubik's group).} consists of:

\begin{itemize}
    \item Placing the edges in their correct positions.
    \item Flipping them two at a time to orient them correctly.
    \item Placing the corners in their correct positions without disturbing the edges.
    \item Rotating the corners two at a time to orient them correctly.
\end{itemize}
With some thought, the first two steps and the last two can be combined.

\begin{itemize}
    \item \emph{Notations.} We denote the faces of the Rubik's Cube by $a$, $b$, $c$, $d$, $e$, and $f$.
    If $r$ is a face, we also denote by $r$ the quarter-turn rotation of the cube layer corresponding to face $r$ (clockwise, with the axis oriented from the center of the cube toward the center of face $r$).
    By definition, $\Rub$ is the subgroup of $\bfG$ generated by $a$, $b$, $c$, $d$, $e$, $f$, and if $r$ is a face, then $r^{-1}$ is the quarter-turn rotation of the corresponding layer of the cube in the counterclockwise direction (i.e., a quarter-turn opposite to $r$).

    If $r$ and $s$ are two faces that share an edge, we denote this edge by $y_{rs}$ (or $y_{sr}$), and if $r$, $s$, $t$ are three faces that share a corner, we denote this corner by $x_{rst}$ (or $x_{srt}$, etc.).

    We index the faces so that $(a, f)$, $(b, e)$, and $(c, d)$ form pairs of opposite faces, and such that $a$ sends $y_{ab} \to y_{ac}$, then $y_{ac} \to y_{ae}$, $y_{ae} \to y_{ad}$, and $y_{ad} \to y_{ab}$.
    The 8 corners are then:
    $x_{abc}$, $x_{ace}$, $x_{aed}$, $x_{adb}$, $x_{fcb}$, $x_{fec}$, $x_{fde}$, and $x_{fbd}$.

    \item \emph{Edge permutation.}
    The permutation of edges uses the element $(a^2b)^5$ from $\Rub$ and its conjugates.
    This element has the property of swapping $y_{ac}$ and $y_{ad}$ by rotating face $a$, while leaving all other edges fixed\footnote{The move $a^2 b$ moves 7 edges, forming one cycle of length 5 and one of length 2; its 5th power therefore eliminates the 5-cycle, but it is somewhat miraculous that it does not flip any element of that cycle.}.
    In particular, its image in $\Perm_\bfY$ via $\sigma_\bfY \circ \pi_\bfY$ is the transposition of edges $y_{ac}$ and $y_{ad}$.

    Moreover, it is easy to verify that if $y$ and $y'$ are any two distinct elements of $Y$, then there exists some $g \in \Rub$ sending $y_{ac} \to y$ and $y_{ad} \to y'$.
    Then the image of $g(a^2b)^5g^{-1}$ under $\sigma_\bfY \circ \pi_\bfY$ is the transposition swapping $y$ and $y'$.
    It follows that $\sigma_\bfY \circ \pi_\bfY(\Rub)$ contains all transpositions, and since these generate $\Perm_\bfY$, this proves the following result.

    \begin{lemma}
        \label{lem:comp}
        The composition $\sigma_\bfY \circ \pi_\bfY$ induces a surjection from $\Rub$ onto $\Perm_\bfY$.
    \end{lemma}

    \item \emph{Edge orientation.} The move $d^2fbd^{-1}$ flips $y_{ad}$ and leaves $y_{ac}$ fixed; thus the element
        $$
        h = (a^2b)^5 (d^2fbd^{-1})^{-1} (a^2b)^5 (d^2fbd^{-1})
        $$
        flips both $y_{ac}$ and $y_{ad}$ without affecting the other edges.
        If $y$ and $y'$ are two distinct elements of $Y$, and if $g \in \Rub$ sends $y_{ac} \to y$ and $y_{ad} \to y'$, then $ghg^{-1}$ flips $y$ and $y'$ without affecting the other edges.
        It follows that $\pi_\bfY(\Rub \cap \ker(\sigma_\bfY \circ \pi_\bfY))$ contains the flips of any two edges.
        Since such elements generate the subgroup $\Rot_\bfY^0$ of $\Rot_\bfY$ consisting of total rotation zero (i.e., even numbers of edge flips), this proves the following:

    \begin{lemma}
        \label{lem:RotY0}
        $\pi_\bfY$ induces a surjection from $\Rub \cap \ker(\sigma_\bfY \circ \pi_\bfY)$ onto $\Rot_\bfY^0$.
    \end{lemma}

    \item \emph{Corner permutation.}
    Permuting the corners uses the element $(b^{-1}a^{-1}ba)^3$ from $\Rub$.
    This element fixes the edges - hence belongs to $\Rub \cap \bfG_\bfX$ - and it swaps the corners $x_{abc}$ and $x_{fcb}$ (by swapping faces $a$ and $f$), and also $x_{adb}$ and $x_{dae}$ (by swapping faces $b$ and $e$), while leaving the others fixed.
    In particular, its image in $\Perm_\bfX$ is a product of two transpositions with disjoint support.

    \begin{lemma}
        \label{lem:trans_quadruple}
        If $x_1, x_2, x_3, x_4$ and $x_1', x_2', x_3', x_4'$ are two families of four distinct elements of $\bfX$, then there exists $g \in \Rub$ such that $\pi_\bfX(g) \cdot x_i = x_i'$ for\footnote{We say that $\Rub$ acts 4-transitively on $\bfX$, or that the action of $\Rub$ on $\bfX$ is 4-transitive.} $i = 1, 2, 3, 4$.
    \end{lemma}
    \begin{proof}
        It suffices to prove that one can map any family to a fixed family, for example: $x_{abc}, x_{fcb}, x_{adb}, x_{dae}$; indeed, if $g \cdot x_1 = x_{abc}$, $g \cdot x_2 = x_{fcb}$, $g \cdot x_3 = x_{adb}$, and $g \cdot x_4 = x_{dae}$, and likewise
        $g' \cdot x_1' = x_{abc}$, $g' \cdot x_2' = x_{fcb}$, $g' \cdot x_3' = x_{adb}$, and $g' \cdot x_4' = x_{dae}$, then
        $((g')^{-1}g) \cdot x_i = x_i'$ for $i = 1, 2, 3, 4$.

        It is very easy to move any two corners to $x_{abc}$ and $x_{fcb}$, and since $d$ and $e$ fix $x_{abc}$ and $x_{fcb}$, it is sufficient to prove that if $x \neq x'$ are two distinct corners different from $x_{abc}$ and $x_{fcb}$, there exists an element $g$ in the subgroup $\bfG_{d,e}$ of $\Rub$, generated by $d$ and $e$, such that
        $g \cdot x = x_{adb}$ and $g \cdot x' = x_{dae}$.

        Now, there exists $h \in \bfG_{d,e}$ such that $h \cdot x = x_{adb}$, and there are three cases:
        \begin{itemize}
            \item $h \cdot x' = x_{ade}$: take $g = h$.
            \item $h \cdot x' = x_{bdf}$: take $g = d^{-1}h$.
            \item $h \cdot x'$ is not on face $b$: then there exists $k$ such that $e^k \cdot (h \cdot x') = x_{ade}$, and take $g = e^k h$.
        \end{itemize}
        This completes the proof.
    \end{proof}

    \begin{lemma}
        \label{lem:AltX}
        The image of $\Rub \cap \bfG_\bfX$ in $\Perm_\bfX$ is the subgroup $\Alt_\bfX$ of permutations of signature 1 (i.e., the alternating group on $\bfX$).
    \end{lemma}

    \begin{proof}
        The image is contained in $\Alt_\bfX$ because $\Rub$ is included in the kernel of $\varepsilon$, and an element of $\bfG_\bfX$ acts trivially on $\bfY$.
        Moreover, the properties of $(b^{-1}a^{-1}ba)^3$ show that its image in $\Perm_\bfX$ contains a product of two disjoint transpositions $(x_1, x_2)(x_3, x_4)$.
        Now, for any $g \in \Rub$, the element $g(b^{-1}a^{-1}ba)^3g^{-1}$ belongs to $\Rub \cap \bfG_\bfX$, and its image in $\Perm_\bfX$ is
        $(g\cdot x_1, g\cdot x_2)(g\cdot x_3, g\cdot x_4)$.
        Using the previous lemma, we deduce that the image contains all products of two disjoint transpositions.
        Since $|\bfX| > 5$, such products generate $\Alt_\bfX$, which completes the proof.
    \end{proof}

    \item \emph{Corner orientation.}
    Let $\Rot_\bfX^0$ be the subgroup of $\Rot_\bfX$ consisting of elements of total rotation zero (i.e., the kernel of $\rt_\bfX$).
    We also have $\Rot_\bfX^0 = \bfH \cap \Rot_\bfX$, since an element of $\Rot_\bfX$ is already in the kernels of $\rt_\bfY \circ \pi_\bfY$ and $\varepsilon$.

    \begin{lemma}
        \label{lem:RotX0}
        We have $\Rot_\bfX^0 \subset \Rub$.
    \end{lemma}
    \begin{proof}
        Observe that the element $ede^{-1}d^{-1}e$ fixes $x_{abc}$, $x_{fcb}$, and $x_{adb}$, and rotates $x_{aed}$ by one third of a turn.
        Now, the element $(b^{-1}a^{-1}ba)^{3}$ belongs to $\Rub \cap \bfG_{x}$ and swaps the corners $x_{abc}$ and $x_{fcb}$, as well as $x_{adb}$ and $x_{dae}$, while fixing the others.
        It follows that
        $$
        (b^{-1}a^{-1}ba)^{3} (ede^{-1}d^{-1}e) (b^{-1}a^{-1}ba)^{3} (ede^{-1}d^{-1}e)^{-1}
        $$
        is an element of $\Rub \cap \ker(\pi_{\bfY})$ that fixes all corners except $x_{dae}$ and $x_{adb}$, each of which it rotates by one-third of a turn (in opposite directions, since the total rotation is zero).
        In other words, letting $x_{1} = x_{adb}$ and $x_{2} = x_{dae}$, this element is $(n_{x})_{x \in \bfX}$ in $\Rot_\bfX^{0}$, with: $n_{x} = 0$ if $x \notin \{x_{1}, x_{2}\}$, and $n_{x_{1}} + n_{x_{2}} = 0$, and $n_{x_{1}} \neq 0$.
        Since $\Rub$ acts 4-transitively on $\bfX$ (and thus in particular 2-transitively), and since
        $ghg^{-1} = (n'_{x})_{x \in \bfX}$ with $n'_{x} = n_{g \cdot x}$, if $h = (n_{x})_{x \in \bfX}$,
        it follows that $\Rub \cap \Rot_\bfX^{0}$ contains all elements of this type.
        Because these generate $\Rot_\bfX^{0}$, we get $\Rub \cap \Rot_\bfX^{0} = \Rot_\bfX^{0}$. This completes the proof.
    \end{proof}

    \item \emph{The inclusion $\bfH \subseteq \Rub$.}
    We can now prove the following result, which completes the proof of Theorem \ref{thm:RubH}.
    \begin{proposition}
        \label{prop:HSubRub}
        We have $\bfH \subseteq \Rub$.
    \end{proposition}
    \begin{proof}
        Let us begin by noting that since $\Rub \subseteq \bfH$, the product of an element of $\Rub$ and an element of $\bfH$ is still an element of $\bfH$. Let $h \in \bfH$.
        \begin{itemize}
            \item Since $\sigma_{\bfY} \circ \pi_{\bfY}$ induces (cf. Lemma \ref{lem:comp}) a surjection from $\Rub$ onto $\Perm_\bfY$, there exists $g_{1} \in \Rub$ such that $\sigma_{\bfY} \circ \pi_{\bfY}(g_{1}) = \sigma_{\bfY} \circ \pi_{\bfY}(h)$, and then $h_{1} = g_{1}^{-1} h$ is an element of $\bfH$ that lies in the kernel of $\sigma_{\bfY} \circ \pi_{\bfY}$.
            \item By Lemma \ref{lem:RotY0}, there exists $g_{2} \in \Rub$ such that $\pi_{\bfY}(g_{2}) = \pi_{\bfY}(h_{1})$, and so $h_{2} = g_{2}^{-1} h_{1}$ is an element of $\bfH$ that belongs to $\bfG_{\bfX}$.
            \item We have $\varepsilon(h_{2}) = 1$, and since $h_{2}$ acts as the identity on $\bfY$, the permutation $\sigma_{\bfX}(h_{2})$ belongs to $\Alt_\bfX$.
            By Lemma \ref{lem:AltX}, this implies there exists $g_{3} \in \Rub \cap \bfG_{\bfX}$ such that $\sigma_{\bfX}(g_{3}) = \sigma_{\bfX}(h_{2})$, and then $g_{4} = g_{3}^{-1} h_{2}$ is an element of $\bfH \cap \Rot_{\bfX}$.
            Now, $\bfH \cap \Rot_{\bfX} = \Rot_{\bfX}^{0}$, which is included in $\Rub$ by Lemma \ref{lem:RotX0}; thus $g_{4} \in \Rub$.
            \item Since $h = g_{1} g_{2} g_{3} g_{4}$, it follows that $h \in \Rub$, which completes the proof.
        \end{itemize}
    \end{proof}
\end{itemize}




