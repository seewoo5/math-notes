% --- LaTeX Lecture Notes Template - S. Venkatraman ---

% --- Set document class and font size ---

\documentclass[letterpaper, 12pt]{article}

% --- Package imports ---

% Extended set of colors
\usepackage[dvipsnames]{xcolor}

\usepackage{
  amsmath, amsthm, amssymb, mathtools, dsfont, units,          % Math typesetting
  graphicx, wrapfig, subfig, float,                            % Figures and graphics formatting
  listings, color, inconsolata, pythonhighlight,               % Code formatting
  fancyhdr, sectsty, hyperref, enumerate, enumitem, framed }   % Headers/footers, section fonts, links, lists

% lipsum is just for generating placeholder text and can be removed
\usepackage{hyperref, lipsum} 

% --- Fonts ---

\usepackage{newpxtext, newpxmath, inconsolata}
\usepackage{amsfonts}

\usepackage{tikz}
\usepackage{tikz-cd}
\usepackage{enumitem}
\usepackage[title]{appendix}
\usepackage{mathdots}
\usepackage{stmaryrd}

% --- Page layout settings ---

% Set page margins
\usepackage[left=1.35in, right=1.35in, top=1.0in, bottom=.9in, headsep=.2in, footskip=0.35in]{geometry}

% Anchor footnotes to the bottom of the page
\usepackage[bottom]{footmisc}

% Set line spacing
\renewcommand{\baselinestretch}{1.2}

% Set spacing between paragraphs
\setlength{\parskip}{1.3mm}

% Allow multi-line equations to break onto the next page
\allowdisplaybreaks

% --- Page formatting settings ---

% Set image captions to be italicized
\usepackage[font={it,footnotesize}]{caption}

% Set link colors for labeled items (blue), citations (red), URLs (orange)
\hypersetup{colorlinks=true, linkcolor=RoyalBlue, citecolor=RedOrange, urlcolor=ForestGreen}

% Set font size for section titles (\large) and subtitles (\normalsize) 
\usepackage{titlesec}
% \titleformat{\section}{\large\bfseries}{{\fontsize{19}{19}\selectfont\textreferencemark}\;\; }{0em}{}
\titleformat{\section}{\large\bfseries}{\thesection\;\;\;}{0em}{}
\titleformat{\subsection}{\normalsize\bfseries\selectfont}{\thesubsection\;\;\;}{0em}{}

% Enumerated/bulleted lists: make numbers/bullets flush left
%\setlist[enumerate]{wide=2pt, leftmargin=16pt, labelwidth=0pt}
\setlist[itemize]{wide=0pt, leftmargin=16pt, labelwidth=10pt, align=left}

% --- Table of contents settings ---

\usepackage[subfigure]{tocloft}

% Reduce spacing between sections in table of contents
\setlength{\cftbeforesecskip}{.9ex}

% Remove indentation for sections
\cftsetindents{section}{0em}{0em}

% Set font size (\large) for table of contents title
\renewcommand{\cfttoctitlefont}{\large\bfseries}

% Remove numbers/bullets from section titles in table of contents
\makeatletter
\renewcommand{\cftsecpresnum}{\begin{lrbox}{\@tempboxa}}
\renewcommand{\cftsecaftersnum}{\end{lrbox}}
\makeatother

% --- Set path for images ---

\graphicspath{{Images/}{../Images/}}

% --- Math/Statistics commands ---

% Add a reference number to a single line of a multi-line equation
% Usage: "\numberthis\label{labelNameHere}" in an align or gather environment
\newcommand\numberthis{\addtocounter{equation}{1}\tag{\theequation}}

% Shortcut for bold text in math mode, e.g. $\b{X}$
\let\b\mathbf

% Shortcut for bold Greek letters, e.g. $\bg{\beta}$
\let\bg\boldsymbol

% Shortcut for calligraphic script, e.g. %\mc{M}$
\let\mc\mathcal

% \mathscr{(letter here)} is sometimes used to denote vector spaces
\usepackage[mathscr]{euscript}

% Convergence: right arrow with optional text on top
% E.g. $\converge[p]$ for converges in probability
\newcommand{\converge}[1][]{\xrightarrow{#1}}

% Weak convergence: harpoon symbol with optional text on top
% E.g. $\wconverge[n\to\infty]$
\newcommand{\wconverge}[1][]{\stackrel{#1}{\rightharpoonup}}

% Equality: equals sign with optional text on top
% E.g. $X \equals[d] Y$ for equality in distribution
\newcommand{\equals}[1][]{\stackrel{\smash{#1}}{=}}

% Normal distribution: arguments are the mean and variance
% E.g. $\normal{\mu}{\sigma}$
\newcommand{\normal}[2]{\mathcal{N}\left(#1,#2\right)}

% Uniform distribution: arguments are the left and right endpoints
% E.g. $\unif{0}{1}$
\newcommand{\unif}[2]{\text{Uniform}(#1,#2)}

% Independent and identically distributed random variables
% E.g. $ X_1,...,X_n \iid \normal{0}{1}$
\newcommand{\iid}{\stackrel{\smash{\text{iid}}}{\sim}}

% Sequences (this shortcut is mostly to reduce finger strain for small hands)
% E.g. to write $\{A_n\}_{n\geq 1}$, do $\bk{A_n}{n\geq 1}$
\newcommand{\bk}[2]{\{#1\}_{#2}}

% \setcounter{section}{-1}

\newcommand{\SL}{\mathrm{SL}}
\newcommand{\Sp}{\mathrm{Sp}}
\newcommand{\Mp}{\mathrm{Mp}}
\newcommand{\GL}{\mathrm{GL}}
\newcommand{\SO}{\mathrm{SO}}
\newcommand{\SU}{\mathrm{SU}}
\newcommand{\PGL}{\mathrm{PGL}}
\newcommand{\PSL}{\mathrm{PSL}}
\newcommand{\rM}{\mathrm{M}}
\newcommand{\rN}{\mathrm{N}}
\newcommand{\rO}{\mathrm{O}}
\newcommand{\rP}{\mathrm{P}}
\newcommand{\rH}{\mathrm{H}}
\newcommand{\rU}{\mathrm{U}}
\newcommand{\JL}{\mathrm{JL}}
\newcommand{\stab}{\mathrm{Stab}}
\newcommand{\cusp}{\mathrm{cusp}}
\newcommand{\reg}{\mathrm{reg}}
\newcommand{\rs}{\mathrm{rs}}
\newcommand{\Irr}{\mathrm{Irr}}
\newcommand{\Tr}{\mathrm{Tr}}
\newcommand{\Hom}{\mathrm{Hom}}
\newcommand{\Gal}{\mathrm{Gal}}
\newcommand{\WD}{\mathrm{WD}}
\newcommand{\Frob}{\mathrm{Frob}}
\newcommand{\Res}{\mathrm{Res}}
\newcommand{\Tam}{\mathrm{Tam}}
\newcommand{\Pet}{\mathrm{Pet}}
\newcommand{\sgn}{\mathrm{sgn}}
\newcommand{\vol}{\mathrm{vol}}
\newcommand{\Aut}{\mathrm{Aut}}
\newcommand{\Ind}{\mathrm{Ind}}
\newcommand{\BC}{\mathrm{BC}}
\newcommand{\Ad}{\mathrm{Ad}}
\newcommand{\chf}{\mathrm{char}}

\newcommand{\what}{\widehat}

\newcommand{\dd}{\mathrm{d}}

\newcommand{\bA}{\mathbb{A}}
\newcommand{\bR}{\mathbb{R}}
\newcommand{\bS}{\mathbb{S}}
\newcommand{\bZ}{\mathbb{Z}}
\newcommand{\bC}{\mathbb{C}}
\newcommand{\bQ}{\mathbb{Q}}
\newcommand{\bH}{\mathbb{H}}
\newcommand{\bfi}{\mathbf{I}}
\newcommand{\bfa}{\mathbf{a}}
\newcommand{\bfb}{\mathbf{b}}
\newcommand{\cS}{\mathcal{S}}
\newcommand{\cO}{\mathcal{O}}
\newcommand{\cV}{\mathcal{V}}
\newcommand{\cP}{\mathcal{P}}
\newcommand{\cH}{\mathcal{H}}

\newcommand{\scA}{\mathscr{A}}
\newcommand{\scB}{\mathscr{B}}
\newcommand{\scV}{\mathscr{V}}
\newcommand{\scT}{\mathscr{T}}
\newcommand{\scU}{\mathscr{U}}
\newcommand{\scW}{\mathscr{W}}
\newcommand{\scO}{\mathscr{O}}
\newcommand{\scL}{\mathscr{L}}

\newcommand{\fra}{\mathfrak{a}}
\newcommand{\frh}{\mathfrak{h}}
\newcommand{\frt}{\mathfrak{t}}
\newcommand{\frg}{\mathfrak{g}}
\newcommand{\frgl}{\mathfrak{gl}}
\newcommand{\fru}{\mathfrak{u}}

% Math mode symbols for common sets and spaces. Example usage: $\R$
\newcommand{\R}{\mathbb{R}}	% Real numbers
\newcommand{\C}{\mathbb{C}}	% Complex numbers
\newcommand{\Q}{\mathbb{Q}}	% Rational numbers
\newcommand{\Z}{\mathbb{Z}}	% Integers
\newcommand{\N}{\mathbb{N}}	% Natural numbers
\newcommand{\F}{\mathcal{F}}	% Calligraphic F for a sigma algebra
\newcommand{\El}{\mathcal{L}}	% Calligraphic L, e.g. for L^p spaces

% Math mode symbols for probability
\newcommand{\pr}{\mathbb{P}}	% Probability measure
\newcommand{\E}{\mathbb{E}}	% Expectation, e.g. $\E(X)$
\newcommand{\var}{\text{Var}}	% Variance, e.g. $\var(X)$
\newcommand{\cov}{\text{Cov}}	% Covariance, e.g. $\cov(X,Y)$
\newcommand{\corr}{\text{Corr}}	% Correlation, e.g. $\corr(X,Y)$
\newcommand{\B}{\mathcal{B}}	% Borel sigma-algebra

% Other miscellaneous symbols
\newcommand{\tth}{\text{th}}	% Non-italicized 'th', e.g. $n^\tth$
\newcommand{\Oh}{\mathcal{O}}	% Big-O notation, e.g. $\O(n)$
\newcommand{\1}{\mathds{1}}	% Indicator function, e.g. $\1_A$

% Additional commands for math mode
\DeclareMathOperator*{\argmax}{argmax}		% Argmax, e.g. $\argmax_{x\in[0,1]} f(x)$
\DeclareMathOperator*{\argmin}{argmin}		% Argmin, e.g. $\argmin_{x\in[0,1]} f(x)$
\DeclareMathOperator*{\spann}{Span}		% Span, e.g. $\spann\{X_1,...,X_n\}$
\DeclareMathOperator*{\bias}{Bias}		% Bias, e.g. $\bias(\hat\theta)$
\DeclareMathOperator*{\ran}{ran}			% Range of an operator, e.g. $\ran(T) 
\DeclareMathOperator*{\dv}{d\!}			% Non-italicized 'with respect to', e.g. $\int f(x) \dv x$
\DeclareMathOperator*{\diag}{diag}		% Diagonal of a matrix, e.g. $\diag(M)$
\DeclareMathOperator*{\trace}{Tr}		% Trace of a matrix, e.g. $\trace(M)$
\DeclareMathOperator*{\supp}{supp}		% Support of a function, e.g., $\supp(f)$

% Numbered theorem, lemma, etc. settings - e.g., a definition, lemma, and theorem appearing in that 
% order in Lecture 2 will be numbered Definition 2.1, Lemma 2.2, Theorem 2.3. 
% Example usage: \begin{theorem}[Name of theorem] Theorem statement \end{theorem}
\theoremstyle{definition}
\newtheorem{theorem}{Theorem}[section]
\newtheorem{conjecture}{Conjecture}[section]
\newtheorem{proposition}[theorem]{Proposition}
\newtheorem{lemma}[theorem]{Lemma}
\newtheorem{corollary}[theorem]{Corollary}
\newtheorem{definition}[theorem]{Definition}
\newtheorem{example}[theorem]{Example}
\newtheorem{remark}[theorem]{Remark}

% Un-numbered theorem, lemma, etc. settings
% Example usage: \begin{lemma*}[Name of lemma] Lemma statement \end{lemma*}
\newtheorem*{theorem*}{Theorem}
\newtheorem*{proposition*}{Proposition}
\newtheorem*{lemma*}{Lemma}
\newtheorem*{corollary*}{Corollary}
\newtheorem*{definition*}{Definition}
\newtheorem*{example*}{Example}
\newtheorem*{remark*}{Remark}
\newtheorem*{claim}{Claim}
\newtheorem*{question*}{Question}
\newtheorem*{problem*}{Problem}

% --- Left/right header text (to appear on every page) ---

% Do not include a line under header or above footer
\pagestyle{fancy}
\renewcommand{\footrulewidth}{0pt}
\renewcommand{\headrulewidth}{0pt}

% Right header text: Lecture number and title
\renewcommand{\sectionmark}[1]{\markright{#1} }
% \fancyhead[R]{\small\textit{\nouppercase{\rightmark}}}

% Left header text: Short course title, hyperlinked to table of contents
% \fancyhead[L]{\hyperref[sec:contents]{\small Gan-Gross-Prasad conjecture}}

\numberwithin{equation}{section}

\makeatletter
\newcommand{\eqnum}{\refstepcounter{equation}\textup{\tagform@{\theequation}}}
\makeatother
% --- Document starts here ---

\begin{document}

% --- Main title and subtitle ---

\title{Automorphic Functions and Number Theory 
\\ - \\
\normalsize Re-\TeX ed by Seewoo Lee\footnote{seewoo5@berkeley.edu.}}

% --- Author and date of last update ---

\author{Goro Shimura}
\date{\normalsize\vspace{-1ex} Last updated: \today}

% --- Add title and table of contents ---

\maketitle


% --- Abstracts ---

% \tableofcontents\label{sec:contents}
% \begin{abstract}
% \end{abstract}

% --- Main content: import lectures as subfiles ---


\section{Introduction}
\label{sec:intro}

The goal of this note is to introduce the arithmetic of function fields, which is the analogue of number theory for polynomials.
Especially, our main goal is to study various evidences of the following claim:

\begin{myquote}
A theorem that holds for integers is also true for polynomials (over finite fields), and latter is often easier to prove.
\end{myquote}
For example, we will see a proof of Fermat's Last Theorem for polynomials, which only requires few pages to prove.

Dictionary between the integers and the polynomials over finite fields can be found in Table \ref{tab:dictionary} of Appendix.

\subsection*{Prerequisites}
We assume that the readers are familiar with undergraduate level algebra (groups, rings, fields, etc.), number theory (congruences, prime numbers, etc.), and a bit of complex analysis.
Some of the theory of finite fields will be reviewed in Appendix \ref{subsec:handbook_galois_ff}.


\subsection*{Notations}

Let $p$ be a prime number. We denote by $\bF_p$ the finite field of order $p$, which is the field with $p$ elements.
We denote the polynomial ring $\bF_p[T]$ by $A$.
For each nonzero polynomial $f \in A$, we denote it's norm by $|f| = p^{\deg (f)}$, where $\deg (f)$ is the degree of $f$, and we set $|0| = 0$.

\subsection*{Sage codes}

There are some codes in this note, which are mostly written in \href{https://www.sagemath.org/}{Sage}.
Sage is a free \href{https://github.com/sagemath/sage}{open-source} mathematics software system, which is built on top of many existing open-source packages and wrappedn in a Python interface.
You can run them online in \href{https://sagecell.sagemath.org/}{SageMathCell}, or install it on your computer.
Especially, a lot of number-theoretic functions are implemented in Sage, so it is much easier to experiment with it than writing your own code from scratch.
For example, to check if a large number is prime, you can simply run
\begin{minted}[fontsize=\footnotesize,framesep=2mm,bgcolor=lightgray!20]{python}
is_prime(10 ^ 9 + 7)
\end{minted}
Several useful Sage functions are listed in Appendix \ref{subsec:handbook_sage}.


\subsection*{Acknowledgements}


\begin{exercise}
    Prove that $\bZ$ is not a polynomial ring over a field. In other words, show that there is no field $k$ such that $\bZ \cong k[T]$ as rings.
\end{exercise}

\begin{exercise}
    Think about your favorite theorems in number theory, and try to find their polynomial analogues. Some of them may appear in this note, but some of them may not.
\end{exercise}


\newpage
\section{Automorphic functions on the upper half plane, especially modular functions}



Let $\cH$ denote the complex upper half plane:
\[
    \cH = \{z \in \bC: \Im(z) > 0\}.
\]
We let every element $\alpha = \left(\begin{smallmatrix}a&b\\c&d\end{smallmatrix}\right)$, with $\det(\alpha) > 0$, act on $\cH$ by
\begin{equation}
    \alpha(z) = \frac{az + b}{cz + d}
\end{equation}
It is well known that the gorup of analytic automorphisms of $\cH$ is isomorphic to $\SL_2(\bR) / \{\pm 1_2\}$. Let $\Gamma$ be a discrete subgroup of $\SL_2(\bR)$. Then the quotient $\cH / \Gamma$ has a structure of Riemann surface such that the natural projection $\cH \to \cH/ \Gamma$ is holomorphic. If $\cH/\Gamma$ is compact, one can simply define an \emph{automorphic function on $\cH$ weith respect to $\Gamma$} to be a meromorphic function on $\cH$ invariant under the elements of $\Gamma$.
Such a function may be regarded as a meromorphic function on the Riemann surface $\cH / \Gamma$ in an obvious way, and vice versa.
We shall later discuss special values of automorphic functions with respect to $\Gamma$ for an \emph{arithmetically defined} $\Gamma$ with compact $\cH / \Gamma$. But we first consider the most classical group $\Gamma = \SL_2(\bZ)$. Since $\cH/ \Gamma$ is not compact in this case, one has to impose a certain condition on automorphic functions. It is well known that every point of $\cH$ can be transformed by an element of $\Gamma = \SL_2(\bZ)$ into the region
\[
    F = \{z \in \cH: |z| \geq 1, |\Re(z)| \leq 1/2\}.
\]
Now two distinct inner points of $F$ can be transformed to each other by an elemtn of $\Gamma$. Now $\cH / \Gamma$ can be compactified by adjoining a point at infinity. By taking $e^{2 \pi i z}$ as a local parameter around this point, we see that $\cH/ \Gamma$ becomes a compact Riemann surface of genus $0$.
Thus we define an automorphic function with respect to $\Gamma$ to be a meromorphic function on this Riemann surface, considered as a function on $\cH$. 
In other words, let $f$ be a $\Gamma$-invariant meromorphic function on $\cH$. For $\gamma = \left(\begin{smallmatrix}
    1 & 1 \\ 0 & 1
\end{smallmatrix}\right)$, we have $\gamma(z) = z + 1$. Since $f(\gamma(z)) = f(z)$, we can express $f(z)$ in the form $f(z) = \sum_{n =-\infty}^{\infty} c_n e^{2 \pi i n z}$ with $c_n \in \bC$. Now an automorphic function with respect to $\Gamma$ is an $f$ such that $c_n = 0$ for all $n < n_0$ for some $n_0$, i.e. meromorphic in the local parameter $q = e^{2 \pi i z}$ at $q = 0$. Such a function is usually called a \emph{modular function of level one}. Since $\cH / \Gamma$ is of genus $0$, all modular functions of level one form a rational function field over $\bC$. As a generator of this field, one can choose a function $j$ such that
\begin{equation}
    \label{eqn:2.2}
    j(\sqrt{-1}) = 1, j\left(\frac{1 + \sqrt{-3}}{2}\right) = 0, j(\infty) = \infty.
\end{equation}
Obviously the function $j$ can be characterized by \eqref{eqn:2.2} and the property of being a generator of the field of all modular functions of level one.

Now let $K$ be an imaginary quadratic field, and $\fra$ a fractional ideal in $K$. Take a basis $\{\omega_1, \omega_2\}$ of $\fra$ over $\bZ$.
Since $K$ is imaginary, $\omega_1 / \omega_2$ is not real. Therefore one may assume that $\omega_1 / \omega_2 \in \cH$, by exchanging $\omega_1$ and $\omega_2$ if necessary. In this setting, we have

\begin{theorem}
\label{thm:2}
The maximal unramified abelian extension of $K$ can be generated by $j(\omega_1 / \omega_2)$ over $K$.
\end{theorem}

This is the first main theorem of the classical theory of complex multiplication. To construct ramified abelian extensions of $K$, one needs modualr functions of higher level (see below) or elliptic functions with periods $\omega_1, \omega_2$.
Even Theorem \ref{thm:2} can be fully understood with the knowledge of elliptic functions of elliptic curves, though such are not explicitly involved in the statement. Therefore, our next task is to recall some elementary facts on this subject. But before that, it will be worth discussing a few elementary facts about the fractional linear transformations and discontinuous groups.

Every 

% --- Bibliography ---

% Start a bibliography with one item.
% Citation example: "\cite{williams}".

% \nocite{*}
\bibliographystyle{acm} % We choose the "plain" reference style
\bibliography{refs} % Entries are in the refs.bib file


% \begin{thebibliography}{1}

% \bibitem{williams}
%    Williams, David.
%    \textit{Probability with Martingales}.
%    Cambridge University Press, 1991.
%    Print.

% % Uncomment the following lines to include a webpage
% % \bibitem{webpage1}
% %   LastName, FirstName. ``Webpage Title''.
% %   WebsiteName, OrganizationName.
% %   Online; accessed Month Date, Year.\\
% %   \texttt{www.URLhere.com}

% \end{thebibliography}

% --- Document ends here ---

\end{document}