\section{Introduction}


Our starting point is the following theorem which was stated by Kronercker and proved by Weber:

\begin{theorem}
Every finite abelian extension of $\bQ$ is contained in a cyclotomic field $\bQ(\zeta)$ with an $m$-th root of unity $\zeta = e^{2\pi i / m}$ for some positive integer $m$.
\end{theorem}

As is imeediately observed, $\zeta$ is the special value of the exponential function $e^{2\pi i z}$ at $z = 1/m$. One can natually ask the follwoing question:

\textbf{Find analytic functions which play a role analogous to $e^{2\pi i z}$ for a given algebraic number field.}

Such a question was raised by Kronecker and later taken up by Hilbert as the 12th of his famous mathematical problems. 
For an imaginary quadratic field $K$, this was settled by the workds of Kronecker himself, Weber, Takagi, and Hasse. It turns out that the maximal abelian extension of $K$ is generated over $K$ by th especial values of certain elliptic functions and elliptic modular functions.
A primary purpose of these lectures is to indicate briefly how this result can be generalized for the number fields of higher degree, making thereby an introduction to the theory of automorphic functions and abelian varieties. I will also include some results concerning the zeta function of an algebraic curve in the sense of Hasse and Weil, since this subject is closely connected with the above question. Further, it should be point out that the automorphic functions are meaningful as a means of generating not only abelian but also non-abelian algebraic extensions of a number field. Some ideas in this direction will be explained in the last part of the lectures.
