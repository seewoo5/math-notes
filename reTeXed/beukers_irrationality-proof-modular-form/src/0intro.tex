\section{Introduction}

In the years following Apery's discovery of his irrationality proofs for $\zeta(2), \zeta(3)$ (see \cite{van1979proof}), it has become clear that these proofs do not only have significance as irrationality proofs, but the numbers that occur in them serve as interesting examples for several phenomena in algebraic geometry and modular form theory.
See \cite{gessel1982some,beukers1982irrationality,beukers1985some} for congruences of the Ap\'ery numbers and \cite{beukers1984family,stienstra1985picard} for geometrical and modular interpretations.\footnote{The original citations were in a different order, but it seems that this is the correct order.}
Furthermore, it turns out that Apery's proofs themselves are in fact simple consequences of elementary complex analysis on spaces of certain modular forms.
In the present paper we describe this analysis together with some generalisations in Theorems 1 to 5. 
For example, we prove that $8\zeta(3) - 5 \sqrt{5} L(3) \not \in \bQ(\sqrt{5})$, where $L(3) = \sum_{n=1}^{\infty} \left(\frac{n}{5}\right) n^{-3}$.
Although the use of modular forms in irrationality proofs looks promising at first sight, the yield of new irrationality results thus far is disappointingly low. 
However, in methods such as this it is easy to overlook some simple tricks that may give new interesting results.

The first section of this paper describes the general framework of the proofs.
This section may seem vague at first sight, but in combination with the proof of Theorem 1 we hope that things will be clear.
We have given the proof of Theorem 1 as extensively as possible in order to set it as an example for the other proofs, where we omit some minor details now and then. 