\section{Local $L$-functions (non-degenerate $\pi$)}
\label{sec:4}


\subsection{}
\label{sec:4.1}


Suppose $E$ (quadratic) over $F$ is local and $\pi$ is an irreducible admissible representation of $G(F)$.
Because of Proposition \ref{prop:3.6}, it is natural to define local zeta-integrals (of Rankin--Selberg--Jacquet type) by
\[
    L^\mu(W, F_\Phi, s) = \int_{Z\backslash H} W(h) F(h) \d dh
\]
where $W \in \cW(\pi, \psi_N)$, $\mu$ is a character of $E^\times$, $\Phi$ is a Schwartz--Bruhat function on the two-dimensional $E$-vector spacae $W = \langle\ell_{-1}, \ell_{1}\rangle$, and
\[
    F^\mu(h) = \int_{E^\times} (h\cdot\Phi)(t \ell_{-1}) \mu(t) |t|_E^s \dd^\times t.
\]
Note
\[
    F(bh) = \mu(\alpha) |\alpha|_E^s F(h)
\]
for all $h \in H$, and
\[
    b = \begin{bmatrix}
        \alpha & 0 & \beta \\ 0 & 1& 0 \\ 0 & 0 & \bar{\alpha}^{-1}
    \end{bmatrix} \in B.
\]


\begin{remark*}
    There are other kinds of local integrals to consider, namely those which arise from the splitting primes for $E$; cf. \S \ref{sec:3.7}.
    Since these zeta integrals involve the more familiar groups $\GL_3$ and $\GL_2$, we shall concentrate on the unitary integrals instead (dealing with the splitting primes only parenthetically).
\end{remark*}


\subsection{}
\label{sec:4.2}


Suitably moddifying the problem in \S 14 of [Jacquet] on can obtain the following:
\begin{proposition}
    For each $W, \Phi$, and $\mu$ as above, the local zeta-integrals $L^\mu(W, F_\Phi, s)$ converge for $\Re(s) \gg 0$ and define rational functions of $q^{-s}$ satisfying the following conditions:
    \begin{enumerate}[label=(\roman*)]
        \item The subspace of $\bC(q^{-s})$ spanned by $L^\mu(W, F_\Phi, s)$ is in fact a fractional ideal of the ring $\bC[q^{-s}, q^s]$ generated by some polynomial $Q_0$ in $\bC[q^{-s}]$ which is Independent of $W$ and $\Phi$;
        \item There is a rational function of $q^{-s}$, denoted $\gamma(s)$, and a ``contragradient'' $L$-function $\tilde{L}^{\mu^{-1}}(W, F_{\widehat{\Phi}}, s)$, such that for all $W$ and $\Phi$ (and $\widehat{\Phi}$ a special kind of ``partial Fourier transform''),
        \[
            \tilde{L}^{\mu^{-1}}(W, F_{\widehat{\Phi}}, 1 -s) = \gamma(s) L^\mu(W, F_{\Phi}, s).
        \]
    \end{enumerate}
\end{proposition}

\subsection{}
\label{sec:4.3}


\begin{remark*}
\begin{enumerate}[label=(\roman*)]
    \item If we deman that $Q_0(0) = 1$, then $L(s, \pi, \mu) = Q_0(q^{-s})^{-1}$ is the unique Euler factor such that
    \[
        \frac{L^\mu(W, \Phi, s)}{L(s, \pi, \mu)}
    \]
    is entire (actually polynomial in $q^s$ and $q^{-s}$) for all $W$ and $\Phi$, and equal to $1$ for appropriately chosen $W$ and $\Phi$.
    A similar statement holds for $L(s, \tilde{\pi}, \mu)$ and $\tilde{L}^{\mu^{-1}}(W, \Phi, s)$.
    As usual, we regard $L(s, \pi, \mu)$ as the normalized g.c.d. of the zeta-functions $L^\mu(W, \Phi, s)$, and as the local component of a (soon to be defined) global $L$-function $L(s, \pi, \mu)$.
    \item If we let
    \[
        \varepsilon(s, \pi, \psi) = \frac{L(1 - s, \tilde{\pi}, \mu^{-1})\gamma(s)}{L(s, \pi, \mu)},
    \]
    then $\varepsilon$ is the monomial factor relating
    \[
        \frac{L^{\mu^{-1}}(W, \widehat{\Phi}, 1 - s)}{L(s, \tilde{\pi}, \mu^{-1})} \text{ and } \frac{L^\mu(W, \Phi, s)}{L(s, \pi)}.
    \]
    \item Throughout this section, we are implicitly dealing only with nonarchimedean fields; the case of $\bR$ (or $\bC$) is an unfortulately thorny yet unavoidable reality.
\end{enumerate} 
\end{remark*}


\subsection{\underline{Unfamified computations}}
\label{sec:4.4}


In the next few sections we shall compute $L(s, \pi, \mu)$ when \underline{everything} in sight is unfamified.

Thus we suppose $F$ is a local \underline{non}archimedean field of characteristic $\neq 2$, and $E$ is an \underline{unramified} quadratic extension of $F$.
Let $\cO_F$ (resp. $\cO_E$) denote the ring of integers of $F$ (resp. $E$), $\varpi$ (resp. $\varpi_E$) a generator of the prime ideal $\frp$ (resp. $\frp_E$) of $\cO_F$ (resp. $\cO_E$), and $\psi$ a character of $F$ trivial on $\cO_F$ but not on $\frp^{-1} \cO_F$.

Let $K$ denote the standard maximal compact subgroup of $G(F)$ consisting of ``integral'' matrices (entries in $\cO_E$, determinant in $\cO_E^\times$).
Because $E$ is unramified over $F$, we have
\[
    G = NAK
\]
where
\[
    A = \left\{
        \begin{bmatrix}
            t & 0 & 0 \\ 0 & 1& 0 \\ 0 & 0 & t^{-1}
        \end{bmatrix}: t \in F^\times
    \right\}
\]
is the maximal $F$-split torus of $G(F)$.


\subsection{\underline{Class $1$ Whittaker functions}}
\label{sec:4.5}

Suppose $\pi$ is a class 1 (with respect to $K$) irreducible admissible representation of $G(F)$. Then $\pi$ is of the form
\[
    \pi = \pi(\nu) = \Ind_{P}^{G} \nu^*
\]
where $\nu$ is an unramified (quasi-)character of $E^\times$,
\[
    P = MN = \left\{
        \begin{bmatrix}
            \delta & * & * \\ 0 & \beta & * \\ 0 & 0 & \bar{\delta}^{-1}
        \end{bmatrix}
    \right\},
\]
and $\nu^*$ is defined on $M =P/N$ by
\[
    \nu^*\left(
        \begin{bmatrix}
            \delta & & \\ & \beta & \\ & & \bar{\delta}^{-1}
        \end{bmatrix} = \nu(\delta)
    \right).
\]

From [Cas Sh] we know $\pi(\nu)$ is non-degenerate; moreover, the $K$-fixed Whittaker function $W$ in $\cW(\pi(\nu), \psi_N)$, normalized by the fondition $W(k) =1$, is uniquely determined by the following formulas:
\begin{enumerate}[label=(\roman*)]
    \item $W(nak) = \psi_N(n) W(a)$ for all $n \in N, a \in A$, and $k \in $K;
    \item $W\left(\left[\begin{smallmatrix}
        \delta & & \\ & 1&  \\ & & \delta^{-1}
    \end{smallmatrix}\right]\right) = 0$ if $|\delta|_F > 1$; and
    \item for all $n \geq 0$,
    \[
        W\left(\begin{bmatrix}
            \varpi^{n} & 0 & 0 \\ 0 & 1 & 0 \\ 0 & 0 & \varpi^{-n}
        \end{bmatrix}\right) = |\varpi|_F^{2n} \frac{\nu(\varphi)^{n+1} - \nu(\varpi)^{-(n+1)}}{\nu(\varpi) - \nu(\varpi)^{-1}}.
    \]
\end{enumerate}
(c.f. Theorem 5.4 of [Cas Sh]).


\subsection{}
\label{sec:4.6}


We compute $L^\mu(W, F_\Phi, s)$ with $\mu$ an unramified character of $E^\times$, $W(g)$ as in \S \ref{sec:4.5}, and $\Phi$ the characteristic function of the $\cO_E$-module in $E\ell_{-1} \oplus E\ell_{1}$ generated by $\ell_{-1}$ and $\ell_{1}$.

Let $K_H = K \cap H$. Since $Z = N \cap H$, we have $H = ZAK_H$, with corresponding integration formula
\[
    \int_{Z\backslash H} f'(h) \dd h = \int_{K_H} \int_{F^\times} f'\left(\begin{bmatrix}
        a & & \\ & 1 & \\ & & a^{-1}       
    \end{bmatrix}k\right) |a|^{-2} \dd^\times a \dd k.
\]
here $f'$ is a function of $Z\backslash H$, and Haar measure $\dd^\times a$ on $F^\times$ is normalized so that $\dd^\times a(\cO_F^\times)= 1$.


Note $k \in K_H$ implies
\begin{align*}
    F_\Phi \left(\begin{bmatrix}
        a & & \\ & 1 & \\ & & a^{-1}
    \end{bmatrix}k\right) &= \mu(a) |a|_E^s F^\mu(k) \\
    &= \mu(a)|a|_E^s \int_{E^\times} (k \cdot \Phi) (t \ell_{1}) \mu(t) |t|_E^s \dd^\times t \\
    &= \mu(a) |a|_E^s F_\Phi(1)
\end{align*}
since $(k \cdot \Phi)(t \ell_{-1}) = \Phi(t \ell_{-1})$ for our unramified choice of $\Phi$.
Thus we have
\[
    L^\mu(W, F_\Phi, s) = F_\Phi(1) \int_{F^\times} W\left(\begin{bmatrix}
        a & & \\ & 1 & \\ & & a^{-1}
    \end{bmatrix}\right) \mu(a) |a|_F^{2s - 2} \dd^\times a
\]
with
\begin{align*}
    F_\Phi(1) &= \int_{E^\times}\Phi(t \ell_{-1}) \mu(t) |t|_E^s \dd^\times t \\
    &= \int_{E^\times} 1_{\cO_E} (t) \mu(t) |t|_E^s \dd^\times t \\
    &= \frac{1}{1 - \mu(\varpi_E) |\varpi_E|^{s}} = L_E(s, \mu).
\end{align*}
Here $L_E(s, \mu)$ is the local Hecke--Tate factor attached to the quasi-character $\mu$ of $E^\times$; c.f. [Goldstein], \S 8.1.


\subsection{}
\label{sec:4.7}


It remains to compute the integral of our class 1 Whittaker function. 
From the formula \S \ref{sec:4.5} (iii), we have
\begin{align*}
    \int_{F^\times} &W\left(\begin{bmatrix}
        a & & \\ & 1 & \\ & & a^{-1}
    \end{bmatrix}\right) \mu(a) |a|^{2s-2} \dd^\times a \\
    &= \frac{1}{\nu(\varpi) - \nu(\varpi)^{-1}} \sum_{n=0}^{\infty} |\varpi^n|^2 \mu(\varpi^n) |\varpi^n|^{2s-2} (\nu(\varpi)^{n+1} - \nu(\varpi)^{-(n+1)}) \\
    &= \sum_{n=0}^{\infty} \mu(\varpi^n) |\varpi^n|^{2s} \sum_{i+j=n} \nu(\varpi)^i \nu(\varpi^{-1})^{j} \\
    &= \left(\sum_{n=0}^{\infty}\mu(\varpi^n)|\varpi^n|^{2s}\nu(\varpi)^n\right) \left(\sum_{m=0}^{\infty} \mu(\varpi^m) |\varpi^m|^{2s} \nu(\varpi^{-1})^{m}\right) \\
    &= \left(\frac{1}{1 - \mu(\varpi)\nu(\varpi)|\varpi|^{2s}}\right)\left(\frac{1}{1 - \mu(\varpi)\nu^{-1}(\varpi)|\varpi|^{2s}}\right) \\
    &= L_F(2s,\mu\nu) L_F(2s, \mu\nu^{-1}).
\end{align*}
Summing up,
\begin{align*}
    L^\mu(W, F_\Phi, s) &= L(s, \pi, \mu)\\
    &= L_E(s, \mu) L(2s, \mu\nu) L(2s, \mu \nu^{-1}).
\end{align*}
Here $\mu$ is regarded as a character of $E^\times$ in the $L$-factor and (by restriction - along with $\nu$) as a character of $F^\times$ in the remaining two $L$-factors. Altogether,
\[
    L(s, \pi, \mu) = Q_0(q^{-s})^{-1}
\]
with $Q_0$ a polynomial \underline{of degree 6} in $q^{-s} = |\varpi_F|^s$.


\subsection{}
\label{sec:4.8}

To which conjugacy class and representation of the $L$-group ${}^L G$ does $L(s, \pi, \mu)$ correspond?

Recall ${}^L G$ is a semi-direct product
\[
    {}^L G = \GL_3(\bC) \rtimes W_{E/F}
\]
where $W_{E/F}$ fits into the exact sequence
\[
    1 \to E^\times \to W_{E/F} \to \Gal(E/F) \to 1
\]
and $\tau$ in $W_{E/F}$ acts on $\GL_3(\bC)$ through its projection onto $\Gal(E/F)$.
In particular, in the present context, $\Gal(E/F) = \{1, \Fr\}$, with $\Fr$ taking
\[
    g \in \GL_3(\bC) \mapsto \begin{bmatrix}
        0 & 0 & 1 \\ 0 & -1 & 0 \\ 1 & 0 & 0
    \end{bmatrix} (g^\intercal)^{-1} \begin{bmatrix}
        0 & 0 & 1 \\ 0 & -1 & 0 \\ 1 & 0 & 0
    \end{bmatrix}.
\]

Now let $\rho_0$ denote the standard representation of $\GL_3(\bC)$ and set
\[
    \rho = \Ind_{\GL_3(\bC) \times E^\times}^{{}^L G} (\rho_0).
\]
Since $W_{E/F}$ acts non-trivially on $\rho_0$ (taking it to its ``twisted'' contragredient), is an irreducible six-dimensional representation of ${}^L G$ whose restriction to ${}^L G^\circ = \GL_3(\bC)$ is the direct sum of $\rho_0(g)$ and $\tilde{\rho}_0(g) = \rho_0\left(\left[\begin{smallmatrix}
    & & 1 \\ & -1 & \\ 1 & & 
\end{smallmatrix}\right] (g^\intercal)^{-1} \left[\begin{smallmatrix}
    & & 1 \\ & -1 & \\ 1 & & 
\end{smallmatrix}\right] \right)$.

\begin{proposition}
\label{prop:4.8}
Let $t_\nu$ denote the conjugacy class in ${}^L G$ determined (via the Satake isomorphism) by the unramified representation $\pi(\nu)$ of $G(F)$.
Then
\[
    L(s, \pi, 1)^{-1} = \det(I - \rho(t_v)|\varpi|^s_F)^{-1},
\]
the (local) $L$-function Langlands attaches to the data $\{t_\nu, \rho\}$.
\end{proposition}
\begin{proof}
The conjugacy class $t_\nu$ determined by $\pi(\nu)$ is
\[
    \begin{bmatrix}
        \nu(\varpi) & 0 & 0 \\ 0 & 1 & 0 \\ 0 & 0 & 1
    \end{bmatrix} \rtimes \Fr \in {}^L G,
\]
so
% https://tex.stackexchange.com/questions/323297/typing-block-matrices-with-zero-blocks-and-separators
\[
    \rho(t_\nu) = 
    \begin{bNiceArray}{ccc|ccc}
        \Block{3-3}{} &&& 1 &  &  \\
        &&&  & 1 &  \\
        &&&  &  & \nu^{-1}(\varpi) \\
        \hline
        \nu(\varpi) &  &  & \Block{3-3}{} \\
         & 1 & \\
         &  & 1\\
    \end{bNiceArray}.
\]
A straightforward calculation with determinants then shows
\[
    \det(I - \rho(t_\nu)|\varpi_F|^{s})^{-1} = L_E(s, 1) L(2s, \nu) L(2s, \nu^{-1}),
\]
as claimed.
\end{proof}


\subsection{}
\label{sec:4.9}


Finally, we relate to the standard $L$-functions on $\GL_3$ over $E$.
Still working locally, let
\[
    G' = \Res_{F}^{E} G_F = \Res_{F}^{E} (\GL_3).
\]
Then $G'(F) = \GL_3(E)$,
\[
    {}^L G'_F = \GL_3(\bC) \times \GL_3(\bC) \rtimes W_{E/F}
\]
and
\[
    {}^L (\GL_3)_E = \GL_3(\bC) \times W_{E/E}.
\]
Now consider the representation $\pi^\nu \otimes \mu$ of $G'(F) = \GL_3(E)$ induced from the unramified character
\[
    \begin{bmatrix}
        a_1 & * & * \\ 0 & a_2 & * \\ 0 & 0 & a_3        
    \end{bmatrix} \mapsto \mu\nu(a_1)\mu(a_2)\mu\nu^{-1}(a_3),
\]
of the standard Borel subgroup of $\GL_3(E)$.
The conjugacy class in ${}^L (\GL_3)_E$ corresponding to $\pi^\nu \otimes \mu$ is
\[
    \begin{bmatrix}
        \mu\nu(\varpi) & 0 & 0 \\ 0 & \mu(\varpi) & 0 \\ 0 & 0 & \mu\nu^{-1}(\varpi)
    \end{bmatrix} \rtimes \Fr.
\]
Therefore, the following $L$-factors coincide:
\begin{enumerate}[label=(\roman*)]
    \item $L(s, \pi^\nu \otimes \mu, \rho_0)$, as an $L$-factor for $\GL_3$ \underline{over $E$}, and
    \item $L(s, \pi(\nu), \mu)$, our $L$-factor attached to $G$ over $F$.
\end{enumerate}
Note that $\pi^\nu$ is clearly a \underline{base change lift} of $\pi(\nu)$ on $G$ to $G'$.

