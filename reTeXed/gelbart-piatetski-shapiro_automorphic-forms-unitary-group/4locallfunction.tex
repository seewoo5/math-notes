\section{Local $L$-functions (non-degenerate $\pi$)}
\label{sec:4}


\subsection{}
\label{sec:4.1}


Suppose $E$ (quadratic) over $F$ is local and $\pi$ is an irreducible admissible representation of $G(F)$.
Because of Proposition \ref{prop:3.6}, it is natural to define local zeta-integrals (of Rankin--Selberg--Jacquet type) by
\[
    L^\mu(W, F_\Phi, s) = \int_{Z\backslash H} W(h) F(h) \d dh
\]
where $W \in \cW(\pi, \psi_N)$, $\mu$ is a character of $E^\times$, $\Phi$ is a Schwartz--Bruhat function on the two-dimensional $E$-vector spacae $W = \langle\ell_{-1}, \ell_{1}\rangle$, and
\[
    F^\mu(h) = \int_{E^\times} (h\cdot\Phi)(t \ell_{-1}) \mu(t) |t|_E^s \dd^\times t.
\]
Note
\[
    F(bh) = \mu(\alpha) |\alpha|_E^s F(h)
\]
for all $h \in H$, and
\[
    b = \begin{bmatrix}
        \alpha & 0 & \beta \\ 0 & 1& 0 \\ 0 & 0 & \bar{\alpha}^{-1}
    \end{bmatrix} \in B.
\]


\begin{remark*}
    There are other kinds of local integrals to consider, namely those which arise from the splitting primes for $E$; cf. \S \ref{sec:3.7}.
    Since these zeta integrals involve the more familiar groups $\GL_3$ and $\GL_2$, we shall concentrate on the unitary integrals instead (dealing with the splitting primes only parenthetically).
\end{remark*}


\subsection{}
\label{sec:4.2}


Suitably moddifying the problem in \S 14 of [Jacquet] on can obtain the following:
\begin{proposition}
    For each $W, \Phi$, and $\mu$ as above, the local zeta-integrals $L^\mu(W, F_\Phi, s)$ converge for $\Re(s) \gg 0$ and define rational functions of $q^{-s}$ satisfying the following conditions:
    \begin{enumerate}[label=(\roman*)]
        \item The subspace of $\bC(q^{-s})$ spanned by $L^\mu(W, F_\Phi, s)$ is in fact a fractional ideal of the ring $\bC[q^{-s}, q^s]$ generated by some polynomial $Q_0$ in $\bC[q^{-s}]$ which is Independent of $W$ and $\Phi$;
        \item There is a rational function of $q^{-s}$, denoted $\gamma(s)$, and a ``contragradient'' $L$-function $\tilde{L}^{\mu^{-1}}(W, F_{\widehat{\Phi}}, s)$, such that for all $W$ and $\Phi$ (and $\widehat{\Phi}$ a special kind of ``partial Fourier transform''),
        \[
            \tilde{L}^{\mu^{-1}}(W, F_{\widehat{\Phi}}, 1 -s) = \gamma(s) L^\mu(W, F_{\Phi}, s).
        \]
    \end{enumerate}
\end{proposition}
