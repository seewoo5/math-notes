\section{$L$-functions \`a la Rankin--Selberg--Jacquet}
\label{sec:3}


We are now ready to attach (global) $L$-functions to (non-degenerate) cuspidal representations $\pi$ of $G(\bA)$, The method used goes back to [Rankin] and [Selberg] who used it to analytically continue the convolution of Dirichlet series corresponding to classical holomorphic modular forms.
The reformulation of their construction in the language of representation theory was carried out in detail by [Jacquet], for $\GL_2 \times \GL_2$, and by [PS 1] in general (but without details or exp1icit computation).
In this Section (and the next), we carry out the construction for $G = \rU_{2, 1}$.


\subsection{}
\label{sec:3.1}


Recall $V$ is a 3-dimensional vector space over $E/F$ and $(-,-)_V$ is a hermitian form on $V$ whose matrix with respect to the basis $\{\ell_{-1}, \ell_{0}, \ell_{1}\}$ is $\left[\begin{smallmatrix}
    0 & 0 & 1 \\ 0 & 1& 0 \\ 1 & 0 & 0
\end{smallmatrix}\right]$.

Let $H \subset G$ denote the stabilizer of the anisotropic line $\langle \ell_0 \rangle$.
Then $H$ also preserves the orthocomplement of $\langle \ell_0 \rangle$, namely
\[
    W = \langle \ell_{-1}, \ell_{1}\rangle.
\]
Regarding $W$ as a 2-dimensional Hermitian space (whose matrix with respect to the basis $\{\ell_{-1}, \ell_{1}\}$ is $\left[\begin{smallmatrix}
    0 & 1 \\ 1& 0
\end{smallmatrix}\right]$), we have $H \simeq \rU(W) \simeq \rU_{1, 1}$ via the embedding
\[
    \begin{bmatrix}
        * & * \\ * & *
    \end{bmatrix} \mapsto \begin{bmatrix}
        * & 0 & * \\ 0 & 1 & 0 \\ * & 0 & *
    \end{bmatrix}.
\]

Let $B$ denote the standard maximal parabolic (Borel) subgroup
\[
    \left\{
    \begin{bmatrix}
        \alpha & 0 & \beta \\ 0 & 1 & 0 \\ 0 & 0 & \bar{\alpha}^{-1}
    \end{bmatrix}
    \right\} \text{ of }H
\]
so that $B \simeq T \ltimes Z$, with $T$ the torus
\[
    T = \left\{
        \begin{bmatrix}
            \alpha & & \\ & 1 & \\ & & \bar{\alpha}^{-1}
        \end{bmatrix}: \alpha \in E^\times
    \right\}
\]
and $Z$ is the center of $N$.


\subsection{}
\label{sec:3.2}

Given an automorphic cuspidal realization $(\pi, H_\pi)$ of $G(\bA)$, and $f \in H_\pi$, we shall analyze a global zeta-integral of the form

\begin{equation}
    \label{eqn:3.2.1}
    L^\mu(f, F, s) = \int_{H(F) \backslash H(\bA)} f(h) E^\mu(h, F, s) \dd h.
\end{equation}
First we need to describe the (as yet) undefined terms $\mu, F, E$ etc.

Fix a (not necessarily unitary) character $\mu$ of the id\`ele class group $E^\times \backslash \bA_E^\times$ of $E$, and $s \in \bC$, define a character $\omega_\mu^s$ of the Borel subgroup $B$ by the formula
\begin{equation}
\label{eqn:3.2.2}
    \omega_\mu^s\left(
        \begin{bmatrix}
            \alpha & 0 & \beta \\ 0 & 1 & 0 \\ 0 & 0 & \bar{\alpha}^{-1}
        \end{bmatrix}
    \right) = \mu(\alpha) |\alpha|_E^s, \quad \alpha \in \bA_E^\times.
\end{equation}

Fixing an arbitrary Schwartz-Bruhat function $\phi$ in the space $\cS(W({\bA_E}))$, set
\begin{equation}
    \label{eqn;3.2.3}
    F_\phi(h) = \int_{\bA_E^\times} (h\cdot \phi)(t \ell_{-1})\mu(t) |t|_E^s \dd^\times t
\end{equation}
where $(h \cdot \phi)(w) = \phi(h^{-1} \cdot w)$, and $h \cdot w$ denotes the natural action of $H(\bA)$ on $W(\bA) \subset V(\bA)$;
as usual, this integral converges for $\Re(s)$ sufficiently large, and continues meromorphically to define a function of $h$ on $H(\bA)$ for all $s$ in $\bC$.
Note
\[
    F(bh) = \omega_{\mu}^{s}(b)F(h) \text{ for } b \in B(\bA), h \in H(\bA).
\]

Finally, the Eisenstein series $E^\mu(h, F, s)$ is defined by the familiar series
\[
    E^\mu(h, F, s) = \sum_{\gamma \in B(F) \backslash H(F)} F(\gamma h);
\]
it converges initially only for $\Re(s)$ large, but the Selberg--Langlands theory of Eisenstein series implies that the function it defines continues meromorphically in $s$ and defines an automorphic form on $H(\bA)$.


\subsection{}
\label{sec:3.3}
\begin{remark*}
\begin{enumerate}[label=(\roman*)]
    \item Because $E(h,F,s)$ is a automorphic form on $H$, and the restriction of the cusp form $f$ from $G(\bA)$ to $H(\bA)$ is still rapidly decreasing, the integral defining $L^\mu(f, F, s)$ (cf. \eqref{eqn:3.2.1}) is convergent.
    The resulting function of $s$ - the global zeta-function of $f$ - has poles which can arise only from poles of $E(h,F,s)$.
    \item The function $E(h, F, s)$ is essentially the familiar $\GL_2$-Eisenstein series discussed (for example) in [Jacquet].
    Indeed, $\SL_2$ is isomorphic to a subgroup of $H$, namely $\SU_{2, 1}$, and $H \simeq \SU_{2, 1} \rtimes \rU_{1}$, where $\rU_{1}\simeq \{\left[\begin{smallmatrix} \alpha & 0 \\ 0 & 1 \end{smallmatrix}: \alpha \in E^1\right]\}$ and $\left[\begin{smallmatrix}
        \alpha & 0 \\ 0 & 1
    \end{smallmatrix}\right]$ acts on $\SU_{2, 1}$ by conjugation.
    Thus the restriction of $E(h, F, s)$ to $\SU_{2, 1}$ is the familiar Eisenstein series on $\SL_2(\bA)$ with functional equation and non-trivial residue given by the constant function (arising from the pole at $s = 1$).
    At the $\rU_{2, 1}$ level, these residues become proportional to $\mu(\det h)$, and the functional equation relates $E(h, F, s)$ to a ``partially Fourier transformed'' Eisenstein series at $1 - s$.
\end{enumerate}
\end{remark*}


\subsection{}
\label{sec:3.4}

From the theory above, we conclude that $L^\mu(f, F, s)$ is meromorphic in $\bC$ with funcitonal equation relating values at $s$ and $1 - s$; more significantly for the sequel, the only possible residues of $L^\mu(f, F, s)$ are proportional to 
\begin{equation}
    \label{eqn:3.4.1}
    \int_{H(F) \backslash H(\bA)} f(h) \mu(\det h) \dd h.
\end{equation}
In particular, if this last integral vanishes, then the zeta-function $L^\mu(f, F, s)$ is entire.

Regarding $H(F) \backslash H(\bA)$ as an (algebraic) cycle in $G(F)\backslash G(\bA)$, we (ultimately) obtain the following statement: The existence of a pole for $L^\mu(f, F, s)$ (and ultimately the $L$-function $L(s, \pi, \mu)$) is related to the non-vanishing integral of $f$ in $H_\pi$ (suitably tensored with $\mu$) over the cycle coming from $\rU_{1, 1}$.

We shall return to these considerations in \S \ref{sec:5}.
For the moment, we content ourselves with a factorization of $L^\mu(f, F, s)$ into local integrals.


\subsection{}
\begin{proposition*}
\[
    L^\mu(f, F, s) = \int_{Z(\bA) \backslash H(\bA)} W_f^\psi(h) F(h) \dd h.
\]
\end{proposition*}
\begin{proof}
From the definition of the series $E$,
\begin{align*}
    L^\mu(f, F, s) &= \int_{H(F) \backslash H(\bA)} \sum_{B(F) \backslash H(F)} f(\gamma h) F(\gamma h) \dd h \\
    &= \int_{B(F) \backslash H(\bA)} f(h) F(h) \dd h.
\end{align*}
Recall our subgroups
\[
    Z = \left\{
        \begin{bmatrix}
            1 & 0 & z \\ 0 & 1& 0 \\ 0 & 0 & 1
        \end{bmatrix}
    \right\} \subset H = \left\{
        \begin{bmatrix}
            * & 0 & * \\ 0 & 1 & 0 \\ * & 0 & *
        \end{bmatrix}
    \right\}
\]
Since $F(h)$ is invariant by $Z(\bA)$ (c.f. \eqref{eqn:3.2.2}),
\begin{align*}
    L^\mu(f, F, s) &= \int_{B(F) Z(\bA) \backslash H(\bA)} F(h) \left(\int_{Z(F) \backslash Z(\bA)} f(zh) \dd z\right) \dd h \\
    &= \int_{B(F) Z(\bA) \backslash H(\bA)} F(h) f_{00}(h) \dd h
\end{align*}
Now recall the Fourier expansion
\[
    f_{00}(h) = \sum_{\delta \in E^\times} W_{f}^{\psi} \left(
        \begin{bmatrix}
            \delta & & \\ & 1 & \\ & & \bar{\delta}^{-1}
        \end{bmatrix}h
    \right),
\]
c.f. \eqref{eqn:2.2.1}. Since
\[
    B_F = \left\{
        \begin{bmatrix}
            \delta & & z \\0 & 1& 0 \\0 & 0 & \bar{\delta}^{-1}
        \end{bmatrix}: \delta \in E^\times, z \in E
    \right\}
\]
we have
\[
    Z(\bA)\backslash B(F)Z(\bA) = \left\{
        \begin{bmatrix}
            \delta & & \\ & 1 & \\ & & \bar{\delta}^{-1}
        \end{bmatrix}: \delta \in E^\times
    \right\}
\]
and therefore
\begin{align*}
    L^\mu(f, F, s) &= \int_{B(F) Z(\bA) \backslash H(\bA)} \left(\sum_{b \in Z(\bA) \backslash B(F) Z(\bA)} W_f^\psi(bh)\right) F(h) \dd h \\
    &= \int_{Z(\bA) \backslash H(\bA)} W_f^\psi(h) F(h) \dd h
\end{align*}
as was to be shown.
\end{proof}
    

\begin{remark*}
\begin{enumerate}[label=(\roman*)]
    \item We defined $L^\mu(f, F, s)$ for any $f \in H_\pi$ without assuming $H_\pi$ orthogonal to all hypercuspforms.
    However, this last proposition shows tha t$L^\mu(f, F, s)$ is identically $0$ if $\cW(\pi, \psi) =\{0\}$, i.e. if $f$ is a hypercuspform.
    This is wy the Rankin--Selberg method fails for arbitrary $\pi$, and why (in \S \ref{sec:9}) we need ot use Shimura's method.
    \item The significance of this Proposition is that it allows to factor $L^\mu(f, F, s)$ into local zeta-integrals, one for each place $v$ of $F$.
    Note that whenever $v$ splits in $E$, i.e. whenever $E\otimes_F F_v$ splits as the direct sum of two fields $E_{w_1}$ and $E_{w_2}$ (isomorphic to $F_v$), we have
    \[
        G_v = G(F_v) \simeq \GL_3(F_v).
    \]
\end{enumerate} 
\end{remark*}


\subsection{}
\label{sec:3.6}
\begin{proposition}
Suppose $\Phi = \prod_v \Phi_w$ in $\cS(W(\bA_E))$, the product being taken over all the primes $w$ of $E$.
Assuming $f$ is not hypercuspidal, we have
\[
    L^\mu(f, F, s) = \prod_v \int_{Z_v \backslash H_v} W_v(h) F_v(h) \dd h
\]
where $v$ is an arbitrary place \underline{of $F$}, $W_v$ belongs to the local Whittaker space $W(\pi_v, \psi_v)$, and
\[
    F_v(h_v) = \prod_{w|v} \int_{E_w^\times} (h_v \cdot \Phi_w) (t \ell_{-1}) \mu_w(t) |t|_w^{s} \dd^\times t.
\]
\end{proposition}
\begin{proof}
Since the domain of the integration factors, we need only check that the integrand does also.
By uniqueness of Whittaker models, $W(h)$ in $\cW(\pi, \psi)$ with $\psi =\prod_v \psi_v$ implies $W(h) = \prod_v W_v(h_v)$, with each $W_v \in\cW(\pi_c, \psi_c)$.
Also, from the definition of $F_\Phi(h)$, it follows
\[
    F_\Phi(h) = \prod_v \left(\prod_{w|v} \int_{E_w^\times} (h_v \cdot \Phi_w) (t \ell_{-1}) \mu_w(t) |t|^{s}_w \dd^\times t\right)
\]
as claimed.
\end{proof}


\subsection{}
\label{sec:3.7}

When $v$ splits in $E$, each integral
\[
    \int_{E_{w_i} \simeq F_v} (h_v \cdot \Phi)(t \ell_{-1}) \mu_{w_i} (t) |t|_{w_i}^{s} \dd^\times t
\]
reduces to the distribution $z(\alpha^s\mu, h \cdot \Phi)$ treated in \S 14 of [Jacquet].
Indeed, in this case, $H_v \simeq \rU_{1, 1} \simeq \GL_2(F_v)$, and $\ell_{-1}$ corresponds to the vector $(0, 1)$ in $F_v^2$ fixed by the unipotent subgroup of $\GL_2(F_v)$ (under the action $(x,y) \left(\begin{smallmatrix}
    a & b \\ c & d
\end{smallmatrix}\right)$).

In this case, our local zeta-integral takes the form
\[
    \int_{\GL_2} W_v(h) z(\alpha^s \mu_{w_1}, h\cdot \Phi_{w_1}) z(\alpha^s \mu_{w_2}, h \cdot \Phi_{w_2}) \dd h
\]
where $W_v(h)$ is a Whittaker function on $\GL_3$ restricted to $\GL_2$ (via the embedding $\left[\begin{smallmatrix}
    * &* \\ * &* 
\end{smallmatrix}\right] \mapsto \left[\begin{smallmatrix} * & 0 & * \\ 0 & 1 & 0 \\ * &0 & * \end{smallmatrix}\right]$), and $w_1$, $w_2$ are the two primes of $E$ which divide $v$.