% --- LaTeX Lecture Notes Template - S. Venkatraman ---

% --- Set document class and font size ---

\documentclass[letterpaper, 12pt]{article}

% --- Package imports ---

% Extended set of colors
\usepackage[dvipsnames]{xcolor}

\usepackage{
  amsmath, amsthm, amssymb, mathtools, dsfont, units,          % Math typesetting
  graphicx, wrapfig, subfig, float,                            % Figures and graphics formatting
  listings, color, inconsolata, pythonhighlight,               % Code formatting
  fancyhdr, sectsty, hyperref, enumerate, enumitem, framed }   % Headers/footers, section fonts, links, lists

% lipsum is just for generating placeholder text and can be removed
\usepackage{hyperref, lipsum} 


% --- Fonts ---

\usepackage{newpxtext, newpxmath, inconsolata}
\usepackage{amsfonts}

\usepackage{tikz}
\usepackage{tikz-cd}
\usepackage{enumitem}
% \usepackage{enumerate}
\usepackage[title]{appendix}


% --- Page layout settings ---

% Set page margins
\usepackage[left=1.35in, right=1.35in, top=1.0in, bottom=.9in, headsep=.2in, footskip=0.35in]{geometry}

% Anchor footnotes to the bottom of the page
\usepackage[bottom]{footmisc}

% Set line spacing
\renewcommand{\baselinestretch}{1.2}

% Set spacing between paragraphs
\setlength{\parskip}{1.3mm}

% Allow multi-line equations to break onto the next page
\allowdisplaybreaks

% --- Page formatting settings ---

% Set image captions to be italicized
\usepackage[font={it,footnotesize}]{caption}

% Set link colors for labeled items (blue), citations (red), URLs (orange)
\hypersetup{colorlinks=true, linkcolor=RoyalBlue, citecolor=RedOrange, urlcolor=ForestGreen}

% Set font size for section titles (\large) and subtitles (\normalsize) 
\usepackage{titlesec}
\titleformat{\section}{\large\bfseries}{{\fontsize{19}{19}\selectfont\textreferencemark}\;\; }{0em}{}
\titleformat{\subsection}{\normalsize\bfseries\selectfont}{\thesubsection\;\;\;}{0em}{}

% Enumerated/bulleted lists: make numbers/bullets flush left
%\setlist[enumerate]{wide=2pt, leftmargin=16pt, labelwidth=0pt}
\setlist[itemize]{wide=0pt, leftmargin=16pt, labelwidth=10pt, align=left}

% --- Table of contents settings ---

\usepackage[subfigure]{tocloft}

% Reduce spacing between sections in table of contents
\setlength{\cftbeforesecskip}{.9ex}

% Remove indentation for sections
\cftsetindents{section}{0em}{0em}

% Set font size (\large) for table of contents title
\renewcommand{\cfttoctitlefont}{\large\bfseries}

% Remove numbers/bullets from section titles in table of contents
\makeatletter
\renewcommand{\cftsecpresnum}{\begin{lrbox}{\@tempboxa}}
\renewcommand{\cftsecaftersnum}{\end{lrbox}}
\makeatother

% --- Set path for images ---

\graphicspath{{Images/}{../Images/}}

% --- Math/Statistics commands ---

% Add a reference number to a single line of a multi-line equation
% Usage: "\numberthis\label{labelNameHere}" in an align or gather environment
\newcommand\numberthis{\addtocounter{equation}{1}\tag{\theequation}}

% Shortcut for bold text in math mode, e.g. $\b{X}$
\let\b\mathbf

% Shortcut for bold Greek letters, e.g. $\bg{\beta}$
\let\bg\boldsymbol

% Shortcut for calligraphic script, e.g. %\mc{M}$
\let\mc\mathcal

% \mathscr{(letter here)} is sometimes used to denote vector spaces
\usepackage[mathscr]{euscript}

% Convergence: right arrow with optional text on top
% E.g. $\converge[p]$ for converges in probability
\newcommand{\converge}[1][]{\xrightarrow{#1}}

% Weak convergence: harpoon symbol with optional text on top
% E.g. $\wconverge[n\to\infty]$
\newcommand{\wconverge}[1][]{\stackrel{#1}{\rightharpoonup}}

% Equality: equals sign with optional text on top
% E.g. $X \equals[d] Y$ for equality in distribution
\newcommand{\equals}[1][]{\stackrel{\smash{#1}}{=}}

% Normal distribution: arguments are the mean and variance
% E.g. $\normal{\mu}{\sigma}$
\newcommand{\normal}[2]{\mathcal{N}\left(#1,#2\right)}

% Uniform distribution: arguments are the left and right endpoints
% E.g. $\unif{0}{1}$
\newcommand{\unif}[2]{\text{Uniform}(#1,#2)}

% Independent and identically distributed random variables
% E.g. $ X_1,...,X_n \iid \normal{0}{1}$
\newcommand{\iid}{\stackrel{\smash{\text{iid}}}{\sim}}

% Sequences (this shortcut is mostly to reduce finger strain for small hands)
% E.g. to write $\{A_n\}_{n\geq 1}$, do $\bk{A_n}{n\geq 1}$
\newcommand{\bk}[2]{\{#1\}_{#2}}

% \setcounter{section}{-1}

\newcommand{\SL}{\mathrm{SL}}
\newcommand{\Sp}{\mathrm{Sp}}
\newcommand{\Mp}{\mathrm{Mp}}
\newcommand{\GL}{\mathrm{GL}}
\newcommand{\SO}{\mathrm{SO}}
\newcommand{\SU}{\mathrm{SU}}
\newcommand{\PGL}{\mathrm{PGL}}
\newcommand{\PSL}{\mathrm{PSL}}
\newcommand{\rM}{\mathrm{M}}
\newcommand{\rN}{\mathrm{N}}
\newcommand{\rU}{\mathrm{U}}
\newcommand{\rO}{\mathrm{O}}
\newcommand{\rP}{\mathrm{P}}
\newcommand{\JL}{\mathrm{JL}}
\newcommand{\Ind}{\mathrm{Ind}}
\newcommand{\stab}{\mathrm{Stab}}

\newcommand{\dd}{\mathrm{d}}

\newcommand{\bA}{\mathbb{A}}
\newcommand{\bR}{\mathbb{R}}
\newcommand{\bZ}{\mathbb{Z}}
\newcommand{\bC}{\mathbb{C}}
\newcommand{\bQ}{\mathbb{Q}}
\newcommand{\cS}{\mathcal{S}}
\newcommand{\cL}{\mathcal{L}}
\newcommand{\cO}{\mathcal{O}}
\newcommand{\cW}{\mathcal{W}}
\newcommand{\scL}{\mathscr{L}}

% Math mode symbols for common sets and spaces. Example usage: $\R$
\newcommand{\R}{\mathbb{R}}	% Real numbers
\newcommand{\C}{\mathbb{C}}	% Complex numbers
\newcommand{\Q}{\mathbb{Q}}	% Rational numbers
\newcommand{\Z}{\mathbb{Z}}	% Integers
\newcommand{\N}{\mathbb{N}}	% Natural numbers
\newcommand{\F}{\mathcal{F}}	% Calligraphic F for a sigma algebra
\newcommand{\El}{\mathcal{L}}	% Calligraphic L, e.g. for L^p spaces

% Math mode symbols for probability
\newcommand{\pr}{\mathbb{P}}	% Probability measure
\newcommand{\E}{\mathbb{E}}	% Expectation, e.g. $\E(X)$
\newcommand{\var}{\text{Var}}	% Variance, e.g. $\var(X)$
\newcommand{\cov}{\text{Cov}}	% Covariance, e.g. $\cov(X,Y)$
\newcommand{\corr}{\text{Corr}}	% Correlation, e.g. $\corr(X,Y)$
\newcommand{\B}{\mathcal{B}}	% Borel sigma-algebra

% Other miscellaneous symbols
\newcommand{\tth}{\text{th}}	% Non-italicized 'th', e.g. $n^\tth$
\newcommand{\Oh}{\mathcal{O}}	% Big-O notation, e.g. $\O(n)$
\newcommand{\1}{\mathds{1}}	% Indicator function, e.g. $\1_A$

% Additional commands for math mode
\DeclareMathOperator*{\argmax}{argmax}		% Argmax, e.g. $\argmax_{x\in[0,1]} f(x)$
\DeclareMathOperator*{\argmin}{argmin}		% Argmin, e.g. $\argmin_{x\in[0,1]} f(x)$
\DeclareMathOperator*{\spann}{Span}		% Span, e.g. $\spann\{X_1,...,X_n\}$
\DeclareMathOperator*{\bias}{Bias}		% Bias, e.g. $\bias(\hat\theta)$
\DeclareMathOperator*{\ran}{ran}			% Range of an operator, e.g. $\ran(T) 
\DeclareMathOperator*{\dv}{d\!}			% Non-italicized 'with respect to', e.g. $\int f(x) \dv x$
\DeclareMathOperator*{\diag}{diag}		% Diagonal of a matrix, e.g. $\diag(M)$
\DeclareMathOperator*{\trace}{Tr}		% Trace of a matrix, e.g. $\trace(M)$
\DeclareMathOperator*{\supp}{supp}		% Support of a function, e.g., $\supp(f)$

% Numbered theorem, lemma, etc. settings - e.g., a definition, lemma, and theorem appearing in that 
% order in Lecture 2 will be numbered Definition 2.1, Lemma 2.2, Theorem 2.3. 
% Example usage: \begin{theorem}[Name of theorem] Theorem statement \end{theorem}
\theoremstyle{definition}
\newtheorem{theorem}{Theorem}[section]
\newtheorem{proposition}[theorem]{Proposition}
\newtheorem{lemma}[theorem]{Lemma}
\newtheorem{corollary}[theorem]{Corollary}
\newtheorem{definition}[theorem]{Definition}
\newtheorem{example}[theorem]{Example}
\newtheorem{remark}[theorem]{Remark}

% Un-numbered theorem, lemma, etc. settings
% Example usage: \begin{lemma*}[Name of lemma] Lemma statement \end{lemma*}
\newtheorem*{theorem*}{Theorem}
\newtheorem*{proposition*}{Proposition}
\newtheorem*{lemma*}{Lemma}
\newtheorem*{corollary*}{Corollary}
\newtheorem*{definition*}{Definition}
\newtheorem*{example*}{Example}
\newtheorem*{remark*}{Remark}
\newtheorem*{claim}{Claim}

% --- Left/right header text (to appear on every page) ---

% Do not include a line under header or above footer
\pagestyle{fancy}
\renewcommand{\footrulewidth}{0pt}
\renewcommand{\headrulewidth}{0pt}

% Right header text: Lecture number and title
\renewcommand{\sectionmark}[1]{\markright{#1} }
\fancyhead[R]{\small\textit{\nouppercase{\rightmark}}}

% Left header text: Short course title, hyperlinked to table of contents
\fancyhead[L]{\hyperref[sec:contents]{\small Automorphic forms on $\rU_{2, 1}$}}

% --- Document starts here ---

\begin{document}

% --- Main title and subtitle ---

\title{Automorphic forms and $L$-functions for the unitary group\footnote{Notes based on the lectures by S. G. at the University of Maryland Special Year on Lie Group Representations, 1982-83.} \\[1em]
\normalsize Re-\TeX ed by Seewoo Lee\footnote{seewoo5@berkeley.edu}}

% --- Author and date of last update ---

\author{Stephen Gelbart \\ Department of Mathematics \\ Cornell University \\ Ithaca, New York 14853/USA \\ \\ and \\  \\ Ilya Piatetski-Shapiro \\ Department of Mathematics \\ Yale University, New Haven, CT. 06520/USA \\ Tel Aviv University, Ramat-Aviv, Israel}
\date{\normalsize\vspace{-1ex} Last updated: \today}

% --- Add title and table of contents ---

\maketitle

% --- Main content: import lectures as subfiles ---

% \input{Lectures/Lecture1}





\section{Introduction}

\begin{conjecture}[Langlands functoriality conjecture] Let $G$ and $G'$ be reductive groups over a global field $F$. 
\end{conjecture}
This is an introductory note on Langlands functoriality conjecture view towards classical examples. Here is a list of topics we are going to study:

\begin{enumerate}
    \item Automorphic induction
    \item Base change
    \item Rankin-Selberg product
    \item Symmetric power lifting and Selberg's 1/4 conjecture
    \item Jacquet-Langlands correspondence
    \item Theta correspondence and Howe duality
\end{enumerate}

\setcounter{tocdepth}{1}
\tableofcontents\label{sec:contents}

\section*{Notation}
\label{sec:notation}

\begin{enumerate}[label=(\roman*)]
    \item $F$ is a field (sometimes local, somtimes a global field), $E$ is a quadratic extension of $F$ with Galois involution $z\mapsto \bar{z}$.
    \item $V$ is a 3-dimensional vector space over $E$, with basis $\{\ell_{-1}, \ell_{0}, \ell_{1}\}$. $(-,-)_V$ is a Hermitian form on $V$, with matrix
    \[
        \begin{bmatrix} 0 & 0& 1 \\ 0 & 1 & 0 \\ 1 & 0 & 0\end{bmatrix}
    \]
    with respect to $\{\ell_{-1}, \ell_{0}, \ell_{1}\}$.
    \item $G = \rU_{2, 1} =\rU(V)$ is the unitary group for the form $(-, -)_V$. $P$=parabolic subgroup stabilizing the isotropic line through $\ell_{-1}$ = $MN$ with
    \[
        M = \left\{\begin{bmatrix} \delta & 0 & 0 \\ 0 & \beta & 0 \\ 0 & 0 & \bar{\delta}^{-1} \end{bmatrix}: \delta \in E^\times, \beta \in E^{1} = \{z: z\bar{z} = 1\}\right\}
    \]
    and unipotent radical
    \[
        N = \left\{\begin{bmatrix} 1 & b& z \\ 0 & 1& -\bar{b} \\ 0 & 0 & 1\end{bmatrix}: z, b \in E, z + \bar{z} = -b\bar{b}\right\}.
    \]
    The center of $N$ is
    \[
        Z = \left\{\begin{bmatrix}
        1 & 0 & z \\ 0 & 1& 0 \\ 0 & 0 & 1          
        \end{bmatrix}: \bar{z} = -z\right\}
    \]
\end{enumerate}



\section{Whittaker Models (Ordinary and Generalized)}
\label{sec:1}


Some kind of Whittaker model is needed in order to introduce $L$-functions on $G$.

Fix $F$ local (not of characteristic two), and suppose $(\pi, H_\pi)$ is an irreducible admissible representation of $G$.
Naively, we should look for functionals on $H_\pi$ which·transform under $N$ according to a one-dimensional
representation.
However, since such functionals need not exist in general, and since there are irreducible representations of N which are not 1-dimensional, it is natural to pursue a more general approach.


\subsection{}

Recall $N$ is the maximal unipotent subgroup of $G$ and $E$ is a quadratic extension of $F$.
We fix, once and for all, an element $i$ in $E$ such that $\bar{i} = -i$, so $\Im(z) = (z - \bar{z}) / 2i$.
Regarding $E$ as a 2-dimensional \underline{symplectic} space over $F$ with skew-form $\langle z_1, z_2\rangle = \Im(z_1\bar{z_2})$ we have
\[
    N = \left\{\begin{bmatrix}1 & b& z \\ 0 & 1 & -\bar{b} \\ 0 & 0 & 1\end{bmatrix}: z, b \in E, z + \bar{z} = -b\bar{b}\right\} \simeq H(E),
\]
the \underline{Heisenberg group} attached to $E$ over $F$.
In particular, $N$ is non-abelian, with commutator subgroup
\[
    [N, N] =\left\{\begin{bmatrix} 1 & 0 & z \\ 0 & 1 & 0 \\ 0 & 0 & 1\end{bmatrix}\right\} = Z,
\]
the center of $N$.
The maximal abelian subgroup of $N$ is
\[
    N' = \left\{\begin{bmatrix} 1 & b& z \\ 0 & 1 & -b \\ 0 & 0 & 1\end{bmatrix} \in N: b \in F\right\}.
\]


\subsection{}
\label{sec:1.2}
The irreducible representations of the \underline{Heisenberg group}, and hence those of $N$, are well known:
\begin{enumerate}[label=(\roman*)]
    \item \underline{$\sigma$ is 1-dimensional.} 
    
    In this case, $\sigma$ must be trivial on 
    \[
        Z = [N, N]
    \]
    and define a character of $N/Z$. So
    \[
        N/Z \simeq \left\{\begin{bmatrix} 1 & a & 0 \\ 0 & 1 & -\bar{a} \\ 0 & 0 & 1 \end{bmatrix}\right\} \simeq E
    \]
    implies $\sigma$ corresponds to a character of $E$, i.e.
    \[
        \sigma = \psi_N\left(\begin{bmatrix}
            1 & a & z \\ 0 & 1 & -\bar{a} \\ 0 & 0 & 1
        \end{bmatrix}\right) = \psi(\Im a)
    \]
    with $\psi$ a character \underline{of $F$}.
    \item \underline{$\sigma$ is infinite-dimensional.}
    
    In this case (by the Stone-von Neumann uniqueness theorem), $\sigma$ is completely determined by its ``central'' character. In particular, if
    \[
        \sigma\left(\begin{bmatrix}
            1 & 0 & z \\ 0 & 1 & 0 \\ 0 & 0 & 1
        \end{bmatrix}\right) = \psi(\Im z) I
    \]
    for some (additive) character $\psi$ of $F$, then
    \[
        \sigma = \rho_\psi = \Ind_{N'}^{N} \psi_{N'},
    \]
    with $\psi_{N'}$ the character of (the maximal abelian subgroup) $N'$ obtained by trivially extending $\psi$ from $Z$ to $N'$.
\end{enumerate}


\subsection{}
\label{sec:1.3}


\underline{Definition}. By a (generalized) Whittaker functional for $(\pi, H_\pi)$ we understand $N$-map from $N_\pi$ to some irreducible representation of $(\sigma, L_\sigma)$ of $N$ (possibly infinite dimensional).


\subsection{}
\label{sec:1.4}


\underline{Remark}. The torus
\[
    T = \left\{\begin{bmatrix} \delta & 0 & 0 \\ 0 & 1& 0 \\ 0 & 0 & \bar{\delta}^{-1}\end{bmatrix}: \delta \in E^\times\right\}
\]
acts by conjugation on $N$, taking
\[
    \begin{bmatrix}
        1 & b& z \\ 0 & 1& -\bar{b} \\ 0 & 0 & 1
    \end{bmatrix} \quad \text{to} \quad
    \begin{bmatrix}
        1 & \delta b & \delta \bar{\delta} z \\ 0 & 1 & -\overline{\delta b} \\ 0 & 0 & 1
    \end{bmatrix}.
\]
So if $\psi_N$ denotes the 1-dimensional representation of $N$ corresponding to the fixed character of $F$ as in \ref{sec:1.2} (i), Pontrygin duality for $E \simeq N/Z$ implies that any other 1-dimensional representation is trivial or of the form
\[
    \psi_N^\delta (n) = \psi_N \left(\begin{bmatrix}
        \delta & & \\ & 1 & \\ & & \bar{\delta}^{-1}
    \end{bmatrix} n \begin{bmatrix}
        \delta & & \\ & 1 & \\ & & \bar{\delta}^{-1}
    \end{bmatrix}^{-1}\right)
\]
for some $\delta \in E^\times$.


\subsection{}
\label{sec:1.5}

If $\sigma$ is a one-dimensional representation of $N$ of the form $\psi_N$, a given irreducible admissible representation $(\pi, H_\pi)$ need \underline{not} possess a nontrivial $\psi_N$-Whittaker functional $\scL$.
However, if it does, then by \ref{sec:1.4} it possesses a $\sigma$-Whittaker functional for \underline{any} one-dimensional representation $\psi_N^\delta$, given by the formula
\[
    \scL^\delta(v) = \scL\left(\pi\left(\begin{bmatrix}
        \delta & & \\ & 1 & \\ & & \bar{\delta}^{-1}
    \end{bmatrix}\right)v\right), \quad v \in H_\pi.
\]
In this case, we call $(\pi, H_\pi)$ \underline{non-degenerate}.
By a well-known theorem of Shalika and Gelfand-Kazhdan (cf. [Sha1]), the space of such $\sigma$-Whittaker functionals is one-dimensional.
In particular, the corresponding Whittaker models
\[
    \cW(\pi, \psi) = \{ W(g) = \scL(\pi(g) v): v \in H_\pi \}
\]
are unique.


\subsection{}
\label{sec:1.6}
In general, $(\pi, H_\pi)$ is not non-degenerate, examples being provided by the Weil representations discussed in \S \ref{sec:6}.
Thus it is necessary to consider $\sigma$-Whittaker models for infinite dimensional $\sigma$ as well.
Such $\sigma$, however, are completely determined by their central character $\psi_Z$, so it is convenient to work with a slight thickening of $N$.
More precisely, consider the stabilizer $R$ in $P$ of the central character $\psi_Z$ of $Z$.
Because 
$\left[\begin{smallmatrix}
    \delta & & \\ & \beta & \\ & & \bar{\delta}^{-1}
\end{smallmatrix}\right]$
conjugates
$
\left[\begin{smallmatrix}
    1 & 0 & z \\ 0 & 1 & 0 \\ 0 & 0 & 1 
\end{smallmatrix}\right]
$
to
$
\left[\begin{smallmatrix}
    1 & 0 & \delta \bar{\delta}z \\ 0 & 1 & 0 \\ 0 & 0 & 1 
\end{smallmatrix}\right],
$
\[
R = \left\{ \begin{bmatrix} \delta & * & * \\ 0 & \beta & * \\ 0 & 0 & \delta\end{bmatrix} \in P: \delta, \beta \in E^1\right\} \simeq (E^1 \times E^1) \ltimes N.
\]
In particular, each irreducible infinite dimensional representation $\rho_\psi$ of $N$ extends to a like representation $\rho_\psi^\alpha$ of $R$ with $\alpha$ a character of $E^1 \times E^1$.

\begin{theorem*}[Existence and Uniqueness of Generalized Whittaker Models: [PS3]]
Any $(\pi, H_\pi)$ possesses a $\rho_\psi^\alpha$-Whittaker functional for some choice of $\rho_\psi^\alpha$; moreover, the space of such functionals is at most one dimensional.
\end{theorem*}

We shall discuss this result in more detail in the global context of \S \ref{sec:7}.
\section{Some Fourier Expansions and Hypercuspidality}
\label{sec:2}


Now $F$ is a global field not of characteristic 2, and $\pi$ is an automorphic cuspidal representation of $G(\bA)$ which we suppose realized in some subspace of cusp forms $H_\pi$ in $L_0^2(G(F) \backslash G(\bA))$.
To attach an $L$-function to $\pi$, it is useful to take forms $f$ in $H_\pi$ and examine their Fourier coefficients along the maximal unipotent subgroup $N$.
When such coefficients are non-zero, $\pi$ is non-degenerate, and we are led back to the local Whittaker models of \ref{sec:1.5}; in this case, we can (and eventually do) introduce $L$-functions using Jacquet's generalization of the ``Rankin--Selberg method''.


On the other hand, if these Fourier coefficients represent zero, then $\pi$ is \underline{hypercuspidal}; in this case, looking at Fourier expansions \underline{along $Z$} will bring us back to the generalized Whittaker models of \ref{sec:1.6}, and ultimately allow us to introduce an $L$-function for $\pi$ using the so-called ``Shimura method''.


Henceforth, let us fix a non-trivial character $\psi$ of $F \backslash \bA$, and define characters $\psi_N$ and $\psi_Z$ of $N = N(\bA)$ and $Z = Z(\bA)$ by
\[
    \psi_N \left(\begin{bmatrix}
        1 & a & z \\ 0 & 1& -\bar{a} \\ 0 &0 & 1
    \end{bmatrix}\right) = \psi(\Im a)
\]
and
\[
    \psi_Z \left(\begin{bmatrix}
        1 & 0 & z \\ 0 & 1& 0 \\ 0 & 0 & 1
    \end{bmatrix}\right) = \psi(\Im z).
\]


\subsection{}
\label{sec:2.1}


Fix $f$ in $H_\pi$.
To obatin a Fourier expansion of $f$ ``along $N$'', we introduce the familiar $\psi$-th coefficient
\[
    W_f^\psi(g) = \int_{N(F) \backslash N(\bA)} f(ng) \overline{\psi_N(n)} \dd n.
\]
The transitivity of $T(\bA) = \left\{\left[\begin{smallmatrix}\delta & & \\ & 1 & \\ & & \bar\delta^{-1}\end{smallmatrix}\right]\right\}$ acting on $Z(\bA) \backslash N(\bA)$ implies - as in the local theory - that
\begin{align*}
    W_f^{\psi^\delta}(g) &= \int_{N(F) \backslash N(\bA)} f(ng) \overline{\psi_N^\delta(n)} \dd n \\
    &= \int_{N(F) \backslash N(\bA)} f(ng) \psi_N \left(\begin{bmatrix}
        \delta & & \\ & 1 & \\ & & \bar{\delta}^{-1}
    \end{bmatrix} n\begin{bmatrix}
        \delta & & \\ & 1 & \\ & & \bar{\delta}^{-1}
    \end{bmatrix}^{-1}\right) \dd n \\
    &= W_f^{\psi} \left(\begin{bmatrix}
        \delta & & \\ & 1&  \\ & & \bar{\delta}^{-1}
    \end{bmatrix}g\right).
\end{align*}
In other words, knowing $W_f^\psi$ determines $W_f^{\psi^\delta}$ for all $\psi^\delta, \delta \in E^\times$.

However, through $N(F) \backslash N(\bA)$ is compact, it is \underline{not} abelian; to obtain a nice Fourier expansion, we must bring into play the compact abelian group $N(F) Z(\bA) \backslash N(\bA)$.


\subsection{}
\label{sec:2.2}

We compute

\begin{align*}
    W_f^\psi(g) &= \int_{N(F) \backslash N(\bA)} f(ng) \overline{\psi_N(n)} \dd n \\
    &= \int_{N(F) Z(\bA) \backslash N(\bA)} \int_{Z(F) \backslash Z(\bA)} f(nzg) \dd z \overline{\psi_N(n)}\dd n \\
    &= \int_{N(F) Z(\bA) \backslash N(\bA)} f_{00}(ng) \overline{\psi_N(n)} \dd n
\end{align*}
with
\begin{equation}
\label{eqn:2.2.1}
    f_{00}(g) = \int_{Z(F) \backslash Z(\bA)} f(zg) \dd z
\end{equation}
the \underline{constant term} (in the Fourier expansion) of $f(zg)$ \underline{along $Z$}.


Fix $g$ in $G(\bA)$.
As a function on the \underline{compact abelian} group $N(F) Z(\bA) \backslash N(\bA)$, $f_{00}(ng)$ has a Fourier expansion
\begin{equation}
    \label{eqn:2.2.2}
    f_{00}(g) = \sum_{\delta \in E^\times} W_{F}^{\psi^\delta}(g) + \int_{N(F) Z(\bA) \backslash N(\bA)} f_{00}(n'g) \dd n'.
\end{equation}
Indeed, the last paragraph says precisely that $W_f^\psi(g)$ is the $\psi$-th Fourier coefficient of $f_{00}(ng)$ along $Z \backslash N \simeq E$. Moreover, the constant term is actually zero since $f$ cuspidal implies
\begin{align*}
    \int_{N(F)Z(\bA) \backslash N(\bA)} f_{00}(n'g) \dd n' &= \int_{N(F)Z(\bA) \backslash N(\bA)} \int_{Z(F) \backslash Z(\bA)} f(zn'g) \dd z \dd n' \\
    &= \int_{N(F) \backslash N(\bA)} f(ng) \dd n = 0.
\end{align*}


\subsection{}
\label{sec:2.3}

Let $\cW(\pi, \psi)$ denote the space of $\psi$-th Fourier coefficients $W_f^\psi(g)$, $f \in H_\pi$.

\begin{proposition}
The vanishing or nonvanishing of $\cW(\pi, \psi)$ is independent of $\psi$; in particular, $\cW(\pi, \psi) = 0$ if and only if
\[
    f_{00}(g) = 0 \quad \forall f \in H_\pi.
\]
\end{proposition}
\begin{proof}
According to \eqref{eqn:2.2.1} and \eqref{eqn:2.2.2},
\begin{equation}
\begin{split}
\label{eqn:2.3.1}
    f_{00}(g) &= \sum_{\delta \in E^\times} W_{f}^{\psi^\delta}(g) \\
    &= \sum_{\delta \in E^\times} W_{f}^{\psi} \left(\begin{bmatrix}
        \delta & & \\ & 1 & \\ & & \bar{\delta}^{-1}
    \end{bmatrix} g\right)
\end{split}
\end{equation}
with
\[
    W_f^\psi(g) = \int_{N(F)Z(\bA) \backslash N(\bA)} f_{00}(ng) \overline{\psi_N(n)} \dd n.
\]
\end{proof}


\subsection{}
\label{sec:2.4}

\begin{definition}
We call $(\pi, H_\pi)$ \underline{hypercuspidal} if $f \in H_\pi$ implies $f_{00} = 0$.
\end{definition}

\begin{proposition}
Let $L_{0, 1}^{2}$ be the orthogonal complement in $L_{0}^{2}$ of all cusp forms. Then    
\begin{enumerate}[label=(\roman*)]
    \item $L_{0, 1}^{2}$ has \underline{multiplicity 1}.
    \item each $(\pi, H_\pi)\subset L_{0, 1}^{2}$ is non-degenerate, and
    \item for any $f \in H_\pi \subset L_{0, 1}^{2}$, the constant term
    \[
        f_{00}(g) = \sum_{\delta \in E^\times}W_{f}^{\psi^\delta}(g)
    \]
    completely determines $f$.
\end{enumerate}
\end{proposition}

\begin{proof}
We start with (iii).
Suppose $f$ and $f'$ are in $H_{\pi}$ such that $f_{00} = f_{00}'$.
Then $(f - f')_{00} = 0$ implies $f - f' = 0$ (by the hypothesis $H_\pi \in L_{0, 1}^{2}$). This proves (iii).
To prove (i) and (ii), suppose there exists $H_\pi' \subset L_{0, 1}^{2}$ such that the right regular representation restricted to $H_{\pi}'$ again realizes $\pi$.
If $f \in H_\pi$ and $f' \in H_{\pi}'$, then
\begin{equation}
\label{eqn:2.4.1}
    f_{00}(g) = \sum_{\delta \in E^\times} W_f^\psi\left(\begin{bmatrix}
        \delta & & \\ & 1 & \\ & & \bar{\delta}^{-1}
    \end{bmatrix}g\right)
\end{equation}
and
\[
    f'_{00}(g) = \sum_{\delta \in E^\times} W_{f'}^\psi\left(\begin{bmatrix}
        \delta & & \\ & 1 & \\ & & \bar{\delta}^{-1}
    \end{bmatrix}g\right).
\]

Note that each $W_f^\psi$ (or $W_{f'}^\psi$) satisfies the condition $W_f^\psi(ng) = \psi(n) W_f^\psi(g)$, $n \in N$, i.e. the spaces $(W_f^\psi)$ and $W_{f'}^\psi$ afford Whittaker models for $\pi$.
But by \S \ref{sec:2.3} these spaces are nonzero (which proves (ii)) and by the uniqueness of Whittaker models quoted in \S \ref{sec:1.5}, these spaces coincide.
Thus by \eqref{eqn:2.4.1}, the spaces $(f_{00})$ and $(f_{00}')$ coincide; by (iii) the spaces $H_\pi = (f)$ and $H_{\pi}' = (f')$ also coincide, thereby proving (i).
\end{proof}


\subsection{}
\label{sec:2.5}

\begin{remark*}
\begin{enumerate}[label=(\roman*)]
    \item It is conjectured (c.f. [Flicker]) that multiplicity one holds for the entire space of cusp forms; however, at the present time, we can prove this only for $L_{0, 1}^{2}$.
    \item Hypercuspforms \underline{do} exist; again, the examples are provided by the Weil representation discussed in \S \ref{sec:6}.
    \item Although $\cW(\pi, \psi) \neq \{0\}$ implies $\pi$ non-degenerate (in the sense that an abstract functional exists), the converse is not clear.
    Indeed, the work of [Wald] indicates that characterizing the nonvanishing of a space of Fourier coefficients is a delicate matter.
\end{enumerate} 
\end{remark*}

\section{$L$-functions \`a la Rankin--Selberg--Jacquet}
\label{sec:3}


We are now ready to attach (global) $L$-functions to (non-degenerate) cuspidal representations $\pi$ of $G(\bA)$, The method used goes back to [Rankin] and [Selberg] who used it to analytically continue the convolution of Dirichlet series corresponding to classical holomorphic modular forms.
The reformulation of their construction in the language of representation theory was carried out in detail by [Jacquet], for $\GL_2 \times \GL_2$, and by [PS 1] in general (but without details or exp1icit computation).
In this Section (and the next), we carry out the construction for $G = \rU_{2, 1}$.


\subsection{}
\label{sec:3.1}


Recall $V$ is a 3-dimensional vector space over $E/F$ and $(-,-)_V$ is a hermitian form on $V$ whose matrix with respect to the basis $\{\ell_{-1}, \ell_{0}, \ell_{1}\}$ is $\left[\begin{smallmatrix}
    0 & 0 & 1 \\ 0 & 1& 0 \\ 1 & 0 & 0
\end{smallmatrix}\right]$.


\section{Local $L$-functions (non-degenerate $\pi$)}
\label{sec:4}


\subsection{}
\label{sec:4.1}


Suppose $E$ (quadratic) over $F$ is local and $\pi$ is an irreducible admissible representation of $G(F)$.
Because of Proposition \ref{prop:3.6}, it is natural to define local zeta-integrals (of Rankin--Selberg--Jacquet type) by
\[
    L^\mu(W, F_\Phi, s) = \int_{Z\backslash H} W(h) F(h) \d dh
\]
where $W \in \cW(\pi, \psi_N)$, $\mu$ is a character of $E^\times$, $\Phi$ is a Schwartz--Bruhat function on the two-dimensional $E$-vector spacae $W = \langle\ell_{-1}, \ell_{1}\rangle$, and
\[
    F^\mu(h) = \int_{E^\times} (h\cdot\Phi)(t \ell_{-1}) \mu(t) |t|_E^s \dd^\times t.
\]
Note
\[
    F(bh) = \mu(\alpha) |\alpha|_E^s F(h)
\]
for all $h \in H$, and
\[
    b = \begin{bmatrix}
        \alpha & 0 & \beta \\ 0 & 1& 0 \\ 0 & 0 & \bar{\alpha}^{-1}
    \end{bmatrix} \in B.
\]


\begin{remark*}
    There are other kinds of local integrals to consider, namely those which arise from the splitting primes for $E$; cf. \S \ref{sec:3.7}.
    Since these zeta integrals involve the more familiar groups $\GL_3$ and $\GL_2$, we shall concentrate on the unitary integrals instead (dealing with the splitting primes only parenthetically).
\end{remark*}


\subsection{}
\label{sec:4.2}


Suitably moddifying the problem in \S 14 of [Jacquet] on can obtain the following:
\begin{proposition}
    For each $W, \Phi$, and $\mu$ as above, the local zeta-integrals $L^\mu(W, F_\Phi, s)$ converge for $\Re(s) \gg 0$ and define rational functions of $q^{-s}$ satisfying the following conditions:
    \begin{enumerate}[label=(\roman*)]
        \item The subspace of $\bC(q^{-s})$ spanned by $L^\mu(W, F_\Phi, s)$ is in fact a fractional ideal of the ring $\bC[q^{-s}, q^s]$ generated by some polynomial $Q_0$ in $\bC[q^{-s}]$ which is Independent of $W$ and $\Phi$;
        \item There is a rational function of $q^{-s}$, denoted $\gamma(s)$, and a ``contragradient'' $L$-function $\tilde{L}^{\mu^{-1}}(W, F_{\widehat{\Phi}}, s)$, such that for all $W$ and $\Phi$ (and $\widehat{\Phi}$ a special kind of ``partial Fourier transform''),
        \[
            \tilde{L}^{\mu^{-1}}(W, F_{\widehat{\Phi}}, 1 -s) = \gamma(s) L^\mu(W, F_{\Phi}, s).
        \]
    \end{enumerate}
\end{proposition}


\nocite{*}

% --- Bibliography ---

% Start a bibliography with one item.
% Citation example: "\cite{williams}".

\bibliographystyle{acm} % We choose the "plain" reference style
\bibliography{refs} % Entries are in the refs.bib file


% \begin{thebibliography}{1}

% \bibitem{williams}
%    Williams, David.
%    \textit{Probability with Martingales}.
%    Cambridge University Press, 1991.
%    Print.

% % Uncomment the following lines to include a webpage
% % \bibitem{webpage1}
% %   LastName, FirstName. ``Webpage Title''.
% %   WebsiteName, OrganizationName.
% %   Online; accessed Month Date, Year.\\
% %   \texttt{www.URLhere.com}

% \end{thebibliography}

% --- Document ends here ---

\end{document}