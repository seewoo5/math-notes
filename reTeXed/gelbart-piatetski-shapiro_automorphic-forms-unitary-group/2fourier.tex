\section{Some Fourier Expansions and Hypercuspidality}
\label{sec:2}


Now $F$ is a global field not of characteristic 2, and $\pi$ is an automorphic cuspidal representation of $G(\bA)$ which we suppose realized in some subspace of cusp forms $H_\pi$ in $L_0^2(G(F) \backslash G(\bA))$.
To attach an $L$-function to $\pi$, it is useful to take forms $f$ in $H_\pi$ and examine their Fourier coefficients along the maximal unipotent subgroup $N$.
When such coefficients are non-zero, $\pi$ is non-degenerate, and we are led back to the local Whittaker models of \ref{sec:1.5}; in this case, we can (and eventually do) introduce $L$-functions using Jacquet's generalization of the ``Rankin--Selberg method''.


On the other hand, if these Fourier coefficients represent zero, then $\pi$ is \underline{hypercuspidal}; in this case, looking at Fourier expansions \underline{along $Z$} will bring us back to the generalized Whittaker models of \ref{sec:1.6}, and ultimately allow us to introduce an $L$-function for $\pi$ using the so-called ``Shimura method''.


Henceforth, let us fix a non-trivial character $\psi$ of $F \backslash \bA$, and define characters $\psi_N$ and $\psi_Z$ of $N = N(\bA)$ and $Z = Z(\bA)$ by
\[
    \psi_N \left(\begin{bmatrix}
        1 & a & z \\ 0 & 1& -\bar{a} \\ 0 &0 & 1
    \end{bmatrix}\right) = \psi(\Im a)
\]
and
\[
    \psi_Z \left(\begin{bmatrix}
        1 & 0 & z \\ 0 & 1& 0 \\ 0 & 0 & 1
    \end{bmatrix}\right) = \psi(\Im z).
\]


\subsection{}
\label{sec:2.1}


Fix $f$ in $H_\pi$.
To obatin a Fourier expansion of $f$ ``along $N$'', we introduce the familiar $\psi$-th coefficient
\[
    W_f^\psi(g) = \int_{N(F) \backslash N(\bA)} f(ng) \overline{\psi_N(n)} \dd n.
\]
The transitivity of $T(\bA) = \left\{\left[\begin{smallmatrix}\delta & & \\ & 1 & \\ & & \bar\delta^{-1}\end{smallmatrix}\right]\right\}$ acting on $Z(\bA) \backslash N(\bA)$ implies - as in the local theory - that
\begin{align*}
    W_f^{\psi^\delta}(g) &= \int_{N(F) \backslash N(\bA)} f(ng) \overline{\psi_N^\delta(n)} \dd n \\
    &= \int_{N(F) \backslash N(\bA)} f(ng) \psi_N \left(\begin{bmatrix}
        \delta & & \\ & 1 & \\ & & \bar{\delta}^{-1}
    \end{bmatrix} n\begin{bmatrix}
        \delta & & \\ & 1 & \\ & & \bar{\delta}^{-1}
    \end{bmatrix}^{-1}\right) \dd n \\
    &= W_f^{\psi} \left(\begin{bmatrix}
        \delta & & \\ & 1&  \\ & & \bar{\delta}^{-1}
    \end{bmatrix}g\right).
\end{align*}
In other words, knowing $W_f^\psi$ determines $W_f^{\psi^\delta}$ for all $\psi^\delta, \delta \in E^\times$.

However, through $N(F) \backslash N(\bA)$ is compact, it is \underline{not} abelian; to obtain a nice Fourier expansion, we must bring into play the compact abelian group $N(F) Z(\bA) \backslash N(\bA)$.


\subsection{}
\label{sec:2.2}

We compute

\begin{align*}
    W_f^\psi(g) &= \int_{N(F) \backslash N(\bA)} f(ng) \overline{\psi_N(n)} \dd n \\
    &= \int_{N(F) Z(\bA) \backslash N(\bA)} \int_{Z(F) \backslash Z(\bA)} f(nzg) \dd z \overline{\psi_N(n)}\dd n \\
    &= \int_{N(F) Z(\bA) \backslash N(\bA)} f_{00}(ng) \overline{\psi_N(n)} \dd n
\end{align*}
with
\begin{equation}
\label{eqn:2.2.1}
    f_{00}(g) = \int_{Z(F) \backslash Z(\bA)} f(zg) \dd z
\end{equation}
the \underline{constant term} (in the Fourier expansion) of $f(zg)$ \underline{along $Z$}.


Fix $g$ in $G(\bA)$.
As a function on the \underline{compact abelian} group $N(F) Z(\bA) \backslash N(\bA)$, $f_{00}(ng)$ has a Fourier expansion
\begin{equation}
    \label{eqn:2.2.2}
    f_{00}(g) = \sum_{\delta \in E^\times} W_{F}^{\psi^\delta}(g) + \int_{N(F) Z(\bA) \backslash N(\bA)} f_{00}(n'g) \dd n'.
\end{equation}
Indeed, the last paragraph says precisely that $W_f^\psi(g)$ is the $\psi$-th Fourier coefficient of $f_{00}(ng)$ along $Z \backslash N \simeq E$. Moreover, the constant term is actually zero since $f$ cuspidal implies
\begin{align*}
    \int_{N(F)Z(\bA) \backslash N(\bA)} f_{00}(n'g) \dd n' &= \int_{N(F)Z(\bA) \backslash N(\bA)} \int_{Z(F) \backslash Z(\bA)} f(zn'g) \dd z \dd n' \\
    &= \int_{N(F) \backslash N(\bA)} f(ng) \dd n = 0.
\end{align*}


\subsection{}
\label{sec:2.3}

Let $\cW(\pi, \psi)$ denote the space of $\psi$-th Fourier coefficients $W_f^\psi(g)$, $f \in H_\pi$.

\begin{proposition}
The vanishing or nonvanishing of $\cW(\pi, \psi)$ is independent of $\psi$; in particular, $\cW(\pi, \psi) = 0$ if and only if
\[
    f_{00}(g) = 0 \quad \forall f \in H_\pi.
\]
\end{proposition}
\begin{proof}
According to \eqref{eqn:2.2.1} and \eqref{eqn:2.2.2},
\begin{equation}
\begin{split}
\label{eqn:2.3.1}
    f_{00}(g) &= \sum_{\delta \in E^\times} W_{f}^{\psi^\delta}(g) \\
    &= \sum_{\delta \in E^\times} W_{f}^{\psi} \left(\begin{bmatrix}
        \delta & & \\ & 1 & \\ & & \bar{\delta}^{-1}
    \end{bmatrix} g\right)
\end{split}
\end{equation}
with
\[
    W_f^\psi(g) = \int_{N(F)Z(\bA) \backslash N(\bA)} f_{00}(ng) \overline{\psi_N(n)} \dd n.
\]
\end{proof}


\subsection{}
\label{sec:2.4}

\begin{definition}
We call $(\pi, H_\pi)$ \underline{hypercuspidal} if $f \in H_\pi$ implies $f_{00} = 0$.
\end{definition}

\begin{proposition}
Let $L_{0, 1}^{2}$ be the orthogonal complement in $L_{0}^{2}$ of all cusp forms. Then    
\begin{enumerate}[label=(\roman*)]
    \item $L_{0, 1}^{2}$ has \underline{multiplicity 1}.
    \item each $(\pi, H_\pi)\subset L_{0, 1}^{2}$ is non-degenerate, and
    \item for any $f \in H_\pi \subset L_{0, 1}^{2}$, the constant term
    \[
        f_{00}(g) = \sum_{\delta \in E^\times}W_{f}^{\psi^\delta}(g)
    \]
    completely determines $f$.
\end{enumerate}
\end{proposition}

\begin{proof}
We start with (iii).
Suppose $f$ and $f'$ are in $H_{\pi}$ such that $f_{00} = f_{00}'$.
Then $(f - f')_{00} = 0$ implies $f - f' = 0$ (by the hypothesis $H_\pi \in L_{0, 1}^{2}$). This proves (iii).
To prove (i) and (ii), suppose there exists $H_\pi' \subset L_{0, 1}^{2}$ such that the right regular representation restricted to $H_{\pi}'$ again realizes $\pi$.
If $f \in H_\pi$ and $f' \in H_{\pi}'$, then
\begin{equation}
\label{eqn:2.4.1}
    f_{00}(g) = \sum_{\delta \in E^\times} W_f^\psi\left(\begin{bmatrix}
        \delta & & \\ & 1 & \\ & & \bar{\delta}^{-1}
    \end{bmatrix}g\right)
\end{equation}
and
\[
    f'_{00}(g) = \sum_{\delta \in E^\times} W_{f'}^\psi\left(\begin{bmatrix}
        \delta & & \\ & 1 & \\ & & \bar{\delta}^{-1}
    \end{bmatrix}g\right).
\]

Note that each $W_f^\psi$ (or $W_{f'}^\psi$) satisfies the condition $W_f^\psi(ng) = \psi(n) W_f^\psi(g)$, $n \in N$, i.e. the spaces $(W_f^\psi)$ and $W_{f'}^\psi$ afford Whittaker models for $\pi$.
But by \S \ref{sec:2.3} these spaces are nonzero (which proves (ii)) and by the uniqueness of Whittaker models quoted in \S \ref{sec:1.5}, these spaces coincide.
Thus by \eqref{eqn:2.4.1}, the spaces $(f_{00})$ and $(f_{00}')$ coincide; by (iii) the spaces $H_\pi = (f)$ and $H_{\pi}' = (f')$ also coincide, thereby proving (i).
\end{proof}


\subsection{}
\label{sec:2.5}

\begin{remark*}
\begin{enumerate}[label=(\roman*)]
    \item It is conjectured (c.f. [Flicker]) that multiplicity one holds for the entire space of cusp forms; however, at the present time, we can prove this only for $L_{0, 1}^{2}$.
    \item Hypercuspforms \underline{do} exist; again, the examples are provided by the Weil representation discussed in \S \ref{sec:6}.
    \item Although $\cW(\pi, \psi) \neq \{0\}$ implies $\pi$ non-degenerate (in the sense that an abstract functional exists), the converse is not clear.
    Indeed, the work of [Wald] indicates that characterizing the nonvanishing of a space of Fourier coefficients is a delicate matter.
\end{enumerate} 
\end{remark*}
