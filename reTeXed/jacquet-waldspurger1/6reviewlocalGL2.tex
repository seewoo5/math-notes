\section{Review on local representations of $\GL(2)$}

\subsection{}
Let $F$ be a local field and $E$ a quadratic extension of $E$.
We will consider again the set $X$ which is reduced to two elements $(G_1', T_1')$ and $(G_2', T_2')$, with say $G_1'$ splits.
It will be convenient to use the following result:
\begin{proposition}
Let $\pi'$ be an irreducible unitary representation of $G_i' / Z_i'$.
Then the dimension of the space of continuous and $T_i'$-invariant linear functionals on the space of smooth vectors of $\pi'$ is at most one.
Moreover such a functional is given by the inner product with a smooth $T_i'$-invariant vector.
\end{proposition}
If $F$ is non-archimedean then the assertion on the dimension is proven in~\cite{waldspurger1991correspondances}, Proposition 9.
It is well-known for $F = \mathbb{R}$.
The rest of the proposition is obvious.

\subsection{}
Likewise:
\begin{proposition}
Let $\pi'$ be an infinite-dimensional irreducible unitary representation of $G_1' / Z_1'$.
Then the dimension of the space of $T$-invariant linear functionals (resp.\ invariant under the character $\eta \circ \det$) on the space of smooth functions of $\pi'$ is one.
\end{proposition}
These are the propositions 9 and 10 of \cite{waldspurger1980correspondances}.

\subsection{}
For $i = 1, 2$, consider irreducible unitary representations $\pi_i'$ of $G_i' / Z_i'$.
Assume that the tuple $(\pi_1', \pi_2')$ satisfies the conditions of the theorem (15.1) of~\cite{jacquet2006automorphic}; in particular $\pi_1'$ is a discrete series representation.

\begin{proposition}
The representations $\pi_i'$ cannot both have a nonzero invariant vector under the group $T_i'$.
\end{proposition}
If $F$ is non-archimedean then our assertion is found in Theorem 2 of~\cite{waldspurger1991correspondances}.
It is well-known for $F = \mathbb{R}$.