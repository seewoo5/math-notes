\section{Double cosets}

\subsection{}
In this paragraph $F$ will be any field of characteristic zero and $E$ be a quadratic extension of $F$.
We will denote $N(E:F)$ or simply $N$ the subgroup of norms of $E$ in the multiplicative group of $F$.
The set $X(E:F)$ or simply $X$ is defined as before.
Consider one of its elements $(G', T')$.
There exists a quaternion algebra $H$ over $F$ and a subfield $L$ of $H$ isomorphic to $E$ such that $G$ is the multiplicative group of $H$ and $T$ is that of $L$.
We are going to parametrize the double cosets $T' \backslash G' / T'$.
For this purpose let's choose an element $\varepsilon$ of the normalizer $N(T')$ of $T'$ which is not in $T'$.
Then every $h$ in $H$ can be uniquely written as:
\begin{equation}
h = h_1 + \varepsilon h_2, \qquad h_i \in L.
\end{equation}
On the other hand, if $z \to \bar{z}$ denotes the unique non-trivial $F$-automorphism of $L$ then:
\begin{equation}
\varepsilon z \varepsilon^{-1} = \bar{z}.
\end{equation}
The square $c = \varepsilon^2$ is in $Z'$, or, in other words, in $F$.
Moreover the class of $c$ modulo $N$ is determined by the isomorphism class of the pair $(G', T')$ and, conversely, determines it.

Define two involutions $j^+$ and $j^-$ of $H$ by the following fomulae:
\begin{equation}
j^{\pm}(h) = \bar{h}_1 \pm \varepsilon h_2, \quad h = h_1 + \varepsilon h_2.
\end{equation}
It is easy to verify that these are the only involutions of $H$ which induce the unique non-trivial $F$-automorphism of $L$.
For $h$ in $G$ we will set\footnote{$\tr$ is a trace map from $H$ to $F$, given as $\tr(h_1 + \varepsilon h_2) = h_1 + \bar{h}_1$.}
\begin{equation}
\label{eqn:1.1.4}
X'(h) = \frac{\frac{1}{2}\tr(hj^+(h))}{\frac{1}{2}\tr(hj^{-}(h))}.
\end{equation}
As the denominator of this faction is nothing but the reduced norm of $h$, $X(h)$ is a well-defined element of $F$ only depends on the double coset of $h$ modulo $T'$.
We also introduce the function $P'(h:T')$ or simply $P'(h)$ defined by
\begin{equation}
X'(h) = \frac{1 + P'(h)}{1 - P'(h)}
\end{equation}
or
\begin{equation}
P'(h) = ch_{2}\bar{h}_2{(h_{1}\bar{h}_1)}^{-1}, \qquad c = \varepsilon^2.
\end{equation}
Thus $P'$ is a function \textcolor{red}{with values in the projective line} which is constant on the double cosets of $T'$ in $G'$.
Note that according to the previous formula, if $P'(h)$ is neither zero nor infinite, then it is an element of the class $cN'$ determined by the tuple $(G', T')$.
Moreover $P'(h)$ can't be 1, otherwise $X'(h)$ would be infinity.
We will say that $h$ (or its double coset) is $T'$-singular if $P'(h)$ is zero or infinity, $T'$-regular otherwise.

\begin{proposition}
Two elements $h$ and $h'$ in $G'$ are in the same double coset of $T'$ if and only if $P'(h) = P'(h')$.
Moreover, if $x$ is in $cN$ and not 1 then there exists a $h$ in $G'$ such that $P'(h) = x$.
\end{proposition}
The proof is left to the reader.

\subsection{}
The following proposition justifies the use of the adjective $T'$-regular.
\begin{proposition}
Suppose $h$ is $T'$-regular.    
The relations
\[
sht = hz,\quad s \in T',\quad t\in T', \quad z \in Z'
\]
imply
\[
s \in Z',\quad t\in Z',\quad st = z.
\]
\end{proposition}
The proof is left to the reader.

\subsection{}
The above applies \emph{mutatis mutandis} to a tuple of the form $(G, T)$ where $G$ is the group $\GL(2)$ and $T$ is a maximal split torus, say the group diagonal matrices in $G$.
Then $H$ is the algrebra of 2 by 2 matrices, $L$ the subalgebra of diagonal matrices and we can take\footnote{In this case, the non-trivial automorphism on $L$ is a map that swaps two diagonal elements, i.e. $\smat{a}{0}{0}{d}\mapsto \smat{d}{0}{0}{a}$.}
\[
\epsilon = \begin{bmatrix}
    0 & 1 \\ 1 & 0
\end{bmatrix},
\quad c = 1.
\]
The functions $X$ and $P(\cdot :T)$ (or simply $P$) are defined as above.
In particular:
\[
P(h)  = bc{(ad)}^{-1}, \quad h = \begin{bmatrix}
    a & b \\ c & d
\end{bmatrix}.
\]
They are constant on the double cosets of $T$ in $G$.
Again $P$ cannot be 1.
We will also say that an element $h$ of $G$ is $T$-singular if $P(h)$ is zero or infinity, $T$-regular otherwise.
There are now 6 $T$-singular double cosets: the cosets on which $P$ takes the value zero:
\begin{equation}
    T, \quad Tn_+T, \quad Tn_{-}T, \quad \text{where } n_{+} = \begin{bmatrix}
        1 & 1 \\ 0 & 1
    \end{bmatrix},  \begin{bmatrix}
        1 & 0 \\ 1 & 1
    \end{bmatrix},
\end{equation}
and the classes on which $P$ takes the infinite value
\begin{equation}
\varepsilon T, \quad T \varepsilon n_{+} T, \quad T\varepsilon n_{-} T.
\end{equation}
Thus these classes cannot be distinguished from each other with $P$.

\subsection{}
However, $P$ distinguishes $T$-regular cosets:
\begin{proposition}
Let $h$ and $h'$ be $T$-regular element of $G$.
Then $h$ and $h'$ are in the same coset if and only if $P(h) = P(h')$.
If $x \in F$ is neither 1 nor 0, there exists an $T$-regular element $h$ such that $P(h) = x$.
\end{proposition}
The proof is left to the reader.