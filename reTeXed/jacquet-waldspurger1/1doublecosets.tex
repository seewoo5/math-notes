\section{Double cosets}

\subsection{}

In this section, let $F$ be any field of characteristic zero, and let $E$ be a quadratic extension of $F$. Denote by $\rN(E:F)$, or simply $\rN$, the subgroup of norms from $E$ to the multiplicative group of $F$.
The set $X(E:F)$, or simply $X$, is defined as in the previous section.
Consider an element $(G', T')$ from this set.
There exists a quaternion algebra $H$ over $F$ and a subfield $L$ of $H$ isomorphic to $E$, such that $G'$ is the multiplicative group of $H$, and $T'$ is the group of units of $L$.
We now aim to parametrize the double cosets $T' \backslash G' / T'$.
To this end, choose an element $\varepsilon$ from the normalizer $N(T')$ of $T'$ that is not in $T'$.
Then, every element $h \in H$ can be uniquely written as:
\begin{equation}
    h = h_1 + \varepsilon h_2, \qquad h_i \in L.
\end{equation}
Moreover, if $\bar{z}$ denotes the non-trivial $F$-automorphism of $L$, then:
\begin{equation}
    \varepsilon z \varepsilon^{-1} = \bar{z}.
\end{equation} 
The square $c = \varepsilon^2$ lies in $Z'$, i.e., in $F$.
Furthermore, the class of $c$ modulo $\rN$ is determined by the isomorphism class of the pair $(G', T')$, and conversely, it determines this class.

Next, define two involutions $j^+$ and $j^-$ on $H$ by:
\begin{equation}
    j^{\pm}(h) = \bar{h}_1 \pm \varepsilon h_2, \quad h = h_1 + \varepsilon h_2.
\end{equation}
It is easy to verify that these are the only involutions of $H$ that induce the non-trivial $F$-automorphism of $L$.
For any $h \in G'$, we define\footnote{Here, $\tr$ is the trace map from $H$ to $F$, given by $\tr(h_1 + \varepsilon h_2) = h_1 + \bar{h}_1$.}
\begin{equation}
    \label{eqn:1.1.4}
    X'(h) = \frac{\frac{1}{2}\tr(hj^+(h))}{\frac{1}{2}\tr(hj^{-}(h))}.
\end{equation}
Since the denominator in this expression is the reduced norm of $h$, $X'(h)$ is a well-defined element of $F$, depending only on the double coset of $h$ modulo $T'$. We also introduce the function $P'(h:T)$, or simply $P'(h)$, by:
\begin{equation}
    X'(h) = \frac{1 + P'(h)}{1 - P'(h)},
\end{equation}
or equivalently,
\begin{equation}
    P'(h) = c h_2 \bar{h}_2 (h_1 \bar{h}_1)^{-1}, \qquad c = \varepsilon^2.
\end{equation}
Thus, $P'$ is a function valued in the projective line\footnote{$F \cup \{\infty\}$}, constant on the double cosets of $T'$ in $G'$.
According to the above formula, if $P'(h)$ is neither zero nor infinity, then it belongs to the class $c\rN'$ determined by the tuple $(G', T')$.
Additionally, $P'(h)$ cannot be equal to $1$, otherwise $X'(h)$ would be infinite.
We say that $h$ (or its double coset) is $T'$-singular if $P'(h)$ is either zero or infinity, and $T'$-regular otherwise.

\begin{proposition}
Two elements $h$ and $h'$ in $G'$ are in the same double coset of $T'$ if and only if $P'(h) = P'(h')$.
Moreover, if $x$ is in $c\rN$ and is not equal to $1$, then there exists an element $h \in G'$ such that $P'(h) = x$.
\end{proposition}
The proof is left to the reader.

\subsection{}
The following proposition justifies the use of the adjective $T'$-regular.
\begin{proposition}
Suppose $h$ is $T'$-regular.    
The relations
\[
sht = hz,\quad s \in T',\quad t\in T', \quad z \in Z'
\]
imply
\[
s \in Z',\quad t\in Z',\quad st = z.
\]
\end{proposition}
The proof is left to the reader.

\subsection{}
The above results apply \emph{mutatis mutandis} to a tuple of the form $(G, T)$ where $G$ is the group $\GL(2)$ and $T$ is a maximal split torus, say the group diagonal matrices in $G$.
In this case, $H$ is the algebra of $2 \times 2$ matrices, and $L$ is the subalgebra of diagonal matrices. We may take\footnote{Here, the non-trivial automorphism of $L$ swaps the two diagonal elements, i.e., $\smat{a}{0}{0}{d} \mapsto \smat{d}{0}{0}{a}$.}
\[
\epsilon = \begin{bmatrix}
    0 & 1 \\ 1 & 0
\end{bmatrix},
\quad c = 1.
\]
The functions $X$ and $P(\cdot :T)$ (or simply $P$) are defined analogously.
In particular:
\[
P(h)  = bc{(ad)}^{-1}, \quad h = \begin{bmatrix}
    a & b \\ c & d
\end{bmatrix}.
\]
These functions are constant on the double cosets of $T$ in $G$.
As before, $P(h)$ cannot be equal to $1$.
We will also say that an element $h \in G$ is $T$-singular if $P(h)$ is either zero or infinity, and $T$-regular otherwise.
There are six $T$-singular double cosets, corresponding to the cosets where $P(h)$ takes the value zero:
\begin{equation}
    T, \quad Tn_+T, \quad Tn_{-}T, \quad \text{where } n_{+} = \begin{bmatrix}
        1 & 1 \\ 0 & 1
    \end{bmatrix}, n_{-} = \begin{bmatrix}
        1 & 0 \\ 1 & 1
    \end{bmatrix},
\end{equation}
and the cosets where $P(h)$ takes the value infinity:
\begin{equation}
\varepsilon T, \quad T \varepsilon n_{+} T, \quad T\varepsilon n_{-} T.
\end{equation}
Thus, these singular cosets cannot be distinguished using the function $P$.

\subsection{}
However, $P$ distinguishes between $T$-regular cosets:
\begin{proposition}
Let $h$ and $h'$ be $T$-regular elements of $G$.
Then $h$ and $h'$ are in the same double coset if and only if $P(h) = P(h')$. Moreover, if $x \in F$ is neither 1 nor 0, there exists a $T$-regular element $h$ such that $P(h) = x$.
\end{proposition}
The proof is left to the reader.
