

\section{Introduction}
\subsection{}

We present a new proof of a remarkable result by Waldspurger~\cite[Theorem 2]{waldspurger1985valeurs}.
While Waldspurger's original proof relies on the properties of Weil's representation, our approach is based on a variant of the trace formula.
We believe that this alternative perspective offers some interest.

We begin by recalling the statement of the result.
Let $F$ be a number field, and $E$ a quadratic extension of $F$.
Denote by $\eta$ the character of the id\'ele class group of $F$ associated with $E$.
Consider the group $\GL(2)$ as an algebraic group $G$ defined over $F$, and let $Z$ be its center.
Let $T$ be a maximal torus in $G$, namely the group of diagonal matrices.
Let $\pi$ be an automorphic representation of $G(\Aa_F)$ that is trivial on the center $Z(\Aa_F)$.

We say that $\pi$ satisfies Waldspurger's first condition (denoted \textbf{W1}) if there exist automorphic forms $\phi_1$ and $\phi_2$ in the space of $\pi$ such that the following integrals are nonzero:
\begin{equation}
    \int_{T(\Aa_F) / Z(\Aa_F)} \phi_1(a) \dd a, \qquad \int_{T(\Aa_F) / Z(\Aa_F)} \phi_2(a) \eta(\det a) \dd a.
\end{equation}
Next, define the set $X(E:F)$, or simply $X$, as the set of isomorphism classes of pairs $(G', T')$, where $G'$ is an inner form of $G$ and $T'$ is a maximal torus in $G'$ isomorphic to $E^\times$ over $F$.
Such a pair arises from another pair $(H, L)$, where $H$ is a quaternion algebra over $F$ and $L$ is a subfield of $H$ isomorphic to $E$ over $F$, by taking $G'$ as the multiplicative group of $H$ and $T'$ as that of $L$.
We denote the center of $G'$ by $Z'$.

We identify the set $X$ with a set of representatives for each isomorphism class in $X$.
Let $X(\pi)$ be the set of triples $(G', T', \pi')$, where the pair $(G', T')$ lies in $X$ and $\pi'$ is a cuspidal automorphic representation of $G'(\Aa_F)$ associated with $\pi$ through the condition outlined in~\cite{jacquet2006automorphic}.
This condition can be stated as follows: there exists a finite set of places $S$ of $F$ such that, for $v \notin S$, the groups $G_v$ and $G_v'$ are isomorphic, and the representations $\pi_v$ and $\pi_v'$ are equivalent under this isomorphism.\footnote{Jacquet--Langlands correspondence.}
We say that $\pi$ satisfies Waldspurger's second condition (denoted \textbf{W2}) if there exists a triple $(G', T', \pi')$ in $X(\pi)$ and an automorphic form $\phi'$ in the space of $\pi'$ such that the following integral is nonzero:
\begin{equation}
    \int_{T'(\Aa_F) / Z'(\Aa_F)} \phi'(t) \dd t.
\end{equation}

We now state the result:
\begin{theorem}[Waldspurger] Conditions \textbf{W1} and \textbf{W2} are equivalent. \end{theorem}

\subsection{}
We now outline the main ideas behind our proof.
First, we identify the set of double cosets $T \backslash G / T$ with the disjoint union of the double cosets $T' \backslash G / T'$ (\S 1).
For this identification, we restrict to ``regular'' double cosets.
This motivates the introduction of a compactly supported smooth function $f$ on $G(\Aa_F) / Z(\Aa_F)$, and for each $(G', T')$, a corresponding compactly supported smooth function $f'$ on $G'(\Aa_F) / Z'(\Aa_F)$.
In fact, $f'$ will be zero for almost all $(G', T')$.
We associate a cuspidal kernel $K_c$ to $f$, and similarly, a cuspidal kernel $K_c'$ to each $f'$.
The relevant conditions imposed on these functions are (\S 7 and \S 10):
\begin{equation}
    \label{eqn:0.2.1}
    \int_{[T]} \int_{[T]} K_{c}(a, b) \eta(\det b) \dd b \dd a = \sum_{(G', T')} \int_{[T']} \int_{[T']} K_{c}'(s, t) \dd t \dd s.
\end{equation}

The relation between $f$ and $f'$ is as follows.
These functions are products of local functions.
If $v$ is a place of $F$ that splits in $E$, then for all $(G', T')$, the groups $G_v'$ and $G_v$ are isomorphic, and we define $f_v$ and $f_v'$ to be identical.
In this case, the set $X(E_v: F_v)$ consists of two elements $(G'_{v_i}, T'_{v_i})$, $i = 1, 2$, with $G'_{v_1}$ being split.
We can still identify the regular double cosets of $T_v$ with the disjoint union of the regular double cosets of $T'_1$ and $T'_2$.
We show that for a given function $f$, there are functions $f'_i$ on $G'_i$ such that:
\begin{equation*}
    \int_{T_v / Z_v} \int_{T_v / Z_v} f(agb) \eta_v(\det b) \dd b \dd a = \int_{T'_{v_i} / Z'_{v_i}} \int_{T'_{v_i} / Z'_{v_i}} f'_{i}(sg't) \dd t \dd s,
\end{equation*}
whenever $g$ corresponds to $g'$ (\S 2 and \S 4). If $v$ is unramified and $f_v$ is a Hecke function\footnote{The characteristic function of the spherical subgroup.}, then we can set $f'_1 = f_v$ and $f'_2 = 0$ (\S 5). The condition simplifies to $f'_v = f'_i$ if $G_v' = G_i'$. Waldspurger's result then follows easily from identity~\eqref{eqn:0.2.1}. Auxiliary results can be found in Section 6.

The proof of formula~\eqref{eqn:0.2.1} relies on a generalization of the trace formula, which can be stated as follows.
Let $G$ be a semisimple group defined over $F$, and let $A$ and $B$ be subgroups of $G$ defined over $F$.
Let $\lambda$ and $\mu$ be characters of $[A]$ and $[B]$, respectively.\footnote{i.e. the characters of $A(\bA_F)$ and $B(\bA_F)$ that are trivial on $A(F)$ and $B(F)$, respectively.}
Let $f$ be a compactly supported smooth function on $[G]$, and consider the integral:
\[
\int_{[A]} \int_{[B]} K_c(a, b) \lambda(a) \mu(b) \dd b \dd a
\]
where $K_c$ is the cuspidal kernel attached to $f$.
This kernel has a complex expression, which involves the sum:
\[
\sum_{\zeta \in G(F)} f(x^{-1} \zeta y).
\]
Choose a system of representatives for the double cosets of the groups $A(F)$ and $B(F)$.
For an element $\eta$  of $G(F)$ let $H_\eta$ the subgroup of $H = A\times B$ of tuples $(\alpha, \beta)$ such that $\alpha^{-1} \eta \beta = \eta$.
Then any element of $G(F)$ can be uniquely written in the form:
\[
\zeta = \alpha^{-1} \eta \beta, \qquad \eta \in A(F) \backslash G(F) / B(F), \quad (\alpha, \beta) \in H_{\eta}(F) \backslash (A(F) \times B(F)).
\]
By formal computation, we arrive at the following expression for the integral:
\[
\sum_{\eta} \mathrm{vol}(H_{\eta}(F) \backslash H(\Aa_F)) \int_{[A]} \int_{[B]} f(a^{-1} \eta b) \lambda(a) \mu(b) \dd b \dd a
\]
where sum on the left hand side is over all $\eta$ such that $\lambda(a) \mu(b) = 1$ if $a^{-1} \eta b = 1$.
Note that we have ignored convergence issues and other terms in the expression of $K_c$.

\subsection{}
I extend my sincere thanks to the Institute for Advanced Study and its permanent members for their hospitality, as the majority of this work was completed during my stay there during the special year 1983-1984 on $L$-functions.
I am particularly grateful to Langlands for his interest in this work.
Lastly, I owe a great debt of gratitude to Piatetski-Shapiro, whose deep understanding of Waldspurger's work proved invaluable.
A conversation with him was, in fact, the starting point of this work.
