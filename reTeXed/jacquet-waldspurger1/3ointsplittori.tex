\section{Orbital integrals: split torus}

\subsection{}
In this section, $F$ is a local field, $E$ a quadratic extension, $\eta$ the quadratic character of $F^\times$ attached to $E$, $G$ the group $\GL(2)$ and $T$ the subgroup of diagonal matrices.
We give the Tamagawa measure on $F^\times$ and product measure of itself on ${(F^{\times})}^2$.
This induces a measure on $T$ and we give $T/Z$ the quotient measure.
For a compactly supported smooth function $f$ on $G/Z$ and $T$-regular element $g \in G$, we set
\begin{align}
    H(g:f:A) = H(g:f:1) = \int_{T/Z} \int_{T/Z} f(agb) dadb \\
    H(g:f:\eta) = \int_{T/Z} \int_{T/Z} f(agb) \eta(\det b) dadb.
\end{align}
The first integral depends only on $P(g:T)$ and we denote $H(x:f:T)$ or $H(x:f:1)$ for its value at a point $g$ such that $P(g:A) = x$.
We also set $H(1:f:T) = H(1:f:1) = 0$.
For $x$ in $F$ neither 0 nor 1,   define a matrix $g(x)$ by
\begin{align}
    g(x) = \begin{bmatrix}
    1 & x \\ 1 & 1
    \end{bmatrix}.
\end{align}
Then $P(g(x)) = x$ so $g(x)$ defines a section of the space of double cosets of $A$ in $G$.
We set $H(x:f:\eta) = H(g(x):f:\eta)$ for $x \neq 0, 1$, and $H(x:f:\eta) = 0$ when $x = 1$.
Let
\begin{align}
    w = \begin{bmatrix}
        0 & -1 \\ 1 & 0        
    \end{bmatrix}.
\end{align}
Let $N_{+}$ be the group of strict upper triangular matrices and $N_{-}$ the group of strict lower triangular matrices.
Then we have a covering of $G$ by two open sets
\begin{align}
\label{3.1.5}
    G = TN_{+}N_{-} \cup TN_{+}wN_{+}.
\end{align}
We can therefore write $f$ as a sum $f_1 + f_2$ where $f_1$ is supported on the first open set and $f_2$ on the second.
Let
\begin{align}
    \phi(g) = \int_{T/Z} f(ag)da,
\end{align}
and define $\phi_1$ and $\phi_2$ in a similar way.
Then $\phi$ is left invariant under $T$ and compactly supported modulo $T$.
The same holds for $\phi_1$ and $\phi_2$.
Moreover, the functions $\Phi_1$ and $\Phi_2$ defined by 
\begin{align}
    \Phi_1(u, v) = \phi_1 \left(\begin{bmatrix} 1 & u \\ 0 & 1\end{bmatrix}\begin{bmatrix}1 & 0 \\ v& 1\end{bmatrix}\right), \\
    \Phi_1(u, v) = \phi_1 \left(\begin{bmatrix} 1 & u \\ 0 & 1\end{bmatrix}w\begin{bmatrix}1 & 0 \\ v& 1\end{bmatrix}\right)
\end{align}
are compactly supported on $F \times F$.
Since
\begin{equation}
\begin{aligned}
    g(x) \bmat{a}{0}{0}{1} &= \bmat{a(1-x)}{0}{0}{1} \bmat{1}{a^{-1}{(1-x)}^{-1}x}{0}{1} \bmat{1}{0}{a}{1} \\
    &= \bmat{1-x}{0}{0}{a} \bmat{1}{a{(1-x)}^{-1}}{0}{1} w \bmat{1}{a^{-1}}{0}{1},
\end{aligned}
\end{equation}
for $x \neq 0, 1$, we have

\begin{equation}
\begin{aligned}
\label{3.1.10}
    H(x:f:T) &= \int_{T/Z} \Phi_1(a^{-1}(1-x)^{-1}x, a) d^{x}a \\
    &+ \int_{T/Z} \Phi_1 (a(1-x)^{-1}, a^{-1}) d^{\times}a. 
\end{aligned}
\end{equation}

To prove the convergence of our integral, we can assume $f$ positive.
Then the RHS of~\eqref{3.1.10} are compactly supported in $F^\times$ and therefore the integral converges.
Similarly, for $x$ different from 0 and 1,

\begin{equation}
\begin{aligned}
\label{3.1.11}
    H(x:f:\eta) &= \int_{T/Z} \Phi_1(a^{-1}(1-x)^{-1}x, a) \eta(a) d^\times a \\
    &+ \int_{T/Z} \Phi_2(a(1-x)^{-1}, a^{-1}) \eta(a) d^{\times} a.
\end{aligned}
\end{equation}

\subsection{}
We are going to study the properties of the functions $H(f:\eta)$.
The equation~\eqref{3.1.11} already shows that $H(x:f:\eta)$ is a smooth function at any point $x$ not equal to 0 or 1.
On the other hand, if $\Phi_1$ and $\Phi_2$ are supported on the set of $(x, y)$ such that $|x| < C$ and $|y| < C$, then in the second integral we have, on the support of $\Phi_2$, $|a(1-x)^{-1}| < C$ and $|a^{-1}| <C$ which gives $C^{-2} < |1-x|$ if the second integral is not zero.
Similarly if the first integral is not zero we have $|(1-x)^{-1}x| < C^2$, which also implies $D < |1 - x|$ for a suitable constant $D > 0$.
It follows that $H(x:f:\eta)$ is zero in the neighborhood of 1.
Therefore the equation~\eqref{3.1.11} holds for all nonzero $x$.

Now, let's study $H(f:\eta)$ in the neighborhood of 0.
In~\eqref{3.1.11} the second integral is obviously a smooth function on $x$ at 0.
To study the first integral we will use the following lemma, the proof of which is left to the reader:
\begin{lemma}\label{lem:3.2}
    Let $\Phi$ be a two variable Schwarts-Bruhat function on $F \times F$.
    There exist two Schwartz-Bruhat functions $A_1(x)$ and $A_2(x)$ on $F$  such that for all $x \neq 0$ in $F$, we have
    \[
    \int_{F^\times} \Phi(a^{-1}x, a)\eta(a) d^\times a = A_1(x) + A_2(x) \eta(x).
    \]
    If $\Phi$ is real valued and compactly supported, then we can take $A_1$ and $A_2$ to be compactly supported.
\end{lemma}

Let's go back to the first integral of~\eqref{3.1.11}.
With the notations used in the lemma, the integral is equal to
\begin{align}
    A_1(x(1-x)^{-1}) + A_2(x(1-x)^{-1})\eta(x(1-x)^{-1}).
\end{align}
If $x$ is close enough to 0 then 1 - $x$ is a norm and $\eta(x(1-x)^{-1}) = \eta(x)$.
On the other hand, $A_i(x(1-x)^{-1})$ are smooth functions of $x$ in a neighborhood of 0 for $i = 1,2$.
As the second integral of~\eqref{3.1.11} is obviously smooth at point 0 we conclude that, in a neighborhood of 0, $H(x:f:\eta)$ has the following form:
\begin{equation}
    H(x:f:\eta) = A_1(x) + A_2(x) \eta(x)
\end{equation}
where $A_i$ are smooth for $i = 1, 2$.


To study $H(x:f:\eta)$ for large $|x|$ note that
\begin{align*}
    \varepsilon g(x) = g(x^{-1}) \bmat{1}{0}{0}{x}\quad \text{if } \varepsilon = \bmat{0}{1}{1}{0}.
\end{align*}
Then we get
\begin{align}
    H(x^{-1}: f: \eta) = H(x:f_0:\eta)\eta(x), \quad f_0(g) = f(\varepsilon g).
\end{align}
Then there exists functions $B_i$, $i = 1, 2$, defined on the neighborhood of 0 and smooth, such that
\begin{align}
    H(x:f:\eta) = B_1(x^{-1}) + B_2(x^{-1})\eta(x)
\end{align}
for $x$ with sufficiently large $|x|$.

\subsection{}
In summary,
\begin{proposition}\label{prop:3.1}
Let $H$ be a function on $F^\times$ such that there exists a compactly supported smooth function $f$ on $G/Z$ with $H(x:f:\eta) =H(x)$.
We have:
\begin{enumerate}
    \item $H$ is smooth on $F^\times$,
    \item $H$ vanishes on a neighborhood of 1,
    \item there exists a neighborhood $U$ of 0 and smooth functions $A_i$ on $U$ for $i = 1, 2$ such that, for $x \in U$, we have
    \[
    H(x) = A_1(x) + A_2(x) \eta(x),
    \]
    \item there exists a neighborhood $U$ of 0 and smooth functions $B_i$ on $U$ for $i = 1, 2$ such that, for $x$ with sufficiently large $|x|$, we have
    \[
    H(x) = B_1(x^{-1}) + B_{2}(x^{-1}) \eta(x).
    \]
\end{enumerate}
\end{proposition}


\subsection{}
We are going to discuss the significance of the zero sets of the  functions $A_i$ and $B_i$ of the proposition \ref{prop:3.1}.
For this purpose let us first recall that, if $\phi$ is a Schwartz-Bruhat function on $F$, then the integral
\[
\int_{F^\times} \phi(x) |x|^{s} d^\times x,
\]
or rather its analytic continuation, has a pole at the point $s = 0$; the residue at this point has the form $C\phi(0)$, where the constant $C$ depends on the choice of the Haar measure on the group $F^\times$.
On the other hand the integral
\[
\int_{F^\times} \phi(x) |x|^{s} \eta(x) d^\times x
\]
admits analytic continuation at $s = 0$ and its value at this point will again be denoted as an integral
\[
\int_{F^\times} \phi(x) \eta(x) d^\times x.
\]
Define the following quantities:
\begin{align}
    H(n_+: f: \eta) &= \int_{F^\times}\int_{F^\times} f \left( \bmat{a}{0}{0}{1} \bmat{1}{b}{0}{1} \right) \eta(b) d^\times a d^\times b, \label{3.4.1} \\
    H(n_-: f: \eta) &= \int_{F^\times}\int_{F^\times} f \left( \bmat{a}{0}{0}{1} \bmat{1}{0}{b}{1} \right) \eta(b) d^\times a d^\times b, \label{3.4.2} \\
    H(\varepsilon n_+: f: \eta) &= \int_{F^\times}\int_{F^\times} f \left( \bmat{a}{0}{0}{1} \varepsilon \bmat{1}{b}{0}{1} \right) \eta(b) d^\times a d^\times b, \label{3.4.3} \\
    H(\varepsilon n_-: f: \eta) &= \int_{F^\times}\int_{F^\times} f \left( \bmat{a}{0}{0}{1} \varepsilon \bmat{1}{0}{b}{1} \right) \eta(b) d^\times a d^\times b. \label{3.4.4}
\end{align}
In general, these integrals are divergent, but they can be interpreted as above.
For example, the first integral is the value at point 0 of the meromorphic function which, for $\Re s > 0$, is given by the convergent integral
\begin{align}
    \int_{F^\times} \int_{F^\times} f\left( \bmat{a}{0}{0}{1} \bmat{1}{b}{0}{1} \right) \eta(b) |b|^{s} d^\times a d^\times b.
\end{align}
However, note that if $f$ is supported in the open set $TN_{+}wN_{+}$ then the integrals~\eqref{3.4.1} and~\eqref{3.4.2} are convergent.
In fact the integral~\eqref{3.4.1} is zero since $TN_{+}$ and $TN_{+}wN_{+}$ are disjoint.
\textcolor{red}{On the other hand the intersection of $TN_{-}$ with a compact set contained in $TN_{+}wN_{+}$ is a disjoint compact of $T$.}
It follows that in~\eqref{3.4.2} the integrand is compactly support in $F^\times \times F^\times$ and the integral clearly converges.
Similarly the integrals~\eqref{3.4.3} and~\eqref{3.4.4} converge if $f$ is supported in $TN_{+}N_{+}$.
\begin{proposition}\label{prop:3.2}
With the notations in the proposition \ref{prop:3.1} we have:
\begin{align}
    H(n_+:f:\eta) &= A_2(0) \label{3.4.6} \\
    H(n_-:f:\eta) &= A_1(0) \label{3.4.7} \\
    H(\varepsilon n_+:f:\eta) &= B_1(0)\label{3.4.8} \\
    H(\varepsilon n_-:f:\eta) &= B_2(0). \label{3.4.9}
\end{align}
\end{proposition}
\begin{proof}
We will prove~\eqref{3.4.6} and~\eqref{3.4.7}.
By the decomposition~\eqref{3.1.5}, it is sufficient to prove when $f$ is supported on $TN_+ N_-$ or $AN_{+}wN_{+}$.
Then we have
\begin{align}
    H(x:f:\eta) = \int_{F^\times} \Phi(a(1-x)^{-1}, a^{-1}) \eta(a) d^\times a
\end{align}
where
\begin{align}
    \Phi(u, v) = \phi\left(\bmat{1}{u}{0}{1} w \bmat{1}{a^{-1}}{0}{1}\right), \\
    \phi(g) = \int_{T/Z} f(ag) da.
\end{align}
Since $H$ is smooth at zero we have $A_2 = 0$ and $A_1 = H(f:\eta)$.
Hence
\[
A_1(0) = \int_{F^\times} \phi\left(\bmat{1}{a}{0}{1} w \bmat{1}{a^{-1}}{0}{1}\right).
\]
Using the fact that $f$ is invariant under the center, we can rite the integral as follows:
\[
A_1(0) = \int_{F^\times} \int_{F} f\left(b \bmat{a^2}{0}{a}{1}\right) \eta(a) db d^\times a.
\]
Change of variable shows that this is nothing but $H(n_-:f:\eta)$.
~\eqref{3.4.6} also holds as both $A_2$ and $H(n_+:f:\eta)$ vanish.

Now assume that $f$ is supported in $TN_+ N_-$.
Then we have
\begin{align}
    H(x:f:\eta) = \int_{F^\times} \Phi(a^{-1}(1-x)^{-1}x, a) \eta(a) d^\times a
\end{align}
where
\begin{align}
    \Phi(u, v) = \phi\left(\bmat{1}{u}{0}{1} \bmat{1}{0}{v}{1} \right), \label{3.4.14} \\
    \phi(g) = \int_{T/Z} f(ag) da. \label{3.4.15}
\end{align}
Recall the definitions of $A_1$ and $A_2$:
\begin{align}
    H(x) = A_1(x) + A_2(x)\eta(x),
\end{align}
on the other hand, lemma~\eqref{lem:3.2} implies
\begin{align}
    \int_{F^\times} \Phi(a^{-1} x, a) \eta(a) d^\times a = C_1(x) + C_2(x) \eta(x).
\end{align}
Comparing with~\eqref{3.4.4} we get $C_i((1-x)^{-1}x) = A_i(x)$.
Hence $C_i$ and $A_i$ vanish on the same set.
By taking the Mellin transform of the above equation we get
\[
\int_{F^\times} \int_{F^\times} \Phi(x, a) |x|^{s} \eta(a) |a|^{s} d^\times a d^\times x = \int_{F^\times} C_1(x) |x|^{s} d^\times x + \int_{F^\times} C_2(x) |x|^{s} \eta(x) d^\times x.
\]
Comparing the residue of both sides at $s = 0$ we get
\[
\int_{F^\times} \Phi(0, a) \eta(a) d^\times a = C_1(0).
\]
By~\eqref{3.4.14} and~\eqref{3.4.15} the left hand side is nothing but $H(n_-:f:\eta)$.
On the other hand the right hand side equals to $A_1(0)$, hence~\eqref{3.4.7} is proven.
~\eqref{3.4.6} can be proved in a similar way.
Equations~\eqref{3.4.8} and~\eqref{3.4.9} follow from the equations~\eqref{3.4.6} and~\eqref{3.4.7} applied to the function $f_0$ defined as $f_0(g) = f(\varepsilon g)$.
\end{proof}