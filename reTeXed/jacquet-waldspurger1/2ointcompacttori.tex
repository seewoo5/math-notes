\section{Orbital integrals: compact torus}

\subsection{}
Let's keep the notations of section 1 but now assume that $F$ is a local field.
Then $T'/Z'$ is compact.
Choose a non-trivial additive character $\Psi$ of $F$.
Endow the additive group $F$ with the self-dual measure $dx$ for the character $\Psi$, the multiplicative group $F^\times$ with the measure $L(1, 1_F)|x|^{-1} dx$ (Tamagawa measure relative to $\Psi$).
Also, on the multiplicative group $E^\times$, we give a Tamagawa measure corresponds to the character $\Psi \circ \tr$.
Then these induce measures on $T'$ and $Z'$.
Equip $T'/Z'$ with the quotient measure.
Let $f'$ be a compactly supported smooth function on $G'/Z'$.
Let
\begin{equation}
H'(g':f':T') = \int_{T'/Z'} \int_{T'/Z'} f(sg't) dsdt.
\end{equation}
It is clear that $H'(g':f':T')$ only depends on the double coset of $g'$ modulo $T'$.
Let $x$ be an element of $F^\times$.
Define $H'(x:f':T')$ as $H'(g':f':T')$ if there's $g'$ in $G'$ such that $P'(g':T') = x$, and 0 otherwise.
Then we obtain a function $H'(f':T')$ on $F^\times$ and we are going to characterize the functions $H'$ on $F^\times$ which are of the form $H' = H'(f': T')$ for an appropriate function $f'$.

\subsection{}
Consider a function $H' = H'(f':T')$.
By definition $H'$ vanishes, therefore is smooth, on the complement of $cN$.
Consider a point $x$ of the form $P'(h':T')$.
As the norm is a submersive map from $E^\times$ to $F^\times$, the map $g' \to P'(g':T')$ is a fortiori submersive at the point $h$.
It follows that $H'$ is smooth at point $x$.
Finally suppose that 1 is in $cN$ (that is, the group $G'$ splits); then we can assume that $c = 1$.
We are going to show that $H'$ is zero near 1.

Since $f'$ is compactly supported modulo $Z'$, there exists a compact subset $C$ of $G'$ such that $H'(g':f':T') \neq 0$ implies $g' \in T'CT'$.
Hence it suffice to show the existence of a number $K$ such that  $|P'(g':T') - 1|>K$ for $g' \in T'CT'$.
Suppose there is no such number.
Then there would exist a sequence $g_i'$ of elements in $T'CT'$ such that $P'(g_i':T')$ tends to 1.
By enlarging $C$ and multiplying the elements of the sequence by elements of $T'$, we can assume that
\[
g_i' = 1 + \varepsilon t_i' = c_i z_i'
\]
with $t_i'$ in $T'$, $c_i$ in $C$ and $z_i'$ in $Z'$. So
\[
P'(g_i':T') = t_i' \bar{t}_i' = 1 + a_i
\]
and $a_i$ tends to zero.
On the other hand, we have:
\[
\det g_i' = -a_i = {(z_i')}^2 \det c_i.
\]
So $z_i'$ tends to zero, and same for $g_i'$.
Since the projection of $g_i'$ onto $L$ is 1, we get a contradiction.

\subsection{}
Let's examine the behavior of the function $H'$ near zero and near infinity.
We will show that there is a neighborhood $U$ of 0 in $F$ and a smooth function $A'$ on $U$ such that
\begin{align}
    H'(x) = A'(x) (1 + \eta(cx)), \quad x \in U \\
    2A'(0) = \mathrm{vol}(T'/Z') \int_{T'/Z'} f'(t) dt.
\end{align}
Similarly we will show that there exists a neighborhood $U$ of 0 in $F$ a smooth function $B'$ on $U$ such that
\begin{align}
    H'(x) = B'(x^{-1}) (1 + \eta(cx)), \quad x^{-1} \in U \\
    2B'(0) = \mathrm{vol}(T'/Z') \int_{T'/Z'} f'(\varepsilon t) dt.
\end{align}
Since $P'(\varepsilon g': T') = {P'(g':T')}^{-1}$ we have
\[
\int_{T'/Z'} \int_{T'/Z'} f'(s\varepsilon g't) dsdt = H'(x^{-1}:f':T')
\]
or
\[
H'(x^{-1}:f': T') = H'(x:f_0':T'), \quad f_0'(g') = f'(\varepsilon g').
\]
Hence it suffices to prove the assertions near zero.
We will consider non-archimedean case first.
Take a $x$ in $cN$.
Then $x = cl\bar{l}$ for some $l \in L$, hence $x = P(h)$ with $h = 1 + \varepsilon l$.
Then we can write
\[
H'(x:f':T')=\int_{T'/Z'} \int_{T'/Z'} f(t_1 (1 + \varepsilon l) t_2) dt_1 dt_2
\]
or with a change of variable
\begin{equation}
\label{2.3.5}
H'(x:f':T') = \int_{T'/Z'} \int_{T'/Z'} f\left(\left(1 + \varepsilon l \frac{\bar{t}_{1}}{t_ {1}} \right)t_{2} \right) dt_1 dt_2.
\end{equation}
Since $f'$ is smooth there exists an ideal $V$ of $E$ such that for $l$ in $V$ we have
\[
f'(g') = f((1 + \varepsilon l) g')
\]
for all $g'$.
Then there exists an ideal $U$ in $F$ such that $l \bar{l} \in U$ is equivalent to $l \in V$.
For $x$ in $cU$ we therefore have $H'(x) = 0$ if $x$ is not in $cN$; if $x$ is in $cN$ then $x = cl\bar{l}$ with $l$ in $V$ and by~\eqref{2.3.5}
\[
H'(x) = \mathrm{vol}(T'/Z') \int_{T'/Z'} f'(t) dt.
\]
Our assertion follows immediately from this.

Let move on to the archimedean case, i.e. $F = \mathbb{R}$ and $L = \mathbb{C}$.
Let $K(x) = H'(cx)$.
Let $V$ be a disk $\{z:z\bar{z} < a\}$ in $L$ such that $1 + \varepsilon V$ is contained in $G$.
Then the right hand side of~\eqref{2.3.5} defines a smooth function on $V$, say $C(l)$, depending only on the norm of $l$.
We have
\[
K(x) = \begin{cases} 0 & x < 0 \\ C(l) & x > 0 \text{ and } x = l \bar{l} \text{ for } x \in V.\end{cases}
\]
In particular, the restriction of $C$ to the real axis is even and smooth and we have
\[
K(x) = \begin{cases} 0 & x < 0 \\ C(y) & 0 <x < a\text{ and } x = y^2, y \in \mathbb{R}. \end{cases}
\]
Our assertion follows from the existence of a smooth function $D$ on $F$ such that $D(x) = K(x)$ for $a > x > 0$, which is a consequence of Whitney's theorem.\footnote{Whitney's approximation theorem: smooth function on $\mathbb{R}$ can be approximated by analytic functions.}

\subsection{}
The following properties characterizes $H'(f':T')$:
\begin{proposition}\label{prop:2.1}
Let $H'$ be a function on $F^\times$.
There exists a compactly supported smooth function $f'$ on $G'/Z'$ with $H' = H'(f':T')$ if and only if the following conditions hold:
\begin{enumerate}[label={(\arabic*)}]
    \item $H'$ vanishes on the complement of $cN$,
    \item $H'$ vanishes on a neighborhood of 1,
    \item There exists a smooth function $A'$ on a neighborhood of 0 in $F$ such that, for $x$ near 0, we have
    \[
    H'(x) = A'(x) (1 + \eta(cx)),
    \]
    \item There exists a smooth function $B'$ on a neighborhood of 0 in $F$ such that, when $|x|$ is sufficiently large,
    \[
    H'(x) = B'(x^{-1}) (1 + \eta(cx)) .
    \]
\end{enumerate}
If $f'$, $A'$ and $B'$ satisfy these conditions then
\[
2A'(0) = \mathrm{vol}(T'/Z') \int_{T'/Z'} f'(t)dt, \quad 2B'(0) = \mathrm{vol}(T'/Z') \int_{T'/Z'} f'(\varepsilon t) dt.
\]
\end{proposition}
We have just shown that the conditions (1) to (4) are necessary.
We will leave it to the reader to show that they are also sufficient.
The last assertion of the proposition has been proved above.