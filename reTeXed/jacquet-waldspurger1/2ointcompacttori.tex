\section{Orbital integrals: compact torus}

\subsection{}
Let's keep the notations from Section 1, but now assume that $F$ is a local field.
In this case, the quotient $T'/Z'$ is compact.
We choose a non-trivial additive character $\Psi$ of $F$.
The additive group $F$ is endowed with the self-dual measure $\dd x$ relative to $\Psi$, while the multiplicative group $F^\times$ is equipped with the measure $L(1, 1_F)|x|^{-1} \dd x$ (the Tamagawa measure relative to $\Psi$).
Similarly, we endow the multiplicative group $E^\times$ with a Tamagawa measure corresponding to the character $\Psi \circ \tr$.
These measures induce corresponding measures on $T'$ and $Z'$, and we equip $T'/Z'$ with the quotient measure.
Let $f'$ be a compactly supported smooth function on $G'/Z'$.
Define
\begin{equation}
    H'(g':f':T') = \int_{T'/Z'} \int_{T'/Z'} f(sg't) \dd s \dd t.
\end{equation}
It is clear that $H'(g':f':T')$ depends only on the double coset of $g'$ modulo $T'$.
Now, let $x$ be an element of $F^\times$.
Define $H'(x:f':T')$ as $H'(g':f':T')$ if there's $g'$ in $G'$ such that $P'(g':T') = x$, and 0 otherwise.
Thus, we obtain a function $H'(f':T')$ on $F^\times$, and we will now characterize the functions $H'$ on $F^\times$ that arise in the form $H' = H'(f':T')$ for some appropriate function $f'$.

\subsection{}
Consider a function $H' = H'(f':T')$. By construction, $H'$ vanishes on the complement of $c\rN$, and therefore is smooth there. Let $x$ be a point of the form $P'(h':T')$.
Since the norm is a submersive map from $E^\times$ to $F^\times$, the map $g' \mapsto P'(g':T')$ is also submersive at the point $h'$. 
Consequently, $H'$ is smooth at $x$.
Next, assume that $1$ belongs to $c\rN$ (i.e., the group $G'$ splits), and without loss of generality, we take $c = 1$.
We will show that $H'$ vanishes near 1.

Since $f'$ is compactly supported modulo $Z'$, there exists a compact subset $C$ of $G'$ such that $H'(g':f':T') \neq 0$ implies $g' \in T'CT'$.
Thus, it suffices to show the existence of a number $K$ such that $|P'(g':T') - 1| > K$ for $g' \in T'CT'$.
Suppose no such number exists. Then, there would exist a sequence $g_i'$ of elements in $T'CT'$ such that $P'(g_i':T') \to 1$. By enlarging $C$ and multiplying the elements of the sequence by elements of $T'$, we can assume that
\[
g_i' = 1 + \varepsilon t_i' = c_i z_i'
\]
where $t_i' \in T'$, $c_i \in C$, and $z_i' \in Z'$. Thus,
\[
P'(g_i':T') = t_i' \bar{t}_i' = 1 + a_i
\]
where $a_i \to 0$.
On the other hand, we have:
\[
\det g_i' = -a_i = {(z_i')}^2 \det c_i.
\]
which implies that $z_i' \to 0$, and consequently, $g_i' \to 0$. Since the projection of $g_i'$ onto $L$ is 1, this leads to a contradiction. Hence, $H'$ must vanish near 1.

\subsection{}
Now, let us examine the behavior of $H'$ near 0 and near infinity. We will show that there exists a neighborhood $U$ of $0$ in $F$ and a smooth function $A'$ on $U$ such that:
\begin{align}
    H'(x) &= A'(x) (1 + \eta(cx)), \quad x \in U \\
    2A'(0) &= \mathrm{vol}(T'/Z') \int_{T'/Z'} f'(t) dt.
\end{align}
Similarly, there exists a neighborhood $U$ of 0 in $F$ and a smooth function $B'$ on $U$ such that:
\begin{align}
    H'(x) &= B'(x^{-1}) (1 + \eta(cx)), \quad x^{-1} \in U \\
    2B'(0) &= \mathrm{vol}(T'/Z') \int_{T'/Z'} f'(\varepsilon t) \dd t.
\end{align}
Since $P'(\varepsilon g': T') = {P'(g':T')}^{-1}$, we have:
\[
\int_{T'/Z'} \int_{T'/Z'} f'(s\varepsilon g't) \dd s \dd t = H'(x^{-1}:f':T')
\]
or
\[
H'(x^{-1}:f': T') = H'(x:f_0':T'), \quad f_0'(g') = f'(\varepsilon g').
\]
Thus, it suffices to prove the assertions near 0.
Let us consider the non-archimedean case first.
Suppose $x \in c\rN$.
Then $x = cl\bar{l}$ for some $l \in L$, and thus $x = P(h)$ for $h = 1 + \varepsilon l$.
We can express
% We will consider non-archimedean case first.
% Take a $x$ in $cN$.
% Then $x = cl\bar{l}$ for some $l \in L$, hence $x = P(h)$ with $h = 1 + \varepsilon l$.
% Then we can write
\[
H'(x:f':T')=\int_{T'/Z'} \int_{T'/Z'} f(t_1 (1 + \varepsilon l) t_2) \dd t_1 \dd t_2
\]
or, after a change of variables:
\begin{equation}
    \label{2.3.5}
    H'(x:f':T') = \int_{T'/Z'} \int_{T'/Z'} f\left(\left(1 + \varepsilon l \frac{\bar{t}_{1}}{t_ {1}} \right)t_{2} \right) \dd t_1 \dd t_2.
\end{equation}
Since $f'$ is smooth, there exists an ideal $V$ of $E$ such that for all $l \in V$, we have:
\[
f'(g') = f((1 + \varepsilon l) g')
\]
for all $g'$.
Therefore, there exists an ideal $U$ of $F$ such that $l\bar{l} \in U$ is equivalent to $l \in V$. For $x \in cU$, we have $H'(x) = 0$ if $x \notin c\rN$; if $x \in c\rN$, then $x = cl\bar{l}$ for $l \in V$, and from \eqref{2.3.5}:
\[
H'(x) = \mathrm{vol}(T'/Z') \int_{T'/Z'} f'(t) \dd t.
\]
Our assertion follows immediately from this.

In the archimedean case, i.e., when $F = \mathbb{R}$ and $L = \mathbb{C}$, let $K(x) = H'(cx)$. Consider a disk $V = \{ z : z \bar{z} < a \}$ in $L$ such that $1 + \varepsilon V \subset G$. Then the right-hand side of \eqref{eqn:2.3.5} defines a smooth function on $V$, say $C(l)$, which depends only on the norm of $l$. Thus,
\[
K(x) = \begin{cases} 0 & x < 0 \\ C(l) & x > 0 \text{ and } x = l \bar{l} \text{ for } x \in V.\end{cases}
\]
In particular, the restriction of $C$ to the real axis is even and smooth, and we obtain:
\[
K(x) = \begin{cases} 0 & x < 0 \\ C(y) & 0 <x < a\text{ and } x = y^2, y \in \mathbb{R}. \end{cases}
\]
Thus, the existence of a smooth function $D$ on $F$ such that $D(x) = K(x)$ for $0 < x < a$ follows from Whitney's approximation theorem.\footnote{Whitney's approximation theorem: a smooth function on $\mathbb{R}$ can be approximated by analytic functions.}

\subsection{}
The following properties characterizes the functions $H'(f':T')$:
\begin{proposition}\label{prop:2.1}
Let $H'$ be a function on $F^\times$. There exists a compactly supported smooth function $f'$ on $G'/Z'$ such that $H' = H'(f':T')$ if and only if the following conditions hold:
\begin{enumerate}[label=(\arabic*)]
    \item $H'$ vanishes on the complement of $c\rN$,
    \item $H'$ vanishes on a neighborhood of $1$,
    \item There exists a smooth function $A'$ on a neighborhood of $0$ in $F$ such that, for $x$ near 0, we have:
    \[
    H'(x) = A'(x) (1 + \eta(cx)),
    \]
    \item There exists a smooth function $B'$ on a neighborhood of $0$ in $F$ such that, for sufficiently large $|x|$,
    \[
    H'(x) = B'(x^{-1}) (1 + \eta(cx)).
    \]
\end{enumerate}
If $f'$, $A'$, and $B'$ satisfy these conditions, then: 
\[
2A'(0) = \mathrm{vol}(T'/Z') \int_{T'/Z'} f'(t) \dd t, \quad 2B'(0) = \mathrm{vol}(T'/Z') \int_{T'/Z'} f'(\varepsilon t) \dd t.
\]
\end{proposition}
We have demonstrated the necessity of conditions (1) to (4).
The proof of their sufficiency is left to the reader.
The last assertion of the proposition has already been established.
