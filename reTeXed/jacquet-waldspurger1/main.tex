% --- LaTeX Lecture Notes Template - S. Venkatraman ---

% --- Set document class and font size ---

\documentclass[letterpaper, 12pt]{article}

% --- Package imports ---

% Extended set of colors
\usepackage[dvipsnames]{xcolor}

\usepackage{
  amsmath, amsthm, amssymb, mathtools, dsfont, units,          % Math typesetting
  graphicx, wrapfig, subfig, float,                            % Figures and graphics formatting
  listings, color, inconsolata, pythonhighlight,               % Code formatting
  fancyhdr, sectsty, hyperref, enumerate, framed }   % Headers/footers, section fonts, links, lists

% lipsum is just for generating placeholder text and can be removed
\usepackage{hyperref, lipsum} 

% --- Fonts ---

\usepackage{newpxtext, newpxmath, inconsolata}
\usepackage{amsfonts}

\usepackage{tikz}
\usepackage{tikz-cd}
\usepackage{enumitem}
\usepackage[title]{appendix}
\usepackage{mathdots}
\usepackage{stmaryrd}

% --- Page layout settings ---

% Set page margins
\usepackage[left=1.35in, right=1.35in, top=1.0in, bottom=.9in, headsep=.2in, footskip=0.35in]{geometry}

% Anchor footnotes to the bottom of the page
\usepackage[bottom]{footmisc}

% Set line spacing
\renewcommand{\baselinestretch}{1.2}

% Set spacing between paragraphs
\setlength{\parskip}{1.3mm}

% Allow multi-line equations to break onto the next page
\allowdisplaybreaks

% --- Page formatting settings ---

% Set image captions to be italicized
\usepackage[font={it,footnotesize}]{caption}

% Set link colors for labeled items (blue), citations (red), URLs (orange)
\hypersetup{colorlinks=true, linkcolor=RoyalBlue, citecolor=RedOrange, urlcolor=ForestGreen}

% Set font size for section titles (\large) and subtitles (\normalsize) 
\usepackage{titlesec}
% \titleformat{\section}{\large\bfseries}{{\fontsize{19}{19}\selectfont\textreferencemark}\;\; }{0em}{}
\titleformat{\section}{\large\bfseries}{\thesection\;\;\;}{0em}{}
\titleformat{\subsection}{\normalsize\bfseries\selectfont}{\thesubsection\;\;\;}{0em}{}

% Enumerated/bulleted lists: make numbers/bullets flush left
%\setlist[enumerate]{wide=2pt, leftmargin=16pt, labelwidth=0pt}
% \setlist[itemize]{wide=0pt, leftmargin=16pt, labelwidth=10pt, align=left}

% --- Table of contents settings ---

\usepackage[subfigure]{tocloft}

% Reduce spacing between sections in table of contents
\setlength{\cftbeforesecskip}{.9ex}

% Remove indentation for sections
\cftsetindents{section}{0em}{0em}

% Set font size (\large) for table of contents title
\renewcommand{\cfttoctitlefont}{\large\bfseries}

% Remove numbers/bullets from section titles in table of contents
\makeatletter
\renewcommand{\cftsecpresnum}{\begin{lrbox}{\@tempboxa}}
\renewcommand{\cftsecaftersnum}{\end{lrbox}}
\makeatother

% --- Set path for images ---

\graphicspath{{Images/}{../Images/}}

% --- Math/Statistics commands ---

% Add a reference number to a single line of a multi-line equation
% Usage: "\numberthis\label{labelNameHere}" in an align or gather environment
\newcommand\numberthis{\addtocounter{equation}{1}\tag{\theequation}}

% Shortcut for bold text in math mode, e.g. $\b{X}$
\let\b\mathbf

% Shortcut for bold Greek letters, e.g. $\bg{\beta}$
\let\bg\boldsymbol

% Shortcut for calligraphic script, e.g. %\mc{M}$
\let\mc\mathcal

% \mathscr{(letter here)} is sometimes used to denote vector spaces
\usepackage[mathscr]{euscript}

% Convergence: right arrow with optional text on top
% E.g. $\converge[p]$ for converges in probability
\newcommand{\converge}[1][]{\xrightarrow{#1}}

% Weak convergence: harpoon symbol with optional text on top
% E.g. $\wconverge[n\to\infty]$
\newcommand{\wconverge}[1][]{\stackrel{#1}{\rightharpoonup}}

% Equality: equals sign with optional text on top
% E.g. $X \equals[d] Y$ for equality in distribution
\newcommand{\equals}[1][]{\stackrel{\smash{#1}}{=}}

% Normal distribution: arguments are the mean and variance
% E.g. $\normal{\mu}{\sigma}$
\newcommand{\normal}[2]{\mathcal{N}\left(#1,#2\right)}

% Uniform distribution: arguments are the left and right endpoints
% E.g. $\unif{0}{1}$
\newcommand{\unif}[2]{\text{Uniform}(#1,#2)}

% Independent and identically distributed random variables
% E.g. $ X_1,...,X_n \iid \normal{0}{1}$
\newcommand{\iid}{\stackrel{\smash{\text{iid}}}{\sim}}

% Sequences (this shortcut is mostly to reduce finger strain for small hands)
% E.g. to write $\{A_n\}_{n\geq 1}$, do $\bk{A_n}{n\geq 1}$
\newcommand{\bk}[2]{\{#1\}_{#2}}

% \setcounter{section}{-1}

\newcommand{\SL}{\mathrm{SL}}
\newcommand{\Sp}{\mathrm{Sp}}
\newcommand{\Mp}{\mathrm{Mp}}
\newcommand{\GL}{\mathrm{GL}}
\newcommand{\SO}{\mathrm{SO}}
\newcommand{\SU}{\mathrm{SU}}
\newcommand{\PGL}{\mathrm{PGL}}
\newcommand{\PSL}{\mathrm{PSL}}
\newcommand{\rM}{\mathrm{M}}
\newcommand{\rN}{\mathrm{N}}
\newcommand{\rO}{\mathrm{O}}
\newcommand{\rP}{\mathrm{P}}
\newcommand{\rH}{\mathrm{H}}
\newcommand{\rU}{\mathrm{U}}
\newcommand{\JL}{\mathrm{JL}}
\newcommand{\stab}{\mathrm{Stab}}
\newcommand{\cusp}{\mathrm{cusp}}
\newcommand{\reg}{\mathrm{reg}}
\newcommand{\rs}{\mathrm{rs}}
\newcommand{\Irr}{\mathrm{Irr}}
\newcommand{\Tr}{\mathrm{Tr}}
\newcommand{\Hom}{\mathrm{Hom}}
\newcommand{\Gal}{\mathrm{Gal}}
\newcommand{\WD}{\mathrm{WD}}
\newcommand{\Frob}{\mathrm{Frob}}
\newcommand{\Res}{\mathrm{Res}}
\newcommand{\Tam}{\mathrm{Tam}}
\newcommand{\Pet}{\mathrm{Pet}}
\newcommand{\sgn}{\mathrm{sgn}}
\newcommand{\vol}{\mathrm{vol}}
\newcommand{\Aut}{\mathrm{Aut}}
\newcommand{\Ind}{\mathrm{Ind}}
\newcommand{\BC}{\mathrm{BC}}
\newcommand{\Ad}{\mathrm{Ad}}

\newcommand{\what}{\widehat}

\newcommand{\dd}{\mathrm{d}}

\newcommand{\bA}{\mathbb{A}}
\newcommand{\bR}{\mathbb{R}}
\newcommand{\bS}{\mathbb{S}}
\newcommand{\bZ}{\mathbb{Z}}
\newcommand{\bN}{\mathbb{N}}
\newcommand{\bC}{\mathbb{C}}
\newcommand{\bQ}{\mathbb{Q}}
\newcommand{\bH}{\mathbb{H}}
\newcommand{\bI}{\mathbb{I}}
\newcommand{\bfi}{\mathbf{I}}
\newcommand{\bfa}{\mathbf{a}}
\newcommand{\bfb}{\mathbf{b}}
\newcommand{\cS}{\mathcal{S}}
\newcommand{\cO}{\mathcal{O}}
\newcommand{\cV}{\mathcal{V}}
\newcommand{\cP}{\mathcal{P}}

\newcommand{\scA}{\mathscr{A}}
\newcommand{\scB}{\mathscr{B}}
\newcommand{\scV}{\mathscr{V}}
\newcommand{\scT}{\mathscr{T}}
\newcommand{\scU}{\mathscr{U}}
\newcommand{\scW}{\mathscr{W}}
\newcommand{\scO}{\mathscr{O}}
\newcommand{\scL}{\mathscr{L}}
\newcommand{\scS}{\mathscr{S}}

\newcommand{\frh}{\mathfrak{h}}
\newcommand{\frt}{\mathfrak{t}}
\newcommand{\frg}{\mathfrak{g}}
\newcommand{\frgl}{\mathfrak{gl}}
\newcommand{\fru}{\mathfrak{u}}

% Math mode symbols for common sets and spaces. Example usage: $\R$
\newcommand{\R}{\mathbb{R}}	% Real numbers
\newcommand{\C}{\mathbb{C}}	% Complex numbers
\newcommand{\Q}{\mathbb{Q}}	% Rational numbers
\newcommand{\Z}{\mathbb{Z}}	% Integers
\newcommand{\N}{\mathbb{N}}	% Natural numbers
\newcommand{\F}{\mathcal{F}}	% Calligraphic F for a sigma algebra
\newcommand{\El}{\mathcal{L}}	% Calligraphic L, e.g. for L^p spaces

% Math mode symbols for probability
\newcommand{\pr}{\mathbb{P}}	% Probability measure
\newcommand{\E}{\mathbb{E}}	% Expectation, e.g. $\E(X)$
\newcommand{\var}{\text{Var}}	% Variance, e.g. $\var(X)$
\newcommand{\cov}{\text{Cov}}	% Covariance, e.g. $\cov(X,Y)$
\newcommand{\corr}{\text{Corr}}	% Correlation, e.g. $\corr(X,Y)$
\newcommand{\B}{\mathcal{B}}	% Borel sigma-algebra

% Other miscellaneous symbols
\newcommand{\tth}{\text{th}}	% Non-italicized 'th', e.g. $n^\tth$
\newcommand{\Oh}{\mathcal{O}}	% Big-O notation, e.g. $\O(n)$
\newcommand{\1}{\mathds{1}}	% Indicator function, e.g. $\1_A$


\newcommand{\tr}{\mathrm{tr}}
\newcommand{\rtr}{\mathrm{rtr}}
\newcommand{\ip}[2]{\langle #1, #2 \rangle}
\newcommand{\RO}{\mathrm{RO}}
\newcommand{\Aa}{\mathbb{A}}
\newcommand{\bmat}[4]{\begin{bmatrix} #1 & #2 \\ #3 & #4 \end{bmatrix}}
\newcommand{\smat}[4]{[\begin{smallmatrix} #1 & #2 \\ #3 & #4 \end{smallmatrix}]}

% Additional commands for math mode
\DeclareMathOperator*{\argmax}{argmax}		% Argmax, e.g. $\argmax_{x\in[0,1]} f(x)$
\DeclareMathOperator*{\argmin}{argmin}		% Argmin, e.g. $\argmin_{x\in[0,1]} f(x)$
\DeclareMathOperator*{\spann}{Span}		% Span, e.g. $\spann\{X_1,...,X_n\}$
\DeclareMathOperator*{\bias}{Bias}		% Bias, e.g. $\bias(\hat\theta)$
\DeclareMathOperator*{\ran}{ran}			% Range of an operator, e.g. $\ran(T) 
\DeclareMathOperator*{\dv}{d\!}			% Non-italicized 'with respect to', e.g. $\int f(x) \dv x$
\DeclareMathOperator*{\diag}{diag}		% Diagonal of a matrix, e.g. $\diag(M)$
\DeclareMathOperator*{\trace}{Tr}		% Trace of a matrix, e.g. $\trace(M)$
\DeclareMathOperator*{\supp}{supp}		% Support of a function, e.g., $\supp(f)$

% Numbered theorem, lemma, etc. settings - e.g., a definition, lemma, and theorem appearing in that 
% order in Lecture 2 will be numbered Definition 2.1, Lemma 2.2, Theorem 2.3. 
% Example usage: \begin{theorem}[Name of theorem] Theorem statement \end{theorem}
\theoremstyle{definition}
\newtheorem{theorem}{Theorem}[section]
\newtheorem{conjecture}{Conjecture}[section]
\newtheorem{proposition}{Proposition}
\newtheorem{lemma}{Lemma}
\newtheorem{corollary}[theorem]{Corollary}
\newtheorem{definition}{Definition}
\newtheorem{example}[theorem]{Example}
\newtheorem{remark}[theorem]{Remark}

% Un-numbered theorem, lemma, etc. settings
% Example usage: \begin{lemma*}[Name of lemma] Lemma statement \end{lemma*}
\newtheorem*{theorem*}{Theorem}
\newtheorem*{proposition*}{Proposition}
\newtheorem*{lemma*}{Lemma}
\newtheorem*{corollary*}{Corollary}
\newtheorem*{definition*}{Definition}
\newtheorem*{example*}{Example}
\newtheorem*{remark*}{Remark}
\newtheorem*{claim}{Claim}
\newtheorem*{question*}{Question}

% --- Left/right header text (to appear on every page) ---

% Do not include a line under header or above footer
\pagestyle{fancy}
\renewcommand{\footrulewidth}{0pt}
\renewcommand{\headrulewidth}{0pt}

% Right header text: Lecture number and title
% \renewcommand{\sectionmark}[1]{\markright{#1} }
% \fancyhead[R]{\small\textit{\nouppercase{\rightmark}}}

% Left header text: Short course title, hyperlinked to table of contents
% \fancyhead[L]{\hyperref[sec:contents]{\small Gan-Gross-Prasad conjecture}}
\setcounter{tocdepth}{1}
\setcounter{section}{-1}
\numberwithin{equation}{subsection}
\numberwithin{theorem}{subsection}
\numberwithin{proposition}{section}
\numberwithin{lemma}{section}
\numberwithin{definition}{subsection}

% --- Document starts here ---

\begin{document}

% --- Main title and subtitle ---

\title{Sur un r\'esultat de Waldspurger \\[1em]
\normalsize Translated and Re-\TeX ed by Seewoo Lee}

% --- Author and date of last update ---

\author{Herv\'e Jacquet}
\date{\normalsize\vspace{-1ex} Last updated: \today}

% --- Add title and table of contents ---

\maketitle


% --- Abstracts ---

% \tableofcontents\label{sec:contents}

% --- Main content: import lectures as subfiles ---

\tableofcontents

\newpage

\textbf{Note from the translator:} 

Most of the translations are done by Google Translator and ChatGPT, where I fixed some grammatical errors and made the translation more readable.
Here are some notations that I used in this translation, which are different from the original paper or do not occur.

\begin{itemize}
    \item For a number field $F$, we denote the ring of ad\'eles over $F$ as $\Aa_F$, instead of $F_A$ as in the original paper.
    \item $G$ would be the group of invertible 2 by 2 matrices (view as an algebraic group) and $T$ be a maximal torus of $G$, chosen as a subgroup of diagonal matrices in $G$ (the original paper uses $A$), and $Z$ be the center of $G$.
    We will denote inner forms of $G$ as $G'$ and the corresponding maximal tori and centers as $T'$ and $Z'$ (the original paper uses $T$ without $'$).
    Also, elements, functions or anything that are related to the inner form $G'$ would be denoted with $'$ ($g' \in G'$, $f': G' \to \mathbb{C}$, ...).
    \item For the above $G$ and $T$, we will use notations $[G] = G(\Aa_F) / G(F)Z(\Aa_F)$ and $[T] = T(\Aa_F) / T(F) Z(\Aa_F)$, and similarly for $G'$ and $T'$ as $[G']$ and $[T']$.
\end{itemize}

Also, all the footnotes are added by myself.
If you find any typos or errors, please contact seewoo5@berkeley.edu.

\newpage



\section{Introduction}
\subsection{}

We present a new proof of a remarkable result by Waldspurger~\cite[Theorem 2]{waldspurger1985valeurs}.
While Waldspurger's original proof relies on the properties of Weil's representation, our approach is based on a variant of the trace formula.
We believe that this alternative perspective offers some interest.

We begin by recalling the statement of the result.
Let $F$ be a number field, and $E$ a quadratic extension of $F$.
Denote by $\eta$ the character of the id\'ele class group of $F$ associated with $E$.
Consider the group $\GL(2)$ as an algebraic group $G$ defined over $F$, and let $Z$ be its center.
Let $T$ be a maximal torus in $G$, namely the group of diagonal matrices.
Let $\pi$ be an automorphic representation of $G(\Aa_F)$ that is trivial on the center $Z(\Aa_F)$.

We say that $\pi$ satisfies Waldspurger's first condition (denoted \textbf{W1}) if there exist automorphic forms $\phi_1$ and $\phi_2$ in the space of $\pi$ such that the following integrals are nonzero:
\begin{equation}
    \int_{T(\Aa_F) / Z(\Aa_F)} \phi_1(a) \dd a, \qquad \int_{T(\Aa_F) / Z(\Aa_F)} \phi_2(a) \eta(\det a) \dd a.
\end{equation}
Next, define the set $X(E:F)$, or simply $X$, as the set of isomorphism classes of pairs $(G', T')$, where $G'$ is an inner form of $G$ and $T'$ is a maximal torus in $G'$ isomorphic to $E^\times$ over $F$.
Such a pair arises from another pair $(H, L)$, where $H$ is a quaternion algebra over $F$ and $L$ is a subfield of $H$ isomorphic to $E$ over $F$, by taking $G'$ as the multiplicative group of $H$ and $T'$ as that of $L$.
We denote the center of $G'$ by $Z'$.

We identify the set $X$ with a set of representatives for each isomorphism class in $X$.
Let $X(\pi)$ be the set of triples $(G', T', \pi')$, where the pair $(G', T')$ lies in $X$ and $\pi'$ is a cuspidal automorphic representation of $G'(\Aa_F)$ associated with $\pi$ through the condition outlined in~\cite{jacquet2006automorphic}.
This condition can be stated as follows: there exists a finite set of places $S$ of $F$ such that, for $v \notin S$, the groups $G_v$ and $G_v'$ are isomorphic, and the representations $\pi_v$ and $\pi_v'$ are equivalent under this isomorphism.\footnote{Jacquet--Langlands correspondence.}
We say that $\pi$ satisfies Waldspurger's second condition (denoted \textbf{W2}) if there exists a triple $(G', T', \pi')$ in $X(\pi)$ and an automorphic form $\phi'$ in the space of $\pi'$ such that the following integral is nonzero:
\begin{equation}
    \int_{T'(\Aa_F) / Z'(\Aa_F)} \phi'(t) \dd t.
\end{equation}

We now state the result:
\begin{theorem}[Waldspurger] Conditions \textbf{W1} and \textbf{W2} are equivalent. \end{theorem}

\subsection{}
We now outline the main ideas behind our proof.
First, we identify the set of double cosets $T \backslash G / T$ with the disjoint union of the double cosets $T' \backslash G / T'$ (\S 1).
For this identification, we restrict to ``regular'' double cosets.
This motivates the introduction of a compactly supported smooth function $f$ on $G(\Aa_F) / Z(\Aa_F)$, and for each $(G', T')$, a corresponding compactly supported smooth function $f'$ on $G'(\Aa_F) / Z'(\Aa_F)$.
In fact, $f'$ will be zero for almost all $(G', T')$.
We associate a cuspidal kernel $K_c$ to $f$, and similarly, a cuspidal kernel $K_c'$ to each $f'$.
The relevant conditions imposed on these functions are (\S 7 and \S 10):
\begin{equation}
    \label{eqn:0.2.1}
    \int_{[T]} \int_{[T]} K_{c}(a, b) \eta(\det b) \dd b \dd a = \sum_{(G', T')} \int_{[T']} \int_{[T']} K_{c}'(s, t) \dd t \dd s.
\end{equation}

The relation between $f$ and $f'$ is as follows.
These functions are products of local functions.
If $v$ is a place of $F$ that splits in $E$, then for all $(G', T')$, the groups $G_v'$ and $G_v$ are isomorphic, and we define $f_v$ and $f_v'$ to be identical.
In this case, the set $X(E_v: F_v)$ consists of two elements $(G'_{v_i}, T'_{v_i})$, $i = 1, 2$, with $G'_{v_1}$ being split.
We can still identify the regular double cosets of $T_v$ with the disjoint union of the regular double cosets of $T'_1$ and $T'_2$.
We show that for a given function $f$, there are functions $f'_i$ on $G'_i$ such that:
\begin{equation*}
    \int_{T_v / Z_v} \int_{T_v / Z_v} f(agb) \eta_v(\det b) \dd b \dd a = \int_{T'_{v_i} / Z'_{v_i}} \int_{T'_{v_i} / Z'_{v_i}} f'_{i}(sg't) \dd t \dd s,
\end{equation*}
whenever $g$ corresponds to $g'$ (\S 2 and \S 4). If $v$ is unramified and $f_v$ is a Hecke function\footnote{The characteristic function of the spherical subgroup.}, then we can set $f'_1 = f_v$ and $f'_2 = 0$ (\S 5). The condition simplifies to $f'_v = f'_i$ if $G_v' = G_i'$. Waldspurger's result then follows easily from identity~\eqref{eqn:0.2.1}. Auxiliary results can be found in Section 6.

The proof of formula~\eqref{eqn:0.2.1} relies on a generalization of the trace formula, which can be stated as follows.
Let $G$ be a semisimple group defined over $F$, and let $A$ and $B$ be subgroups of $G$ defined over $F$.
Let $\lambda$ and $\mu$ be characters of $[A]$ and $[B]$, respectively.\footnote{i.e. the characters of $A(\bA_F)$ and $B(\bA_F)$ that are trivial on $A(F)$ and $B(F)$, respectively.}
Let $f$ be a compactly supported smooth function on $[G]$, and consider the integral:
\[
\int_{[A]} \int_{[B]} K_c(a, b) \lambda(a) \mu(b) \dd b \dd a
\]
where $K_c$ is the cuspidal kernel attached to $f$.
This kernel has a complex expression, which involves the sum:
\[
\sum_{\zeta \in G(F)} f(x^{-1} \zeta y).
\]
Choose a system of representatives for the double cosets of the groups $A(F)$ and $B(F)$.
For an element $\eta$  of $G(F)$ let $H_\eta$ the subgroup of $H = A\times B$ of tuples $(\alpha, \beta)$ such that $\alpha^{-1} \eta \beta = \eta$.
Then any element of $G(F)$ can be uniquely written in the form:
\[
\zeta = \alpha^{-1} \eta \beta, \qquad \eta \in A(F) \backslash G(F) / B(F), \quad (\alpha, \beta) \in H_{\eta}(F) \backslash (A(F) \times B(F)).
\]
By formal computation, we arrive at the following expression for the integral:
\[
\sum_{\eta} \mathrm{vol}(H_{\eta}(F) \backslash H(\Aa_F)) \int_{[A]} \int_{[B]} f(a^{-1} \eta b) \lambda(a) \mu(b) \dd b \dd a
\]
where sum on the left hand side is over all $\eta$ such that $\lambda(a) \mu(b) = 1$ if $a^{-1} \eta b = 1$.
Note that we have ignored convergence issues and other terms in the expression of $K_c$.

\subsection{}
I extend my sincere thanks to the Institute for Advanced Study and its permanent members for their hospitality, as the majority of this work was completed during my stay there during the special year 1983-1984 on $L$-functions.
I am particularly grateful to Langlands for his interest in this work.
Lastly, I owe a great debt of gratitude to Piatetski-Shapiro, whose deep understanding of Waldspurger's work proved invaluable.
A conversation with him was, in fact, the starting point of this work.

\section{Double cosets}

\subsection{}
In this paragraph $F$ will be any field of characteristic zero and $E$ be a quadratic extension of $F$.
We will denote $N(E:F)$ or simply $N$ the subgroup of norms of $E$ in the multiplicative group of $F$.
The set $X(E:F)$ or simply $X$ is defined as before.
Consider one of its elements $(G', T')$.
There exists a quaternion algebra $H$ over $F$ and a subfield $L$ of $H$ isomorphic to $E$ such that $G$ is the multiplicative group of $H$ and $T$ is that of $L$.
We are going to parametrize the double cosets $T' \backslash G' / T'$.
For this purpose let's choose an element $\varepsilon$ of the normalizer $N(T')$ of $T'$ which is not in $T'$.
Then every $h$ in $H$ can be uniquely written as:
\begin{equation}
h = h_1 + \varepsilon h_2, \qquad h_i \in L.
\end{equation}
On the other hand, if $z \to \bar{z}$ denotes the unique non-trivial $F$-automorphism of $L$ then:
\begin{equation}
\varepsilon z \varepsilon^{-1} = \bar{z}.
\end{equation}
The square $c = \varepsilon^2$ is in $Z'$, or, in other words, in $F$.
Moreover the class of $c$ modulo $N$ is determined by the isomorphism class of the pair $(G', T')$ and, conversely, determines it.

Define two involutions $j^+$ and $j^-$ of $H$ by the following fomulae:
\begin{equation}
j^{\pm}(h) = \bar{h}_1 \pm \varepsilon h_2, \quad h = h_1 + \varepsilon h_2.
\end{equation}
It is easy to verify that these are the only involutions of $H$ which induce the unique non-trivial $F$-automorphism of $L$.
For $h$ in $G$ we will set\footnote{$\tr$ is a trace map from $H$ to $F$, given as $\tr(h_1 + \varepsilon h_2) = h_1 + \bar{h}_1$.}
\begin{equation}
\label{eqn:1.1.4}
X'(h) = \frac{\frac{1}{2}\tr(hj^+(h))}{\frac{1}{2}\tr(hj^{-}(h))}.
\end{equation}
As the denominator of this faction is nothing but the reduced norm of $h$, $X(h)$ is a well-defined element of $F$ only depends on the double coset of $h$ modulo $T'$.
We also introduce the function $P'(h:T')$ or simply $P'(h)$ defined by
\begin{equation}
X'(h) = \frac{1 + P'(h)}{1 - P'(h)}
\end{equation}
or
\begin{equation}
P'(h) = ch_{2}\bar{h}_2{(h_{1}\bar{h}_1)}^{-1}, \qquad c = \varepsilon^2.
\end{equation}
Thus $P'$ is a function \textcolor{red}{with values in the projective line} which is constant on the double cosets of $T'$ in $G'$.
Note that according to the previous formula, if $P'(h)$ is neither zero nor infinite, then it is an element of the class $cN'$ determined by the tuple $(G', T')$.
Moreover $P'(h)$ can't be 1, otherwise $X'(h)$ would be infinity.
We will say that $h$ (or its double coset) is $T'$-singular if $P'(h)$ is zero or infinity, $T'$-regular otherwise.

\begin{proposition}
Two elements $h$ and $h'$ in $G'$ are in the same double coset of $T'$ if and only if $P'(h) = P'(h')$.
Moreover, if $x$ is in $cN$ and not 1 then there exists a $h$ in $G'$ such that $P'(h) = x$.
\end{proposition}
The proof is left to the reader.

\subsection{}
The following proposition justifies the use of the adjective $T'$-regular.
\begin{proposition}
Suppose $h$ is $T'$-regular.    
The relations
\[
sht = hz,\quad s \in T',\quad t\in T', \quad z \in Z'
\]
imply
\[
s \in Z',\quad t\in Z',\quad st = z.
\]
\end{proposition}
The proof is left to the reader.

\subsection{}
The above applies \emph{mutatis mutandis} to a tuple of the form $(G, T)$ where $G$ is the group $\GL(2)$ and $T$ is a maximal split torus, say the group diagonal matrices in $G$.
Then $H$ is the algrebra of 2 by 2 matrices, $L$ the subalgebra of diagonal matrices and we can take\footnote{In this case, the non-trivial automorphism on $L$ is a map that swaps two diagonal elements, i.e. $\smat{a}{0}{0}{d}\mapsto \smat{d}{0}{0}{a}$.}
\[
\epsilon = \begin{bmatrix}
    0 & 1 \\ 1 & 0
\end{bmatrix},
\quad c = 1.
\]
The functions $X$ and $P(\cdot :T)$ (or simply $P$) are defined as above.
In particular:
\[
P(h)  = bc{(ad)}^{-1}, \quad h = \begin{bmatrix}
    a & b \\ c & d
\end{bmatrix}.
\]
They are constant on the double cosets of $T$ in $G$.
Again $P$ cannot be 1.
We will also say that an element $h$ of $G$ is $T$-singular if $P(h)$ is zero or infinity, $T$-regular otherwise.
There are now 6 $T$-singular double cosets: the cosets on which $P$ takes the value zero:
\begin{equation}
    T, \quad Tn_+T, \quad Tn_{-}T, \quad \text{where } n_{+} = \begin{bmatrix}
        1 & 1 \\ 0 & 1
    \end{bmatrix},  \begin{bmatrix}
        1 & 0 \\ 1 & 1
    \end{bmatrix},
\end{equation}
and the classes on which $P$ takes the infinite value
\begin{equation}
\varepsilon T, \quad T \varepsilon n_{+} T, \quad T\varepsilon n_{-} T.
\end{equation}
Thus these classes cannot be distinguished from each other with $P$.

\subsection{}
However, $P$ distinguishes $T$-regular cosets:
\begin{proposition}
Let $h$ and $h'$ be $T$-regular element of $G$.
Then $h$ and $h'$ are in the same coset if and only if $P(h) = P(h')$.
If $x \in F$ is neither 1 nor 0, there exists an $T$-regular element $h$ such that $P(h) = x$.
\end{proposition}
The proof is left to the reader.
\section{Orbital integrals: compact torus}

\subsection{}
Let's keep the notations of section 1 but now assume that $F$ is a local field.
Then $T'/Z'$ is compact.
Choose a non-trivial additive character $\Psi$ of $F$.
Endow the additive group $F$ with the self-dual measure $dx$ for the character $\Psi$, the multiplicative group $F^\times$ with the measure $L(1, 1_F)|x|^{-1} dx$ (Tamagawa measure relative to $\Psi$).
Also, on the multiplicative group $E^\times$, we give a Tamagawa measure corresponds to the character $\Psi \circ \tr$.
Then these induce measures on $T'$ and $Z'$.
Equip $T'/Z'$ with the quotient measure.
Let $f'$ be a compactly supported smooth function on $G'/Z'$.
Let
\begin{equation}
H'(g':f':T') = \int_{T'/Z'} \int_{T'/Z'} f(sg't) dsdt.
\end{equation}
It is clear that $H'(g':f':T')$ only depends on the double coset of $g'$ modulo $T'$.
Let $x$ be an element of $F^\times$.
Define $H'(x:f':T')$ as $H'(g':f':T')$ if there's $g'$ in $G'$ such that $P'(g':T') = x$, and 0 otherwise.
Then we obtain a function $H'(f':T')$ on $F^\times$ and we are going to characterize the functions $H'$ on $F^\times$ which are of the form $H' = H'(f': T')$ for an appropriate function $f'$.

\subsection{}
Consider a function $H' = H'(f':T')$.
By definition $H'$ vanishes, therefore is smooth, on the complement of $cN$.
Consider a point $x$ of the form $P'(h':T')$.
As the norm is a submersive map from $E^\times$ to $F^\times$, the map $g' \to P'(g':T')$ is a fortiori submersive at the point $h$.
It follows that $H'$ is smooth at point $x$.
Finally suppose that 1 is in $cN$ (that is, the group $G'$ splits); then we can assume that $c = 1$.
We are going to show that $H'$ is zero near 1.

Since $f'$ is compactly supported modulo $Z'$, there exists a compact subset $C$ of $G'$ such that $H'(g':f':T') \neq 0$ implies $g' \in T'CT'$.
Hence it suffice to show the existence of a number $K$ such that  $|P'(g':T') - 1|>K$ for $g' \in T'CT'$.
Suppose there is no such number.
Then there would exist a sequence $g_i'$ of elements in $T'CT'$ such that $P'(g_i':T')$ tends to 1.
By enlarging $C$ and multiplying the elements of the sequence by elements of $T'$, we can assume that
\[
g_i' = 1 + \varepsilon t_i' = c_i z_i'
\]
with $t_i'$ in $T'$, $c_i$ in $C$ and $z_i'$ in $Z'$. So
\[
P'(g_i':T') = t_i' \bar{t}_i' = 1 + a_i
\]
and $a_i$ tends to zero.
On the other hand, we have:
\[
\det g_i' = -a_i = {(z_i')}^2 \det c_i.
\]
So $z_i'$ tends to zero, and same for $g_i'$.
Since the projection of $g_i'$ onto $L$ is 1, we get a contradiction.

\subsection{}
Let's examine the behavior of the function $H'$ near zero and near infinity.
We will show that there is a neighborhood $U$ of 0 in $F$ and a smooth function $A'$ on $U$ such that
\begin{align}
    H'(x) = A'(x) (1 + \eta(cx)), \quad x \in U \\
    2A'(0) = \mathrm{vol}(T'/Z') \int_{T'/Z'} f'(t) dt.
\end{align}
Similarly we will show that there exists a neighborhood $U$ of 0 in $F$ a smooth function $B'$ on $U$ such that
\begin{align}
    H'(x) = B'(x^{-1}) (1 + \eta(cx)), \quad x^{-1} \in U \\
    2B'(0) = \mathrm{vol}(T'/Z') \int_{T'/Z'} f'(\varepsilon t) dt.
\end{align}
Since $P'(\varepsilon g': T') = {P'(g':T')}^{-1}$ we have
\[
\int_{T'/Z'} \int_{T'/Z'} f'(s\varepsilon g't) dsdt = H'(x^{-1}:f':T')
\]
or
\[
H'(x^{-1}:f': T') = H'(x:f_0':T'), \quad f_0'(g') = f'(\varepsilon g').
\]
Hence it suffices to prove the assertions near zero.
We will consider non-archimedean case first.
Take a $x$ in $cN$.
Then $x = cl\bar{l}$ for some $l \in L$, hence $x = P(h)$ with $h = 1 + \varepsilon l$.
Then we can write
\[
H'(x:f':T')=\int_{T'/Z'} \int_{T'/Z'} f(t_1 (1 + \varepsilon l) t_2) dt_1 dt_2
\]
or with a change of variable
\begin{equation}
\label{2.3.5}
H'(x:f':T') = \int_{T'/Z'} \int_{T'/Z'} f\left(\left(1 + \varepsilon l \frac{\bar{t}_{1}}{t_ {1}} \right)t_{2} \right) dt_1 dt_2.
\end{equation}
Since $f'$ is smooth there exists an ideal $V$ of $E$ such that for $l$ in $V$ we have
\[
f'(g') = f((1 + \varepsilon l) g')
\]
for all $g'$.
Then there exists an ideal $U$ in $F$ such that $l \bar{l} \in U$ is equivalent to $l \in V$.
For $x$ in $cU$ we therefore have $H'(x) = 0$ if $x$ is not in $cN$; if $x$ is in $cN$ then $x = cl\bar{l}$ with $l$ in $V$ and by~\eqref{2.3.5}
\[
H'(x) = \mathrm{vol}(T'/Z') \int_{T'/Z'} f'(t) dt.
\]
Our assertion follows immediately from this.

Let move on to the archimedean case, i.e. $F = \mathbb{R}$ and $L = \mathbb{C}$.
Let $K(x) = H'(cx)$.
Let $V$ be a disk $\{z:z\bar{z} < a\}$ in $L$ such that $1 + \varepsilon V$ is contained in $G$.
Then the right hand side of~\eqref{2.3.5} defines a smooth function on $V$, say $C(l)$, depending only on the norm of $l$.
We have
\[
K(x) = \begin{cases} 0 & x < 0 \\ C(l) & x > 0 \text{ and } x = l \bar{l} \text{ for } x \in V.\end{cases}
\]
In particular, the restriction of $C$ to the real axis is even and smooth and we have
\[
K(x) = \begin{cases} 0 & x < 0 \\ C(y) & 0 <x < a\text{ and } x = y^2, y \in \mathbb{R}. \end{cases}
\]
Our assertion follows from the existence of a smooth function $D$ on $F$ such that $D(x) = K(x)$ for $a > x > 0$, which is a consequence of Whitney's theorem.\footnote{Whitney's approximation theorem: smooth function on $\mathbb{R}$ can be approximated by analytic functions.}

\subsection{}
The following properties characterizes $H'(f':T')$:
\begin{proposition}\label{prop:2.1}
Let $H'$ be a function on $F^\times$.
There exists a compactly supported smooth function $f'$ on $G'/Z'$ with $H' = H'(f':T')$ if and only if the following conditions hold:
\begin{enumerate}[label={(\arabic*)}]
    \item $H'$ vanishes on the complement of $cN$,
    \item $H'$ vanishes on a neighborhood of 1,
    \item There exists a smooth function $A'$ on a neighborhood of 0 in $F$ such that, for $x$ near 0, we have
    \[
    H'(x) = A'(x) (1 + \eta(cx)),
    \]
    \item There exists a smooth function $B'$ on a neighborhood of 0 in $F$ such that, when $|x|$ is sufficiently large,
    \[
    H'(x) = B'(x^{-1}) (1 + \eta(cx)) .
    \]
\end{enumerate}
If $f'$, $A'$ and $B'$ satisfy these conditions then
\[
2A'(0) = \mathrm{vol}(T'/Z') \int_{T'/Z'} f'(t)dt, \quad 2B'(0) = \mathrm{vol}(T'/Z') \int_{T'/Z'} f'(\varepsilon t) dt.
\]
\end{proposition}
We have just shown that the conditions (1) to (4) are necessary.
We will leave it to the reader to show that they are also sufficient.
The last assertion of the proposition has been proved above.
\section{Orbital integrals: split torus}

\subsection{}
In this section, let $F$ be a local field, $E$ a quadratic extension, $\eta$ the quadratic character of $F^\times$ associated with $E$, $G$ the group $\GL(2)$, and $T$ the subgroup of diagonal matrices.
We define the Tamagawa measure on $F^\times$ and its product measure on $F^\times \times F^\times$.
This induces a measure on $T$, and we give $T/Z$ the quotient measure.
For a compactly supported smooth function $f$ on $G/Z$ and a $T$-regular element $g \in G$, we define the orbital integrals as:
\begin{align}
    H(g:f:A) = H(g:f:1) = \int_{T/Z} \int_{T/Z} f(agb) \dd a \dd b \\
    H(g:f:\eta) = \int_{T/Z} \int_{T/Z} f(agb) \eta(\det b) \dd a \dd b.
\end{align}
The first integral depends only on the projection $P(g:T)$, and we denote it as $H(x:f:T)$ or $H(x:f:1)$, where $x$ is such that $P(g:T) = x$.
Additionally, we set $H(1:f:T) = H(1:f:1) = 0$.
For $x$ in $F$ different from 0 or 1, we define a matrix $g(x)$ as:
\begin{align}
    g(x) = \begin{bmatrix}
    1 & x \\ 1 & 1
    \end{bmatrix}.
\end{align}
Since $P(g(x)) = x$, $g(x)$ defines a section of the space of double cosets of $A$ in $G$.
We set $H(x:f:\eta) = H(g(x):f:\eta)$ for $x \neq 0, 1$, and define $H(x:f:\eta) = 0$ for $x = 1$.
Let $w$ be the matrix
\begin{align}
    w = \begin{bmatrix}
        0 & -1 \\ 1 & 0        
    \end{bmatrix}.
\end{align}
Let $N_+$ denote the group of strictly upper triangular matrices and $N_-$ the group of strictly lower triangular matrices.
Then $G$ can be covered by the two open sets:
\begin{align}
\label{3.1.5}
    G = TN_{+}N_{-} \cup TN_{+}wN_{+}.
\end{align}
Then we can write $f$ as a sum $f_1 + f_2$, where $f_1$ is supported on the first open set, and $f_2$ on the second.
We define the function $\phi(g)$ by:
\begin{align}
    \phi(g) = \int_{T/Z} f(ag) \dd a,
\end{align}
and similarly define $\phi_1$ and $\phi_2$ using $f_1$ and $f_2$, respectively.
These functions are left-invariant under $T$ and compactly supported modulo $T$. 
Furthermore, the functions $\Phi_1$ and $\Phi_2$ defined by:
\begin{align}
    \Phi_1(u, v) &= \phi_1 \left(\begin{bmatrix} 1 & u \\ 0 & 1\end{bmatrix}\begin{bmatrix}1 & 0 \\ v& 1\end{bmatrix}\right), \\
    \Phi_1(u, v) &= \phi_1 \left(\begin{bmatrix} 1 & u \\ 0 & 1\end{bmatrix}w\begin{bmatrix}1 & 0 \\ v& 1\end{bmatrix}\right)
\end{align}
which are compactly supported on $F \times F$.
Since
\begin{equation}
\begin{aligned}
    g(x) \bmat{a}{0}{0}{1} &= \bmat{a(1-x)}{0}{0}{1} \bmat{1}{a^{-1}{(1-x)}^{-1}x}{0}{1} \bmat{1}{0}{a}{1} \\
    &= \bmat{1-x}{0}{0}{a} \bmat{1}{a{(1-x)}^{-1}}{0}{1} w \bmat{1}{a^{-1}}{0}{1},
\end{aligned}
\end{equation}
for $x \neq 0, 1$, we have
\begin{equation}
\begin{aligned}
\label{3.1.10}
    H(x:f:T) &= \int_{T/Z} \Phi_1(a^{-1}(1-x)^{-1}x, a) \dd^{x}a \\
    &+ \int_{T/Z} \Phi_2 (a(1-x)^{-1}, a^{-1}) \dd^{\times}a. 
\end{aligned}
\end{equation}
To prove the convergence of these integrals, we can assume that $f$ is positive.
The right-hand side of \eqref{3.1.10} is compactly supported, ensuring the integral converges.
Similarly, for $x$ different from 0 and 1,
\begin{equation}
\begin{aligned}
\label{3.1.11}
    H(x:f:\eta) &= \int_{T/Z} \Phi_1(a^{-1}(1-x)^{-1}x, a) \eta(a) \dd^\times a \\
    &+ \int_{T/Z} \Phi_2(a(1-x)^{-1}, a^{-1}) \eta(a) \dd^{\times} a.
\end{aligned}
\end{equation}

\subsection{}
We will now study the properties of the functions $H(f:\eta)$.
The equation \eqref{3.1.11} already indicates that $H(x:f:\eta)$ is smooth at any $x \ne 0, 1$. 
On the other hand, if $\Phi_1$ and $\Phi_2$ are supported within the region where teh both absolute values $|x|$ and $|y|$ are less than some constant $C$, then, in the second integral, the condition $|a(1 - x)^{-1}| < C$ and $|a^{-1}| < C$ holds on the support of $\Phi_2$, which implies $C^{-2} < |1 - x|$ if the second integral is not zero.

Similarly, if the first integral is non-zero, we deduce $|(1-x)^{-1}x| < C^2$, which also implies $D < |1 - x|$ for a suitable constant $D > 0$.
It follows that $H(x:f:\eta)$ vanishes in the neighborhood around $x = 1$.
This shows that equation \eqref{3.1.11} holds for all nonzero $x$.

Next, let's examine $H(f:\eta)$ near $x = 0$.
In \eqref{3.1.11} the second integral is obviously a smooth at $x = 0$.
To analyze the first integral, we will apply the following lemma, leaving the proof as an exercise for the reader:
\begin{lemma}\label{lem:3.2}
    Let $\Phi$ be a Schwartz--Bruhat function of two variables defined on $F \times F$.
    There exist two Schwartz--Bruhat functions $A_1(x)$ and $A_2(x)$ on $F$, such that for all $x \neq 0$ in $F$, we have
    \[
    \int_{F^\times} \Phi(a^{-1}x, a)\eta(a) \dd^\times a = A_1(x) + A_2(x) \eta(x).
    \]
    If $\Phi$ is real valued and compactly supported, then we can take $A_1$ and $A_2$ to be compactly supported.
\end{lemma}

Returning to the first integral in equation \eqref{3.1.11}, and using the notations from the lemma, we find that the integral is equal to:
\begin{align}
    A_1(x(1-x)^{-1}) + A_2(x(1-x)^{-1})\eta(x(1-x)^{-1}).
\end{align}
When $x$ is sufficiently close to $0$, then $1 - x$ is a norm and $\eta(x(1-x)^{-1}) = \eta(x)$.
Moreover, the functions $A_i(x(1-x)^{-1})$ are smooth functions of $x$ in a neighborhood of 0 for $i = 1,2$.
As the second integral of \eqref{3.1.11} is clearly smooth at point $x = 0$, we conclude that, in a neighborhood of $0$, $H(x:f:\eta)$ can be expressed as
\begin{equation}
    H(x:f:\eta) = A_1(x) + A_2(x) \eta(x)
\end{equation}
where both $A_1$ and $A_2$ are smooth functions.

% To study $H(x:f:\eta)$ for large $|x|$ note that
Next, we study $H(x:f:\eta)$ for large $|x|$.
We have
\begin{align*}
    \varepsilon g(x) = g(x^{-1}) \bmat{1}{0}{0}{x}\quad \text{if } \varepsilon = \bmat{0}{1}{1}{0}.
\end{align*}
This leads to
\begin{align}
    H(x^{-1}: f: \eta) = H(x:f_0:\eta)\eta(x), \quad f_0(g) = f(\varepsilon g).
\end{align}
Thus, there exist functions $B_i$, $i = 1, 2$, smooth at $x = 0$, such that
\begin{align}
    H(x:f:\eta) = B_1(x^{-1}) + B_2(x^{-1})\eta(x)
\end{align}
for $x$ with sufficiently large $|x|$.

\subsection{}
In summary,
\begin{proposition}\label{prop:3.1}
Let $H$ be a function on $F^\times$, and suppose there exists a smooth, compactly supported function $f$ on $G/Z$ with $H(x:f:\eta) = H(x)$.
Then the following hold:
\begin{enumerate}
    \item $H$ is smooth on $F^\times$,
    \item $H$ vanishes on a neighborhood of 1,
    \item there exists a neighborhood $U$ of 0 and smooth functions $A_i$ on $U$ for $i = 1, 2$ such that, for $x \in U$, we have
    \[
    H(x) = A_1(x) + A_2(x) \eta(x),
    \]
    \item there exists a neighborhood $U$ of 0 and smooth functions $B_i$ on $U$ for $i = 1, 2$ such that, for $x$ with sufficiently large $|x|$, we have
    \[
    H(x) = B_1(x^{-1}) + B_{2}(x^{-1}) \eta(x).
    \]
\end{enumerate}
\end{proposition}


\subsection{}
We will now discuss the significance of the zero sets of the functions $A_i$ and $B_i$ from Proposition \ref{prop:3.1}.
To do so, let us first recall a few key facts.
If $\phi$ is a Schwartz--Bruhat function on $F$, the integral
\[
\int_{F^\times} \phi(x) |x|^{s} \dd^\times x,
\]
(or rather its analytic continuation) has a pole at $s = 0$.
The residue at this point takes the form $C\phi(0)$, where $C$ is a constant depending on the choice of the Haar measure on $F^\times$.
On the other hand, the integral
\[
\int_{F^\times} \phi(x) |x|^{s} \eta(x) \dd^\times x
\]
admits analytic continuation at $s = 0$, and its value at this point is written as:
\[
\int_{F^\times} \phi(x) \eta(x) \dd^\times x.
\]
Next, we define the following quantities:
\begin{align}
    H(n_+: f: \eta) &= \int_{F^\times}\int_{F^\times} f \left( \bmat{a}{0}{0}{1} \bmat{1}{b}{0}{1} \right) \eta(b) \dd^\times a \dd^\times b, \label{3.4.1} \\
    H(n_-: f: \eta) &= \int_{F^\times}\int_{F^\times} f \left( \bmat{a}{0}{0}{1} \bmat{1}{0}{b}{1} \right) \eta(b) \dd^\times a \dd^\times b, \label{3.4.2} \\
    H(\varepsilon n_+: f: \eta) &= \int_{F^\times}\int_{F^\times} f \left( \bmat{a}{0}{0}{1} \varepsilon \bmat{1}{b}{0}{1} \right) \eta(b) \dd^\times a \dd^\times b, \label{3.4.3} \\
    H(\varepsilon n_-: f: \eta) &= \int_{F^\times}\int_{F^\times} f \left( \bmat{a}{0}{0}{1} \varepsilon \bmat{1}{0}{b}{1} \right) \eta(b) \dd^\times a \dd^\times b. \label{3.4.4}
\end{align}
In general, these integrals are divergent but can be interpreted as meromorphic continuations.
For example, the first integral is the value at $s = 0$ of a meromorphic function which, for $\Re(s) > 0$, is given by the convergent integral
\begin{align}
    \int_{F^\times} \int_{F^\times} f\left( \bmat{a}{0}{0}{1} \bmat{1}{b}{0}{1} \right) \eta(b) |b|^{s} \dd^\times a \dd^\times b.
\end{align}
However, if $f$ is supported in the open set $TN_{+}wN_{+}$, the integrals \eqref{3.4.1} and \eqref{3.4.2} are convergent.
Specifically, the integral \eqref{3.4.1} is zero, as the sets $TN_{+}$ and $TN_{+}wN_{+}$ are disjoint.
% \textcolor{red}{On the other hand the intersection of $TN_{-}$ with a compact set contained in $TN_{+}wN_{+}$ is a disjoint compact of $T$.}
% It follows that in \eqref{3.4.2} the integrand is compactly support in $F^\times \times F^\times$ and the integral clearly converges.
Furthermore, the intersection of $TN_{-}$ with a compact set contained in $TN_{+}wN_{+}$ is compact and disjoint from $T$. This implies that in \eqref{3.4.2}, the integrand is compactly supported in $F^\times \times F^\times$, ensuring the integral converges.
% Similarly the integrals \eqref{3.4.3} and \eqref{3.4.4} converge if $f$ is supported in $TN_{+}N_{+}$.
Similarly, the integrals \eqref{3.4.3} and \eqref{3.4.4} converge if $f$ is supported in $TN_{+}N_{+}$.
\begin{proposition}\label{prop:3.2}
With the notation from Proposition \ref{prop:3.1}, we have:
\begin{align}
    H(n_+:f:\eta) &= A_2(0) \label{3.4.6} \\
    H(n_-:f:\eta) &= A_1(0) \label{3.4.7} \\
    H(\varepsilon n_+:f:\eta) &= B_1(0)\label{3.4.8} \\
    H(\varepsilon n_-:f:\eta) &= B_2(0). \label{3.4.9}
\end{align}
\end{proposition}
\begin{proof}
We will prove equations \eqref{3.4.6} and \eqref{3.4.7}. By the decomposition in \eqref{3.1.5}, it is sufficient to prove the result when $f$ is supported on either $TN_+ N_-$ or $AN_{+}wN_{+}$.
Then we have
\begin{align}
    H(x:f:\eta) = \int_{F^\times} \Phi(a(1-x)^{-1}, a^{-1}) \eta(a) \dd^\times a
\end{align}
where
\begin{align}
    \Phi(u, v) &= \phi\left(\bmat{1}{u}{0}{1} w \bmat{1}{a^{-1}}{0}{1}\right), \\
    \phi(g) &= \int_{T/Z} f(ag) da.
\end{align}
Since $H$ is smooth at $x = 0$, we deduce that $A_2 = 0$ and $A_1 = H(f: \eta)$.
Hence
\[
A_1(0) = \int_{F^\times} \phi\left(\bmat{1}{a}{0}{1} w \bmat{1}{a^{-1}}{0}{1}\right).
\]
Using the invariance of $f$ under the center, we can rewrite the integral as:
\[
A_1(0) = \int_{F^\times} \int_{F} f\left(b \bmat{a^2}{0}{a}{1}\right) \eta(a) \dd b \dd^\times a.
\]
A change of variable shows that this is equivalent to $H(n_-: f: \eta)$.
\eqref{3.4.6} also holds as both $A_2$ and $H(n_+:f:\eta)$ vanish.

Now assume that $f$ is supported in $TN_+ N_-$.
Then we have
\begin{align}
    H(x:f:\eta) = \int_{F^\times} \Phi(a^{-1}(1-x)^{-1}x, a) \eta(a) \dd^\times a
\end{align}
where
\begin{align}
    \Phi(u, v) &= \phi\left(\bmat{1}{u}{0}{1} \bmat{1}{0}{v}{1} \right), \label{3.4.14} \\
    \phi(g) &= \int_{T/Z} f(ag) \dd a. \label{3.4.15}
\end{align}
Recall the definitions of $A_1$ and $A_2$:
\begin{align}
    H(x) = A_1(x) + A_2(x)\eta(x),
\end{align}
on the other hand, lemma \eqref{lem:3.2} implies
\begin{align}
    \int_{F^\times} \Phi(a^{-1} x, a) \eta(a) \dd^\times a = C_1(x) + C_2(x) \eta(x).
\end{align}
Comparing with \eqref{3.4.4} we get $C_i((1-x)^{-1}x) = A_i(x)$.
Hence $C_i$ and $A_i$ vanish on the same set.
By taking the Mellin transform of the above equation we get
\[
\int_{F^\times} \int_{F^\times} \Phi(x, a) |x|^{s} \eta(a) |a|^{s} \dd^\times a \dd^\times x = \int_{F^\times} C_1(x) |x|^{s} \dd^\times x + \int_{F^\times} C_2(x) |x|^{s} \eta(x) \dd^\times x.
\]
Comparing the residue of both sides at $s = 0$ we get
\[
\int_{F^\times} \Phi(0, a) \eta(a) \dd^\times a = C_1(0).
\]
By \eqref{3.4.14} and \eqref{3.4.15} the left hand side is nothing but $H(n_-:f:\eta)$.
On the other hand, the right hand side equals to $A_1(0)$, hence \eqref{3.4.7} is proven, and \eqref{3.4.6} can be proved in a similar way.
Equations \eqref{3.4.8} and \eqref{3.4.9} follow from the equations \eqref{3.4.6} and \eqref{3.4.7}, applied to the function $f_0$ defined as $f_0(g) = f(\varepsilon g)$.
\end{proof}

\section{Matching functions}

\subsection{}
In this section, $E$ is a local field that is a quadratic extension of $F$, and $\eta$ is a quadratic character attached to $E$.
We again consider the pair $(G, T)$ of the group $\GL(2)$ and the subgroup of diagonal matrices, and the set $X = X(E:F)$ that contains two elements $(G_i', T_i')$, $i =1, 2$, where $G_1'$ split.
Let $f$ be a compactly supported smooth function on $G/Z$ and $f_i'$, $i=1, 2$ compactly supported smooth functions on $G_i' /Z_i'$.
We say that $f$ and the pair $(f_1', f_2')$ are matched if the following condition holds:
For all $x \in F \backslash \{0, 1\}$, choose $i$ and $g' \in G_i'$ such that $x = P'(g':T_i')$ ($i=1$ if $x$ is a norm of $E$, and $i=2$ otherwise).
Then
\[
H(x:f:\eta) = H'(g':f_i':T_i').
\]
\begin{proposition}\label{prop:4.1}
For a given function $f$, there exists a pair of functions $(f_1', f_2')$ that matches $f$.
Moreover, we have
\begin{align*}
    \mathrm{vol}(T_i' / Z_i') \int_{T_i'} f_i'(t_i') \dd t_i' &= H(n_+:f:\eta) \pm H(n_-:f:\eta) \\
    \mathrm{vol}(T_i' / Z_i') \int_{T_i'} f_i'(\varepsilon t_i') \dd t_i' &= H(\varepsilon n_+:f:\eta) \pm H(\varepsilon n_-:f:\eta)
\end{align*}
where the sign is $+$ for $i=1$ and $-$ for $i = 2$.
\end{proposition}
\begin{proof}
This result follows from the propositions \ref{prop:2.1}, \ref{prop:3.1}, and \ref{prop:3.2}.
\end{proof}

\subsection{}
When $F = \mathbb{R}$, let $K$ denote the orthogonal subgroup in $G$.
We define $U$ to be the set of pairs $(f_1', f_2')$ that match a compactly supported smooth function $f$ on $G/Z$, where $f$ is $K$-finite if $F = \mathbb{R}$.

Let $U_1$ (resp. $U_2$) be a projection of $U$ to the first and second components, respectively.
Then the sets $U_i$ is \emph{dense} in the following sense.
\begin{proposition}\label{prop:4.2}
    Let $\phi'$ be a continuous function on $G_i' / Z_i'$ that is bi-invariant under $T_i'$. 
    If $\int_{G_i' / Z_i'} \phi'(g') f_i'(g') \dd g' = 0$ for all $f_i' \in U_i$, then $\phi'$ must be a zero function.
\end{proposition}
The proof of \ref{prop:4.2} will occupy the remainder of this section.

\subsection{}
Let $f$ be a compactly supported smooth function on $G/Z$.
Then
\begin{align}
\label{4.3.1}
    \int_{G/Z} f(g)\dd g = c \int_{x \in F^\times} H(x:f:A)|1 - x|^{-2} \dd x
\end{align}
where $c$ is a constant that does not depend on $f$.
We also need an estimate for the functions $H(x:f:T)$, wehre $f$ is compactly supported continuous on $G/Z$.

\begin{lemma}
Let $f$ be a compactly supported function on $G/Z$, and $H(x) = H(x:f:T)$.
Then $H$ vanishes in a neighborhood of 1 and behaves as $O(\log |x|)$ for both small or large $|x|$.
\end{lemma}
\begin{proof}
The proof is similar to that given in Section 3.2, except that Lemma \ref{lem:3.2} is replaced by the following assertion: if $\Phi$ is a Schwarts--Bruhat function of two variables, then there exist two Schwartz--Bruhat functions $A_1(x)$ and $A_2(x)$, such that
\begin{align*}
    \int_{F^\times} \Phi(a^{-1}x a) \dd^\times a = A_1(x) + A_2(x) \log |x|.
\end{align*}
We need integration formulas for groups $G_i'$ analogous to \eqref{4.3.1}:
\begin{align}
    \int_{G_1' / Z_1'} f_1'(g) \dd g &= c_1 \int_{0}^{\infty} H'(x:f_1':T_1') |1 - x|^{-2} \dd x, \label{4.3.3} \\
    \int_{G_2' / Z_2'} f_2'(g) \dd g &= c_2 \int_{-\infty}^{0} H'(x:f_2':T_2') |1-x|^{-2} \dd x, \label{4.3.4}
\end{align}
where $c_i$ is a constant and $f_i'$ is a compactly supported continuous function on $G_i' / Z_i'$, for $i = 1, 2$.
\end{proof}

\subsection{}
\emph{Proof of the proposition \ref{prop:4.2}}.
Suppose $i = 1$.
Let $H'(x) = H'(x:\phi':T_1')$.
In particuler, $H'(x) = 0$ if $x$ is not a norm.
We will compute the integrals up to multiplicative constants.
Suppose $f$ and $(f_1', f_2')$ match with $f'$ being $K$-finite if $F = \mathbb{R}$.
By \eqref{4.3.3},
\begin{align*}
    \int_{G_1' / Z_1'} \phi'(g_1') f_1'(g_1') \dd g_1' = \int_{0}^{\infty} H'(x) H'(x:f_1':T_1') |1 - x|^{-2} \dd x.
\end{align*}
Define $\phi_0$ as a function on $G$ by
\begin{align*}
    \phi_0(g) = H'(x) \eta(\det b) \quad \text{if } g = ag(x) b.
\end{align*}
By the properties of $H'$ and the integration formula~\eqref{4.3.1} $\phi_0$ is locally integrable and
\begin{align*}
    \int_{G/Z} \phi_0(g) f(g) = \int_{F^\times} H'(x) H'(x:f:\eta) |1 - x|^{-2} \dd x.
\end{align*}
From $H(x:f:\eta) = H'(x:f_1':T_1')$, when $x$ is a norm, we find that
\begin{align*}
    \int_{G/Z} \phi_0(g) f(g) \dd g = \int_{G_1' / Z_1'} \phi'(g_1') f_1'(g_1') \dd g_1'.
\end{align*}
By assumption, the second integral vanishes.
Therefore $\phi_0$ is orthogonal to any smooth function (resp. any $K$-finite smooth function if $F$ is real), implying that $\phi_0$ must be zero.
Consequently, $H'$ must also be zero: since $\phi'$ is bi-invariant under $T_1'$, it is completely determined by $H'$ and we get $\phi' = 0$.

\section{Orbital integrals: unramified case}

\subsection{}
In this section, $F$ is a non-archimedean local field and $E$ is an unramified quadratic extension of $F$.
Assume that the residual characteristic of $F$ is not 2 and the order of character $\psi$ is 0.\footnote{Order of a character $\psi$ is a smallest integer $r\geq 0$ such that $\varpi ^r R \subseteq \ker \psi$.}
Consider a tuple $(G, T)$ of a group $\GL(2)$ and a diagonal subgroup $T$.
We denote by $R$ the ring of integers of $F$, $\mathfrak{p}$ a maximal ideal of $R$, $\varpi$ a  uniformizer and $K = \GL(2, R)$.
The set $X = X(E:F)$ reduces to a set of two elements $(G_1', T_1')$ and $(G_2', T_2')$.
Now suppose $G_1' = G$ and $T_1'$ is contained in the subgroup $ZK$.
We will simply write $T'$ for $T_1'$.
The measures of $T\cap K / Z \cap K$ and $T' \cap K / Z\cap K$ are therefore equal to 1.
The goal of this section is to prove the following proposition.

\begin{proposition}\label{prop:5.1}
Lef $f$ be a $K$ bi-invariant compactly supported function on $G/Z$.
The $f$ matches with the pair $(f, 0)$.
Also, 
\begin{align*}
    H(n_+: f: \eta) = H(n_-: f: \eta) = \frac{1}{2} \mathrm{vol}(T'/Z) \int_{T'/Z} f(t')dt'.
\end{align*}
\end{proposition}
It will be convenient to consider functions with compact support on $G$ rather than functions with compact support on $G/Z$.
Of course the measures of the sets $T\cap K$, $T'\cap K$, and $Z\cap K$ are therefore equal to 1.
If $f$ is a $K$ bi-invariant function with compact support on $G$, then we set
\begin{align}
    H'(g:f:T') = \int_{T'/Z} \int_{T'} f(s'gt') ds'dt'.
\end{align}
Since $T'$ is contained in $ZK$ this reduces to
\begin{align}
    H'(g:f:T') = \int_{Z} f(zg) dz.
\end{align}
Likewise, we define
\begin{align}
    H(g:f:\eta) = \int_{T/Z} \int_{T} f(agb) \eta(\det b) dadb.
\end{align}
We write $H(x:f:\eta)$ again for $H(g(x):f:\eta)$.
Then we have to prove the following identities
\begin{align}
    H(x:f:\eta) &= \int_{Z} f(zg) dz \quad \text{if } v(x) \text{ is even and } P(g:T') = x \label{eqn:5.1.4}\\
    H(x:f:\eta) &= 0 \quad \text{if } v(x) \text{ is odd}. \label{eqn:5.1.5}
\end{align}
By linearity we can assume that $f$ is either the characteristic function $f_0$ of $K$, or the characteristic function $f_m$ of the set
\begin{align}
    K \bmat{\varpi^m}{0}{0}{1}K, \quad m > 0.
\end{align}
Note that $f_m(g) \neq 0$ if and only if the following conditions hold:
\begin{itemize}
    \item the entries of $g$ are integers,
    \item $v(\det g) = m$,
    \item at least one of the entries of $g$ is a unit.
\end{itemize}
Note that the condition trivially holds when $m = 0$.

\subsection{}
We will first compute $H(x:f_m:\eta)$ which, for simplicity, denote by $H(x:m)$.
Let's assume $m >0$ first.
\begin{proposition}\label{prop:5.2}
Suppose $m > 0$.Then $H(x:m)$ is given by the following formulas:
\begin{enumerate}
    \item if $v(x)$ is odd then $H(x:m)=0$.
    \item if $v(x)$ is even then $H(x:m)=0$, unless $v(x)=0$ and $v(1-x)=m$ in which case $H(x:m )=1$.
\end{enumerate}
\end{proposition}
\begin{proof}
We will use the following lemma:
\begin{lemma}
Let
\[
S = \sum_{i, j} (-1)^{i+j}
\]
where the summation is over all the pairs of integers $(i, j)$ on the edge of the rectangle defined by the inequalities
\[
0 \leq i \leq P, \quad 0 \leq j \leq Q.
\]
Then $S$ is given by the following formulas:
\begin{enumerate}
    \item if $PQ > 0$ then $S =0$,
    \item if $P = 0$ and $Q >0$, then $S=1$ if $Q$ is even and $S=0$ if $Q$ is odd,
    \item if $Q=0$ and $P >0$, the $S=1$ if $P$ is even and $S=0$ if $P$ is odd,
    \item if $P=Q=0$ then $S=1$.
\end{enumerate}
\end{lemma}
We not prove the proposition.
We write $\mathrm{Mat}[a, b, c, d]$ for the matrices with entries $a, b, c, d$.
With this notation we have
\begin{align}
\label{5.2.7}
    H(x:m) = \sum_{i, j, k} f_{m}(\mathrm{Mat}[\varpi^{i+k}, x\varpi^{j+k}, \varpi^{i}, \varpi^{j}]) (-1)^{i+j},
\end{align}
where the sum is over all triples of integers $(i, j, k)$.
As the determinant of the matrices in~\eqref{5.2.7} has valuations equal to $i+j+k+v(1-x)$, from the condition $v(\det g) = m$ we can restrict ourselves to triples $(i, j, k)$ with 
\begin{align*}
    i+j+k+v(1-x) = m.
\end{align*}
This allows us to eliminate $k$ and, by the previous conditions on the integrality of entries of $g$,
\begin{align}
    H(x:m) = \sum_{i, j}(-1)^{i+j}
\end{align}
where the summation is over all the pairs of integers $(i, j)$ such that
\begin{gather}
    0 \leq i \leq m - v(1-x) + v(x) \\
    0 \leq j \leq m - v(1-x) \\
    ij(m-v(1-x)+v(x)-i)(m-v(1-x)-j)=0.
\end{gather}
The sum is empty and $H(x:m)$ is zero unless
\begin{align}
\label{5.2.12}
    m - v(1-x) \geq 0 \quad \text{and} \quad m - v(1-x) + v(x) \geq 0.
\end{align}
Suppose~\eqref{5.2.12} holds.
Then we can apply the lemma, and we have $H(x:m)=0$ unless
\begin{align}
    (m-v(1-x))(m-v(1-x)+v(x)) =0.
\end{align}
Then the proposition follows from elementary calculations.
\end{proof}

\subsection{}
Let's compute $H(x:0)$.
\begin{proposition}\label{prop:5.3}
    $H(x:0)$ is given by the following formulas:
    \begin{enumerate}
        \item if $v(x)$ is odd then $H(x:0) =0$,
        \item if $v(x)$ is even then $H(x:0) =1$, unless $v(x)=0$ and $v(1-x)>0$ in which case $H(x:0)=0$.
    \end{enumerate}
\end{proposition}
\begin{proof}
We have
\begin{align}
    H(x:0) = \sum_{i, j, k} f_{0}(\mathrm{Mat}[\varpi^{i+k}, x\varpi^{j+k}, \varpi^i, \varpi^j]) (-1)^{i+j},
\end{align}
where the sum is over all triples of integers $(i, j, k)$.
As above, based on the conditions on $g$ to be $f_0(g)\neq 0$, we can eliminate $k$ and write
\begin{align}
    H(x:0) = \sum_{i, j} (-1)^{i+j}
\end{align}
where the sum is over all pairs of integers $(i, j)$ such that
\begin{align}
    0 \leq i \leq v(x) - v(1-x) \\
    0 \leq j \leq -v(1-x).
\end{align}
Then the proposition follows from elementary calculations.
\end{proof}

\subsection{}
We will compute $\int_Z f_m(zg) dz$.
It only depends on $x = P(g:T')$ and we denote $H'(x:m:T')$ for its value.
Recall that by definition $x$ is a norm, in other words the valuation of $x$ is even.
We will start with the case $m > 0$.
\begin{proposition}\label{prop:5.4}
    Suppose $m > 0$.
    Then $H'(x:m:T')=0$, unless $v(x) = 0$ and $v(1-x)=m$ in which case $H'(x:m:T')=1$.
\end{proposition}
\begin{proof}
We can assume that $E$ is an extension generated by the square root of $\tau$, where $\tau$ is a unit.
Then we can take for $T'$ the multiplicative group of the following algebra
\begin{align}
    L = \left\{\bmat{a}{b}{b\tau}{a}\right\}
\end{align}
and $\varepsilon$ is a matrix
\begin{align}
    \varepsilon = \bmat{1}{0}{0}{-1}.
\end{align}
Let's compute $H'(x:m:T')$ for $x = P'(g:T')$.
We can assume that 
\begin{align}
    g = \bmat{1}{0}{0}{1} + \varepsilon \bmat{u}{v}{v\tau}{u}.
\end{align}
so $\det g = 1 - x$ and $x = y^2 - v^2 \tau$.
We have
\begin{align}
    H'(x:m:T') = \sum_{k}f_{m}(\varpi^k g)
\end{align}
and
\begin{align}
    \label{5.4.5}
    \varpi^k g = \bmat{\varpi^k(1+u)}{\varpi^k v}{-\varpi^k v\tau}{\varpi^k(1-u)}.
\end{align}
By the integrality condition of entries of $g$, this sum consists of a single term where $k$ is determined by the equation
\begin{align}
    \label{5.4.6}
    k = \frac{1}{2}(m - v(1-x)).
\end{align}
In particuler $H'(x:m:T') = 0$ or 1.
By the integrality conditions again and $\det g = m$, $H(x:m:T') = 1$ if and only if the followings hold:
\begin{itemize}
    \item $m = v(1-x) \,\mathrm{\mod}\,2$,
    \item the entries of the matrix in~\eqref{5.4.5} with $k$ given by~\eqref{5.4.6} are integers,
    \item at least one of the entries of this matrix is a unit.
\end{itemize}
First assume $v(x) <0$.
We have $v(1-x) = v(x)$.
Since $x =u^2 - v^2 \tau$ and $\tau$ is not a square $v(x)$ is even.
Then $H'(x:m:T')=0$ unless $m$ is even.
So let's assume that this is the case.
We can write
\begin{align*}
    u = u_0 \varpi^{v(x) /2}, \quad v = v_0 \varpi^{v(x) / 2},
\end{align*}
where $u_0, v_0$ are integrals, with at least one being a unit.
Then the entries of the matrix~\eqref{5.4.5} are:
\begin{align*}
    \varpi^{(m  - v(x)) / 2} (1 + u_0 \varpi^{v(x) / 2}), \quad \varpi^{m/2}v_0 \\
    -\varpi^{m/2} v_0 \tau, \quad \varpi^{(m - v(x))/2} (1 - u_0 \varpi^{v(x) / 2}).
\end{align*}
All of these are in $\mathfrak{p}$ so $H'(x:m:T') = 0$.

Supose $v(x) > 0$. We have $v(1 - x) = 0$.
From $m \equiv v(1-x)\,(\mathrm{\mod}\,2)$ we have $H'(x:m:T') = 0$ unless $m$ is even.
Assume $m$ is even.
Since $k = m / 2$, $u$ and $v$ are integral.
The entries of the matrix in~\eqref{5.4.5} are
\begin{align*}
    \varpi^{m/2} (1+u), \quad \varpi^{m/2} v, \\
    -\varpi^{m/2} v\tau, \quad \varpi^{m/2} (1 - u).
\end{align*}
All of these are in $\mathfrak{p}$ so $H'(x:m:T') =0$.

Lastly, assume $v(x) = 0$. We have $v(1 - x) \geq 0$.
If $m - v(1-x)$ is odd, then $H'(x:m:T') =0$. Let's assume that $m - v(1-x)$ is even.
Then the entries of the matrix in~\eqref{5.4.5} are
\begin{align*}
    \varpi^{(m - v(1-x))/2} (1 +u), \quad \varpi^{(m - v(1-x))/2} v \\
    -\varpi^{(m - v(1-x)) /2} v\tau, \quad \varpi^{(m - v(1-x)) / 2} (1 - u).
\end{align*}
Since $x$ is a unit, $u$ and $v$ are both integral and at least one is a unit.
If $1 + u$ and $1 - u$ are both in $\mathfrak{p}$ we would have $2 \in \mathfrak{p}$, a contradiction.
So at least one of the numbers $1 +u$ and $1 - u$ is a unit.
If $H(x:m:T')$ is not zero, integrality condition gives $m = v(1-x)$.
The entries of the matrix~\eqref{5.4.5} are therefore reduced to
\begin{align*}
    1+u, \quad v, \quad -v\tau, \quad 1 -u.
\end{align*}
These are all integral and at least one of them is a unit. Hence $H'(x:m:T') = 1$.

So we have computed $H'$ completely and the proposition follows.
\end{proof}

\subsection{}
Let's compute $H'(x:0:T')$.
Recall that $v(x)$ is even.
\begin{proposition}\label{prop:5.5}
$H'(x:0:T')=1$, unless $v(x)=0$ and $v(1-x)>0$ in which case $H'(x:0:T')=0$.
\end{proposition}
\begin{proof}
As above we have
\begin{align}
    H'(x:m:T') = \sum_{k} f_{0}(\varpi^k g).
\end{align}
The sum has at most one term whose index $k$ is given by
\begin{align}
\label{eqn:5.5.2}
    k = - \frac{v(1-x)}{2}.
\end{align}
In particular, $H'(x:0:T')=0$ or $1$.
Moreover $H'(x:0:T')=1$ if and only if $v(1-x)$ is even and the matrix
\begin{align}
\label{eqn:5.5.3}
    \varpi^k g = \bmat{\varpi^k(1+u)}{\varpi^k v}{-\varpi^k v\tau}{\varpi^k(1-u)}
\end{align}
with $k$ given by~\eqref{eqn:5.5.2} is in $\GL(2, R)$.

Assume that $v(x) < 0$ and $v(1-x)$ is even.
Then $v(1-x) = v(x)$, $v(x)$ is even and
\begin{align*}
    u = u_0 \varpi^{v(x)/2}, \quad v = v_0 \varpi^{v(x) / 2}
\end{align*}
where $u_0$ and $v_0$ are integral, at least one being unit.
Then the entries of the matrix~\eqref{eqn:5.5.3} are the numbers
\begin{align*}
    \varpi^{-v(x) / 2} + u_0, \quad v_0, \quad -v_0 \tau, \quad \varpi^{-v(x) / 2} - u_0.
\end{align*}
They are integral.
As the determinant of the matrix~\eqref{eqn:5.5.3} is a unit according to the choice of $k$ the matrix~\eqref{eqn:5.5.3} is in $\GL(2, R)$ and $H'(x:0:T')=1$.

Suppose $v(x) \geq 0$ and $v(1-x)=0$ (of course $v(x) > 0$ leads to $v(1-x)=0$).
Then $k=0$ and the entries of the matrices~\eqref{eqn:5.5.3} reduce to the numbers
\begin{align*}
    & \varpi^{-v(1-x) / 2} (1+u), \quad \varpi^{-v(1-x)/2}v \\
    & \varpi^{-v(1-x)/2}v\tau, \quad \varpi^{-v(1-x)/2}(1-u).
\end{align*}
As $1+u$ or $1-u$ is a unit at least one of these numbers is not integral so~\eqref{eqn:5.5.3} is not in $\GL(2,R) $ and $H'(x:0:T')=0$.
\end{proof}

\subsection{}
By comparing the propositions~\eqref{prop:5.2},~\eqref{prop:5.3}, 
~\eqref{prop:5.4} and~\eqref{prop:5.5} we see that we have proved the identities~\eqref{eqn:5.1.4} and~\eqref{eqn:5.1.5}.
This therefore completes the proof of the first assertion of proposition~\eqref{prop:5.1}.
The second then follows from proposition~\eqref{prop:4.1}.

\subsection{}
To establish the convergence of the global orbital integrals we will need an additional result, the proof of which we will leave to the reader.

\begin{lemma}
Suppose $h$ in $KZ$ and let $x = P(h:T)$.
Suppose $v(x) = 0$ and $v(1-x)=0$.
Then the relation
\begin{align*}
ahb \in KZ, \quad a \in T, \quad b \in T
\end{align*}
leads to
\begin{align*}
a \in Z(K \cap T), \quad b \in Z(K \cap T).
\end{align*}
\end{lemma}
The lemma implies the following proposition.
\begin{proposition}
Let $f$ be the characteristic function of $KZ$.
Suppose $E$ is an unramifeid quadratic extension and $T'$ contained in $KZ$.
Let $h$ be an element of $KZ$ and $x = P(h:T)$.
If $x$ and $1-x$ are units then
\begin{align*}
H(h:f:T) =1, \quad H(h:f:\eta) = 1.
\end{align*}
\end{proposition}
\section{Review on local representations of $\GL(2)$}

\subsection{}
Let $F$ be a local field and $E$ a quadratic extension of $E$.
We will consider again the set $X$ which is reduced to two elements $(G_1', T_1')$ and $(G_2', T_2')$, with say $G_1'$ splits.
It will be convenient to use the following result:
\begin{proposition}
Let $\pi'$ be an irreducible unitary representation of $G_i' / Z_i'$.
Then the dimension of the space of continuous and $T_i'$-invariant linear functionals on the space of smooth vectors of $\pi'$ is at most one.
Moreover such a functional is given by the inner product with a smooth $T_i'$-invariant vector.
\end{proposition}
If $F$ is non-archimedean then the assertion on the dimension is proven in~\cite{waldspurger1991correspondances}, Proposition 9.
It is well-known for $F = \mathbb{R}$.
The rest of the proposition is obvious.

\subsection{}
Likewise:
\begin{proposition}
Let $\pi'$ be an infinite-dimensional irreducible unitary representation of $G_1' / Z_1'$.
Then the dimension of the space of $T$-invariant linear functionals (resp.\ invariant under the character $\eta \circ \det$) on the space of smooth functions of $\pi'$ is one.
\end{proposition}
These are the propositions 9 and 10 of \cite{waldspurger1980correspondances}.

\subsection{}
For $i = 1, 2$, consider irreducible unitary representations $\pi_i'$ of $G_i' / Z_i'$.
Assume that the tuple $(\pi_1', \pi_2')$ satisfies the conditions of the theorem (15.1) of~\cite{jacquet2006automorphic}; in particular $\pi_1'$ is a discrete series representation.

\begin{proposition}
The representations $\pi_i'$ cannot both have a nonzero invariant vector under the group $T_i'$.
\end{proposition}
If $F$ is non-archimedean then our assertion is found in Theorem 2 of~\cite{waldspurger1991correspondances}.
It is well-known for $F = \mathbb{R}$.
\section{Global orbital integrals: split torus}

\subsection{}
In the rest of this work $F$ will be a number field and $E$ a quadratic extension of $F$, $\eta$ the quadratic character of the ideles of $F$ attached to $E$.
In this and the next section we will consider the pair $(G, T)$ and a compactly supported smooth function $f$ on $G(\Aa_F) /Z(\Aa_F)$.
Denote $K_c$ for the cuspidal kernel attached to $f$.
Let $\phi_j$ be an orthonormal basis of the space of cusp forms of the group $G/Z$.

Then, by definition
\begin{align}
    K_c(x, y) = \sum_{j} \rho(f) \phi_{j}(x) \overline{\phi}_{j}(y)
\end{align}
and
\begin{align}
    \rho(f) \phi(x) = \int f(g) \phi(xg) \dd g.
\end{align}
In this and the following section we will give a nice expression of the integral
\begin{align}
    \int_{[T]} \int_{[T]} K_c(a, b) \eta(\det b) da db
\end{align}
We have chosen a nontrivial character $\psi$ of $\Aa_F / F$.
Then for each place $v$ we have the Tamagawa measure attached to $\psi_v$ induced on $A_v$ and $Z_v$.
We therefore have the product measure on $A(\Aa_F/F)$ and the quotient measure on $T(\Aa_F) / Z(\Aa_F)$.
We will denote by $S$ a finite set of places containing the infinite places, the ramified places in $E$ and the places of residual characteristic 2.
For each place $v$ of $F$ we will denote $K_v$ for the usual maximal compact subgroup.
In particular $K_v = \GL(2, R_v)$ if $v$ is finite.
We will take the function $f$ product of local functions $f_v$ which are $K_v$-finite at all places.
We will assume that $f_v$ is bi-$K_v$-invariant for all $v$ not in $S$.
We have a decomposition of $K_c$ as a following sum:
\begin{align}
    K_c(x, y) = \sum_{\gamma \in G(F) / Z(F)} f(x^{-1}\gamma y) - K_{\mathrm{sp}}(x, y) - K_{\mathrm{ei}}(x, y),
\end{align}
where $K_{\mathrm{sp}}$ denotes the special kernel and $K_{\mathrm{ei}}$ the Eisenstein kernel (the definition will be recalled later).
We can write the first term of this sum as the sum of two other terms $K_r$ and $K_s$ where
\begin{align}
    K_{r}(x, y) &= \sum_{\gamma \in G(F) / Z(F)} f(x^{-1}\gamma y), \quad \gamma\text{ is }T\text{-regular} \\
    K_{s}(x, y) &= \sum_{\gamma \in G(F) / Z(F)} f(x^{-1}\gamma y), \quad \gamma\text{ is }T\text{-singular}
\end{align}
Then $K_c$ can be written as
\begin{align}
    K_c = K_r + K_s - K_{\mathrm{sp}} - K_{\mathrm{et}}.
\end{align}

\subsection{}
We first consider the integral of $K_r$.
Any $T$-regular element $\gamma$ of $G(F) /Z(F)$ can be uniquely written in the form
\begin{align}
    \gamma = \alpha g(\xi) \beta, \quad \alpha, \beta \in T(F) /Z(F)\text{ and } \xi \neq 0, 1
\end{align}
(cf. (3.1.3) for the notation and \S 1)
This implies
\begin{align}
    \int_{T(\Aa_F) / Z(\Aa_F)} \int_{T(\Aa_F) / Z(\Aa_F)} K_r(a, b) \eta(\det b) da db = \sum_{\xi \in F^\times - \{1\}} H(\xi: f: \eta),
\end{align}
where
\begin{align}
\label{eqn:7.2.3}
    H(\xi: f: \eta) = \int_{T(\Aa_F) / Z(\Aa_F)}\int_{T(\Aa_F)/Z(\Aa_F)} f(ag(\xi)b) \eta(\det b) da db.
\end{align}
Let's justify our formal computations.
Assme that the support of $f$ has only a finite number of regular classes.
The function $X$ introduced in the section 1 (equation~\eqref{eqn:1.1.4}) defines a continuous function of the group $G(\Aa_F)/Z(\Aa_F)$ over $\Aa_F$.
Hence it only takes a finite number of values on the intersection of the support of $f$ with the set of rational points:
the same is therefore true for the function $P(\cdot:A)$, which gives us our assertion.
On the other hand, each of the integrals~\eqref{eqn:7.2.3} converges absolutely: it suffices to prove it for the integral
\begin{align}
    \label{eqn:7.2.4}
    H(\xi:f:T) = \int_{T(\Aa_F) / Z(\Aa_F)} \int_{T(\Aa_F) / Z(\Aa_F)} f(ag(\xi)b) dadb.
\end{align}
Each of the local integrals $H(\xi: f_v: T_v)$ converges; almost all are equal to 1 (cf. (5.7)).
Hence~\eqref{eqn:7.2.4} converges.
It is true for (3) and (3) is the product of the corresponding local integrals
\begin{align}
    H(\xi:f:\eta) = \prod_v H(\xi: f_v: \eta_v).
\end{align}
All but finitely many local integrals in the product are equal to 1 (cf. (5.7)).

\subsection{}
Consider the integral of $K_s$.
It is not absolutely convergent, but it is weakly convergent in the following sense.
Let $c$ be a number greater than 1.
Define
\begin{align}
    \label{eqn:7.3.1}
    \int_{c^{-1}}^{c} \int_{c^{-1}}^{c} K_{s}(a, b) \eta(\det b) da db,
\end{align}
the integral of $K_s(a, b)\eta(\det b)$ over the set of pairs $(a, b)$ satisfying $c^{-1} < |a_1 / a_2| < c$, $c^{-1} < |b_1 / b_2| < c$; where $a_1$ and $a_2$ are the diagonal entries $a$ (and similar for $b_1$ and $b_2$).
The integral exists since it is over a compact set.
We will see that the integral~\eqref{eqn:7.3.1} converges as $c$ goes to infinity.
Then we define the weak integral of $K_s(a, b) \eta(\det b)$ as the limit.
We have seen in (1.3) that there are 6 singular double cosets in $T$, namely the double cosets of the following elements: $e, n_+, n_-, \varepsilon, \varepsilon n_+, \varepsilon n_-$.
Let's number them from 1 to 6.
Then we have a decomposition of $K_s$ into 6 terms $K_i$, $1\leq i\leq 6$, where $K_i$ is the sum of the $f(x^{-1} \gamma y)$ for all $\gamma$ in the $i$-th double coset.
Let's study the integral of $K_1$ for example.
We have
\begin{align*}
    K_1(x, y) = \sum_{\alpha \in T(F) / Z(F)} f(x^{-1} \alpha y).
\end{align*}
We have
\begin{align*}
    \int_{c^{-1}}^{c} \int_{c^{-1}}^{c} K_{1}(a, b) \eta (\det b) da db = \int_{c^{-1}}^{c} \int_{c^{-1}}^{c} f(ab) \eta (\det b) da db,
\end{align*}
in the left integral $a$ and $b$ vary in the compact subset of $[T]$ defined above;
in the right integral $b$ still varies in the compact subset of $[T]$ defined by $c^{-1} < |b_1 / b_2| < c$, but $a$ varies in the subset of $T(\Aa_F)/Z(\Aa_F)$ defined by $c^{-1} < |a_1/a_2| < c$.\footnote{The integral with respect to $a$ over $[T] = T(\Aa_F) / T(F)Z(\Aa_F)$ and the summation over $T(F)$ are combined as an integral over $T(\Aa_F)/Z(\Aa_F)$.}
Apply change of variable from $a$ to $ab^{-1}$ in the left integral.
We get a double integral, with the inner integral only depending on $|b_1 / b_2|$.
This inner integral is written as\footnote{Integration over the elements $b = \smat{b_1}{}{}{b_2} \in [T]$ satisfying $c^{-1} < |b_1 / b_2| < c^2$.}
\begin{align*}
    \int_{c^{-1}}^{c} \eta(\det b) db.
\end{align*}
It is 0 because the restriction of $\eta$ to the group of id\'eles of absolute value 1 is nontrivial.
The integral of $K_1$ is therefore weakly convergent and its value is 0.
The same holds for the integral of $K_4$.

Let's examie the integrals of the other terms, $K_2$ for example.
We have
\begin{align}
    K_2(x, y) = \sum_{\alpha, \beta \in T(F) / Z(F)} f(x^{-1} \alpha n_+ \beta y).
\end{align}
It follows that
\begin{align*}
    \int_{c^{-1}}^{c} \int_{c^{-1}}^{c} K_2(a, b) \eta(a, b) da db = \int_{c^{-1}}^{c} \int_{c^{-1}}^{c} \sum_{\beta \in T(F)/Z(F)} f(a n_+ \beta b) \eta (\det b) da db;
\end{align*}
in the right integral $b$ still varies in the compact subset of $[T]$ defined by $c^{-1} < |b_1 / b_2| <c$, but $a$ varies the subset of $T(\Aa_F) / Z(\Aa_F)$ defined by $c^{-1} < |a_1 / a_2| < c$.
Let's introduce the function $\phi$ on $\Aa_F^\times \times \Aa_F$ defined by
\begin{align}
    \phi(x, y) = f\left(\bmat{x}{0}{0}{1} \bmat{1}{y}{0}{1}\right).
\end{align}
It has a compact support. 
{\color{red}
Write our integral as
\begin{align*}
    \int_{\Aa_F^\times / F^\times} \sum_{\zeta \in F^\times} \int_{\Aa_F^\times} \phi(ab^{-1}, b\zeta) \eta(b) da db, \quad c^{-1} < |a| < c, \,\, c^{-1} < |b| < c.
\end{align*}
}
Using the Poisson summation formula with respect to the second variable and taking the Fourier transform with respect to the second variable we obtain for this integral the expression
\begin{align*}
    \int_{\Aa_F^\times / F^\times} \sum_{\zeta \in F^\times} \int_{\Aa_F^\times} \phi(ab^{-1}, b\zeta) \eta(b)dadb + \int_{\Aa_F^\times /F^\times} \sum_{\zeta \in F^\times} \int_{\Aa_F^\times} \hat{\phi} (ab, b\zeta) |b| \eta(b) dadb
\end{align*}
with $c^{-1} < |a| < c$ and $1 < |b| < c$.
It is obvious that the integrals extend to the domain
\begin{align*}
    a\in\Aa_F^\times, \quad b\in \Aa_F^\times / F^\times, \quad 1 < |b|,
\end{align*}
that converges absolutely.
Moreover in the integrals with extended domains we can change a variable $a$ into $ab^{\pm 1}$.
We conclude that the integral of $K_2$ is weakly convergent and that its value is the following sum:
\begin{align*}
    \int_{\Aa_F^\times / F^\times}\int_{\Aa_F^\times} \sum_{\zeta \in F^\times} \phi(a, b\zeta) \eta(b) da db + \int_{\Aa_F^\times /F^\times} \int_{\Aa_F^\times} \sum_{\zeta \in F^\times} \hat{\phi}(a, b\zeta) |b| \eta(b) da db,
\end{align*}
with $1 < |b|$.
The integral is nothing but the value of the analytical continuation of the following integral at $s=0$:
\begin{align}
    \int_{\Aa_F^\times} \int_{\Aa_F^\times} \phi(a, b) |b|^s \eta(b) dadb.
\end{align}
The value will be denoted as an integral
\begin{align}
    \int_{\Aa_F^\times} \int_{\Aa_F^\times} \phi(a, b) \eta(b) dadb.
\end{align}
With this convention we can write that the weak integral of $K_2$ as
\begin{align}
    \label{eqn:7.3.6}
    H(n_+: f:\eta) = \int_{\Aa_F^\times}\int_{\Aa_F^\times} f\left(\bmat{a}{0}{0}{1} \bmat{1}{b}{0}{1} \right) \eta(b) dadb.
\end{align}
An analogous result is valid for the integrals of the other $K_i$.
Finally we see that the weak integral of $K_s$ exists and is equal to the sum
\begin{align}
    H(n_+:f:\eta) + H(n_-:f:\eta) + H(n\varepsilon_+ :f:\eta) + H(\varepsilon n_-:f:\eta),
\end{align}
where the first term is defined by~\eqref{eqn:7.3.6} and the others are defined similarly:
\begin{align}
    H(n_-:f:\eta) &= \iint f\left(\bmat{a}{0}{0}{1} \bmat{1}{0}{b}{1} \right) \eta(b) da db, \\
    H(\varepsilon n_+:f:\eta) &= \iint f\left(\bmat{a}{0}{0}{1} \varepsilon \bmat{1}{b}{0}{1} \right) \eta(b) da db, \\
    H(\varepsilon n_-:f:\eta) &= \iint f\left(\bmat{a}{0}{0}{1} \varepsilon \bmat{1}{0}{b}{1} \right) \eta(b) da db.
\end{align}

\subsection{}
Let's move on to the integral of $K_{\mathrm{sp}}$.
Recall the definition of $K_{\mathrm{sp}}$:
\begin{align*}
    K_{\mathrm{sp}}(x, y) = \mathrm{vol([G])}^{-1} \sum_{\chi} \int f(g) \chi(\det g) \dd g \cdot \chi(\det x)\chi(\det y^{-1})
\end{align*}
where the sum over all the quadratic characters $\chi$ of the group of id\'ele classes of $F$ and $\mathrm{vol}([G])$ is the volume of the quotient $[G]$.
If $\chi$ is such a character then either $\chi$ or $\chi\eta$ has a non-trival restriction to the groups of id\'ele classes of norm 1.
Reasoning as for $K_1$ we immediately see that $K_\mathrm{sp}$ is weakly integrable and the integral vanishes.



\section{Eisenstein kernel}

\subsection{}
We continue with the same notations in section 7.
We will see that the integral
\begin{align}
\label{eqn:8.1.1}
    \int_{[T]}\int_{[T]} K_{\mathrm{ei}}(a, b) \eta(\det b) dadb
\end{align}
weakly converges.
For the value of the integral, that is a classic application of the trace formula, we will only need a fairly small result.
Choose a place $u$ not in $S$ that splits in $E$.
Fix the components of $f$ at the other places and view the integral~\eqref{eqn:8.1.1} as a function of $f_u$.
Denote $\hat{f_u}$ for the Satake transform of $f_u$.
We prove the following result:
\begin{proposition}\label{prop:8.1}
There exists an integrable function $\phi$ on $\mathbb{R}$ and a constant $c$ such that
\begin{align}
    \int_{[T]} \int_{[T]} K_{\mathrm{ei}}(a, b) \eta(\det b) dadb = \int_{-\infty}^{\infty} \phi(t) \hat{f_u}(q_{u}^{-it}) dt + c \hat{f_u}(q^{-1}).
\end{align}
\end{proposition}


\subsection{}
We need standard results on the Mellin transform of an Eisenstein series.
We fix a subgroup $C$ of $\Aa_F^\times$ isomorphic to the group of real numbers $>0$ such that $\Aa_F^\times$ is the product of $C$ and $\Aa_F^{1}$,
the group of id\'eles of norm 1.
The group $C$ is equipped with the pullback measure of the measure $t^{-1} dt$ by the map $c \to |c|$ and $\Aa_F^1$ admits the quotient measure.
We assume that all characters on the id\'ele class group are trivial on $C$.
Let $\chi$ be such a character and $V(\chi)$ the space of functions $\phi$ on $K$ (the product of $K_v$) such that
\begin{align}
    \phi\left(\bmat{a}{x}{0}{b}k\right) = \chi(ab^{-1})\phi(k)
\end{align}
if 
\begin{align*}
    \bmat{a}{x}{0}{b} \in K.
\end{align*}
Now consider a function $\phi$ on $K\times \mathbb{C}$ such that for each $u \in\mathbb{C}$ the function $\phi(\cdot, u)$ is in $V(\chi)$.
The function is assumed to be holomorphic, or at least meromorphic with respect to $u$; for example it can be independent of $u$.
We will extend $\phi$ into a function $\phi(g, u, \chi)$ on $G(\Aa_F)$ such that
\begin{align}
    \phi\left(\bmat{a}{x}{0}{b}g, u, \chi\right) = \chi(ab^{-1}) |ab^{-1}|^{u+ 1/2}\phi(g, u, \chi).
\end{align}
Then the Eisenstein series is the analytic continuation of the series
\begin{align}
    E(g, \phi, u, \chi) = \sum_{\gamma \in G(F) / T(F) N_+(F)} \phi(\gamma g, u, \chi).
\end{align}
The series converges absolutely if $\Re u > 1/2$.
The constant term of $E$ along $N_{+}$, the group of strictly upper triangular matrices, is by definition the integral
\begin{align}
    E_{N_{+}}(g, \phi, u, \chi) = \int_{N_+(\Aa_F) / N_+(F)} E(ng, \phi, u, \chi) dn.
\end{align}
It has a form of 
\begin{align}
    E_{N_{+}}(g, \phi, u, \chi) = \phi(g, u, \chi) + M(u, \chi)\phi(g, -u, \chi^{-1})
\end{align}
where $M(u, \chi)$ is the intertwining operator from $V(\chi)$ to $V(\chi^{-1})$.
We also need another Fourier coefficient of $E$, namely
\begin{align}
    W(g, \phi, u, \chi) = \int_{\Aa_F /F} E\left(\bmat{1}{x}{0}{1}g, \phi, u, \chi\right)\psi(-x)\dd x,
\end{align}
where $\psi$ is a fixed character on the group $\Aa_F/F$.
Then the Fourier series of $E$ is written as
\begin{align}
    E(g, \phi, u, \chi) = \phi(g, u, \chi) + M(u, \chi) \phi(g, -u, \chi^{-1}) + \sum_{\alpha\in T(F)/Z(F)} W(\alpha g, \phi, u, \chi).
\end{align}
We can also consider a Fourier series for the group $N_{-}$ of strictly lower triangular matrices.
Since
\begin{align}
    N_- = wN_{+}w^{-1}, \quad w = \bmat{0}{-1}{1}{0},
\end{align}
the Fourier series is written as
\begin{align}
    E(g, \phi, u, \chi) = \phi(wg, u, \chi) + M(u, \chi) \phi(wg, -u, \chi^{-1}) + \sum_{\alpha\in A(F)/Z(F)} W(\alpha w g, \phi, u, \chi).
\end{align}
The Mellin transform $L(s, \lambda:\phi:u, \chi)$ of $E$ is defined by the following integral (or its analytic continuation)
\begin{align}
    L(s, \lambda:\phi:u, \chi) = \int_{\Aa_F^\times /F^\times} \left( E\left(\bmat{a}{0}{0}{1}\right) - E_{N_{+}}\left(\bmat{a}{0}{0}{1}\right)\right) |a|^{s-1/2} \lambda(a) da.
\end{align}
Let's ignore the variables of $E$ for simplification.
By replacing $E$ with its Fourier series, we immediately obtain the following in the Mellin transform
\begin{align}
    \int_{\Aa_F^\times /F^\times} W\left(\bmat{a}{0}{0}{1}\right) |a|^{s-1/2} \lambda(a) da.
\end{align}
We can also write the Mellin transform of $E$ as follows:
\begin{align}
\label{eqn:8.2.12}
    L(s, \dots) =\int_{1}^{\infty} (E-E_{N_{+}}) + \int_{0}^{1} (E-E_{N_{-}}) + \int_{1}^{\infty} E_{N_+} + \int_{0}^{1} E_{N_{-}}
\end{align}
In each of these integrals, the function is evaluated at the point $\mathrm{diag}(a, 1)$ and integrated against $|a|^{s-1/2}\lambda(a)$ on a subset of the id\'ele class group.
For the first integral, for example, we integrate over the subset of $a$ with $1<|a|$.
Using the Fourier series of $E$ we easily obtain another expression for the Mellin transform:
\begin{equation}
\begin{aligned}
    &\int_{1}^{\alpha} W\left(\bmat{a}{0}{0}{1}\right)|a|^{s-1/2} \lambda(a) da \\
    &+ \int_{1}^{a} W\left(\bmat{a}{0}{0}{1}w\right)|a|^{s-1/2} \lambda(a) da  \\
    &+ \int_{1}^{\alpha} (|a|^{s+u}(\lambda \chi)(a)\phi(e) + |a|^{s-u}(\lambda\chi^{-1})(a)M(u, \chi)\phi(e)) da  \\
    &+ \int_{0}^{1} (|a|^{s-u-1} (\lambda \chi^{-1})(a) \phi(w) + |a|^{s+u-1}(\lambda\chi)(a) M(u, \chi)\phi(w)) da. 
\end{aligned}
\end{equation}
The first two integrals converge for all $s$ and the last two for $\Re s > 1/2$.
The last two integrals can be easily computed.
In particular, for $s = 1/2$ and $u \in i\mathbb{R}$, 
we obtain the following expression for the Melline transform of $E$ at the point $s = 1/2$:
\begin{equation}
\begin{aligned}
\label{eqn:8.2.14}
    L(1/2, \lambda: \phi:u, \chi) &= \int_{1}^{\infty} W\left(\bmat{a}{0}{0}{1}\right) \lambda(a) da \\
    &+ \int_{1}^{\infty} W\left(\bmat{a}{0}{0}{1}w\right) \lambda(a) da \\
    &- \frac{1}{u + 1/2} (\phi(w)\delta(\lambda\chi^{-1}) + \phi(e) \delta(\lambda \chi)) \\
    &+ \frac{1}{u - 1/2} (M(u, \chi)\phi(w) \delta(\lambda\chi) + M(u, \chi)\phi(e) \delta(\lambda\chi^{-1})),
\end{aligned}
\end{equation}
where 
\begin{align*}
    \delta(\chi) = \int_{\Aa_F^{1} / F^\times} \chi(a) da
\end{align*}
for a character $\chi$ on the id\'ele class group.
We have to compute the difference between the Mellin transform and the following integral
\begin{align}
\label{eqn:8.2.15}
    \int_{c^{-1}}^{c} E\left(\bmat{a}{0}{0}{1}\right) \lambda(a) da.
\end{align}
Recall that this notation means that the integral is taken over the compact subset of id\'ele classes a such that $c^{-1} < |a|< c$.
Instead of~\eqref{eqn:8.2.12} we have for the integral~\eqref{eqn:8.2.15} the expression:
\begin{align}
    \int_{1}^{c} (E - E_{N_+}) + \int_{c^{-1}}^{1} (E - E_{N_-}) + \int_{1}^{c} E_{N_+} + \int_{c^{-1}}^{1} E_{N_-}.
\end{align}
Replacing $E$ again by its Fourier series we obtain for~\eqref{eqn:8.2.15} the expression:
\begin{equation}
\begin{aligned}
    &\int_{1}^{c} W\left(\bmat{a}{0}{0}{1}\right)\lambda(a) da \\
    &+ \int_{1}^{c} W\left(\bmat{a}{0}{0}{1}w\right)\lambda(a)da \\
    &+ \int_{1}^{c} (|a|^{1/2 + u}(\lambda\chi)(a)\phi(e) + |a|^{1/2 - u}(\lambda \chi^{-1})(a) M(u, \chi)\phi(e)) da \\
    &+ \int_{c^{-1}}^{1} (|a|^{-u-1/2}(\lambda\chi^{-1})(a) \phi(w) + |a|^{u-1/2} (\lambda\chi)(a) M(u, \chi) \phi(w)) da
\end{aligned}
\end{equation}
Calculating the last two integrals and comparing to~\eqref{eqn:8.2.14} we finally get the expression we had in mind:
\begin{equation}
    \label{eqn:8.2.18}
    \begin{aligned}
        &\int_{c^{-1}}^{c} E\left(\bmat{a}{0}{0}{1}\right) \lambda(a)da \\
        &=L(1/2, \lambda :\phi: u, \chi) \\
        &+ \frac{c^{u+1/2}}{u + 1/2} \delta(\chi\lambda) \phi(e) + \frac{c^{-u+1/2}}{-u+1/2} \delta(\chi^{-1}\lambda) M(u, \chi)\phi(e) \\
        &+ \frac{c^{u+1/2}}{u  +1/2} \delta(\chi^{-1}\lambda) \phi(w) +\frac{c^{-u+1/2}}{-u+1/2} \delta(\chi\lambda) M(u, \chi)\phi(w) + R(c)
    \end{aligned}
\end{equation}
where $R(c)$ is defined as
\begin{align}
    \label{eqn:8.2.19}
    -R(c) = \int_{c}^{\infty} W\left(\bmat{a}{0}{0}{1}\right) \lambda(a) da + \int_{c}^{\infty} W\left(\bmat{a}{0}{0}{1}w\right) \lambda(a) da.
\end{align}
It is clear that $R(c)$ tends to zero as $c$ goes to infinity.

\subsection{}
We need precise estimates for $R(c)$.
Recall that $R$ depends not only on $c$, but also on $u$, $\lambda$ and $\phi$.
Our estimates will be a consequence of the following lemma:

\begin{lemma}
Assume that $\phi$ is independent of $u$.
There exists a Schwartz-Bruhat function $\Phi$ such that, for $u \in i\mathbb{R}$, we have
\begin{align*}
    \Bigg|W\left(\bmat{a}{0}{0}{1}, \phi, u, \chi\right)\Bigg| \leq \Phi(a) |a|^{-1/2} |L(2u+1, \chi^{2S})|^{-1}
\end{align*}
\end{lemma}
Here $L(s, \chi^{S})$ is defined as a product of local factors $L(s, \chi_v)$ for $v \not\in S$
We also assume that $\phi$ is invariant under $K_v$ for $v\not\in S$.

\begin{proof}
There is a two-variable Schwartz-Bruhat function $\Phi$ such that
\begin{align*}
    \phi(g, u, \chi) = \int \Phi((0, t)g) \chi^2(t) |t|^{2u+1} dt \times \chi(\det g)|\det g|^{u + 1/2} \times L(2u+1, \chi^{2S})^{-1}.
\end{align*}
A formal computation (see~\cite{jacquet2006automorphic}, chapter 3 for details) gives
\begin{align*}
    W\left(\bmat{a}{0}{0}{1}, \dots \right) = L(2u+1, \chi^{2S}) \chi(a)|a|^{u+1/2} \int \hat{\Phi}(at, t^{-1}) \chi^{2}(t)|t|^{2u+1} dt,
\end{align*}
where $\hat{\Phi}$ is a Fourier transform with respect to the second variable.
Hence it is suffice to prove the following assertion: given a Schwartz-Bruhat function $\Phi\geq 0$ with two variables,
there exists a Schwartz-Bruhat function with one variable $\phi\geq 0$ such that for id\'ele $a$ we have
\begin{align*}
    \int \Phi(at, t^{-1}) dt \leq \phi(a) |a|^{-1}.
\end{align*}
Consider the analogous local problem.
To be precise, let us first consider the case where the local field $F$ is non-archimedean and the function $\Phi$ is the characteristic function of the integers.
{\color{red}
Then the integral is 0 except for the set defined by the inequalities $|a|\leq|t|\leq 1$.
The integral is therefore 0 unless $a$ is integral.
In this case the integral is $1+v(a)$.
Since $q\geq a$ this is smaller than $q^{v(a)}$.
}
So our integral is at most $\phi(a)|a|^{-1}$, where $\phi$ is the characteristic function of the integers.
Then the integral, considered as a function of $a$, has the form
\begin{align*}
    \int \Phi(at, t^{-1})dt = \phi_1(a) + \phi_2(a) \log |a|
\end{align*}
for some Schwartz-Bruhat functions $\phi_i$ (cf. (4.3)).
It is clear that the right hand side is bounded by $\phi(a)|a|^{-1}$, where $\phi$ is a suitable Schwartz-Bruhat function.
By multiplying these local inequalites we easily obtain the required global inequality.
\end{proof}

It is well known that the function $L(2u+1, \chi^{2S})^{-1}$ has polynomial growth on the line $\Re(u)=0$.
On the other hand, if $\phi$ is a Schwartz-Bruhat function, there exists for all $N>0$ a constant $C(N)$ such that
\begin{align*}
    \int_{c}^{\infty} \phi(a)|a|^{-1}da \leq C(N) c^{-N}.
\end{align*}
By comparing with definition~\eqref{eqn:8.2.19} of $R$ we immediately obtain the following:

\begin{lemma}
For all $N$ there exist constants $C(N)$ and $M$ such that for all imaginary $u$ we have
\begin{align*}
    |R(c, u)| \leq C(N)|c|^{-N} |u|^{M}.
\end{align*}
\end{lemma}
In the same way using the expression for the Mellin transform and the fact that the operator $M(u, \chi)$ is unitary on the imaginary axis we obtain the following estimate:

\begin{lemma}\label{lem:8.3}
On the imaginary axis $M(u, \chi)\phi(k)$ and $L(1/2, \lambda:\phi:u, \chi)$ have polynomial growth.
\end{lemma}


\subsection{}
Let's study the integral of the kernel $K_{\mathrm{et}}$.
Let's recall its definition.
For any character $\chi$ choose an orthonormal basis $\phi_i$ of the Hilbert space $V(\chi)$;
denote by $\rho(u, \chi)$ the representation of $G(\Aa_F)$ by right translations in the space of functions $\phi$ such that
\begin{align}
    \phi\left(\bmat{a}{x}{0}{b} g\right) = \chi(ab^{-1}) |ab^{-1}|^{u + 1/2} \phi(g).
\end{align}
We can identify the space of $\rho(u, \chi)$ with $V(\chi)$ and set:
\begin{align}
    F(u, \chi:i, j) = (\rho(u, \chi)\phi_i, \phi_j).
\end{align}
We will write $E_{\mathrm{ei}}(x, i, \dots)$ for $E_{\mathrm{ei}}(x, \phi_i, \dots)$.
With these notations
\begin{align}
    \label{eqn:8.4.3}
    K_{\mathrm{ei}}(x, y) = \sum_{\chi} K_{\chi}(x, y)
\end{align}
where, for each character of the id\'ele class group, 
\begin{align}
    \label{eqn:8.4.4}
    K_{\chi}(x, y) = \frac{1}{2i\pi} \sum_{i, j} \int_{-i\infty}^{i\infty} F(u, \chi: i, j) E(x, j, u, \chi) \overline{E(y, i, u, \chi)} du.
\end{align}
For a given $f$ the sum~\eqref{eqn:8.4.3} and~\eqref{eqn:8.4.4} are finite.
Define
\begin{align}
    I(c, \chi) = \int_{c^{-1}}^{c} \int_{c^{-1}}^{c} K_\chi(a, b) \eta(\det b) dadb.
\end{align}
We can obviously change the order of integrations for $u$ and the tuple $(a, b)$.
Using~\eqref{eqn:8.2.12} we obtain the following expression on $I(c, \chi)$:
\begin{equation}
\label{eqn:8.4.6}
\begin{aligned}
    &\frac{1}{i\pi} \sum_{i, j} \int_{-i\infty}^{i\infty} F(u, \chi: i, j) \\
    &\times \bigg( L(1/2, 1:j, u, \chi) + R(c, u) 
    + \frac{c^{u+ 1/2}}{u + 1/2}\delta(\chi)\phi_j(e)
    + \frac{c^{-u+1/2}}{-u+1/2}\delta(\chi^{-1})M(u, \chi)\phi_j(c) \\
    &+ \frac{c^{u+1/2}}{u + 1/2} \delta(\chi^{-1}) \phi_j(w) + \frac{c^{-u+1/2}}{-u+1/2} \delta(\chi) M(u, \chi) \phi_j(w) \bigg) \\
    &\times \bigg( \overline{L(1/2, \eta, i, u, \chi)} + R'(c, u) 
    + \frac{c^{-u+1/2}}{-u+1/2} \delta(\eta \chi)\overline{\phi_i}(e) 
    + \frac{c^{u+1/2}}{u + 1/2} \delta(\chi^{-1}\eta) M(u, \chi) \overline{\phi_i}(e) \\
    &+ \frac{c^{-u+1/2}}{-u+1/2}\delta(\chi^{-1}\eta)\overline{\phi_i}(w) + \frac{c^{u+1/2}}{u+1/2} \delta(\chi\eta) M(u, \chi)\overline{\phi_i}(w)\bigg) du
\end{aligned}
\end{equation}

For each $(i, j)$, the terms $R(c, u)$ and $R'(c, u)$ satisfy the conclutions of the Lemma \ref{lem:8.3}.
For each $f$,  $F(u, \chi: i, j)$ is zero for all but finitely many $(i, j)$. 
In particular, $F(u, \chi: i, j)$ is zero unless $\phi_i$ and $\phi_j$ are both invariant under all $K_v$ with $v$ not in $S $.
Moreover, on the imaginary axis, $F(u, \chi: i, j)$ rapidly decreases (faster than the inverse of a polynomial in $u$).
On the other hand, according to (8.3), the terms $L(\dots)$ and the terms containing the powers of $c$ moderately grows (at most polynomial in $u$).
It follows that when we expand expression~\eqref{eqn:8.4.6} we find a number of terms which tend to zero as $c$ tends to infinity, and we can ignore these terms.
The remaining terms become an  integral independent of $c$:
\begin{align}
    \label{eqn:8.4.7}
    \sum_{i, j}\int_{-i\infty}^{i\infty} F(u, \chi: i, j) L(1/2, 1:j:u, \chi) \overline{L(1/2, \eta, i, u, \chi)} du.
\end{align}
The other terms does not vanish only if $\chi=1$ or $\chi=\eta$.
Each of these terms is of one of the following types:
\begin{align}
    &\int F(u,1:i, j)\overline{L(1/2, \eta:i:u, 1)} \frac{c^{1/2+u}}{1/2+u}(\phi_j(e) + \phi_j(w))du, \label{eqn:8.4.8} \\
    &\int F(u, \eta:i, j) L(1/2, 1: j: u, \eta) \frac{c^{1/2-u}}{1/2-u} (\overline{\phi_i}(e) + \overline{\phi_i}(w)) du, \label{eqn:8.4.9} \\
    &\int F(u, 1:i, j) \overline{L(1/2, \eta:i:u, 1)} \frac{c^{1/2-u}}{1/2-u} M(u, 1)(\phi_j(e) +\phi_j(w)) du, \label{eqn:8.4.10}\\
    &\int F(u, \eta:i, j)L(1/2, 1:j:u, \eta) \frac{c^{1/2+u}}{1/2+u} M(u, \eta)(\overline{\phi_i}(e) +\overline{\phi_i}(w)) du. \label{eqn:8.4.11}
\end{align}

Integral~\eqref{eqn:8.4.7} obviously has the properties required by Proposition \ref{prop:8.1}.
To prove the proposition, it suffices to show that each of the expressions~\eqref{eqn:8.4.8} to~\eqref{eqn:8.4.11} converges when $c$ tends to infinity
and that, moreover, the limit is zero if the Satake transform of the function $f_u$ is zero at $q^{-1}$.
This last condition implies that the integral of $f_u$ over $G_u/Z_u$ is zero and  $F(u, 1:u, j)$ and $F(u, \eta:i, j)$ cancel out at points $u =1/2$ and $u=-1/2$.


\subsection{}
Let's study~\eqref{eqn:8.4.8}.
We will move the contour of integration from line $\Re u = 0$ to line $\Re u = -1/2$; 
but for the latter one, we will replace the segment joining the point $-1/2-i\varepsilon$ and the point $-1/2+i\varepsilon$ by the semi-circle centered at $-1/2$ and of radius $\varepsilon$
which passes through the points $-1/2-\varepsilon i$, $\varepsilon-1/2$ and $-1/2+i\varepsilon$.
Let's prove that such a transformation of cantour is valid.
The factor
\begin{align*}
    F(u) = F(u, 1:i, j)(\phi_j(e) + \phi_j(w))
\end{align*}
and its derivatives, are holomorphic and rapidly decreasing on the vertical strip $-1/2\leq \Re u \leq 0$.
The exponential function remains bounded.
The factor $(1/2+u)^{-1}$ also remains bounded at infinity on this vertical strip.
Now we study the Mellin transform.
Recall that we have an integral representation of $\phi_i(g, u, 1)$:
\begin{align*}
    \phi_i(g, u, 1) = \int \Phi((0, t) g) |t|^{2u+1} dt \times |\det g|^{u+1/2} L(2u+1, 1^{S})^{-1}.
\end{align*}
A formal computation gives the following expression of the Mellin transform (denoted as $L(u)$ in short):
\begin{align}
    L(u) = L(2u+1, 1^{S})^{-1} \iint \hat{\Phi}(a, b) |a|^{1/2+u} \eta(a) |b|^{1/2-u} \eta(b) dadb,
\end{align}
where $\hat{\Phi}$ is the Fourier transform of $\Phi$ with respec to the second variable.
By taking the complex conjugation of the second variable we obtain
\begin{align}
    \overline{L(-\bar{u})} = L(-2u+1, 1^{S})^{-1}T(u),
\end{align}
where
\begin{align}
    T(u) = \iint \Phi_1(a, b)|a|^{1/2-u} \eta(a) |b|^{1/2+u} \eta(b) dadb.
\end{align}
Here $\Phi_1$ is a Schwartz-Bruhat function; the \emph{Tate's double integral} $T(u)$, as well as all its derivatives, is bounded on the vertical strip $-1/2 \le \Re u \le 0$.
At last, on the vertical strip we have $1\leq \Re(-2u+1) \leq 2$ and the function $L(-2u+1, 1^S)^{-1}$ is holomorphic and bounded by a polynomial in $\Im u$.
We can write the integral as
\begin{align*}
    \int F(u)L(1-2u, 1^{S})^{-1} T(u) c^{1/2+u} (1/2+u)^{-1} du.
\end{align*}
Hence our transformation of the contour of the integration is valid.
By replacing $u$ with $u-1/2$, we obtain the following expression for~\eqref{eqn:8.4.8}:

\begin{align}
    \label{eqn:8.5.4}
    \int F(u-1/2) L(2-2u, 1^{S})^{-1} T(u-1/2) c^{u} u^{-1}du
\end{align}
In~\eqref{eqn:8.5.4} the contour of the integration is the line $\Re u = 0$,
except that the segment joining the point $-i\varepsilon$ to the point $i\varepsilon$ is replaced by the semicircle centered at 0 which goes through the points $-i\varepsilon$, $\varepsilon$, $i\varepsilon$.
Now let $\varepsilon \to 0$.
Then the integral over the semicircle tends to
\begin{align*}
    i\pi F(-1/2) L(2, 1^{S})^{-1} T(-1/2)
\end{align*}
while the integral on the linnear part of the contour tends to the Cauchy's principal value.
In terms of the real variable $t$ the integral~\eqref{eqn:8.5.4} is also equal to
\begin{align}
    \label{eqn:8.5.5}
    \int_{-\infty}^{\infty} F(it-1/2)L(2-2it, 1^{S})^{-1} T(it-1/2)c^{it}t^{-1}dt + i\pi F(-1/2)L(2, 1^{S})^{-1} T(-1/2).
\end{align}
For real $r$ the function $L(-2it+2, 1^S)$ is given by an absolutely and uniformly convergent infinite product (or a Dirichlet series).
Its derivatives are therefore bounded and its inverse is also bounded.
The derivatives of the factor $L(-2it+2, 1^S)^{-1}$ are therefore bounded.
In~\eqref{eqn:8.5.5} the product of the first three terms is therefore a Schwartz function of $t$.
When $c$ tends to infinity the Cauchy integral tends to $i\pi$ times the value of the Schwartz function at point 0.
In total we see that~\eqref{eqn:8.5.5}, i.e. the term~\eqref{eqn:8.4.8}, converges as $c \to \infty$, to
\begin{align*}
    2i \pi F(-1/2)L(2, 1^S)^{-1} T(-1/2),
\end{align*}
where the limit vanishes if and only if $F(-1/2, 1: i, j)$ does.
This is what we had to prove.
Similar argument holds for~\eqref{eqn:8.4.9}.


\subsection{}
Let's move on to~\eqref{eqn:8.4.10}.
We'll simplify the notation as
\begin{align*}
    F(u) = F(u, 1:i, j).
\end{align*}
We are going to use a slightly different expression from the one we have used so far for the Mellin transform.

Write $\phi$ for $\phi_j$ and suppose that $\phi$ is a product of local functions $\phi_v$.
We can also assume that, for each place $v$, $\phi_v$ is either $K_v$ invariant, or has a vanishing integral over $K_v$.
Let $S_0$ denote the set of places where this last condition is satisfied.
Then $S_0$ is finite and contains $S$. 
We can find an integral representation for $\phi(g, u, 1)$ of the form
\begin{align}
    \phi(g, u, 1) = \int \Phi((0, t)g) |t|^{2u+1} dt \times |\det g|^{u+1/2} L(2u+1, 1^{S_0})^{-1}
\end{align}
We can conclude that, as before, the Mellin transform in~\eqref{eqn:8.4.10} can be written as
\begin{align}
    \label{eqn:8.6.2}
    L(-2u+1, 1^{S_0}) T(u)
\end{align}
where $T(u)$ is defined as a Tate's double integral, holomorphic in $u$.
On the other hand we can write the intertwining operator $M(u, 1)$ as a product
\begin{align}
    \label{eqn:8.6.3}
    M(u, 1) = L(2u, 1) L(2u+1, 1)^{-1}N(u, 1)
\end{align}
where $N$ is the normalized intertwining operator.
Now the quotient of $L(2u, 1)$ by $L(-2u+1, 1)$ is an exponential function $ab^u$.
It follows that the product of factors~\eqref{eqn:8.6.2} and~\eqref{eqn:8.6.3} reduces to
\begin{align}
    ab^u L(-2u + 1, 1_{S_0}) L(2u+1, 1_{S_0})^{-1} L(2u+1, 1^{S_0})^{-1} T(u)N(u, 1).
\end{align}
Then~\eqref{eqn:8.4.10} is given by the following integral
\begin{equation}
    \int F(u) L(2u+1, 1^{S_0})^{-1} T(u) c^{1/2-u} (1/2-u)^{-1} A(u) du,
\end{equation}
with
\begin{align*}
    A(u) = ab^{u}L(-2u+1, 1_{S_0}) L(2u+1, 1_{S_0})^{-1} N(u, 1)(\phi(e) + \phi(w)).
\end{align*}
We are going to move the contour of integration.
The present contour is the line $\Re u = 0$.
The new contour will be the line $\Re u = 1/2$,
except that the segment joining the points $1/2-i\varepsilon$ and $1/2+i\varepsilon$ will be replaced by the semicircle passing through the points $1 /2-i\varepsilon$, $1/2-\varepsilon$, $1/2+i\varepsilon$. 
Remaining part of the proof will then be the same as in the previous case, except that we have to show that the factor $A(u)$ is holomorphic and has a moderate growth on the strip $0\leq \Re u \leq 1/2$.
The ratio of the factors $L$ which appears in $A$ is the product of the ratios
\begin{equation*}
    L(-2u+1, 1_v) L(2u+1, 1_v)^{-1}
\end{equation*}
for all $v$ in $S_0$.
If $v$ is finite, then the ratio is a rational function in $q_v^{-u}$ and has a moderate growth.
If $v$ is infinite, then the Stirling's formula implies that the ratio has a moderate growth.
Recall that $\phi_v$ equals to 1 on $K_v$ for all $v$ not in $S_0$.
For such $v$, we have $N(u, 1_v)\phi_v(k_v)=1$ for all $u$. 
So $N(u, 1)\phi(e)$  is in fact the product over all $v$ in $S_0$ of
\begin{equation*}
    N(u, 1_v) \phi_v(e).
\end{equation*}
If $v$ is finite this is still has a moderate growth.
If $v$ is infinite, this is a polynomial in $u$, and so $A$ has a moderate growth.
At last, let's prove that $A$ is holomorphic at the poles of the factor $L(-2u+1, 1_{S_0})$ on the strip.
Let's prove, for example, holomorphy at $1/2$ of 
\begin{equation*}
    L(-2u+1, 1_{S_0})L(2u+1, 1_{S_0})^{-1} N(u, 1)\phi(e).
\end{equation*}
The previous product can be written as
\begin{equation*}
    \prod_{v\in S_0}L(-2u+1, 1_v)L(2u+1, 1_v)^{-1} N(u, 1_v) \phi_v(e).
\end{equation*}
Take a $v$ in $S_0$.
As the integral of $\varphi_v$ over $K_v$ is zero,
$N(u, 1_v)\phi_v(e)$ vanishes at the point $u=1/2$ and this zero cancel out the pole of the factor $L (-2u+1, 1_v)$ at the same point.
The product is therefore holomorphic at the point $1/2$ and this concludes our proof for the term~\eqref{eqn:8.4.10}.
A similar argument applies to the term~\eqref{eqn:8.4.11}.
Hence we complete the proof of the assertions in (8.1).
\section{Global orbital integrals: compact torus}

\subsection{}

In this section $F$ is again a number field and $E$ a quadratic extension of $F$.
We will fix an element $(G', T')$ of the set $X(E:F)$ and an element $\varepsilon$ of $N(T')-T'$.
Then the square $c$ of $\varepsilon$ is an element of $F^\times$ and the class $cN$ of the group of norms $N$ of $E$ 
determines the isomorphism class of $(G', T')$.
Let $f'$ be a compactly supported smooth function on the group $G'(\Aa_F) /Z'(\Aa_F)$.
We have a cuspidal kernel $K_c'$ attached to $f'$.
Let $\phi_i'$ be an orthonormal basis of the space of automorphic forms which are cuspidal and orthogonal to the functions s$g \mapsto \chi(\det g)$, 
where $\chi$ is a character of id\'ele class group whose square is trivial.
By definition:
\begin{equation}\label{eqn:9.1.1}
    K_c'(x, y) = \sum_j \rho(f') \phi_j'(x) \overline{\phi_j'}(y).
\end{equation}
We will give a useful expression of the integral
\begin{equation}\label{eqn:9.1.2}
    \int_{[T']}\int_{[T']} K_c'(s, t) dsdt.
\end{equation}
Of course $\psi\circ\mathrm{tr}$ is a character of $\Aa_E /E$ and we therefore have for each place $v$ of $E$ the Tamagawa measure on the group $E_v^\times$ a
and an induced measure on $T_v'$.
We also have the product measure on the group $T'(\Aa_E)$ and the quotient measure on $T'(\Aa_F)/Z'(\Aa_F)$. 
We will denote by $S$ a finite set of places of $F$ containing the places at infinity, 
the ramified places in $E$, 
the places where $G'$ does not split, the places where $\psi_v'$  is not of order 0 and places of residual characteristic 2.
We choose for all $v$ a maximal compact subgroup $K_v'$ of $G_v'$ such that $T_v'$ is contained in $K_v' Z_v'$ if $v$ does not split in $E$ and $G'(\Aa_F)$ be 
the restricted product of $G_v'$ with respect to $K_v'$. 
We suppose that $f'$ is the product of compactly supported smooth local functions $f_v'$ on $G_v'/Z_v'$.
We assume $f_v'$ bi-$K_v'$-invariant for each $v$ not in $S$.
For $v$ that does not split in $E$ we replace $f_v$ for~\eqref{eqn:9.1.1} as $f_v'$ defined by
\begin{equation*}
    f_{v0}'(g') = \frac{1}{\mathrm{vol}(T_v)} \int_{T_v'} \int_{T_v'} f_v'(s_v g' t_v) ds_v dt_v.
\end{equation*}
We can therefore assume that each $v$ which does not split in $E$ the function $f_v'$ is bi-$T_v'$-invariant, in particular 
bi-$K_v'$-finite. 
At last we assume that $f_v'$ is bi-$K_v'$-finite at $v$'s split in $E$. 
Then we have:
\begin{equation}
    K_c'(x, y) = \sum_{\gamma \in G'(F)/Z'(F)} f'(x^{-1}\gamma y) - K_{\mathrm{sp}}' (x, y) - K_{\mathrm{ei}}' (x, y).
\end{equation}
where $K_{\mathrm{sp}}'$ denote the special kernel and $K_{\mathrm{ei}}'$ is the Eisenstein kernel.
The Eisenstein kernel is zero if $G'$ does not split.
The kernel $K_{\mathrm{sp}}'$ is defined by the following sum
\begin{equation}
    \label{eqn:9.1.4}
    K_{\mathrm{sp}}'(x, y) = \sum_{\chi} \frac{1}{\mathrm{vol}([G'])} \int f'(\det g') dg' \cdot \chi(\det x)\chi^{-1}(\det y)
\end{equation}
where the sum is over all the quadratic characters $\chi$ of the id\'ele class group of $F$.
We define two other kernels
\begin{align}
    K_{r}'(x, y) &= \sum_{\gamma\text{ is }T'\text{-regular}} f(x^{-1}\gamma y) \\
    K_{s}'(x, y) &= \sum_{\gamma\text{ is }T'\text{-singular}} f(x^{-1}\gamma y),
\end{align}
then $K_c'$ can be written as a sum
\begin{equation}
    \label{eqn:9.1.7}
    K_c' = K_r' + K_s' - K_{\mathrm{sp}}' - K_{\mathrm{ei}}'.
\end{equation}
Since $[T']$ is compact,~\eqref{eqn:9.1.2} is simply an integral of each term in~\eqref{eqn:9.1.7}.


\subsection{}
Let's consider $K_r'$ first. 
Each $T'$-regular element $\gamma \in G'(F)/Z'(F)$ can be uniquely written as
a form 
\begin{equation}
    \gamma = \sigma^{-1} \mu \tau,
\end{equation} 
where $\sigma$ and $\tau$ are in $T'(F)/Z'(F)$ and $\mu$ is a representative of a $T'$-regular double coset of $T'(F)$ in $G'(F)$ (Proposition (1.2)).
Then we get\footnote{Of course, the summation on the RHS is over $T'(F)$-regular double coset representatives of $T'(F)$ in $G'(F)$.}
\begin{equation}
    \iint K_r'(s, t)dsdt = \sum_{\mu} \iint f'(s^{-1} \mu t) ds dt,
\end{equation}
where the integral on the RHS is over $T'(\Aa_F) / Z'(\Aa_F)$.
The double integral of the right-hand side only depends on $\zeta=P'(\mu:T')$ and we will note $H'(\zeta:f':T')$ its value.
Then we can write as
\begin{equation}
    \iint K_r'(s, t)dsdt = \sum_{\zeta \in cN - \{1\}} H(\zeta: f': T'),
\end{equation}
since the function $P'$ parametrizes the regular double cosets and its values, on the regular elements, 
are all the points of the class $cN$ associated with the pair $(G', T')$ minus identity (Proposition (1.1)). 
Of course the orbital integral $H'(\zeta:f':T')$ is the product of the local orbital integrals:
\begin{equation}
    H'(\zeta:f':T') = \prod_v H'(\zeta:f_v':T_v').
\end{equation}
Almost all the factors are equal to 1.
Indeed when $v$ be a place of $F$ which is not in $S$; suppose that $f_v'$ is the characteristic function of $Z_v' K_v'$.
If $v$ does not split in $E$, then $T_v'$ is contained in $Z_v' K_v'$ and the integral is 1.
If $v$ splits in $E$ then the local integral is still 1 by Proposition (5.7).


\subsection{}
Let's consider the integral of the term $K_s'$.
There are only two singular double cosets, $T'(F)$ and $\varepsilon T'(F)$.
Then we get
\begin{equation}
    \iint K_s'(s, t) = \mathrm{vol}([T']) \int_{T'(\Aa_F)/Z'(\Aa_F)} f'(t)dt + \mathrm{vol}([T']) \int_{T'(\Aa_F)/Z'(\Aa_F)} f'(\varepsilon t) dt.
\end{equation}


\subsection{}
Consider $K_{\mathrm{sp}}'$.
By~\eqref{eqn:9.1.4}, we have
\begin{equation}
    \iint K_{\mathrm{sp}}'(s, t)dsdt = \sum_{\chi} \frac{1}{\mathrm{vol}([G'])} \int f'(g) \chi(\det g) dg \int \chi(\det s) ds \int \chi^{-1}(\det t) dt,
\end{equation}
each of the integral of characters over $[T']$ is 0 unless $s\mapsto \chi(\det s)$ is trivial over $T'(\Aa_F) $; 
this is the case if and only if $\chi=1$ or $\chi=\eta$.
The integral of $K_{\mathrm{sp}}$ therefore reduces to two terms:
\begin{equation}
    \iint K_{\mathrm{sp}}'(s, t) dsdt = \frac{\mathrm{vol}([T'])^2}{\mathrm{vol}([G'])} \left(\int f'(g)\chi(\det g)dg + \int f'(g) dg\right)
\end{equation}
In particular, let's choose as in (8.1) a place $z$ of $F$ not in $S$, fix the components of $f$ at places other than $z$ and look at the integral as a function of $\hat{f_z'}$.
Then \textcolor{red}{the integral (5)} is of the form $c \hat{f_z'}(q_{z}^{-1})$ for a constant $c$.



\subsection{}
Consider $K_{\mathrm{ei}}$.
It vanishes if $G'$ does not split over $F$.
Suppose $G'$ splits and recall the notations in (8.4).
we have:
\begin{equation}
    K_{\mathrm{ei}}'(x, y) = \frac{1}{i\pi} \int_{-i\infty}^{i\infty} A(x, y, u)
\end{equation}
where
\begin{equation}\label{eqn:9.5.2}
    A(x, y, u) = \sum_{\chi, j} (\rho(f')E)(x, j, u, \chi) \overline{E(y, j, u, \chi)}.
\end{equation}
Note that the maximal compact subgroup implicit in the definition of Eisenstein series is now the product of the groups $K_v'$, with $K_v' Z_v' =T_v'$ if $v$ is infinite.
In particular the series~\eqref{eqn:9.5.2} is finite.
As we integrate over a compact set we get:
\begin{equation}
    \int_{[T']}\int_{[T']} K_{\mathrm{ei}}'(s, t)dsdt = \frac{1}{i\pi} \int_{-i\infty}^{i\infty} A(u:f') du,
\end{equation}
where
\begin{equation}
    A(u:f') = \sum_{\chi, j} \int_{[T']} (\rho(f')E)(s, j, u,\chi)ds \int_{[T']} \overline{E(t, j, u, \chi)} dt.
\end{equation}
Now $(\rho(f')E)(x, j, u, \chi)$ is zero unless $\phi_j'$ is $K_v'$-invariant for all places $v$ not in $S$.
In particular, let's choose as above a place $z$ of $F$ which is not in $S$ and splits in $E$.
Then $f' =f^{z}{'} f_{z}'$ where $f^{z}{'}$ is the product of $f_v'$ for $v\neq z$ and
\begin{equation}
    (\rho(f')E)(x, j, u, \chi) = \hat{f_z'} (q_z^{-2iu}) (\rho(f^z{'})E)(x, j, u, \chi).
\end{equation}
Hence we get
\begin{equation}
    \int_{[T']}\int_{[T']} K_{\mathrm{ei}}'(s, t)dsdt = \frac{1}{2i\pi} \int_{-i\infty}^{i\infty} \hat{f_z'}(q_{z}^{-2iu}) A(u:f^z{'}) du,
\end{equation}
where $A(u:f^z{'})$ is integrable, a result which will be sufficient for our purpose.
\section{Fundamental lemma}


\subsection{}
In this section, we again assume that $F$ is a number field, $E$ is a quadratic extension of $G$,
and $\eta$ is a quadratic character attached to $E$.
We consider the pair $(G, T)$ of a group $\GL(2)$ and a subgroup $T$ of diagonal matrices;
fix an element $\varepsilon$ in the normalizer of $T$ which is not in $T$.
We denote by $K_v$ the usual maximal compact subgroup of $G_v$ and we assume $\varepsilon$ contained in $K_v$ for all $v$.
We define a finite set $S$ of places of $F$, containing the infinite places, the places which ramify in $E$, the places where $\psi$ has no order 0 and the places of residual characteristic 2.
It will be convenient to assume that $S$ has an even number of elements.
Let $X(S)$ be the set of pairs $(G', T')$ in $X(E:F)$ such that $G'$ splits outside $S$.
For each $(G', T')$ in $X(S)$ and each place $v$, we choose a maximal compact subgroup $K_v'$ of $G_v'$ so that $G'(\Aa_F)$ is the restricted product of $G_v'$ with respect to $K_v'$.
We assume that if $v$ does not split in $E$ then $T_v'$ is contained in $K_v' Z_v'$.
For all $v$ not in $S$ the measures of $T_v' \cap K_v' / K_v' \cap Z_v'$ and $T_v \cap K_v / K_v \cap Z_v$ are 1.
We fix an element in the normalizer of $T'$ which is not in $T'$, and assume that  $\varepsilon'$ is in $K_v'$ for all $v$ not in $S$.
Let $f$ be a compactly supported smooth function on $G(\Aa_F)/Z(\Aa_F)$ and, for each $(G', T')$ in $X(S)$, $f'$ be a compactly supported smooth function on $G'(\Aa_F) / Z'(\Aa_F)$.
Of course, these functions are assumed to be products of local functions.
We also make the following assumptions:
\begin{enumerate}
    \item Let $v\in S$ be a place that does not split in $E$.
    Then $f_v'$ is $T_v'$-bi-invariant.
    Moreover if $x$ is an element of $F_v$ that is not 1 or zero, $(G', T') \in X(S)$ and $g' \in G_v'$ with $x = P'(g': T_v')$ then
    \begin{equation*}
        H(x:f_v:\eta_v) = H(g':f_v':T_v').
    \end{equation*}
    \item Let $v \in S$ be a place that splits in $E$.
    Then $f_v$ is $K_v$-finite and $f_v'$ is $K_v'$-finite.
    Let $g$ be a $A_v$-regular element in $G_v$.
    If $(G', T') \in X(S)$ and $g' \in G_v$ satisfy
    \begin{equation*}
        P(g:T_v) = P(g':T_v')
    \end{equation*}
    then
    \begin{enumerate}
        \item \begin{equation*}
            H(g:f_v:T_v) = H(g':f_v':T_v')
        \end{equation*}
        \item \begin{equation*}
            \int_{T_v} f_v(a_v) da_v = \int_{T_v'} f_v'(t_v') dt_v'
        \end{equation*}
        \item \begin{equation*}
            \int_{T_v} f_v(\varepsilon a_v) da_v = \int_{T_v'} f_v'(\varepsilon't_v') dt_v'.
        \end{equation*}
    \end{enumerate}
    \item If $v$ is not in $S$ then $f_v$ is $K_v$-bi-invariant, $f_v'$ is $K_v'$-bi-invariant and the isomorphism between $(G_v, K_v)$ and $(G_v', K_v')$ induces a map from $f_v$ to $f_v'$.
    \item \emph{Remark}. In the situation of assumption (2) there is an isomorphism between $(G_v, T_v)$ and $(G_v', T_v')$.
    Condition (2) is satisfied if we take for $f_v'$ the image of $f_v$ under the isomorphism.
    Indeed this is clear for (2.a) and (2.b).
    For (2.c), the integral of the right hand side does not change if we replace $\varepsilon'$ by the image of $\varepsilon$ under the isomorphism in question and then our assertion is obvious.
\end{enumerate}

For given function $f$, we have a cuspidal kernel $K_c$ for the group $G$ associated to it.
Similarly, for each $(G', T')$, theres a cuspidal kernel $K_c'$ for the group $G'$ attached to the function $g'$.
In this section we will prove the following result:
\begin{theorem}
With the previous assumptions, we have
\begin{equation}
    \iint K_c(a, b) \eta(\det b) dadb = \sum_{(G', T') \in X(S)} \iint K_{c}'(s, t)dsdt.
\end{equation}
\end{theorem}


\subsection{}
To prove the identity, as in the section 7 and 9, we write
\begin{align}
    K_c &= K_r + K_s - K_{\mathrm{sp}} - K_{\mathrm{ei}}, \\
    K_c' &= K_r' + K_s' - K_{\mathrm{sp}}' - K_{\mathrm{ei}}'.
\end{align}
We will first prove the following identities:
\begin{align}
    \iint K_r(a, b) \eta(\det b) dadb &= \sum_{(G', T') \in X(S)} \iint K_{r}'(s, t) dsdt, \label{eqn:10.2.3} \\
    \iint K_s(a, b) \eta(\det b) dadb &= \sum_{(G', T') \in X(S)} \iint K_{s}'(s, t) dsdt. \label{eqn:10.2.4}
\end{align}
We first assume these identities and will show how the theorem follows from.
Consider the difference
\begin{equation}
\label{eqn:10.2.5}
    \iint K_c(a, b)\eta(\det b)dadb - \sum_{(G', T') \in X(S)} \iint K_c'(s, t)dsdt.
\end{equation}
Considering~\eqref{eqn:10.2.3} and~\eqref{eqn:10.2.4}, we write 
\begin{align*}
    -\iint K_{\mathrm{sp}}(a, b) \eta(\det b) dadb + \sum_{(G', T') \in X(S)} \iint K_{\mathrm{sp}}'(s, t)dsdt \\
    -\iint K_{\mathrm{ei}}(a, b) \eta(\det b) dadb + \sum_{(G', T') \in X(S)} \iint K_{\mathrm{ei}}'(s, t)dsdt.
\end{align*}
Recall that these are weak integrals for the group $G$.

Now choose a place $z$ of $E$ which is not in $S$ and splits in $E$.
Fix local factors of $f$ and $f'$ at the other places.
At the place $z$ the Satake transforms of $f_z$ and $f_z'$ are the same.
Hence we can regard our integrals as functions of $\hat{f_z}$.
Then by (8.1), (9.3) and (9.4) the above sum has the form of
\begin{equation}
    \int_{-\infty}^{\infty} \phi(t) \hat{f_z}(q_{z}^{-2it})dt + c \hat{f_z}(q_z^{-1}),
\end{equation}
where $\phi$ is integrable.
We finish the proof as in~\cite{langlands1980base} by using the fact that the integrals of $K_c$ and $K_c'$ also have the form
\begin{equation}
    \sum_{t} a_{t} \hat{f_z}(t),
\end{equation}
where the complex numbers $t$ are either on the unit circle or on the real axis between $q_z^{-1}$ and $q_z$ and the series $\sum_t a_t$ absolutely converges.
The uniqueness of the decomposition of a measure into an atomic measure and a continuous measure implies that the difference~\eqref{eqn:10.2.5} is zero.


\subsection{}
Let's prove~\eqref{eqn:10.2.3}.
LHS can be written as
\begin{equation*}
    \sum_{\zeta} H(\zeta:f:\eta), \quad \zeta \neq 0, 1,
\end{equation*}
where RHS can be written as a double sum
\begin{equation*}
    \sum_{(G', T')}\sum_{\zeta}H'(\zeta:f':T')
\end{equation*}
the inner sum is over all $1 \neq \zeta \in cN$, determined by the pair $(G', T')$.
We can combine the two sums and write RHS as a sum
\begin{equation*}
    \sum_{\zeta} H(\zeta:f':T'), \quad \zeta \in N(S) - \{1\},
\end{equation*}
where $N(S)$ is the union of the classes $cN$ corresponds to  the elements of $X(S)$.
According to the class field theory the elements of $F^\times - N(S)$ are exactly the $\zeta$ in $F^\times$ which satisfy the following condition: there exists a place $v$ of $F$, which is not in $S$, \textcolor{red}{inert} in $E$ and not a norm of the quadratic extension $E_v$ of $F_v$.
By Proposition (5.1) we have, for such a $\zeta$, $H(\zeta:f_v:\eta_v)=0$ if $v$ is the place in question.
This results in $H(\zeta:f:\eta)=0$.
Therefore it is sufficient to show the equality of the orbital integrals $H(\zeta:f:\eta)$ and $H(\zeta:f':T')$ when $\zeta$ is in $N(S)$.
Decompose these integrals into products of local integrals $H(\zeta:f_v:\eta_v)$ and $H(\zeta:f_v':T_v')$ respectively.
For $v$ in $S$ the equality of these integrals results from hypotheses (1) and (2).
For $v$ not in $S$ the equality follows from hypothesis (3) and proposition (5.1).
This proves the equality of the global orbital integrals, and the formula (3).


\subsection{}
Let's prove~\eqref{eqn:10.2.4}.
We can use (7.3.7) to compute LHS and (9.3.1) to compute RHS.
The equality~\eqref{eqn:10.2.4} will then be a consequence of the following two identities
\begin{align}
    H(n_+: f: \eta) + H(n_-:f:\eta) &= \sum_{(G',T') \in X(S)} \mathrm{vol}([T']) \int_{[T']} f'(t')\dd t', \label{eqn:10.4.1}\\
    H(\varepsilon n_+: f:\eta) + H(\varepsilon n_-:f:\eta) &= \sum_{(G',T')\in X(S)} \mathrm{vol}([T']) \int_{[T']} f'(\varepsilon't')\dd t'.
\end{align}

The second identity results from the first identity applied to the function $f_1$ defined by $f_1'(g) = f'(\varepsilon' g)$.
It is indeed easy to verify that the conditions (10.1.1) to (10.1.3) are satisfied by $f_1$ and $f_1'$.
Therefore let's prove the first identity.

Let's compute~\eqref{eqn:10.4.1}.
Let $a$ and $b$ be id\'eles of $E$ and $F$.
The analytic continuation of the Tate integral is 
\begin{equation}\label{eqn:10.4.3}
    \int \phi(t) |t|^{s} \eta(t) dt
\end{equation}
where $\phi$ is a Schwartz-Bruhat function, whose value at $s=0$ is
\begin{equation*}
    L(0, \eta) \prod_{v\in W} \int_{T_v} \phi_v(t_v) \eta(t_v) dt_v L(0, \eta_v)^{-1} \prod_{v\in V} \phi_v(0) |a_v|^{1/2},
\end{equation*}
where $W$ is a set of places of $F$ that do not split in $E$ and $V$ is a set of places split in $E$.

Apply this formula to the functions $\phi_{+}$ and $\phi_{-}$ defined as
\begin{align*}
    \phi_{+}(x) &= \int_{T(\Aa_F) / Z(\Aa_F)} f\left(a \bmat{1}{x}{0}{1}\right) da, \\
    \phi_{-}(x) &= \int_{T(\Aa_F) / Z(\Aa_F)} f\left(a \bmat{1}{0}{x}{1}\right) da.
\end{align*}

The local components of $\phi_{+}$ and $\phi_{-}$ are defined analogously in terms of the local decomponents of $f$.
Then the right hand side of~\eqref{eqn:10.4.1} is nothing but the sum of the values of the Tate integrals~\eqref{eqn:10.4.3} of $\phi_{+}$ and $\phi_{-}$ at the point $s=0$.
Moreover we obviously have for each $v$ in $V$:
\begin{equation*}
    \phi_{+v}(0) = \phi_{-v}(0) = \int_{T_v/Z_v} f_v(a_v) da_v.
\end{equation*}
On the other hand, for each $v$ in $W$ the values at point 0 of the Tate integrals of $\phi_{+v}$ and $\phi_{-v}$ are nothing but the singular orbital integrals 
of the points $n_+$ and $n_-$.
To simplify the notations, define
\begin{align*}
    M_v &= \int_{T_v/Z_v} f_v(a_v) da_v, \quad v\in V, \\
    M_{v\pm} &= 2 H(n_{\pm}:f_v:\eta_v), \quad v\in W.
\end{align*}
Then the LHS of~\eqref{eqn:10.4.1} can be written as
\begin{equation}\label{eqn:10.4.4}
    L(0, \eta) \prod_{v\in W} \frac{1}{2L(0, \eta_v)} \prod_{v \in V}|a_v|^{1/2} \times \prod_{v\in V}M_v  \left(\prod_{v\in W}M_{v+} + \prod_{v\in W}M_{v-}\right).
\end{equation}
Note that all but finitely many factors of each product are equal to 1.

Let's move on to RHS of~\eqref{eqn:10.4.1}.
The integral is obviously the product of similar local integrals:
\begin{equation*}
    \int_{T'(\Aa_F) / Z'(\Aa_F)} f'(t')\dd t' = \prod_{v} \int_{T_v' / Z_v'} f'_v(t_v') dt_v'.
\end{equation*}
If $v\in V$, the local integral is $M_v$ by the assumption (2.b).
If $v\in W$ the integral equals to
\begin{equation*}
    \frac{1}{2\mathrm{vol}(T_v'/Z_v')} (M_{v-} + \eta_v(c) M_{v+})
\end{equation*}
by Proposition (4.1) and Proposition (5.1).
The volume that appears in this formula is $|b_w|^{1/2}|a_v|^{-1/2}$, where $w$ is the only place of $E$ above $v$.
Hence RHS of~\eqref{eqn:10.4.1} is equal to the product
\begin{equation}
    2L(1,\eta) \sum_{c \in N(S)/N}\prod_{v\in W}\frac{1}{2L(0, \eta_v)} \prod_{v\in W} \bigg|\frac{a_v}{b_w}\bigg|^{1/2} \prod_{v\in V} M_v \prod_{v\in W} \frac{1}{2} (M_{v-} + \eta_v(c)M_{v-}).
\end{equation}
Comparing with~\eqref{eqn:10.4.4}, we can find that it suffices to prove the following identities:
\begin{align}
    L(0, \eta) \prod_{v\in V}|a_v|^{1/2} &= L(1, \eta) \prod_{v\in W} \bigg|\frac{a_v}{b_w}\bigg|^{1/2}, \label{eqn:10.4.6}\\
    \prod_{v\in W}M_{v+} + \prod_{v\in W} M_{v-} &= 2 \sum_{c \in N(S) / N} \prod_{v\in W} \frac{1}{2}(M_{v-} + \eta_v(c) M_{v+}). \label{eqn:10.4.7}
\end{align} 

The equality~\eqref{eqn:10.4.6} immediately follows from the functional equations of the terms $L(s, 1_E)$ and $L(s, 1_F)$ and their relation to $L(s, \eta)$.


Let's move on to~\eqref{eqn:10.4.7}.
For $v \in W-S$ we have $\eta_v(c)=1$ by the definition of $N(S)$ and $M_{v+} = M_{v-}$ (Proposition (5.1)); moreover for almost all $v \in W-V$, $M_{v+} = M_{v-} = 1$.
For $U = W \cap S$, we see that the identity~\eqref{eqn:10.4.7} reduces to
\begin{equation*}
    \prod_{v\in U} M_{v+} + \prod_{v\in U}M_{v-} = 2 \sum_{c\in N(S)/N} \prod_{v\in U} \frac{1}{2} (M_{v-} + \eta_v(c)M_{v+}).
\end{equation*}
Let $H = \{1, -1\}^{U}$.
For a place $v \in U$ define a character $\chi_v$ of $H$ by the formula $\chi_v(h) = h_v$.
Let $H'$ be a subgroup of $H$ defined by the equation $\prod_{v\in U}\chi_v(h) = 1$.
Then the map $c \mapsto (\eta_v(c))$ defines a bijection between $N(S)/N$ and $H'$.
Then the formula that we want to prove can be written as
\begin{equation*}
    \prod_{v\in U} M_{v+} + \prod_{v\in U} M_{v-} = 2 \sum_{h\in H} \prod_{v\in U} \frac{1}{2}(M_{v-} + \chi_v(h)M_{v+}).
\end{equation*}
Since $|H| = 2|H'|$, RHS can be written as
\begin{equation*}
    \frac{1}{|H'|} \sum_{Y} \sum_{h\in H'} \prod_{v\in Y} \chi_v(h) \prod_{v \in Y} M_{v+} \prod_{v\in U-Y} M_{v-},
\end{equation*}
where the outer sum is over the subset $Y$ of $U$.
The character $\prod_{v\in Y} \chi_v$ is nontrivial on $H'$, unless $Y$ is empty or equal to $U$.
Therefore only the terms corresponding to the empty set and $U$ contributes to the sum; this gives us our equality.

\section{Waldspurger's result}


\subsection{}
We can finally prove Waldspurger's result using the identites from section 10.
We will denote by $S$ a finite set of places of $F$ satisfying the conditions of section 10.
By taking $S$ sufficiently large, we can only consider the cuspidal representations of $G$ unramified outside of $S$, pairs $(G', T')$ belongto $X(S)$ and, for such a pair, cuspidal representations of $G'$ unramified outside $S$.
We will denote by $K$ (resp. $K^S$) for the product of the compact subgroups $K_v$ for all $v$ (resp.\ all $v$ not in $S$) and $G^S$ the restricted product of $G_v$ for $v$ not in $S$.
For a pair $(G', T')$ in $X(S)$, we denote as $K', K'^S$ and $G'^S$ for the analogous groups.

To describe Waldspurger's first condition, note that if
\begin{equation*}
    \int \phi_1(a)da \quad \text{and} \quad \int \phi_2(b)\eta(\det b)db
\end{equation*}
is nonzero for some pair $(\phi_1, \phi_2)$ of smooth vectors, then it is also nonzero for some $K$-finite vectors $(\phi_1, \phi_2)$; moreover, if $S$ is large enough, then $\phi$ and $\phi'$ are $K^S$-invariant.
Similarly if there exists a pair $(G', T')$ in $X$, a cuspidal representation $\pi'$ and a smooth vector $\phi'$ in the space of $\pi'$ such that the integral $\int \phi'(t')dt'$ is nonzero, then we can take $\phi'$ to be $K'$-finite; moreover, if $S$ is large enough, $(G', T')$ is in $X(S)$ and $\phi'$ is invariant under $K'^S$.


\subsection{}
Consider a set $S$ and functions $f$ and $f'$ satisfying the conditions in section 10.
In particular $f_v$ (resp. $f_v'$) is bi-invariant under $K_v$ (resp. $K_v'$).
Consider the kernel $K_c$.
We can write as
\begin{equation}
    \label{eqn:11.2.1}
    K_c = \sum_{\pi} K_\pi
\end{equation}
where, for each cuspidal automorphic representation $\pi$ (unramified outside of $S$), we have
\begin{equation}
    \label{eqn:11.2.2}
    K_\pi(x, y) = \sum_{j} \rho(f) \phi_j(x) \overline{\phi_j}(y),
\end{equation}
where $\{\phi_j\}$ is an orthonormal basis of the subspace of $K^S$-invariant vectors in the space of $\pi$.
The series~\eqref{eqn:11.2.1} not only converges in the Hilbert space of square integrable functions on the quotient $[G]$, but also in the space of rapidly decreasing functions on the quotient $[G]$.
Moreover, since $f_v$ is $K_v$-finite for all $v$ infinite, the series~\eqref{eqn:11.2.2} has only a finite number of nonzero terms for given $f$.
Denote by $H(S)$ the Hecke algebra of the group $G^S$ relative to the subgroup $K^S$. 
Write $f = f_S f^S$, where $f_S$ (resp. $f^S$) is the product of $f_v$ for $v$ in $S$ (resp.\ not in $S$).
Let $\Lambda_\pi$ be the character of $H(S)$ attached to a representation $\pi$.
Then we have
\begin{equation}
    \iint K_c(a, b) \eta(\det b) dadb = \sum_{\pi} a(\pi, f_S) \Lambda_\pi(f^S),
\end{equation}
where $a(\pi, f_S)$ is
\begin{equation}
    a(\pi, f_S) = \sum_j \int \rho(f_S) \phi_j(a) da \int \overline{\phi_j}(b) \eta(\det b) db.
\end{equation}


\subsection{}
Consider $(G', T') \in X(S)$.
We have a similar decomposition
\begin{align}
    K_c' &= \sum_{\pi'} K_{\pi'}, \label{eqn:11.3.1} \\
    K_{\pi'}(x, y) &= \sum_j \rho(f') \phi_j'(x) \overline{\phi_j'}(y), \label{eqn:11.3.2}
\end{align}
where $\{\phi_j'\}$ is an orthonormal basis of the subspace of $K'^S$-invariant vectors in the space of $\pi'$.
The series~\eqref{eqn:11.3.1} also converges in the space of rapidly decreasing functions and the series~\eqref{eqn:11.3.2} has only finitely many nonzero terms.
By integrating term by term, we get
\begin{equation}
    \iint K_\pi'(s, t)dsdt = a(\pi', f_S') \Lambda_\pi'(f'^S)
\end{equation}
where $a(\pi', f_S')$ is given by
\begin{equation}
    a(\pi', f_S') = \sum_j \int \rho(f) \phi_j'(s) ds \int \overline{\phi_j'}(t) dt.
\end{equation}
Then the whole integral of $K_c'$ becomes
\begin{equation*}
    \iint K_c'(s, t) dsdt = \sum_{\pi'}a(\pi', f_S') \Lambda_{\pi'}(f'^S).
\end{equation*}



\subsection{}
Now let's use our fundamental lemma.
Note that if $\pi'$ is a cuspidal representation of $G'$ and $\pi$ the corresponding cuspidal representation of $G$, then $\Lambda_\pi(f_S) = \Lambda_{\pi'}( f_{S}')$.
Since $\pi'$ determines $\pi$ we can write $a(\pi, f_S')$ for $a(\pi', f_S')$.
On the other hand, for a representation $\pi$ of the group $G$, it will be convenient to set $a(\pi, f_S')=0$ if there is no representation $\pi'$ of $G'$ corresponding to $\pi$.
Then our fundamental lemma is written as
\begin{equation}
    \label{eqn:11.4.1}
    \sum_{\pi} a(\pi, f_S)\Lambda_{\pi}(f^S) = \sum_{\pi} \sum_{(G', T')} a(\pi, f_S') \Lambda_{\pi}(f^S)
\end{equation}
Here $f^S$ is an arbitrary element of $H(S)$.
Let $v$ be a place in $S$.
Then the function $f_v$ is an arbitrary $K_v$-finite function.
The function $f_v'$ matches with $f_v$ by the conditions (10.1.1) and (10.1.2).
If $v$ splits then $f_v'$ is in fact can be chosen as an arbitrary $K_v'$-finite function.
If $v$ does not split, then $f_v'$ is no longer arbitrary but satisfies a density condition: if a function $h$ is bi-$T_v$-invariant on $G_v/Z_v$ and orthogonal to all $f_v'$ then $h$ is zero (Proposition 4.2).
Suppose that $\pi$ satisfies Waldspurger's first condition; then there exist $K$-finite vectors $\phi_1$ and $\phi_2$ in the space of $\pi$ such that:
\begin{equation*}
    \int \phi_1(a)da \neq 0 \quad \text{and}\quad \int \phi_2(b) \eta(\det b) db \neq 0,
\end{equation*}
and we can assume $\phi_1$ and $\phi_2$ to be $K^S$-invariant.
Choose a basis $\{\widetilde{\phi_j}\}$ such that $\widetilde{\phi_1} = \phi_2 / \|\phi_2\|$.
There exists $f_S$ such that $\rho(f_S)\widetilde{\phi_1} = \phi_1$ and $\rho(f_S) \widetilde{\phi_j} = 0$ if $j\neq 1$.
Then we have
\begin{equation*}
    a(\pi, f_S) = \int \phi_1(a) da  \overline{\int \widetilde{\phi_1}(b) \eta(\det b)db}\neq 0.
\end{equation*}

According to the principle of \emph{infinite} linear independence of the characters of $H(S)$~\cite{langlands1980base}, 
there exists a pair $(G', T')$ such that $a(\pi, f_S')$ is nonzero.
It clearly follows that there exists $\phi'$ in the space of $\pi'$ such that $\int \phi'(t') dt'$ is nonzero.
Thus $\pi$ satisfies the second condition of Waldspurger.

Now assume that there exists a pair $(G', T')$, a representation $\pi'$ and a $K'$-finite vector $\phi'$ in the space of $\pi'$ such that the integral $\int \phi'(t')dt'$ is nonzero; we can assume that $(G', T')$ is in $X(S)$ and $\phi'$ is invariant under $K'^S$.
We will see that we can choose $f_S'$ so that $a(\pi', f_S')$ is nonzero.
The integral over $T'$ defines a continuous linear functional on the space of smooth vectors of $\pi'$ fixed by $K'^S$.
Let's write it as the dot product with a \emph{generalized} vector $e_{T'}$:
\begin{equation*}
    \int \phi'(t') dt' = (\phi', e_{T'}).
\end{equation*}
If $h$ is a compactly supported smooth function on $G_S / Z_S$, then $\pi'(h)(e_{T'})$ is defined: it is a smooth vector satisfies $(\phi, \pi'(h)e_{T'}) = (\pi'(h^*)\phi, e_{T'})$ for any vector $\phi$, smooth or not.
With this notation we have
\begin{equation*}
    a(\pi, f_S') = (\pi'(f_S')e_{T'}, e_{T'}).
\end{equation*}
The subspace of $K'^S$-invariant vectors of $\pi'$ is isomorphic to the tensor product of the spaces of $\pi_v'$ with $v$ in $S$.
For each $v$ in $S$, there exists a continuous linear functional on the space of smooth vectors of $\pi'$ nonzero at $e_v'$ which is invariant under $T_v'$.
This functional is unique up to scalar (cf. (6.1) and (6.2)), and we can therefore write:
\begin{equation*}
    a(\pi', f_S') = (\pi'(f_S')e_{T'}, e_{T'}) = C \prod_{v\in S} (\pi_v'(f_v')e_v', e_v'),
\end{equation*}
where $C$ is a nonzero constant.
We want to show that we can choose $f_v'$ so that $(\pi_v(f_v')e_v', e_v')$ is nonzero.
It is obvious when $v$ splits since $f_v'$ is then arbitrary $K_v'$-finite.
If $v$ splits $e_v'$ is in fact an ordinary vector since $T_v'$ is compact.
Then $(\pi_v(f_v')e_v', e_v')$ is the scalar product of the function $f_v'$ with the continuous matrix coefficient $(\pi'(g)e_v', e_v')$; therefore it cannot be zero for any $f_v'$ according to the density property of $f_v'$.
On the other hand if $(G'', T'')$ is another element of $X(S)$, then $a(\pi, f_S'')=0$; otherwise there would exist at least one place $v$ in $S$ where the groups $G_v'$ and $G_v''$ are not isomorphic and the representations $\pi_v'$ and $\pi_v''$ admit nonzero vectors that are invariant under $T_v'$ and $T_v''$ respectively.
But this is impossible (Proposition (6.3)).
The coefficient of $\Lambda_\pi$ in the RHS of~\eqref{eqn:11.4.1} is therefore nonzero, for a suitable choice of $f_S'$ (i.e. $f_S$).
This results that $a(\pi, f_S)$ is not zero.
This obviously implies that $\pi$ satisfies Waldspurger's first condition.
Therefore this completes the proof of the equivalence of the two Waldspurger conditions.


% --- Bibliography ---

% Start a bibliography with one item.
% Citation example: "\cite{williams}".

\nocite{*}
\bibliographystyle{acm} % We choose the "plain" reference style
\bibliography{refs} % Entries are in the refs.bib file


% \begin{thebibliography}{1}

% \bibitem{williams}
%    Williams, David.
%    \textit{Probability with Martingales}.
%    Cambridge University Press, 1991.
%    Print.

% % Uncomment the following lines to include a webpage
% % \bibitem{webpage1}
% %   LastName, FirstName. ``Webpage Title''.
% %   WebsiteName, OrganizationName.
% %   Online; accessed Month Date, Year.\\
% %   \texttt{www.URLhere.com}

% \end{thebibliography}

% --- Document ends here ---

\end{document}