\section{Orbital integrals: unramified case}

\subsection{}
In this section, $F$ is a non-archimedean local field and $E$ is an unramified quadratic extension of $F$.
Assume that the residual characteristic of $F$ is not 2 and the order of character $\psi$ is 0.\footnote{Order of a character $\psi$ is a smallest integer $r\geq 0$ such that $\varpi ^r R \subseteq \ker \psi$.}
Consider a tuple $(G, T)$ of a group $\GL(2)$ and a diagonal subgroup $T$.
We denote by $R$ the ring of integers of $F$, $\mathfrak{p}$ a maximal ideal of $R$, $\varpi$ a  uniformizer and $K = \GL(2, R)$.
The set $X = X(E:F)$ reduces to a set of two elements $(G_1', T_1')$ and $(G_2', T_2')$.
Now suppose $G_1' = G$ and $T_1'$ is contained in the subgroup $ZK$.
We will simply write $T'$ for $T_1'$.
The measures of $T\cap K / Z \cap K$ and $T' \cap K / Z\cap K$ are therefore equal to 1.
The goal of this section is to prove the following proposition.

\begin{proposition}\label{prop:5.1}
Lef $f$ be a $K$ bi-invariant compactly supported function on $G/Z$.
The $f$ matches with the pair $(f, 0)$.
Also, 
\begin{align*}
    H(n_+: f: \eta) = H(n_-: f: \eta) = \frac{1}{2} \mathrm{vol}(T'/Z) \int_{T'/Z} f(t')dt'.
\end{align*}
\end{proposition}
It will be convenient to consider functions with compact support on $G$ rather than functions with compact support on $G/Z$.
Of course the measures of the sets $T\cap K$, $T'\cap K$, and $Z\cap K$ are therefore equal to 1.
If $f$ is a $K$ bi-invariant function with compact support on $G$, then we set
\begin{align}
    H'(g:f:T') = \int_{T'/Z} \int_{T'} f(s'gt') ds'dt'.
\end{align}
Since $T'$ is contained in $ZK$ this reduces to
\begin{align}
    H'(g:f:T') = \int_{Z} f(zg) dz.
\end{align}
Likewise, we define
\begin{align}
    H(g:f:\eta) = \int_{T/Z} \int_{T} f(agb) \eta(\det b) dadb.
\end{align}
We write $H(x:f:\eta)$ again for $H(g(x):f:\eta)$.
Then we have to prove the following identities
\begin{align}
    H(x:f:\eta) &= \int_{Z} f(zg) dz \quad \text{if } v(x) \text{ is even and } P(g:T') = x \label{eqn:5.1.4}\\
    H(x:f:\eta) &= 0 \quad \text{if } v(x) \text{ is odd}. \label{eqn:5.1.5}
\end{align}
By linearity we can assume that $f$ is either the characteristic function $f_0$ of $K$, or the characteristic function $f_m$ of the set
\begin{align}
    K \bmat{\varpi^m}{0}{0}{1}K, \quad m > 0.
\end{align}
Note that $f_m(g) \neq 0$ if and only if the following conditions hold:
\begin{itemize}
    \item the entries of $g$ are integers,
    \item $v(\det g) = m$,
    \item at least one of the entries of $g$ is a unit.
\end{itemize}
Note that the condition trivially holds when $m = 0$.

\subsection{}
We will first compute $H(x:f_m:\eta)$ which, for simplicity, denote by $H(x:m)$.
Let's assume $m >0$ first.
\begin{proposition}\label{prop:5.2}
Suppose $m > 0$.Then $H(x:m)$ is given by the following formulas:
\begin{enumerate}
    \item if $v(x)$ is odd then $H(x:m)=0$.
    \item if $v(x)$ is even then $H(x:m)=0$, unless $v(x)=0$ and $v(1-x)=m$ in which case $H(x:m )=1$.
\end{enumerate}
\end{proposition}
\begin{proof}
We will use the following lemma:
\begin{lemma}
Let
\[
S = \sum_{i, j} (-1)^{i+j}
\]
where the summation is over all the pairs of integers $(i, j)$ on the edge of the rectangle defined by the inequalities
\[
0 \leq i \leq P, \quad 0 \leq j \leq Q.
\]
Then $S$ is given by the following formulas:
\begin{enumerate}
    \item if $PQ > 0$ then $S =0$,
    \item if $P = 0$ and $Q >0$, then $S=1$ if $Q$ is even and $S=0$ if $Q$ is odd,
    \item if $Q=0$ and $P >0$, the $S=1$ if $P$ is even and $S=0$ if $P$ is odd,
    \item if $P=Q=0$ then $S=1$.
\end{enumerate}
\end{lemma}
We not prove the proposition.
We write $\mathrm{Mat}[a, b, c, d]$ for the matrices with entries $a, b, c, d$.
With this notation we have
\begin{align}
\label{5.2.7}
    H(x:m) = \sum_{i, j, k} f_{m}(\mathrm{Mat}[\varpi^{i+k}, x\varpi^{j+k}, \varpi^{i}, \varpi^{j}]) (-1)^{i+j},
\end{align}
where the sum is over all triples of integers $(i, j, k)$.
As the determinant of the matrices in~\eqref{5.2.7} has valuations equal to $i+j+k+v(1-x)$, from the condition $v(\det g) = m$ we can restrict ourselves to triples $(i, j, k)$ with 
\begin{align*}
    i+j+k+v(1-x) = m.
\end{align*}
This allows us to eliminate $k$ and, by the previous conditions on the integrality of entries of $g$,
\begin{align}
    H(x:m) = \sum_{i, j}(-1)^{i+j}
\end{align}
where the summation is over all the pairs of integers $(i, j)$ such that
\begin{gather}
    0 \leq i \leq m - v(1-x) + v(x) \\
    0 \leq j \leq m - v(1-x) \\
    ij(m-v(1-x)+v(x)-i)(m-v(1-x)-j)=0.
\end{gather}
The sum is empty and $H(x:m)$ is zero unless
\begin{align}
\label{5.2.12}
    m - v(1-x) \geq 0 \quad \text{and} \quad m - v(1-x) + v(x) \geq 0.
\end{align}
Suppose~\eqref{5.2.12} holds.
Then we can apply the lemma, and we have $H(x:m)=0$ unless
\begin{align}
    (m-v(1-x))(m-v(1-x)+v(x)) =0.
\end{align}
Then the proposition follows from elementary calculations.
\end{proof}

\subsection{}
Let's compute $H(x:0)$.
\begin{proposition}\label{prop:5.3}
    $H(x:0)$ is given by the following formulas:
    \begin{enumerate}
        \item if $v(x)$ is odd then $H(x:0) =0$,
        \item if $v(x)$ is even then $H(x:0) =1$, unless $v(x)=0$ and $v(1-x)>0$ in which case $H(x:0)=0$.
    \end{enumerate}
\end{proposition}
\begin{proof}
We have
\begin{align}
    H(x:0) = \sum_{i, j, k} f_{0}(\mathrm{Mat}[\varpi^{i+k}, x\varpi^{j+k}, \varpi^i, \varpi^j]) (-1)^{i+j},
\end{align}
where the sum is over all triples of integers $(i, j, k)$.
As above, based on the conditions on $g$ to be $f_0(g)\neq 0$, we can eliminate $k$ and write
\begin{align}
    H(x:0) = \sum_{i, j} (-1)^{i+j}
\end{align}
where the sum is over all pairs of integers $(i, j)$ such that
\begin{align}
    0 \leq i \leq v(x) - v(1-x) \\
    0 \leq j \leq -v(1-x).
\end{align}
Then the proposition follows from elementary calculations.
\end{proof}

\subsection{}
We will compute $\int_Z f_m(zg) dz$.
It only depends on $x = P(g:T')$ and we denote $H'(x:m:T')$ for its value.
Recall that by definition $x$ is a norm, in other words the valuation of $x$ is even.
We will start with the case $m > 0$.
\begin{proposition}\label{prop:5.4}
    Suppose $m > 0$.
    Then $H'(x:m:T')=0$, unless $v(x) = 0$ and $v(1-x)=m$ in which case $H'(x:m:T')=1$.
\end{proposition}
\begin{proof}
We can assume that $E$ is an extension generated by the square root of $\tau$, where $\tau$ is a unit.
Then we can take for $T'$ the multiplicative group of the following algebra
\begin{align}
    L = \left\{\bmat{a}{b}{b\tau}{a}\right\}
\end{align}
and $\varepsilon$ is a matrix
\begin{align}
    \varepsilon = \bmat{1}{0}{0}{-1}.
\end{align}
Let's compute $H'(x:m:T')$ for $x = P'(g:T')$.
We can assume that 
\begin{align}
    g = \bmat{1}{0}{0}{1} + \varepsilon \bmat{u}{v}{v\tau}{u}.
\end{align}
so $\det g = 1 - x$ and $x = y^2 - v^2 \tau$.
We have
\begin{align}
    H'(x:m:T') = \sum_{k}f_{m}(\varpi^k g)
\end{align}
and
\begin{align}
    \label{5.4.5}
    \varpi^k g = \bmat{\varpi^k(1+u)}{\varpi^k v}{-\varpi^k v\tau}{\varpi^k(1-u)}.
\end{align}
By the integrality condition of entries of $g$, this sum consists of a single term where $k$ is determined by the equation
\begin{align}
    \label{5.4.6}
    k = \frac{1}{2}(m - v(1-x)).
\end{align}
In particuler $H'(x:m:T') = 0$ or 1.
By the integrality conditions again and $\det g = m$, $H(x:m:T') = 1$ if and only if the followings hold:
\begin{itemize}
    \item $m = v(1-x) \,\mathrm{\mod}\,2$,
    \item the entries of the matrix in~\eqref{5.4.5} with $k$ given by~\eqref{5.4.6} are integers,
    \item at least one of the entries of this matrix is a unit.
\end{itemize}
First assume $v(x) <0$.
We have $v(1-x) = v(x)$.
Since $x =u^2 - v^2 \tau$ and $\tau$ is not a square $v(x)$ is even.
Then $H'(x:m:T')=0$ unless $m$ is even.
So let's assume that this is the case.
We can write
\begin{align*}
    u = u_0 \varpi^{v(x) /2}, \quad v = v_0 \varpi^{v(x) / 2},
\end{align*}
where $u_0, v_0$ are integrals, with at least one being a unit.
Then the entries of the matrix~\eqref{5.4.5} are:
\begin{align*}
    \varpi^{(m  - v(x)) / 2} (1 + u_0 \varpi^{v(x) / 2}), \quad \varpi^{m/2}v_0 \\
    -\varpi^{m/2} v_0 \tau, \quad \varpi^{(m - v(x))/2} (1 - u_0 \varpi^{v(x) / 2}).
\end{align*}
All of these are in $\mathfrak{p}$ so $H'(x:m:T') = 0$.

Supose $v(x) > 0$. We have $v(1 - x) = 0$.
From $m \equiv v(1-x)\,(\mathrm{\mod}\,2)$ we have $H'(x:m:T') = 0$ unless $m$ is even.
Assume $m$ is even.
Since $k = m / 2$, $u$ and $v$ are integral.
The entries of the matrix in~\eqref{5.4.5} are
\begin{align*}
    \varpi^{m/2} (1+u), \quad \varpi^{m/2} v, \\
    -\varpi^{m/2} v\tau, \quad \varpi^{m/2} (1 - u).
\end{align*}
All of these are in $\mathfrak{p}$ so $H'(x:m:T') =0$.

Lastly, assume $v(x) = 0$. We have $v(1 - x) \geq 0$.
If $m - v(1-x)$ is odd, then $H'(x:m:T') =0$. Let's assume that $m - v(1-x)$ is even.
Then the entries of the matrix in~\eqref{5.4.5} are
\begin{align*}
    \varpi^{(m - v(1-x))/2} (1 +u), \quad \varpi^{(m - v(1-x))/2} v \\
    -\varpi^{(m - v(1-x)) /2} v\tau, \quad \varpi^{(m - v(1-x)) / 2} (1 - u).
\end{align*}
Since $x$ is a unit, $u$ and $v$ are both integral and at least one is a unit.
If $1 + u$ and $1 - u$ are both in $\mathfrak{p}$ we would have $2 \in \mathfrak{p}$, a contradiction.
So at least one of the numbers $1 +u$ and $1 - u$ is a unit.
If $H(x:m:T')$ is not zero, integrality condition gives $m = v(1-x)$.
The entries of the matrix~\eqref{5.4.5} are therefore reduced to
\begin{align*}
    1+u, \quad v, \quad -v\tau, \quad 1 -u.
\end{align*}
These are all integral and at least one of them is a unit. Hence $H'(x:m:T') = 1$.

So we have computed $H'$ completely and the proposition follows.
\end{proof}

\subsection{}
Let's compute $H'(x:0:T')$.
Recall that $v(x)$ is even.
\begin{proposition}\label{prop:5.5}
$H'(x:0:T')=1$, unless $v(x)=0$ and $v(1-x)>0$ in which case $H'(x:0:T')=0$.
\end{proposition}
\begin{proof}
As above we have
\begin{align}
    H'(x:m:T') = \sum_{k} f_{0}(\varpi^k g).
\end{align}
The sum has at most one term whose index $k$ is given by
\begin{align}
\label{eqn:5.5.2}
    k = - \frac{v(1-x)}{2}.
\end{align}
In particular, $H'(x:0:T')=0$ or $1$.
Moreover $H'(x:0:T')=1$ if and only if $v(1-x)$ is even and the matrix
\begin{align}
\label{eqn:5.5.3}
    \varpi^k g = \bmat{\varpi^k(1+u)}{\varpi^k v}{-\varpi^k v\tau}{\varpi^k(1-u)}
\end{align}
with $k$ given by~\eqref{eqn:5.5.2} is in $\GL(2, R)$.

Assume that $v(x) < 0$ and $v(1-x)$ is even.
Then $v(1-x) = v(x)$, $v(x)$ is even and
\begin{align*}
    u = u_0 \varpi^{v(x)/2}, \quad v = v_0 \varpi^{v(x) / 2}
\end{align*}
where $u_0$ and $v_0$ are integral, at least one being unit.
Then the entries of the matrix~\eqref{eqn:5.5.3} are the numbers
\begin{align*}
    \varpi^{-v(x) / 2} + u_0, \quad v_0, \quad -v_0 \tau, \quad \varpi^{-v(x) / 2} - u_0.
\end{align*}
They are integral.
As the determinant of the matrix~\eqref{eqn:5.5.3} is a unit according to the choice of $k$ the matrix~\eqref{eqn:5.5.3} is in $\GL(2, R)$ and $H'(x:0:T')=1$.

Suppose $v(x) \geq 0$ and $v(1-x)=0$ (of course $v(x) > 0$ leads to $v(1-x)=0$).
Then $k=0$ and the entries of the matrices~\eqref{eqn:5.5.3} reduce to the numbers
\begin{align*}
    & \varpi^{-v(1-x) / 2} (1+u), \quad \varpi^{-v(1-x)/2}v \\
    & \varpi^{-v(1-x)/2}v\tau, \quad \varpi^{-v(1-x)/2}(1-u).
\end{align*}
As $1+u$ or $1-u$ is a unit at least one of these numbers is not integral so~\eqref{eqn:5.5.3} is not in $\GL(2,R) $ and $H'(x:0:T')=0$.
\end{proof}

\subsection{}
By comparing the propositions~\eqref{prop:5.2},~\eqref{prop:5.3}, 
~\eqref{prop:5.4} and~\eqref{prop:5.5} we see that we have proved the identities~\eqref{eqn:5.1.4} and~\eqref{eqn:5.1.5}.
This therefore completes the proof of the first assertion of proposition~\eqref{prop:5.1}.
The second then follows from proposition~\eqref{prop:4.1}.

\subsection{}
To establish the convergence of the global orbital integrals we will need an additional result, the proof of which we will leave to the reader.

\begin{lemma}
Suppose $h$ in $KZ$ and let $x = P(h:T)$.
Suppose $v(x) = 0$ and $v(1-x)=0$.
Then the relation
\begin{align*}
ahb \in KZ, \quad a \in T, \quad b \in T
\end{align*}
leads to
\begin{align*}
a \in Z(K \cap T), \quad b \in Z(K \cap T).
\end{align*}
\end{lemma}
The lemma implies the following proposition.
\begin{proposition}
Let $f$ be the characteristic function of $KZ$.
Suppose $E$ is an unramifeid quadratic extension and $T'$ contained in $KZ$.
Let $h$ be an element of $KZ$ and $x = P(h:T)$.
If $x$ and $1-x$ are units then
\begin{align*}
H(h:f:T) =1, \quad H(h:f:\eta) = 1.
\end{align*}
\end{proposition}