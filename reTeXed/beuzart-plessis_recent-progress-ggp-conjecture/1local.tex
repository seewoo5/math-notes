\section{The Local Conjectures}

\subsection{The groups}

Let $E/F$ be a quadratic extension of local fields of characteristic zero.
We therefore have either $E/F = \bC / \bR$ or that $E$ and $F$ are finite extensions of the field of $p$-adic numbers $\bQ_p$ for a certain prime number $p$ ($\bQ_p$ is the completion of $\bQ$ by the $p$-adic absolute value $|\cdot|_p$ defined by $|p^k \frac{a}{b}|
_p = p^{-k}$ for $a$ and $b$ integers prime to $p$).
We denote by $\sigma$  the unique non-trivial element of the Galois
group $\Gal(E/F)$ and $\sgn_{E/F}$ the quadratic character of $F$ associated with the extension $E/F$ by the class field theory (it is therefore the unique quadratic character with kernel $\rN_{E/F}(E^\times)$, the image of the norm map).
Finally, we will fix two non-trivial additive characters $\psi_0: F \to \bS^1$ and $\psi: E \to \bS^1$ with the property that $\psi$ is trivial on $F$.

Let $V$ be a finite dimensional vector space of dimension $n$ over $E$ and $\varepsilon \in \{ \pm 1 \}$.
We assume $V$ is equipped with a non-degenerate $\varepsilon$-hermitian form
\[
    \langle -, -\rangle: V \times V \to E.
\]
By definition a $\varepsilon$-hermitian form satisfies
\begin{align*}
    \langle \lambda v + \mu w, u \rangle &= \lambda \langle v, u \rangle + \mu \langle w, u \rangle \\
    \langle v, u \rangle &= \varepsilon \langle u, v \rangle^\sigma
\end{align*}
for all $u, v, w \in V$ and $\lambda, \mu \in E$.
Depending on whether $\varepsilon = 1$ or $-1$ we call it hermitian or skew-hermitian.
Let $W$ be a non-detenerate subspace of $V$ with
\[
    \dim(V) - \dim(W) = \begin{cases} 1 &\text{if }\varepsilon = 1 \\ 0 & \text{if } \varepsilon = -1. \end{cases}
\]
Let $\rU(V) \subset \GL(V)$ and $\rU(W) \subset \GL(W)$ be the algebraic subgroups (defined over $F$) of linear automorphisms of $V$ and $W$ preserving the form $\langle -, - \rangle$.
Then $\rU(V)$ and $\rU(W)$ are unitary groups and we have a natural embeddign $\rU(W) \hookrightarrow \rU(V)$ where $\rU(W)$ acts trivially on $W^\perp$ (which of dimension at most 1).
In the following we will (abusively) identify an algebraic group defined on $F$ with the group of $F$-points corresponding to it.

The following discussion also extends to the case where $E = F \times F$ equipped with the involution $\sigma(x, y) = (y, x)$, a case which it will be necessary to include anyway when we will deal with the global conjecture.
In such a situation, a non-degenerate form $\langle -, -\rangle$ as above identifies $V$ and $W$ to direct sums $V_0 \oplus V_0^\vee$ and $W_0 \oplus W_0^\vee$ where $W_0 \subset V_0$ are the finite dimensional vector spaces over $F$ and $V_0^\vee$, $W_0^\vee$ denote their duals.
We then have a natural identifications $\rU(V) \simeq \GL(V_0)$ and $\rU(W) \simeq \GL(W_0)$.

In all cases, we put $G = \rU(W) \times \rU(V)$, $H = \rU(W)$ and we embed $H$ into $G$ diagonally.
The groups $H$ and $G$ inherit from the field $F$ topologies which make them Lie groups in the archimedean case (i.e. when $F = \bR$) and locally profinite groups in the non-archimedean case (i.e. when $F$ is a finite extension of $\bQ_p$; recall that a topological group is locally profinite if it has a basis of neighborhoods of the identity element consist of compact subgroups).


\subsection{The restriction problem}

Let $(\pi, \scV)$ be a smooth and irreducible complex representation of $G$.
In the $p$-adic case, this means that $\pi$ is a representation of $G$ on a $\bC$-vector space $\scV$ (typically of infinite dimension) all of whose vectors have a open stabilizer, irreducibility is then
an algebraic notion (i.e. no non-trivial subspace stable under $G$).
In the archimedean case, this means that $\scV$ is a Fréchet space and that $\pi$ is a smooth representation (in the $C^\infty$ sense), admissible (i.e. the irreducible representations of a maximal compact subgroup appear with finite multiplicities) on $\scV$ satisfying a certain condition of “moderate growth” (which was introduced by Casselman and Wallach, see \cite{casselman1989canonical} and \cite{wallachreal} Chap. 11); irreducibility is then a topological notion (ie no non-trivial closed subspace stable by $G$).
In any case, such an irreducible representation decomposes as a tensor product $\pi = \pi_W \boxtimes \pi_V$ where $\pi_W$ and $\pi_V$ are irreducible (smooth) representations of $\rU(W)$ and $\rU(V)$ respectively (and where the tensor product is a topological tensor product in the archimedean case).
We will denote as $\Irr(G)$ for the set of isomorphism classes of smooth irreducible representations of $G$.

To define the restriction problem that will interest us, we must also introduce a certain “small” representation $\nu$ of $H$.
In the hermitian case (i.e. if $\varepsilon = 1$), $\nu$ is the trivial representation that we will denote as $1$ or simply $\bC$ in the following.
In the skew-hermitian case (i.e. if $\varepsilon = -1
$), we have an inclusion
\[
    \rU(W) \subset \Sp(\Res_{E/F}W)
\]
where $\Res_{E/F}W$ denotes the restriction of the scalars from $E$ to $F$ of $W$ equipped with the symplectic form $\Tr_{E/F} \circ \langle -, -\rangle$ and $\Sp(\Res_{E/F}W)$ denotes the corresponding symplectic group.
Let $\Mp(\Res_{E/F}W)$ be the metaplectic group associated with this symplectic space (it is a $\bZ/2\bZ$-extension of $\Sp(\Res_{E/F}W)$).
The metaplectic covering splits over $\rU(W)$ but this splitting is not unique (because there are non-trivial characters $\rU(W) \to \{\pm 1\}$).
We can, however, fix such a splitting by choosing a character $\mu: E^\times \to \bS^1$ with $\mu|_{F^\times} = \sgn_{E/F}$ from now on.
Let $\omega_{\psi_0, W}$ be the Weil representation of $\Mp(\Res_{E/F} W)$ associated to the character $\psi_0$ (c.f. \cite{moeglin2006correspondances} Chap. 2. II).
Then $\nu = \omega_{\psi_0, W, \mu}$ is the restriction of this Weil representation to $\rU(W)$ via the splitting that we have just fixed.

For every case, the space of intertwining maps which is of our interest is the following
\begin{equation}
    \label{eqn:2}
    \Hom_H(\pi, \nu)
\end{equation}
where implicitly we only consider the continuous maps in the archimedean case (for the underlying Fr\'echet topologies).
We denote $m(\pi)$ for the dimension of this space
\[
    m(\pi):= \dim \Hom_H(\pi, \nu).
\]
Note that in the hermitian case we have identifications
\[
    \Hom_H(\pi, \nu) = \Hom_{\rU(W)}(\pi_W \boxtimes \pi_V, \bC) = \Hom_{\rU(W)}(\pi_V, \pi_W^\vee)
\]
where $\pi_W^\vee$ denotes the (smooth) contragredient representation of $\pi_W$.

An element of space \eqref{eqn:2} is called a Bessel functional if $\varepsilon = 1$ and a Fourier-Jacobi functional if $\varepsilon =-1$.
We will then talk in parallel about the Bessel and Fourier-Jacobi cases of the conjecture.


\subsection{Multiplicity $1$}

The following theorem is due to Aizenbud--Gourevitch--Rallis--Schiffmann \cite{aizenbud2010multiplicity} and Sun \cite{sun2012multiplicity} in the $p$-adic case and to Sun--Zhu \cite{sun2012multiplicityarchimedean} in the archimedean case.

\begin{theorem}
For any smooth irreducible representations $\pi$ of $G$ we have
\[m(\pi) \leq 1.\]
\end{theorem}

The local Gan--Gross--Prasad conjecture then essentially provides an answer to the following simple question: when do we have $m(\pi) = 1$? 
Just as for the law of branching between real compact unitary groups discussed in the introduction, any comprehensible answer to this question requires knowing how to parameterize the (isomorphism classes of) irreducible representations of $G$.
Such a parameterization is precisely the object of the local Langlands correspondence (for unitary groups) whose main properties we now recall.


\subsection{Local Langlands correspondence for unitary groups}

In this section we consider a hermitian or skew-hermitian space $V$ of finite dimension $n$ over $E$ and we denote by $\rU(V)$ the corresponding unitary group.

\subsubsection{Weil-Deligne group}

Let $W_F$ be the Weil group of $F$.
If $F$ is non-archimedean, we have the following commutative diagram where each row are exact
\begin{center}
\begin{tikzcd}
    1 \arrow[r] & I_F \arrow[r] \arrow[d, equal] &  \Gal(\overline{F} / F) \arrow[r] & \Gal(\overline{k_F} / k_F) \simeq \widehat{\bZ} \arrow[r] & 1 \\
    1 \arrow[r] & I_F \arrow[r] & W_F \arrow[u] \arrow[r] & \bZ \arrow[r] \arrow[u] & 1
\end{tikzcd}
\end{center}
where $\overline{F}$ is an algebraic closure of $F$, $k_F$ is the residue field of $F$, the isomorphism $\Gal(\overline{k_F} / k_F) \simeq \widehat{\bZ}$ correspond to the choice of the geometric Frobeinus $\Frob_F$ as a topological generator of $\Gal(\overline{k_F} / k_F)$ and $I_F$ is the inertia subgroup (i.e. the kernel of the arrow $\Gal(\overline{F}/ F) \to \Gal(\overline{k_F} / k_F)$).
We then equip $W_F$ with the topology that mesk $I_F$ as an open subgroup (the topology induced from that of $\Gal(\overline{F} / F)$).
If $F$ is archimedean, we have
\[
    W_F = \begin{cases} \bC^\times \cup \bC^\times j & \text{if }F = \bR \\ \bC^\times & \text{if }F = \bC,
\end{cases}
\]
where $j^2 = -1$ and $jzj^{-1} = \bar{z}$ for all $z \in \bC^\times$.
The Weil-Deligne group $\WD_F$ of $F$ is defined by 
\[
    \WD_F = \begin{cases} W_F \times \SL_2(\bC) & \text{if }F\text{ is non-archimedean} \\ W_F & \text{if }F \text{ is archimedean}. \end{cases}
\]


\subsubsection{Langlands parameters}

Langlands associates with $\rU(V)$, and more generally with any connected reductive group over $F$, an \emph{$L$-group} ${}^L \rU(V)$ that is a semi-direct product of a complex reductive group $\widehat{\rU(V)}$ with the Weil group $W_F$: ${}^L \rU(V) = \widehat{\rU(V)} \rtimes W_F$.
Here, the $L$-group is explicitly described as follows: we have $\widehat{\rU(V)} = \GL_n(\bC)$ and the action of $W_F$ factors through $W_F \to W_F / W_E = \Gal(E/F)$ with $\sigma$ acts as $\sigma(g) = J^{t}g^{-1} J^{-1}$, where
\[
J = \begin{pmatrix} & & & 1 \\
& & -1 & \\
& \iddots & & \\
(-1)^{n-1} & & & 
\end{pmatrix}.
\]

A Langlands parameter for $\rU(V)$ is then a $\widehat{\rU(V)}$-conjugacy class of "admissible" homomorphisms (i.e. satisfying certain properties of continuity, semi-simplicity and algebraicity)
\[
    \phi: \WD_F \to {}^L \rU(V)
\]
commuting with projections on $W_F$.
We denote $\Phi(\rU(V))$ the set of Langlands parameters for $\rU(V)$.
For the unitary groups we have the following more explicit description (c.f. \cite{gan2011symplectic} Theorem 8.1): the restriction to $\WD_E$ induces a bijection between $\Phi(\rU(V))$ and the set of isomorphism classes of the complex continuous semi-simple and algebraic representations on $\SL_2(\bC)$ of dimension $n$ of $\WD_E$ which are $(-1)^{n+1}$-conjugate dual.
Let's recall what this last term means.
Fix $c \in W_F \backslash W_E$ maps to $\sigma$.
A representation $\varphi: \WD_E \to \GL(M)$ is called \emph{conjugate dual} if there exists a non-degenerate bilinear form
\[
    B: M \times M \to \bC
\]
satisfying
\[
    B(\varphi(\tau)u, \varphi(c\tau c^{-1})v) = B(u, v), \quad \forall u, v \in M, \tau \in \WD_E.
\]
It is equivalent to ask if $M$ is isomorphic to $(M^c)^\vee$ where $M^c$ is the $c$-conjugate of $M$ and $(-)^\vee$ is the contragredient representation.
We further say that $\varphi: \WD_E \to \GL(M)$ is $\varepsilon$-conjugate-dual, where $\varepsilon \in \{\pm 1\}$, if we can choose a bilinear form satisfying the additional condition
\[
    B(u, \varphi(c^2)v) = \varepsilon B(v, u), \quad \forall u, v \in M.
\]
We will call such a form an $\varepsilon$-conjugate-dual form.

To state the Langlands correspondence in its most complete version, it is necessary to introduce for all $\phi \in \Phi(\rU(V))$ a certain finite group $S_\phi$.
The latter is defined as the group of connected components of the centralizer in $\widehat{\rU(V)}$ of the image of $\phi$.
If we identify $\phi$ with a $(-1)^{n+1}$-conjugate-dual representation $\varphi: \WD_E \to \GL(M)$, we have the following more concrete description of $S_\phi$.
Let $B$ be a conjugate-dual form of sign $(-1)^{n+1}$ as above and denote $\Aut(\varphi, B)$ the group of linear automorphisms of $M$ commutes with the image of $\varphi$ and preserve the form $B$.
We then have (canonically)
\[S_\phi = \Aut(\varphi, B) / \Aut(\varphi, B)^\circ\]
where we denote as $\Aut(\varphi, B)^\circ$ for the connected component of the identity element.
Moreover, this group is always abelian and isomorphic to a product of finitely many copies of $\bZ / 2\bZ$.


\subsubsection{Pure inner forms}

Following an idea from Vogan \cite{vogan1993local}, the Langlands correspondence should be formulated more simply if we consider several groups at the same time.
More precisely, we must take account the pure inner forms of $\rU(V)$.
These forms are naturally parameterized by the Galois cohomology set $\rH^1(F, \rU(V))$ and all admit the same $L$-group as $\rU(V)$ (so that a Langlands parameter for $\rU(V)$ can also be considered as Langlands parameter of all its pure inner forms).
For unitary groups we know how to describe the pure inner forms explicitly: $\rH^1(F, \rU(V))$ naturally classifies the isomorphism classes of (skew-)Hermitian spaces of dimension $n$ and the pure inner forms of $\rU(V)$ are then the unitary groups of the latter spaces.
For a class $\alpha \in \rH^1(F, \rU(V))$, we denote $V_\alpha$ the (skew-)hermitian space it determines and $\rU(V_\alpha)$ the corresponding pure inner form.

In the non-archimedean case, and for $n \neq 0$, there exist exactly two isomorphism classes of (skew-)hermitian spaces of dimension $n$, which can be distinguished by their discriminants, and therefore as many pure inner forms.
In the archimedean case, there are $n+1$ pure interior forms of $\rU(V)$ corresponding to $\rU(p, q)$ for $p + q = n$.
Note that two distinct pure inner forms of $\rU(V)$ can be isomorphic (e.g. $\rU(p, q) \simeq \rU(q,p)$) but from the point of view of the Langlands correspondence these must be considered separately.


\subsubsection{The correspondence}

We can now state the local Langlands correspondence for $\rU(V)$ (and its pure inner forms) in the following informal way.
For all $\alpha \in \rH^1(F, \rU(V))$, there should exist a partition
\[
    \Irr(\rU(V_\alpha)) = \bigsqcup_{\phi \in \Phi(\rU(V))} \Pi^{\rU(V_\alpha)}(\phi) 
\]
into finite (possibly empty) subsets called \emph{$L$-packets} and for all $\phi \in \Phi(\rU(V))$ there should exists a bijection
\begin{equation}
\label{eqn:3}
\begin{aligned}
    \bigsqcup_{\alpha \in \rH^{1}(F, \rU(V))} \Pi^{\rU(V_\alpha)}(\phi) &\simeq \widehat{S_\phi} \\
    \pi(\varphi, \chi) &\mapsfrom \chi
\end{aligned}
\end{equation}
where $\widehat{S_\phi}$ is the group of characters of the finite abelian group $S_\phi$.
This data must of course satisfy a certain number of properties.
In fact, the famous \emph{endoscopic relations}, which we will not explain here, characterize, if it exists, the local Langlands correspondence for the unitary groups from the known correspondence (\cite{harris2001geometry,henniart2000preuve,scholze2013local}), for linear groups.
These endoscopic relations depend however on a certain choice corresponding to the normalization of \emph{transfer factors}.
The composition of the $L$-packets does not depend on this choice but the bijection \eqref{eqn:3} depends on it.
We will give more details about the choices involved in this normalization in section 1.4.7.


\subsubsection{Status}

In the archimedean case, the local correspondence was constructed by Langlands himself \cite{langlands1989irreducible} for all real reductive groups from the results of Harish-Chandra.
This correspondence verifies the expected endoscopic relations follows from  the work of Shelstad \cite{shelstad1982indistinguishability,shelstad2008tempered,shelstad2010tempered} and Mezo \cite{mezo2016tempered} (see also \cite{clozel1982changement} for the case of unitary groups).

In the non-archimedean case, the correspondence was obtained much more recently by Mok \cite{mok2015endoscopic} for quasi-split unitary groups and then by Kaletha--Minguez--Shin--White \cite{kaletha2014endoscopic} for all unitary groups following the founding work of Arthur \cite{arthur2013endoscopic} on symplectic and orthogonal groups.
Until recently these results were still conditional on the stabilization of the twisted trace formula now established in full generality by Waldspurger and Moeglin--Waldspurger in an impressive series of papers \cite{moeglin2016stabilisation}.


\subsubsection{$L$-functions and $\varepsilon$-factors}

For a given Langlands parameter $\phi: \WD_F \to {}^L \rU(V)$ we can associate certain arithmetic invariants with it.
More precisely, for any algebraic representation $\rho: {}^L \rU(V) \to \GL(M)$ where $M$ is a finite dimensional complex vector space, the composition $\rho \circ \phi$ is a representation of the Weil-Deligne group $\WD_F$ to which we can associate a local $L$-function $L(s, \rho\circ \phi) = L(s, \rho, \phi)$ and a local $\varepsilon$ factor $\varepsilon(s, \rho\circ\phi, \psi_0) = \varepsilon(s, \phi, \rho, \psi_0)$ which depends on the additive character $\psi_0: F \to \bC^\times$.
The local $L$-functions are meromorphic functions on $\bC$ without zero while the local epsilon factors are invertible holomorphic functions on $\bC$.
In the case where $F$ is non-archimedean and $\rho \circ \phi$ is trivial on the factor $\SL_2(\bC)$ the $L$-function is defined by
\[
    L(s, \phi, \rho) = \frac{1}{\det(1 - q^{-s} (\rho \circ \phi)(\Frob_F)|_{M^{I_F}})},
\]
where we denote $q$ for the cardinality of the residue field of $F$, $M^{I_F}$ the subspace of $I_F$-invarians and $\Frob_F$ a (geometric) Frobenius in $W_F$.
We have an analogous formula in the general case if $F$ is non-archimedean (cf. \cite{tate1979number} 4.1.6) and if $F$ is archimedean the local $L$ factors are explicit products of gamma functions and powers of $\pi$ and $2$ (cf. \cite{tate1979number} \S 3 ).
Local epsilon factors are much more subtle invariants.
Indeed, these must satisfy a certain number of simple properties characterizing them only but their existence is a difficult theorem due independently to Langlands and Deligne (\cite{deligne1973constantes}).

Let us mention here a property of these factors that we will need.
Let $\varphi: \WD_E \to \GL(M)$ be a $(-1)$-conjugate-dual representation of the Weil-Deligne group of $E$.
Then, $\varepsilon(\frac{1}{2}, \varphi, \psi) \in \{ \pm 1\}$ where we recall that the character $\psi: E \to \bS^1$ is trivial on $F$.
Moreover, this epsilon factor depends only on the $\rN(E^\times)$-orbit of $\psi$ and in fact does not depend on $\psi$ at all if $\dim(\varphi)$ is even.


\subsubsection{Whittaker datum and normalization of the correspondence}

As explained in 1.4.4, the bijection \eqref{eqn:3} depends on a choice allowing to normalize certain transfer factors.
According to \cite{kottwitz1999foundations}, such a choice can be made by fixing a Whittaker datum of a pure inner form of $\rU(V)$.
More precisely, we first choose a quasi-split pure inner form $\rU(V_\alpha)$ of $\rU(V)$ having a Borel subgroup $B \subset \rU(V_\alpha)$ defined on $F$.
Such a group exists and even it means that by replacing $V$ by $V_\alpha$ (which does not modify the family of pure inner forms), we can assume that we have chosen $\rU(V)$ (which we therefore assume quasi-split).
A Whittaker data on $\rU(V)$ is then a conjugacy class of pairs $(N, \theta)$ where $N$ is the unipotent radical of a Borel subgroup $B = TN$ defined over $F$ and $\theta: N \to \bS^1$ is a \emph{generic} character whose stabilizer in $T$ equals to the center of $\rU(V)$.
There is only one conjugacy class of Whittaker data if $n = \dim(V)$ is odd while if $n$ is even there are two and one can be fixed from the
character $\psi: E /F \to \bS^1$ in hermitian case and $\psi_0: F \to \bS^1$ in the skew-hermitian case.


\subsubsection{Generic, tempered, and discrete $L$-packets}

A Langlands parameter $\phi: \WD_F \to {}^L \rU(V)$ is said to be generic if $L(s, \phi, \Ad)$ has no pole at $s = 1$ where $\Ad$ denotes the adjoint representation of ${}^L \rU(V)$ on its Lie algebra.
The corresponding L-packet $\Pi^{\rU(V)}(\phi)$ then contains one and only one representation $\pi$ admitting a Whittaker model for $(N, \theta)$ i.e. $\Hom_{N}(\pi, \theta) \neq 0$ (we then say that $\pi$ is \emph{generic} with respect to $(N, \theta)$) and moreover this representation corresponds via the bijection \eqref{eqn:3} to the trivial character of $S_\phi$.

A Langlands parameter $\phi: \WD_F \to {}^L \rU(V)$ is \emph{tempered} if the projection of the image of $W_F$ onto $\widehat{\rU(V)}$ is relatively compact.
A tempered parameter is automatically generic and the corresponding $L$-packet $\Pi^{\rU(V)}(\phi)$ only contains \emph{tempered} representations, i.e. representations which weakly contained in $L^2(\rU(V))$ (there is also a characterization of tempered representations by a condition of growth of coefficients).
In fact, one can reconstruct the Langlands correspondence for $\rU(V)$ from the correspondence restricted to the tempered parameters of $\rU(V)$ and its Levi subgroups.
This follows from the Langlands classification which makes it possible to obtain all the irreducible representations of a reductive group from the tempered representations of its Levi subgroups by a classical process called parabolic induction.

Finally, a Langlands parameter $\phi: \WD_F \to {}^L \rU(V)$ is said to be \emph{discrete} if the centralizer of its image in $\widehat{\rU(V)}$ is finite.
A discrete parameter is automatically tempered (therefore also generic) and determines an $L$-packet of representations of the discrete series which appear as submodules of $L^2(\rU(V))$.
