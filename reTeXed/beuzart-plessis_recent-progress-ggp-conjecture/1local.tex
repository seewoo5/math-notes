\section{The Local Conjectures}

\subsection{The groups}

Let $E/F$ be a quadratic extension of local fields of characteristic zero.
We therefore have either $E/F = \bC / \bR$ or that $E$ and $F$ are finite extensions of the field of $p$-adic numbers $\bQ_p$ for a certain prime number $p$ ($\bQ_p$ is the completion of $\bQ$ by the $p$-adic absolute value $|\cdot|_p$ defined by $|p^k \frac{a}{b}|
_p = p^{-k}$ for $a$ and $b$ integers prime to $p$).
We denote by $\sigma$  the unique non-trivial element of the Galois
group $\Gal(E/F)$ and $\sgn_{E/F}$ the quadratic character of $F$ associated with the extension $E/F$ by the class field theory (it is therefore the unique quadratic character with kernel $\rN_{E/F}(E^\times)$, the image of the norm map).
Finally, we will fix two non-trivial additive characters $\psi_0: F \to \bS^1$ and $\psi: E \to \bS^1$ with the property that $\psi$ is trivial on $F$.

Let $V$ be a finite dimensional vector space of dimension $n$ over $E$ and $\varepsilon \in \{ \pm 1 \}$.
We assume $V$ is equipped with a non-degenerate $\varepsilon$-hermitian form
\[
    \langle -, -\rangle: V \times V \to E.
\]
By definition a $\varepsilon$-hermitian form satisfies
\begin{align*}
    \langle \lambda v + \mu w, u \rangle &= \lambda \langle v, u \rangle + \mu \langle w, u \rangle \\
    \langle v, u \rangle &= \varepsilon \langle u, v \rangle^\sigma
\end{align*}
for all $u, v, w \in V$ and $\lambda, \mu \in E$.
Depending on whether $\varepsilon = 1$ or $-1$ we call it hermitian or skew-hermitian.
Let $W$ be a non-detenerate subspace of $V$ with
\[
    \dim(V) - \dim(W) = \begin{cases} 1 &\text{if }\varepsilon = 1 \\ 0 & \text{if } \varepsilon = -1. \end{cases}
\]
Let $\rU(V) \subset \GL(V)$ and $\rU(W) \subset \GL(W)$ be the algebraic subgroups (defined over $F$) of linear automorphisms of $V$ and $W$ preserving the form $\langle -, - \rangle$.
Then $\rU(V)$ and $\rU(W)$ are unitary groups and we have a natural embeddign $\rU(W) \hookrightarrow \rU(V)$ where $\rU(W)$ acts trivially on $W^\perp$ (which of dimension at most 1).
In the following we will (abusively) identify an algebraic group defined on $F$ with the group of $F$-points corresponding to it.

The following discussion also extends to the case where $E = F \times F$ equipped with the involution $\sigma(x, y) = (y, x)$, a case which it will be necessary to include anyway when we will deal with the global conjecture.
In such a situation, a non-degenerate form $\langle -, -\rangle$ as above identifies $V$ and $W$ to direct sums $V_0 \oplus V_0^\vee$ and $W_0 \oplus W_0^\vee$ where $W_0 \subset V_0$ are the finite dimensional vector spaces over $F$ and $V_0^\vee$, $W_0^\vee$ denote their duals.
We then have a natural identifications $\rU(V) \simeq \GL(V_0)$ and $\rU(W) \simeq \GL(W_0)$.

In all cases, we put $G = \rU(W) \times \rU(V)$, $H = \rU(W)$ and we embed $H$ into $G$ diagonally.
The groups $H$ and $G$ inherit from the field $F$ topologies which make them Lie groups in the archimedean case (i.e. when $F = \bR$) and locally profinite groups in the non-archimedean case (i.e. when $F$ is a finite extension of $\bQ_p$; recall that a topological group is locally profinite if it has a basis of neighborhoods of the identity element consist of compact subgroups).


\subsection{The restriction problem}

Let $(\pi, \scV)$ be a smooth and irreducible complex representation of $G$.
In the $p$-adic case, this means that $\pi$ is a representation of $G$ on a $\bC$-vector space $\scV$ (typically of infinite dimension) all of whose vectors have a open stabilizer, irreducibility is then
an algebraic notion (i.e. no non-trivial subspace stable under $G$).
In the archimedean case, this means that $\scV$ is a Fréchet space and that $\pi$ is a smooth representation (in the $C^\infty$ sense), admissible (i.e. the irreducible representations of a maximal compact subgroup appear with finite multiplicities) on $\scV$ satisfying a certain condition of “moderate growth” (which was introduced by Casselman and Wallach, see \cite{casselman1989canonical} and \cite{wallachreal} Chap. 11); irreducibility is then a topological notion (ie no non-trivial closed subspace stable by $G$).
In any case, such an irreducible representation decomposes as a tensor product $\pi = \pi_W \boxtimes \pi_V$ where $\pi_W$ and $\pi_V$ are irreducible (smooth) representations of $\rU(W)$ and $\rU(V)$ respectively (and where the tensor product is a topological tensor product in the archimedean case).
We will denote as $\Irr(G)$ for the set of isomorphism classes of smooth irreducible representations of $G$.

To define the restriction problem that will interest us, we must also introduce a certain “small” representation $\nu$ of $H$.
In the hermitian case (i.e. if $\varepsilon = 1$), $\nu$ is the trivial representation that we will denote as $1$ or simply $\bC$ in the following.
In the skew-hermitian case (i.e. if $\varepsilon = -1
$), we have an inclusion
\[
    \rU(W) \subset \Sp(\Res_{E/F}W)
\]
where $\Res_{E/F}W$ denotes the restriction of the scalars from $E$ to $F$ of $W$ equipped with the symplectic form $\Tr_{E/F} \circ \langle -, -\rangle$ and $\Sp(\Res_{E/F}W)$ denotes the corresponding symplectic group.
Let $\Mp(\Res_{E/F}W)$ be the metaplectic group associated with this symplectic space (it is a $\bZ/2\bZ$-extension of $\Sp(\Res_{E/F}W)$).
The metaplectic covering splits over $\rU(W)$ but this splitting is not unique (because there are non-trivial characters $\rU(W) \to \{\pm 1\}$).
We can, however, fix such a splitting by choosing a character $\mu: E^\times \to \bS^1$ with $\mu|_{F^\times} = \sgn_{E/F}$ from now on.
Let $\omega_{\psi_0, W}$ be the Weil representation of $\Mp(\Res_{E/F} W)$ associated to the character $\psi_0$ (c.f. \cite{moeglin2006correspondances} Chap. 2. II).
Then $\nu = \omega_{\psi_0, W, \mu}$ is the restriction of this Weil representation to $\rU(W)$ via the splitting that we have just fixed.

For every case, the space of intertwining maps which is of our interest is the following
\begin{equation}
    \label{eqn:2}
    \Hom_H(\pi, \nu)
\end{equation}
where implicitly we only consider the continuous maps in the archimedean case (for the underlying Fr\'echet topologies).
We denote $m(\pi)$ for the dimension of this space
\[
    m(\pi):= \dim \Hom_H(\pi, \nu).
\]
Note that in the hermitian case we have identifications
\[
    \Hom_H(\pi, \nu) = \Hom_{\rU(W)}(\pi_W \boxtimes \pi_V, \bC) = \Hom_{\rU(W)}(\pi_V, \pi_W^\vee)
\]
where $\pi_W^\vee$ denotes the (smooth) contragredient representation of $\pi_W$.

An element of space \eqref{eqn:2} is called a Bessel functional if $\varepsilon = 1$ and a Fourier-Jacobi functional if $\varepsilon =-1$.
We will then talk in parallel about the Bessel and Fourier-Jacobi cases of the conjecture.


\subsection{Multiplicity $1$}

The following theorem is due to Aizenbud--Gourevitch--Rallis--Schiffmann \cite{aizenbud2010multiplicity} and Sun \cite{sun2012multiplicity} in the $p$-adic case and to Sun--Zhu \cite{sun2012multiplicityarchimedean} in the archimedean case.

\begin{theorem}
For any smooth irreducible representations $\pi$ of $G$ we have
\[m(\pi) \leq 1.\]
\end{theorem}

The local Gan--Gross--Prasad conjecture then essentially provides an answer to the following simple question: when do we have $m(\pi) = 1$? 
Just as for the law of branching between real compact unitary groups discussed in the introduction, any comprehensible answer to this question requires knowing how to parameterize the (isomorphism classes of) irreducible representations of $G$.
Such a parameterization is precisely the object of the local Langlands correspondence (for unitary groups) whose main properties we now recall.


\subsection{Local Langlands correspondence for unitary groups}

In this section we consider a hermitian or skew-hermitian space $V$ of finite dimension $n$ over $E$ and we denote by $\rU(V)$ the corresponding unitary group.

\subsubsection{Weil-Deligne group}

Let $W_F$ be the Weil group of $F$.
If $F$ is non-archimedean, we have the following commutative diagram where each row are exact
\begin{center}
\begin{tikzcd}
    1 \arrow[r] & I_F \arrow[r] \arrow[d, equal] &  \Gal(\overline{F} / F) \arrow[r] & \Gal(\overline{k_F} / k_F) \simeq \widehat{\bZ} \arrow[r] & 1 \\
    1 \arrow[r] & I_F \arrow[r] & W_F \arrow[u] \arrow[r] & \bZ \arrow[r] \arrow[u] & 1
\end{tikzcd}
\end{center}
where $\overline{F}$ is an algebraic closure of $F$, $k_F$ is the residue field of $F$, the isomorphism $\Gal(\overline{k_F} / k_F) \simeq \widehat{\bZ}$ correspond to the choice of the geometric Frobeinus $\Frob_F$ as a topological generator of $\Gal(\overline{k_F} / k_F)$ and $I_F$ is the inertia subgroup (i.e. the kernel of the arrow $\Gal(\overline{F}/ F) \to \Gal(\overline{k_F} / k_F)$).
We then equip $W_F$ with the topology that mesk $I_F$ as an open subgroup (the topology induced from that of $\Gal(\overline{F} / F)$).
If $F$ is archimedean, we have
\[
    W_F = \begin{cases} \bC^\times \cup \bC^\times j & \text{if }F = \bR \\ \bC^\times & \text{if }F = \bC,
\end{cases}
\]
where $j^2 = -1$ and $jzj^{-1} = \bar{z}$ for all $z \in \bC^\times$.
The Weil-Deligne group $\WD_F$ of $F$ is defined by 
\[
    \WD_F = \begin{cases} W_F \times \SL_2(\bC) & \text{if }F\text{ is non-archimedean} \\ W_F & \text{if }F \text{ is archimedean}. \end{cases}
\]


\subsubsection{Langlands parameters}

Langlands associates with $\rU(V)$, and more generally with any connected reductive group over $F$, an \emph{$L$-group} ${}^L \rU(V)$ that is a semi-direct product of a complex reductive group $\widehat{\rU(V)}$ with the Weil group $W_F$: ${}^L \rU(V) = \widehat{\rU(V)} \rtimes W_F$.
Here, the $L$-group is explicitly described as follows: we have $\widehat{\rU(V)} = \GL_n(\bC)$ and the action of $W_F$ factors through $W_F \to W_F / W_E = \Gal(E/F)$ with $\sigma$ acts as $\sigma(g) = J^{t}g^{-1} J^{-1}$, where
\[
J = \begin{pmatrix} & & & 1 \\
& & -1 & \\
& \iddots & & \\
(-1)^{n-1} & & & 
\end{pmatrix}.
\]

A Langlands parameter for $\rU(V)$ is then a $\widehat{\rU(V)}$-conjugacy class of "admissible" homomorphisms (i.e. satisfying certain properties of continuity, semi-simplicity and algebraicity)
\[
    \phi: \WD_F \to {}^L \rU(V)
\]
commuting with projections on $W_F$.
We denote $\Phi(\rU(V))$ the set of Langlands parameters for $\rU(V)$.
For the unitary groups we have the following more explicit description (c.f. \cite{gan2011symplectic} Theorem 8.1): the restriction to $\WD_E$ induces a bijection between $\Phi(\rU(V))$ and the set of isomorphism classes of the complex continuous semi-simple and algebraic representations on $\SL_2(\bC)$ of dimension $n$ of $\WD_E$ which are $(-1)^{n+1}$-conjugate dual.
Let's recall what this last term means.
Fix $c \in W_F \backslash W_E$ maps to $\sigma$.
A representation $\varphi: \WD_E \to \GL(M)$ is called \emph{conjugate dual} if there exists a non-degenerate bilinear form
\[
    B: M \times M \to \bC
\]
satisfying
\[
    B(\varphi(\tau)u, \varphi(c\tau c^{-1})v) = B(u, v), \quad \forall u, v \in M, \tau \in \WD_E.
\]
It is equivalent to ask if $M$ is isomorphic to $(M^c)^\vee$ where $M^c$ is the $c$-conjugate of $M$ and $(-)^\vee$ is the contragredient representation.
We further say that $\varphi: \WD_E \to \GL(M)$ is $\varepsilon$-conjugate-dual, where $\varepsilon \in \{\pm 1\}$, if we can choose a bilinear form satisfying the additional condition
\[
    B(u, \varphi(c^2)v) = \varepsilon B(v, u), \quad \forall u, v \in M.
\]
We will call such a form an $\varepsilon$-conjugate-dual form.

To state the Langlands correspondence in its most complete version, it is necessary to introduce for all $\phi \in \Phi(\rU(V))$ a certain finite group $S_\phi$.
The latter is defined as the group of connected components of the centralizer in $\widehat{\rU(V)}$ of the image of $\phi$.
If we identify $\phi$ with a $(-1)^{n+1}$-conjugate-dual representation $\varphi: \WD_E \to \GL(M)$, we have the following more concrete description of $S_\phi$.
Let $B$ be a conjugate-dual form of sign $(-1)^{n+1}$ as above and denote $\Aut(\varphi, B)$ the group of linear automorphisms of $M$ commutes with the image of $\varphi$ and preserve the form $B$.
We then have (canonically)
\[S_\phi = \Aut(\varphi, B) / \Aut(\varphi, B)^\circ\]
where we denote as $\Aut(\varphi, B)^\circ$ for the connected component of the identity element.
Moreover, this group is always abelian and isomorphic to a product of finitely many copies of $\bZ / 2\bZ$.