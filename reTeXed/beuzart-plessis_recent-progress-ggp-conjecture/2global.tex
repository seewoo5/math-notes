\section{The Global Conjectures}


Through this section we fix a quadratic extension $k'/k$ of number fields.
We denote the set of places of $k$ as $|k|$ and for all $v \in |k|$ the corresponding completion as $k_v$.
We also have $k_v' = k' \times_k k_v$.
Thus $k_v' \simeq k_v \times k_v$ if $v$ splits in $k'$ and $k_V'$ is a unique quadratic extension of $k_v$ otherwise.
Let $\bA$ the ring of ad\`eles of $k$.
Recall that it is a restricted product of $k_v$ over all places $v \in |k|$ which is the set of families $(x_v)_{v\in |k|}$ with $x_v \in k_v$ for all $v$ and $x_v \in \scO_v$ for almost all non-archimedean $v$ where $\scO_v$ is the ring of integers of $k_v$.
We have a diagonal embedding $k \hookrightarrow \bA$ and we will also denote $\bA_f$ the ring of finite adeles (i.e. the restricted product of non-archimedean completions of $k$) and $k_\infty = k \otimes_\bQ \bR \simeq \prod_{v|\infty}k_v$ the archimedean part of $\bA$.
We denote by $\sgn_{k'/k}$ the quadratic character of the id\`ele class group $\bA^\times / k^\times$ associated with, by the class field theory, the extension $k'/k$.
For all $v \in |k|$, the restriction of $\sgn_{k'/k}$ to $k_v^\times$ coincides with $\sgn_{k_v'/k_v}$ (and it is trivial if $v$ splits in $k'$).
Let $W \subset V$ be two hermitian or skew-hermitian spaces over $k'$ with
\[
    \dim(V) - \dim(W) = \begin{cases} 1 & \text{in the hermitian case} \\ 0 & \text{in the skew-herimitian case},
    \end{cases}
\]
and set $G = \rU(W) \times \rU(V)$, $H = \rU(W)$ (algebraic groups defined over $k$).
As in the local case we have a ``diagonal'' inclusion $H \hookrightarrow G$ and the group of adelic points $G(\bA)$ is a locally compact group admitting the following more explicit description.
Fix a model of $G$ on $\scO_k[1/N]$ for some integer $N \geq 1$ where $\scO_k$ denotes the ring of algebraic integers of $k$.
Then for almost all place $v \in |k|$, the group of points $G(\scO_v)$ of this model over $\scO_v$ is a maximal compact subgroup of $G(k_v)$ and $G(\bA)$ is the restricted product of $G(k_V)$, $v \in |k|$ with respect to $G(\scO_v)$ which is the set of families $(g_v)_{v \in |k|}$ with $g_v \in G(k_v)$ for all $v \in |k|$ and $g_v \in G(\scO_v)$ for almost all $v \in |k|$.
A similar description obviously applies to $H(\bA)$.


\subsection{Automorphic forms and periods}

Recall that the automorphic forms on $G(\bA)$ is a function $\varphi: G(\bA)\to \bC$ satisfying the following conditions:
\begin{itemize}
    \item[--] $\varphi$ is left invariant by $G(k)$.
    \item[--] $\varphi$ is right invariant by an open compact subgroup $K_f \subset G(\bA_f)$.
    \item[--] For all $g \in G(\bA)$ the function $g_\infty \in G(k_\infty) \mapsto \varphi(gg_\infty)$ is $C^\infty$, in particular we have an action of Lie algebra $\frg_\infty$ of $G(k_\infty)$ on $\varphi$ by $(X.\varphi)(g) = \frac{\dd}{\dd t}\varphi(g e^{tX})|_{t=0}$ that extends to the complexified universal enveloping algebra $\scU(\frg_\infty)$.
    \item[--] $\varphi$ satisfies a certain \emph{moderate growth} at infinity (cf. \cite{moeglin1994decomposition} \S 1.2.3).
\end{itemize}

We denote by $\scA(G)$ the space of automorphic forms over $G(\bA)$.
Let us point out that the definition above differs from the one usually admitted which imposes an additional condition of $K_\infty$-finitness where $K_\infty$ is a maximal compact subgroup of $G(k_\infty)$ fixed in advance.
This definition has however the advantage of providing a stable space $\scA(G)$ under the right translation action of $G(\bA)$ (whereas with the usual definition one only obtains a structure of $(\frg_\infty, K_\infty)$-module at archimedean places) and seems more natural for the questions we are going to discuss.
We denote $\scA_\cusp(G) \subset \scA(G)$ the subspace of cusp forms, i.e. automorphic forms which are rapidly decreasing as well as all their derivatives (in a sense which we will not make precise here).
This space is stable under the action by the right translation of $G(\bA)$ and admits a decomposition
\[
    \scA_\cusp(G) = \bigoplus_\pi \pi
\]
as a direct sum of irreducible representations of $G(\bA)$.
Each of these irreducible representations decomposes as a restricted tensor product $\pi = \otimes_{v \in |k|}' \pi_v$.
More precisely, there exists a family of smooth irreducible representations $(\pi_v)_{v \in |k|}$ of local groups $G(k_v)$ that are unramified at almost all places $v \in |k|$ (which means $\pi_v^{G(\scO_v)} \neq 0$ and it is one-dimensional by the \emph{Satake isomorphis}) and the nonzero vectors $\varphi_v^\circ \in \pi_v^{G(\scO_v)}$ such that $\pi$ is isomorphic to the natural representation of $G(\bA)$ on
\[
    \lim_{\substack{\longrightarrow \\ S}} \bigotimes_{v \in S}\pi_v
\]
where the limit is over the sufficiently ``large'' finite sets of places of $k$ (i.e. containing the archimedean places and the ramified finite places) and the transition maps $\bigotimes_{v \in S} \pi_v \to \bigotimes_{v \in T} \pi_v$ for $S \subset T$ are defined by $\varphi_S \mapsto \varphi_S \otimes \bigotimes_{v \in T\backslash S} \varphi_v^\circ$ (to be precise, one needs to consider topological tensor products at archimedean places).

The constructions of section 1.2 provide for every $v$ a representation $\nu_v$ of the local group $H(k_v)$ (the trivial representation in the hermitian case and a certain Weil representation in the skew-hermitian case).
We can form their restricted tensor product $\nu = \otimes_v' \nu_v$ and it turns out that there exists a natural realization $\nu \hookrightarrow \scA(H)$ in the space of automorphic forms on $H$ (in the hermitian case the trivial representation is realized
as the space of constant functions on $H(k) \backslash H(\bA)$ while in the skew-hermitian case $\nu$ is a certain global Weil representation and the embedding $\nu \hookrightarrow \scA(H)$ is obtained via the
theta series).
In particular, in the skew-hermitian case we must choose local data $(\psi_{0, v}, \mu_v)$ for all $v \in |k|$ as in 1.2 in order to specify the representations $\nu_v$.
To obtain the embedding $\nu \hookrightarrow \scA(H)$ we must assume  that these local data come by localization of global characters $\psi_0: \bA/k \to \bC^\times$ and $\mu: \bA^\times_{k'} / (k')^\times \to \bC^\times$.

We define an \emph{automorphic period}
\[
    \cP_H: \scA_\cusp(G) \otimes \overline{\nu} \to \bC,
\]
where $\overline{\nu}$ denotes the complex conjugation of the realization of $\nu$ in $\scA(H)$, by
\[
    \cP_H(\varphi \otimes \overline{\theta}) := \int_{H(k) \backslash H(\bA)} \varphi(h) \overline{\theta(h)} \dd h
\]
for all $\varphi \in \scA_\cusp(G)$ and $\overline{\theta} \in \overline{\nu}$.
The integral is absolutely convergent due to the rapid decrease of $\varphi$ and the measure $\dd h$ for which we are integrating is a $H(\bA)$-invariant measure which can be obtained as the quotient of a Haar measure on $H(\bA)$ because $H(k)$ is a discrete subgroup.
For the global Gan--Gross--Prasad conjecture the specific choice of this Haar measure does not matter because we are only interested in questions of non-vanishing of the period.
On the other hand, this choice matters for the Ichino--Ikeda conjecture which predicts an explicit formula for (the square of) the period above.
Note that in the hermitian case, the representation $\nu$ being trivial, the period $\cP_H$ is simply the linear form $\scA_\cusp(G) \to \bC$ given by
\[
    \cP_H(\varphi) = \int_{H(k) \backslash H(\bA)} \varphi(h) \dd h.
\]
In any case, if $\pi \subset \scA_\cusp(G)$ is a cuspidal irreducible representation, the restriction of the preiod $\cP_H$ to $\pi$ defines an element of the space of intertwining maps
\[
    \Hom_{H(\bA)}(\pi \otimes \overline{\nu}, \bC) = \Hom_{H(\bA)}(\pi, \nu)
\]
which decomposes into a (restricted) tensor product of the local spaces of intertwining maps $\Hom_{H_v}(\pi_v, \nu_v)$ for all $v \in |k|$.
Thus, a necessary condition for this restriction to be nonzero is that these local spaces is nontrivial (a condition which is itself made explicit by the local conjecture).

