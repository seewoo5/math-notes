% --- LaTeX Lecture Notes Template - S. Venkatraman ---

% --- Set document class and font size ---

\documentclass[letterpaper, 12pt]{article}

% --- Package imports ---

% Extended set of colors
\usepackage[dvipsnames]{xcolor}

\usepackage{
  amsmath, amsthm, amssymb, mathtools, dsfont, units,          % Math typesetting
  graphicx, wrapfig, subfig, float,                            % Figures and graphics formatting
  listings, color, inconsolata, pythonhighlight,               % Code formatting
  fancyhdr, sectsty, hyperref, enumerate, enumitem, framed }   % Headers/footers, section fonts, links, lists

% lipsum is just for generating placeholder text and can be removed
\usepackage{hyperref, lipsum} 

% --- Fonts ---

\usepackage{newpxtext, newpxmath, inconsolata}
\usepackage{amsfonts}

\usepackage{tikz}
\usepackage{tikz-cd}
\usepackage{enumitem}
\usepackage[title]{appendix}
\usepackage{mathdots}
\usepackage{stmaryrd}

% --- Page layout settings ---

% Set page margins
\usepackage[left=1.35in, right=1.35in, top=1.0in, bottom=.9in, headsep=.2in, footskip=0.35in]{geometry}

% Anchor footnotes to the bottom of the page
\usepackage[bottom]{footmisc}

% Set line spacing
\renewcommand{\baselinestretch}{1.2}

% Set spacing between paragraphs
\setlength{\parskip}{1.3mm}

% Allow multi-line equations to break onto the next page
\allowdisplaybreaks

% --- Page formatting settings ---

% Set image captions to be italicized
\usepackage[font={it,footnotesize}]{caption}

% Set link colors for labeled items (blue), citations (red), URLs (orange)
\hypersetup{colorlinks=true, linkcolor=RoyalBlue, citecolor=RedOrange, urlcolor=ForestGreen}

% Set font size for section titles (\large) and subtitles (\normalsize) 
\usepackage{titlesec}
% \titleformat{\section}{\large\bfseries}{{\fontsize{19}{19}\selectfont\textreferencemark}\;\; }{0em}{}
\titleformat{\section}{\large\bfseries}{\thesection\;\;\;}{0em}{}
\titleformat{\subsection}{\normalsize\bfseries\selectfont}{\thesubsection\;\;\;}{0em}{}

% Enumerated/bulleted lists: make numbers/bullets flush left
%\setlist[enumerate]{wide=2pt, leftmargin=16pt, labelwidth=0pt}
\setlist[itemize]{wide=0pt, leftmargin=16pt, labelwidth=10pt, align=left}

% --- Table of contents settings ---

\usepackage[subfigure]{tocloft}

% Reduce spacing between sections in table of contents
\setlength{\cftbeforesecskip}{.9ex}

% Remove indentation for sections
\cftsetindents{section}{0em}{0em}

% Set font size (\large) for table of contents title
\renewcommand{\cfttoctitlefont}{\large\bfseries}

% Remove numbers/bullets from section titles in table of contents
\makeatletter
\renewcommand{\cftsecpresnum}{\begin{lrbox}{\@tempboxa}}
\renewcommand{\cftsecaftersnum}{\end{lrbox}}
\makeatother

% --- Set path for images ---

\graphicspath{{Images/}{../Images/}}

% --- Math/Statistics commands ---

% Add a reference number to a single line of a multi-line equation
% Usage: "\numberthis\label{labelNameHere}" in an align or gather environment
\newcommand\numberthis{\addtocounter{equation}{1}\tag{\theequation}}

% Shortcut for bold text in math mode, e.g. $\b{X}$
\let\b\mathbf

% Shortcut for bold Greek letters, e.g. $\bg{\beta}$
\let\bg\boldsymbol

% Shortcut for calligraphic script, e.g. %\mc{M}$
\let\mc\mathcal

% \mathscr{(letter here)} is sometimes used to denote vector spaces
\usepackage[mathscr]{euscript}

% Convergence: right arrow with optional text on top
% E.g. $\converge[p]$ for converges in probability
\newcommand{\converge}[1][]{\xrightarrow{#1}}

% Weak convergence: harpoon symbol with optional text on top
% E.g. $\wconverge[n\to\infty]$
\newcommand{\wconverge}[1][]{\stackrel{#1}{\rightharpoonup}}

% Equality: equals sign with optional text on top
% E.g. $X \equals[d] Y$ for equality in distribution
\newcommand{\equals}[1][]{\stackrel{\smash{#1}}{=}}

% Normal distribution: arguments are the mean and variance
% E.g. $\normal{\mu}{\sigma}$
\newcommand{\normal}[2]{\mathcal{N}\left(#1,#2\right)}

% Uniform distribution: arguments are the left and right endpoints
% E.g. $\unif{0}{1}$
\newcommand{\unif}[2]{\text{Uniform}(#1,#2)}

% Independent and identically distributed random variables
% E.g. $ X_1,...,X_n \iid \normal{0}{1}$
\newcommand{\iid}{\stackrel{\smash{\text{iid}}}{\sim}}

% Sequences (this shortcut is mostly to reduce finger strain for small hands)
% E.g. to write $\{A_n\}_{n\geq 1}$, do $\bk{A_n}{n\geq 1}$
\newcommand{\bk}[2]{\{#1\}_{#2}}

% \setcounter{section}{-1}

\newcommand{\SL}{\mathrm{SL}}
\newcommand{\Sp}{\mathrm{Sp}}
\newcommand{\Mp}{\mathrm{Mp}}
\newcommand{\GL}{\mathrm{GL}}
\newcommand{\SO}{\mathrm{SO}}
\newcommand{\SU}{\mathrm{SU}}
\newcommand{\PGL}{\mathrm{PGL}}
\newcommand{\PSL}{\mathrm{PSL}}
\newcommand{\rM}{\mathrm{M}}
\newcommand{\rN}{\mathrm{N}}
\newcommand{\rO}{\mathrm{O}}
\newcommand{\rP}{\mathrm{P}}
\newcommand{\rH}{\mathrm{H}}
\newcommand{\rU}{\mathrm{U}}
\newcommand{\JL}{\mathrm{JL}}
\newcommand{\stab}{\mathrm{Stab}}
\newcommand{\cusp}{\mathrm{cusp}}
\newcommand{\reg}{\mathrm{reg}}
\newcommand{\rs}{\mathrm{rs}}
\newcommand{\Irr}{\mathrm{Irr}}
\newcommand{\Tr}{\mathrm{Tr}}
\newcommand{\Hom}{\mathrm{Hom}}
\newcommand{\Gal}{\mathrm{Gal}}
\newcommand{\WD}{\mathrm{WD}}
\newcommand{\Frob}{\mathrm{Frob}}
\newcommand{\Res}{\mathrm{Res}}
\newcommand{\Tam}{\mathrm{Tam}}
\newcommand{\Pet}{\mathrm{Pet}}
\newcommand{\sgn}{\mathrm{sgn}}
\newcommand{\vol}{\mathrm{vol}}
\newcommand{\Aut}{\mathrm{Aut}}
\newcommand{\Ind}{\mathrm{Ind}}
\newcommand{\BC}{\mathrm{BC}}
\newcommand{\Ad}{\mathrm{Ad}}

\newcommand{\what}{\widehat}

\newcommand{\dd}{\mathrm{d}}

\newcommand{\bA}{\mathbb{A}}
\newcommand{\bR}{\mathbb{R}}
\newcommand{\bS}{\mathbb{S}}
\newcommand{\bZ}{\mathbb{Z}}
\newcommand{\bC}{\mathbb{C}}
\newcommand{\bQ}{\mathbb{Q}}
\newcommand{\bH}{\mathbb{H}}
\newcommand{\bfi}{\mathbf{I}}
\newcommand{\bfa}{\mathbf{a}}
\newcommand{\bfb}{\mathbf{b}}
\newcommand{\cS}{\mathcal{S}}
\newcommand{\cO}{\mathcal{O}}
\newcommand{\cV}{\mathcal{V}}
\newcommand{\cP}{\mathcal{P}}

\newcommand{\scA}{\mathscr{A}}
\newcommand{\scB}{\mathscr{B}}
\newcommand{\scV}{\mathscr{V}}
\newcommand{\scT}{\mathscr{T}}
\newcommand{\scU}{\mathscr{U}}
\newcommand{\scW}{\mathscr{W}}
\newcommand{\scO}{\mathscr{O}}
\newcommand{\scL}{\mathscr{L}}

\newcommand{\frh}{\mathfrak{h}}
\newcommand{\frt}{\mathfrak{t}}
\newcommand{\frg}{\mathfrak{g}}
\newcommand{\frgl}{\mathfrak{gl}}
\newcommand{\fru}{\mathfrak{u}}

% Math mode symbols for common sets and spaces. Example usage: $\R$
\newcommand{\R}{\mathbb{R}}	% Real numbers
\newcommand{\C}{\mathbb{C}}	% Complex numbers
\newcommand{\Q}{\mathbb{Q}}	% Rational numbers
\newcommand{\Z}{\mathbb{Z}}	% Integers
\newcommand{\N}{\mathbb{N}}	% Natural numbers
\newcommand{\F}{\mathcal{F}}	% Calligraphic F for a sigma algebra
\newcommand{\El}{\mathcal{L}}	% Calligraphic L, e.g. for L^p spaces

% Math mode symbols for probability
\newcommand{\pr}{\mathbb{P}}	% Probability measure
\newcommand{\E}{\mathbb{E}}	% Expectation, e.g. $\E(X)$
\newcommand{\var}{\text{Var}}	% Variance, e.g. $\var(X)$
\newcommand{\cov}{\text{Cov}}	% Covariance, e.g. $\cov(X,Y)$
\newcommand{\corr}{\text{Corr}}	% Correlation, e.g. $\corr(X,Y)$
\newcommand{\B}{\mathcal{B}}	% Borel sigma-algebra

% Other miscellaneous symbols
\newcommand{\tth}{\text{th}}	% Non-italicized 'th', e.g. $n^\tth$
\newcommand{\Oh}{\mathcal{O}}	% Big-O notation, e.g. $\O(n)$
\newcommand{\1}{\mathds{1}}	% Indicator function, e.g. $\1_A$

% Additional commands for math mode
\DeclareMathOperator*{\argmax}{argmax}		% Argmax, e.g. $\argmax_{x\in[0,1]} f(x)$
\DeclareMathOperator*{\argmin}{argmin}		% Argmin, e.g. $\argmin_{x\in[0,1]} f(x)$
\DeclareMathOperator*{\spann}{Span}		% Span, e.g. $\spann\{X_1,...,X_n\}$
\DeclareMathOperator*{\bias}{Bias}		% Bias, e.g. $\bias(\hat\theta)$
\DeclareMathOperator*{\ran}{ran}			% Range of an operator, e.g. $\ran(T) 
\DeclareMathOperator*{\dv}{d\!}			% Non-italicized 'with respect to', e.g. $\int f(x) \dv x$
\DeclareMathOperator*{\diag}{diag}		% Diagonal of a matrix, e.g. $\diag(M)$
\DeclareMathOperator*{\trace}{Tr}		% Trace of a matrix, e.g. $\trace(M)$
\DeclareMathOperator*{\supp}{supp}		% Support of a function, e.g., $\supp(f)$

% Numbered theorem, lemma, etc. settings - e.g., a definition, lemma, and theorem appearing in that 
% order in Lecture 2 will be numbered Definition 2.1, Lemma 2.2, Theorem 2.3. 
% Example usage: \begin{theorem}[Name of theorem] Theorem statement \end{theorem}
\theoremstyle{definition}
\newtheorem{theorem}{Theorem}[section]
\newtheorem{conjecture}{Conjecture}[section]
\newtheorem{proposition}[theorem]{Proposition}
\newtheorem{lemma}[theorem]{Lemma}
\newtheorem{corollary}[theorem]{Corollary}
\newtheorem{definition}[theorem]{Definition}
\newtheorem{example}[theorem]{Example}
\newtheorem{remark}[theorem]{Remark}

% Un-numbered theorem, lemma, etc. settings
% Example usage: \begin{lemma*}[Name of lemma] Lemma statement \end{lemma*}
\newtheorem*{theorem*}{Theorem}
\newtheorem*{proposition*}{Proposition}
\newtheorem*{lemma*}{Lemma}
\newtheorem*{corollary*}{Corollary}
\newtheorem*{definition*}{Definition}
\newtheorem*{example*}{Example}
\newtheorem*{remark*}{Remark}
\newtheorem*{claim}{Claim}
\newtheorem*{question*}{Question}

% --- Left/right header text (to appear on every page) ---

% Do not include a line under header or above footer
\pagestyle{fancy}
\renewcommand{\footrulewidth}{0pt}
\renewcommand{\headrulewidth}{0pt}

% Right header text: Lecture number and title
\renewcommand{\sectionmark}[1]{\markright{#1} }
% \fancyhead[R]{\small\textit{\nouppercase{\rightmark}}}

% Left header text: Short course title, hyperlinked to table of contents
% \fancyhead[L]{\hyperref[sec:contents]{\small Gan-Gross-Prasad conjecture}}

% --- Document starts here ---

\begin{document}

% --- Main title and subtitle ---

\title{Recent Progress on the Gan--Gross--Prasad Conjectures (after Jacquet--Rallis, Waldspurger, W. Zhang, etc.) \\[1em]
\normalsize Progr\`es R\'ecents sur les Conjectures de Gan--Gross--Prasad (d'apr\`es Jacquet--Rallis, Waldspurger, W. Zhang, etc.)\\ - \\
\normalsize Re-\TeX ed by Seewoo Lee\footnote{seewoo5@berkeley.edu. Most of the translation is due to Google Translator, and I only fixed a little.}}

% --- Author and date of last update ---

\author{Rapha\"el Beuzart-Plessis}
\date{\normalsize\vspace{-1ex} Last updated: \today}

% --- Add title and table of contents ---

\maketitle


% --- Abstracts ---

% \tableofcontents\label{sec:contents}

% --- Main content: import lectures as subfiles ---


\section{Introduction}

\begin{conjecture}[Langlands functoriality conjecture] Let $G$ and $G'$ be reductive groups over a global field $F$. 
\end{conjecture}
This is an introductory note on Langlands functoriality conjecture view towards classical examples. Here is a list of topics we are going to study:

\begin{enumerate}
    \item Automorphic induction
    \item Base change
    \item Rankin-Selberg product
    \item Symmetric power lifting and Selberg's 1/4 conjecture
    \item Jacquet-Langlands correspondence
    \item Theta correspondence and Howe duality
\end{enumerate}
\section{The Local Conjectures}

\subsection{The groups}

Let $E/F$ be a quadratic extension of local fields of characteristic zero.
We therefore have either $E/F = \bC / \bR$ or that $E$ and $F$ are finite extensions of the field of $p$-adic numbers $\bQ_p$ for a certain prime number $p$ ($\bQ_p$ is the completion of $\bQ$ by the $p$-adic absolute value $|\cdot|_p$ defined by $|p^k \frac{a}{b}|
_p = p^{-k}$ for $a$ and $b$ integers prime to $p$).
We denote by $\sigma$  the unique non-trivial element of the Galois
group $\Gal(E/F)$ and $\sgn_{E/F}$ the quadratic character of $F$ associated with the extension $E/F$ by the class field theory (it is therefore the unique quadratic character with kernel $\rN_{E/F}(E^\times)$, the image of the norm map).
Finally, we will fix two non-trivial additive characters $\psi_0: F \to \bS^1$ and $\psi: E \to \bS^1$ with the property that $\psi$ is trivial on $F$.

Let $V$ be a finite dimensional vector space of dimension $n$ over $E$ and $\varepsilon \in \{ \pm 1 \}$.
We assume $V$ is equipped with a non-degenerate $\varepsilon$-hermitian form
\[
    \langle -, -\rangle: V \times V \to E.
\]
By definition a $\varepsilon$-hermitian form satisfies
\begin{align*}
    \langle \lambda v + \mu w, u \rangle &= \lambda \langle v, u \rangle + \mu \langle w, u \rangle \\
    \langle v, u \rangle &= \varepsilon \langle u, v \rangle^\sigma
\end{align*}
for all $u, v, w \in V$ and $\lambda, \mu \in E$.
Depending on whether $\varepsilon = 1$ or $-1$ we call it hermitian or skew-hermitian.
Let $W$ be a non-detenerate subspace of $V$ with
\[
    \dim(V) - \dim(W) = \begin{cases} 1 &\text{if }\varepsilon = 1 \\ 0 & \text{if } \varepsilon = -1. \end{cases}
\]
Let $\rU(V) \subset \GL(V)$ and $\rU(W) \subset \GL(W)$ be the algebraic subgroups (defined over $F$) of linear automorphisms of $V$ and $W$ preserving the form $\langle -, - \rangle$.
Then $\rU(V)$ and $\rU(W)$ are unitary groups and we have a natural embeddign $\rU(W) \hookrightarrow \rU(V)$ where $\rU(W)$ acts trivially on $W^\perp$ (which of dimension at most 1).
In the following we will (abusively) identify an algebraic group defined on $F$ with the group of $F$-points corresponding to it.

The following discussion also extends to the case where $E = F \times F$ equipped with the involution $\sigma(x, y) = (y, x)$, a case which it will be necessary to include anyway when we will deal with the global conjecture.
In such a situation, a non-degenerate form $\langle -, -\rangle$ as above identifies $V$ and $W$ to direct sums $V_0 \oplus V_0^\vee$ and $W_0 \oplus W_0^\vee$ where $W_0 \subset V_0$ are the finite dimensional vector spaces over $F$ and $V_0^\vee$, $W_0^\vee$ denote their duals.
We then have a natural identifications $\rU(V) \simeq \GL(V_0)$ and $\rU(W) \simeq \GL(W_0)$.

In all cases, we put $G = \rU(W) \times \rU(V)$, $H = \rU(W)$ and we embed $H$ into $G$ diagonally.
The groups $H$ and $G$ inherit from the field $F$ topologies which make them Lie groups in the archimedean case (i.e. when $F = \bR$) and locally profinite groups in the non-archimedean case (i.e. when $F$ is a finite extension of $\bQ_p$; recall that a topological group is locally profinite if it has a basis of neighborhoods of the identity element consist of compact subgroups).


\subsection{The restriction problem}

Let $(\pi, \scV)$ be a smooth and irreducible complex representation of $G$.
In the $p$-adic case, this means that $\pi$ is a representation of $G$ on a $\bC$-vector space $\scV$ (typically of infinite dimension) all of whose vectors have a open stabilizer, irreducibility is then
an algebraic notion (i.e. no non-trivial subspace stable under $G$).
In the archimedean case, this means that $\scV$ is a Fréchet space and that $\pi$ is a smooth representation (in the $C^\infty$ sense), admissible (i.e. the irreducible representations of a maximal compact subgroup appear with finite multiplicities) on $\scV$ satisfying a certain condition of “moderate growth” (which was introduced by Casselman and Wallach, see \cite{casselman1989canonical} and \cite{wallachreal} Chap. 11); irreducibility is then a topological notion (ie no non-trivial closed subspace stable by $G$).
In any case, such an irreducible representation decomposes as a tensor product $\pi = \pi_W \boxtimes \pi_V$ where $\pi_W$ and $\pi_V$ are irreducible (smooth) representations of $\rU(W)$ and $\rU(V)$ respectively (and where the tensor product is a topological tensor product in the archimedean case).
We will denote as $\Irr(G)$ for the set of isomorphism classes of smooth irreducible representations of $G$.

To define the restriction problem that will interest us, we must also introduce a certain “small” representation $\nu$ of $H$.
In the hermitian case (i.e. if $\varepsilon = 1$), $\nu$ is the trivial representation that we will denote as $1$ or simply $\bC$ in the following.
In the skew-hermitian case (i.e. if $\varepsilon = -1
$), we have an inclusion
\[
    \rU(W) \subset \Sp(\Res_{E/F}W)
\]
where $\Res_{E/F}W$ denotes the restriction of the scalars from $E$ to $F$ of $W$ equipped with the symplectic form $\Tr_{E/F} \circ \langle -, -\rangle$ and $\Sp(\Res_{E/F}W)$ denotes the corresponding symplectic group.
Let $\Mp(\Res_{E/F}W)$ be the metaplectic group associated with this symplectic space (it is a $\bZ/2\bZ$-extension of $\Sp(\Res_{E/F}W)$).
The metaplectic covering splits over $\rU(W)$ but this splitting is not unique (because there are non-trivial characters $\rU(W) \to \{\pm 1\}$).
We can, however, fix such a splitting by choosing a character $\mu: E^\times \to \bS^1$ with $\mu|_{F^\times} = \sgn_{E/F}$ from now on.
Let $\omega_{\psi_0, W}$ be the Weil representation of $\Mp(\Res_{E/F} W)$ associated to the character $\psi_0$ (c.f. \cite{moeglin2006correspondances} Chap. 2. II).
Then $\nu = \omega_{\psi_0, W, \mu}$ is the restriction of this Weil representation to $\rU(W)$ via the splitting that we have just fixed.

For every case, the space of intertwining maps which is of our interest is the following
\begin{equation}
    \label{eqn:2}
    \Hom_H(\pi, \nu)
\end{equation}
where implicitly we only consider the continuous maps in the archimedean case (for the underlying Fr\'echet topologies).
We denote $m(\pi)$ for the dimension of this space
\[
    m(\pi):= \dim \Hom_H(\pi, \nu).
\]
Note that in the hermitian case we have identifications
\[
    \Hom_H(\pi, \nu) = \Hom_{\rU(W)}(\pi_W \boxtimes \pi_V, \bC) = \Hom_{\rU(W)}(\pi_V, \pi_W^\vee)
\]
where $\pi_W^\vee$ denotes the (smooth) contragredient representation of $\pi_W$.

An element of space \eqref{eqn:2} is called a Bessel functional if $\varepsilon = 1$ and a Fourier-Jacobi functional if $\varepsilon =-1$.
We will then talk in parallel about the Bessel and Fourier-Jacobi cases of the conjecture.


\subsection{Multiplicity $1$}

The following theorem is due to Aizenbud--Gourevitch--Rallis--Schiffmann \cite{aizenbud2010multiplicity} and Sun \cite{sun2012multiplicity} in the $p$-adic case and to Sun--Zhu \cite{sun2012multiplicityarchimedean} in the archimedean case.

\begin{theorem}
For any smooth irreducible representations $\pi$ of $G$ we have
\[m(\pi) \leq 1.\]
\end{theorem}

The local Gan--Gross--Prasad conjecture then essentially provides an answer to the following simple question: when do we have $m(\pi) = 1$? 
Just as for the law of branching between real compact unitary groups discussed in the introduction, any comprehensible answer to this question requires knowing how to parameterize the (isomorphism classes of) irreducible representations of $G$.
Such a parameterization is precisely the object of the local Langlands correspondence (for unitary groups) whose main properties we now recall.


\subsection{Local Langlands correspondence for unitary groups}

In this section we consider a hermitian or skew-hermitian space $V$ of finite dimension $n$ over $E$ and we denote by $\rU(V)$ the corresponding unitary group.

\subsubsection{Weil-Deligne group}

Let $W_F$ be the Weil group of $F$.
If $F$ is non-archimedean, we have the following commutative diagram where each row are exact
\begin{center}
\begin{tikzcd}
    1 \arrow[r] & I_F \arrow[r] \arrow[d, equal] &  \Gal(\overline{F} / F) \arrow[r] & \Gal(\overline{k_F} / k_F) \simeq \widehat{\bZ} \arrow[r] & 1 \\
    1 \arrow[r] & I_F \arrow[r] & W_F \arrow[u] \arrow[r] & \bZ \arrow[r] \arrow[u] & 1
\end{tikzcd}
\end{center}
where $\overline{F}$ is an algebraic closure of $F$, $k_F$ is the residue field of $F$, the isomorphism $\Gal(\overline{k_F} / k_F) \simeq \widehat{\bZ}$ correspond to the choice of the geometric Frobeinus $\Frob_F$ as a topological generator of $\Gal(\overline{k_F} / k_F)$ and $I_F$ is the inertia subgroup (i.e. the kernel of the arrow $\Gal(\overline{F}/ F) \to \Gal(\overline{k_F} / k_F)$).
We then equip $W_F$ with the topology that mesk $I_F$ as an open subgroup (the topology induced from that of $\Gal(\overline{F} / F)$).
If $F$ is archimedean, we have
\[
    W_F = \begin{cases} \bC^\times \cup \bC^\times j & \text{if }F = \bR \\ \bC^\times & \text{if }F = \bC,
\end{cases}
\]
where $j^2 = -1$ and $jzj^{-1} = \bar{z}$ for all $z \in \bC^\times$.
The Weil-Deligne group $\WD_F$ of $F$ is defined by 
\[
    \WD_F = \begin{cases} W_F \times \SL_2(\bC) & \text{if }F\text{ is non-archimedean} \\ W_F & \text{if }F \text{ is archimedean}. \end{cases}
\]


\subsubsection{Langlands parameters}

Langlands associates with $\rU(V)$, and more generally with any connected reductive group over $F$, an \emph{$L$-group} ${}^L \rU(V)$ that is a semi-direct product of a complex reductive group $\widehat{\rU(V)}$ with the Weil group $W_F$: ${}^L \rU(V) = \widehat{\rU(V)} \rtimes W_F$.
Here, the $L$-group is explicitly described as follows: we have $\widehat{\rU(V)} = \GL_n(\bC)$ and the action of $W_F$ factors through $W_F \to W_F / W_E = \Gal(E/F)$ with $\sigma$ acts as $\sigma(g) = J^{t}g^{-1} J^{-1}$, where
\[
J = \begin{pmatrix} & & & 1 \\
& & -1 & \\
& \iddots & & \\
(-1)^{n-1} & & & 
\end{pmatrix}.
\]

A Langlands parameter for $\rU(V)$ is then a $\widehat{\rU(V)}$-conjugacy class of "admissible" homomorphisms (i.e. satisfying certain properties of continuity, semi-simplicity and algebraicity)
\[
    \phi: \WD_F \to {}^L \rU(V)
\]
commuting with projections on $W_F$.
We denote $\Phi(\rU(V))$ the set of Langlands parameters for $\rU(V)$.
For the unitary groups we have the following more explicit description (c.f. \cite{gan2011symplectic} Theorem 8.1): the restriction to $\WD_E$ induces a bijection between $\Phi(\rU(V))$ and the set of isomorphism classes of the complex continuous semi-simple and algebraic representations on $\SL_2(\bC)$ of dimension $n$ of $\WD_E$ which are $(-1)^{n+1}$-conjugate dual.
Let's recall what this last term means.
Fix $c \in W_F \backslash W_E$ maps to $\sigma$.
A representation $\varphi: \WD_E \to \GL(M)$ is called \emph{conjugate dual} if there exists a non-degenerate bilinear form
\[
    B: M \times M \to \bC
\]
satisfying
\[
    B(\varphi(\tau)u, \varphi(c\tau c^{-1})v) = B(u, v), \quad \forall u, v \in M, \tau \in \WD_E.
\]
It is equivalent to ask if $M$ is isomorphic to $(M^c)^\vee$ where $M^c$ is the $c$-conjugate of $M$ and $(-)^\vee$ is the contragredient representation.
We further say that $\varphi: \WD_E \to \GL(M)$ is $\varepsilon$-conjugate-dual, where $\varepsilon \in \{\pm 1\}$, if we can choose a bilinear form satisfying the additional condition
\[
    B(u, \varphi(c^2)v) = \varepsilon B(v, u), \quad \forall u, v \in M.
\]
We will call such a form an $\varepsilon$-conjugate-dual form.

To state the Langlands correspondence in its most complete version, it is necessary to introduce for all $\phi \in \Phi(\rU(V))$ a certain finite group $S_\phi$.
The latter is defined as the group of connected components of the centralizer in $\widehat{\rU(V)}$ of the image of $\phi$.
If we identify $\phi$ with a $(-1)^{n+1}$-conjugate-dual representation $\varphi: \WD_E \to \GL(M)$, we have the following more concrete description of $S_\phi$.
Let $B$ be a conjugate-dual form of sign $(-1)^{n+1}$ as above and denote $\Aut(\varphi, B)$ the group of linear automorphisms of $M$ commutes with the image of $\varphi$ and preserve the form $B$.
We then have (canonically)
\[S_\phi = \Aut(\varphi, B) / \Aut(\varphi, B)^\circ\]
where we denote as $\Aut(\varphi, B)^\circ$ for the connected component of the identity element.
Moreover, this group is always abelian and isomorphic to a product of finitely many copies of $\bZ / 2\bZ$.


\subsubsection{Pure inner forms}

Following an idea from Vogan \cite{vogan1993local}, the Langlands correspondence should be formulated more simply if we consider several groups at the same time.
More precisely, we must take account the pure inner forms of $\rU(V)$.
These forms are naturally parameterized by the Galois cohomology set $\rH^1(F, \rU(V))$ and all admit the same $L$-group as $\rU(V)$ (so that a Langlands parameter for $\rU(V)$ can also be considered as Langlands parameter of all its pure inner forms).
For unitary groups we know how to describe the pure inner forms explicitly: $\rH^1(F, \rU(V))$ naturally classifies the isomorphism classes of (skew-)Hermitian spaces of dimension $n$ and the pure inner forms of $\rU(V)$ are then the unitary groups of the latter spaces.
For a class $\alpha \in \rH^1(F, \rU(V))$, we denote $V_\alpha$ the (skew-)hermitian space it determines and $\rU(V_\alpha)$ the corresponding pure inner form.

In the non-archimedean case, and for $n \neq 0$, there exist exactly two isomorphism classes of (skew-)hermitian spaces of dimension $n$, which can be distinguished by their discriminants, and therefore as many pure inner forms.
In the archimedean case, there are $n+1$ pure interior forms of $\rU(V)$ corresponding to $\rU(p, q)$ for $p + q = n$.
Note that two distinct pure inner forms of $\rU(V)$ can be isomorphic (e.g. $\rU(p, q) \simeq \rU(q,p)$) but from the point of view of the Langlands correspondence these must be considered separately.


\subsubsection{The correspondence}

We can now state the local Langlands correspondence for $\rU(V)$ (and its pure inner forms) in the following informal way.
For all $\alpha \in \rH^1(F, \rU(V))$, there should exist a partition
\[
    \Irr(\rU(V_\alpha)) = \bigsqcup_{\phi \in \Phi(\rU(V))} \Pi^{\rU(V_\alpha)}(\phi) 
\]
into finite (possibly empty) subsets called \emph{$L$-packets} and for all $\phi \in \Phi(\rU(V))$ there should exists a bijection
\begin{equation}
\label{eqn:3}
\begin{aligned}
    \bigsqcup_{\alpha \in \rH^{1}(F, \rU(V))} \Pi^{\rU(V_\alpha)}(\phi) &\simeq \widehat{S_\phi} \\
    \pi(\varphi, \chi) &\mapsfrom \chi
\end{aligned}
\end{equation}
where $\widehat{S_\phi}$ is the group of characters of the finite abelian group $S_\phi$.
This data must of course satisfy a certain number of properties.
In fact, the famous \emph{endoscopic relations}, which we will not explain here, characterize, if it exists, the local Langlands correspondence for the unitary groups from the known correspondence (\cite{harris2001geometry,henniart2000preuve,scholze2013local}), for linear groups.
These endoscopic relations depend however on a certain choice corresponding to the normalization of \emph{transfer factors}.
The composition of the $L$-packets does not depend on this choice but the bijection \eqref{eqn:3} depends on it.
We will give more details about the choices involved in this normalization in section 1.4.7.


\subsubsection{Status}

In the archimedean case, the local correspondence was constructed by Langlands himself \cite{langlands1989irreducible} for all real reductive groups from the results of Harish-Chandra.
This correspondence verifies the expected endoscopic relations follows from  the work of Shelstad \cite{shelstad1982indistinguishability,shelstad2008tempered,shelstad2010tempered} and Mezo \cite{mezo2016tempered} (see also \cite{clozel1982changement} for the case of unitary groups).

In the non-archimedean case, the correspondence was obtained much more recently by Mok \cite{mok2015endoscopic} for quasi-split unitary groups and then by Kaletha--Minguez--Shin--White \cite{kaletha2014endoscopic} for all unitary groups following the founding work of Arthur \cite{arthur2013endoscopic} on symplectic and orthogonal groups.
Until recently these results were still conditional on the stabilization of the twisted trace formula now established in full generality by Waldspurger and Moeglin--Waldspurger in an impressive series of papers \cite{moeglin2016stabilisation}.


\subsubsection{$L$-functions and $\varepsilon$-factors}

For a given Langlands parameter $\phi: \WD_F \to {}^L \rU(V)$ we can associate certain arithmetic invariants with it.
More precisely, for any algebraic representation $\rho: {}^L \rU(V) \to \GL(M)$ where $M$ is a finite dimensional complex vector space, the composition $\rho \circ \phi$ is a representation of the Weil-Deligne group $\WD_F$ to which we can associate a local $L$-function $L(s, \rho\circ \phi) = L(s, \rho, \phi)$ and a local $\varepsilon$ factor $\varepsilon(s, \rho\circ\phi, \psi_0) = \varepsilon(s, \phi, \rho, \psi_0)$ which depends on the additive character $\psi_0: F \to \bC^\times$.
The local $L$-functions are meromorphic functions on $\bC$ without zero while the local epsilon factors are invertible holomorphic functions on $\bC$.
In the case where $F$ is non-archimedean and $\rho \circ \phi$ is trivial on the factor $\SL_2(\bC)$ the $L$-function is defined by
\[
    L(s, \phi, \rho) = \frac{1}{\det(1 - q^{-s} (\rho \circ \phi)(\Frob_F)|_{M^{I_F}})},
\]
where we denote $q$ for the cardinality of the residue field of $F$, $M^{I_F}$ the subspace of $I_F$-invarians and $\Frob_F$ a (geometric) Frobenius in $W_F$.
We have an analogous formula in the general case if $F$ is non-archimedean (cf. \cite{tate1979number} 4.1.6) and if $F$ is archimedean the local $L$ factors are explicit products of gamma functions and powers of $\pi$ and $2$ (cf. \cite{tate1979number} \S 3 ).
Local epsilon factors are much more subtle invariants.
Indeed, these must satisfy a certain number of simple properties characterizing them only but their existence is a difficult theorem due independently to Langlands and Deligne (\cite{deligne1973constantes}).

Let us mention here a property of these factors that we will need.
Let $\varphi: \WD_E \to \GL(M)$ be a $(-1)$-conjugate-dual representation of the Weil-Deligne group of $E$.
Then, $\varepsilon(\frac{1}{2}, \varphi, \psi) \in \{ \pm 1\}$ where we recall that the character $\psi: E \to \bS^1$ is trivial on $F$.
Moreover, this epsilon factor depends only on the $\rN(E^\times)$-orbit of $\psi$ and in fact does not depend on $\psi$ at all if $\dim(\varphi)$ is even.


\subsubsection{Whittaker datum and normalization of the correspondence}

As explained in 1.4.4, the bijection \eqref{eqn:3} depends on a choice allowing to normalize certain transfer factors.
According to \cite{kottwitz1999foundations}, such a choice can be made by fixing a Whittaker datum of a pure inner form of $\rU(V)$.
More precisely, we first choose a quasi-split pure inner form $\rU(V_\alpha)$ of $\rU(V)$ having a Borel subgroup $B \subset \rU(V_\alpha)$ defined on $F$.
Such a group exists and even it means that by replacing $V$ by $V_\alpha$ (which does not modify the family of pure inner forms), we can assume that we have chosen $\rU(V)$ (which we therefore assume quasi-split).
A Whittaker data on $\rU(V)$ is then a conjugacy class of pairs $(N, \theta)$ where $N$ is the unipotent radical of a Borel subgroup $B = TN$ defined over $F$ and $\theta: N \to \bS^1$ is a \emph{generic} character whose stabilizer in $T$ equals to the center of $\rU(V)$.
There is only one conjugacy class of Whittaker data if $n = \dim(V)$ is odd while if $n$ is even there are two and one can be fixed from the
character $\psi: E /F \to \bS^1$ in hermitian case and $\psi_0: F \to \bS^1$ in the skew-hermitian case.


\subsubsection{Generic, tempered, and discrete $L$-packets}

A Langlands parameter $\phi: \WD_F \to {}^L \rU(V)$ is said to be generic if $L(s, \phi, \Ad)$ has no pole at $s = 1$ where $\Ad$ denotes the adjoint representation of ${}^L \rU(V)$ on its Lie algebra.
The corresponding L-packet $\Pi^{\rU(V)}(\phi)$ then contains one and only one representation $\pi$ admitting a Whittaker model for $(N, \theta)$ i.e. $\Hom_{N}(\pi, \theta) \neq 0$ (we then say that $\pi$ is \emph{generic} with respect to $(N, \theta)$) and moreover this representation corresponds via the bijection \eqref{eqn:3} to the trivial character of $S_\phi$.

A Langlands parameter $\phi: \WD_F \to {}^L \rU(V)$ is \emph{tempered} if the projection of the image of $W_F$ onto $\widehat{\rU(V)}$ is relatively compact.
A tempered parameter is automatically generic and the corresponding $L$-packet $\Pi^{\rU(V)}(\phi)$ only contains \emph{tempered} representations, i.e. representations which weakly contained in $L^2(\rU(V))$ (there is also a characterization of tempered representations by a condition of growth of coefficients).
In fact, one can reconstruct the Langlands correspondence for $\rU(V)$ from the correspondence restricted to the tempered parameters of $\rU(V)$ and its Levi subgroups.
This follows from the Langlands classification which makes it possible to obtain all the irreducible representations of a reductive group from the tempered representations of its Levi subgroups by a classical process called parabolic induction.

Finally, a Langlands parameter $\phi: \WD_F \to {}^L \rU(V)$ is said to be \emph{discrete} if the centralizer of its image in $\widehat{\rU(V)}$ is finite.
A discrete parameter is automatically tempered (therefore also generic) and determines an $L$-packet of representations of the discrete series which appear as submodules of $L^2(\rU(V))$.



\subsection{The conjecture}



We return to the situation introduced in 1.1 and 1.2.
Let us call \emph{pure inner form} of $(G, H)$ a pair $(G_\alpha,H_\alpha)$ obtained in the following way.
Let $\alpha \in \rH^1(F, H)$ and $W_\alpha$ the corresponding (skew-)hermitian space.
We then set $V_\alpha = W_\alpha \oplus L$, where $L$ is a space such as $V = W \oplus L$, $H = \rU(W_\alpha)$ and $G_\alpha = \rU(W_\alpha) \times \rH(V_\alpha)$.
We again have an injection $H_\alpha \hookrightarrow G_\alpha$ and we define as in section 1.2 a “small” representation $\nu_\alpha$ of $H_\alpha$ (which depends, like $\nu$, in the Fourier--Jacobi case on the choices of $\psi_0$ and $\mu$) as well as a multiplicity function $\pi\in \Irr(G_\alpha) \mapsto m(\pi)$ by
\[
    m(\pi) = \dim \Hom_{H_\alpha} (\pi, \nu_\alpha).
\]
Note that $G_\alpha$ is then a pure inner form of $G$ but that in general we do not obtain all the pure inner forms of $G$ in this way.
The pure inner forms of $G$ thus obtained will be called \emph{relevant}.
There always happens to be a relevant pure inner form which is quasi-split.
By changing our initial pair if needed, we will therefore assume that $G$ itself is quasi-split.
Then we fix the Langlands correspondence for $\rU(V)$ and $\rU(W)$ (and their pure inner forms) as in 1.4.7.

Let $\phi: \WD_F \to {}^L \rU(V)$ and $\phi': \WD_F \to {}^L \rU(W)$ be two Langlands parameters identified with complex representations $\varphi: \WD_E \to \GL(M)$ and $\varphi': \WD_E \to \GL(N)$ of dimensions $d_V = \dim(V)$ and $d_W = \dim(W)$ and which are $(-1)^{d_V + 1}$- and $(-1)^{d_W + 1}$-conjugate-dual respectively.
According to Gan, Gross and Prasad, we define two characteristics
\[
    \chi_{\phi, \phi'}: S_\phi \to \{ \pm 1\}\text{ and } \chi_{\phi', \phi}:S_{\phi'} \to \{\pm1\}
\]
as follows.
Fix non-degenerate forms $B$ and $B'$ on $M$ and $N$ which are $(-1)^{d_V + 1}$ and $(-1)^{d_W + 1}$-conjugate-dual respectively so we have identifications
\[
    S_\phi = \Aut(\varphi, B) / \Aut(\varphi, B)^\circ \text{ and } S_{\phi'} = \Aut(\varphi', B) / \Aut(\varphi', B)^\circ.
\]
Let $s \in S_\phi$ and $s' \in S_{\phi'}$, regarding as elements of $\Aut(\varphi, B)$ and $\Aut(\varphi', B')$ respectively.
In the Bessel case (i.e. $\varepsilon = 1$), we set
\[
    \chi_{\phi, \phi'}(s) = \varepsilon\left(\frac{1}{2}, \varphi^{s = -1} \otimes \varphi', \psi_{-2\delta}\right) \text{ and } \chi_{\phi', \phi}(s') = \varepsilon\left(\frac{1}{2}, \varphi \otimes (\varphi')^{s' = -1}, \psi_{-2\delta}\right)
\]
where $\varphi^{s=-1}$ (resp. $(\varphi')^{s' = -1}$) denote the subrepresentation of $\varphi$ (resp. $\varphi'$) where $s$ (resp. $s'$) acts as $-1$, $\delta$ is the discriminant of the unique odd-dimensional hermidian space in the pair $(W, V)$ and $\psi_{-2\delta}(x) = \psi(-2\delta x)$.
In the Fourier--Jacobi case (i.e. $\varepsilon = -1$), we set
\[
    \chi_{\phi, \phi'}(s) = \varepsilon\left(\frac{1}{2}, \varphi^{s = -1} \otimes \varphi' \otimes \mu^{-1}, \psi_\lambda\right) \text{ and } \chi_{\phi', \phi}(s) = \varepsilon\left(\frac{1}{2}, \varphi \otimes (\varphi')^{s=-1} \otimes \mu^{-1}, \psi_\lambda\right)
\]
where $\mu$ is the multiplicative character of $E^\times$ that we fixed to define the representation $\nu$, $\lambda = 1$ in the case where $\dim(V)$ is even and $\lambda$ is the unique element of $F^\times$ such that $\psi(\lambda x) = \psi_0(\Tr_{E/F}(ex))$ for all $x \in E$ with $e$ the discriminant of the skew-hermitian space $V$ in the case where $\dim(V)$ is odd.
In any case, we show that the result does not depend on the choices of representatives of $s$ and $s'$ and thus we have defined the characters of $S_\phi$ and $S_{\phi'}$ (\cite{gan2011symplectic} Theorem 6.1).


\begin{conjecture}[Gan--Gross--Prasad]
\label{conj:localggp}
Let $\phi$ and $\phi'$ be generic Langlands parameters.
Then
\begin{enumerate}
    \item We have $\sum_{\alpha \in \rH^1(F, H)} \sum_{\pi \in \Pi^{G_\alpha}(\phi \times \phi')} m(\pi) = 1$.
    \item More precisely, for all pair of charaters $(\chi, \chi') \in \widehat{S_\phi} \times \widehat{S_{\phi'}}$ such that $\pi(\phi, \chi) \boxtimes \pi(\phi', \chi')$ is a representation of a pure inner form of $G$, we have
    \[
        m(\pi(\phi, \chi) \boxtimes \pi(\phi', \chi')) = 1 \Leftrightarrow \chi = \chi_{\phi, \phi'}\text{ and }\chi = \chi_{\phi', \phi}.
    \]
\end{enumerate}
\end{conjecture}


\subsection{Status}

\subsubsection{Bessel case}

In a series of four seminal papers \cite{waldspurger1990demonstration,waldspurger2010formule,waldspurger2012calcul,waldspurger2012conjecture}, Waldspurger established the analogue of conjecture \ref{conj:localggp} for orthogonal special groups (an analogue which exists only in the Bessel case) when $F$ is $p$-adic and when $\phi$ and $\phi'$ are tempered Langlands parameters.
This result was then extended by Moeglin and Waldspurger \cite{moeglin2012conjecture} to all generic parameters.
In my thesis \cite{beuzart2014expression,beuzart2015endoscopie,beuzart2016conjecture} I adapted Waldspurger's method in order to prove conjecture \ref{conj:localggp} in the Bessel case and for tempered Langlands parameters of $p$-adic unitary groups.
The extension to all generic $L$-packets was done by Gan and Ichino in \cite{gan2016gross} (section 9.3) crucially using a result of Heiermann \cite{heiermann2016note} which generalizes part of the Moeglin--Waldspurger argument.
Still following the method initiated by Waldspurger, I established in \cite{beuzart2015local} the property of multiplicity one in $L$-packets (i.e. the 1 of conjecture \ref{conj:localggp}) still for tempered parameters and in the Bessel case but this time for real unitary groups. 
The preliminary work carried out in \cite{beuzart2015local} should make it possible to completely adapt Waldspurger's method for archimedean fields and thus to obtain a complete proof of the conjecture in the Bessel case (for unitary groups).
Unfortunately, the sequel to \cite{beuzart2015local} has not yet been written.
Finally, by a completely different method using theta correspondence and specific to the archimedean case, Hongyu He obtained in \cite{he2017gan} a proof of the conjecture in the Bessel case for the discrete Langlands parameters of real unitary groups.


\subsubsection{Fourier-Jacobi case}

For $p$-adic fields and shortly after the proof of the conjecture in the Bessel case, Gan and Ichino \cite{gan2016gross} showed how, by using the local theta correspondence, one could deduce the conjecture in the Fourier--Jacobi case.
This method was later adapted by Hiraku Atobe \cite{atobe2018local} to establish the analogue of conjecture \ref{conj:localggp} for symplectic/metaplectic groups (analogue that only exists in the Fourier--Jacobi case) over a $p$-adic field.
On the other hand, the Fourier--Jacobi case of the conjecture remains completely open for archimedean fields.


\subsection{Brief overview of proofs}

\subsubsection{The Bessel case}

We present here a rapid overview of the method initiated by Waldspurger, and adapted by the author to the case of unitary groups, to prove the local conjecture in the Bessel case for tempered representations.
This method is based on an integral formula calculating the multiplicity $m(\pi)$ when $\pi$ is tempered.
Let us first present this formula in the simplest case, i.e. when the groups $G$ and $H$ are compact (this can happen at any rank for the real groups but over the $p$-adic fields it implies $\dim(V) \leq 2$).
The representation $\pi$ is then of finite dimension and has a character $\theta_\pi$ defined by $\theta_\pi(g) = \Tr(\pi(g))$ for all $g \in G$.
By the orthogonality relations between characters we immediately get
\[
    m(\pi) = \int_H \theta_\pi(h) \dd h
\]
where $\dd h$ is the unique Haar measure on $H$ of total mass $1$.
By the Weyl's integration formula, this can be written as
\begin{equation}
\label{eqn:4}
    m(\pi) = \sum_{T \in \scT(H)} |W(H, T)|^{-1} \int_T D^H(t) \theta_\pi(t) \dd t
\end{equation}
where $\scT(H)$ denotes a set of representatives of the conjugacy classes of maximal tori in $H$, $W(H, T)$ is the Weyl group $N_H(T) / T$ where $N_H(T)$ is the normalizer of $T$ in $H$ and $D^H(t) = |\det(1 - \Ad(t))_{\frh / \frt}|$ is the Weyl discriminant (with $\frh$ and $\frt$ the Lie algebras of $H$ and $T$ respectively).
In the case of real compact groups, we know explicit formulas (also due to Weyl) for the characters $\theta_\pi$ and the above formula then makes it possible to find directly the branching law presented in the introduction (note that in this case $\scT(H)$ is reduced to one element).

Waldspurger's method makes it possible to generalize formula \eqref{eqn:4} to groups that are not necessarily compact.
Several difficulties then arise.
First of all, the character $\theta_\pi$ no
longer has any meaning \emph{a priori} since the representation $\pi$ is in general of infinite dimension and the operators $\pi(g)$, $g \in G$ are not trace-class. 
A very deep result of Harish-Chandra nevertheless allows us to define such a character $\theta_\pi(g)$.
More precisely, Harish-Chandra first defines a character distribution $f \in C_c^\infty(G) \mapsto \theta_\pi(f):= \Tr(\pi(f))$ where $\pi(f) = \int_G f(g)\pi(g) \dd g$ (we show without too much difficulty that these
operators are of trace-class; they even have finite rank for $p$-adic groups) and proves that this distribution is representable by a locally integrable function $\theta_\pi$ on $G$ with nice properties.
In particular, this function is smooth (i.e. locally constant in the $p$-adic case) on the open locus $G_{\reg}$ of regular semi-simple elements and Harish-Chandra even described the singularities that $\theta_\pi$ can have in the neighborhood of the singular elements. 
Since intersections of maximal tori of $H$ with $G_\reg$ are open and have negligible complements, the formula \eqref{eqn:4} makes sense in general (modulo convergence issue).
However, the Waldspurger's formula differs from \eqref{eqn:4} by two aspects.
First, not all maximal tori contribute to it, but only those that are compact (which essentially settles questions of convergence)
Secondly, certain tori are not maximal (but still compact).
This last property implies in particular the existence of non-negligible contributions from certain singular conjugacy classes which requires defining a regularization $x \mapsto c_\pi(x)$ of the character $\theta_\pi$  at these points.
When $\dim(V) = 2$ (so that $\dim(W) = 1$ and $H$ is itself a torus) but when the group $G$ is not compact the formula has the following form
\begin{equation}
\label{eqn:5}
    m(\pi) = \int_H \theta_\pi(t) \dd t + c_\pi(1).
\end{equation}
Thus, here the only singular contribution comes from the trivial conjugacy class.
We refer to the introductions of \cite{waldspurger2010formule} (for the case of orthogonal groups) and \cite{beuzart2016conjecture} for more details on the formula in the general case.
Now let's explain briefly how we can deduce from formulas \eqref{eqn:4} and \eqref{eqn:5} the first part of conjecture \ref{conj:localggp} in the case of $\dim(V) = 2$ and $F$ is $p$-adic.
More precisely, we denote $(G_i, H_i)$ for the unique pure inner form with quasi-split $G_i$ (hence non-compact) and $(G_a, H_a)$ the only other pure inner form which is, in contrary, compact.
Let $\phi: \WD_F \to {}^L G_i = {}^L G_a$ be a tempered Langlands parameter.
Then each of the $L$-packets $\Pi^{G_i}(\phi)$ and $\Pi^{G_a}(\phi)$ can contain at most two elements (and $\Pi^{G_a}(\phi)$ can be empty).
Let $\theta_{\phi, \natural} = \sum_{\pi \in \Pi^{G_\natural}(\phi)}\theta_\pi$ for $\natural \in \{i, a\}$.
By \eqref{eqn:4} and \eqref{eqn:5}, we have $\sum_{\pi \in \Pi^{G_a}(\phi)} m(\pi) = \int_{H_a} \theta_{\phi, a}(h) \dd h$ and $\sum_{\pi \in \Pi^{G_i}(\phi)} m(\pi) = \int_{H_i} \theta_{\phi, i}(h) \dd h + c_{\phi, i}(1)$ where $c_{\phi, i}(1)$ is a certain regularization of $\theta_{\phi, i}$ at $1$.
Since the groups $H_i$ and $H_a$ are unitary groups of rank one, we have a natural isomorphism $H_i \simeq H_a$ and furthermore via this isomorphism we have the equality $\theta_{\phi, i}(h) = -\theta_{\phi, a}(h)$ (this is the simplest example of the famous \emph{endoscopic relations}).
By summing the two formulas, we therefore obtain
\[
    \sum_{\pi \in \Pi^{G_i}(\phi) \cup \Pi^{G_a}(\phi)} m(\pi) = c_{\phi, i}(1).
\]
Finally, according to a result of Rodier \cite{rodier2006modele} and the definition of $c_{\phi, i}(1)$ (which we have not given here), this last term counts the number of generic representations in the $L$-packet $\Pi^{G_i}(\phi)$ corresponding to a certain Whittaker datum.
Since $\phi$ is generic this number becomes $1$, which concludes the proof.
The same idea (slightly more elaborate) leads to a proof for genera ranks of the first part of conjecture \ref{conj:localggp} from Waldspurger's formula.

To obtain the second part of the conjecture, Waldspurger introduces a second essential ingredient: an integral formula, analogous to the previous one, expressing certain epsilon factors of pairs.
In the context of the conjecture for unitary groups, this formula expresses more precisely a factor of the form $\varepsilon(\frac{1}{2}, \pi \times \pi', \psi)$, where $\pi$ and $\pi'$ are tempered irreducible representations of $\GL_k(E)$ and $\GL_l(E)$ which are conjugate-dual (i.e. $\pi^\sigma \simeq \pi^\vee$ and $(\pi')^\sigma \simeq (\pi')^\vee$) with $k\not\equiv l\,\mathrm{mod}\,2$ as a function of “twisted” characters associated with $\pi$ and $\pi'$ (more precisely the restriction to the connected component of the identity of the extension of characters of $\pi$ and $\pi'$ to $\GL_i^+(E) = \GL_i(E) \rtimes \langle \theta_i\rangle$ where $\theta_i g \theta_i^{-1} = {}^t(g^\sigma)^{-1}$ for $i = k, l$).
Here $\varepsilon(s, \pi \times \pi', \psi)$ is a certain epsilon factor defined by Jacquet, Piatetski-Shapiro, and Shalika \cite{jacquet1983rankin} and which is equal to the Artin's epsilon factor $\varepsilon(s, \phi \otimes \phi', \psi)$ where $\phi: \WD_E \to \GL_k(\bC)$ and $\phi': \WD_E \to \GL_l(\bC)$ are the Langlands parameters of $\pi$ and $\pi'$ obtained via the local Langlands correspondence for linear groups (proved by Harris-Taylor \cite{harris2001geometry}, Henniart \cite{henniart2000preuve} and more recently Scholze \cite{scholze2013local}; this compatibility with the $\varepsilon$ factors of pairs is moreover an essential ingredient to characterize this correspondence).
We will not give more details on this formula (nor on its proof) and we will simply refer the reader to the introductions of \cite{waldspurger2012calcul} and \cite{beuzart2014expression} for more details.
This formula for the epsilon factors of pairs has not yet been proved in the archimedean case and is the missing part to finish the proof of conjecture \ref{conj:localggp} in general.


Finally, the last part of the proof for tempered representations \cite{waldspurger2012conjecture}, \cite{beuzart2015endoscopie} consists of relating the two previous formulas, for the multiplicity $m(\pi)$ and for the epsilon factors of pairs, via the \emph{endoscopic relations} between characters.
Indeed, these relations which, let us recall, characterize the local correspondence for unitary groups, essentially make it possible to express the character of any tempered representation of $G$ from “twisted” characters on linear groups as above.
Then we come across, quite miraculously, an expression of $m(\pi)$, for a tempered representation $\pi$ of $G$, in terms of epsilon factors of pairs which is exactly the formula predicted by the Gan--Gross--Prasad conjecture.


\subsubsection{The Fourier--Jacobi case}

We explain here the outline of the proof by Gan and Ichino \cite{gan2016gross} of conjecture \ref{conj:localggp} in the Fourier--Jacobi case.
For this, we need to make some reminders about the local theta correspondence for unitary groups.
For simplicity, we will restrict ourselves to the case of tempered representations.


Recall that a \emph{reductive dual pair} of a symplectic group $\Sp(\scW)$ is a pair $(\rU_1, \rU_2)$ of reductive subgroups of $\Sp(\scW)$ which are centralizers of each other.
Let $W_n$ be an $n$-dimensional skew-hermitian space, $V_{n+1}$ an ($n+1$)-dimensional hermitian space, $V_n \subset V_{n+1}$ a non-degenerate hyperplane and $V_1$ the orthogonal complement of $V_n$ in $V_{n+1}$.
Then in the symplectic group $\Sp(\Res_{E/F} W_n \otimes V_{n+1})$ we have the following two dual reductive pairs
\[
    (\rU(W_n), \rU(V_{n+1}))\text{ and } (\rU(W_n) \times \rU(W_{n}), \rU(V_n) \times \rU(V_{1}))
\]
where the action of the first pair on $W_n \otimes V_{n+1}$ is the obvious action while the action of the second pair respects the decomposition $W_n \otimes V_{n+1} = (W_n \otimes V_n) \oplus (W_n \otimes V_1)$.
Moreover, we have inclusions $\rU(W_n) \subset \rU(W_n) \times \rU(W_n)$ and $\rU(V_n) \times \rU(V_1) \subset \rU(V_{n+1})$.
We summarize the situation in the form of the following diagram, called ``see-saw'' diagram (or \emph{scissor} diagram),

\begin{equation}
\label{diag:6}
\begin{tikzcd}
    \rU(W_n) \times \rU(W_n) \arrow[d, dash] \arrow[dr, dash] & \rU(V_{n+1}) \arrow[d, dash] \arrow[dl, dash]\\
    \rU(W_n) & \rU(V_n) \times \rU(V_1).
\end{tikzcd}
\end{equation}
The character $\mu: E^\times \to \bC^\times$ which we have fixed, and whose restriction to $F^\times$ coincides with $\sgn_{E/F}$, makes it possible to identify all these groups in the metaplectic covering $\Mp(\Res_{E/F} W_n \otimes V_{n+1})$.
Let $\omega_{\psi_0, W_n, V_{n+1}}$ be the Weil representation of $\Mp(\Res_{E/F} W_n \otimes V_{n+1})$ associated with the additive character $\psi_0$.
For any irreducible representation $\sigma$ of $\rU(W_n)$, the maximal $\sigma$-isotypic quotient of $\omega_{\psi_0, W_n, V_{n+1}}$ is of the form $\sigma \boxtimes \Theta_{W_n, V_{n+1}}(\sigma)$ where $\Theta_{W_n, V_{n+1}}(\sigma)$ is a representation of $\rU(V_{n+1})$ which always happens to be of finite length.
Moreover, if this representation is non zero then it admits a unique irreducible quotient according to \cite{waldspurger1990demonstration} and \cite{gan2016howe} (this is the famous “Howe's duality conjecture”).
Let's denote this irreducible quotient as $\theta_{W_n, V_{n+1}}(\sigma)$ when it exists.
Then, the partially defined map $\sigma \mapsto \theta_{W_n, V_{n+1}}(\sigma)$ gives a bijection between a part of $\Irr(\rU(W_n))$ and a part of $\Irr(\rU(V_{n+1}))$: this is what called the local theta correspondence.
We define in the same way a map $\pi \in\Irr(\rU(V_n)) \mapsto \Theta_{W_n, V_{n}}(\pi)$ with values in the representations of finite lengths of $\rU(W_n)$ and a partially defined map $\pi \in \Irr(\rU(V_n)) \mapsto \theta_{W_n, V_n}(\pi) \in \Irr(\rU(W_n))$ by realizing $(\rU(V_n), \rU(W_n))$ as a dual reductive pair in $\Sp(\Res_{E/F} V_n \otimes W_n)$ (also in the metaplectic cover as before).
The restriction of the Weil representation $\omega_{\psi_0, W_n, V_{n+1}}$ to $\Mp(\Res_{E/F}W_n \otimes V_n) \times \Mp(\Res_{E/F} W_n \otimes V_1)$ via the natural morphism $\Mp(\Res_{E/F} W_n \otimes V_n) \times \Mp(\Res_{E/F} W_n \otimes V_1) \to \Mp(\Res_{E/F} W_n \otimes V_{n+1})$ is isomorphic to the tensor product $\omega_{\psi_0, W_n, V_n} \boxtimes \omega_{\psi_0, W_n, V_1}$.
It follows by the diagram \eqref{diag:6} that there is a natural isomorphism (the ``see-saw'' identity)
\begin{equation}
    \label{eqn:7}
    \Hom_{\rU(W_n)}(\Theta_{W_n, V_n}(\pi) \otimes \omega_{\psi_0, W_n, V_1}, \sigma) \simeq \Hom_{\rU(V_n)}(\Theta_{W_n, V_{n+1}}(\sigma), \pi)
\end{equation}
for all representations $\pi \in \Irr(\rU(V_n))$ and $\sigma \in \Irr(\rU(W_n))$.
The contragredient of the Weil representation $\omega_{\psi_0, W_n, V_1}$ is $\omega_{\psi_0^{-1}, W_n, V_1}$, the space on the left hand side is essentially the space of Fourier--Jacobi functionals on the representation $\Theta_{W_n, V_n}(\pi) \boxtimes \sigma^\vee$ of the group $\rU(W_n) \times \rU(W_n)$ while the space on the right hand side is the space of the Bessel functionals on the representation $\pi^\vee \boxtimes \Theta_{W_n, V_{n+1}}(\sigma)$ of the group $\rU(V_n) \times \rU(V_{n+1})$.
Thus the identity \eqref{eqn:7} relates the Bessel and Fourier--Jacobi cases of the conjecture and to deduce the Fourier-Jacobi case from the Bessel case, at least for tempered representations, it is roughly sufficient to
\begin{itemize}
    \item[--] Show that for tempered $\pi$ and $\sigma$ the representations $\Theta_{W_n, V_n}(\pi)$ and $\Theta_{W_n, V_{n+1}}(\sigma)$ are zero or irreducible (in which case they coincide with $\theta_{W_n, V_n}(\pi)$ and $\theta_{W_n, V_{n+1}}(\sigma)$ respectively).
    \item[--] Explain the local theta correspondences $\pi \mapsto \theta_{W_n, V_n}(\pi)$ and $\sigma\mapsto \theta_{W_n, V_{n+1}}(\sigma)$ in terms of the local Langlands correspondence.
    \item[--] Show that by changing the hermitian space $V_n$ if necessary, the local theta correspondence $\pi \mapsto \theta_{W_n, V_n}(\pi)$ restricted to tempered representations is surjective.
\end{itemize}
The first and third points had already been established by Gan and Ichino in \cite{gan2014formal} for another purpose.
Moreover, Prasad \cite{prasad1993local,prasad2000theta} had stated precise conjectures concerning the second point.
These conjectures are demonstrated in \cite{gan2016gross} by Gan and Ichino by methods that we will not describe here.

\section{The Global Conjectures}


Through this section we fix a quadratic extension $k'/k$ of number fields.
We denote the set of places of $k$ as $|k|$ and for all $v \in |k|$ the corresponding completion as $k_v$.
We also have $k_v' = k' \times_k k_v$.
Thus $k_v' \simeq k_v \times k_v$ if $v$ splits in $k'$ and $k_V'$ is a unique quadratic extension of $k_v$ otherwise.
Let $\bA$ the ring of ad\`eles of $k$.
Recall that it is a restricted product of $k_v$ over all places $v \in |k|$ which is the set of families $(x_v)_{v\in |k|}$ with $x_v \in k_v$ for all $v$ and $x_v \in \scO_v$ for almost all non-archimedean $v$ where $\scO_v$ is the ring of integers of $k_v$.
We have a diagonal embedding $k \hookrightarrow \bA$ and we will also denote $\bA_f$ the ring of finite adeles (i.e. the restricted product of non-archimedean completions of $k$) and $k_\infty = k \otimes_\bQ \bR \simeq \prod_{v|\infty}k_v$ the archimedean part of $\bA$.
We denote by $\sgn_{k'/k}$ the quadratic character of the id\`ele class group $\bA^\times / k^\times$ associated with, by the class field theory, the extension $k'/k$.
For all $v \in |k|$, the restriction of $\sgn_{k'/k}$ to $k_v^\times$ coincides with $\sgn_{k_v'/k_v}$ (and it is trivial if $v$ splits in $k'$).
Let $W \subset V$ be two hermitian or skew-hermitian spaces over $k'$ with
\[
    \dim(V) - \dim(W) = \begin{cases} 1 & \text{in the hermitian case} \\ 0 & \text{in the skew-hermiitian case},
    \end{cases}
\]
and set $G = \rU(W) \times \rU(V)$, $H = \rU(W)$ (algebraic groups defined over $k$).
As in the local case we have a ``diagonal'' inclusion $H \hookrightarrow G$ and the group of adelic points $G(\bA)$ is a locally compact group admitting the following more explicit description.
Fix a model of $G$ on $\scO_k[1/N]$ for some integer $N \geq 1$ where $\scO_k$ denotes the ring of algebraic integers of $k$.
Then for almost all place $v \in |k|$, the group of points $G(\scO_v)$ of this model over $\scO_v$ is a maximal compact subgroup of $G(k_v)$ and $G(\bA)$ is the restricted product of $G(k_V)$, $v \in |k|$ with respect to $G(\scO_v)$ which is the set of families $(g_v)_{v \in |k|}$ with $g_v \in G(k_v)$ for all $v \in |k|$ and $g_v \in G(\scO_v)$ for almost all $v \in |k|$.
A similar description obviously applies to $H(\bA)$.


\subsection{Automorphic forms and periods}

Recall that the automorphic forms on $G(\bA)$ is a function $\varphi: G(\bA)\to \bC$ satisfying the following conditions:
\begin{itemize}
    \item[--] $\varphi$ is left invariant by $G(k)$.
    \item[--] $\varphi$ is right invariant by an open compact subgroup $K_f \subset G(\bA_f)$.
    \item[--] For all $g \in G(\bA)$ the function $g_\infty \in G(k_\infty) \mapsto \varphi(gg_\infty)$ is $C^\infty$, in particular we have an action of Lie algebra $\frg_\infty$ of $G(k_\infty)$ on $\varphi$ by $(X.\varphi)(g) = \frac{\dd}{\dd t}\varphi(g e^{tX})|_{t=0}$ that extends to the complexified universal enveloping algebra $\scU(\frg_\infty)$.
    \item[--] $\varphi$ satisfies a certain \emph{moderate growth} at infinity (cf. \cite{moeglin1994decomposition} \S 1.2.3).
\end{itemize}

We denote by $\scA(G)$ the space of automorphic forms over $G(\bA)$.
Let us point out that the definition above differs from the one usually admitted which imposes an additional condition of $K_\infty$-finitness where $K_\infty$ is a maximal compact subgroup of $G(k_\infty)$ fixed in advance.
This definition has however the advantage of providing a stable space $\scA(G)$ under the right translation action of $G(\bA)$ (whereas with the usual definition one only obtains a structure of $(\frg_\infty, K_\infty)$-module at archimedean places) and seems more natural for the questions we are going to discuss.
We denote $\scA_\cusp(G) \subset \scA(G)$ the subspace of cusp forms, i.e. automorphic forms which are rapidly decreasing as well as all their derivatives (in a sense which we will not make precise here).
This space is stable under the action by the right translation of $G(\bA)$ and admits a decomposition
\[
    \scA_\cusp(G) = \bigoplus_\pi \pi
\]
as a direct sum of irreducible representations of $G(\bA)$.
Each of these irreducible representations decomposes as a restricted tensor product $\pi = \otimes_{v \in |k|}' \pi_v$.
More precisely, there exists a family of smooth irreducible representations $(\pi_v)_{v \in |k|}$ of local groups $G(k_v)$ that are unramified at almost all places $v \in |k|$ (which means $\pi_v^{G(\scO_v)} \neq 0$ and it is one-dimensional by the \emph{Satake isomorphism}) and the nonzero vectors $\varphi_v^\circ \in \pi_v^{G(\scO_v)}$ such that $\pi$ is isomorphic to the natural representation of $G(\bA)$ on
\[
    \lim_{\substack{\longrightarrow \\ S}} \bigotimes_{v \in S}\pi_v
\]
where the limit is over the sufficiently ``large'' finite sets of places of $k$ (i.e. containing the archimedean places and the ramified finite places) and the transition maps $\bigotimes_{v \in S} \pi_v \to \bigotimes_{v \in T} \pi_v$ for $S \subset T$ are defined by $\varphi_S \mapsto \varphi_S \otimes \bigotimes_{v \in T\backslash S} \varphi_v^\circ$ (to be precise, one needs to consider topological tensor products at archimedean places).

The constructions of section 1.2 provide for every $v$ a representation $\nu_v$ of the local group $H(k_v)$ (the trivial representation in the hermitian case and a certain Weil representation in the skew-hermitian case).
We can form their restricted tensor product $\nu = \otimes_v' \nu_v$ and it turns out that there exists a natural realization $\nu \hookrightarrow \scA(H)$ in the space of automorphic forms on $H$ (in the hermitian case the trivial representation is realized
as the space of constant functions on $H(k) \backslash H(\bA)$ while in the skew-hermitian case $\nu$ is a certain global Weil representation and the embedding $\nu \hookrightarrow \scA(H)$ is obtained via the
theta series).
In particular, in the skew-hermitian case we must choose local data $(\psi_{0, v}, \mu_v)$ for all $v \in |k|$ as in 1.2 in order to specify the representations $\nu_v$.
To obtain the embedding $\nu \hookrightarrow \scA(H)$ we must assume  that these local data come by localization of global characters $\psi_0: \bA/k \to \bC^\times$ and $\mu: \bA^\times_{k'} / (k')^\times \to \bC^\times$.

We define an \emph{automorphic period}
\[
    \cP_H: \scA_\cusp(G) \otimes \overline{\nu} \to \bC,
\]
where $\overline{\nu}$ denotes the complex conjugation of the realization of $\nu$ in $\scA(H)$, by
\[
    \cP_H(\varphi \otimes \overline{\theta}) := \int_{H(k) \backslash H(\bA)} \varphi(h) \overline{\theta(h)} \dd h
\]
for all $\varphi \in \scA_\cusp(G)$ and $\overline{\theta} \in \overline{\nu}$.
The integral is absolutely convergent due to the rapid decrease of $\varphi$ and the measure $\dd h$ for which we are integrating is a $H(\bA)$-invariant measure which can be obtained as the quotient of a Haar measure on $H(\bA)$ because $H(k)$ is a discrete subgroup.
For the global Gan--Gross--Prasad conjecture the specific choice of this Haar measure does not matter because we are only interested in questions of non-vanishing of the period.
On the other hand, this choice matters for the Ichino--Ikeda conjecture which predicts an explicit formula for (the square of) the period above.
Note that in the hermitian case, the representation $\nu$ being trivial, the period $\cP_H$ is simply the linear form $\scA_\cusp(G) \to \bC$ given by
\[
    \cP_H(\varphi) = \int_{H(k) \backslash H(\bA)} \varphi(h) \dd h.
\]
In any case, if $\pi \subset \scA_\cusp(G)$ is a cuspidal irreducible representation, the restriction of the period $\cP_H$ to $\pi$ defines an element of the space of intertwining maps
\[
    \Hom_{H(\bA)}(\pi \otimes \overline{\nu}, \bC) = \Hom_{H(\bA)}(\pi, \nu)
\]
which decomposes into a (restricted) tensor product of the local spaces of intertwining maps $\Hom_{H_v}(\pi_v, \nu_v)$ for all $v \in |k|$.
Thus, a necessary condition for this restriction to be nonzero is that these local spaces is nontrivial (a condition which is itself made explicit by the local conjecture).


\subsection{Automorphic $L$-functions and base change}

Let $\pi \subset \scA_\cusp(G)$ be an irreducible cuspidal representation that we assume to be almost everywhere generic, i.e. the representation $\pi_v$ is generic in the sense of \S 1.4.8 for almost all $v \in |k|$.
We also say that $\pi$ is of \emph{Ramanujan type} because these are the representations for which we hope to have a generalized Ramanujan conjecture (i.e. $\pi_v$ is tempered for all $v \in |k|$).
This assumption also allows us to define global $L$-functions only in terms of the local Langlands correspondence (whereas in general one should consider more general packets of representations called \emph{Arthur packets}).
Moreover, the two conjectures that will interest us only relate, for the moment, to this type of cuspidal representation.

Let $\rho: {}^L G \to \GL(M)$ be an algebraic representation of the $L$-group of G.
According to the section 1.4.6, we can associate, at any place $v \in |k|$, to the representation $\pi_v$ a local $L$-function $L(s, \pi_v, \rho)$.
We then define a global function $L(s, \pi, \rho)$ by
\[
    L(s, \pi, \rho) = \prod_{v \in |k|} L(s, \pi_v, \rho).
\]
The product converges for $\Re(s) \gg 1$ and conjecturally $L(s, \pi, \rho)$ admits a meromorphic continuation to the complex plane and a functional equation relating $L(s, \pi, \rho)$ to $L(1 - s, \pi, \rho^\vee)$.
In what follows, only two global $L$-functions will appear.
The first, which plays a minor role, is the adjoint $L$-function $L(s, \pi, \Ad)$ associated to the adjoint representation of ${}^L G$.
The second $L$-function, the one that will interest us the most, is associated to the following representation $R$ of ${}^L G$:
\[
    R = \begin{cases} \Ind_{\widehat{G} \times W_{k'}}^{{}^L G}(M_W \otimes M_V) & \text{hermitian case}; \\
        \Ind_{\widehat{G} \times W_{k'}}^{{}^L G} ((M_W \otimes M_V) \otimes \mu^{-1}) & \text{skew-hermitian case},
    \end{cases}
\]
where $M_W$ and $M_V$ denote the standard representations of $\widehat{\rU(W)}$ and $\widehat{\rU(V)}$ (recall that these are linear groups over $\bC$) respectively and we have identified $\mu$ with a character of the Weil group $W_k$ via the class field theory.
We can also describe the local factors of this $L$-function in the following explicit way.
Since $G = \rU(W) \times \rU(V)$, we can decompose $\pi$ as a tensor product $\pi = \pi_1 \boxtimes \pi_2$ where $\pi_1$ (resp. $\pi_2$) is an irreducible cuspidal representation of $\rU(W)$ (resp. $\rU(V)$).

According to the section 1.4.2, for all $v \in |k|$ not split in $k'$ the Langlands parameters of $\pi_{1, v}$ and $\pi_{2, v}$ can be identified with conjugate-dual representations $\varphi_{1, v}: \WD_{k_v'} \to \GL(M_1)$ and $\varphi_{2, v}: \WD_{k_v'} \to \GL(M_2)$ of a certain sign.
Then we have $L(s, \pi_v, R) = L(s, \varphi_{1, v} \otimes \varphi_{2, v})$ in the hermitian case and $L(s, \pi_v, R) = L(s, \varphi_{1, v} \otimes \varphi_{2, v} \otimes \mu_v^{-1})$ in the skew-hermitian case.
For a place $v$ splits in $k'$, the groups $\rU(W)_v$ and $\rU(V)_v$ are linear groups over $k_v$ and the Langlands parameters of $\pi_{1, v}$ and $\pi_{2, v}$ can be identified with the representations $\varphi_{1, v}$ and $\varphi_{2, v}$ of $\WD_{k_v}$.
Then we have $L(s, \pi_v, R) = L(s, \varphi_{1, v} \otimes \varphi_{2, v}) L(s, \varphi_{1, v}^\vee \otimes \varphi_{2, v}^\vee)$ in the hermitian case and $L(s, \pi_v, R) = L(s, \varphi_{1, v} \otimes \varphi_{2, v} \otimes \mu_{v}') L(s, \varphi_{1, v}^\vee \otimes \varphi_{2, v}^\vee \otimes (\mu_v')^{-1})$ in the skew-hermitian case where we write $\mu_v = (\mu_v')^{-1} \boxtimes \mu_v'$ under the identification $k_v' \simeq k_v \times k_v$.

By the results of Mok \cite{mok2015endoscopic} and Kaletha--Minguez--Shin--White \cite{kaletha2014endoscopic} on the classification of automorphic representations of unitary groups as well as the works of Jacquet, Piatetski-Shapiro and Shalika \cite{jacquet1983rankin} and Shahidi on the $L$-functions of pairs and Asai for linear groups respectively, we know that $L(s, \pi, \Ad)$ and $L(s, \pi, R)$ admit analytic continuations to $\bC$ and satisfy the expected functional equations.
To be more precise, according to \cite{mok2015endoscopic} and \cite{kaletha2014endoscopic} there exist, with the above notations, irreducible automorphic representations $\BC(\pi_1)$ and $\BC(\pi_2)$ ($\BC$ for \emph{base change}) of $\GL_{d_W}(\bA_{k'})$ and $\GL_{d_V}(\bA_{k'})$ respectively (where we denote $\bA_{k'}$ for the ad\`eles over $k'$, $d_W = \dim(W)$ and $d_V = \dim(V)$) whose local components at $v \in |k|$ have Langlands parameters $\varphi_{1, v}$, $\varphi_{2, v}$ respectively if $v$ does not split in $k'$ and $\varphi_{1, v} \times \varphi_{1, v}^\vee$ and $\varphi_{2, v} \times \varphi_{2, v}^\vee$ respectively if $v$ splits in $k'$ (which is the case $\GL_{d_W}(k_v') \simeq \GL_{d_W}(k_v) \times \GL_{d_W}(k_v)$ and $\GL_{d_V}(k_v') \simeq \GL_{d_V}(k_v) \times \GL_{d_V}(k_v)$).
Moreover, $L(s, \pi, R)$ coincides with the $L$-function of the pair $L(s, \BC(\pi_1) \times \BC(\pi_2))$ defined by Jacquet, Piatetski-Shapiro, and Shalika \cite{jacquet1983rankin} in the hermitian case and with $L(s, \BC(\pi_1) \times \BC(\pi_2) \otimes \mu^{-1})$ in the skew-hermitian case while $L(s, \pi, \Ad)$ coincides with the product of Asai $L$-functions of $\BC(\pi_1)$ and $\BC(\pi_2)$ defined by Shahidi.
Note that $\BC(\pi) = \BC(\pi_1) \boxtimes \BC(\pi_2)$.
It is an automorphic representation of $G(\bA_{k'}) \simeq \GL_{d_W}(\bA_{k'}) \times \GL_{d_V}(\bA_{k'})$ which is called the (quadratic) \emph{base change} of $\pi$, and we set $L(s, \BC(\pi)) = L(s, \BC(\pi_1) \times \BC(\pi_2))$.
Thus, in the hermitian case we simply have
\[
    L(s, \pi, R) = L(s, \BC(\pi)).
\]


\subsection{The conjecture}

We now have all the ingredients to state the global Gan--Gross--Prasad conjecture.


\begin{conjecture}[Gan--Gross--Prasad]
\label{conj:global}
Let $\pi \subset \scA_\cusp(G)$ be a cuspidal irreducible almost everywhere generic representation.
Then the following are equivalent:
\begin{enumerate}
    \item The restriction of the period $\cP_H$ to $\pi$ is nonvanishing.
    \item We have $L(\frac{1}{2}, \pi, R) \neq 0$ and for all $v \in |k|$ we have $\Hom_{H_v}(\pi_v, \nu_v) \neq 0$.
\end{enumerate}
\end{conjecture}


\subsection{The refinement of Ichino--Ikeda}


There is a refinement of conjecture 2.1 in the form of an identity directly linking $L(\frac{1}{2}, \pi, R)$ to the period $\cP_H$.
This conjecture is due to Ichino and Ikeda \cite{ichino2010periods} in the case of orthogonal groups and was extended to unitary groups by N. Harris \cite{harris2014refined} (in the hermitian case) and
H. Xue \cite{xue2017refined} (in the skew-hermitian case).
To simplify the exposition, we will limit ourselves here to the Bessel case of the conjecture (i.e. $W$ and $V$ are hermitian spaces).


Let $v \in |k|$ and $\pi_v$ a smooth irreducible tempered representation of $G(k_v)$.
In particular, $\pi_v$ is unitary and we can fix a $G(k_v)$-invariant inner product $(-,-)_v$ on (the space of) $\pi_v$.
Thanks to this inner product, we can define a \emph{local period}
\[
    \cP_{H, v}: \pi_v \times \pi_v \to \bC
\]
by
\[
    \cP_{H, v}(\varphi_v, \varphi_v') = \int_{H(k_v)} (\pi_v(h_v) \varphi_v, \varphi_v')_v \dd h_v, \quad \varphi_v, \varphi_v' \in \pi_v
\]
where $\dd h_v$ is a Haar measure on $H(k_v)$.
The above integral is absolutely convergent by the assumption that $pi_v$ is tempered and moreover $\cP_{H, v}$ induces a $H(k_v) \times H(k_v)$-invariant hermitian form on $\pi_v$.
In particular, if the local period $\cP_{H, v}$ is nonzero on $\pi_v$ then $\Hom_{H(k_v)}(\pi_v, \bC) \neq 0$.
An important step in the proof of the local Gan--Gross--Prasad conjecture is to show the reverse implication (cf. \cite{beuzart2015local} Theorem 8.2.1): $\cP_{H, v}$ is nonzero on $\pi_v$ if and only if $\Hom_{H(k_v)}(\pi_v, \bC) \neq 0$.

Now let $\pi \subset \scA_\cusp(G)$ be an irreducible cuspidal representation.
We can equip the ad\`elic groups $G(\bA)$ and $H(\bA)$ with canonical Haar measures $\dd g_\Tam$, $\dd h_\Tam$ called Tamagawa measures (cf. \cite{weil2012adeles} Chap.II).
We normalize the global period $\cP_H$ through the Tamagara measure on $H(\bA)$ and we normalize the local periods $\cP_{H, v}$ by the local measures which factorizes the Tamagawa measure:
\[
    \dd h_\Tam = \prod_{v \in |k|} \dd h_v.
\]
We endow $\pi$ with the following inner product (the \emph{Petersson} inner product)
\[
    (\varphi_1, \varphi_2) \in \pi \times \pi \mapsto \langle \varphi_1, \varphi_2 \rangle_\Pet = \int_{G(k) \backslash G(\bA)} \varphi_1(g) \overline{\varphi_2(g)} \dd g_\Tam
\]
and we choose the local inner products $(-, -)_v$, $v \in |k|$, so they factor the global inner product $\langle-, -\rangle_\Pet$:
\[
    \langle \varphi, \varphi \rangle_\Pet = \prod_{v \in |k|} (\varphi_v, \varphi_v)_v, \quad \forall \varphi = \otimes_{v}' \varphi_v \in \pi = \otimes_v' \pi_v.
\]
We associate to $\pi$ the following quotient of $L$-functions
\[
    \scL(s, \pi) := \Delta_{n+1} \frac{L(s, \BC(\pi))}{L(s + 1/2, \pi, \Ad)}
\]
where
\[
    \Delta_{n+1} := \prod_{i=1}^{n+1} L(i, \sgn_{k'/k}^i)
\]
is a finite product of special values of Hecke $L$-functions.
We define a local analogue $\scL(s, \pi_v)$ of $\scL(s, \pi)$ for all place $v \in |k|$.
Suppose $\pi$ is tempered everywhere, which means that $\pi_v$ is tempered for all $v \in |k|$.
Then, for almost all place $v \in |k|$ and for all vector $\varphi_v^\circ \in \pi_v^{G(\scO_v)}$, we have (\cite{harris2014refined} Theorem 2.12)
\[
    \cP_{H, v}(\varphi_v^\circ, \varphi_v^\circ) = \scL\left(\frac{1}{2}, \pi_v \right) \vol(H(\scO_v)) (\varphi_v^\circ, \varphi_v^\circ)_v.
\]
This leads us to define the \emph{normalized local periods} by
\[
    \cP_{H, \pi_v}^{\natural} := \scL\left(\frac{1}{2}, \pi_v\right)^{-1} \cP_{H, v}|_{\pi_v}, \quad v\in |k|.
\]
Now we can state the Ichino--Ikeda conjecture for hermitian unitary groups in the following slightly informal way:
\begin{conjecture}
\label{conj:ichinoikeda}
Let $\pi \subset \scA_\cusp(G)$ be a cuspidal irreducible tempered representation.
For all factorizable vector $\varphi = \otimes_v' \varphi_v\in\pi$, we have
\[
    |\cP_H(\varphi)|^2 = \frac{1}{|S_\pi|} \scL\left(\frac{1}{2}, \pi\right) \prod_{v \in |k|} \cP_{H, \pi_v}^\natural(\varphi_v, \varphi_v)
\]
where $S_\pi$ is the centralizer of the ``Langlands parameter'' of $\pi$ (this is a global analogue of the group $S_\phi$ introduced in section 1.4.2 which we will not try to define here).
\end{conjecture}

\begin{remark}
\begin{itemize}
    \item[--] By the equivalence mentioned
    \[
        \cP_{H, v}|_{\pi_v} \neq 0 \Leftrightarrow \Hom_{H(k_v)}(\pi_v, \bC) \neq 0,
    \]
    and since $L(s, \pi, \Ad)$ has no pole at $s = 1$, the above conjecture implies conjecture \ref{conj:global} in the case where $\pi_v$ is tempered everywhere.
    \item[--] By the generalized Ramanujan conjecture we expect to be able to replace the hypothesis ``$\pi$ is tempered everywhere'' by the hypothesis ``$\pi$ is almost everywhere generic''.
    However, even if the generalized Ramanujan conjecture is far from being established in full generality, one can state a similar conjecture under the (\emph{a priori}) weaker hypothesis ``$\pi$ is almost everywhere generic'' to extend the definition of normalized local periods $\cP_{H, \pi_v}^\natural$ (which is no longer \emph{a priori} defines absolutely convergent integrals) by a certain ``analytic continuation''.
    Existence of such extension follows from the main results of \cite{beuzart2021comparison} and \cite{beuzart2017factorisations}.
    In this more general form (which we will not state) conjecture \ref{conj:ichinoikeda} then becomes strictly stronger than conjecture \ref{conj:global}.
    \item[--] In the case where $\dim(W) = 1$ (and therefore $\dim(V) = 2$), conjecture \ref{conj:ichinoikeda} is essentially equivalent to a well-known formula of Waldspurger \cite{waldspurger1985valeurs} for toric periods on quaternion algebras (see \cite{harris2014refined} Sect.3 for a formal reduction) of which we gave an example in the introduction.
    It also seems that this formula of Waldspurger was one of the main inspirations of Ichino and Ikeda to state their conjecture.
\end{itemize}
\end{remark}

\subsection{Status}



Even before the Gan--Gross--Prasad conjectures were extended to all classical groups in \cite{gan2011symplectic}, Ginzburg, Jiang and Rallis showed in a series of papers \cite{ginzburg488models,ginzburg2004nonvanishing,ginzburg2005nonvanishing} the implication $1 \Rightarrow 2$ of the conjecture \ref{conj:global} under the assumption that $\pi$ is globally generic, i.e. there exists a global Whittaker datum $(N, \theta)$ of $G$ with $\theta$ trivial on $N(k)$ and for $\varphi \in \pi$ we have $\int_{N(k) \backslash N(\bA)} \varphi(u) \theta(u) \dd u \neq 0$, in particular this implies that the group $G$ is quasi-split (because otherwise $G$ has no Whittaker datum).
It is not clear whether the Ginzburg--Jiang--Rallis approach could be able to establish the reverse implication (at least in the globally generic case).

As we have already indicated, in the hermitian case and where $\dim(W) = 1$ conjectures \ref{conj:global} and \ref{conj:ichinoikeda} arise from the Waldspurger formula \cite{waldspurger1985valeurs}.
The same Waldspurger formula also covers the analogues of these conjectures for a pair of orthogonal special groups $\SO(W) \subset \SO(V)$ with $\dim(W) = 2$ and $\dim(V) = 3$.
For the special orthogonal groups of the case $\dim(W) = 3$ and $\dim(V) = 4$, the analogue of conjecture \ref{conj:ichinoikeda} has been established by Ichino \cite{ichino2008trilinear} (then the $L$-function  appears in the numerator of the right hand side of the formula is essentially associated with the triple product of three cuspidal representations of $\PGL_2$).

More recently, following an approach proposed by Jacquet--Rallis \cite{jacquet2011gross} for comparing relative trace formulas and thanks to the fundamental lemma proved by Z. Yun \cite{yun2011fundamental}, W. Zhang proved conjecture \ref{conj:global} in the hermitian case under the following simplifying assumptions (\cite{zhang2014fourier} Theorem 1.1):
\begin{itemize}
    \item[--] All the archimedean places of $k$ split in $k'$ .
    \item[--] There are two distinct non-archimedean places $v_0, v_1 \in |k|$ splits in $k'$ such that $\pi_{v_0}$ is supercuspidal and $\pi_{v_1}$ is tempered.
\end{itemize}

Subsequent work by H. Xue \cite{xue2019global} on the one hand and by Chaudouard--Zydor \cite{chaudouard2021transfert} on the other hand now allows all these restrictions to be removed except the existence of a split place $v \in |k|$ in which $\pi_v$ is supercuspidal.
This last hypothesis seems inevitable by Zhang's method which uses simple versions of the Jacquet--Rallis trace formulas.
The ongoing work of M. Zydor and P.-H. Chaudouard to establish complete versions of these relative trace formulas (work already initiated in \cite{zydor2016variante,zydor2018variante,zydor2020formules}) should make it possible to remove this last hypothesis.


Following his first paper, Zhang \cite{zhang2014automorphic} tackled the conjecture \ref{conj:ichinoikeda}.
His main result is a proof of this conjecture essentially under the following assumptions (we leave aside a technical detail):
\begin{itemize}
    \item[--] Every archimedean place of $k$ split in $k'$.
    \item[--] There is a non-archimedean place $v\in |k|$ split in $k'$ where $\pi$ is supercuspidal.
    \item[--] For any place $v \in |k|$ not split in $k'$ either $\pi_v$ is supercuspidal or unramified.
\end{itemize}


This third hypothesis is much more restrictive than the others and comes from the following problem: in addition to the global comparison of relative trace formulas proposed by Jacquet and Rallis, Zhang must also compare certain local periods which he can only do for unramidied or supercuspidal representations at non-split places.
In \cite{beuzart2021comparison}, the author managed to extend this comparison to all tempered representations at non-archimedean places, which finally allows us to remove the third hypothesis.
Furthermore, in a work currently being written \cite{beuzart2017factorisations}, the author obtained, by another method, the desired comparison between local periods at archimedean places modulo an indeterminate sign which should make it possible to establish without the first hypothesis the formula conjectured by Ichino--Ikeda up to a single sign.
Finally, let us point out that modulo this sign problem the current work of Chaudouard--Zydor should also make it possible to remove the second hypothesis.

Following the work of Zhang, conjectures \ref{conj:global} and \ref{conj:ichinoikeda} have also been partially established in the Fourier--Jacobi case (i.e. skew-hermitian).
More precisely, in \cite{liu2014relative} Y. Liu proposed a comparison of relative trace formulas analogous to that of Jacquet--Rallis to attack these conjectures.
Following Zhang's method, Hang Xue \cite{xue2014gan,xue2016fourier,xue2017fourier} was then able to demonstrate conjectures \ref{conj:global} and \ref{conj:ichinoikeda} in the Fourier-Jacobi case, under the same hypothesis of the existence of two distinct split places in which $\pi$ is supercuspidal and temperate respectively (which allows him, like Zhang, to consider only simple forms of relative trace formulas).


\subsection{Sketch of the proof of the global conjecture in the hermitian case}


In this section, we give an outline of the proof by Zhang \cite{zhang2014fourier} of the Gan--Gross--Prasad conjecture (Conjecture \ref{conj:global}) under certain local assumptions in the hermitian case.
In particular, we will not talk about the proof of the Ichino--Ikeda conjecture (Conjecture \ref{conj:ichinoikeda}) nor of the skew-hermitian case.

\subsubsection{Jacquet--Rallis' approach}
In \cite{jacquet2011gross} Jacquet and Rallis propose an approach to conjecture \ref{conj:global} via a comparison of ``relative trace formulas''.
The latter, which generalize the famous Arthur--Selberg trace formula, were introduced and studied in many cases by Jacquet and his co-authors (see \cite{lapid2006relative} for an introduction to this subject).
In its purest version, a relative trace formula consists of expressing in two ways an integral of the following form
\begin{equation}
\label{eqn:8}
    \int_{H_1(k) \backslash H_1(\bA) \times H_2(k) \backslash H_2(\bA)} K_f(h_1, h_2)\eta_1(h_1) \eta_2(h_2) \dd h_1 \dd h_2
\end{equation}
where $G_0$ is a connected reductive group over $k$, $H_1$ and $H_2$ are algebraic subgroups of $G_0$ defined over $k$, $\eta_1: H_1(k) \backslash H_1(\bA) \to \bC^\times$ and $\eta_2: H_2(k) \backslash H_2(\bA) \to \bC^\times$ are automorphic characters, $f$ is a compactly supported function on $G_0(\bA)$ and
\[
    K_f(x, y) = \sum_{\gamma \in G_0(k)} K_f(x^{-1} \gamma y), \quad x,  y \in G_0(\bA)
\]
is  the kernel of the action by convolution to the right of $f$ on $L^2(G_0(k) \backslash G_0(\bA))$.
The trace formulas introduced by Jacquet and Rallis correspond to the following two cases:

\begin{itemize}
    \item[--] $G_0 = G = \rU(W) \times \rU(V)$ where $W \subset V$ are hermitian spaces over $k'$ with $\dim(W) = \dim(V) - 1$, $H_1 = H_2 = H = \rU(W)$ (equipped with the diagonal inclusion $H \hookrightarrow G$) and $\eta_1$, $\eta_2$ are trivial.
    \item[--] $G_0 = G' = \Res_{k'/k} \GL_n \times \Res_{k'/k} \GL_{n+1}$ where $\Res_{k'/k}$ denotes the Weil restriction of scalar and $n = \dim(W)$, $H_1 = H_1' = \Res_{k'/k} \GL_n$ equipped with the natural inclusion $H_1' \hookrightarrow G'$, $H_2 = H_2' = \GL_n \times \GL_{n+1}$ equipped with the natural inclusion $H_2' \hookrightarrow G'$, $\eta_1$ trivial and $\eta_2 = \eta$ a certain quadratic character of $H_2'(\bA)$.
\end{itemize}


Even in these particular cases, the expression \eqref{eqn:8} can't be done as is because the integral is in general divergent and it must be regularized (for example by introducing truncations as Arthur does).
To circumvent this problem, Zhang only considers ``good'' functions $f$ for which expression \eqref{eqn:8} is absolutely convergent and which also allow him to obtain absolutely convergent geometric and spectral expansions.
This is called a \emph{simple} trace formula (because of the restriction on test functions).
Let $J(f)$ and $I(f')$ be the Jacquet--Rallis relative trace formulas applied to ``good'' functions $f \in C_c^\infty(G(\bA))$ and $f' \in C_c^\infty(G'(\bA))$ respectively.
Then simple formal manipulations which are justified by the choice of ``good'' test functions gives the equalities
\begin{align}
    \sum_{\gamma \in H(k) \backslash G_\rs(k) / H(k)}  \cO(\gamma, f) &= J(f) = \sum_{\pi \subset \scA_\cusp(G)} J_\pi(f) \label{eqn:9} \\
    \sum_{\delta \in H_1'(k) \backslash G_\rs'(k) / H_2'(k)} \cO(\delta, f') &=  I(f') = \sum_{\substack{\Pi \subset \scA_\cusp(G') \\ \omega_\Pi|_{\bA^\times \times \bA^\times = 1}}} I_\Pi(f') \label{eqn:10}
\end{align}
where
\begin{itemize}
    \item[--] $G_\rs \subset G$ denotes the Zariski open subset of regular semisimple elements for the action by bimultiplication of $H \times H$ i.e. $g \in G_\rs$ if and only if the double coset $HgH$ is closed under the Zariski topology and $g^{-1} Hg \cap H = \{1\}$ (that is, the stabilizer of $g$ in $H \times H$ is trivial).
    We define in the same way the open subset $G_\rs' \subset G'$ of regular semisimple elements for the action of $H_1' \times H_2'$ by bimultiplication.
    \item[--] For $\gamma \in G_\rs(k)$,
    \[
        \cO(\gamma, f) = \int_{H(\bA) \times H(\bA)} f(h_1^{-1} \gamma h_2) \dd h_1 \dd h_2
    \]
    denotes the corresponding relative orbital integral.
    For $\delta \in G_\rs'(k)$, we define $\cO(\delta, f')$ in a similar way by twisting by the character $\eta$ on $H_2'(\bA)$.
    \item[--] The right summation of the first formula is over all irreducible cuspidal representations of $G(\bA)$ and the right summation of the second formula is over the set of irreducible cuspidal representations $\Pi$ of $G(\bA)$ whose central character (here we denote $\omega_\Pi$) is trivial on $\bA^\times \times \bA^\times = Z_{H_2'}(\bA)$ (the center of $H_2'(\bA)$).
    \item[--] The distributinos $f \mapsto J_\pi(f)$ and $f'\mapsto I_\Pi(f')$ are the \emph{relative characters} defined by
    \begin{align*}
        J_\pi(f) &= \sum_{\varphi \in \scB_\pi} \cP_H(\pi(f) \varphi) \overline{\cP_H(\varphi)} \\
        I_\Pi(f') &= \sum_{\varphi \in \scB_\Pi} \cP_{H_1'}(\Pi(f')\varphi) \overline{\cP_{H_2', \eta}(\varphi)}
    \end{align*}
    where $\scB_\pi$, $\scB_\Pi$ denote (good) orthonormal bases of $\pi$, $\Pi$ respectively for Peterssen scalar products, $\cP_H$ is the Gan--Gross--Prasad period, $\cP_{H_1'}$ (resp. $\cP_{H_2', \eta}$) is the period that maps a cusp form to the integral over $H_1'(k) \backslash H_1'(\bA)$ (resp. over $Z_{H_2'}(\bA) H_2'(k) \backslash H_2'(\bA)$ against the character $\eta$).
\end{itemize}


The relative characters $H_\pi$ and $I_\Pi$ are immediately related to the periods $\cP_H$ and $\cP_{H_1'}$, $\cP_{H_2', \eta}$.
In fact, we can show without too much difficulty that
\begin{align*}
    J_\pi \neq 0 &\Leftrightarrow \cP_H|_\pi \neq 0 \\
    I_\Pi \neq 0 &\Leftrightarrow \cP_{H_1'}|_{\Pi} \neq 0 \text{ and } \cP_{H_2', \eta}|_\Pi \neq 0.
\end{align*}

According to the work of Rallis and Flicker \cite{flicker1988twisted} on the period $\cP_{H_2', \eta}$ and the classification of the 
automorphic representations of unitary groups by Mok \cite{mok2015endoscopic} and Kaletha--Minguez--Shin--White \cite{kaletha2014endoscopic}, the period $\cP_{H_2', \eta}|_\Pi$ is nonzero if and only if $\Pi$ comes from a basis change of a product of unitary groups $\rU(W') \times \rU(V')$.
On the other hand, according to the work of
Jacquet, Piatetski-Shapiro and Shalika \cite{jacquet1983rankin} on the Rankin--Selberg convolution, the period $\cP_{H_1'}|_\Pi$ is non-zero if and only if $L(\frac{1}{2}, \Pi) \neq 0$ (where, recall that $L(s, \Pi)$ is the $L$-function of the pair).

To prove the conjecture \ref{conj:global}, it is enough to show that
\[
    J_\pi \neq 0 \Leftrightarrow I_{\BC(\pi)} \neq 0
\]
and the strategy proposed by Jacquet-Rallis to establish this equivalence consists of comparing formulas \eqref{eqn:9} and \eqref{eqn:10}.
More precisely, it involves comparing the geometric sides (i.e. the left hand sides) to deduce an identity between the spectral sides (i.e. the right hand sides).
To this end, Jacquet and Rallis begin by defining an injection
\[
    H(k) \backslash G_\rs(k) / H(k) \hookrightarrow H_1'(k) \backslash G_\rs'(k) / H_2'(k)
\]
and more generally
\begin{equation}
    \label{eqn:11}
    H(F) \backslash G_\rs(F) / H(F) \hookrightarrow H_1'(F) \backslash G_\rs'(F) / H_2'(F)
\end{equation}
for all extension $F$ of $k$ (in particular the completions $k_v$ of $k$).
We will then say that two elements $(\gamma, \delta) \in G_\rs(F) \times G_\rs'(F)$ \emph{match} if $H_1'(F) \delta H_2'(F)$ is the image of $H(F) \gamma H(F)$ under \eqref{eqn:11}.
In order to compare formulas \eqref{eqn:9} and \eqref{eqn:10}, we look for functions $f \in C_c^\infty(G(\bA))$ and $f' \in C_c^\infty(G'(\bA))$ such that
\[
    \cO(\gamma, f) = \cO(\delta, f')
\]
for the matching pair of elements $(\gamma, \delta) \in G_\rs(F) \times G_\rs'(F)$.
When $f$ (resp. $f$) decomposes as a product $f = \prod_v f_v$ (resp. $f' = \prod_v f_v'$) of local functions $f_v \in C_c^\infty(G(k_v))$ (resp. $f_v' \in C_c^\infty(G'(k_v))$) (we have $f_v = \mathbf{1}_{G(\scO_v)}$, resp. $f_v' = \mathbf{1}_{G'(\scO_v)}$ for almost all $v \in |k|$), we have a corresponding factorization of orbital integrals:
\[
    \cO(\gamma, f) = \prod_v \cO(\gamma, f_v) \quad(\text{resp. } \cO(\delta, f') = \prod_v \cO(\delta, f_v'))
\]
and we can first try to compare the local orbital integrals $\cO(\gamma, f_v)$ and $\cO(\delta, f_v')$.
For this, Jacquet and Rallis introduce a family of ``transfer factors''
\[
    \Delta_v: G_\rs(k_v) \times G_\rs'(k_v) \to \bC
\]
for all $v \in |k|$ which are defined by explicit formulas and have the following essential properties:
\begin{itemize}
    \item[--] $\Delta_v(\gamma, \delta) = 0$ unless $\gamma$ and $\delta$ match.
    \item[--] $\Delta_v(h_1 \gamma h_2, h_1' \delta h_2') = \eta_v(h_2')\Delta_v(\gamma, \delta)$ for all $h_1, h_2 \in H(k_v)$, $h_1' \in H_1'(k_v)$, and $h_2' \in H_2'(k_v)$ and $\eta_v$ denotes the local component of $\eta$ at $v$.
    \item[--] If $\gamma \in G_\rs(k)$ and $\eta \in G_\rs'(k)$ match then
    \[
        \prod_{v \in |k|} \Delta_v(\gamma, \delta) = 1.
    \]
\end{itemize}

Following Jacquet and Rallis, we then say that two functions $(f_v, f_v') \in C_c^\infty(G(k_v)) \times C_c^\infty(G'(k_v))$ \emph{match} or they are \emph{transfer} of each other if
\[
    \cO(\gamma, f_v) = \Delta_v(\gamma, \delta) \cO(\delta, f_v')
\]
for all matching pair of elements $(\gamma, \delta) \in G_\rs(k_v) \times G_\rs'(k_v)$.
In order to construct sufficient pairs of global functions $(f, f')$ that allow us to compare formulas \eqref{eqn:9} and \eqref{eqn:10} effectively, we are naturally led to consider the following two local problems:
\begin{itemize}
    \item[--] Fundamental lemma: Show that $f = \mathbf{1}_{G(\scO_v)}$ and $f' = \mathbf{1}_{G'(\scO_v)}$ match for almost all place $v$.
    \item[--] Existence of transfer: Show that for any function $f_v \in C_c^\infty(G(k_v))$ there exists a matching function $f_v' \in C_c^\infty(G'(k_v))$ and vice versa.
\end{itemize}
These two statements are easy to prove at split places $v \in |k|$ hence the essential problem therefore lies in the non-split places.
Let us point out that once the comparison has been made, it is still necessary to separate the contributions from the spectral side (in order to obtain an identity directly relates $J_\pi$ and $I_{\BC(\pi)}$).
For this, in addition to the local conjecture (which ensures that given $\Pi$ there exists at most one cuspidal representation $\pi$ satisfying $\Hom_{H(k_v)}(\pi_v, \bC) \neq 0$ everywhere and $\Pi = \BC(\pi)$) we need \emph{a priori} an extended version of the fundamental lemma for all the elements of certain spherical Hecke algebras (this is what really allows us to separate the spectral contributions by applying Stone--Weierstrass theorem).
However, this fundamental lemma for spherical Hecke algebras is also very easy to establish at split places and Zhang remarked that this was sufficient to isolate all spectral terms from a recent result of Ramakrishnan \cite{ramakrishnan2015mild}.


\subsubsection{Work of Z. Yun}

In \cite{yun2011fundamental}, Zhiwei Yun establishes the Jacquet--Rallis fundamental lemma for function fields.
His proof uses geometric methods close to those introduced by Ngô Bao Châu in his thesis to demonstrate the fundamental lemma of Jacquet--Ye (which is the fundamental lemma resulting from another relative trace formula).
In the appendix of \cite{yun2011fundamental} and by methods of model theory, Julia Gordon shows how one can transfer this result to number fields and deduce from it the fundamental lemma in any place $v$ of sufficiently large residual characteristic.


\subsubsection{Work of W. Zhang}

The main result demonstrated by Zhang in \cite{zhang2014fourier} is the existence of transfer at non-archimedean places.
The main steps are as follows:

\begin{itemize}
    \item[--] Using the partition of unity and a decent inspired by Harish-Chandra, Zhang localizes the problem to the neighborhoods of semi-simple (not necessarily regular) elements $\gamma \in G(k_v)$ and $\delta \in G'(k_v)$.
    Then it shows that if $\gamma$ and $\delta$ are not central we reduce a transfer problem to smaller groups, which allow us to treat these cases inductively.

    \item[--] At neighborhoods of central elements, via Cayley transform, it suffices to consider an analogous problem on Lie algebras.
    More precisely, the problem is reduced to compare orbital integrals for the adjoint action of $\rU(V)$ on $\fru(V)$ (the Lie algebra of $\rU(V)$) with orbital integrals for the adjoint action of $\GL_{n+1}$ on $\frgl_{n+1}$.

    \item[--] The same descent method inspired by Harish-Chandra makes it possible to show the desired result for the functions on these Lie algebras with disjoint support of the nilpotent cone (i.e. the set of elements whose orbits contain the origin in their closure).

    \item[--] Zhang shows that if $f$ and $f'$ are functions on $\fru(V)$ and $\frgl_{n+1}$ respectively, whose orbital integrals match then so do some of their partial Fourier transforms up
    to an explicit multiplicative constant (in fact we must consider four Fourier transforms, including the identity).
    This is the heart of the proof and this step is based on local analogues for the Lie algebras of the Jacquet--Rallis trace formulas as well as an ingenious induction which uses the fact that the adjoint representations of $\rU(W)$ and $\GL_n$ on $\fru(V)$ and $\frgl_{n+1}$ respectively are reducible.
    
    \item[--] Finally, according to a result of Aizenbud \cite{aizenbud2013partial}, one can write any smooth function with compact support on Lie algebras as a sum of images of the various Fourier transforms of functions whose supports are disjoint  nilpotent cone and of a function whose orbital integrals are all zero.
    According to the previous steps, this concludes the proof of the existence of the transfer.
\end{itemize}


\subsubsection{Work of H. Xue and Chaudouard--Zydor}

In \cite{xue2019global}, H. Xue adapted Zhang's proof to archimedean places and obtained the existence of the transfer for a dense subspace of the space of test functions.
Indeed, in this case only the dual assertion of Aizenbud's result is known (i.e. there is no invariant distribution on the Lie algebra of which all the Fourier transforms have support in the nilpotent cone) which only implies that any test function is approximable by sums of Fourier transforms of functions with disjoint support of the nilpotent cone and of a function of zero orbital integrals.
Besides this difference, Xue's proof is similar to that in the non-archimedean case and allows remove Zhang's hypothesis at archimedean places.

At the same time, Zydor \cite{zydor2016variante,zydor2018variante} and Chaudouard--Zydor \cite{chaudouard2021transfert} began to develop Jacquet--Rallis trace formulas in complete generality (i.e. without restriction on the test functions).
As a corollary of their first results, we can apply and compare these trace formulas for ``good'' functions a little more general than those of Zhang, which in particular makes it possible to remove the hypothesis of the existence of a split place where the representation $\pi$ is tempered.

% --- Bibliography ---

% Start a bibliography with one item.
% Citation example: "\cite{williams}".

\nocite{*}
\bibliographystyle{acm} % We choose the "plain" reference style
\bibliography{refs} % Entries are in the refs.bib file


% \begin{thebibliography}{1}

% \bibitem{williams}
%    Williams, David.
%    \textit{Probability with Martingales}.
%    Cambridge University Press, 1991.
%    Print.

% % Uncomment the following lines to include a webpage
% % \bibitem{webpage1}
% %   LastName, FirstName. ``Webpage Title''.
% %   WebsiteName, OrganizationName.
% %   Online; accessed Month Date, Year.\\
% %   \texttt{www.URLhere.com}

% \end{thebibliography}

% --- Document ends here ---

\end{document}