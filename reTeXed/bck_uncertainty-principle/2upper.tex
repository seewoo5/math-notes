\section{Upper bound of $B_{1}$}

A first idea is to associate $f$ with the Hermite series
\[
    f(x) \sim \sum_{n = 0}^{\infty}a_n h_n(x)
\]
where $h_n$ are eigenvectors of the Fourier transform $\mathcal{F}$ corresponding to the eigenvalues $i^n$.
Since $f = \what{f}$ the expression becomes
\[
    f(x) \sim \sum_{m=0}^{\infty} a_{4m} h_{4m}(x).
\]
Each $h_{n}$ has a form of $h_{n} = e^{-\pi x^2} P_{n}(x)$ where $P_{n}$ is a polynomial of degree $n$.
A suitable linear combination of $h_{0}$ and $h_{4}$ (satisfying $f(0) = 0$) gives $\pi A^{2} \leq 3$.
The calculations seem difficult and we will not proceed in this direction further.

We can also consider the functions
\begin{equation}
    \label{eqn:2.1}
    g_{a}(x) = a \gamma(ax) + \gamma\left(\frac{x}{a}\right) - (1 + a) \gamma(x),\quad  a > 1
\end{equation}
which satisfy the requirements for \eqref{eqn:1.1}.
Then any expression of the form
\begin{equation}
    \label{eqn:2.2}
    \int_{1}^{\infty} g_{a}(x) \dd \tau(a)
\end{equation}
where $\tau$ is a measure on $[1, \infty)$ such that the integral converges absolutely and $\geq 0$ at infinity is our candidate (it seems difficult to characterize such measures where \eqref{eqn:2.2} converges absolutely and positive at infinity).

We first study $A(g_{a})$.
It is convenient to put $X = \pi x^2$, and $G_{a}(X) = g_{a}(x)$, so
\[
    G_{a}(X) = a e^{-a^{2}X} + e^{-a^{-2}X} - (1 + a)e^{-X}.
\]
The function
\begin{equation}
    \label{eqn:2.3}
    H_{a}(X) = e^{X} G_{a}(X) = a e^{(1 - a^{2})X} + e^{(1 - a^{-2})X} - 1 - a
\end{equation}
is convex and satisfies
\[
    H_{a}(0) = 0, \quad H_{a}'(0) = -a^{2}(a^{2} - 1)(a^{3} - 1) < 0
\]
and tends to $+\infty$ as $X \to \pm \infty$.
So it has a unique zero $X_{a} > 0$, and
\[
    A(g_{a}) = \sqrt{\frac{X_{a}}{\pi}}.
\]
It is natural to study how $X_{a}$ varies, and we first consider those for $a$ close to $1$.
Put $a = 1 + h$, $h > 0$, then for fixed $X$ it can be written as
\[
    H_{a}(X) = (1 + h)(e^{-X(2h + h^2)} - 1) + e^{X(2h - 3h^{2} + 3h^{3} - 4h^{4})X} - 1
\]
modulo $O(h^{5})$.
It can be written as $P_{1}h + P_{2}h^{2} + P_{3}h^{3} + P_{4}h^{4} + O(h^{5})$, where the polynomials $P_{i}$ are
\begin{align*}
    P_{1} &= 0 \\
    P_{2} &= 2X(2X - 3)\\
    P_{3} &= -X(2X - 3) \\
    P_{4} &= -5X + 15 X^{2} - \frac{28}{3} X^{3} + \frac{4}{3}X^{4}.
\end{align*}
From the expression of $P_{2}$, for sufficiently small $h$, $H_a(X) > 0$ if $X > \frac{3}{2}$ and $H_a(X) < 0$ if $X < \frac{3}{2}$.
As a result,
\begin{equation}
    \label{eqn:2.4}
    \lim_{a \to 1^+} X_a = \frac{3}{2}.
\end{equation}
which gives an explicit bound
\begin{equation}
    \label{eqn:2.5}
    A \leq \sqrt{\frac{3}{2 \pi}}.
\end{equation}

But this simple bound cannot be the true value of $A$.
For $X = \frac{3}{2}$, $P_2$ and $P_3$ cancel out, and
\[
    P_4\left(\frac{3}{2}\right) = \frac{3}{2}.
\]
For nonzero small $h$, we therefore have $X_a < \frac{3}{2}$.

If $a \to +\infty$, $X_a \to +\infty$; in fact, a simple calculation shows that
\[
    X_a = \log a + O(1) \quad (a \to +\infty).
\]
We have not determined the minimum value of $X_a$, but it is easy to estimate it semi-heuristically.
The value $a = \sqrt{2}$ satisfies, for $q = e^{\frac{1}{2}X_a}$,
\[
    q^3 - (1+\sqrt{2})q^2 + \sqrt{2} = 0;
\]
if $q \neq 1$, it becomes the quadratic equation
\[
    q^2 - \sqrt{2} q - \sqrt{2} = 0
\]
with a zero $q = \frac{\sqrt{2}}{2}(1 + \sqrt{1+2\sqrt{2}})$,
\[
    X_a = 2 \log q = 1.4749\cdots < \frac{3}{2}\quad(a = \sqrt{2}).
\]
The value $a = 2$ gives, for $q = e^{\frac{3}{4}X}$,
\[
    q^4 - 2 \frac{q^4 - 1}{q - 1} = 0.
\]
The unique zero $q > 1$ is $q = 2.9744\cdots$, from where
\[
    X_a = 1.4534\cdots\quad(a = 2).
\]
It seems that we can approximate the optimal value by this method.
Indeed, if we solve $H_a(X) = 0$ for $H_a$ given by \eqref{eqn:2.3}, and if we assume $a \geq 2$, the first term is negligible.
So $X_a$ is approximately
\[
    \frac{\log(1+a)}{1 - a^{-2}}.
\]
The extremal value of this expression is attained when $a(1-a) = 2 \log(1+a)$, which gives
\[
    a = 2.08137\cdots.
\]

In all cases, the minimum value of $A(g_a)$ we obtain is not the optimum of \eqref{eqn:1.1} that we are looking for.
Consider $a_0$ such that $X_0 = X_{a_0}$ is minimal, and $H_0 = H_{a_0}$ is positive on $[X_a, \infty)$.
Let $a$ be a number (for example, near $1$) such that $X_a > X_0$.
On $[X_a, \infty)$, $H_a \geq 0$ and its order of growth as $X \to +\infty$, $e^{(1-a^{-2})X}$, is smaller than that of $H_{a_0}$ if $a < a_0$.
So there exists $T > 0$ such that $H_{a_0} - TH_{a}$ is $\geq 0$ on $[X_a, \infty)$.
But this function is positive on $[X_0, X_a)$, so is for $X \geq X'$ with $X' < X_0$.

The same argument holds for all $a_0$ with $X_0 < \frac{3}{2}$.
For $a_0 = 2$, we can determin the optimal value (corresponds to $a$ very close to $1$), giving a function $\geq 0$ on $[X'', \infty)$ where
\begin{equation}
    \label{eqn:2.6}
    \begin{aligned}
        X'' &= 1.25 \cdots \\
        A &\leq 0.63\cdots.
    \end{aligned}
\end{equation}
We only made a very rough calculation.
Nevertheless we state the result, to compare with Theorem \ref{thm:1.1}.
\begin{theorem}
    \label{thm:2.1}
    We have $A \leq 0.64\cdots$ and $B_1 \leq 0.41\cdots$.
\end{theorem}
