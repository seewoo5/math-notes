% --- LaTeX Lecture Notes Template - S. Venkatraman ---

% --- Set document class and font size ---

\documentclass[letterpaper, 12pt]{article}

% --- Package imports ---

% Extended set of colors
\usepackage[dvipsnames]{xcolor}

\usepackage{
  amsmath, amsthm, amssymb, mathtools, dsfont, units,          % Math typesetting
  graphicx, wrapfig, subfig, float,                            % Figures and graphics formatting
  listings, color, inconsolata, pythonhighlight,               % Code formatting
  fancyhdr, sectsty, hyperref, enumerate, enumitem, framed }   % Headers/footers, section fonts, links, lists

% lipsum is just for generating placeholder text and can be removed
\usepackage{hyperref, lipsum} 

% --- Fonts ---

\usepackage{newpxtext, newpxmath, inconsolata}
\usepackage{amsfonts}

\usepackage{tikz}
\usepackage{tikz-cd}
\usepackage{enumitem}
\usepackage[title]{appendix}
\usepackage{mathdots}
\usepackage{stmaryrd}

% --- Page layout settings ---

% Set page margins
\usepackage[left=1.35in, right=1.35in, top=1.0in, bottom=.9in, headsep=.2in, footskip=0.35in]{geometry}

% Anchor footnotes to the bottom of the page
\usepackage[bottom]{footmisc}

% Set line spacing
\renewcommand{\baselinestretch}{1.2}

% Set spacing between paragraphs
\setlength{\parskip}{1.3mm}

% Allow multi-line equations to break onto the next page
\allowdisplaybreaks

% --- Page formatting settings ---

% Set image captions to be italicized
\usepackage[font={it,footnotesize}]{caption}

% Set link colors for labeled items (blue), citations (red), URLs (orange)
\hypersetup{colorlinks=true, linkcolor=RoyalBlue, citecolor=RedOrange, urlcolor=ForestGreen}

% Set font size for section titles (\large) and subtitles (\normalsize) 
\usepackage{titlesec}
% \titleformat{\section}{\large\bfseries}{{\fontsize{19}{19}\selectfont\textreferencemark}\;\; }{0em}{}
\titleformat{\section}{\large\bfseries}{\thesection\;\;\;}{0em}{}
\titleformat{\subsection}{\normalsize\bfseries\selectfont}{\thesubsection\;\;\;}{0em}{}

% Enumerated/bulleted lists: make numbers/bullets flush left
%\setlist[enumerate]{wide=2pt, leftmargin=16pt, labelwidth=0pt}
\setlist[itemize]{wide=0pt, leftmargin=16pt, labelwidth=10pt, align=left}

% --- Table of contents settings ---

\usepackage[subfigure]{tocloft}

% Reduce spacing between sections in table of contents
\setlength{\cftbeforesecskip}{.9ex}

% Remove indentation for sections
\cftsetindents{section}{0em}{0em}

% Set font size (\large) for table of contents title
\renewcommand{\cfttoctitlefont}{\large\bfseries}

% Remove numbers/bullets from section titles in table of contents
\makeatletter
\renewcommand{\cftsecpresnum}{\begin{lrbox}{\@tempboxa}}
\renewcommand{\cftsecaftersnum}{\end{lrbox}}
\makeatother

% --- Set path for images ---

\graphicspath{{Images/}{../Images/}}

% --- Math/Statistics commands ---

% Add a reference number to a single line of a multi-line equation
% Usage: "\numberthis\label{labelNameHere}" in an align or gather environment
\newcommand\numberthis{\addtocounter{equation}{1}\tag{\theequation}}

% Shortcut for bold text in math mode, e.g. $\b{X}$
\let\b\mathbf

% Shortcut for bold Greek letters, e.g. $\bg{\beta}$
\let\bg\boldsymbol

% Shortcut for calligraphic script, e.g. %\mc{M}$
\let\mc\mathcal

% \mathscr{(letter here)} is sometimes used to denote vector spaces
\usepackage[mathscr]{euscript}

% Convergence: right arrow with optional text on top
% E.g. $\converge[p]$ for converges in probability
\newcommand{\converge}[1][]{\xrightarrow{#1}}

% Weak convergence: harpoon symbol with optional text on top
% E.g. $\wconverge[n\to\infty]$
\newcommand{\wconverge}[1][]{\stackrel{#1}{\rightharpoonup}}

% Equality: equals sign with optional text on top
% E.g. $X \equals[d] Y$ for equality in distribution
\newcommand{\equals}[1][]{\stackrel{\smash{#1}}{=}}

% Normal distribution: arguments are the mean and variance
% E.g. $\normal{\mu}{\sigma}$
\newcommand{\normal}[2]{\mathcal{N}\left(#1,#2\right)}

% Uniform distribution: arguments are the left and right endpoints
% E.g. $\unif{0}{1}$
\newcommand{\unif}[2]{\text{Uniform}(#1,#2)}

% Independent and identically distributed random variables
% E.g. $ X_1,...,X_n \iid \normal{0}{1}$
\newcommand{\iid}{\stackrel{\smash{\text{iid}}}{\sim}}

% Sequences (this shortcut is mostly to reduce finger strain for small hands)
% E.g. to write $\{A_n\}_{n\geq 1}$, do $\bk{A_n}{n\geq 1}$
\newcommand{\bk}[2]{\{#1\}_{#2}}

% \setcounter{section}{-1}

\newcommand{\SL}{\mathrm{SL}}
\newcommand{\Sp}{\mathrm{Sp}}
\newcommand{\Mp}{\mathrm{Mp}}
\newcommand{\GL}{\mathrm{GL}}
\newcommand{\SO}{\mathrm{SO}}
\newcommand{\SU}{\mathrm{SU}}
\newcommand{\PGL}{\mathrm{PGL}}
\newcommand{\PSL}{\mathrm{PSL}}
\newcommand{\rM}{\mathrm{M}}
\newcommand{\rN}{\mathrm{N}}
\newcommand{\rO}{\mathrm{O}}
\newcommand{\rP}{\mathrm{P}}
\newcommand{\rH}{\mathrm{H}}
\newcommand{\rU}{\mathrm{U}}
\newcommand{\JL}{\mathrm{JL}}
\newcommand{\stab}{\mathrm{Stab}}
\newcommand{\cusp}{\mathrm{cusp}}
\newcommand{\reg}{\mathrm{reg}}
\newcommand{\rs}{\mathrm{rs}}
\newcommand{\Irr}{\mathrm{Irr}}
\newcommand{\Tr}{\mathrm{Tr}}
\newcommand{\Hom}{\mathrm{Hom}}
\newcommand{\Gal}{\mathrm{Gal}}
\newcommand{\WD}{\mathrm{WD}}
\newcommand{\Frob}{\mathrm{Frob}}
\newcommand{\Res}{\mathrm{Res}}
\newcommand{\Tam}{\mathrm{Tam}}
\newcommand{\Pet}{\mathrm{Pet}}
\newcommand{\sgn}{\mathrm{sgn}}
\newcommand{\vol}{\mathrm{vol}}
\newcommand{\Aut}{\mathrm{Aut}}
\newcommand{\Ind}{\mathrm{Ind}}
\newcommand{\BC}{\mathrm{BC}}
\newcommand{\Ad}{\mathrm{Ad}}
\newcommand{\chf}{\mathrm{char}}

\newcommand{\what}{\widehat}

\newcommand{\dd}{\mathrm{d}}

\newcommand{\bA}{\mathbb{A}}
\newcommand{\bR}{\mathbb{R}}
\newcommand{\bS}{\mathbb{S}}
\newcommand{\bZ}{\mathbb{Z}}
\newcommand{\bC}{\mathbb{C}}
\newcommand{\bQ}{\mathbb{Q}}
\newcommand{\bH}{\mathbb{H}}
\newcommand{\bfi}{\mathbf{I}}
\newcommand{\bfa}{\mathbf{a}}
\newcommand{\bfb}{\mathbf{b}}
\newcommand{\cS}{\mathcal{S}}
\newcommand{\cO}{\mathcal{O}}
\newcommand{\cV}{\mathcal{V}}
\newcommand{\cP}{\mathcal{P}}

\newcommand{\scA}{\mathscr{A}}
\newcommand{\scB}{\mathscr{B}}
\newcommand{\scV}{\mathscr{V}}
\newcommand{\scT}{\mathscr{T}}
\newcommand{\scU}{\mathscr{U}}
\newcommand{\scW}{\mathscr{W}}
\newcommand{\scO}{\mathscr{O}}
\newcommand{\scL}{\mathscr{L}}

\newcommand{\frh}{\mathfrak{h}}
\newcommand{\frt}{\mathfrak{t}}
\newcommand{\frg}{\mathfrak{g}}
\newcommand{\frgl}{\mathfrak{gl}}
\newcommand{\fru}{\mathfrak{u}}

% Math mode symbols for common sets and spaces. Example usage: $\R$
\newcommand{\R}{\mathbb{R}}	% Real numbers
\newcommand{\C}{\mathbb{C}}	% Complex numbers
\newcommand{\Q}{\mathbb{Q}}	% Rational numbers
\newcommand{\Z}{\mathbb{Z}}	% Integers
\newcommand{\N}{\mathbb{N}}	% Natural numbers
\newcommand{\F}{\mathcal{F}}	% Calligraphic F for a sigma algebra
\newcommand{\El}{\mathcal{L}}	% Calligraphic L, e.g. for L^p spaces

% Math mode symbols for probability
\newcommand{\pr}{\mathbb{P}}	% Probability measure
\newcommand{\E}{\mathbb{E}}	% Expectation, e.g. $\E(X)$
\newcommand{\var}{\text{Var}}	% Variance, e.g. $\var(X)$
\newcommand{\cov}{\text{Cov}}	% Covariance, e.g. $\cov(X,Y)$
\newcommand{\corr}{\text{Corr}}	% Correlation, e.g. $\corr(X,Y)$
\newcommand{\B}{\mathcal{B}}	% Borel sigma-algebra

% Other miscellaneous symbols
\newcommand{\tth}{\text{th}}	% Non-italicized 'th', e.g. $n^\tth$
\newcommand{\Oh}{\mathcal{O}}	% Big-O notation, e.g. $\O(n)$
\newcommand{\1}{\mathds{1}}	% Indicator function, e.g. $\1_A$

% Additional commands for math mode
\DeclareMathOperator*{\argmax}{argmax}		% Argmax, e.g. $\argmax_{x\in[0,1]} f(x)$
\DeclareMathOperator*{\argmin}{argmin}		% Argmin, e.g. $\argmin_{x\in[0,1]} f(x)$
\DeclareMathOperator*{\spann}{Span}		% Span, e.g. $\spann\{X_1,...,X_n\}$
\DeclareMathOperator*{\bias}{Bias}		% Bias, e.g. $\bias(\hat\theta)$
\DeclareMathOperator*{\ran}{ran}			% Range of an operator, e.g. $\ran(T) 
\DeclareMathOperator*{\dv}{d\!}			% Non-italicized 'with respect to', e.g. $\int f(x) \dv x$
\DeclareMathOperator*{\diag}{diag}		% Diagonal of a matrix, e.g. $\diag(M)$
\DeclareMathOperator*{\trace}{Tr}		% Trace of a matrix, e.g. $\trace(M)$
\DeclareMathOperator*{\supp}{supp}		% Support of a function, e.g., $\supp(f)$

% Numbered theorem, lemma, etc. settings - e.g., a definition, lemma, and theorem appearing in that 
% order in Lecture 2 will be numbered Definition 2.1, Lemma 2.2, Theorem 2.3. 
% Example usage: \begin{theorem}[Name of theorem] Theorem statement \end{theorem}
\theoremstyle{definition}
\newtheorem{theorem}{Theorem}[section]
\newtheorem{conjecture}{Conjecture}[section]
\newtheorem{proposition}[theorem]{Proposition}
\newtheorem{lemma}[theorem]{Lemma}
\newtheorem{corollary}[theorem]{Corollary}
\newtheorem{definition}[theorem]{Definition}
\newtheorem{example}[theorem]{Example}
\newtheorem{remark}[theorem]{Remark}

% Un-numbered theorem, lemma, etc. settings
% Example usage: \begin{lemma*}[Name of lemma] Lemma statement \end{lemma*}
\newtheorem*{theorem*}{Theorem}
\newtheorem*{proposition*}{Proposition}
\newtheorem*{lemma*}{Lemma}
\newtheorem*{corollary*}{Corollary}
\newtheorem*{definition*}{Definition}
\newtheorem*{example*}{Example}
\newtheorem*{remark*}{Remark}
\newtheorem*{claim}{Claim}
\newtheorem*{question*}{Question}
\newtheorem*{problem*}{Problem}

% --- Left/right header text (to appear on every page) ---

% Do not include a line under header or above footer
\pagestyle{fancy}
\renewcommand{\footrulewidth}{0pt}
\renewcommand{\headrulewidth}{0pt}

% Right header text: Lecture number and title
\renewcommand{\sectionmark}[1]{\markright{#1} }
% \fancyhead[R]{\small\textit{\nouppercase{\rightmark}}}

% Left header text: Short course title, hyperlinked to table of contents
% \fancyhead[L]{\hyperref[sec:contents]{\small Gan-Gross-Prasad conjecture}}

\numberwithin{equation}{section}

\makeatletter
\newcommand{\eqnum}{\refstepcounter{equation}\textup{\tagform@{\theequation}}}
\makeatother
% --- Document starts here ---

\begin{document}

% --- Main title and subtitle ---

\title{Heisenberg's principle and positive functions \\[1em]
\normalsize Principe d'Heisenberg et fonctions positives \\ - \\
\normalsize Re-\TeX ed by Seewoo Lee\footnote{seewoo5@berkeley.edu. Most of the translation is due to Google Translator, and I only fixed a little.}}

% --- Author and date of last update ---

\author{Jean Bourgain, Laurent Clozel, Jean-Pierre Kahane}
\date{\normalsize\vspace{-1ex} Last updated: \today}

% --- Add title and table of contents ---

\maketitle


% --- Abstracts ---

% \tableofcontents\label{sec:contents}
\begin{abstract}
We consider a natural problem concerning Fourier transforms.
In one variable, one seeks functions $f$ and $\what{f}$, both positive for $|x| \geq a$ and vanishing at $0$.
What is the lowest bound for $a$?
In higher dimension, the same problem can be posed by replacing the interval by the ball of radius $a$.
We show that there is indeed a strictly positive lower bound, which is estimated as a function of the dimension.
In the last section the question, and its solution, are shown to be naturally related to the theory of zeta functions.
\end{abstract}

% --- Main content: import lectures as subfiles ---


\section{Introduction}

\begin{conjecture}[Langlands functoriality conjecture] Let $G$ and $G'$ be reductive groups over a global field $F$. 
\end{conjecture}
This is an introductory note on Langlands functoriality conjecture view towards classical examples. Here is a list of topics we are going to study:

\begin{enumerate}
    \item Automorphic induction
    \item Base change
    \item Rankin-Selberg product
    \item Symmetric power lifting and Selberg's 1/4 conjecture
    \item Jacquet-Langlands correspondence
    \item Theta correspondence and Howe duality
\end{enumerate}
\section{Statement of the problem and lower bound of $B_{1}$}

Consider a pair of functions $(f, \what{f})$ on reals: they are Fourier pairs if
\[
    \begin{cases}
    \what{f}(y) = \int f(x) e^{-2 i \pi x y} \dd x, \quad f \in L^{1}(\mathbb{R}) \\
    f(x) = \int \what{f}(y) e^{2 i \pi x y} \dd y, \quad \what{f} \in L^{1}(\mathbb{R}).
    \end{cases}
\]
So $f$ and $\what{f}$ are continuous and converges to $0$ at infinity.
We are interested in the Fourier pairs $(f, \what{f})$ such that
\begin{enumerate}
    \item $f$ and $\what{f}$ are real-valued, even, and not identically zero,
    \item $f(0) \leq 0$ and $\what{f}(0) \geq 0$,
    \item $f(x) \geq 0$ for $x \geq a_f$ and $\what{f}(y) \geq 0$ for $y \geq a_{\what{f}}$.
\end{enumerate}
Note that the condition 2 and the non-vanishing assumptions on $f$ and $\what{f}$ imply $a_f$ and $a_{\what{f}} > 0$.

\begin{problem*}
What is the infimum of the product $a_{f} a_{\what{f}}$ for the Fourier pairs $(f, \what{f})$ satisfying 1--3?
\end{problem*}

We denote the infimum as $B_1 \geq 0$ (note that the pair attaining infimum clearly exists).
We will show, which is not obvious a priori, that $B_1$ is strictly positive.

Until section 3, we will focus on dimension $1$.
For a Fourier pair $(f, \what{f})$ satisfying 1--3 let
\begin{align*}
    A(f) &= \inf\{x > 0: f((x, \infty)) \subset \mathbb{R}^{+}\} \\
    A(\what{f}) &= \inf\{y > 0: \what{f}((t, \infty)) \subset \mathbb{R}^{+}\} .
\end{align*}
The product $A(f)A(\what{f})$ is invariant under scaling, i.e. replacing $f(x)$, $\what{f}(y)$ by $f(x /\lambda)$, $\lambda \what{f}(\lambda y)$, $\lambda > 0$.
Since
\[
    B_1 = \inf A(f) A(\what{f})
\]
for all Fourier pairs satisfying 1--3, we only consider pairs satisfying $A(f) = A(\what{f})$.
Then $f + \what{f} \neq 0$ (consider their values at points near $A(f)$), and
\[
    A(f + \what{f}) \leq A(f) = A(\what{f}).
\]
So $B_1 = \inf A^{2}(f + \what{f})$. Hence we see that
\[
    B_1 = A^2, \quad A = \inf A(f)
\]
where infimum is taken over all functions $f \in L^{1}(\mathbb{R})$, real-valued and even, not identically zero, equal to their own Fourier transforms, and $f(0) < 0$.

Let
\[
    \gamma(x) = e^{-\pi x^2}
\]
so that $\gamma = \what{\gamma}$.
If $f(0) < 0$, $f - f(0)\gamma$ satisfies the same conditions as $f$, and
\[
    A(f - f(0)\gamma) \leq A(f).
\]
Finally,
\begin{equation}
\label{eqn:1.1}
    A = \inf A(f)
\end{equation}
\textbf{where infimum is taken over all $f \in L^{1}(\mathbb{R})$, real-valued, even, not identically zero, $f = \what{f}$, and $f(0) = 0$.}

Here is an important result.

\begin{theorem}
Let $\lambda = -\inf\left( \frac{\sin x}{x}\right) = 0.2712\cdots$.
Then
\[
    A \geq \frac{1}{2(1 + \lambda)} = 0.4107\cdots
\]
so
\[
    B \geq 0.1687\cdots.
\]
\end{theorem}
\begin{proof}
Choose $f = \what{f}$, $f(0) = 0$, and $\int_{\mathbb{R}} |f(x)| \dd x := \int_{\mathbb{R}} |f| = 1$.
Write $A = A(f)$.
Put $f = f^{+} - f^{-}$, $|f| = f^{+} + f^{-}$.
Since $\int_{\mathbb{R}}f = \what{f}(0) = 0$, we have $\int_{\mathbb{R}} f^{+} = \int_{\mathbb{R}} f^{-} = \int_{-A}^{A} f^{-} = \frac{1}{2}$.
So $\int_{-A}^{A} |f| \geq \frac{1}{2}$.
From $|f(x)| \leq \int |\what{f}| = 1$, $2A \geq \frac{1}{2}$ and we obtain a first bound $A \geq \frac{1}{4}$.
We will see that this argument extends to higher dimensions.

In dimension $1$, we can refine it in the following way.
From $f = \what{f}$,
\begin{align*}
    f(x) &= \int f(y) \cos 2 \pi y x \dd y = \int f(y) (\cos 2 \pi y x - 1) \dd y \\
    &= \int f^{-}(y) (1 - \cos 2 \pi y x) \dd y - \int f^{+}(y) (\cos 2 \pi y x - 1) \dd y.
\end{align*}
This implies, ???
\[
    f^{-}(x) \leq \int f^{+}(y) (1 - \cos 2 \pi y x) \dd y
\]
and
\[
    \frac{1}{4} = \int_{0}^{A} f^{-} \leq \int_{-\infty}^{\infty} f^{+}(y) \left(A - \frac{\sin 2 \pi y A}{2 \pi y}\right) \dd y
\]
so
\[
    \frac{1}{4} \leq \frac{A}{2}\sup_{u \in \mathbb{R}} \left(1 - \frac{\sin u}{u}\right) = \frac{A}{2}(1 + \lambda)
\]
and we obtain the theorem.
\end{proof}
Later, we will need to consider functions that are regular enough.
A natural class is the Schwartz space $\mathcal{S}$.
It is not obvious that the infimum $A$ defined by \eqref{eqn:1.1}, taken only over the functions in $\mathcal{S}$, coincides with that over all $f \in L^{1}(\mathbb{R})$.

Let $\mathcal{B}_{1}$ be $A^{2}$, where $A$ is defined by \eqref{eqn:1.1} for $f \in \mathcal{S}$.
We will see that $B_{1}$ and $\mathcal{B}_{1}$ are not much different.
Clearly, we have
\begin{equation}
    \label{eqn:1.2}
    B_{1} \leq \mathcal{B}_{1}.
\end{equation}
Let
\[
    B_{1}^{-} = \inf \{A^{2}: f(0) < 0, f = \what{f}\text{ even} \neq 0, f \in L^{1}(\mathbb{R})\}.
\]
Hence $B_{1}^{-}$ is defined by \eqref{eqn:1.1}, with additional assumption $f(0) < 0$.
Define $\mathcal{B}_{1}^{-}$ similarly for $f \in \mathcal{S}$.
Clearly,
\begin{equation}
    \label{eqn:1.3}
    B_{1}^{-} \leq \mathcal{B}_{1}^{-}
\end{equation}
\begin{equation}
    \label{eqn:1.4}
    \mathcal{B}_{1} \leq \mathcal{B}_{1}^{-}, \quad B_{1} \leq B_{1}^{-}.
\end{equation}
To prove $\mathcal{B}_{1}^{-} \leq B_{1}^{-}$, let $f \in L^{1}(\mathbb{R})$ be a function satisfying the conditions for \eqref{eqn:1.1} but $f(0) < 0$, and let $a = A(f)$.
Let $\varphi = \psi \ast \psi$, where $\psi$ is $C^{\infty}$, even, positive, and compactly supported near $0$, and $g = f \ast \varphi$.
Then $A(g) \leq a + \varepsilon$ and $g(0) < 0$.
We have $\what{g} = \what{f} \what{\psi}^{2}$; by applying the same operation on $\what{g}$ we obtain a function $h \in \mathcal{S}$ such that $h = \what{h}$, $h(0) < 0$, and $A(h) \leq a + \varepsilon$; from this we get $\mathcal{B}_{1}^{-} \leq B_{1}^{-}$ and
\begin{equation}
    \label{eqn:1.5}
    \mathcal{B}_{1}^{-} = B_{1}^{-}.
\end{equation}
Note that the argument does not work if $f(0) = 0$.
We will show
\begin{equation}
    \label{eqn:1.6}
    B_{1}^{-} \leq 2B_{1};
\end{equation}
combining \eqref{eqn:1.4} and \eqref{eqn:1.6} we obtain
\begin{equation}
    \label{eqn:1.7}
    B_{1} \leq \mathcal{B}_{1} \leq 2 B_{1}.
\end{equation}
Let $f$ be a function satisfying the conditions for \eqref{eqn:1.1} and $a = A(f)$.
Since $\what{f}(0) = \int f(x) \dd x = 0$, $f$ takes a negative value on $[-a, a]$.
Let $b > 0$ be such a number, and consider the distribution
\[
    T = \delta_{b} + \delta_{-b} + 2 \delta_{0}.
\]
It is a positive measure with positive Fourier transform
\[
    \what{T} = 2 \cos (2 \pi b y) + 2 \geq 0.
\]
We have
\[
    (T \ast f)(0) = f(b) + f(-b) < 0.
\]
Since $b < a$, $g = T \ast f$ satisfies
\[
    g(0) < 0, \quad g \geq 0 \text{ on }(2a, \infty).
\]
Moreover $\what{g} = \what{T}\what{f}$ is nonnegative on $[0, \infty)$, and $\what{g}(0) = 0$.
By scaling, we obtain a function $h$ such that
\begin{align*}
    h &\geq 0 \text{ on } [a \sqrt{2}, \infty),\quad h(0) < 0 \\
    \what{h} &\geq 0 \text{ on } [a \sqrt{2}, \infty),\quad \what{h}(0) = 0. \\
\end{align*}
The functions $h$ and $\what{h}$ are real-valued and even.
Hence $h + \what{h}$ satisfy the conditions defining $B_{1}^{-}$.
So $B_{1}^{-} \leq (a \sqrt{2})^{2} = 2a^{2}$; by varying $f$, we obtain \eqref{eqn:1.6}.

\section{Upper bound of $B_{1}$}

An important idea is to use Hermite series
\[
    f(x) \sim \sum_{n = 0}^{\infty}a_n h_n(x)
\]
associated to $f$, where $h_n$ are eigenvectors of the Fourier transform $\mathcal{F}$ corresponding to the eigenvalues $i^n$.
Since $f = \what{f}$ the expression becomes
\[
    f(x) \sim \sum_{m=0}^{\infty} a_{4m} h_{4m}(x).
\]
Each $h_{n}$ has a form of $h_{n} = e^{-\pi x^2} P_{n}(x)$ where $P_{n}$ is a polynomial of degree $n$.
A suitable linear combination of $h_{0}$ and $h_{4}$ (satisfying $f(0) = 0$) gives $\pi A^{2} \leq 3$.
The calculations seem difficult and we will not proceed in this direction further.

We can also consider the functions
\begin{equation}
    \label{eqn:2.1}
    g_{a}(x) = a \gamma(ax) + \gamma\left(\frac{x}{a}\right) - (1 + a) \gamma(x),\quad  a > 1
\end{equation}
which satisfy the requirements for \eqref{eqn:1.1}.
Then any expression of the form
\begin{equation}
    \label{eqn:2.2}
    \int_{1}^{\infty} g_{a}(x) \dd \tau(a)
\end{equation}
where $\tau$ is a measure on $[1, \infty)$ such that the integral converges absolutely and positive is our candidates (it seems difficult to characterize such measures where \eqref{eqn:2.2} converges absolutely and positive).

We first study $A(g_{a})$.
It is convenient to put $X = \pi x^2$, and $G_{a}(X) = g_{a}(x)$, so
\[
    G_{a}(X) = a e^{-a^{2}X} + e^{-a^{-2}X} - (1 + a)e^{-X}.
\]
The function
\begin{equation}
    \label{eqn:2.3}
    H_{a}(X) = e^{X} G_{a}(X) = a e^{(1 - a^{2})X} + e^{(1 - a^{-2})X} - 1 - a
\end{equation}
is convex and satisfying
\[
    H_{a}(0) = 0, \quad H_{a}'(0) = -a^{2}(a^{2} - 1)(a^{3} - 1) < 0
\]
and tends to $+\infty$ as $X \to \pm \infty$.
So it has a unique zero $X_{a} > 0$, and
\[
    A(g_{a}) = \sqrt{\frac{X_{a}}{\pi}}.
\]
It is natural to study with varying $X_{a}$, and we first consider those for $a$ near $1$.
Put $a = 1 + h$, $h > 0$, then $H_{a}(X)$ can be written as
\[
    H_{a}(X) = (1 + h)(e^{-X(2h + h^2)} - 1) + e^{X(2h - 3h^{2} + 3h^{3} - 4h^{4})X} - 1
\]
modulo $O(h^{5})$.
It can be written as $P_{1}h + P_{2}h^{2} + P_{3}h^{3} + P_{4}h^{4} + O(h^{5})$, where the polynomials $P_{i}$ are
\begin{align*}
    P_{1} &= 0 \\
    P_{2} &= 2X(2X - 3)\\
    P_{3} &= -X(2X - 3) \\
    P_{4} &= -5X + 15 X^{2} - \frac{28}{3} X^{3} + \frac{4}{3}X^{4}.
\end{align*}
From the expression of $P_{2}$, for sufficiently small $h$, $H_a(X) > 0$ if $X > \frac{3}{2}$ and $H_a(X) < 0$ if $X < \frac{3}{2}$.
As a result,
\begin{equation}
    \label{eqn:2.4}
    \lim_{a \to 1^+} X_a = \frac{3}{2}.
\end{equation}
This provides an explicit bound
\begin{equation}
    \label{eqn:2.5}
    A \leq \sqrt{\frac{3}{2 \pi}}.
\end{equation}

But this simple bound cannot be the true value of $A$.
For $X = \frac{3}{2}$, $P_2$ and $P_3$ cancel out, and
\[
    P_4\left(\frac{3}{2}\right) = \frac{3}{2}.
\]
For nonzero small $h$, we therefore have $X_a < \frac{3}{2}$.

If $a \to +\infty$, $X_a \to +\infty$; in fact, a simple calculation shows that
\[
    X_a = \log a + O(1) \quad (a \to +\infty).
\]
We have not determined the minimum value of $X_a$, but it is easy to estimate it, in a semi-heuristic way.
The value $a = \sqrt{2}$ satisfies, for $q = e^{\frac{1}{2}X_a}$,
\[
    q^3 - (1+\sqrt{2})q^2 + \sqrt{2} = 0;
\]
if $q \neq 1$, it becomes the quadratic equation
\[
    q^2 - \sqrt{2} q - \sqrt{2} = 0
\]
with a zero $q = \frac{\sqrt{2}}{2}(1 + \sqrt{1+2\sqrt{2}})$,
\[
    X_a = 2 \log q = 1.4749\cdots < \frac{3}{2}\quad(a = \sqrt{2}).
\]
The value $a = 2$ gives, for $q = e^{\frac{3}{4}X}$,
\[
    q^4 - 2 \frac{q^4 - 1}{q - 1} = 0.
\]
The unique zero $q > 1$ is $q = 2.9744\cdots$, where
\[
    X_a = 1.4534\cdots\quad(a = 2).
\]
It seems that we can approximate the optimal value by this method.
Indeed, if we solve $H_a(X) = 0$ for $H_a$ given by \eqref{eqn:2.3}, and if we assume $a \geq 2$, the first term is negligible.
So $X_a$ is approximately
\[
    \frac{\log(1+a)}{1 - a^{-2}}.
\]
The extremal value of this expression is attained when $a(1-a) = 2 \log(1+a)$, which gives
\[
    a = 2.08137\cdots.
\]

In all cases, the minimum value of $A(g_a)$ we obtain is not the value for \eqref{eqn:1.1} that we are looking for.
Consider $a_0$ such that $X_0 = X_{a_0}$ is minimal, and $H_0 = H_{a_0}$ is positive on $[X_a, \infty)$.
Let $a$ be a number (for example, near $1$) such that $X_a > X_0$.
On $[X_a, \infty)$, $H_a \geq 0$ and 
\section{Higher dimensions}

On Euclidean space $\mathbb{R}^{d}$ with inner product
\[
    x \cdot y = \sum_{i=1}^{d} x_i y_i, \quad \|x\| = (x \cdot x)^{1/2},
\]
Fourier transform is defined by
\begin{equation}
    \label{eqn:3.1}
    \what{f}(y) = \int f(x) e^{-2 i \pi x \cdot y} \dd x
\end{equation}
where $\dd x = \dd x_1 \cdots \dd x_d$ is the Lebesgue measure; then
\begin{equation}
    \label{eqn:3.2}
    f(x) = \int \what{f}(y) e^{2 i \pi x \cdot y} \dd y.
\end{equation}
We suppose that $f$ and $\what{f}$ are continuous and integrable.
More generally, if $E$ is a Euclidean space of dimension $d$, if the invariant measure $\dd x$ on $E$ is chosen so that the cube formed by the orthonormal basis has measure $1$, and if $x\cdot y$ is the corresponding inner product, Fourier transform and its inverse is defined by \eqref{eqn:3.1} and \eqref{eqn:3.2}.

Consider the Fourier pairs $(f, \what{f})$ satisfying
\begin{enumerate}
    \item $f$, $\what{f}$ are not identically zero,
    \item $f(0) \leq 0$ and $\what{f}(0) \leq 0$,
    \item $f(x) \geq 0$ for $\|x\| \geq a_f$, $\what{f}(0) \geq 0$ for $\|y\| \geq a_{\what{f}}$.
\end{enumerate}
Define $A(f)$ and $A(\what{f})$ as in \S 1:
\[
    A(f) = \inf\{r >0: f(x) \geq 0 \text{ if } \|x\| > r\},
\]
and
\[
    B_d = \inf A(f) A(\what{f})
\]
for pairs satisfying 1--3.
Let $f^\natural(x)$ be the (invariant) integral of $f$ on the sphere of radius $\|x\|$: $\what{f}^\natural = (\what{f})^{\natural}$ and $f^\natural$ and $\what{f}^{\natural}$ are nonzero; otherwise $f$ and $\what{f}$ are compactly supported from 3.
Since $A(f^\natural) \leq A(f)$ and $A(\what{f}^{\natural}) \leq A(\what{f})$, we can limit ourselves to the radial functions.
Since
\[
    (f(x/\lambda))^\wedge = \lambda^d \what{f}(\lambda y)\quad (\lambda > 0),
\]
we can follow the argument in \S 1 and we have
\begin{equation}
    \label{eqn:3.4}
    B_d = A^2, \quad A = \inf A(f)
\end{equation}
\textbf{where the infimum is over the functions $f \in L^1(\mathbb{R}^d)$, radial, not identically zero, such that $f = \what{f}$ and $f(0) = 0$.}

We have, as in \S 1, can add multiple of the following radial and self-dual function if necessary
\[
    \gamma(x) = e^{-\pi \|x\|^2}.
\]

\begin{theorem}
\label{thm:3.1}
We have
\[
B_d \geq \frac{1}{\pi}\left(\frac{1}{2}\Gamma\left(\frac{d}{2} + 1\right)\right)^{2/d} > \frac{d}{2 \pi e}.
\]
\end{theorem}
\begin{proof}
Follow the argument of the case $d = 1$, where we replace the interval $(-A(f), A(f))$ with the ball of radius $A(f)$ centered at the origin, whose volume ($\geq \frac{1}{2}$) is $\frac{1}{\Gamma(\frac{d}{2} + 1)}(A(f))^{d} \pi^{d/2}$.
\end{proof}

Put $X = \pi \|x\|^2$, the argument in \S 2 natually leads us to consider the functions
\[
    g_a(x) = G_a(X) \quad (x \in \mathbb{R}^{d})
\]
where
\[
    G_a(X) = a^d e^{-X a^2} + e^{-X a^2} - (1 + a^d) e^{-X},
\]
and set
\[
    H_a(X) = a^d e^{(1-a^2)X} + e^{(1 - a^{-2})X} - (1 + a^d), \quad a > 1.
\]
It is convenient to define $a^2 = 1 + k$, $d = 2c$, which gives
\[
    H_a(X) = (1 + k)^{c} e^{-kX} + e^{(1 - (1+k)^{-1})X} - 1 - (1 + k)^{c}.
\]
The derivative in $X$ at the origin is
\[
    \frac{k}{1 + k}\left(1 - (1+k)^{c+1}\right) < 0;
\]
the convexity argument in \S 2 shows that $H_a$ has a unique positive zero $X_a$.
As before, we compute the expansion of $H_a(X)$ in $k$ up to order $4$.
It is
\[
    H_a(X) = P_1 k + P_2 k^2 + P_3 k^3 + P_4 k^4 + O(k^5)
\]
where
\begin{align*}
    P_1 &= 0 \\
    P_2 &= X(X - c - 1) \\
    P_3 &= \frac{1}{2}(c - 2)X(X-c-1) \\
    P_4 &= \frac{1}{12} X(X^3 - (2c + 6)X^2 + (3c(c-1) + 18)X - (2c(c-1)(c-2) + 12)).
\end{align*}
As in dimension $1$ case, we see that $P_2$ and $P_3$ cancel out for
\begin{equation}
    \label{eqn:3.5}
    X = X(d) := \frac{d}{2} + 1.
\end{equation}
Moreover, $P_2 > 0$  for $X > X(d)$, $<0$ for $X < X(d)$.
Taking the limit $k \to 0$ gives
\[
    \lim_{a \to 1} X_a = \frac{d}{2} + 1.
\]
To understand the location of $X_a$ with respect to $X(d)$ as $a \to 1$, compute $Q_4(X(d))$ or $P_4 = \frac{X}{12}Q_4$.
Calculation gives
\[
    Q_4(c+1) = -c^2 + 1.
\]
For $d > 2$, the term is $<0$, so $H_a(X(d)) < 0$ for $a$ close to $1$, which shows that
\[
    X_a > \frac{d}{2} + 1 \quad(a > 1, \text{ close to }1).
\]
Therefore it is possible that the value in \eqref{eqn:3.5} is optimal.
This is not the case when $d = 1$ as we saw in \S 2.

For $d = 2$, $Q_4(c+1) = 0$, so we need to compute up to degree $5$, where
\begin{equation}
    \label{eqn:3.6}
    H_a(2) = (1+k)e^{-2k} + e^{2(1 - \frac{1}{1+k})} - 2 - k.
\end{equation}
The Taylor series at $0$ of
\begin{align*}
    f(z) &= e^{2(1 - \frac{1}{1+z})} = e^{2\frac{z}{1+z}},\\
    f(z) &= \sum_{n=0}^{\infty} q_n z^n,
\end{align*}
can be calculated using the residue theorem.
Let
\[
    w = \frac{z}{1+z}, \,\,z = \frac{w}{1-w}, \,\,\dd z = \frac{\dd w}{(1-w)^2},
\]
by taking a small contour around $0$:
\begin{align*}
    q_n &= \Res_{z=0} \frac{f(z)}{z^{n+1}} = \frac{1}{2 i \pi} \oint e^{\frac{2z}{1+z}} \frac{\dd z}{z^{n+1}} \\
    &= \frac{1}{2 i \pi} \oint e^{2w} \frac{(1-w)^{n+1}}{w^{n+1}} \frac{\dd w}{(1-w)^{2}} \\
    &= \Res_{w=0} \frac{(1-w)^{n-1}}{w^{n+1}} e^{2w}.
\end{align*}
In particular, $q_5$ is the sum of
\begin{equation}
    \label{eqn:3.7}
    \frac{2^4}{4!} - \frac{2^5}{5!}
\end{equation}
coming from the first term of \eqref{eqn:3.6}, and the coefficient of $w^5$ in $e^{2w}(1-w)^{4}$, equal to
\begin{equation}
    \label{eqn:3.8}
    \frac{2^5}{5!} - 4 \cdot \frac{2^4}{4!} +  6 \cdot \frac{2^3}{3!} - 4 \cdot \frac{2^2}{2!} + 2.
\end{equation}
We found that $q_5 = 0$.

Similarly, $q_6$ is the sum of
\begin{equation}
    \label{eqn:3.9}
    -\frac{2^5}{5!} + \frac{2^6}{6!}
\end{equation}
and
\begin{equation}
    \label{eqn:3.10}
    \frac{2^6}{6!} - 5 \cdot \frac{2^5}{5!} + 10 \cdot \frac{2^4}{4!} - 10 \cdot \frac{2^3}{3!} + 5 \cdot \frac{2^2}{2!} - 2,
\end{equation}
which is
\[
    q_6 = -\frac{4}{45} < 0.
\]
When $a$ is sufficiently close to $1$, we therefore have $H_a(2) < 0$ and $X_a > X(2) = 2$.
Again, the bound given by \eqref{eqn:3.5} could be optimal.

Concluding this section, note that for all $d \geq 2$ we obtain the upper bound
\begin{equation}
    \label{eqn:3.11}
    B_d \leq \mathcal{B}_{d} \leq \frac{d+2}{2\pi}
\end{equation}
where $\mathcal{B}_{d}$ is defined, as in \S 1, by the functions in the space $\mathcal{S}(\mathbb{R}^{d})$.
Also following the argument in the end of \S 1, relating the bounds for $L^1$ and $\mathcal{S}$ applies.
To prove the inequality \eqref{eqn:1.6}, we have to consider $T = \delta_b + \delta_{-b} + 2 \delta_{0}$, where $\|b\| < a = A(f)$ and $f(b) < 0$; $\what{T} = 2 \cos(2 \pi b \cdot y) + 2$ is a positive plane wave function.
The rest of the argument is the same, replacing $h + \what{h}$ with the spherical average of $h + \what{h}$ if we want to limit ourselves to the radial functions.
In conclusion,
\begin{theorem}
\label{thm:3.2}
We have
\begin{equation}
    \label{eqn:3.12}
    B_d \leq \mathcal{B}_{d} \leq \frac{d+2}{2 \pi}, \quad B_d \geq \frac{1}{2} \mathcal{B}_d.
\end{equation}
\end{theorem}


\section{Arithmetic}

Let $F$ be a number field of degree $d$ over $\mathbb{Q}$.
We denote as $v$ for the places of $F$ (finite or archimedean), and $F_{v}$ for the corresponding completion; for finite $v$, $\mathcal{O}_{v} \subset F_{v}$ is the ring of integers of $F_{v}$ and $\mathcal{O}_{v}^\times$ is the group of unities; $q_{v}$ is the cardinality of the residue field.
Let
\[
    \mathbb{A}_{F} = \sideset{}{'}\prod_{v} F_{v}
\]
(restricted product) be the ring of ad\`eles of $F$, and $\mathbb{A}_{F}^\times = I_{F}$ the group of id\`eles.
Let $x: I_{F} \mapsto \prod_{v} |x|_{v}$ be the id\`ele norm,
\begin{align*}
    I_F^{1} &= \{ x \in I_F: |x| = 1\} \\
    \text{and }I_{F}^{+} &= \{ x \in I_F: |x| \geq 1\}.
\end{align*}
Consider the invariant measure $\dd x = \prod \dd x_v$ on $\mathbb{A}_F$, where $\dd x_v$ is Haar measure on $F_v$.
At finite places, $\dd x_v$ are self-dual measures of Tate \cite{tate1967fourier}; at a real place, $\dd x_v$ is the Lebesgue measure; at a complex place, if we write the variable $z = x + iy$, $\dd z = 2 \dd x \dd y$.
At a real place, the Fourier transform $\hat{f}(y)$ of a function $f$ is defined as before.

If $z = x + iy$ is a complex variable and $w = \xi + i \eta$, Tate define the transfom $\what{f}(w)$ of a function $f(z)$ by
\begin{align*}
    \what{f}(w) &= \int f(z) e^{-2 i \pi \Tr(zw)} \dd z \\
    \text{where } \Tr(zw) &= 2 \Re(zw) = 2(x \xi - y \eta).
\end{align*}
For radial functions, in each of the variables, it coincides with the Fourier transform defined in \S 3 with the inner product $z \cdot w = 2(x\xi + y\eta)$.
The self-dual measure $\dd z$ of Tate is the normalized measure considered in the beginning of \S 3 for abstract Euclidean spaces.

Let $f$ be a function in the Schwartz space of $\mathbb{A}_F$ given by
\begin{equation}
    \label{eqn:4.1}
    f(x) = \prod_{v|\infty} f_{v}(x_{v}) \prod_{v\text{ finite}} f_{v}^{0}(x_{v})
\end{equation}
where $f_{v}^{0}$ is the characteristic function of $\mathcal{O}_{v}$ and, for archimedean $v$, $f_{v}$ is for a moment an arbitrary Schwartz function.
Tate's zeta function associated to $f$ is defined for $\Re(s) > 1$ by
\[
    Z(f, s) = \int_{I_F} f(x) |x|^{s} \dd^\times x,
\]
where $\dd^\times x$ is the product of $\dd^\times x_v = \frac{\dd x_v}{|x_v|}$ (multiplied by $(1 - q_{v}^{-1})^{-1}$ at finite places).

Rather than the factorizable functions in \eqref{eqn:4.1}, we will consider the functions of the form $g_{a}(x)$ (\S 3) on $\mathbb{R}^{d}$, where $\mathbb{R}^{d}$ is regarded as an inner product space via
\[
    \|x_\infty\|^{2} = \sum_{v \text{ real}} |x_v|^{2} + \sum_{v \text{ complex}} 2 \|z_{v}\|^{2}
\]
where $\|z\|$ is the usual absolute value of a complex number (We denote $|z| = \|z\|^{2}$ the normalized absolute norm as in Tate's theory).
More generally,
\begin{equation}
    \label{eqn:4.2}
    f(x) = f_\infty(x_\infty) \prod_{v\text{ finite}} f_v^0(x_v)
\end{equation}
where $f_\infty(x_\infty) \in \mathcal{S}(\mathbb{R}^{d})$.
The conditions imposed by Tate (i.e., $(z_1), (z_2), (z_3)$ in \cite[\S 4.4]{tate1967fourier}) are satisfied by these functions.
For example, $(z_3)$ says that the integral
\[
    \int_{F_\infty} f_\infty(x_\infty) \prod_{v|\infty} |x_v|_v^{\sigma - 1} \dd x
\]
where $F_\infty = \prod_{v|\infty}F_v$, converges absolutely for $\sigma > 1$.
In fact, it holds for $\sigma > 0$ and all $f_\infty \in \mathcal{S}(F_\infty)$.
Hence the same condition holds for $\what{f}$.

In the case where $f_\infty = \prod f_v^0$ with
\begin{align*}
    f_v^0(x) &= e^{-\pi x^2} \quad (\text{real variable}) \\
    f_v^0(x) &= e^{-2 \pi \|z\|^{2}} \quad (\text{complex variable}),
\end{align*}
$Z(f, s)$ is the zeta function $\zeta_F(s)$, multiplied by the usual archimedean factors (product of $\Gamma$ functions) and $|D_F^{-1/2}|$.
Following Tate \cite{tate1967fourier}, we write
\begin{equation}
    \label{eqn:4.3}
    Z(f, s) = \int_{I_F^{+}} f(x) |x|^{s} \dd^\times x + \int_{I_F^{+}} \what{f}(x)|x|^{1-s} \dd^\times x + \kappa \frac{\what{f}(0)}{s - 1} - \kappa \frac{f(0)}{s}
\end{equation}
following the usual notations \cite[Th\'eor\`eme 4.3.2]{tate1967fourier},
\[
    \kappa = \frac{2^{r_1}(2 \pi)^{r_2} h R}{\sqrt{|D_F|}w}
\]
is the residue of $\zeta_F(s)$ at $s = 1$.
In particular, $D_F$ is the absolute discriminant of $F$, and $d = r_1 + 2 r_2$, where $r_1$ is the number of real places and $r_2$ is the number of complex places.
Then the two integrals in \eqref{eqn:4.3} converges absolutely for all $s \in \mathbb{C}$.


\begin{lemma}
\label{lem:4.1}
Let $s$ be a zero of $\zeta_F(s)$ with $\Re(s) > 0$.
Then $Z(f, s)$ vanishes for all $f_\infty \in \mathcal{S}(F_\infty)$.
\end{lemma}
Indeed, we can write $Z(f, s)$ for $\Re(s) > 1$ as
\[
    Z(f, s) = |D_F|^{-1/2} Z(f_\infty, s) \zeta_F(s).
\]
Since $Z(f, s)$, $\zeta_F(s)$, and $Z(f_\infty, s)$ are holomorphic for $s \neq 1$ and $\Re (s) > 0$, the Lemma follows.

For every finite place $v$, $\what{f_v^0}$ is equal to $|\mathfrak{d}_v|^{-1/2} \mathds{1}_{\mathfrak{d}_{v}^{-1}}$.
Here $\mathfrak{d}_v \subset F_v$ is the different, $\mathfrak{d}_v^{-1}$ is inverse, $\mathds{1}_{\mathfrak{d}_v^{-1}}$ is the characteristic function, and $|\mathfrak{d}_v|$ is the ideal norm (a positive power of $q_v$).
Recall that
\[
    \prod_{v\text{ finite}} |\mathfrak{d}_v| = |D_F|.
\]

Consider the first integral of \eqref{eqn:4.3}:
\begin{equation}
\label{eqn:4.4}
    \int_{I_F^+} f(x) |x|^{s} \dd^\times x.
\end{equation}
If $f(x) \neq 0$ for $x = (x_\infty, x_f)$, the decomposition $f_f = \prod_{v\text{ finite}}f_v$ shows $|x_f| \leq 1$; since $|x_\infty x_f| \geq 1$,
\begin{equation}
\label{eqn:4.5}
    |x_\infty| = \prod_{v | \infty} |x_v| \geq 1.
\end{equation}
For the second integral, we have $|x_v| \leq |\mathfrak{d}_v|$ if $x_v \in \mathfrak{d}_v^{-1}$, so $|x_f| \leq \prod_v |\mathfrak{d}_v| = |D_F|$ and
\begin{equation}
\label{eqn:4.6}
    |x_\infty| \geq |D_F|^{-1}.
\end{equation}

\begin{lemma}
\label{lem:4.2}
Suppose that there exists a Fourier pair $(f, \what{f})$ on $F_\infty = \mathbb{R}^{d}$ such that $f(x_\infty) \geq 0$ if $|x_\infty| \geq 1$, $f$ is strictly positive on the neighborhood of $1$ in the set $|x_\infty| \geq 1$, $\what{f}(y_\infty) \geq 0$ if $|y_\infty| \geq D_F^{-1}$ and $f(0) = \what{f}(0) = 0$.
Then $\zeta_F(s) \neq 0$ for all $s$ in the interval $(0, 1)$.
\end{lemma}

\eqref{eqn:4.3}
In fact \eqref{eqn:4.3} allows us to focus on its integrand: $|x|^s$ is strictly positive on the domain of integration, and the integral \eqref{eqn:4.4} is strictly positive by the assumptions on $f$.
Hence $Z(f, s) > 0$ and $\zeta_F(s) \neq 0$ follows from Lemma \ref{lem:4.1}.

Let $x = (x_v) \in F_\infty$.
The Euclidean norm compatible with Tate's Fourier transform is
\[
    \|x\|^{2} = \sum_{v\text{ real}} |x_v|^2 + 2 \sum_{v\text{ complex}} \|x_v\|^2.
\]
Since
\[
    |x|^{2} = \prod_{v\text{ real}} |x_v|^{2} \prod_{v\text{ complex}} \|x_v\|^{4},
\]
arithmetic-geometric mean inequality gives
\[
    |x|^{2/d} \leq \frac{1}{d}\|x\|^{2}
\]
For $r = \|x\|$, $\rho = \|y\|$ ($y \in F_\infty$) we see that
\begin{align*}
    |x| \geq 1 &\Rightarrow r \geq \sqrt{d} \\
    |y| \geq |D_F|^{-1} &\Rightarrow \rho \geq |D_F|^{-1/d} \sqrt{d}
\end{align*}

\begin{proposition}
\label{prop:4.3}
Suppose that there exists a number field of degree $d$ and discriminant $D$ such that $\zeta_F$ has a zero in $(0, 1)$.
Then
\[
    \mathcal{B}_{d} \geq d |D|^{-1/d}.
\]
\end{proposition}

Conversely, $\zeta_F$ has no zero if
\[
    d |D|^{-1/d} > \mathcal{B}_{d}.
\]
The proof is now obvious.
Suppose $d |D|^{-1/d} > \mathcal{B}_{d}$.
As in \S 3, we can find radial $f$ and $\what{f}$ that are nonnegative for $r \geq \sqrt{d}$ and $\rho \geq |D|^{-1/d} \sqrt{d}$.
We can assume that $f$ is strictly positive for $x$ with $\sqrt{d} \leq \|x\| \leq \sqrt{d} + \varepsilon$.
Then the assumptions for Lemma \ref{lem:4.2} are satisfied since $\|1\| = \sqrt{d}$.

It is difficult to find a field $F$ satisfying the hypothesis of Proposition \ref{prop:4.3}.
However, the decomposition of $\zeta_F(s)$ in terms of Artin $L$-functions of Galois extensions $E$ over $F$ allowed Armitage to exhibit such a zero (which is $s = 1/2$, as predicted by Riemann's hypothesis).
More precisely, Armitage considered an explicit extension $F$ over $E = \mathbb{Q}(\sqrt{3(1+i)})$ of degree $12$ constructed by Serre \cite{serre1971conducteurs}, which is of degree $48$ over $\mathbb{Q}$ and satisfies $\zeta_F\left(\frac{1}{2}\right) = 0$ \cite[\S 4]{armitage1971zeta}.

As a consequence, we have a weaker version of Theorem \ref{thm:3.1} from number theory.

\begin{proposition}
\label{prop:4.4}
For $d$ multiple of $48$, $\mathcal{B}_{d}$ is strictly positive.
\end{proposition}

For $d = 48$, this follows from the existence of $F$.
Assume that $d = 48c$.
There exists a cyclotomic extension $L$ over $\mathbb{Q}$ linearly disjoint with $F$.
Then $LF$ has degree $d$ over $\mathbb{Q}$, and $\zeta_F$ divides $\zeta_{LF}$ since $LF/F$ is abelian, and $\zeta_{LF}$ factorizes as a product of Dirichlet $L$-functions over $F$.
Hence the result follows.

You may wonder if Proposition \ref{prop:4.4} provides any restriction on the discriminant of a number field where $\zeta_F$ has a real zero.
In this case, we have
\begin{equation}
    \label{eqn:4.7}
    |D|^{1/d} \geq \frac{d}{\mathcal{B}_d}.
\end{equation}
By Theorem \ref{thm:3.1},
\[
    \frac{d}{\mathcal{B}_d} < 2 \pi e = 17.079\cdots.
\]
Odlyzyko \cite{odlyzko1977lower} proved a general unconditional bound 
\[
    |D|^{1/d} \geq 22.2(1 + o(d))
\]
for $d \to \infty$.
As result we get \eqref{eqn:4.7}, at least for large enough $d$.

Hence Proposition \ref{prop:4.4} does not give any interesting improvement of the lower bound of $\mathcal{B}_{d}$.
However, it is striking to note that, at least for some degrees, number theory provides a linear growth in $d$ given by Theorem \ref{thm:3.1}.
Let $p$ be a prime number
By theorems of Golod-Shafarevi\v{c} and Brumer, there exists a tower of number fields
\[
    E_p^1 \subset E_p^2 \subset \cdots \subset E_p^n \subset \cdots
\]
where $E_p^1$, that has degree $p(p-1)$ over $\mathbb{Q}$, is a degree $p$ extension of $\mathbb{Q}(\zeta_p)$, and $E_{p}^{n+1} / E_{p}^{n}$ is unramified are unramified extensions of degree $p$.
See \cite[Cor 7]{towers1967peter}; we adjoint $\zeta_p$ by two successive abelian extensions of $\mathbb{Q}$ to obtain $E_p^1$.

Consider the series of extensions $F_i = F E_p^i$ of $F_i$, where $F_{i+1}/F_{i}$ is abelian of degree $1$ or $p$.
We can extract a minimal, strictly increasing subsequence
\[
    F_0 = F E_{p}^{n_0} \subset F_{1} \subset \cdots \subset F_{m} = F E_{p}^{n_m}
\]
where each extension is abelian of degree $p$.
Since the extensions are all relatively unramified, a classical formula for absolute discriminants gives
\begin{equation}
    \label{eqn:4.8}
    D_{F_m} = D_{F_0}^{p^m} =: D^{p^m}.
\end{equation}
The successive extensions of $F$ are abelian, so $\zeta_F$ divides $\zeta_{F_m}$ for all $m$.
Then Proposition \ref{prop:4.3} shows that for $d = d_0 p^m$, $d_0 = [F_0:\mathbb{Q}]$:
\begin{equation}
    \label{eqn:4.9}
    \mathcal{B}_d \geq C d, \quad C = |D|^{-1/d_0}.
\end{equation}

For such degress, \eqref{eqn:3.11} and \eqref{eqn:4.9} shows that the growth of $\mathcal{B}_d$ - so is $B_d \geq \frac{1}{2}\mathcal{B}_d$, is linear in $d$.
If $p$ does not divide $D_F$, $F$ and $\mathbb{Q}(\zeta_p)$ are linearly disjoint and we can choose $E_p^1$ to be linearly disjoint with $F$.
Then $F_0 = F E_p^1$ and the inequality \eqref{eqn:4.8} is valid for $d = 48(p-1)p^n$, $n \geq 1$.
Of course, the $(p-1)$ term is not necessary if one assumes Artin' conjecture or Dedekind's divisibility conjecture. (Dedekind's conjecture claims that $\zeta_F(s)$ is divisible by $\zeta_E(s)$ for all extensions $E/F$. Then you can choose $E_p^1$, possibly non-Galois, to be degree $p$ over $\mathbb{Q}$.
Then the Artin's conjecture on the holomorphicity of non-abelian $L$-functions implies Dedekind's conjecture.)




% --- Bibliography ---

% Start a bibliography with one item.
% Citation example: "\cite{williams}".

% \nocite{*}
\bibliographystyle{acm} % We choose the "plain" reference style
\bibliography{refs} % Entries are in the refs.bib file


% \begin{thebibliography}{1}

% \bibitem{williams}
%    Williams, David.
%    \textit{Probability with Martingales}.
%    Cambridge University Press, 1991.
%    Print.

% % Uncomment the following lines to include a webpage
% % \bibitem{webpage1}
% %   LastName, FirstName. ``Webpage Title''.
% %   WebsiteName, OrganizationName.
% %   Online; accessed Month Date, Year.\\
% %   \texttt{www.URLhere.com}

% \end{thebibliography}

% --- Document ends here ---

\end{document}