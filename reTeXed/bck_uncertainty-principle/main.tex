% --- LaTeX Lecture Notes Template - S. Venkatraman ---

% --- Set document class and font size ---

\documentclass[letterpaper, 12pt]{article}

% --- Package imports ---

% Extended set of colors
\usepackage[dvipsnames]{xcolor}

\usepackage{
  amsmath, amsthm, amssymb, mathtools, dsfont, units,          % Math typesetting
  graphicx, wrapfig, subfig, float,                            % Figures and graphics formatting
  listings, color, inconsolata, pythonhighlight,               % Code formatting
  fancyhdr, sectsty, hyperref, enumerate, enumitem, framed }   % Headers/footers, section fonts, links, lists

% lipsum is just for generating placeholder text and can be removed
\usepackage{hyperref, lipsum} 

% --- Fonts ---

\usepackage{newpxtext, newpxmath, inconsolata}
\usepackage{amsfonts}

\usepackage{tikz}
\usepackage{tikz-cd}
\usepackage{enumitem}
\usepackage[title]{appendix}
\usepackage{mathdots}
\usepackage{stmaryrd}

% --- Page layout settings ---

% Set page margins
\usepackage[left=1.35in, right=1.35in, top=1.0in, bottom=.9in, headsep=.2in, footskip=0.35in]{geometry}

% Anchor footnotes to the bottom of the page
\usepackage[bottom]{footmisc}

% Set line spacing
\renewcommand{\baselinestretch}{1.2}

% Set spacing between paragraphs
\setlength{\parskip}{1.3mm}

% Allow multi-line equations to break onto the next page
\allowdisplaybreaks

% --- Page formatting settings ---

% Set image captions to be italicized
\usepackage[font={it,footnotesize}]{caption}

% Set link colors for labeled items (blue), citations (red), URLs (orange)
\hypersetup{colorlinks=true, linkcolor=RoyalBlue, citecolor=RedOrange, urlcolor=ForestGreen}

% Set font size for section titles (\large) and subtitles (\normalsize) 
\usepackage{titlesec}
% \titleformat{\section}{\large\bfseries}{{\fontsize{19}{19}\selectfont\textreferencemark}\;\; }{0em}{}
\titleformat{\section}{\large\bfseries}{\thesection\;\;\;}{0em}{}
\titleformat{\subsection}{\normalsize\bfseries\selectfont}{\thesubsection\;\;\;}{0em}{}

% Enumerated/bulleted lists: make numbers/bullets flush left
%\setlist[enumerate]{wide=2pt, leftmargin=16pt, labelwidth=0pt}
\setlist[itemize]{wide=0pt, leftmargin=16pt, labelwidth=10pt, align=left}

% --- Table of contents settings ---

\usepackage[subfigure]{tocloft}

% Reduce spacing between sections in table of contents
\setlength{\cftbeforesecskip}{.9ex}

% Remove indentation for sections
\cftsetindents{section}{0em}{0em}

% Set font size (\large) for table of contents title
\renewcommand{\cfttoctitlefont}{\large\bfseries}

% Remove numbers/bullets from section titles in table of contents
\makeatletter
\renewcommand{\cftsecpresnum}{\begin{lrbox}{\@tempboxa}}
\renewcommand{\cftsecaftersnum}{\end{lrbox}}
\makeatother

% --- Set path for images ---

\graphicspath{{Images/}{../Images/}}

% --- Math/Statistics commands ---

% Add a reference number to a single line of a multi-line equation
% Usage: "\numberthis\label{labelNameHere}" in an align or gather environment
\newcommand\numberthis{\addtocounter{equation}{1}\tag{\theequation}}

% Shortcut for bold text in math mode, e.g. $\b{X}$
\let\b\mathbf

% Shortcut for bold Greek letters, e.g. $\bg{\beta}$
\let\bg\boldsymbol

% Shortcut for calligraphic script, e.g. %\mc{M}$
\let\mc\mathcal

% \mathscr{(letter here)} is sometimes used to denote vector spaces
\usepackage[mathscr]{euscript}

% Convergence: right arrow with optional text on top
% E.g. $\converge[p]$ for converges in probability
\newcommand{\converge}[1][]{\xrightarrow{#1}}

% Weak convergence: harpoon symbol with optional text on top
% E.g. $\wconverge[n\to\infty]$
\newcommand{\wconverge}[1][]{\stackrel{#1}{\rightharpoonup}}

% Equality: equals sign with optional text on top
% E.g. $X \equals[d] Y$ for equality in distribution
\newcommand{\equals}[1][]{\stackrel{\smash{#1}}{=}}

% Normal distribution: arguments are the mean and variance
% E.g. $\normal{\mu}{\sigma}$
\newcommand{\normal}[2]{\mathcal{N}\left(#1,#2\right)}

% Uniform distribution: arguments are the left and right endpoints
% E.g. $\unif{0}{1}$
\newcommand{\unif}[2]{\text{Uniform}(#1,#2)}

% Independent and identically distributed random variables
% E.g. $ X_1,...,X_n \iid \normal{0}{1}$
\newcommand{\iid}{\stackrel{\smash{\text{iid}}}{\sim}}

% Sequences (this shortcut is mostly to reduce finger strain for small hands)
% E.g. to write $\{A_n\}_{n\geq 1}$, do $\bk{A_n}{n\geq 1}$
\newcommand{\bk}[2]{\{#1\}_{#2}}

% \setcounter{section}{-1}

\newcommand{\SL}{\mathrm{SL}}
\newcommand{\Sp}{\mathrm{Sp}}
\newcommand{\Mp}{\mathrm{Mp}}
\newcommand{\GL}{\mathrm{GL}}
\newcommand{\SO}{\mathrm{SO}}
\newcommand{\SU}{\mathrm{SU}}
\newcommand{\PGL}{\mathrm{PGL}}
\newcommand{\PSL}{\mathrm{PSL}}
\newcommand{\rM}{\mathrm{M}}
\newcommand{\rN}{\mathrm{N}}
\newcommand{\rO}{\mathrm{O}}
\newcommand{\rP}{\mathrm{P}}
\newcommand{\rH}{\mathrm{H}}
\newcommand{\rU}{\mathrm{U}}
\newcommand{\JL}{\mathrm{JL}}
\newcommand{\stab}{\mathrm{Stab}}
\newcommand{\cusp}{\mathrm{cusp}}
\newcommand{\reg}{\mathrm{reg}}
\newcommand{\rs}{\mathrm{rs}}
\newcommand{\Irr}{\mathrm{Irr}}
\newcommand{\Tr}{\mathrm{Tr}}
\newcommand{\Hom}{\mathrm{Hom}}
\newcommand{\Gal}{\mathrm{Gal}}
\newcommand{\WD}{\mathrm{WD}}
\newcommand{\Frob}{\mathrm{Frob}}
\newcommand{\Res}{\mathrm{Res}}
\newcommand{\Tam}{\mathrm{Tam}}
\newcommand{\Pet}{\mathrm{Pet}}
\newcommand{\sgn}{\mathrm{sgn}}
\newcommand{\vol}{\mathrm{vol}}
\newcommand{\Aut}{\mathrm{Aut}}
\newcommand{\Ind}{\mathrm{Ind}}
\newcommand{\BC}{\mathrm{BC}}
\newcommand{\Ad}{\mathrm{Ad}}

\newcommand{\what}{\widehat}

\newcommand{\dd}{\mathrm{d}}

\newcommand{\bA}{\mathbb{A}}
\newcommand{\bR}{\mathbb{R}}
\newcommand{\bS}{\mathbb{S}}
\newcommand{\bZ}{\mathbb{Z}}
\newcommand{\bC}{\mathbb{C}}
\newcommand{\bQ}{\mathbb{Q}}
\newcommand{\bH}{\mathbb{H}}
\newcommand{\bfi}{\mathbf{I}}
\newcommand{\bfa}{\mathbf{a}}
\newcommand{\bfb}{\mathbf{b}}
\newcommand{\cS}{\mathcal{S}}
\newcommand{\cO}{\mathcal{O}}
\newcommand{\cV}{\mathcal{V}}
\newcommand{\cP}{\mathcal{P}}

\newcommand{\scA}{\mathscr{A}}
\newcommand{\scB}{\mathscr{B}}
\newcommand{\scV}{\mathscr{V}}
\newcommand{\scT}{\mathscr{T}}
\newcommand{\scU}{\mathscr{U}}
\newcommand{\scW}{\mathscr{W}}
\newcommand{\scO}{\mathscr{O}}
\newcommand{\scL}{\mathscr{L}}

\newcommand{\frh}{\mathfrak{h}}
\newcommand{\frt}{\mathfrak{t}}
\newcommand{\frg}{\mathfrak{g}}
\newcommand{\frgl}{\mathfrak{gl}}
\newcommand{\fru}{\mathfrak{u}}

% Math mode symbols for common sets and spaces. Example usage: $\R$
\newcommand{\R}{\mathbb{R}}	% Real numbers
\newcommand{\C}{\mathbb{C}}	% Complex numbers
\newcommand{\Q}{\mathbb{Q}}	% Rational numbers
\newcommand{\Z}{\mathbb{Z}}	% Integers
\newcommand{\N}{\mathbb{N}}	% Natural numbers
\newcommand{\F}{\mathcal{F}}	% Calligraphic F for a sigma algebra
\newcommand{\El}{\mathcal{L}}	% Calligraphic L, e.g. for L^p spaces

% Math mode symbols for probability
\newcommand{\pr}{\mathbb{P}}	% Probability measure
\newcommand{\E}{\mathbb{E}}	% Expectation, e.g. $\E(X)$
\newcommand{\var}{\text{Var}}	% Variance, e.g. $\var(X)$
\newcommand{\cov}{\text{Cov}}	% Covariance, e.g. $\cov(X,Y)$
\newcommand{\corr}{\text{Corr}}	% Correlation, e.g. $\corr(X,Y)$
\newcommand{\B}{\mathcal{B}}	% Borel sigma-algebra

% Other miscellaneous symbols
\newcommand{\tth}{\text{th}}	% Non-italicized 'th', e.g. $n^\tth$
\newcommand{\Oh}{\mathcal{O}}	% Big-O notation, e.g. $\O(n)$
\newcommand{\1}{\mathds{1}}	% Indicator function, e.g. $\1_A$

% Additional commands for math mode
\DeclareMathOperator*{\argmax}{argmax}		% Argmax, e.g. $\argmax_{x\in[0,1]} f(x)$
\DeclareMathOperator*{\argmin}{argmin}		% Argmin, e.g. $\argmin_{x\in[0,1]} f(x)$
\DeclareMathOperator*{\spann}{Span}		% Span, e.g. $\spann\{X_1,...,X_n\}$
\DeclareMathOperator*{\bias}{Bias}		% Bias, e.g. $\bias(\hat\theta)$
\DeclareMathOperator*{\ran}{ran}			% Range of an operator, e.g. $\ran(T) 
\DeclareMathOperator*{\dv}{d\!}			% Non-italicized 'with respect to', e.g. $\int f(x) \dv x$
\DeclareMathOperator*{\diag}{diag}		% Diagonal of a matrix, e.g. $\diag(M)$
\DeclareMathOperator*{\trace}{Tr}		% Trace of a matrix, e.g. $\trace(M)$
\DeclareMathOperator*{\supp}{supp}		% Support of a function, e.g., $\supp(f)$

% Numbered theorem, lemma, etc. settings - e.g., a definition, lemma, and theorem appearing in that 
% order in Lecture 2 will be numbered Definition 2.1, Lemma 2.2, Theorem 2.3. 
% Example usage: \begin{theorem}[Name of theorem] Theorem statement \end{theorem}
\theoremstyle{definition}
\newtheorem{theorem}{Theorem}[section]
\newtheorem{conjecture}{Conjecture}[section]
\newtheorem{proposition}[theorem]{Proposition}
\newtheorem{lemma}[theorem]{Lemma}
\newtheorem{corollary}[theorem]{Corollary}
\newtheorem{definition}[theorem]{Definition}
\newtheorem{example}[theorem]{Example}
\newtheorem{remark}[theorem]{Remark}

% Un-numbered theorem, lemma, etc. settings
% Example usage: \begin{lemma*}[Name of lemma] Lemma statement \end{lemma*}
\newtheorem*{theorem*}{Theorem}
\newtheorem*{proposition*}{Proposition}
\newtheorem*{lemma*}{Lemma}
\newtheorem*{corollary*}{Corollary}
\newtheorem*{definition*}{Definition}
\newtheorem*{example*}{Example}
\newtheorem*{remark*}{Remark}
\newtheorem*{claim}{Claim}
\newtheorem*{question*}{Question}
\newtheorem*{problem*}{Problem}

% --- Left/right header text (to appear on every page) ---

% Do not include a line under header or above footer
\pagestyle{fancy}
\renewcommand{\footrulewidth}{0pt}
\renewcommand{\headrulewidth}{0pt}

% Right header text: Lecture number and title
\renewcommand{\sectionmark}[1]{\markright{#1} }
% \fancyhead[R]{\small\textit{\nouppercase{\rightmark}}}

% Left header text: Short course title, hyperlinked to table of contents
% \fancyhead[L]{\hyperref[sec:contents]{\small Gan-Gross-Prasad conjecture}}

% --- Document starts here ---

\begin{document}

% --- Main title and subtitle ---

\title{Heisenberg's principle and positive functions \\[1em]
\normalsize Principe d'Heisenberg et fonctions positives \\ - \\
\normalsize Re-\TeX ed by Seewoo Lee\footnote{seewoo5@berkeley.edu. Most of the translation is due to Google Translator, and I only fixed a little.}}

% --- Author and date of last update ---

\author{Jean Bourgain, Laurent Clozel, Jean-Pierre Kahane}
\date{\normalsize\vspace{-1ex} Last updated: \today}

% --- Add title and table of contents ---

\maketitle


% --- Abstracts ---

% \tableofcontents\label{sec:contents}
\begin{abstract}
We consider a natural problem concerning Fourier transforms.
In one variable, one seeks functions $f$ and $\what{f}$, both positive for $|x| \geq a$ and vanishing at $0$.
What is the lowest bound for $a$?
In higher dimension, the same problem can be posed by replacing the interval by the ball of radius $a$.
We show that there is indeed a strictly positive lower bound, which is estimated as a function of the dimension.
In the last section the question, and its solution, are shown to be naturally related to the theory of zeta functions.
\end{abstract}

% --- Main content: import lectures as subfiles ---


\section{Introduction}

\begin{conjecture}[Langlands functoriality conjecture] Let $G$ and $G'$ be reductive groups over a global field $F$. 
\end{conjecture}
This is an introductory note on Langlands functoriality conjecture view towards classical examples. Here is a list of topics we are going to study:

\begin{enumerate}
    \item Automorphic induction
    \item Base change
    \item Rankin-Selberg product
    \item Symmetric power lifting and Selberg's 1/4 conjecture
    \item Jacquet-Langlands correspondence
    \item Theta correspondence and Howe duality
\end{enumerate}
\section{Statement of the problem and lower bound of $B_{1}$}

Consider a pair of functions $(f, \what{f})$ on reals: they are Fourier pairs if
\[
    \begin{cases}
    \what{f}(y) = \int f(x) e^{-2 i \pi x y} \dd x, \quad f \in L^{1}(\mathbb{R}) \\
    f(x) = \int \what{f}(y) e^{2 i \pi x y} \dd y, \quad \what{f} \in L^{1}(\mathbb{R}).
    \end{cases}
\]
So $f$ and $\what{f}$ are continuous and converges to $0$ at infinity.
We are interested in the Fourier pairs $(f, \what{f})$ such that
\begin{enumerate}
    \item $f$ and $\what{f}$ are real-valued, even, and not identically zero,
    \item $f(0) \leq 0$ and $\what{f}(0) \geq 0$,
    \item $f(x) \geq 0$ for $x \geq a_f$ and $\what{f}(y) \geq 0$ for $y \geq a_{\what{f}}$.
\end{enumerate}
Note that the condition 2 and the non-vanishing assumptions on $f$ and $\what{f}$ imply $a_f$ and $a_{\what{f}} > 0$.

\begin{problem*}
What is the infimum of the product $a_{f} a_{\what{f}}$ for the Fourier pairs $(f, \what{f})$ satisfying 1--3?
\end{problem*}

We denote the infimum as $B_1 \geq 0$ (note that the pair attaining infimum clearly exists).
We will show, which is not obvious a priori, that $B_1$ is strictly positive.

Until section 3, we will focus on dimension $1$.
For a Fourier pair $(f, \what{f})$ satisfying 1--3 let
\begin{align*}
    A(f) &= \inf\{x > 0: f((x, \infty)) \subset \mathbb{R}^{+}\} \\
    A(\what{f}) &= \inf\{y > 0: \what{f}((t, \infty)) \subset \mathbb{R}^{+}\} .
\end{align*}
The product $A(f)A(\what{f})$ is invariant under scaling, i.e. replacing $f(x)$, $\what{f}(y)$ by $f(x /\lambda)$, $\lambda \what{f}(\lambda y)$, $\lambda > 0$.
Since
\[
    B_1 = \inf A(f) A(\what{f})
\]
for all Fourier pairs satisfying 1--3, we only consider pairs satisfying $A(f) = A(\what{f})$.
Then $f + \what{f} \neq 0$ (consider their values at points near $A(f)$), and
\[
    A(f + \what{f}) \leq A(f) = A(\what{f}).
\]
So $B_1 = \inf A^{2}(f + \what{f})$. Hence we see that
\[
    B_1 = A^2, \quad A = \inf A(f)
\]
where infimum is taken over all functions $f \in L^{1}(\mathbb{R})$, real-valued and even, not identically zero, equal to their own Fourier transforms, and $f(0) < 0$.

Let
\[
    \gamma(x) = e^{-\pi x^2}
\]
so that $\gamma = \what{\gamma}$.
If $f(0) < 0$, $f - f(0)\gamma$ satisfies the same conditions as $f$, and
\[
    A(f - f(0)\gamma) \leq A(f).
\]
Finally,
\begin{equation}
\label{eqn:1.1}
    A = \inf A(f)
\end{equation}
\textbf{where infimum is taken over all $f \in L^{1}(\mathbb{R})$, real-valued, even, not identically zero, $f = \what{f}$, and $f(0) = 0$.}

Here is an important result.

\begin{theorem}
Let $\lambda = -\inf\left( \frac{\sin x}{x}\right) = 0.2712\cdots$.
Then
\[
    A \geq \frac{1}{2(1 + \lambda)} = 0.4107\cdots
\]
so
\[
    B \geq 0.1687\cdots.
\]
\end{theorem}
\begin{proof}
Choose $f = \what{f}$, $f(0) = 0$, and $\int_{\mathbb{R}} |f(x)| \dd x := \int_{\mathbb{R}} |f| = 1$.
Write $A = A(f)$.
Put $f = f^{+} - f^{-}$, $|f| = f^{+} + f^{-}$.
Since $\int_{\mathbb{R}}f = \what{f}(0) = 0$, we have $\int_{\mathbb{R}} f^{+} = \int_{\mathbb{R}} f^{-} = \int_{-A}^{A} f^{-} = \frac{1}{2}$.
So $\int_{-A}^{A} |f| \geq \frac{1}{2}$.
From $|f(x)| \leq \int |\what{f}| = 1$, $2A \geq \frac{1}{2}$ and we obtain a first bound $A \geq \frac{1}{4}$.
We will see that this argument extends to higher dimensions.

In dimension $1$, we can refine it in the following way.
From $f = \what{f}$,
\begin{align*}
    f(x) &= \int f(y) \cos 2 \pi y x \dd y = \int f(y) (\cos 2 \pi y x - 1) \dd y \\
    &= \int f^{-}(y) (1 - \cos 2 \pi y x) \dd y - \int f^{+}(y) (\cos 2 \pi y x - 1) \dd y.
\end{align*}
This implies, ???
\[
    f^{-}(x) \leq \int f^{+}(y) (1 - \cos 2 \pi y x) \dd y
\]
and
\[
    \frac{1}{4} = \int_{0}^{A} f^{-} \leq \int_{-\infty}^{\infty} f^{+}(y) \left(A - \frac{\sin 2 \pi y A}{2 \pi y}\right) \dd y
\]
so
\[
    \frac{1}{4} \leq \frac{A}{2}\sup_{u \in \mathbb{R}} \left(1 - \frac{\sin u}{u}\right) = \frac{A}{2}(1 + \lambda)
\]
and we obtain the theorem.
\end{proof}
Later, we will need to consider functions that are regular enough.
A natural class is the Schwartz space $\mathcal{S}$.
It is not obvious that the infimum $A$ defined by \eqref{eqn:1.1}, taken only over the functions in $\mathcal{S}$, coincides with that over all $f \in L^{1}(\mathbb{R})$.

Let $\mathcal{B}_{1}$ be $A^{2}$, where $A$ is defined by \eqref{eqn:1.1} for $f \in \mathcal{S}$.
We will see that $B_{1}$ and $\mathcal{B}_{1}$ are not much different.
Clearly, we have
\begin{equation}
    \label{eqn:1.2}
    B_{1} \leq \mathcal{B}_{1}.
\end{equation}
Let
\[
    B_{1}^{-} = \inf \{A^{2}: f(0) < 0, f = \what{f}\text{ even} \neq 0, f \in L^{1}(\mathbb{R})\}.
\]
Hence $B_{1}^{-}$ is defined by \eqref{eqn:1.1}, with additional assumption $f(0) < 0$.
Define $\mathcal{B}_{1}^{-}$ similarly for $f \in \mathcal{S}$.
Clearly,
\begin{equation}
    \label{eqn:1.3}
    B_{1}^{-} \leq \mathcal{B}_{1}^{-}
\end{equation}
\begin{equation}
    \label{eqn:1.4}
    \mathcal{B}_{1} \leq \mathcal{B}_{1}^{-}, \quad B_{1} \leq B_{1}^{-}.
\end{equation}
To prove $\mathcal{B}_{1}^{-} \leq B_{1}^{-}$, let $f \in L^{1}(\mathbb{R})$ be a function satisfying the conditions for \eqref{eqn:1.1} but $f(0) < 0$, and let $a = A(f)$.
Let $\varphi = \psi \ast \psi$, where $\psi$ is $C^{\infty}$, even, positive, and compactly supported near $0$, and $g = f \ast \varphi$.
Then $A(g) \leq a + \varepsilon$ and $g(0) < 0$.
We have $\what{g} = \what{f} \what{\psi}^{2}$; by applying the same operation on $\what{g}$ we obtain a function $h \in \mathcal{S}$ such that $h = \what{h}$, $h(0) < 0$, and $A(h) \leq a + \varepsilon$; from this we get $\mathcal{B}_{1}^{-} \leq B_{1}^{-}$ and
\begin{equation}
    \label{eqn:1.5}
    \mathcal{B}_{1}^{-} = B_{1}^{-}.
\end{equation}
Note that the argument does not work if $f(0) = 0$.
We will show
\begin{equation}
    \label{eqn:1.6}
    B_{1}^{-} \leq 2B_{1};
\end{equation}
combining \eqref{eqn:1.4} and \eqref{eqn:1.6} we obtain
\begin{equation}
    \label{eqn:1.7}
    B_{1} \leq \mathcal{B}_{1} \leq 2 B_{1}.
\end{equation}
Let $f$ be a function satisfying the conditions for \eqref{eqn:1.1} and $a = A(f)$.
Since $\what{f}(0) = \int f(x) \dd x = 0$, $f$ takes a negative value on $[-a, a]$.
Let $b > 0$ be such a number, and consider the distribution
\[
    T = \delta_{b} + \delta_{-b} + 2 \delta_{0}.
\]
It is a positive measure with positive Fourier transform
\[
    \what{T} = 2 \cos (2 \pi b y) + 2 \geq 0.
\]
We have
\[
    (T \ast f)(0) = f(b) + f(-b) < 0.
\]
Since $b < a$, $g = T \ast f$ satisfies
\[
    g(0) < 0, \quad g \geq 0 \text{ on }(2a, \infty).
\]
Moreover $\what{g} = \what{T}\what{f}$ is nonnegative on $[0, \infty)$, and $\what{g}(0) = 0$.
By scaling, we obtain a function $h$ such that
\begin{align*}
    h &\geq 0 \text{ on } [a \sqrt{2}, \infty),\quad h(0) < 0 \\
    \what{h} &\geq 0 \text{ on } [a \sqrt{2}, \infty),\quad \what{h}(0) = 0. \\
\end{align*}
The functions $h$ and $\what{h}$ are real-valued and even.
Hence $h + \what{h}$ satisfy the conditions defining $B_{1}^{-}$.
So $B_{1}^{-} \leq (a \sqrt{2})^{2} = 2a^{2}$; by varying $f$, we obtain \eqref{eqn:1.6}.

\section{Upper bound of $B_{1}$}

An important idea is to use Hermite series
\[
    f(x) \sim \sum_{n = 0}^{\infty}a_n h_n(x)
\]
associated to $f$, where $h_n$ are eigenvectors of the Fourier transform $\mathcal{F}$ corresponding to the eigenvalues $i^n$.
Since $f = \what{f}$ the expression becomes
\[
    f(x) \sim \sum_{m=0}^{\infty} a_{4m} h_{4m}(x).
\]
Each $h_{n}$ has a form of $h_{n} = e^{-\pi x^2} P_{n}(x)$ where $P_{n}$ is a polynomial of degree $n$.
A suitable linear combination of $h_{0}$ and $h_{4}$ (satisfying $f(0) = 0$) gives $\pi A^{2} \leq 3$.
The calculations seem difficult and we will not proceed in this direction further.

We can also consider the functions
\begin{equation}
    \label{eqn:2.1}
    g_{a}(x) = a \gamma(ax) + \gamma\left(\frac{x}{a}\right) - (1 + a) \gamma(x),\quad  a > 1
\end{equation}
which satisfy the requirements for \eqref{eqn:1.1}.
Then any expression of the form
\begin{equation}
    \label{eqn:2.2}
    \int_{1}^{\infty} g_{a}(x) \dd \tau(a)
\end{equation}
where $\tau$ is a measure on $[1, \infty)$ such that the integral converges absolutely and positive is our candidates (it seems difficult to characterize such measures where \eqref{eqn:2.2} converges absolutely and positive).

We first study $A(g_{a})$.
It is convenient to put $X = \pi x^2$, and $G_{a}(X) = g_{a}(x)$, so
\[
    G_{a}(X) = a e^{-a^{2}X} + e^{-a^{-2}X} - (1 + a)e^{-X}.
\]
The function
\begin{equation}
    \label{eqn:2.3}
    H_{a}(X) = e^{X} G_{a}(X) = a e^{(1 - a^{2})X} + e^{(1 - a^{-2})X} - 1 - a
\end{equation}
is convex and satisfying
\[
    H_{a}(0) = 0, \quad H_{a}'(0) = -a^{2}(a^{2} - 1)(a^{3} - 1) < 0
\]
and tends to $+\infty$ as $X \to \pm \infty$.
So it has a unique zero $X_{a} > 0$, and
\[
    A(g_{a}) = \sqrt{\frac{X_{a}}{\pi}}.
\]
It is natural to study with varying $X_{a}$, and we first consider those for $a$ near $1$.
Put $a = 1 + h$, $h > 0$, then $H_{a}(X)$ can be written as
\[
    H_{a}(X) = (1 + h)(e^{-X(2h + h^2)} - 1) + e^{X(2h - 3h^{2} + 3h^{3} - 4h^{4})X} - 1
\]
modulo $O(h^{5})$.
It can be written as $P_{1}h + P_{2}h^{2} + P_{3}h^{3} + P_{4}h^{4} + O(h^{5})$, where the polynomials $P_{i}$ are
\begin{align*}
    P_{1} &= 0 \\
    P_{2} &= 2X(2X - 3)\\
    P_{3} &= -X(2X - 3) \\
    P_{4} &= -5X + 15 X^{2} - \frac{28}{3} X^{3} + \frac{4}{3}X^{4}.
\end{align*}
From the expression of $P_{2}$, for sufficiently small $h$, $H_a(X) > 0$ if $X > \frac{3}{2}$ and $H_a(X) < 0$ if $X < \frac{3}{2}$.
As a result,
\begin{equation}
    \label{eqn:2.4}
    \lim_{a \to 1^+} X_a = \frac{3}{2}.
\end{equation}
This provides an explicit bound
\begin{equation}
    \label{eqn:2.5}
    A \leq \sqrt{\frac{3}{2 \pi}}.
\end{equation}

But this simple bound cannot be the true value of $A$.
For $X = \frac{3}{2}$, $P_2$ and $P_3$ cancel out, and
\[
    P_4\left(\frac{3}{2}\right) = \frac{3}{2}.
\]
For nonzero small $h$, we therefore have $X_a < \frac{3}{2}$.

If $a \to +\infty$, $X_a \to +\infty$; in fact, a simple calculation shows that
\[
    X_a = \log a + O(1) \quad (a \to +\infty).
\]
We have not determined the minimum value of $X_a$, but it is easy to estimate it, in a semi-heuristic way.
The value $a = \sqrt{2}$ satisfies, for $q = e^{\frac{1}{2}X_a}$,
\[
    q^3 - (1+\sqrt{2})q^2 + \sqrt{2} = 0;
\]
if $q \neq 1$, it becomes the quadratic equation
\[
    q^2 - \sqrt{2} q - \sqrt{2} = 0
\]
with a zero $q = \frac{\sqrt{2}}{2}(1 + \sqrt{1+2\sqrt{2}})$,
\[
    X_a = 2 \log q = 1.4749\cdots < \frac{3}{2}\quad(a = \sqrt{2}).
\]
The value $a = 2$ gives, for $q = e^{\frac{3}{4}X}$,
\[
    q^4 - 2 \frac{q^4 - 1}{q - 1} = 0.
\]
The unique zero $q > 1$ is $q = 2.9744\cdots$, where
\[
    X_a = 1.4534\cdots\quad(a = 2).
\]
It seems that we can approximate the optimal value by this method.
Indeed, if we solve $H_a(X) = 0$ for $H_a$ given by \eqref{eqn:2.3}, and if we assume $a \geq 2$, the first term is negligible.
So $X_a$ is approximately
\[
    \frac{\log(1+a)}{1 - a^{-2}}.
\]
The extremal value of this expression is attained when $a(1-a) = 2 \log(1+a)$, which gives
\[
    a = 2.08137\cdots.
\]

In all cases, the minimum value of $A(g_a)$ we obtain is not the value for \eqref{eqn:1.1} that we are looking for.
Consider $a_0$ such that $X_0 = X_{a_0}$ is minimal, and $H_0 = H_{a_0}$ is positive on $[X_a, \infty)$.
Let $a$ be a number (for example, near $1$) such that $X_a > X_0$.
On $[X_a, \infty)$, $H_a \geq 0$ and 

% --- Bibliography ---

% Start a bibliography with one item.
% Citation example: "\cite{williams}".

\nocite{*}
\bibliographystyle{acm} % We choose the "plain" reference style
\bibliography{refs} % Entries are in the refs.bib file


% \begin{thebibliography}{1}

% \bibitem{williams}
%    Williams, David.
%    \textit{Probability with Martingales}.
%    Cambridge University Press, 1991.
%    Print.

% % Uncomment the following lines to include a webpage
% % \bibitem{webpage1}
% %   LastName, FirstName. ``Webpage Title''.
% %   WebsiteName, OrganizationName.
% %   Online; accessed Month Date, Year.\\
% %   \texttt{www.URLhere.com}

% \end{thebibliography}

% --- Document ends here ---

\end{document}