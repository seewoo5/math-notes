\section*{Introduction}


The inequalities of Heisenberg's experiments, with the notations of the present article, have the form
\[
    \int x^2 |f(x)|^2 \dd x \int y^2 |\what{f}(y)|^2 \dd y \geq 1 / 16 \pi^2
\]
(if $f$ is of norm $1$)s, and they are optimal, since equality holds for $f(x) = e^{-\pi x^2}$.
In the following form
\[
    \Delta p \Delta x \geq \hbar
\]
they are interpreted by physicists as a relationship between the spatial extension of a quantum object and the width of the spectrum of values of its momentum; when these relations appeared, they are called as "uncertainty relations", since it is a question about determining the position and the momentum of a point precisely.
For mathematicians, this is a simple observation, but with varying aspects: for a Fourier pair, one cannot make both to be concentrated near $0$.
This common but important fact is called Heisenberg's principle.

In this article, we introduce an another problem about positivity outside a neighborhood of zero.
This would not give anything new if we do not add a new condition, for both functions, to be negative at zero.
Can the neighborhoods of zero where the functions are positive outside of those be arbitrarily small?
We will see that the answer is negative.

The problem, and the first answers for higher dimensions, comes from number theory, more precisely from Tate's theory of zeta functions of number fields.
The functional equation of ad\'elic zeta functions proposes the problem in the most natural way, as we show in \S 4 of this article; however, it is already implicitly proposed in the classical article of Landau \cite{landau1918uber}.
But the classical Fourier analysis gives the best results, as shown in the first three sections.
The results are about lower and upper bounds of natural constants associated to the problem, denoted as $B_d$ and $\mathcal{B}_d$ in terms of the dimension $d$.
They are far from being optimal.
Section 1 and 2 treats upper and lower bounds for $d = 1$; section 3 for the case $d \geq 2$.

Finally, the section 4, arithmetic, relates this problem it Tate's method, to study real zeros of zeta functions by relating it with the discriminant.
The arithmetic argument shows that the linear growth of $B_d$ as a function in the dimension is natural in view of known ramification properties of these fields.
