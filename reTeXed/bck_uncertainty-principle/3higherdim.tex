\section{Higher dimensions}

On Euclidean space $\mathbb{R}^{d}$ with inner product
\[
    x \cdot y = \sum_{i=1}^{d} x_i y_i, \quad \|x\| = (x \cdot x)^{1/2},
\]
Fourier transform is defined by
\begin{equation}
    \label{eqn:3.1}
    \what{f}(y) = \int f(x) e^{-2 i \pi x \cdot y} \dd x
\end{equation}
where $\dd x = \dd x_1 \cdots \dd x_d$ is the Lebesgue measure; then
\begin{equation}
    \label{eqn:3.2}
    f(x) = \int \what{f}(y) e^{2 i \pi x \cdot y} \dd y.
\end{equation}
We suppose that $f$ and $\what{f}$ are continuous and integrable.
More generally, if $E$ is a Euclidean space of dimension $d$, if the invariant measure $\dd x$ on $E$ is chosen so that the cube formed by the orthonormal basis has measure $1$, and if $x\cdot y$ is the corresponding inner product, Fourier transform and its inverse is defined by \eqref{eqn:3.1} and \eqref{eqn:3.2}.

Consider the Fourier pairs $(f, \what{f})$ satisfying
\begin{enumerate}
    \item $f$, $\what{f}$ are not identically zero,
    \item $f(0) \leq 0$ and $\what{f}(0) \leq 0$,
    \item $f(x) \geq 0$ for $\|x\| \geq a_f$, $\what{f}(0) \geq 0$ for $\|y\| \geq a_{\what{f}}$.
\end{enumerate}
Define $A(f)$ and $A(\what{f})$ as in \S 1:
\[
    A(f) = \inf\{r >0: f(x) \geq 0 \text{ if } \|x\| > r\},
\]
and
\[
    B_d = \inf A(f) A(\what{f})
\]
for pairs satisfying 1--3.
Let $f^\natural(x)$ be the (invariant) integral of $f$ on the sphere of radius $\|x\|$: $\what{f}^\natural = (\what{f})^{\natural}$ and $f^\natural$ and $\what{f}^{\natural}$ are nonzero; otherwise $f$ and $\what{f}$ are compactly supported from 3.
Since $A(f^\natural) \leq A(f)$ and $A(\what{f}^{\natural}) \leq A(\what{f})$, we can limit ourselves to the radial functions.
Since
\[
    (f(x/\lambda))^\wedge = \lambda^d \what{f}(\lambda y)\quad (\lambda > 0),
\]
we can follow the argument in \S 1 and we have
\begin{equation}
    \label{eqn:3.4}
    B_d = A^2, \quad A = \inf A(f)
\end{equation}
\textbf{where the infimum is over the functions $f \in L^1(\mathbb{R}^d)$, radial, not identically zero, such that $f = \what{f}$ and $f(0) = 0$.}

We have, as in \S 1, can add multiple of the following radial and self-dual function if necessary
\[
    \gamma(x) = e^{-\pi \|x\|^2}.
\]

\begin{theorem}
\label{thm:3.1}
We have
\[
B_d \geq \frac{1}{\pi}\left(\frac{1}{2}\Gamma\left(\frac{d}{2} + 1\right)\right)^{2/d} > \frac{d}{2 \pi e}.
\]
\end{theorem}
\begin{proof}
Follow the argument of the case $d = 1$, where we replace the interval $(-A(f), A(f))$ with the ball of radius $A(f)$ centered at the origin, whose volume ($\geq \frac{1}{2}$) is $\frac{1}{\Gamma(\frac{d}{2} + 1)}(A(f))^{d} \pi^{d/2}$.
\end{proof}

Put $X = \pi \|x\|^2$, the argument in \S 2 natually leads us to consider the functions
\[
    g_a(x) = G_a(X) \quad (x \in \mathbb{R}^{d})
\]
where
\[
    G_a(X) = a^d e^{-X a^2} + e^{-X a^2} - (1 + a^d) e^{-X},
\]
and set
\[
    H_a(X) = a^d e^{(1-a^2)X} + e^{(1 - a^{-2})X} - (1 + a^d), \quad a > 1.
\]
It is convenient to define $a^2 = 1 + k$, $d = 2c$, which gives
\[
    H_a(X) = (1 + k)^{c} e^{-kX} + e^{(1 - (1+k)^{-1})X} - 1 - (1 + k)^{c}.
\]
The derivative in $X$ at the origin is
\[
    \frac{k}{1 + k}\left(1 - (1+k)^{c+1}\right) < 0;
\]
the convexity argument in \S 2 shows that $H_a$ has a unique positive zero $X_a$.
As before, we compute the expansion of $H_a(X)$ in $k$ up to order $4$.
It is
\[
    H_a(X) = P_1 k + P_2 k^2 + P_3 k^3 + P_4 k^4 + O(k^5)
\]
where
\begin{align*}
    P_1 &= 0 \\
    P_2 &= X(X - c - 1) \\
    P_3 &= \frac{1}{2}(c - 2)X(X-c-1) \\
    P_4 &= \frac{1}{12} X(X^3 - (2c + 6)X^2 + (3c(c-1) + 18)X - (2c(c-1)(c-2) + 12)).
\end{align*}
As in dimension $1$ case, we see that $P_2$ and $P_3$ cancel out for
\begin{equation}
    \label{eqn:3.5}
    X = X(d) := \frac{d}{2} + 1.
\end{equation}
Moreover, $P_2 > 0$  for $X > X(d)$, $<0$ for $X < X(d)$.
Taking the limit $k \to 0$ gives
\[
    \lim_{a \to 1} X_a = \frac{d}{2} + 1.
\]
To understand the location of $X_a$ with respect to $X(d)$ as $a \to 1$, compute $Q_4(X(d))$ or $P_4 = \frac{X}{12}Q_4$.
Calculation gives
\[
    Q_4(c+1) = -c^2 + 1.
\]
For $d > 2$, the term is $<0$, so $H_a(X(d)) < 0$ for $a$ close to $1$, which shows that
\[
    X_a > \frac{d}{2} + 1 \quad(a > 1, \text{ close to }1).
\]
Therefore it is possible that the value in \eqref{eqn:3.5} is optimal.
This is not the case when $d = 1$ as we saw in \S 2.

For $d = 2$, $Q_4(c+1) = 0$, so we need to compute up to degree $5$, where
\begin{equation}
    \label{eqn:3.6}
    H_a(2) = (1+k)e^{-2k} + e^{2(1 - \frac{1}{1+k})} - 2 - k.
\end{equation}
The Taylor series at $0$ of
\begin{align*}
    f(z) &= e^{2(1 - \frac{1}{1+z})} = e^{2\frac{z}{1+z}},\\
    f(z) &= \sum_{n=0}^{\infty} q_n z^n,
\end{align*}
can be calculated using the residue theorem.
Let
\[
    w = \frac{z}{1+z}, \,\,z = \frac{w}{1-w}, \,\,\dd z = \frac{\dd w}{(1-w)^2},
\]
by taking a small contour around $0$:
\begin{align*}
    q_n &= \Res_{z=0} \frac{f(z)}{z^{n+1}} = \frac{1}{2 i \pi} \oint e^{\frac{2z}{1+z}} \frac{\dd z}{z^{n+1}} \\
    &= \frac{1}{2 i \pi} \oint e^{2w} \frac{(1-w)^{n+1}}{w^{n+1}} \frac{\dd w}{(1-w)^{2}} \\
    &= \Res_{w=0} \frac{(1-w)^{n-1}}{w^{n+1}} e^{2w}.
\end{align*}
In particular, $q_5$ is the sum of
\begin{equation}
    \label{eqn:3.7}
    \frac{2^4}{4!} - \frac{2^5}{5!}
\end{equation}
coming from the first term of \eqref{eqn:3.6}, and the coefficient of $w^5$ in $e^{2w}(1-w)^{4}$, equal to
\begin{equation}
    \label{eqn:3.8}
    \frac{2^5}{5!} - 4 \cdot \frac{2^4}{4!} +  6 \cdot \frac{2^3}{3!} - 4 \cdot \frac{2^2}{2!} + 2.
\end{equation}
We found that $q_5 = 0$.

Similarly, $q_6$ is the sum of
\begin{equation}
    \label{eqn:3.9}
    -\frac{2^5}{5!} + \frac{2^6}{6!}
\end{equation}
and
\begin{equation}
    \label{eqn:3.10}
    \frac{2^6}{6!} - 5 \cdot \frac{2^5}{5!} + 10 \cdot \frac{2^4}{4!} - 10 \cdot \frac{2^3}{3!} + 5 \cdot \frac{2^2}{2!} - 2,
\end{equation}
which is
\[
    q_6 = -\frac{4}{45} < 0.
\]
When $a$ is sufficiently close to $1$, we therefore have $H_a(2) < 0$ and $X_a > X(2) = 2$.
Again, the bound given by \eqref{eqn:3.5} could be optimal.

Concluding this section, note that for all $d \geq 2$ we obtain the upper bound
\begin{equation}
    \label{eqn:3.11}
    B_d \leq \mathcal{B}_{d} \leq \frac{d+2}{2\pi}
\end{equation}
where $\mathcal{B}_{d}$ is defined, as in \S 1, by the functions in the space $\mathcal{S}(\mathbb{R}^{d})$.
Also following the argument in the end of \S 1, relating the bounds for $L^1$ and $\mathcal{S}$ applies.
To prove the inequality \eqref{eqn:1.6}, we have to consider $T = \delta_b + \delta_{-b} + 2 \delta_{0}$, where $\|b\| < a = A(f)$ and $f(b) < 0$; $\what{T} = 2 \cos(2 \pi b \cdot y) + 2$ is a positive plane wave function.
The rest of the argument is the same, replacing $h + \what{h}$ with the spherical average of $h + \what{h}$ if we want to limit ourselves to the radial functions.
In conclusion,
\begin{theorem}
\label{thm:3.2}
We have
\begin{equation}
    \label{eqn:3.12}
    B_d \leq \mathcal{B}_{d} \leq \frac{d+2}{2 \pi}, \quad B_d \geq \frac{1}{2} \mathcal{B}_d.
\end{equation}
\end{theorem}

