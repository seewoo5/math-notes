\section{The trace formula}
\label{sec:traceformula}

We shall denote by $\scS(S(\bA))$ the space of Schwartz-Bruhat functions with compact support on $S(\bA)$.
It follows from \cite{duflo1971formule} that for any $f \in \scS(S(\bA))$, the operator $\rho(f)$ is of trace class and its trace is given by Selberg's Trace Formula.
Given $f \in \scS(S(\bA))$ and $g \in G(\bA)$ we define a function $f^g$ on $S(\bA)$ by
\[
    f^g(s) = f(g s g^{-1});
\]
we have $\Tr \rho^g(f) = \Tr \rho(f^{g^{-1}})$; moreover the function
\[
    g \mapsto \Tr \rho(f^{g})
\]
is continuous and constant on the cosets modulo $S(\bA)Z(\bA)G(F)$, so that one can define
\[
    I_\varepsilon(f) = \int_{G(\bA) / S(\bA) Z(\bA) G(F)} \varepsilon(\det g) \Tr \rho(f^g) \dd g.
\]
This integral is merely a finite sum, and is zero except for finitely many values of $\varepsilon \in \mathscr{E}$.
On the other hand we have
\[
    \Tr \rho(f) = \sum_{U \in \widehat{S(\bA)}} m(U) \Tr U(f),
\]
the series being absolutely convergent. It is then clear that
\[
    I_\varepsilon(f) = \sum_{U \in \widehat{S(\bA)}} m_\varepsilon(U) \Tr U(f)
\]
and that
\[
    \Tr \rho(f) = \sum_{\varepsilon \in \mathscr{E}} I_\varepsilon(f).
\]
The values of the factors $m_\varepsilon(U)$ will be deduced from the explicit knowledge of the integrals $I_\varepsilon(f)$ for sufficiently many functions $f$, in application of the principle that a representation is determined by its trace.
Before we explain the computation of the integrals $I_\varepsilon(f)$, let us state the result.
