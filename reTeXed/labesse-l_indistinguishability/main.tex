% --- LaTeX Lecture Notes Template - S. Venkatraman ---

% --- Set document class and font size ---

\documentclass[letterpaper, 12pt]{article}

% --- Package imports ---

% Extended set of colors
\usepackage[dvipsnames]{xcolor}

\usepackage{
  amsmath, amsthm, amssymb, mathtools, dsfont, units,          % Math typesetting
  graphicx, wrapfig, subfig, float,                            % Figures and graphics formatting
  listings, color, inconsolata, pythonhighlight,               % Code formatting
  fancyhdr, sectsty, hyperref, enumerate, framed }   % Headers/footers, section fonts, links, lists

% lipsum is just for generating placeholder text and can be removed
\usepackage{hyperref, lipsum} 

% --- Fonts ---

\usepackage{newpxtext, newpxmath, inconsolata}
\usepackage{amsfonts}

\usepackage{tikz}
\usepackage{tikz-cd}
% \usepackage{enumitem}
\usepackage[title]{appendix}
\usepackage{mathdots}
\usepackage{stmaryrd}

% --- Page layout settings ---

% Set page margins
\usepackage[left=1.35in, right=1.35in, top=1.0in, bottom=.9in, headsep=.2in, footskip=0.35in]{geometry}

% Anchor footnotes to the bottom of the page
\usepackage[bottom]{footmisc}

% Set line spacing
\renewcommand{\baselinestretch}{1.2}

% Set spacing between paragraphs
\setlength{\parskip}{1.3mm}

% Allow multi-line equations to break onto the next page
\allowdisplaybreaks

% --- Page formatting settings ---

% Set image captions to be italicized
\usepackage[font={it,footnotesize}]{caption}

% Set link colors for labeled items (blue), citations (red), URLs (orange)
\hypersetup{colorlinks=true, linkcolor=RoyalBlue, citecolor=RedOrange, urlcolor=ForestGreen}

% Set font size for section titles (\large) and subtitles (\normalsize) 
\usepackage{titlesec}
% \titleformat{\section}{\large\bfseries}{{\fontsize{19}{19}\selectfont\textreferencemark}\;\; }{0em}{}
\titleformat{\section}{\large\bfseries}{\thesection\;\;\;}{0em}{}
\titleformat{\subsection}{\normalsize\bfseries\selectfont}{\thesubsection\;\;\;}{0em}{}

% Enumerated/bulleted lists: make numbers/bullets flush left
%\setlist[enumerate]{wide=2pt, leftmargin=16pt, labelwidth=0pt}
% \setlist[itemize]{wide=0pt, leftmargin=16pt, labelwidth=10pt, align=left}

% --- Table of contents settings ---

\usepackage[subfigure]{tocloft}

% Reduce spacing between sections in table of contents
\setlength{\cftbeforesecskip}{.9ex}

% Remove indentation for sections
\cftsetindents{section}{0em}{0em}

% Set font size (\large) for table of contents title
\renewcommand{\cfttoctitlefont}{\large\bfseries}

% Remove numbers/bullets from section titles in table of contents
\makeatletter
\renewcommand{\cftsecpresnum}{\begin{lrbox}{\@tempboxa}}
\renewcommand{\cftsecaftersnum}{\end{lrbox}}
\makeatother

% --- Set path for images ---

\graphicspath{{Images/}{../Images/}}

% --- Math/Statistics commands ---

% Add a reference number to a single line of a multi-line equation
% Usage: "\numberthis\label{labelNameHere}" in an align or gather environment
\newcommand\numberthis{\addtocounter{equation}{1}\tag{\theequation}}

% Shortcut for bold text in math mode, e.g. $\b{X}$
\let\b\mathbf

% Shortcut for bold Greek letters, e.g. $\bg{\beta}$
\let\bg\boldsymbol

% Shortcut for calligraphic script, e.g. %\mc{M}$
\let\mc\mathcal

% \mathscr{(letter here)} is sometimes used to denote vector spaces
\usepackage[mathscr]{euscript}

% Convergence: right arrow with optional text on top
% E.g. $\converge[p]$ for converges in probability
\newcommand{\converge}[1][]{\xrightarrow{#1}}

% Weak convergence: harpoon symbol with optional text on top
% E.g. $\wconverge[n\to\infty]$
\newcommand{\wconverge}[1][]{\stackrel{#1}{\rightharpoonup}}

% Equality: equals sign with optional text on top
% E.g. $X \equals[d] Y$ for equality in distribution
\newcommand{\equals}[1][]{\stackrel{\smash{#1}}{=}}

% Normal distribution: arguments are the mean and variance
% E.g. $\normal{\mu}{\sigma}$
\newcommand{\normal}[2]{\mathcal{N}\left(#1,#2\right)}

% Uniform distribution: arguments are the left and right endpoints
% E.g. $\unif{0}{1}$
\newcommand{\unif}[2]{\text{Uniform}(#1,#2)}

% Independent and identically distributed random variables
% E.g. $ X_1,...,X_n \iid \normal{0}{1}$
\newcommand{\iid}{\stackrel{\smash{\text{iid}}}{\sim}}

% Sequences (this shortcut is mostly to reduce finger strain for small hands)
% E.g. to write $\{A_n\}_{n\geq 1}$, do $\bk{A_n}{n\geq 1}$
\newcommand{\bk}[2]{\{#1\}_{#2}}

% \setcounter{section}{-1}

\newcommand{\SL}{\mathrm{SL}}
\newcommand{\Sp}{\mathrm{Sp}}
\newcommand{\Mp}{\mathrm{Mp}}
\newcommand{\GL}{\mathrm{GL}}
\newcommand{\SO}{\mathrm{SO}}
\newcommand{\SU}{\mathrm{SU}}
\newcommand{\PGL}{\mathrm{PGL}}
\newcommand{\PSL}{\mathrm{PSL}}
\newcommand{\rM}{\mathrm{M}}
\newcommand{\rN}{\mathrm{N}}
\newcommand{\rO}{\mathrm{O}}
\newcommand{\rP}{\mathrm{P}}
\newcommand{\rH}{\mathrm{H}}
\newcommand{\rU}{\mathrm{U}}
\newcommand{\JL}{\mathrm{JL}}
\newcommand{\stab}{\mathrm{Stab}}
\newcommand{\cusp}{\mathrm{cusp}}
\newcommand{\reg}{\mathrm{reg}}
\newcommand{\rs}{\mathrm{rs}}
\newcommand{\Irr}{\mathrm{Irr}}
\newcommand{\Tr}{\mathrm{Tr}}
\newcommand{\Hom}{\mathrm{Hom}}
\newcommand{\Gal}{\mathrm{Gal}}
\newcommand{\WD}{\mathrm{WD}}
\newcommand{\Frob}{\mathrm{Frob}}
\newcommand{\Res}{\mathrm{Res}}
\newcommand{\Tam}{\mathrm{Tam}}
\newcommand{\Pet}{\mathrm{Pet}}
\newcommand{\sgn}{\mathrm{sgn}}
\newcommand{\vol}{\mathrm{vol}}
\newcommand{\Aut}{\mathrm{Aut}}
\newcommand{\Ind}{\mathrm{Ind}}
\newcommand{\BC}{\mathrm{BC}}
\newcommand{\Ad}{\mathrm{Ad}}

\newcommand{\what}{\widehat}

\newcommand{\dd}{\mathrm{d}}

\newcommand{\bA}{\mathbb{A}}
\newcommand{\bR}{\mathbb{R}}
\newcommand{\bS}{\mathbb{S}}
\newcommand{\bZ}{\mathbb{Z}}
\newcommand{\bN}{\mathbb{N}}
\newcommand{\bC}{\mathbb{C}}
\newcommand{\bQ}{\mathbb{Q}}
\newcommand{\bH}{\mathbb{H}}
\newcommand{\bI}{\mathbb{I}}
\newcommand{\bfi}{\mathbf{I}}
\newcommand{\bfa}{\mathbf{a}}
\newcommand{\bfb}{\mathbf{b}}
\newcommand{\cS}{\mathcal{S}}
\newcommand{\cO}{\mathcal{O}}
\newcommand{\cV}{\mathcal{V}}
\newcommand{\cP}{\mathcal{P}}

\newcommand{\scA}{\mathscr{A}}
\newcommand{\scB}{\mathscr{B}}
\newcommand{\scV}{\mathscr{V}}
\newcommand{\scT}{\mathscr{T}}
\newcommand{\scU}{\mathscr{U}}
\newcommand{\scW}{\mathscr{W}}
\newcommand{\scO}{\mathscr{O}}
\newcommand{\scL}{\mathscr{L}}
\newcommand{\scS}{\mathscr{S}}

\newcommand{\frh}{\mathfrak{h}}
\newcommand{\frt}{\mathfrak{t}}
\newcommand{\frg}{\mathfrak{g}}
\newcommand{\frgl}{\mathfrak{gl}}
\newcommand{\fru}{\mathfrak{u}}

% Math mode symbols for common sets and spaces. Example usage: $\R$
\newcommand{\R}{\mathbb{R}}	% Real numbers
\newcommand{\C}{\mathbb{C}}	% Complex numbers
\newcommand{\Q}{\mathbb{Q}}	% Rational numbers
\newcommand{\Z}{\mathbb{Z}}	% Integers
\newcommand{\N}{\mathbb{N}}	% Natural numbers
\newcommand{\F}{\mathcal{F}}	% Calligraphic F for a sigma algebra
\newcommand{\El}{\mathcal{L}}	% Calligraphic L, e.g. for L^p spaces

% Math mode symbols for probability
\newcommand{\pr}{\mathbb{P}}	% Probability measure
\newcommand{\E}{\mathbb{E}}	% Expectation, e.g. $\E(X)$
\newcommand{\var}{\text{Var}}	% Variance, e.g. $\var(X)$
\newcommand{\cov}{\text{Cov}}	% Covariance, e.g. $\cov(X,Y)$
\newcommand{\corr}{\text{Corr}}	% Correlation, e.g. $\corr(X,Y)$
\newcommand{\B}{\mathcal{B}}	% Borel sigma-algebra

% Other miscellaneous symbols
\newcommand{\tth}{\text{th}}	% Non-italicized 'th', e.g. $n^\tth$
\newcommand{\Oh}{\mathcal{O}}	% Big-O notation, e.g. $\O(n)$
\newcommand{\1}{\mathds{1}}	% Indicator function, e.g. $\1_A$

% Additional commands for math mode
\DeclareMathOperator*{\argmax}{argmax}		% Argmax, e.g. $\argmax_{x\in[0,1]} f(x)$
\DeclareMathOperator*{\argmin}{argmin}		% Argmin, e.g. $\argmin_{x\in[0,1]} f(x)$
\DeclareMathOperator*{\spann}{Span}		% Span, e.g. $\spann\{X_1,...,X_n\}$
\DeclareMathOperator*{\bias}{Bias}		% Bias, e.g. $\bias(\hat\theta)$
\DeclareMathOperator*{\ran}{ran}			% Range of an operator, e.g. $\ran(T) 
\DeclareMathOperator*{\dv}{d\!}			% Non-italicized 'with respect to', e.g. $\int f(x) \dv x$
\DeclareMathOperator*{\diag}{diag}		% Diagonal of a matrix, e.g. $\diag(M)$
\DeclareMathOperator*{\trace}{Tr}		% Trace of a matrix, e.g. $\trace(M)$
\DeclareMathOperator*{\supp}{supp}		% Support of a function, e.g., $\supp(f)$

% Numbered theorem, lemma, etc. settings - e.g., a definition, lemma, and theorem appearing in that 
% order in Lecture 2 will be numbered Definition 2.1, Lemma 2.2, Theorem 2.3. 
% Example usage: \begin{theorem}[Name of theorem] Theorem statement \end{theorem}
\theoremstyle{definition}
\newtheorem{theorem}{Theorem}[section]
\newtheorem{conjecture}{Conjecture}[section]
\newtheorem{proposition}[theorem]{Proposition}
\newtheorem{lemma}[theorem]{Lemma}
\newtheorem{corollary}[theorem]{Corollary}
\newtheorem{definition}[theorem]{Definition}
\newtheorem{example}[theorem]{Example}
\newtheorem{remark}[theorem]{Remark}

% Un-numbered theorem, lemma, etc. settings
% Example usage: \begin{lemma*}[Name of lemma] Lemma statement \end{lemma*}
\newtheorem*{theorem*}{Theorem}
\newtheorem*{proposition*}{Proposition}
\newtheorem*{lemma*}{Lemma}
\newtheorem*{corollary*}{Corollary}
\newtheorem*{definition*}{Definition}
\newtheorem*{example*}{Example}
\newtheorem*{remark*}{Remark}
\newtheorem*{claim}{Claim}
\newtheorem*{question*}{Question}

% --- Left/right header text (to appear on every page) ---

% Do not include a line under header or above footer
\pagestyle{fancy}
\renewcommand{\footrulewidth}{0pt}
\renewcommand{\headrulewidth}{0pt}

% Right header text: Lecture number and title
\renewcommand{\sectionmark}[1]{\markright{#1} }
% \fancyhead[R]{\small\textit{\nouppercase{\rightmark}}}

% Left header text: Short course title, hyperlinked to table of contents
% \fancyhead[L]{\hyperref[sec:contents]{\small Gan-Gross-Prasad conjecture}}

% --- Document starts here ---

\begin{document}

% --- Main title and subtitle ---

\title{L-indistinguishable representations and trace formula for $\SL(2)$ \\[1em]
\normalsize Re-\TeX ed by Seewoo Lee\footnote{seewoo5@berkeley.edu. Some notations in the paper are ``modernized'' or changed a bit.}}

% --- Author and date of last update ---

\author{Jean-Pierre Labesse}
\date{\normalsize\vspace{-1ex} Last updated: \today}

% --- Add title and table of contents ---

\maketitle


% --- Abstracts ---

% \tableofcontents\label{sec:contents}

% --- Main content: import lectures as subfiles ---


\newpage
\section{Introduction}

Modular forms and Maass wave forms are certain functions defined on the complex upper half plane that satisfies $\SL(2, \Zz)$-transformations laws (or more generally, transform under congruence subgroups $\Gamma_{0}(N)$). 
There are a lot of applications of modular forms in number theory, such as sum of squares and the irrationality of $\zeta(3)$, and the Wiles' famous proof of Fermat's Last Theorem. 
There are also applications in other subjects, such as combinatorics (partition numbers), physics, representation theory (monstrous moonshine), knot theory, etc. 
 
In this note, we will study how to interpret such functions (so-called classical automorphic forms) as a representation of ad\'ele groups $\GL(2, \Aa)$ (here $\Aa$ is a ring of ad\'eles of global fields such as $\Qq$), and study representation theory of it. 
This can be a starting point of the \emph{Langlands' Program}, which connects representation of Galois groups, algebraic geometry, and automorphic forms (representations). 

To study such representations, we first study local representations. 
There are two kinds of local representations - archimedean and non-archimedean. 
For the archimedean cases, we study representation theory of $\GL(2, \Rr)$  via so-called ($\frag, K$)-modules. $(\frag, K)$-module is a vector space with compatible $\frag_{\Cc} = \mathfrak{gl}(2, \Cc)$ and $K = \rO(2)$-actions. It is easier to study $(\frag, K)$-modules than studying the representation of $\GL(2, \Rr)$ directly since $(\frag, K)$-modules are more \emph{algebraic}. We will classify $(\frag, K)$-modules for $\GL(2, \Rr)$ and also study which of them are unitarizable, since we are interested in the representation that lives in $L^{2}$ space. Also, we will see how these representations are related to classical automorphic forms (such as modular forms and Maass wave forms). 

We also have non-archimedean representations - which are representation of $p$-adic groups $\GL(2, \Qq_{p})$ for a prime $p$. They are very different from archimedean cases because of their topology. This makes the situation easier or harder, but anyway, we will also classify all the representations of such groups and study their unitarizability. 

When we finish the local theories, we can \emph{glue} these representations to obtain the representation of the ad\'ele group $\GL(2, \Aa)$. (In fact, this is not a true representation of $\GL(2, \Aa)$, but a representation of $(\frag_{\infty}, K_{\infty}) \times \GL(2, \Aa_{\fin})$.) 
While we are studying such representations (local or global), we will only concentrate on some \emph{nice} representations (\emph{admissible} representations) that are close to the representation of finite groups. 
\emph{Automorphic} representations are some nice representations that also satisfies some analytic conditions on growth. 
Later, we will see that Flath's decomposition theorem tells us that it is enough to study such glued representations to study automorphic representations. 

Before we get into the representation theory of $\GL(2, \Aa)$, we will study $\GL(1, \Aa)$ first, which are  completed by Tate in his celebrated thesis. He find a natural way to prove the analytic continuation and the functional equation of Hecke's $L$-function using local-global principle, and such idea will be used to define $L$-functions attached to automorphic representations of $\GL(2,\Aa)$. 

It may be hard to study an abstract representation of a given group (such as $\GL(2, \Rr), \GL(2, \Qq_{p})$ or $\GL(2, \Aa)$). 
Whittaker model (or Whittaker functional) help us to study such representations as a very concrete representation that functions on the group lives (and the group acts as a right translation). 
Most case, such Whittaker model exist and unique, and such results are called (local or global) multiplicity one theorem. 
In the last section, we will see how the multiplicity one theorem is related to the classical modular forms. 

%%%%%%%%%%%%%%%%%%%%%%%%%%%%
\section{Arithmetic equivalence}
\label{sec:arithmetic}

The group $S(\bA)$ is an invariant subgroup in $G(\bA)$, so that if $g$ is an element of $G(\bA)$ and $U$ any unitary representation of $S(\bA)$, we can define a unitary representation $U^g$ of $S(\bA)$ by
\[
    U^g(s) = U(g s g^{-1}), \quad s \in S(\bA).
\]
The two representations $U$ and $U^g$ will be called \emph{arithmetically equivalent}. 
If $U$ is irreducible then $U$ and $U^g$ will also sometimes be called \emph{L-indistinguishable};
in fact from some arithmetical points of view they seem indistinguishable despite the fact they are not in general equivalent in the ordinary sense.
One must remark that if $U$ occurs in the restriction $V$ to $S(\bA)$ of some representation $\widetilde{V}$ of $G(\bA)$, then $U^g$ will occur in $V$ with the same multiplicity for any $g \in G(\bA)$.

There is an obvious local analogue of this global arithmetical equivalence.
For instance a member of the discrete series of representations of $\SL(2, \bR)$ and its complex conjugate are arithmetically equivalent in the local sense.
It is well known that the representation $\rho$ of $S(\bA)$ in $L_0^2$, splits into a direct sum of irreducible representations with finite multiplicities.
Let us denote by $\widehat{S(\bA)}$, the set of equivalence classes of irreducible unitary representations of $S(\bA)$, and by $m(U)$ the multiplicity of $U \in \widehat{S(\bA)}$ in $\rho$, so that
\[
    \rho = \sum_{U \in \widehat{S(\bA)}} m(U) U.
\]
In the same way, given any $g \in G(\bA)$, we have
\[
    \rho^g = \sum_{U \in \widehat{S(\bA)}} m(U) U^g = \sum_{U \in \widehat{S(\bA)}} m(U^{g^{-1}}) U.
\]

We see that the comparison of $\rho$ and $\rho^g$ amounts to comparing $m(U)$ and $m(U^g)$.
Our problem can be restated ni the folowing way: do $U$ and $U^g$ occur with the same multiplicity (may be zero) in $\rho$?
We are thus led to study the continuous functions
\[
    g \mapsto m(U^g)
\]
from $G(\bA)$ to the set $\bN$ of natural numbers.
If this function is constant (resp. non-constant) the representation $U$ will be called \emph{stable} (resp. \emph{unstable}).

Denoting by $Z$ with the center of $G$, we have

\begin{lemma}
If $g$ belongs to $Z(\bA)S(\bA)G(F)$, then $\rho^g$ is equivalent to $\rho$.
\end{lemma}

\begin{proof}
If $g$ lies in $S(\bA)$ or in $Z(\bA)$ the statement is obvious.
If $g$ lies in $G(F)$ the conjugation by $g$ defines an automorphism of the space $S(\bA) / S(F)$, and so determines an intertwining operator between $\rho$ and $\rho^g$, hence the lemma.
\end{proof}

As a corollary we see that the function $g \mapsto m(U^g)$ is constant on the invariant subgroup $S(\bA) Z(\bA) G(F)$ of $G(\bA)$.
If we denote by $\bI$ the group of ideles of $F$, by $\bI^2$ the group of square ideles and by $F^\times$ the multiplicative group of F, then $G(\bA) / Z(\bA)S(\bA) G(F)$ is isomorphic via the determinant mapping to the compact abelian group $\bI / \bI^2 F^\times$.
Now the function $g \mapsto m(U^g)$ induces a discrete valued continuous function on a compact group and hence assumes only finitely many values.

Now let $\varepsilon$ be an element of $\mathscr{E}$ the group of characters of $\bI / \bI^2 F^\times$. We can define
\[
    m_\varepsilon(U) = \int_{G(\bA) / S(\bA) Z(\bA) G(F)} \varepsilon(\det g) m(U^g) \dd g
\]
where $\dd g$ is the Haar measure with total mass one.

We have the following
\begin{lemma}
\begin{enumerate}
    \item A representation $U \in \widehat{S(\bA)}$ is stable if and only if $m_\varepsilon(U) = 0$ for any nontrivial $\varepsilon \in \mathscr{E}$.
    \item A representation $U \in \widehat{S(\bA)}$ is unstable if and only if $m_\varepsilon(U) \neq 0$ for some nontrivial $\varepsilon \in \mathscr{E}$.
    \item $m(U) = \sum_{\varepsilon \in \mathscr{E}} m_\varepsilon(U)$, the series benig only a sum over a finite subgroup of $\mathscr{E}$ depending on $U$.
\end{enumerate}
\end{lemma}

The study of the numbers $m_{\varepsilon}(U)$ will now be made using Selberg's trace formula.

\section{The trace formula}
\label{sec:traceformula}

We shall denote by $\scS(S(\bA))$ the space of Schwartz-Bruhat functions with compact support on $S(\bA)$.
It follows from \cite{duflo1971formule} that for any $f \in \scS(S(\bA))$, the operator $\rho(f)$ is of trace class and its trace is given by Selberg's Trace Formula.
Given $f \in \scS(S(\bA))$ and $g \in G(\bA)$ we define a function $f^g$ on $S(\bA)$ by
\[
    f^g(s) = f(g s g^{-1});
\]
we have $\Tr \rho^g(f) = \Tr \rho(f^{g^{-1}})$; moreover the function
\[
    g \mapsto \Tr \rho(f^{g})
\]
is continuous and constant on the cosets modulo $S(\bA)Z(\bA)G(F)$, so that one can define
\[
    I_\varepsilon(f) = \int_{G(\bA) / S(\bA) Z(\bA) G(F)} \varepsilon(\det g) \Tr \rho(f^g) \dd g.
\]
This integral is merely a finite sum, and is zero except for finitely many values of $\varepsilon \in \mathscr{E}$.
On the other hand we have
\[
    \Tr \rho(f) = \sum_{U \in \widehat{S(\bA)}} m(U) \Tr U(f),
\]
the series being absolutely convergent. It is then clear that
\[
    I_\varepsilon(f) = \sum_{U \in \widehat{S(\bA)}} m_\varepsilon(U) \Tr U(f)
\]
and that
\[
    \Tr \rho(f) = \sum_{\varepsilon \in \mathscr{E}} I_\varepsilon(f).
\]
The values of the factors $m_\varepsilon(U)$ will be deduced from the explicit knowledge of the integrals $I_\varepsilon(f)$ for sufficiently many functions $f$, in application of the principle that a representation is determined by its trace.
Before we explain the computation of the integrals $I_\varepsilon(f)$, let us state the result.

\section{The formula for $I_\varepsilon(f)$}
\label{sec:formulaI}

Throughout this paragraph we fix a nontrivial character $\varepsilon \in \mathscr{E}$.
Global class field theory associates to $\varepsilon$ a separable quadratic extension $L$ of $F$.
We shall denote by $E$ the kernel of the norm mapping $\rN_{L/F}$.
The algebraic group $E$ is an $F$-anisotropic torus of dimension one; we shall denote by $\Theta(\varepsilon)$ the dual group of the compact group $E(\bA) / E(F)$.
If $\psi$ is a nontrivial character of $\bA/F$, then using the Weil representation one can define for any character $\theta$ of $E(\bA)$, a representation $U(\psi, \theta)$ of $S(\bA)$ (cf. Shalika and Tanaka \cite{shalika1969explicit}).
The equivalence class of $U(\psi, \theta)^{g}$ depends only on $d = \det(g)$ and we shall denote it by $U(\psi, \theta, d)$; moreover two such equivalence classes for $d$ and $d'$ in $\bI$ coincide if and only if $d'd^{-1} \in \rN_{L/F}(\bI_L)$.
We recall that $\rN_{L/F}(\bI_L)$ is an open subgroup of $\bI$.
Let us now define
\[
U^+_\theta = \sum_{\substack{d \in \bI / \rN_{L/F}(\bI_L) \\ \varepsilon(d) = 1}} U(\psi, \theta, d)
\]
and
\[
U^-_\theta = \sum_{\substack{d \in \bI / \rN_{L/F}(\bI_L) \\ \varepsilon(d) = -1}} U(\psi, \theta, d).
\]
The representations $U^+_\theta$ and $U^-_\theta$, are independent of the choice of $\psi$; moreover their intertwining number is zero. We can now state the

\begin{theorem}
Let $\varepsilon$ be a nontrivial character of $\bI / \bI^2 F^\times$, and let $f \in \scS(S(\bA))$, then:
\begin{enumerate}[(i)]
    \item The operators $U^+_\theta(f)$ and $U^-_\theta(f)$ are of trace class.
    \item The function $\theta \mapsto [\Tr U^+_\theta(f) - \Tr U^-_\theta(f)]$ is a Schwartz--Bruhat function on the dual $\widehat{E(\bA)}$ of $E(\bA)$.
    \item $$I_\varepsilon(f) = \frac{1}{4} \sum_{\substack{\theta \in \Theta(\varepsilon) \\ \theta \ne 1}} [\Tr U^+_\theta(f) - \Tr U^-_\theta(f)].$$
\end{enumerate}
\end{theorem}

\begin{proof}
The proof of assertion (iii) will be outlined in Section \ref{sec:computation}; let us only observe now that assertion (ii) implies the absolute convergence of the series.
To prove (i) and (ii) we may asume that $f$ is adecomposable function $f = \otimes_v f_v$ where for almost all $v$ (the places of $F$) $f_v = \chi_v$ the characteristic function of the standard maximal compact subgroup $K_v$ of $S_v = \SL(2, F_v)$.
Now if $v$ is a finite place of $F$ and if $U_v$ is an irreducible unitary representation of $S$, then $U_v(\chi_v) = 0$ unless $U_v$ contains the trivial representation of $K_v$.
The representation $U(\psi, \theta, d)$ is a tensor product of representations $U(\psi_v, \theta_v, d_v)$ and if $v$ is finite then $U(\psi_v, \theta_v, d_v)(\chi_v) = 0$ if $\theta_v$ is ramified (i.e. if $\theta_v$ is nontrivial on the maximal compact subgroup of $E_v$).
If $v$ is finite and if $\theta_v$ is unramified then $\Tr U(\psi_v, \theta_v, d_v)(\chi_v)$ is independent of $\theta_v$ and can only assume the values +1 or 0 if the local Haar measure is such that $\vol(K_v) = 1$, depending on the choice of $\psi_v$, $d_v$, and the structure of $L_v$.
In any case $\theta_v \mapsto \Tr U(\psi_v, \theta_v, d_v)$ is a Schwartz--Bruhat function on $\widehat{E_v}$.
We deduce from these facts that $\theta \mapsto \Tr U(\psi, \theta, d)$ is a Schwartz--Bruhat function on $\widehat{E(\bA)}$.
But one can show that outside a finite subgroup of $\bI / \rN_{L/F}(\bI_L)$, independent of $\theta$ then $U(\psi, \theta, d)(f) = 0$, and hence the assertions (i) and (ii) are proved.
\end{proof}

\begin{corollary}
Let $U \in \widehat{S(\bA)}$ then $m_\varepsilon(U) = 0$ or $|m_\varepsilon(U)| \ge 1/4$.
\end{corollary}

Using a slightly generalized form of the Lemma 16.1.1, p. 195 in \cite{jacquet1970automorphic} one see that if $m_\varepsilon(U) \ne 0$ then $U$ is equivalent to an irreducible part of some $U(\psi, \theta, d)$, and the corollary is proved.
More precise information can be obtained from the knowledge of the decomposition of the representations $U(\psi, \theta, d)$ and of the intertwining operators between $U(\psi, \theta, d)$ and $U(\psi, \theta', d')$, so that one obtains a complete classification of the unstable representations and the exact value of $m_\varepsilon(U)$.
Let us simply say here that the local representations $U(\psi_v, \theta_v, d_v)$ are irreducible if $\theta_v = 1$ or if $\theta_v^2 \ne 1$ and that the intertwining number between $U(\psi_v, \theta_v, d_v)$ and $U(\psi_v, \theta_v', d_v')$ is zero unless $d_v = d_v'$ and $\theta_v = \theta_v'$ or $\theta_v^{-1} = \theta_v'$ in which case they are equivalent.
If $\theta_v \ne 1$ and $\theta_v^2 = 1$ then $U(\psi_v, \theta_v, d_v)$ splits into two inequivalent irreducible parts.
One must remark that if $U \in \widehat{S(\bA)}$, and if $m_\varepsilon(U) \ne 0$ for some $\varepsilon \in \mathscr{E}$ then there exists at least one $g \in G(\bA)$ such that $m(U^g) \ne 0$ and we thus get a new proof (but in a less explicit way) of the result of Shalika and Tanaka \cite{shalika1969explicit} according ot which certain representations of $S(\bA)$ attached ot global characters of a nonsplit $F$-torus occur in $\rho$. (See also \cite{jacquet1970automorphic} p. 396.)


\section{Computation of $I_\varepsilon(f)$}
\label{sec:computation}

We shall denote by $r$ the natural representation of $S(\bA)$ on the space $L^2$ of square integrable functions on $S(\bA) / S(F)$; the representation $\rho$ is the restriction of $r$ to the invariant subspace $L^2_0$.
The orthogonal complement of $L^2_0$ in $L^2$ is the direct sum of the one dimensional space of constant functions, and of a space $L^2_1$ on which $r$ induces a representation equivalent to a continuous integral of representations of the global principal series (i.e. representations associated to unitary Hecke characters of a $F$-split torus).
This spectral decomposition is constructed by the use of Eisenstein series (cf. \cite{godement1964analyse}, see also \cite{duflo1971formule}, \cite{gelfand1966} and \cite{jacquet1970automorphic}) in an explicit way, and one finds that if $\delta$ denotes the unit representation of $S(\bA)$, then
\[
    \Tr \rho(f) + \Tr \delta(f) = \int_{[S]} [K_f(s, s) - H_f(s, s)] \, \dd s
\]
where
\[
K_f(x, y) = \sum_{\gamma \in S(F)} f(x \gamma y^{-1})
\]
is the kernel of $r(f)$ in $L^2$ and where $H_f(x, y)$ is the kernel of the operator $r(f)$ restricted to $L^2_1$.
Let us denote by $S_e$ the set of elliptic elements in $S(F)$ (i.e. the set of elements whose eigenvalues do not lie in $F$) and by $S_p$ its complementary set, the set of parabolic elements.
The above integral can be split into two convergent parts.
In fact it is easy to show that the integral
\[
J(f) = \int_{[S]}  \sum_{\gamma \in S_e} f(s \gamma s^{-1}) \, \dd s
\]
is absolutely convergent.
The other part is
\[
C(f) = \int_{[S]}  \sum_{\gamma \in S_p} f(s \gamma s^{-1}) - H_f(s, s)\, \dd s
\]
and is called the complementary term (c.f. \cite{duflo1971formule}).
We now introduce the integrals
\[
J_\varepsilon(f) = \int_{G(\bA) / Z(\bA)E(F)} \varepsilon(\det g) \sum_{\gamma \in S_e} f(g s\gamma g^{-1}) \, \dd g.
\]
The invariant measures are chosen in such a way that
\[
J(f) = \sum_{\varepsilon \in \mathscr{E}} J_\varepsilon(f).
\]
Now fix some non-trivial $\varepsilon \in \mathscr{E}$, and taking over the notations of \S \ref{sec:formulaI} choose some embedding of $L^\times$ in $G(F)$; it is not difficult to prove that
\[
J_\varepsilon(f) = \frac{1}{4} \vol([E]) \sum_{g \in G(\bA) / S(\bA) \bI_L} \varepsilon(\det g) \int_{S(\bA) / E(\bA)} \sum_{\substack{\gamma \in E(F) \\ \gamma \ne \pm 1}} f(g s \gamma s^{-1} g^{-1}) \, \dd s.
\]
We now introduce local factors:
\[
F_{f_v}^{\varepsilon}(e_v) = \sum_{g \in G_v / S_v L_v^\times} \varepsilon_v(\det g) \int_{S_v / E_v} a_v(e_v) f_v(gse_v s^{-1}g^{-1})\, \dd s
\]
where
\[
a_v(e_v) = b_v(e_v) |\rN_{L/F}(e_v - e_v^{-1})|_v^{1/2}
\]
and
\[
b_v\begin{pmatrix}
    x & y \\ z & t
\end{pmatrix} = \varepsilon_v(y).
\]
The function $e_v \mapsto F_{f_v}(e_v)$ is a priori defined only when $e_v \ne \pm 1$ but it can be continued in a Schwartz--Bruhat function on $E_v$.
This can be checked directly or by computing its Fourier transform, namely
\[
\widehat{F_{f_v}^{\varepsilon}}(\theta_v) = \lambda(L_v / F_v, \psi_v) \sum_{d \in F_v / \rN_{L/F}(L_v^\times)} \varepsilon_v(d) \Tr U(\psi_v, \theta_v, d)(f_v)
\]
(c.f. \cite{gelfand1966}) where $\lambda(L_v / F_v, \psi_v)$ is a factor of absolute value one defined in \cite{jacquet1970automorphic} p. 6. Using the assertions (i) and (ii) of the theorem in \S 3, we have then proved that
\[
\widehat{F_{f}^{\varepsilon}}(\theta) = \prod_v \widehat{F_{f_v}^{\varepsilon}}(\theta_v) = \Tr U^+_\theta(f) - \Tr U^-_\theta(f)
\]
is a Schwartz-Bruhat function. (We recall that the factors $\lambda(L_v / F_v, \psi_v)$ have a product equal to 1.)
Now applying Poisson's summation formula we obtain
\[
J_\varepsilon(f) = \frac{1}{4} \sum_{\theta \in \Theta(\varepsilon)} [\Tr U^+_\theta(f) - \Tr U^-_\theta(f)] - (F_f^{\varepsilon}(1) + F_f^{\varepsilon}(-1)) \frac{\vol([E])}{4}
\]
if the base field is of characteristic zero or $p \ne 2$; if the characteristic is 2 one should read $F^{\varepsilon}_f(1)$ instead of $F^{\varepsilon}_f(1) + F^{\varepsilon}_f(-1)$.

Let $K = \prod_v K_v$ be the maximal compact subgroup of $S(\bA)$.
If $s = k \left(\begin{smallmatrix}
    t & \\ & t^{-1}
\end{smallmatrix}\right) \left( \begin{smallmatrix}
    1 & u \\ & 1
\end{smallmatrix}\right)$ is an Iwasawa decomposition of $s \in S(\bA)$, a Haar measure on $S(\bA)$ will be given by $\dd s = |t|^2\dd k \dd^\times t \dd u$.
The measures $\dd k$ and $\dd^\times t$ are left arbitrary but $\dd u$ wil be the Tamagawa measure on $\bA$ (i.e. $\vol(\bA/ F) = 1$).
If we now choose the measure on $E(\bA)$ in such a way that $\vol([E]) = 2$ we find
\[
F_f^{\varepsilon}(\pm 1) = \prod_v F_{f_v}^{\varepsilon}(\pm 1) = \lim_{\sigma \to 1} \int_{\bI} f^{K} \begin{pmatrix} \pm 1 & 0 \\ 0 & \pm 1 \end{pmatrix} \varepsilon(t) |t|^{\sigma} \, \dd^\times t
\]
where
\[
f^K(s) = \int_{K} f(ksk^{-1}) \, \dd k.
\]

We now have to study the complementary term $C(f)$.
One can exhibit expressions $C_\varepsilon(f)$ such that
\[
C(f) = \sum_{\varepsilon \in \mathscr{E}} C_\varepsilon(f)
\]
with
\[
C_\varepsilon(f^g) = \varepsilon(\det g) C_\varepsilon(f)\quad \text{for all } g \in G(\bA).
\]
Let $(\varphi_i)$ be a sequence of Schwartz--Bruhat functions on $S(\bA)$ converging to the constant function 1 in $L^2$ defined like in \cite{duflo1971formule} p. 235 and put
\[
C'(f, \varphi_i) = \int_{[S]} \varphi_i(s)\sum_{\gamma \in S_p} f(s \gamma s^{-1}) \, \dd s.
\]
The evaluation of such integrals leads to the computation of the residue at $z = 1$ of the analytic continuation of the functions
\[
Z(f, z;a) = \int_{\bA} \int_{\bI / F^\times} \sum_{\eta \in F^\times} f^K\begin{pmatrix}
    a & (a - a^{-1})u + t^2 u \\ 0 & a^{-1}
\end{pmatrix} \frac{|t|^{z + 1}}{R(u)^{z + 1}} \dd^\times t \dd u
\]
if $\Re(z) > 1$ where $R(u)$ is defined in \cite{duflo1971formule} p. 244.
One sees immediately that
\[
Z(f, z; a) = \sum_{\varepsilon \in \mathscr{E}} \widetilde{Z}(f, z, \varepsilon; a)
\]
where
\[
\widetilde{Z}(f, z, \varepsilon; a) = \frac{1}{2} \int_{\bA} \int_{\bI} f^K\begin{pmatrix}
    a & (a - a^{-1})u + t \\ 0 & a^{-1}
\end{pmatrix} \frac{\varepsilon(t) |t|^{\frac{z + 1}{2}}}{R(u)^{z + 1}} \, \dd^\times t \dd u
\]
if $\Re z > 1$.
These integrals are introduced in \cite{duflo1971formule} p. 243 with similar notation and are studied in detail when $\varepsilon = 1$, the trivial character.
When $\varepsilon \ne 1$ one can show that if $a \ne a^{-1}$ then $\widetilde{Z}(f, z, \varepsilon; a)$ is holomorphic at $z = 1$ and hence gives no contribution.
If $\varepsilon \ne 1$ and $a = a^{-1}$ then $\widetilde{Z}(f, z, \varepsilon; a)$ has a simple pole at $z = 1$ and gives a contribution to $C(f', \varphi_i)$ which with our choice of the Haar measures is
\[
\frac{1}{2}\lim_{z \to 1} \int_{\bI} f^K \begin{pmatrix}
    a & t \\ 0 & a
\end{pmatrix} |t|^{\frac{z + 1}{2}} \varepsilon(t) \, \dd^\times t
\]
and is in $C_\varepsilon(f)$.
(This contribution is Independent of $\varphi_i$ because we assume that $\int_{[S]} \varphi_i(s) \dd s = \vol([S])$ for all $i$.)

The computation of the remaining part
\[
C''(f, \varphi_i) = - \int_{[S]} \varphi_i(s) H_f(s, s) \, \dd s
\]
can be done exactly as in \cite{duflo1971formule}.
In the result appears, among other terms, the expression $-\frac{1}{4} \Tr \widehat{f}(\varepsilon) M(\varepsilon)$ where $\widehat{f}(\varepsilon)$ is the convolution operator by $f$ in the space of the representation of the global principal series for $S(\bA)$ relative to the character $\varepsilon$; and $M(\varepsilon)$ is an intertwining operator of that representation.
One can show that this term is in $C_\varepsilon(f)$ and that if $\varepsilon \ne 1$ then
\[
C_\varepsilon(f) = \frac{1}{2} \lim_{\sigma \to 1} \int_{\bI} \sum_{\substack{a^2 = 1 \\ a \in F^\times}} f^K\begin{pmatrix}
    a & t \\ 0 & a
\end{pmatrix} \varepsilon(t) |t|^{\sigma} \, \dd^\times t - \frac{1}{4} \Tr (\widehat{f}(\varepsilon) M(\varepsilon)).
\]
Assertion (iii) of the theorem in \S 3 then is the consequence of the relation
\[
\Tr \widehat{f}(\varepsilon) M(\varepsilon) = \Tr U^+_{\theta_0}(f) - \Tr U^-_{\theta_0}(f)
\]
where $\theta_0$ is the unit element of $\Theta(\varepsilon)$.

\section{Conjectures}
\label{sec:conjectures}

We do not give here the expression for the ``stable'' term of the trace formula.
Let us simply say it is very close to the complete trace formula for $G(\bA)$.
I am led to formulate the following conjectures:
\begin{enumerate}[(i)]
    \item A representation $U \in \widehat{S(\bA)}$ occurs in $\rho^g$ for some $g \in G(\bA)$, if and only if $U$ occurs in the restriction ot $S(\bA)$ of some representation $\widetilde{U}$ of $G(\bA)$ occuring in the space of cusp forms for $G(\bA)$ relative to some character of the centre.
    \item The multiplicity one theorem is true for the representation of $S(\bA)G(F)$ on $L^2_0(S(\bA) G(F) / G(F)) = L^2_0(S(\bA) / S(F))$.
    \item If $U \in \widehat{S(\bA)}$ is unstable $m(U) = 1$ if and only if $m_\varepsilon(U) \ge 1$ for all $\varepsilon \in \mathscr{E}$ and $m(U) = 0$ otherwise.
\end{enumerate}



% --- Bibliography ---

% Start a bibliography with one item.
% Citation example: "\cite{williams}".

\nocite{*}
\bibliographystyle{acm} % We choose the "plain" reference style
\bibliography{refs} % Entries are in the refs.bib file


% \begin{thebibliography}{1}

% \bibitem{williams}
%    Williams, David.
%    \textit{Probability with Martingales}.
%    Cambridge University Press, 1991.
%    Print.

% % Uncomment the following lines to include a webpage
% % \bibitem{webpage1}
% %   LastName, FirstName. ``Webpage Title''.
% %   WebsiteName, OrganizationName.
% %   Online; accessed Month Date, Year.\\
% %   \texttt{www.URLhere.com}

% \end{thebibliography}

% --- Document ends here ---

\end{document}