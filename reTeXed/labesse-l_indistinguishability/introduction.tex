\section*{Introduction}


The results announced in this paper rest to a large extent on research done by
R. P. Langlands and the author in Bonn in May and June 1971. We shall only
sketch the proofs, and state some conjectures.

Let $F$ be a global field and $\bA$ be the ring of adeles of $F$. 
Put $G = \GL(2)$ and $S = \SL(2)$.
Let us denote by $L_0^2$ the space of cusp forms on $[S] = S(\bA) / S(F)$ and by $\rho$ the natural representation of $S(\bA)$ on $L_0^2$.
If $g$ is an element of $G(\bA)$ one can define a representation $\rho^g$ of $S(\bA)$ on $L_0^2$ by
\[
    \rho^g(s) = \rho(g s g^{-1}), \quad s \in S(\bA).
\]

The problem to be solved here answers a question asked by R. P. Langlands: ``Is the representation $\rho^g$ equivalent to $\rho$, and if not how do they differ?''
We shall show that in general they are not equivalent and give some explicit expression of their ``difference''.
The result and the method ot prove ti were guessed by Langlands.
It is a direct application of Selberg's Trace Formula for $\SL(2)$, and even merely a way of writing it.
The complete knowledge of this Trace Formula leads to some conjectures, namely the description of the spectrum of $\rho$ in terms of the spectrum of the natural representation of $G(\bA)$ in the space of cusp forms on $G(\bA) / G(F)$ studied by Jacquet and Langlands in \cite{jacquet1970automorphic}.
