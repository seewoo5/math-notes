\documentclass{article}
\usepackage{amsfonts, amssymb, amsmath, amsthm}
\usepackage{tikz}
\usepackage{tikz-cd}
\usepackage{hyperref}
\usepackage{comment}
\usepackage{mathtools}
\usetikzlibrary{positioning}

\title{Low Dimensional Complex Spin Groups}
\author{Seewoo Lee}


\newtheorem{theorem}{Theorem}
\newtheorem{lemma}{Lemma}
\newtheorem{corollary}{Corollary}
\newtheorem{proposition}{Proposition}

\newcommand{\Mod}[1]{(\text{mod }#1)}
\newcommand{\PSL}{\mathrm{PSL}}
\newcommand{\SO}{\mathrm{SO}}
\newcommand{\SL}{\mathrm{SL}}
\newcommand{\GL}{\mathrm{GL}}
\newcommand{\Spin}{\mathrm{Spin}}
\newcommand{\ad}{\mathrm{ad}}
\newcommand{\Sp}{\mathrm{Sp}}
\newcommand{\Hom}{\mathrm{Hom}}
\newcommand{\Stab}{\mathrm{Stab}}
\newcommand{\Ma}{\mathrm{M}}
\newcommand{\Tr}{\mathrm{Tr}}

\begin{document}
\maketitle

\emph{Spin group} is a universal cover of an orthogonal group. In this note, we will show that low dimensional complex spin groups, $\Spin(3, \mathbb{C}), \Spin(4, \mathbb{C}), \Spin(5, \mathbb{C})$ and $\Spin(6, \mathbb{C})$ are isomorphic to familiar groups. 

\section{$\Spin(3, \mathbb{C})\simeq \SL(2, \mathbb{C})$} 
First, we are going to prove the following theorem:
\begin{theorem}
$$
\mathrm{PSL}(2, \mathbb{C}) \simeq \mathrm{SO}(3, \mathbb{C})
$$
\end{theorem}
Using this Theorem, we can prove that  $\Spin(3, \mathbb{C})$ is isomorphic to  a well-known group.
\begin{corollary}
$$
\Spin(3, \mathbb{C})\simeq \SL(2, \mathbb{C})
$$
\end{corollary}
\begin{proof}
First, we have to show that $\SL(2, \mathbb{C})$ is a simply connected group. 
To prove this, consider a natural action $\SL(2, \mathbb{C})$ on $\mathbb{C}^{2}\backslash \{0\}$. 
Then this is a transitive action and the stabilizer subgroup of $\mathbf{e}_{1} = (1, 0)^{T}$ is 
$$
\Stab(\mathbf{e}_{1}) = \left\{ \begin{pmatrix} 1& z \\ 0 & 1 \end{pmatrix}\,:\, z\in \mathbb{C}\right\} \simeq \mathbb{C}.
$$
Hence we have a diffeomorphism 
$$
\SL(2, \mathbb{C})/\Stab(\mathbf{e}_{1})\simeq \mathbb{C}^{2}\backslash\{0\}.
$$
We know that $\mathbb{C}^{2}\backslash \{0\}$ is homotopic to $S^{3}$, which is simply connected. 
Also, since $\Stab(\mathbf{e}_{1})\simeq \mathbb{C}$ is contractible,  $\SL(2, \mathbb{C})$ is homotopic to $S^{3}$, so is simply connected. 
Now we have a 2-cover $\SL(2, \mathbb{C})\twoheadrightarrow \PSL(2, \mathbb{C})\simeq \SO(3, \mathbb{C})$, so $\SL(2, \mathbb{C})$ is a universal cover of $\SO(3, \mathbb{C})$. 
\end{proof}
So, how we can prove the Theorem 1? Actually, there is a one line proof:
\begin{proof}[proof of the Theorem 1]
The map $\Phi:\SL(2, \mathbb{C})\to \SO(3, \mathbb{C})$ defined as
$$
\begin{pmatrix}a&b\\c&d\end{pmatrix} \mapsto \begin{pmatrix} \frac{1}{2}(a^{2}-b^{2}-c^{2} + d^{2}) & -(ab-cd) & \frac{i}{2}(a^{2}+b^{2}-c^{2}-d^{2}) \\ -(ac-bd) & ad+bc & -i(ac+bd)  \\ 
-\frac{i}{2}(a^{2}-b^{2}+c^{2}-d^{2}) & i(ab+cd) & \frac{1}{2}(a^{2}+b^{2}+c^{2}+d^{2}) \end{pmatrix}
$$
is a surjective homomorphism whose kernel is the center of $\SL(2, \mathbb{C})$. 
\end{proof}

Our main question is: where does the isomorphism come from? This map seems very complicated and unnatural. 
It is not even trivial that $\phi$ is a group homomorphism and surjective. 
We will construct such homomorphism by using the \emph{adjoint} action of a Lie group on a Lie algebra. 

Let $\mathfrak{sl}(2, \mathbb{C})$ be a Lie algebra of $\SL(2, \mathbb{C})$. 
This is a 3-dimensional complex vector space with a basis 
$$
E = \begin{pmatrix} 0&1\\0&0\end{pmatrix}, \quad H = \begin{pmatrix} 1&0\\0&-1\end{pmatrix}, \quad F = \begin{pmatrix} 0&0 \\ 1&0 \end{pmatrix}.
$$
Let $G = \SL(2, \mathbb{C})$. Consider a left $G$-action on $G$ itself by a conjugation, i.e. $g\in G$ acts on $G$ by $h\mapsto ghg^{-1}$. 
Then we have an induced action of $G$ on its Lie algebra $\mathfrak{g} = \mathfrak{sl}(2, \mathbb{C})$ by conjugation $v\mapsto gvg^{-1}$ again. 
Hence we get an adjoint representation $\ad:G\to \mathrm{GL}(\mathfrak{g})$ of $G$, and we will analyze this map more rigorously. 

Let $$g = \begin{pmatrix} a&b\\c&d\end{pmatrix}\in \SL(2, \mathbb{C}).$$
Then its action on $E, H, F$ is given by 
\begin{align*}
E&\mapsto gEg^{-1}  = \begin{pmatrix} -ac & a^{2} \\ -c^{2} & ac\end{pmatrix} \\
H&\mapsto gHg^{-1} = \begin{pmatrix} ad+bc & -2ab \\ 2cd & -(ad+bc)\end{pmatrix} \\
F&\mapsto gFg^{-1}  = \begin{pmatrix} bd & -b^{2} \\ d^{2} & -bd\end{pmatrix} 
\end{align*}
so the automorphism $\phi_{g}:\mathfrak{sl}(2, \mathbb{C})\to \mathfrak{sl}(2, \mathbb{C})$ corresponds to the $g$ can be represented as a 3 by 3 matrix 
$$
\phi_{g} = \begin{pmatrix} a^{2} & -2ab & -b^{2} \\ -ac & ad+bc & bd \\ -c^{2} & 2cd & d^{2} \end{pmatrix}
$$
with respect to the ordered basis $\{E, H, F\}$ of $\mathfrak{sl}(2, \mathbb{C})$. We have $\det(\phi_{g}) = 1$: if we expand determinant by the second row, we have 
\begin{align*}
\det(\phi_{g}) &= ac(-2abd^{2}+2cdb^{2}) + (ad+bc)(a^{2}d^{2}-b^{2}c^{2}) - bd(2a^{2}cd-2c^{2}ab) \\
&= -2abcd + (ad+bc)^{2} - 2abcd = (ad-bc)^{2} =1.
\end{align*}
Now define an inner product on $\mathfrak{sl}(2, \mathbb{C})$ by 
$
\langle v, w\rangle = \Tr(vw)
$, which is clearly $\mathbb{C}$-bilinear. 
The associated quadratic form is 
$$Q_{1}(v)  = \langle v, v\rangle = \Tr(v^{2}) = 2(y^{2}+xz)$$
for $v = xE + yH + zF$, and this is a nondegenerated quadratic form. 
Clearly, this inner product and the quadratic form is invariant under the $G$-action: 
$$
\langle \phi_{g}(v), \phi_{g}(w)\rangle = \Tr(gvg^{-1}gwg^{-1}) = \Tr(gvwg^{-1}) = \Tr(vw) = \langle v, w\rangle. 
$$
Hence image of the map $g\mapsto \phi_{g}$ lies in $\SO(Q_{1})$, special orthogonal group which preserves the quadratic form $Q_{1}$. 
If we denote $Q_{2}$ as a standard quadratic form defined as $Q_{2}(xE+yH+zF) = x^{2}+y^{2}+z^{2}$, a basis change matrix
$$
B = \begin{pmatrix} \frac{1}{\sqrt{2}} & 0 & \frac{i}{\sqrt{2}} \\ 0 & \frac{1}{\sqrt{2}} & 0 \\ \frac{1}{\sqrt{2}} & 0 & -\frac{i}{\sqrt{2}} \end{pmatrix}
$$
relates two quadratic forms as $Q_{2}(v) = Q_{1}(Bv)$. 
Since $\SO(Q_{2}) = \SO(3, \mathbb{C})$, we have an isomorphism $\SO(Q_{1})\simeq \SO(3, \mathbb{C})$ given by $A\mapsto B^{-1}AB$. 
Explicit computation gives us that 
\begin{align*}
B^{-1}\phi_{g}B &= \begin{pmatrix} \frac{1}{\sqrt{2}} & 0 & \frac{1}{\sqrt{2}} \\ 0 & \sqrt{2} & 0 \\ -\frac{i}{\sqrt{2}} & 0 & \frac{i}{\sqrt{2}} \end{pmatrix}\begin{pmatrix} a^{2} & -2ab & -b^{2} \\ -ac & ad+bc & bd \\ -c^{2} & 2cd & d^{2} \end{pmatrix}\begin{pmatrix} \frac{1}{\sqrt{2}} & 0 & \frac{i}{\sqrt{2}} \\ 0 & \frac{1}{\sqrt{2}} & 0 \\ \frac{1}{\sqrt{2}} & 0 & -\frac{i}{\sqrt{2}} \end{pmatrix} \\
&=\begin{pmatrix} \frac{1}{2}(a^{2}-b^{2}-c^{2} + d^{2}) & -(ab-cd) & \frac{i}{2}(a^{2}+b^{2}-c^{2}-d^{2}) \\ -(ac-bd) & ad+bc & -i(ac+bd)  \\ 
-\frac{i}{2}(a^{2}-b^{2}+c^{2}-d^{2}) & i(ab+cd) & \frac{1}{2}(a^{2}+b^{2}+c^{2}+d^{2}) \end{pmatrix}
\end{align*}
which gives the previous homomorphism $\SL(2, \mathbb{C}) \to \SO(3, \mathbb{C})$. 

Since $\SO(Q_{1})\simeq \SO(3, \mathbb{C})$, $\ker \Phi$ is same as $\ker \phi$. 
If $\phi_{g} = \mathrm{id}$, then we should have $b = c = 0$ and $a^{2} = d^{2} = ad = 1$. 
So the only possible choice is $(a, d) = (1, 1)$ or $(-1, -1)$, which corresponds to elements in the center.
Hence we have an induced map $\PSL(2, \mathbb{C})\to \SO(3, \mathbb{C})$, which is an embedding. 

Now we only need to show that this map is an isomorphism. First, both have dimension 3: since $\SL(2, \mathbb{C})\to \PSL(2, \mathbb{C})$ is a finite cover, both group have a same dimension, which is 3 since $\dim_{\mathbb{C}} \mathfrak{sl}(2, \mathbb{C}) = 3$. 
For $\SO(3, \mathbb{C})$, one can check that $\mathfrak{so}(3, \mathbb{C}) = \{X\in M_{3\times 3}(\mathbb{C})\,:\, X^{T}+X = 0\}$, and this space also has dimension 3 over $\mathbb{C}$. 
We need the following lemma:
\begin{lemma}
Let $G$ be a conneted Lie group of dimension $n$ and $H$ be a Lie subgroup of $G$ with same dimension. 
Then $G = H$. 
\end{lemma}
\begin{proof}
Since $H\subseteq G$ is a Lie subgroup, $G/H$ has a smooth manifold structure. Since $\dim G = \dim H$, $\dim (G/H) = 0$ and thus $G/H$ is a $0$-dimensional smooth manifold, i.e. a set of points endowed wit a discrete topology. 
Since $G = \coprod_{g\in G/H} gH$, $G$ is not connected if $G\neq H$. 
\end{proof}
By lemma, it is enough to show that $\SO(3, \mathbb{C})$ is connected. Actually, we can prove more general result:
\begin{lemma}
$\SO(n, \mathbb{C})$ is connected for $n\geq 1$. 
\end{lemma}
\begin{proof}
Clearly, $\SO(1, \mathbb{C}) = \{1\}$ is connected. We will use induction on $n$. Assume that $\SO(n-1, \mathbb{C})$ is connected for some $n\geq 2$. Let 
$$
X_n = \{(x_{1}, \dots, x_{n})\in \mathbb{C}^{n}\,:\, x_{1}^{2} + \cdots + x_{n}^{2}= 1\}. 
$$
$g\in \SO(n, \mathbb{C})$ acts on $X$ as $v\mapsto gv$. 
If $ge_{1} = e_{1}$ for $e_{1} = (1, 0, \dots, 0)^{T}$, since $g\in \SO(n, \mathbb{C})$, $g$ should has a form 
$$
g = \begin{pmatrix} 1 & \mathbf{0}^{T} \\ \mathbf{0} & g'\end{pmatrix}
$$
where $g'\in \SO(n-1, \mathbb{C})$. Thus $\mathrm{Stab}(e_{1})\simeq \SO(n-1, \mathbb{C})$. 
Also, we can show that the action is transitive, hence we have $\SO(n, \mathbb{C})/\SO(n-1, \mathbb{C})\simeq X_{n}$. 
It is known that $X_{n}$ is connected, so $\SO(n, \mathbb{C})$ is also connected since both $\SO(n-1, \mathbb{C})$ and $X_{n}$ are connected.  (First one is because of the induction hypothesis and the Lemma \ref{con} in the Appendix. For the second one, in general, for any irreducible polynomial $f(x_{1}, \dots, x_{n})\in \mathbb{C}[x_{1}, \dots, x_{n}]$, zero set of $f$ in $\mathbb{C}^{n}$ is connected with respect to the usual topology on $\mathbb{C}^{n}$, which is hard to prove in general.)
\end{proof}
Thus we get $\PSL(2, \mathbb{C})\simeq \SO(3, \mathbb{C})$ with an explicit isomorphism. 
Since $\SL(2, \mathbb{C})$ is a double cover of $\PSL(2, \mathbb{C})$ and is simply connected, we just showed that the complex spin group $\Spin(3, \mathbb{C})$ is $\SL(2, \mathbb{C})$. 


\section{$\Spin(4, \mathbb{C})\simeq \SL(2, \mathbb{C})\times \SL(2, \mathbb{C})$}
By the similar way, we can also show the following:
\begin{theorem}
$$
(\SL(2, \mathbb{C})\times\SL(2, \mathbb{C})) / \langle (-I, -I)\rangle \simeq \SO(4, \mathbb{C}). 
$$
\end{theorem}
\begin{corollary}
$$\Spin(4, \mathbb{C})\simeq \SL(2, \mathbb{C})\times \SL(2, \mathbb{C}).$$ 
\end{corollary}
\begin{proof}
By the Theorem 2, there exists a surjective homomorphism $\SL(2, \mathbb{C})\times \SL(2, \mathbb{C})\to \SO(4, \mathbb{C})$, which is a double cover of $\SO(4, \mathbb{C})$. Since $\SL(2, \mathbb{C})$ is simply connected, $\SL(2, \mathbb{C})\times \SL(2, \mathbb{C})$ is also simply connected. 
\end{proof}

To prove the Theorem 2, we may find some appropriate $\SL(2, \mathbb{C})\times \SL(2, \mathbb{C})$ on a 4-dimensional $\mathbb{C}$-vector space. 
Consider the action of the group on a space $\Ma_{2\times 2}(\mathbb{C})$ (the space of complex $2\times 2$ matrices) defined as 
$$
(g, h)\cdot v: = gvh^{-1}, \quad (g, h)\in \SL(2, \mathbb{C})\times\SL(2, \mathbb{C}), \, v\in \Ma_{2\times 2}(\mathbb{C}). 
$$
Then this is a well-defined action on $\Ma_{2\times 2}(\mathbb{C})$. With respect to the basis $\{\mathbf{e}_{11}, \mathbf{e}_{12}, \mathbf{e}_{21}, \mathbf{e}_{22}\}$, the map $\phi_{g, h}:\Ma_{2\times 2}(\mathbb{C})\to \Ma_{2\times 2}(\mathbb{C})$ corresponds to a $4\times 4$ matrix
$$
A_{g, h}=
\begin{pmatrix}
a\delta & -a\gamma & b\delta & -b\gamma \\
-a\beta & a\alpha & -b\beta & b\alpha \\
c\delta & -c\gamma & d\delta & -d\gamma \\
-c\beta & c\alpha & -d\beta & d\alpha
\end{pmatrix}, \quad g = \begin{pmatrix} a&b\\c&d \end{pmatrix} , \,\, h = \begin{pmatrix} \alpha&\beta \\ \gamma & \delta\end{pmatrix} \in \SL(2, \mathbb{C}).
$$
We can check that $\det(A_{g, h}) = (ad-bc)(\alpha\delta - \beta\gamma) = 1$, so the image of $$\phi:\SL(2, \mathbb{C})\times\SL(2, \mathbb{C})\to \GL(\Ma_{2\times 2}(\mathbb{C}))$$ is contained in $\SL(\Ma_{2\times 2}(\mathbb{C}))$. 
Now we need a bilinear map and a (non-degenerate) quadratic form on $\Ma_{2\times 2}(\mathbb{C})$ so that $\phi_{g, h}$ preserves the quadratic form. 
Define $Q:\Ma_{2\times 2}(\mathbb{C})\to \mathbb{C}$ as the \emph{determinant}, i.e. $Q(v) = \det(v)$. 
Then we have $$Q(\phi_{g, h}(v)) = \det(gvh^{-1}) = \det(g)\det(v)\det(h)^{-1} = \det(v) = Q(v),$$ so it is preserved by the action. 
The corresponding inner product can be defined as
$$
\langle v, w\rangle = \frac{1}{2}\left( Q(v+w)-Q(v)-Q(w)\right) = \frac{1}{2}(\det(v+w) - \det(v) - \det(w))
$$
which satisfies $\langle v, v\rangle = Q(v)$. Simple computation shows that the inner product is given as
$$
\langle v, w\rangle = \frac{1}{2}(x_{1}w_{2} - y_{1}z_{2} + x_{2}w_{1}- y_{2}z_{1}), \quad v = \begin{pmatrix} x_{1} & y_{1} \\ z_{1} & w_{1}\end{pmatrix}, \,\, w = \begin{pmatrix} x_{2} & y_{2} \\ z_{2} & w_{2}\end{pmatrix} \in \Ma_{2\times 2}(\mathbb{C}).  
$$
Hence we obtain a map $\phi:\SL(2, \mathbb{C})\times \SL(2, \mathbb{C})\to \SO(4, \mathbb{C})$. 
If $(g, h)\in \ker\phi$, then $gvh^{-1} = v$ for any $v\in \Ma_{2\times 2}(\mathbb{C})$. If we put $v = h$, we get $g = h$ and $gvg^{-1} = v$. Thus $g\in Z(\Ma_{2\times 2}(\mathbb{C})) = \mathbb{C}I_{2}$. 
Since $\det(g) = 1$, we should have $g = h = \pm I_{2}$, and $\phi$ induces an injection 
$$
\phi:(\SL(2, \mathbb{C})\times\SL(2, \mathbb{C})) / \langle (-I, -I)\rangle \to  \SO(4, \mathbb{C}). 
$$
One can check that both groups has dimension 6 (by computing dimensions of Lie algebra of each groups), so $\phi$ is an isomorphism by the Lemma 1 and 2. 




\section{$\Spin(6, \mathbb{C})\simeq \SL(4, \mathbb{C})$}
We are going to see $\Spin(6, \mathbb{C})$ first since $\Spin(5, \mathbb{C})$ uses similar technic but slightly more difficult. 
For $\Spin(6, \mathbb{C})$, we have the following isomorphism
\begin{theorem}
$$\SL(4, \mathbb{C})/\langle -I\rangle \simeq \SO(6, \mathbb{C}).$$
\end{theorem}
\begin{corollary}
$$
\Spin(6, \mathbb{C})\simeq \SL(4, \mathbb{C}). 
$$
\end{corollary}

\begin{proof}
More generally, we will prove that $\SL(n, \mathbb{C})$ is simply connected for $n\geq 2$. 
The natural action of $\SL(n, \mathbb{C})$ on $\mathbb{C}^{n}\backslash \{0\}$ is transitive, so we have a diffeomorphism 
$$
\SL(n, \mathbb{C})/\Stab(\mathbf{e}_{1}) \simeq \mathbb{C}^{n}\backslash \{0\}. 
$$
We know that $\mathbb{C}^{n}\backslash \{0\}$ is homotopic to $S^{2n-1}$, which is simply connected. 
Also, we have
$$
\Stab(\mathbf{e}_{1}) = \left\{\begin{pmatrix} 1 & \mathbf{v}^{T} \\ \mathbf{0} & g\end{pmatrix}\,:\, g\in \SL(n-1, \mathbb{C})\right\}\simeq \SL(n-1, \mathbb{C}) \rtimes\mathbb{C}^{n-1}
$$
which is diffeomorphic to $\SL(n-1, \mathbb{C})\times \mathbb{C}^{n-1}$, so is homotopic to $\SL(n-1, \mathbb{C})$. 
Thus by induction with the Lemma \ref{simply}, $\SL(n, \mathbb{C})$ is simply connected for any $n\geq 2$. 
\end{proof}

What is a 6-dimensional $\mathbb{C}$-vector space that $\SL(4, \mathbb{C})$ can act on?  $\SL(4, \mathbb{C})$ naturally acts on $\mathbb{C}^{4}$, and this induces an action on $\wedge^{2}\mathbb{C}^{4}$, which has a dimension $\binom{4}{2}= 6$. The action is defined as
$$
g(v_{1}\wedge v_{2}) = gv_{1}\wedge gv_{2},\quad g\in \SL(4, \mathbb{C}), \,v_{1}, v_{2}\in \mathbb{C}^{4}. 
$$
It is easy to prove that this action has determinant 1. Recall that for any finite $d$-dimensional $\mathbb{C}$-vector space $V$, we have the determinant map $\det:\GL(V)\to \mathbb{C}^{\times}$ defined as $g\mapsto \det(g) = \wedge^{d}g$, where $\wedge^{d}g$ is the induced map on $\wedge^{d}V$ which is isomorphic to $\mathbb{C}$. 
Now we define $\SL(V) = \ker(\det)$. 
In our case, for $g\in \SL(4, \mathbb{C}) = \SL(\mathbb{C}^{4})$, we have to check that $\wedge^{2}g\in \SL(\wedge^{2}\mathbb{C}^{4})$, which trivially follows from $\wedge^{4}g \in \ker(\det:\GL(\wedge^{4}\mathbb{C}^{4})\to \mathbb{C}^{\times})$: then $\wedge^{4}g$ acts on $\wedge^{4}\mathbb{C}^{4}$ trivially, and then $\wedge^{6}(\wedge^{2}g) = \wedge^{3}(\wedge^{4}g)$ also acts on $\wedge^{6}(\wedge^{2}\mathbb{C}^{4})$ trivially.  

For a bilinear pairing on $\wedge^{2}\mathbb{C}^{4}$, define it as 
$$
\langle v_{1}\wedge v_{2}, w_{1}\wedge w_{2}\rangle = v_{1}\wedge v_{2}\wedge w_{1}\wedge w_{2}\in \wedge^{4}\mathbb{C}^{4}\simeq \mathbb{C}, \quad v_{1}, v_{2}, w_{1}, w_{2}\in \mathbb{C}^{4}
$$
This is a symmetric pairing on $\wedge^{2}\mathbb{C}^{4}$ since $(13)(24)\in S_{4}$ is an even permutation. Actually, this pairing can be considered as a determinant of the matrix with column vectors $v_{1}, v_{2}, w_{1}, w_{2}$, and linearly extends to $\wedge^{2}\mathbb{C}^{4}$. 
One can check that this is a nondegenerate paring on $\wedge^{2}\mathbb{C}^{4}$, so we get a map 
$$
\phi:\SL(4, \mathbb{C})\to \SO(Q, \wedge^{2}\mathbb{C}^{4}) \simeq \SO(6, \mathbb{C}). 
$$
where the nondegenerate quadratic form $Q$ on $\wedge^{2}\mathbb{C}^{4}$ corresponds to the above pairing is 
$$
Q(v) = \langle v, v\rangle = 2(a_{12}a_{34}-a_{13}a_{24}+a_{14}a_{23}), \quad v = \sum_{i<j}a_{ij}(e_{i}\wedge e_{j}). 
$$
To show that the kernel of $\phi$ is $\langle -I\rangle$, assume that $g = (b_{ij})_{1\leq i, j\leq 4}\in SL(4, \mathbb{C})$ trivially acts on $\wedge^{2}\mathbb{C}^{4}$. From $ge_{1}\wedge ge_{2} = e_{1}\wedge e_{2}$, we get, for example, the following equations:
\begin{align*}
b_{11}b_{22}-b_{21}b_{12} = 1 \\
b_{11}b_{32} - b_{31}b_{12} = 0 \\
b_{21}b_{32}-b_{31}b_{22} = 0
\end{align*}
by comparing coefficients of each basis elements $e_{i}\wedge e_{j}$ ($i<j$). Then 
$$
b_{31} = b_{31}b_{11}b_{22} - b_{31}b_{21}b_{12} = b_{21}b_{32}b_{11} - b_{11}b_{21}b_{32} = 0.
$$
By the similar way, we can prove that all off-diagonal entries of $g$ are zero, and $b_{ii}b_{jj}  = 1$ for all $i<j$ gives $b_{11} = b_{22} = b_{33} = b_{44} = \pm 1$. 




\section{$\Spin(5, \mathbb{C})\simeq \Sp(4, \mathbb{C})$}
This is the most difficult one among $3, 4, 5$ and $6$, since we need to consider symplectic structure.
The \emph{standard symplectic form} on $\mathbb{C}^{4}$ is a bilinear map $\omega:\mathbb{C}^{4}\times \mathbb{C}^{4}\to \mathbb{C}$ which is anti-symmetic and nondegenerate. 
More explicitly, it is defined as
$$
\omega(v_{1}, v_{2}):= v_{1}^{T}Jv_{2}, \quad J = \begin{pmatrix} O & I_{2} \\ -I_{2} & O \end{pmatrix}, \,\, v_{1}, v_{2}\in \mathbb{C}^{4}.
$$
Since $\omega$ is anti-symmetric, we have an induced map $\omega:\wedge^{2}\mathbb{C}^{4}\to \mathbb{C}$. The standard action of $\Sp(4, \mathbb{C})$ on $\mathbb{C}^{4}$ induces the action on the space $\wedge^{2}\mathbb{C}^{4}$ as before, and its dual action on $V' = \Hom_{\mathbb{C}}(\wedge^{2}\mathbb{C}^{4}, \mathbb{C})$ defined as 
$$
(g\cdot f)(v_{1}\wedge v_{2}) = f(g^{-1}v_{1}\wedge g^{-1}v_{2}), \quad f:\wedge^{2}\mathbb{C}^{4}\to \mathbb{C}. 
$$
We can easily check that $\omega$ is fixed by the action (in some sense, $\Sp(4, \mathbb{C})$ is \emph{defined} to be the group that fixes $\omega$), and in fact, it is a unique such element in $V'$ up to constant multiplication. 
This would be the heart of the our following proof. 

\begin{theorem}
$$\mathrm{PSp}(4, \mathbb{C}) = \Sp(4, \mathbb{C})/\langle -I\rangle \simeq \SO(5, \mathbb{C})$$
\end{theorem}
\begin{corollary}
$$\Spin(5, \mathbb{C})\simeq \Sp(4, \mathbb{C}).$$
\end{corollary}
\begin{proof}
It is enough to show that $\Sp(4, \mathbb{C})$ is simply connected. We use the argument of Eric Wofsey in \cite{mse}. Consider the standard action of $\Sp(4, \mathbb{C})$ on $\mathbb{C}^{4}\backslash \{0\}$. This action is transitive: choose any nonzero vector $v = (a_{11}, a_{21}, c_{11}, c_{21})^{T}\in\mathbb{C}^{4}$. 
Note that the matrix $g = \left(\begin{smallmatrix} A&B\\C&D\end{smallmatrix}\right)$ is in $\Sp(4, \mathbb{C})$ if and only if 
$$
A^{T}C = C^{T}A, \quad B^{T}D=D^{T}B, \quad A^{T}D-C^{T}B=I. 
$$
Assume that $(a_{11}, a_{21})\neq (0, 0)$. Then we can find $a_{21}, a_{22}\in \mathbb{C}$ s.t. $a_{11}a_{22}-a_{12}a_{21}\neq 0$. Then we can also find $c_{12}, c_{22}\in \mathbb{C}$ s.t. $$a_{11}c_{12}+a_{21}c_{22}=a_{12}c_{11}+a_{22}c_{21},$$
which implies $A^{T}C = C^{T}A$ for $A = \left(\begin{smallmatrix}a_{11}&a_{12} \\ a_{21}&a_{22}\end{smallmatrix}\right)$ and $C = \left(\begin{smallmatrix}c_{11}&c_{12}\\c_{21}&c_{22}\end{smallmatrix}\right)$. 
Now take $D = A^{-T} = (A^{-1})^{T}$ and $B=O$, then we have $\left(\begin{smallmatrix}A&B\\C&D\end{smallmatrix}\right)\in \Sp(4, \mathbb{C})$. If $(a_{11}, a_{21})=(0, 0)$, we have $(c_{11}, c_{21})\neq (0, 0)$ and do the similar thing with $D=O$. 

Now consider the diagonal action of $\Sp(4, \mathbb{C})$ on $(\mathbb{C}^{4}\backslash\{0\})\times (\mathbb{C}^{4}\backslash\{0\})$. We will figure out what is the orbit and the stabilizer of the element $(e_{1}, e_{3})\in (\mathbb{C}^{4}\backslash\{0\})\times(\mathbb{C}^{4}\backslash\{0\})$. 
First, assume that $g\in \Sp(4, \mathbb{C})$ fixes $e_{1}$ and $e_{3}$. 
Since $\omega$ is preserved under the action, $g$ must also fix their orthogonal complement with respect to the symplectic form, which is $\mathbb{C}e_{2}\oplus \mathbb{C}e_{4}$. 
So we can see that the stabilizer group of $(e_{1}, e_{3})$ is isomorphic to $\Sp(2, \mathbb{C}) =\SL(2, \mathbb{C})$, which is simply connected. 

For the orbit of $(e_{1}, e_{3})$, we just saw that $\Sp(4, \mathbb{C})$ acts on $\mathbb{C}^{4}\backslash\{0\}$ transitively, so we can map $e_{1}$ to the any vector in $\mathbb{C}^{4}\backslash\{0\}$. Once we choose the image of $e_{1}$, then $e_{3}$ may goes to some vector that lies on the affine space $$S = \{v\in \mathbb{C}^{4}\backslash\{0\}\,:\, \omega(ge_{1}, v) = \omega(e_{1}, e_{3})=1\}$$
which is just $\mathbb{C}^{3}$ topologically. 
Hence our orbit space is a fiber bundle over $\mathbb{C}^{4}\backslash\{0\}$ with fiber $\mathbb{C}^{3}$. Since both $\mathbb{C}^{4}\backslash\{0\}$ and $\mathbb{C}^{3}$ are simply connected, the orbit space should be simply connected, too. 

So both stabilizer and the orbit of $(e_{1}, e_{3})$ are simply connected, and so $\Sp(4, \mathbb{C})$ too by the Lemma \ref{simply}. 
\end{proof}
Now consider the  non-degenerate bilinear paring on $\wedge^{2}\mathbb{C}^{4}$, defined as
$$
\langle v_{1}\wedge v_{2}, v_{3}\wedge v_{4}\rangle = v_{1}\wedge v_{2}\wedge v_{3}\wedge v_{4}\in \wedge^{4}\mathbb{C}^{4}\simeq \mathbb{C}, \quad v_{i}\in\mathbb{C}^{4}\,\text{ for }1\leq i\leq 4. 
$$ 
Then we have an isomorphism 
$$
\wedge^{2}\mathbb{C}^{4}\simeq \Hom_{\mathbb{C}}(\wedge^{2}\mathbb{C}^{4}, \mathbb{C}), \quad v_{1}\wedge v_{2}\mapsto \langle v_{1}\wedge v_{2}, -\rangle 
$$
which is a $\Sp(4, \mathbb{C})$-equivariant isomorphism, from the fact that $\Sp(4, \mathbb{C})\subseteq \SL(4, \mathbb{C})$. (This was proven in the previous section.) 
Hence there exists a nonzero vector $v_{\omega}$ in $\wedge^{2}\mathbb{C}^{4}$ which is fixed by the action that corresponds to $\omega$, and we can compute it explicitly - $\{e_{i}\wedge e_{j}\}_{1\leq i<j\leq 4}$ is a basis of $\wedge^{2}\mathbb{C}^{4}$, and if we write dual basis of $e_{i}\wedge e_{j}$ as $(e_{i}\wedge e_{j})^{*}$, then $\omega = (e_{1}\wedge e_{3})^{*}+(e_{2}\wedge e_{4})^{*}$ and the corresponding element that fixed by $\Sp(4, \mathbb{C})$-action is 
$$
v_{\omega} = - e_{1}\wedge e_{3} - e_{2}\wedge e_{4}. 
$$
So we have an induced action on the 5-dimensional vector space $V = \wedge^{2}\mathbb{C}^{4}/\langle v_{\omega}\rangle$, which is the desired space that $\Sp(4, \mathbb{C})$ acts on. 
Now define a bilinear paring on $V$ as
$$
\langle v_{1}\wedge v_{2}, v_{3}\wedge v_{4}\rangle_{V} := \omega(v_{1}\wedge v_{3})\omega(v_{2}\wedge v_{4}) - \omega(v_{1}\wedge v_{4})\omega(v_{2}\wedge v_{3}). 
$$
We can check that this is a well-defined on $V$ by checking that $\langle v_{\omega}, e_{i}\wedge e_{j}\rangle_{V} = 0$ for any $1\leq i<j\leq 4$. Also, this is a $\Sp(4, \mathbb{C})$-invariant bilinear paring since $\omega$ does. 
By the Lemma \ref{det} and the previous section, the action on $V$ has determinant 1, hence the image of the representation $\Sp(4, \mathbb{C})\hookrightarrow \GL(V)$ lies in $\SO(5, \mathbb{C})$. 

To prove that the kernel of the map is $\langle -I\rangle$, assume that $g\in \Sp(4, \mathbb{C})$ is in the kernel, so that $gv_{1}\wedge gv_{2} = v_{1}\wedge v_{2}$ for all $v_{1}\wedge v_{2}\in V$. This means that $gv_{1}\wedge gv_{2}  = v_{1}\wedge v_{2} + \lambda v_{\omega}$ for some $\lambda \in \mathbb{C}$. 
Now define $\lambda_{ij}\in \mathbb{C}$ as $ge_{i}\wedge ge_{j} = e_{i}\wedge e_{j} + \lambda_{ij}(e_{1}\wedge e_{3} + e_{2}\wedge e_{4})$. Let $g = (b_{ij})_{1\leq i, j\leq 4}$. For $(i, j) = (1, 3)$,  we get the following equations
\begin{align*}
b_{11}b_{23}-b_{21}b_{13} &=0 \\
b_{11}b_{33}-b_{13}b_{31} &=\lambda_{13}+1 \\
b_{11}b_{43}-b_{41}b_{13} &=0\\
b_{21}b_{33}-b_{31}b_{23} &=0 \\
b_{21}b_{43}-b_{41}b_{23} &=\lambda_{13}\\
b_{31}b_{43}-b_{41}b_{33} &= 0
\end{align*}
and by using the same trick as before, we get 
\begin{align*}
(\lambda_{13}+1)(b_{21}, b_{23}, b_{41}, b_{43}) &= (0, 0, 0, 0) \\
\lambda_{13}(b_{11}, b_{13}, b_{31}, b_{33}) &= (0, 0, 0, 0). \\
\end{align*}
Now we can prove that $\lambda_{13}=0$ - if not, we must have $b_{11} = b_{31} = b_{13} = b_{33} =0$, and this gives a contradiction when we do the similar computation for $(i, j) = (1, 2)$ and $(i, j) = (1, 4)$. (I'm not going to write down all the equations since margin is too small to contain.) 
Hence we must have $\lambda_{13} =0$ and $b_{21} =b_{23} = b_{41} = b_{43} =0$. Similar argument shows that the off-diagonal elements should be all zero, and the diagonal entries should satisfy $b_{ii}b_{jj} = 1$, which implies $b_{11} = b_{22} = b_{33} = b_{44} = \pm 1$. 




\subsection{Appendix}

\begin{lemma}
\label{con}
Let $G$ be a Lie group and $H$ be a closed Lie subgroup. 
If both $H$ and $G/H$ are connected, then $G$ is also connected. 
\end{lemma}
\begin{proof}
Assume that $G$ is not connected. 
Then there exists a proper clopen subset $U$ of $G$. Since $H$ is connected, $U$ is a union of some cosets of $H$, and then $U/H$ is a proper clopen subset of $G/H$, which contradicts to the connectedness of $G/H$. 
\end{proof}


\begin{lemma}
\label{simply}
Let $G$ be a connected Lie group and $H$ be a closed Lie subgroup. If both $H$ and $G/H$ are simply connected, then $G$ is also simply connected. 
\end{lemma}
\begin{proof}
The canonical projection $G\to G/H$ is an $H$-fibration, so we obtain a long exact sequence of homotopy groups
$$
\cdots \to \pi_{2}(G/H)\to \pi_{1}(H) \to \pi_{1}(G) \to \pi_{1}(G/H) \to \pi_{0}(H) \to \cdots. 
$$
Since both $\pi_{1}(H)$ and $\pi_{1}(G/H)$ are trivial, so is $\pi_{1}(G)$. 
\end{proof}

\begin{lemma}
\label{det}
Let $V$ be a $d$-dimensional $\mathbb{C}$-vector space and $\phi:V\to V$ be the invertible linear map, i.e. $\phi\in \GL(V)$. Assume that there exists a nonzero vector $v_{0}\in V$ which is fixed by $\phi$. 
If $\phi\in \SL(V)$, then the induced map $\overline{\phi}:V/\langle v_{0}\rangle \to V/\langle v_{0}\rangle$ also satisfies $\overline{\phi}\in \SL(V/\langle v_{0}\rangle)$. 
\end{lemma}
\begin{proof}
Consider a basis $\mathcal{B} = \{v_{0}, v_{1}, \dots, v_{d-1}\}$ which contains $v_{0}\neq 0$. 
Then $\overline{\mathcal{B}} = \{\overline{v_{1}}, \dots, \overline{v_{d-1}}\}$ is a basis of $V/\langle v_{0}\rangle$. 
Since $\phi\in \SL(V)$, it acts on $\wedge^{d}V$ trivially, i.e. $$\phi(v_{0})\wedge \phi(v_{1})\wedge \cdots \wedge \phi(v_{d-1}) = v_{0}\wedge v_{1}\wedge \cdots \wedge v_{d-1}.$$
Since $\phi(v_{0}) = v_{0}$,  we have
$$
v_{0}\wedge (\phi(v_{1})\wedge \cdots \wedge \phi(v_{d-1}) - v_{1}\wedge \cdots \wedge v_{d-1}) = 0, 
$$
which implies that 
$$
\phi(v_{1})\wedge\cdots\wedge \phi(v_{d-1}) - v_{1}\wedge\cdots \wedge v_{d-1} = \sum_{i=1}^{d-1}c_{i}(v_{1}\wedge\cdots\wedge\widehat{v_{i}}\wedge v_{d-1})
$$
for some $c_{1}, \dots, c_{d-1}\in \mathbb{C}$. 
Since the image of RHS in $\wedge^{d-1}(V/\langle v_{0}\rangle)$ is 0, we get
Now we have to show that $\overline{\phi(v_{1})}\wedge\cdots\wedge \overline{\phi(v_{d-1})} = \overline{v_{1}}\wedge\cdots\wedge \overline{v_{d-1}}$ and so $\overline{\phi}\in \SL(V/\langle v_{0}\rangle)$.  
\end{proof}


\begin{thebibliography}{5}
\bibitem{spinor}
P. Deligne, \emph{Notes on Spinors}, Quantum fields and strings: a course for mathematicians 1 (1999):2. 
\bibitem{mse}
Eric Wofsey, Answer to the MSE question \emph{"$\Sp(4, \mathbb{C})$ is simply connected"}, \url{https://math.stackexchange.com/a/2931022/350772}. 
\end{thebibliography}
\end{document}