\documentclass{amsart}
\usepackage[utf8]{inputenc}
\usepackage{amsmath, amsthm, amssymb}
\usepackage{url}
\usepackage{mathrsfs}
\usepackage{empheq}
\usepackage{tikz-cd}

\newtheorem{theorem}{Theorem}[section]
\newtheorem{conjecture}{Conjecture}[section]
% \newtheorem*{conjecture*}{Conjecture}[section]
\newtheorem{lemma}{Lemma}
\newtheorem{proposition}{Proposition}[section]
\newtheorem{corollary}{Corollary}[section]
\newtheorem{definition}{Definition}[section]

\DeclareMathOperator{\GL}{\mathrm{GL}}
\DeclareMathOperator{\SL}{\mathrm{SL}}
\DeclareMathOperator{\PGL}{\mathrm{PGL}}
\DeclareMathOperator{\Ad}{\mathrm{Ad}}
\DeclareMathOperator{\Oo}{\mathrm{O}}
\DeclareMathOperator{\Gal}{\mathrm{Gal}}
\DeclareMathOperator{\AI}{\mathrm{AI}}
\DeclareMathOperator{\L2}{\mathrm{L}^{2}}
\DeclareMathOperator{\cusp}{\mathrm{cusp}}
\DeclareMathOperator{\Spec}{\mathrm{Spec}}
\DeclareMathOperator{\Pic}{\mathrm{Pic}}
\DeclareMathOperator{\Sym}{\mathrm{Sym}}
\DeclareMathOperator{\Jac}{\mathrm{Jac}}
\DeclareMathOperator{\Fr}{\mathrm{Fr}}
\DeclareMathOperator{\Tr}{\mathrm{Tr}}
\DeclareMathOperator{\rH}{\mathrm{H}}
\DeclareMathOperator{\Loc}{\mathsf{Loc}}
\DeclareMathOperator{\Bun}{\mathsf{Bun}}
\DeclareMathOperator{\AJ}{\mathrm{AJ}}

\newcommand{\xrightarrowdbl}[2][]{%
    \xrightarrow[#1]{#2}\mathrel{\mkern-14mu}\rightarrow
}

\title{Analytic Langlands Program}
\author{Seewoo Lee}


\begin{document}

\begin{abstract}
This is a \LaTeX-ed note for the special lecture on \emph{Analytic Langlands Program} by Edward Frankel at UC Berkeley in 2022 Fall.

\end{abstract}

\maketitle
\tableofcontents

\newpage
\section{Introduction to Langlands correspondence (August 25)}

This course is on a new aspect of Langlands program, so-called \emph{Analytic Langlands Program}.
The classical Langlands program is originated from Langlands' letter to Andr\'e Weil in 1967,
and also from Andr\'e Weil's letter to his sister  (Simone Weil) on his conjecture (Weil's conjecture on
zeta functions of curves over finite fields, which was resolved by Dwork, Grothendieck, and Deligne) in 1940.
Weil's \emph{Rosetta stone} relates two different topics in mathematics: number theory and complex curves (Riemann surfaces).
A goal is to find something happens in parallel between two, and we need another bridge - curves over finite fields.
The difference between complex curves and curves over finite fields is the fact that those are defined over different fields.
A similarity between number theory sied and the curves over finite fields side is that 
the number fields (finite extensions of $\mathbb{Q}$) are similar to the function fields of curves (over finite fields - we denote it as $\mathbb{F}_{q}(X)$ for a curve $X/\mathbb{F}_{q}$).
The most simplest example is a comparison between $\mathbb{Q}$ and $\mathbb{F}_{q}(\mathbb{P}^{1}) \simeq \mathbb{F}_{q}(t)$:
\begin{align*}
    \mathbb{Q} = \left\{ \frac{p}{q}\,:\,p, q\,\text{rel. prime} \in \mathbb{Z}\right\} &\leftrightarrow \mathbb{F}_{q}(t) = \left\{ \frac{P(t)}{Q(t)}\,:\, P, Q\,\text{rel. prime} \in \mathbb{F}_{q}[t]\right\} \\
    \text{ring of integers: }\mathbb{Z} &\leftrightarrow \mathbb{F}_{q}[t] \\
    \text{completions: }\mathbb{Q}_{p} &\leftrightarrow \mathbb{F}_{q}((t))
\end{align*}
Sometimes we include one more topic in this Rosetta stone, which originates from Physics -
Quantum Field Theory, Electro-Magnetic Duality, and Gauge Theory (developed by Edward Witten and other physicists).

The (classical) Langlands correspondence is about interplays between the Galois representations and Automorphic representations.\footnote{The \emph{geometric} Langlands correspondence is about curves over $\mathbb{C}$, which is mainly developed by Drinfeld, Laumon, Beilinson, Gaitsgory, ...}
It dealts with two different (but similar) types of fields - number fields and function fields of curves over finite fields.
Fix such a field $F$.
Then we can describe a Langlands correspondence for $\GL_{n}$ over $F$ as follows:
\begin{enumerate}
    \item \textbf{Galois side:} Let $\overline{F}$ be a (separable) algebraic closure of $F$, and $\Gal(\overline{F}/F)$ 
    be the absolute Galois group of $F$ (i.e. the group of automorphisms of $\overline{F}$ that fix $F$ pointwisely).
    This is one of the most important groups in number theory, and its structure is higly complicated.
    Hence, instead of studying the group $\Gal(\overline{F}/F)$ directly, we study the representations of it.
    Especially, we are going to consider the (equivalence) classes of $n$-dimensional representations of $\Gal(\overline{F} / F)$ over some field that would be determined later.    
    This is just an equivalence class of homomorphisms
    $$
        \sigma: \Gal(\overline{F} / F) \to \GL_{n}(?).
    $$
    \item \textbf{Automorphic side:} We first need to define the notion of \emph{Adele}.
    Let $\mathscr{V} = \mathscr{V}_{F}$ be the set of equivalent classes of the norms (places) on $F$.
    For each $v \in \mathscr{V}$, we can define a completion $F_{v}$ with respect to $v$.
    For example, when $F = \mathbb{Q}$, Ostrowski's theorem states that the places of $\mathbb{Q}$ corresponds to
    the set of primes (each prime $p$ gives $p$-adic norms, which is non-archimedean) along with the ``infinite'' prime (corresponds to the usual archimedean norm).
    In this case, we have two types of completions, either $p$-adic numbers $\mathbb{Q}_{p}$ or real numbers $\mathbb{R}$.
    And we have the ring of $p$-adic integers $\mathbb{Z}_{p}$ as a subring of $\mathbb{Q}_{p}$.

    In case of function field $F =\mathbb{F}_{q}(\mathbb{P}^{1}) \simeq \mathbb{F}_{q}(t)$ of $X = \mathbb{P}^{1}$,
    the places of $F$ corresponds to the closed points of $X$, which again corresponds to
    the maximal ideals of $\mathbb{F}_{q}[t]$ (and the point at infinity).
    For example, any $a \in \mathbb{F}_{q}$ actually gives a closed point that corresponds to
    the maximal ideal $(x-a)$.
    Any other irreducible polynomials over $\mathbb{F}_{q}$ of higher degree also give closed points in $X$.
    Completion of $F$ at $x \in X$ is isomorphic to the field of formal Laurent series $(\mathbb{F}_{q})_{x}((t_{x}))$,
    where $(\mathbb{F}_{q})_x$ is the residue field at $x$ and $t_{x}$ is some parameter.
    We also have a ring of integers in these completions, which is $(\mathbb{F}_{q})_{x}[[t_{x}]]$.

    The ring of adeles is defined as a restricted product of all completions of $F$, which is
    $$
        \quad\quad\quad\quad \mathbb{A}_{F} = \prod_{v \in \mathscr{V}_{F}} F_{v} = \left\{(f_{v})\,:\, f_{v} \in F_{v}, f_{v} \in \mathcal{O}_{v}\text{ for all but finitely many $v$.}\right\}
    $$
    where $\mathcal{O}_{v} \subset F_{v}$ is the ring of integers of $F_{v}$, which is the set of elements with norm at most 1.
    Then we have a diagonal embedding $F \hookrightarrow \mathbb{A}_{F}$ that sends $a \in F$ to $(a, a, \dots) \in \mathbb{A}_{F}$,
    and this induces an embedding $\GL_{n}(F) \hookrightarrow \GL_{n}(\mathbb{A}_{F})$.
    Now we can think of a Hilbert space $\mathscr{H}(\GL_{n}(F) \backslash \GL_{n}(\mathbb{A}_{F}))$ of $\L2$-functions on the quotient space $\GL_{n}(F) \backslash \GL_{n}(\mathbb{A}_{F})$,
    with the Haar measure on the quotient space.
    Then we have a right regular representation of $\GL_{n}(\mathbb{A}_{F})$ on $\mathscr{H}$, and it is known that this representation decomposes
    into continuous part and discrete part:
    $$
        \mathscr{H} = \mathscr{H}_{\mathrm{cont}} \oplus \mathscr{H}_{\mathrm{disc}}
    $$
    and the discrete part decomposes into irreducible representations as 
    $$\mathscr{H}_{\mathrm{disc}} = \oplus_{\pi}\pi$$
    without multiplicity (multiplicity one theorem).\footnote{
        It is not always the case that any representation decomposes into irredubiles - 
        consider the regular representation of $\mathbb{R}$ on $\L2(\mathbb{R})$ that acts as a translation.
        The irreducible sub-representations of it correspons to the exponential function $\exp(i\lambda x)$ for $\lambda \in \mathbb{R}$,
        but we can't write a function $f(x)$ as a discrete sum of these in general. 
        We can only write it as an integral of these, which is the Fourier transform.
        Note that the regular representation on $\L2(\mathbb{Z}\backslash \mathbb{R})$ decomposes into irreducibles (which gives Fourier series),
        and the reason behind is that the circle group $\mathbb{S}^{1} = \mathbb{Z} \backslash\mathbb{R}$ is compact.
    }
    The irreducible constituents of $\mathscr{H}_{\mathrm{disc}}$ is called \emph{cuspidal automorphic rerpesentations of $\GL_{n}(\mathbb{A}_{F})$},
    up to technical conditions on the center of the group and the archimedean places.
    \item \textbf{Correspondence:} The Langlands correspondence states that there is a one-to-one correspondence between these two different objects that preserves some special invariants.
    More precisely, the equivalence classes of $n$-dimensional Galois representation $\sigma$ of $\Gal(\overline{F}/F)$ (over some field)
    corresponds to an irreducible automorphic representation $\pi = \pi(\sigma)$ of $\GL_{n}(\mathbb{A}_{F})$, and vice versa.
    The irreducibility of $\sigma$ corresponds to cuspidality of $\pi$.

    One of the invariant of Galois representation $\sigma$ is the conjugacy classes of $\sigma(\mathrm{Fr}_{x})$ in $\GL_{n}$, where
    $\mathrm{Fr}_{x}$ is the \emph{Frobenius} element corresponds to a closed point $x \in X$ (for almost all $x$).
    On the automorphic side, there is so-called \emph{Hecke operators} $h_{x}$ for each $x \in X$.
    Then the conjectural Langlands correspondence should give correspondences between these two invariants.
    
\end{enumerate}

A celebrated example of Langland's correspondence is the Shimura-Taniyama-Weil conjecture, which is now a theorem by Andrew Wiles and Richard Taylor (modularity theorem).
It is a special case of Langland's correspondence for $n = 2$ that  relates an elliptic curve and a modular form.

Let $E$ be an elliptic curve over $\mathbb{Q}$, i.e. a smooth projective curve defined over $\mathbb{Q}$ by equation
$$
    y^{2} = x^{3} + ax + b
$$
for $a, b \in \mathbb{Q}$ and $\Delta = 4a^{3} + 27b^{2} \neq 0$ (the discriminant of $E$).
Then for each prime $\ell$ not dividing $\Delta$ (or any $\ell \neq p = \mathrm{char}(F)$ when $F$ is a function field),  the first \'etale cohomology $\mathrm{H}^{1}_{\text{\'et}}(E_{\overline{\mathbb{Q}}}, \mathbb{Q}_{\ell})$ is isomorphic to $\mathbb{Q}_{\ell}^{2}$,
2-dimensional vector space over $\mathbb{Q}_{\ell}$.
Since $E$ is defined over $\mathbb{Q}$, we have a natural action of $\Gal(\overline{\mathbb{Q}}/\mathbb{Q})$ on $E_{\mathbb{Q}} = E \otimes_{\mathbb{Q}}\overline{\mathbb{Q}}$ which
induces an action on $\mathrm{H}^{1}_{\text{\'et}}(E_{\overline{\mathbb{Q}}}, \mathbb{Q}_{\ell})$i, i.e. a 2-dimensional representation $\sigma_{E, \ell}$ over $\mathbb{Q}_{\ell}$
(so the mysterious field where the representation is defined that we didn't defined before is $\mathbb{Q}_{\ell}$ in this case).
For $p \nmid \Delta$, Frobenious conjugacy class $\sigma_{E, \ell}(\mathrm{Fr}_{p})$ has a trace
$$
    \mathrm{Tr}(\sigma_{E, \ell}(\mathrm{Fr}_{p})) = p + 1 - \# E(\mathbb{F}_{p})
$$
and determinant $p$, which completely determines the conjugacy class of $\sigma_{E, \ell}(\mathrm{Fr}_{p})$ for $\GL_{2}$.
Here $\#E(\mathbb{F}_{p})$ is the number of $\mathbb{F}_{p}$-points on $E$.

Assuming Langlands' correspondence, such $\sigma_{E, \ell}$ (or family of $\sigma_{E, \ell})$ for varying $\ell$'s) should corresponds
to an irreducible automorphic representation of $\GL_{2}(\mathbb{A}_{\mathbb{Q}})$.
In case of $\GL_{2}$, an automorphic representation $\pi$ corresponds to certain holomorphic function $f_{\pi}$ called \emph{modular form} on
the complex upper half plane $\mathfrak{H}$ satisfying a functional equation
$$
    f_{\pi} \left(\frac{a\tau + b}{c\tau + d}\right) = (c\tau + d)^{2}f_{\pi}(\tau)
$$
for $(\begin{smallmatrix} a & b \\ c & d\end{smallmatrix}) \in \Gamma_{0}(N) = \{(\begin{smallmatrix} a &b \\ c & d\end{smallmatrix}) \in \mathrm{SL}_{2}(\mathbb{Z}), c\equiv 0\,(\mathrm{mod}\,N)\}$
where $N = N_{E}$ is the conductor of $E$.
Also, $f_{\pi}(\tau)  \to 0$ as $\tau \to \infty$ (in other words, $f_{\pi}$ is a cusp form).
It admits a Fourier expansion
$$
    f_{\pi}(\tau) = \sum_{n\geq 1} a_{n} q^{n}, \quad q = e^{2 \pi i \tau}
$$
and the Langlands correspondence for this case becomes the equality
$$
a_{p} = \mathrm{Tr}(\sigma_{E, \ell}(\mathrm{Fr}_p)) = p + 1 - \$ E(\mathbb{F}_{p}).
$$
for all $p \nmid \ell N$.

Langlands correspondence for $\GL_n$ is now a theorem when $F$ is a function field of some curve, and this is proven by Drinfeld ($n = 2$) and Laurent Lafforgue ($n>2$).\footnote{Lafforgue got a Fields medal for this work.}

\newpage
\section{More on Classical Langlands Correspondence (August 30)}

We are going to give more detailed explanations on the classical Langlands correspondence
and give an explicit example of a correspondence between elliptic curves and modular forms
(Taniyama-Shimura-Weil conjecture, now a theorem by Wiles-Taylor and Breuil-Conrad-Diamond-Taylor).

First, irreducible cuspidal automorphic representations $\pi$ of $\GL_{n}(\mathbb{A}_{F})$
always decomposes into \emph{local} representations as\footnote{this is called Flath's theorem.}
$$
\pi = \bigotimes_{v \in \mathscr{V}}\pi_{v}
$$
(this is also a kind of restricted product).
When $F = \mathbb{F}_{q}(X)$ is a function field, then there is a 1-1 correspondence between
the set of places (completions) $\mathscr{V} = \mathscr{V}_{F}$ and the set of closed points $|X|$ of a curve $X$.
(There are only non-archimedean places.)
If a place $v \in \mathscr{V}$ corresponds to a point $x\in |X|$, and the completion of $F$
by $v$ is isomorphic to $(\mathbb{F}_{q})_{x}((t_{x}))$, where $t_{x}$ is a local coordinate at $x$.
In this case, each $\pi_{v}$ becomes a representation of $\GL_{n}(F_{v})$.
When $F$ is a number field, there exist archimedean places, which has a different nature from nonarchimedean places.
For example, when $F = \mathbb{Q}$, we have $\mathscr{V}_{\mathbb{Q}} = \{p\,:\,p\text{ prime}\} \cup \{\infty\}$, and
$\pi$ decomposes as
$$
\pi = \left(\bigotimes_{p < \infty} \pi_{p}\right) \otimes \pi_{\infty}.
$$
Although $\pi_{p}$'s are representations of $\GL_{2}(\mathbb{Q}_{p})$, $\pi_{\infty}$ is \emph{not} an
irreducible representation of $\GL_{2}(\mathbb{R})$.
It is acually a representation of $(\mathfrak{gl}_{2}(\mathbb{R}), \mathrm{O}_{2}(\mathbb{R}))$ - in other words,
it is a representation of Lie algebra $\mathfrak{gl}_{2}(\mathbb{R)}$ and a (maximal compact subgroup) $\mathrm{O}_{2}(\mathbb{R})$
with compatibility condition on their actions.

Recall that the classical Langlands correspondence for $\GL_{n}$ is a correspondence between (equivalence classes of)
$n$-dimensional irreducible ($\ell$-adic) Galois representations $\sigma$ of $\Gal(\overline{F}/F)$
and (equivalence classes of ) cuspidal automorphic representations of $\GL_{n}(\mathbb{A}_{F})$.
It is not an arbitrary 1-1 correspondence - certain \emph{invariants} should match.
The Galois-side invariant is semisimple Frobenius conjugacy classes in $\GL_{n}(\overline{\mathbb{Q}_{\ell}})$: it is
$$
    \{\sigma(\mathrm{Fr}_v), v\in \mathscr{V}\backslash S_{\sigma}\}
$$
where $S_{\sigma}$ is a finite subset of $\mathscr{V}$.
Note that the topology matters for Galois side - we have Krull topology (profinite topology) on $\Gal(\overline{F}/F)$
and we only consider continuous representations.
On the automorphic side, there are certain semisimple conjugacy classes in $\GL_{n}(\mathbb{C})$, which we call
Hecke conjugacy classes.
These record eigenvalues of the (spherical) Hecke algebra associated to each $v\in \mathscr{V}$.
We denote it as
$$
    \{\pi(h_{v}),v\in \mathscr{V}\backslash S_{\pi}\}
$$
where $S_{\pi}$ is a finite subset of $\mathscr{V}$.
Note that we can identify $\overline{\mathbb{Q}_{\ell}}$ and $\mathbb{C}$ since they have
the same transcendence degree over $\overline{\mathbb{Q}}$, and the correspondence is independent of
the choice of identification.
Also, the invariants uniquely determine representation themselves.


Now, we will introduce an explicit correspondence between a certain elliptic curve and a modular form.
Let $E$ be an elliptic curve over $\mathbb{Q}$ defined by
$$
y^{2} + y = x^{3} - x^{2}.
$$
Then the only bad prime of reduction is 11, and the conductor of the elliptic curve is also 11.
We can count the number of $\mathbb{F}_{p}$-points on the curve.
For example, when $p = 5$, there are exactly 5 points: $\{(0, 0), (1, 0), (0, 4), (1, 4), \infty\}$.
Now, consider the following function defined as an infinite product:
$$
    f(\tau) = q\prod_{n=1}^{\infty} (1 - q^{n})^{2}(1-q^{11n})^{2}, \quad q = e^{2\pi i \tau}.
$$
It turns out that this is a modular frm of weight 2 and level 11. Its expansion is
$$
    f(\tau) = q - 2q^{2} - q^{3} + 2q^{r} + q^{5} + 2q^{6} - 2q^{7} + \cdots
$$
and the 5th coefficient of $f$ is $a_{5}(f) = 1$, which equals to $a_{5}(E) = 5 + 1 - 5 = 1$.
In fact, this is the modular form corresponds to $E$, and $a_{p}(E) = a_{p}(f)$ holds for all $p\neq 11$.

As an aside, Langlands correspondence for $\GL_{1}$ has long been known as \emph{abelian class field theory}.
Since $\GL_{1}$ is an abelian group, 1-dimensional Galois represention should factor through $\Gal(F^{\mathrm{ab}}/F)$
and the structure of the latter group is well known for some cases.
For example, we have a Kronecker-Weber theorem when $\mathbb{Q}$, which states that $\mathbb{Q}^{\mathrm{ab}} = \cup_{n\geq 1}\mathbb{Q}(\zeta_{n})$.


Now we will explain Frobenius automorphisms and conjugacy classes in detail.
The Galois group of finite extension of finite fields has a simple structure.
For the extension $\mathbb{F}_{q^{n}} /\mathbb{F}_{q}$, its Galois group is just
a cyclic group of order $n$ generated by the Frobenius automorphism $x \mapsto x^{q}$.
Now let $K/F$ be a finite extension of number fields, and $\mathcal{O}_{F} \subset \mathcal{O}_{K}$ be the ring of integers.
These are Dedekind domain: any ideal admits a prime ideal factorization.
For a prime ideal $v \subset \mathcal{O}_{F}$, regarding it as an ideal $\mathcal{O}_{K}$, it splits as a
product of prime ideals in $\mathcal{O}_{K}$ as $v = w_{1}\cdots w_{g}$.
Then $\mathcal{O}_{F}/v \subset \mathcal{O}_{K} / w_{j}$ is a finite extension of finite fields, so is cyclic.
Although we can't directly link $\Gal(K/F)$ with $\Gal((\mathcal{O}_{K}/w_{j})/(\mathcal{O}_{F}/v))$, there exists
a \emph{decomposition group} $D_{w_{j}} \subset \Gal(K/F)$ defined as
$$
    D_{w_{j}}:=\{g \in \Gal(K/F)\,:\, gw_{j} = w_{j}\} \xrightarrow{\alpha_{w_{j}}} \Gal((\mathcal{O}_{K}/w_{j})/(\mathcal{O}_{F}/v))
$$
where $\alpha_{w_j}$ is surjective.
We also define \emph{inertia subgroup} $I_{w_{j}}$ as $\ker \alpha_{w_{j}}$, so that $D_{w_{1}}/I_{w_{1}} \simeq \mathbb{Z}/n\mathbb{Z}$
for some $n$.
Now, when $I_{w_{j}} = 1$, we have $D_{w_{j}} \simeq \mathbb{Z}/n\mathbb{Z}$ and we can define a Frobenius
conjugacy class in $\GL_{n}(\overline{\mathbb{Q}_{\ell}})$ by composing the isomorphism with $\sigma|_{D_{w_{j}}}$.
It is known that $I_{w_{j}} = 1$ for all but finitely many $v$ (we call such $v$ \emph{unramified}),
and since different choices of $w_{j}$ gives conjugated decomposition groups,
the Frobenius conjugacy class $\sigma(\mathrm{Fr}_{w_{j}})$ does not depend on the choice of $w_{j}$ and only on $v$. 
\newpage
\section{Classical Langlands correspondence over function fields (September 1)}


We are going to explain classical Langlands correspondence over function fields in (more) detail.
Let $X$ be a smooth, geometrically irreducible, projective curve over $\mathbb{F}_{q}$ and $F = \mathbb{F}_{q}(X)$ be a function field.
Let $|X|$ be a set of closed points of $X$, which has a 1-1 correspondence with $\mathscr{V}$ - the set of
places (completions) of $F$.
Recall that the completion $F_{x}$ at $x \in |X|$ is isomorphic to $(\mathbb{F}_{q})_{x}((t_{x}))$, where 
$(\mathbb{F}_{q})_{x}$ residue field at $x$ and $t_{x}$ is a rational functino on $X$ with order 1 zero at $x$
(In other words, it is a generator of maximal ideal $\mathfrak{m}_{x}$ corresponds to $x$).
Then we have a ring of integer $\mathcal{O}_{x} \subset F_{x}$ isomorphic to the ring of
formal power series $(\mathbb{F}_{q})_{x}[[t_{x}]]$.
We also defined the ad\'ele ring $\mathbb{A}_{F}$ for $F$.

Now we define the \emph{Weil group} $W(\overline{F}/F)$ as follows.
Let $\overline{F}$ be a (separable) algebraic closure of $F$, then we have the action 
of $\Gal(\overline{F}/F)$ on the subfield $\overline{\mathbb{F}_{q}}$ (the field of constants) that fixes $\mathbb{F}_{q}$.
Then we have a surjective map 
$$\Gal(\overline{F}/F) \xrightarrowdbl{\mathrm{res}} \Gal(\overline{\mathbb{F}_{q}}/\mathbb{F}_{q})$$
and the latter group is an inverse limit of Galois groups of finite extensions of $\mathbb{F}_{q}$, so
$$
\Gal(\overline{\mathbb{F}_{q}}/\mathbb{F}_{q}) \simeq \varprojlim \Gal(\mathbb{F}_{q^{n}} / \mathbb{F}_{q}) \simeq \varprojlim \mathbb{Z}/n\mathbb{Z} =: \widehat{\mathbb{Z}},
$$
which is the profinite completion of $\mathbb{Z}$.
It is topologically generated by Frobenius automorphism $\mathrm{Fr}$, and it has a subgroup isomorphic to $\mathbb{Z}$ generated (not topologically, but just algebraically) by $\mathrm{Fr}$.
Then we define the \emph{Weil group} $W(\overline{F}/F)$ as an inverse image of $\mathbb{Z} \simeq \langle \mathrm{Fr} \rangle \subset \Gal(\overline{\mathbb{F}_{q}} / \mathbb{F}_{q})$
of restriction map, which is a subgroup of $\Gal(\overline{F}/F)$.
For Galois side of Langlands correspondence over function field, we are going to consider irreducible representations of
$W(\overline{F}/F)$ instead of $\Gal(\overline{F}/F)$.
More precisely, we consider the (equivalence classes of) irreducible $n$-dimensional $\ell$-adic representations of $W(\overline{F}/F)$,
$$
\sigma: W(\overline{F}/F) \to \GL_{n}(\overline{\mathbb{Q}_{\ell}})
$$
such that 
\begin{enumerate}
    \item Image of $\sigma$ in $\GL_{n}(\overline{\mathbb{Q}_{\ell}})$ is in $\GL_{n}(E)$ for some finite extension $E/\mathbb{Q}_{\ell}$.
    \item $\sigma$ is continuous where $W(\overline{F}/F)$ is given Krull topology (profinite toppology) and $\GL_{n}(E)$ is given subspace topology of $M_{n}(E)$.\footnote{
        This explains somehow why we are considering $\ell$-adic representations instead of complex representations.
        As a toy example, consider continuous 1-dimensional complex representations of $(\mathbb{Z}_{\ell}, +)$, i.e. an additive character $\sigma: \mathbb{Z}_{\ell} \to \GL_{1}(\mathbb{C}) = \mathbb{C}^{\times}$.
        Then it should factor through $\mathbb{Z}_{\ell} / \ell^{n}\mathbb{Z}_{\ell}$ for some $n$, so that the image is always finite.
        However, if we consider $\ell$-adic characters $\sigma:\mathbb{Z}_{\ell} \to \mathbb{Q}_{\ell}^{\times}$, then there are non-trivial characters with infinite image, e.g. $x \mapsto \exp_{\ell}(\ell x)$ 
        where $\exp_{ell}$ is an $\ell$-adic exponential function.
    }
    \item $\sigma$ is unramified for all but finitely many $x \in |X|$.
    Note that the unramifiedness is defined using decomposition group and inertia group as before.
\end{enumerate}

On the automorphic side, we wiil explain cuspidality and unramifiedness in more detail.
The space of cusp forms $\mathrm{L}^{2}_{\cusp}(\GL_{n}(F) \backslash \GL_{n}(\mathbb{A}_{F}), \chi)$\footnote{Here $\chi$ is a continuous unitary character on center $Z(\mathbb{A}_{F})$ trivial on $Z(F)$,
and $\mathrm{L}^{2}(\GL_{n}(F)\backslash\GL_{n}(\mathbb{A}_{F}), \chi)$ is a space of functions where the center acts as the character $\chi$.
}
are functions satisfying the following vanishing condition: for $0 < n_{1}, n_{2} < n$ with $n = n_{1} + n_{2}$,
we have
$$
\int_{N_{n_{1}, n_{2}}(F) \backslash N_{n_{1}, n_{2}}(\mathbb{A}_{F})} f(ng)dn = 0
$$
for all $g\in \GL_{n}(\mathbb{A}_{F})$, where $N_{n_{1}, n_{2}} < \GL_{n}$ is the unipotent group of matrices of the form
$$
\begin{pmatrix}
    I_{n_{1}} & * \\ \mathbf{0} & I_{n_2}
\end{pmatrix}
$$
Note that non-example of cuspidal represenetation is Eisenstein series representation, which is obtained from
two representations $\pi_{1}, \pi_{2}$ of $\GL_{n_{1}}(\mathbb{A}_{F})$ and $\GL_{n_2}(\mathbb{A}_F)$ respectively, by inflation and (parabolic induction).
Then it is a theorem (from Flath) that any irreducible cuspidal represenetations of $\GL_{n}(\mathbb{A}_F)$ decomposes as restricted product
of local representations,
$$
\pi \simeq \bigotimes_{x\in |X|} \pi_{x}
$$
where each $\pi_{x}$ are irreducible representation of $\GL_{n}(F_x)$.
In this case, for all but finitely many $x$, $\GL_{n}(\mathcal{O}_{x})$-fixed subspace
$\pi^{\GL_{n}(\mathcal{O}_x)}$ is non-trivial and one-dimensional.
We call that $\pi$ is \emph{unramified at $x$} for such $x$.
For $x$ where $pi$ is unramified, we have a represenetation of \emph{spherical Hecke algebra $\mathcal{H}_{x}$}, which 
is a sub-algebra of compactly supported functions on $\GL_{n}(F_{x})$ that are $\GL_{n}(\mathcal{O}_{x})$-biinvariant.
Then $\mathcal{H}_{x}$ is a convolution algebra which is commutative and 
isomorphic to $\mathbb{C}[x_{1}^{\pm}, \dots, x_{n}^{\pm}]^{S_{n}}$
and corresponds to semisimple conjugacy classes in $\GL_{n}(\mathbb{C})$, which we will denote $\pi(h_{x})$.

Also, as in the case of Galois side, we impose some conditions on the automorphic side.
We will only consider automorphic representations of $\GL_{n}(\mathbb{A}_{F})$ with some finiteness conditions, i.e.
for any compact subgroup $K$ of $\GL_{n}(\mathbb{A}_F)$, the translates of any $f\in \pi$ span a
finite dimensional vector space.

Then the Langlands correspondence becomes as follows.
It is a 1-1 correspondence between the irreducible $\ell$-adic $n$-dimensional representations of Weil group $W(\overline{F}/F)$
(with some conditions) and irreducible cupsidal automorphic representations of $\GL_{n}(\mathbb{A}_{F})$ (with some conditions).
The invariants, Frobenius conjugacy classes $\{\sigma(\mathrm{Fr}_{x})\}$ on the Galois side, matches with
the Hecke conjugacy classes $\{\pi(h_{x})\}$, for all $x \not \in S_{\sigma} \cup S_{\pi}$.
Here $S_{\sigma}$ (resp. $S_{\pi}$) is the set of unramified places for $\sigma$ (resp. $\pi$), and
we actually have $S_{\sigma} = S_{\pi}$ for corresponding $\sigma - \pi$ pairs.
\newpage
\section{More on Hecke algebra and Langlands correspondence for general reductive groups (September 6)}

We said that Langlands correspondence for $\GL_{n}$ gives a correspondence between invariants,
which are the Frobenius conjugacy classes (on Galois side) and the Hecke conjugacy classes (on automorphic side).
We are going to explain about Hecke conjugacy classes more in detail.

Let $F = \mathbb{F}_{q}(X)$ be a function field for a curve over finite field.
Let $\pi$ be an automorphic representation of $\GL_n(\mathbb{A}_F)$.
It decomposes as a restricted product of local representations of $\GL_{n}(F_x)$ as $\pi = \otimes'_{x\in |X|}\pi_x$.
Then there exists a finite set of (closed) points $S_\pi \subset |X|$ such that
$\pi_{x}^{\GL_{n}(\mathcal{O}_x)}\neq 0$, i.e. there exists $\GL_{n}(\mathcal{O}_{x})$-fixed vector in $\pi_x$.
We define Hecke algebra $\mathcal{H}_x$ as a convolution algebra on the set of 
compactly supported $\GL_{n}(\mathcal{O}_{x})$-bi-invariant functions with Haar measure on $\GL_n(F_x)$
normalized by $\mu(\GL_n(\mathcal{O}_x))=1$.
Then the Hecke algebra is actually commutative, and using this we can show that $\pi_x^{\GL_n(\mathcal{O}_x)}$ is actually 1-dimensional,
i.e. there exists a unique vector (up to scaling) $v_x \in \pi_x$ fixed by $\GL_n(\mathcal{O}_x)$.
The restricted product $\otimes_{x\in |X|}' \pi_x$ of local representations are defined as a span of vectors $\otimes_x w_x$
where $w_x\in \pi_x$ and $w_x = v_x$ for all but finitely many $x$.
In this case, the group $\GL_n(\mathbb{A}_F) = \otimes_x' \GL_n(F_x)$ acts on the space componenti-wise
$g.\otimes w_x := \otimes_x (g_x.w_x)$
and the previous argument this actually gives an action on the space $\otimes_{x}'\pi_x$.

Now for given local representation $\pi_x$, we can attach a representation of the Hecke algebra $\mathcal{H}_x$
where the representation space is $\pi_x^{\GL_n(\mathcal{O}_x)}$ action is given by (we use the same notation $\pi_x$ for the representation of $\mathcal{H}_x$)
$$
f \mapsto \pi_x(f): v \mapsto \int_{\GL_n(F_x)} f(g) \pi_x(g)v dg.
$$
The integral is well-defined since $f$ is compactly supported and $\GL_n(\mathcal{O}_x)$ preserves $\pi^{\GL_n(\mathcal{O}_x)}$.
This gives a functor from the category of representations of $\GL_n(F_x)$ and the category of representations of $\mathcal{H}_x$.
In fact, this sets up bijection between irreducible unramified representations of $\GL_n(F_x)$
and irreducible representations of $\mathcal{H}_x$.
Since $\mathcal{H}_x$ is commutative, irreducible representations of $\mathcal{H}_x$ are just characters of $\mathcal{H}_x$,
which we denote it as $\chi_x: \mathcal{H}_x \to \mathbb{C}$.

We can describe the structure of $\mathcal{H}_x$ more precisely as follows, which is a special case of so-called \emph{Satake isomorphism}.

\begin{theorem}
    $$
    \mathcal{H}_x \simeq \mathbb{C}[x_1^{\pm}, \dots, x_n^{\pm}]^{S_n}
    $$
    where the RHS is a space of symmetric Laurent polynonmials in $n$-variables.
\end{theorem}
Here's a sketch of proof.
First, the double coset space $\GL_n(\mathcal{O}_x) \backslash \GL_n(F_x) / \GL_n(\mathcal{O}_x)$ can be identified with $\mathbb{Z}^{n}/S_n$ as follows.
We have a map 
$$
\mathbb{Z}^{n} \to \GL_n(\mathcal{O}_x) \backslash \GL_n(F_x) / \GL_n(\mathcal{O}_x)
$$
that maps $(\lambda_1, \dots, \lambda_n) \in \mathbb{Z}^{n}$ to the double coset of the diagonal matrix
$$
\begin{pmatrix}
    t_{x}^{\lambda_1} & & & \\
    & t_{x}^{\lambda_@} & & \\
    & & \ddots & \\
    & & & t_{x}^{\lambda_n}
\end{pmatrix}
$$
where $t_x$ is a choosen uniformizer of $F_x$, so that $F_x \simeq (\mathbb{F}_q)_x((t_x))$ 
and $\mathcal{O}_x \simeq (\mathbb{F}_{q})_x[[t_x]]$.
Such a map is well-defined in the sense tha the double coset is independent of the choice of uniformizer $t_x$ - any other choice $t_x'$ satisfies $t_x'/t_x \in \mathcal{O}_x$.
The map is surjective, and it factors through $\mathbb{Z}^{n}/S_n$ with permutation on $\mathbb{Z}^n$ since
$(\lambda_1, \dots, \lambda_n)$ and $(\lambda_1', \dots, \lambda_n')$ with $\tau\lambda = \lambda'$ maps to the diagonal matrices
that are conjugate to each other by permutation matrix (corresponds to $\tau\in S_n$).

Using the identification, we can get the Satake isomorphism as follows.
Any $f \in \mathcal{H}_{x}$ as a form of $\sum_{\lambda} a_{\lambda} c_{\lambda}$
where $c_{\lambda}$ is a characteristic function on a double coset corresponds to an unordered set $\lambda = (\lambda_1, \dots, \lambda_n)$.
This sum is a finite sum since $f$ is compactly supported.
And the function $f$ corresponds to a symmetrized Laurent polynomial
$\sum_{\tau\in S_n} \tau(\sum a_{\lambda}x_1^{\lambda_1}\cdots x_{n}^{\lambda_n})$.

As we said before, Langlands correspondence for $\GL_n$ over a global function field is now a theorem.
\begin{theorem}[Deligne, Lafforgue]
   Let $F$ be a function field of a curve over a finite field.
   There is such a bijection between $n$-dimensional continuous irreducible representations of the Weil group $W(\overline{F}/F)$
   (with some technical conditions mentioned before) and irreducible cuspidal automorphic representations of $\GL_n(\mathbb{A}_F)$.
   This also gives a bijection between the Frobenius conjugacy classes and the Hecke conjugacy classes, and also these conjugacy classes
   are actually in $\GL_n(\overline{\mathbb{Q}})$.
\end{theorem}
$n=2$ case is proven by Delign, and $n > 2$ is by L. Lafforgue.
The difference between two cases are on the existence of \emph{nice} moduli space.
Lafforgue invented objects called \emph{Shutuka} and use moduli space of them for the cases $n >2$.


How can we state the Langlands correspondence for general reductive group $G$ (over function fields)?
First, the concepts we defined (Galois representations, automorphic representations, Hecke algebra, Frobenius conjugacy classes, ...)
generalizes to general reductive groups.
For example, Hecke algebra $\mathcal{H}_{x}$ is a convolution algebra of compactly supported functions on $G(F_x)$
which is bi-invariant under $G(\mathcal{O}_{x})$.
Then it is a commutative algebra for $x \in |X|$ with $\pi_x^{G(\mathcal{O}_x)}\neq 0$ (unramified), and
the invariant subspace $\pi_{x}^{G(\mathcal{O}_{x})}$ is 1-dimensional.
Also, we have a Satake isomorphism for $G$.

But if $\chi_x : \mathcal{H}_x \to \mathbb{C}$ is a character of $\mathcal{H}_{x}$, then where the corresponding
conjugacy class would live in?
For general reductive group $G$, Hecke conjugacy classes does not live in $G(\mathbb{C})$, but in different group called \emph{dual group} of $G$.
To define the notion of dual group, we have to define a \emph{root datum} first.

Let $T$ be a maximal torus of $G$, i.e. maximal commutative subgroup of $G$.
For $\GL_n$, it is a set of diagonal matrices.
For now, we will regard $G$ and $T$ as a group over $\mathbb{C}$.
Then we define the latteices of characters and cocharacters of $T$ as
\begin{align*}
X^{*}(T) = \mathrm{Hom}(T, \mathbb{G}_{m}) \\
X_{*}(T) = \mathrm{Hom}(\mathbb{G}_{m}, T)
\end{align*}
which are free abelian groups of finite rank.
We have a pairing $X^{*}(T)\times X_{*}(T) \to \mathbb{Z}$ defined by composition (note that all the morphisms $\mathbb{G}_{m} \to \mathbb{G}_{m}$
has a form of $x \mapsto x^{n}$).
We also have roots $\Delta \subset X^{*}(T)$ and coroots $\Delta ^{\vee} \subset X_{*}(T)$
which are nonzero eigenvalues of adjoint action of $T$ (and their duals).
Then we can associate a quadruple $(X^{*}(T), X_{*}(T), \Delta, \Delta^{\vee})$, a \emph{root datum} of $G$,
and it determines a group $G$ upto isomorphism when $G$ is split (i.e. admits a split maximal torus).
By simply flipping a root datum, we get another root datum
$$
(X_*(T), X^{*}(T), \Delta^{\vee}, \Delta)
$$
called dual root datum, and the group determined by this new root datum is called the \emph{dual} group of $G$, denoted by $\widehat{G}$.

Now let's get back to the Hecke algebra.
In case of $\GL_{n}$, we have a map $\mathbb{Z}^{n} \to \GL_2(\mathcal{O}_{x}) \backslash \GL_2(F_{x}) / \GL_2(\mathcal{O}_x)$ 
that induces an isomorphism 
$$
\mathbb{Z}^{n}/S_{n} \to \GL_2(\mathcal{O}_x) \backslash \GL_2(F_x) / \GL_2(\mathcal{O}_x).
$$
For general reductive group $G$, we have a map $X_{*}(T) \to G(\mathcal{O}_x) \backslash G(F_x) / G(\mathcal{O}_x)$ defined as an evaluation of character at uniformizer $t_{x}$.
Then this map factors through the quotient of $X_{*}(T)$ by the \emph{Weyl group} $W = W(G, T)$, the symmetry group of a root datum.
This induces an isomorphism 
$$
X_{*}(T) / W \simeq G(\mathcal{O}_{x}) \backslash G(F_x) / G(\mathcal{O}_x)
$$
which gives $\mathcal{H}_{x} \simeq \mathbb{C}[X_{*}(T)/W]$.
This saids that the character of $\mathcal{H}_x$ corresponds to an element in $X_{*}(T)/W$.
In this case, which kind of a conjugacy would corresponds to the element?
In fact, we have a canonical isomorphism $\mathbb{C}[X^{*}(T)/W] \simeq \mathbb{C}[T]^{W}$.
which gives a correspondence between $X^{*}(T)/W$ and semisimple conjugacy class in $G$.
However, we have $X_{*}(T)/W$ instead, and since $X_{*}(T)$ is a character group of the dual group $\widehat{G}$, 
we can concludes that $X_{*}(T)/W$ corresponds to semisimple conjugacy classes in $\widehat{G}$.

Based on this observation, the Langlands correspondence for a reductive group $G$ (over a function field) 
is a correspondence between
\begin{align*}
    \boxed{
        \text{irreducible representation } \sigma:W(\overline{F}/F) \to \widehat{G}(\overline{\mathbb{Q}_{\ell}})
    }
\end{align*}
and
\begin{align*}
    \boxed{
        \text{irreducible cuspidal automorphic representation of }G(\mathbb{A}_F)
    }
\end{align*}
where the invariants match: the Frobenius conjugacy classes 
$$
[\sigma(\mathrm{Fr}_{x})] \subset \widehat{G}(\overline{\mathbb{Q}_{\ell}}), \quad x \in |X| \backslash S_{\sigma}
$$
and the Hecke conjugacy classes (corresponds to characters of Hecke algebra $\mathcal{H}_x$)
$$
\pi(h_x) \subset \widehat{G}(\mathbb{C}),\quad x \in |X| \backslash S_\pi
$$
with a choice of identification $\iota: \mathbb{C} \simeq \overline{\mathbb{Q}_{\ell}}$, where $S_\sigma = S_\pi$ is a finite subset of $|X|$.
\newpage
\section{Satake isomorphism (September 8)}

% \subsection{Preliminaries on affine group schemes}
% Before we rigorously state the Satake isomorphism, we'll go through some basics on affine group schemes.


\subsection{Satake isomorphism}
Recall that for given cocharacter $\lambda \in X_{*}(T)$, we can associate an element in $T(F_x)$ 
by evaluating $\lambda$ at chosen uniformizer $t_x \in F_x$.
Then it defines a well-defined double coset in $G(\mathcal{O}_x) \backslash G(F_x) / G(\mathcal{O}_x)$
since the choice of uniformizer is only differ by $\mathcal{O}_x^{\times}$.
Now we can state the Satake isomorphism for genral (split) reductive groups.
\begin{theorem}
We have an isomorphism of $\mathbb{C}$-algebra
$$
\mathcal{H}_{x} \simeq \mathbb{C}[X_{*}(T)]^{W} \simeq \mathbb{C}[X^{*}(\widehat{T})]^{W}.
$$
\end{theorem}
As a consequence, Hecke algebras are commutative.
\begin{proof}
Satake provided explicit isomorphism from $\mathcal{H}_{x}$ to $\mathbb{C}[X_{*}(T)]^{W}$, which is
$$
f \mapsto \sum_{\lambda \in X_{*}(T)} \left(q_{x}^{\langle \rho, \lambda\rangle} \int_{N(F_x)}f(u \lambda(t_x))du \right) \lambda
$$
Here $\rho = (1/2)\sum_{\alpha \in \Delta^{+}}\alpha$ is the half-sum of the positive roots
and $du$ is the Haar measure on $N(F_x)$ normalized as $du(N(\mathcal{O}_x)) = 1$.
Since $f \in \mathcal{H}_x$ is compactly supported, the image under Satake's map is actually a finite sum.
He showed that this is an injective algebra homomorphism whose image is $\mathbb{C}[X_*(T)]^W$.
\end{proof}
Now, observe that we have following isomorphisms
$$
\mathbb{C}[X_{*}(T)]^{W} \simeq \mathbb{C}[T]^{W} \simeq \mathbb{C}[G]^{G}
$$
where the last space is the space of class functions on $G$, i.e. functions on $G$ that are invariant under conjugation.
The second isomorphism comes from Chevalley's restriction theorem.
Hence the characters of $\mathcal{H}_x$ correspond to the characters of $\mathbb{C}[G]^{G}$, which again correspond to
the semisimple conjugacy classes in $G$.
This is the way how we associate ``Hecke conjugacy class $\pi(h_x)$''\footnote{Usually, it is called as \emph{Satake parameters}.} to an automorphic representation $\pi$.

\subsection{Why Weil group?}
On the Galois side, why do we care about representations of Weil groups, not the whole Galois group?
The reason can be found from the $n=1$ case, i.e. Langlands correspondence for $\GL_1 = \mathbb{G}_m$, which is also
called \emph{Abelian Class Field Theory}.

Let $\rho : \Gal(\overline{F}/F) \to \GL_1(\overline{\mathbb{Q}}_\ell) = \overline{\mathbb{Q}_{\ell}}^{\times}$
be an irreducible representation.
Since $\GL_1$ is abelian, the map factors through the abelianization of $\Gal(\overline{F}/F)$, which equals
$\Gal(F^{\mathrm{ab}}/F)$, the Galois group of maximal abelian extension.
On the automorphic side, (cuspidal) automorphic representations of $\GL_1$ are essentially the characters of $\GL_1(F) \backslash \GL_1(\mathbb{A}_F) = F^{\times} \backslash \mathbb{A}_F^{\times}$.
Then the Langlands correspondence is given through Artin reciprocity map
$$
\theta: F^{\times} \backslash \mathbb{A}_F^{\times} \to \Gal(F^{\mathrm{ab}} / F).
$$
By the way, this map is injective but not surjective, and the image is exactly the ablianization of Weil group, $W(F^{\mathrm{ab}}/F)$.
Hence, generalizing this idea for $n > 1$ makes us to consider representations of Weil group instead of the
full absolute Galois group.
\newpage
\section{Glimpse on analytic langlands (September 13)}

Now, our goal is to formulate Langlands correspondence in geometric terms, so that we could
find their analogues when a curve $X$ is over $\mathbb{C}$, rather than a finite field $\mathbb{F}_q$.
For simplicity, we'll assume that everything is unramified, i.e. the ``Galois representations'' and ``automorphic representations''
are unramified everywhere ($S_\sigma = S_\pi = \emptyset$), whatever it means.

On the Galois side, our slogan is the following: the geometric object corresponds to 
Galois group is essentially the (\'etale) \emph{fundamental group} of $X$.
In this case, extensions of a function field corresponds to (\'etale) coverings $Y \to X$.
For example, if we have a curve $X$ (over a base field $k = \mathbb{C}$ or $\mathbb{F}_q$) and a covering $Y \to X$,
then we get a field extension $k(X) \hookrightarrow k(Y)$.
Then the Galois group $\Gal(k(Y) / k(X))$ is essentially the group of deck transformations of this covering.
Hence the Galois group of maximal unramified extension of $F = k(X)$, $\Gal(F^{\mathrm{un}}/F)$,
is the group of deck transformations of maximal unramified cover $\widetilde{X}$, which is a fundamental group $\pi_1(X)$ of $X$.

% Recall that, for topological spaces and their (universal) coverings, fundamental group is defined with a base point

Now, let $X$ be a smooth projective connected algebraic curve over $\mathbb{C}$, or in other words, 
a compact Riemann surface.
Then our analogy for the Galois representations becomes
\begin{align*}
    \boxed{
        \text{Equivalence class of } \sigma: \pi_1(X) \to \widehat{G}(\overline{\mathbb{Q}_{\ell}})
    }
\end{align*}
Now, such a collection has a correspondence with flat connections:
\begin{align*}
    \boxed{
        (E, \nabla): C^{\infty} \text{ principal } \widehat{G} \text{-bundles with } C^{\infty}\text{ flat connection}
    }
\end{align*}
where $C^{\infty}$ principal $G$-bundle is a principal $G$-bundle where transition maps are smooth.
By decomposing it into holomorphic and anti-holomorphic part ($\nabla = \nabla^{(1, 0)} + \nabla^{(0, 1)}$), one can see that this also has a correspondence with
\begin{align*}
    \boxed{
        (E^{\mathrm{hol}}, \nabla^{(1, 0)}): \text{ flat holomorphic $\widehat{G}$-bundle with a holomorphic connection $\nabla^{(1, 0)}$}
    }
\end{align*}
Note that, on a curve, any such connection becomes automatically flat.

For example, let $X = \mathbb{C}$ and consider trivial $\GL_1 = \mathbb{G}_{m}$-bundle on $X$, $\mathbb{C}^{\times} \times \mathbb{C}$.
Let $\nabla^{(1, 0)} = \frac{\partial}{\partial z} + \frac{\lambda}{z}$ be a connection on $X$.
Then the solution of $\nabla^{(1, 0)}\phi = 0$ is $\phi(z) = z^{-\lambda} = e^{-\lambda \log z}$, 
and this gives a monodromy action where a generator of $\pi(\mathbb{C}^{\times}, 1) \simeq \mathbb{Z}$ acts as multiplication by $e^{-2\pi i \lambda}$.

Now let's move on to the automorphic side.
Recall that automorphic representation $\pi$ can be thought as a space of compactly supported
smooth (i.e. locally constant) functions on $G(F)\backslash G(\mathbb{A}_F)$.
If it is unramified everywhere, then its $G(\mathcal{O}_F) = \prod_{v}' G(\mathcal{O}_{F_v})$-invariant subspace 
$\pi^{G(\mathcal{O}_F)}$ is a subspace of functions on a double coset 
$$
G(F) \backslash G(\mathbb{A}_F) / G(\mathcal{O}_F).
$$
Upshot is, the above double coset space is on bijection with the set of isomorphism lasses of principal (holomorphic or algebraic) $G$-bundles on $X$.
For example, in case of $G = \mathrm{GL}_1$, the double coset space is
$$
F^{\times} \backslash \mathbb{A}_F^{\times} / \sideset{}{'}\prod_{x} \mathcal{O}_{F_x}^{\times} \simeq F^{\times} \backslash \sideset{}{'}\prod_{x}(F_{x}^{\times} / \mathcal{O}_{F_{x}}^{\times}) \simeq F^{\times} \backslash \sideset{}{'}\prod_{v}\mathbb{Z}.
$$
On the other hand, $\GL_1$-bundles on $X$ are the same as line bundles on $X$, 
and this again corresponds to (Weil) divisors on $X$.
Divisor is just a formal $\mathbb{Z}$ combination of closed points on $X$.
To divisors are equivalent when the difference is principal divisor, i.e. divisor of a form
$$
\mathrm{div}(f):= \sum_{x \in |X|} \mathrm{ord}_{x}(f)\cdot [x]
$$
and the set of equivalence classes of divisors is in bijection with the divisor class group, which is
$$
\mathrm{Cl}(X) = \mathrm{Div}(X) / \mathrm{PDiv}(X)
$$
(quotient of group of divisors by group of principal divisors).
One can find natural bijection between $\mathrm{Cl}(X)$ and the above double coset space.

\newpage
\section{On vector bundles and flat connections (September 15)}


Let $X$ be a smooth real manifold.
A (complex) rank $n$ vector bundle $\mathcal{P}$ on $X$ is a manifold with projection map $\mathcal{P} \to X$ such that
there exists a covering $\{ U_{\alpha}\}$ of $X$ with local trivializations $u_{\alpha}: \mathcal{P}|_{U_\alpha} \to U_\alpha \times \mathbb{C}^{n}$.
On the intersections $U_\alpha \cap U_\beta$, we have two trivializations
$u_\alpha|_{U_\alpha \cap U_\beta}$ and $u_\beta|_{U_\alpha \cap U_\beta}$, and this gives $\GL_n(\mathbb{C})$-valued functions
$g_{\alpha\beta}$ comes from $u_{\beta}u_{\alpha}^{-1}: (U_\alpha \cap U_\beta) \times \mathbb{C}^n \to (U_\alpha\cap U_\beta)\times\mathbb{C}^n$.
We assume that such a transition function is smooth. Also, on triple intersections $U_\alpha \cap U_\beta \cap U_\gamma$, 
we assume that transition functions are compatible, in the sense that they satisfy cocycle conditions $g_{\alpha\gamma} = g_{\beta\gamma}g_{\alpha \beta}$.
Hence giving a vector bundle is equivalent to giving a covering $\{U_\alpha\}$ and smooth transition functions $\{g_{\alpha\beta}\}$
satisfying cocycle conditions.

We call that a vector bundle $\mathcal{P}$ is \emph{flat} if $g_{\alpha\beta}:U_{\alpha}\cap U_{\beta}\to \GL_n(\mathbb{C})$is a \emph{constant} function.
It means that we have an identification of all nearby fibers $\mathcal{P}_{x} \simeq \mathcal{P}_{x'}$ for all $x, x'\in U_\alpha$.
Equivalent way to say this is using a \emph{flat connection}.
Connection is a way of differentiating sections of vector bundles.
If a vector bundle is trivial, then this corresponds to an ordinary differentiation of a vector-valued function on $X$.
Hence we can do the same thing for general vector bundles locally, but this may depends on the choice of local trivializations.

\emph{Connection} is a map between between a sheaf of smooth vector fields on $X$ and a sheaf of smooth sections
of $\mathrm{End}(\mathcal{P})$, satisfying some linearity and Leibniz rules.
More precisely, it is a map
$$
\nabla: \mathcal{T}\to \mathcal{E}nd_{\mathbb{C}}(\mathcal{P})
$$
with $\xi \mapsto \nabla_{\xi}$, such that 
\begin{enumerate}
    \item ($\mathcal{O}_X$-linear) $\nabla_{\xi + \eta} = \nabla_{\xi} + \nabla_{\eta}$ and $\nabla_{f\xi} = f\nabla_{\xi}$
    for all $f \in \mathcal{O}_{X}$ and $\xi \in \mathcal{T}$.
    \item (Leibniz rule) $\nabla_{\xi}(f\cdot \phi) = f\cdot \nabla_{\xi}(\phi) + (\xi\cdot f)\phi$
    for all $f\in\mathscr{O}_X, \xi \in\mathcal{T}$, and $\phi \in \mathcal{P}$.
\end{enumerate}
A connection is called \emph{flat} if $\nabla$ is a Lie algebra homomorphism, i.e. 
$$
\nabla_{[\xi, \eta]} = [\nabla_{\xi}, \nabla_{\eta}].
$$
In other words, the connection form $R_{\xi\eta}:= \nabla_{[\xi, \eta]} - [\nabla_{\xi}, \nabla_{\eta}]$ is identically 0.
If $U\subset X$ is an open subset fih local coordinates $x_1, \dots, x_k$ and trivialization $\mathcal{P}|_U \simeq U \times \mathbb{C}^n$, it can be written as
$$
\nabla_{\frac{\partial}{\partial x_i}} = \frac{\partial}{\partial x_i} + A_i
$$
fomr some $\mathrm{M}_n(\mathbb{C})$-valued function $A_i$.
Then flatness of $\nabla$ is equivalent to 
$$
\frac{\partial A_{j}}{\partial x_{i}} - \frac{\partial A_{i}}{\partial x_{j}} + [A_i, A_j] =0
$$
for all $i, j$.

Let $\mathcal{S}_U$ be a set of all trivializations of $\mathcal{P}|_U$.
Then $\mathcal{G}_U:=\{f:U \to \GL_n(\mathbb{C})\}$ acts simply transitively on $\mathcal{S}_U$ (i.e. $\mathcal{S}_U$
is $\mathcal{G}_U$-torsor) via conjugation:
$$
g\left(\frac{\partial}{\partial x_i} + A_i\right)g^{-1} = \frac{\partial}{\partial x_i} + gA_i g^{-1} - \left(\frac{\partial g}{\partial x_i}\right)g^{-1},
$$
we call $A_i \mapsto gA_i g^{-1} - \left(\frac{\partial g}{\partial x_i}\right)g^{-1}$ as \emph{gauge} transform.

One can prove that flatness of a vector bundle is equivalent to imposing a flat connection on the vector bundle.
For example, if a flat connection $\nabla$ of $\mathcal{P}$ is given, then we can obtain local identification of nearby fibers
by solving a differential equation $\nabla \phi = 0$ locally, and connect from $p \in \mathcal{P}_{x}$ to $p' \in \mathcal{P}_{x'}$ along $\phi$.
By the theory of PDE, such $\phi$ (called \emph{horizontal section}) always uniquely exists (locally) for given initial condition.
Also, this gives a monodromy action of fundamental group.
For $p \in \mathcal{P}_x$ and a loop $\gamma$, we can ``solve'' $\nabla \phi = 0$ locally along a loop
and coming back to $x$, which gives $\rho_{\mathcal{P}}: \pi_1(X, x) \to \GL_n(\mathbb{C})$.
Since changing initialization gives conjugated representation, and we get a bijection between
equivalence classes of flat vector bundles of rank $n$ and equivalence classes of representations $\pi_1(X, x) \to \GL_n(\mathbb{C})$.

\newpage
\section{Principal $G$-bundles and topologies (September 20)}

Let $G$ be a reductive group over $\mathbb{C}$, and let $\mathcal{P}$ be a principal
$G$-bundle on a curve $X$ over $\mathbb{C}$ (which is also a $G$-torsor on $X$).
Tannaka's theorem tells us that we can reconstruct $G$ from its category of representations $\mathsf{Rep}_{G}$
which is naturally a tensor category (for given two representations of $G$, we have a tensor product representation of them).
Using this, we can prove that there's a natural bijection between 
\begin{align*}
    \boxed{
        \text{principal $G$-bundles on $X$ (w.r.t. particular topology) with connection}
    }
\end{align*}
and
\begin{align*}
    \boxed{
        \text{Tensor functors $\mathsf{Rep}_{G}(\mathbb{C}) \to \mathsf{VecBun}^{\nabla}(X)$ (w.r.t. same topology)}
    }
\end{align*}
where the map is given by
$$
(\mathcal{P}, \nabla) \mapsto \mathcal{F}_{\mathcal{P}}: (\rho, V) \mapsto \mathcal{P} \times V
$$
where $\mathcal{P}$ acts on $V$ via $\rho$.
Note that tensor product on the category $\mathsf{VecBun}^{\nabla}$ (vector bundle with connection) is defined as
$$
(V, \nabla_V) \otimes (W, \nabla_W) := (V \otimes W, \nabla_V \otimes \mathrm{id}_W + \mathrm{id}_V \otimes \nabla_W).
$$

Last week, we gave a correspondence between (flat) holomorphic/algebraic $\hat{G}$-bundles on $X$ with 
a double coset space $G(F)\backslash G(\mathbb{A}_F) / G(\mathcal{O}_F)$, for everywhere-unramified cases.
What if we allow ramifications? What kind of geometric objects would correspond to the flat $\hat{G}$-bundles?
Before we give an answer, we need to specify what kind of topology are we going to impose on $X$.

In algebraic geometry, there's a versatile notion of \emph{Grothendieck topology}
in which open set $U\subset X$ is replaced by morphisms $U \to X$ of certain type.\footnote{To be more precise, Grothendieck topology is a topology
on a certain category of morphisms associated to $X$}
For example, Grothendieck topology defined with \'etale morphisms $U \to X$ is called \emph{\'etale topology}.
We have the following inclusion of topologies: $A \rightarrow B$ means that $B$ is finer than $A$.\footnote{fppf: flat and locally of finite presentation, fpqc: faithfully-flat and quasi-compact.}
\begin{align*}
    \boxed{
        \text{Zariski $\rightarrow$ \'etale $\rightarrow$ fppf $\rightarrow$ fpqc}
    }
\end{align*}
With Grothendieck topology on $X$, principal $G$-bundle is a scheme $\mathcal{P}$ with $G$-equivariant morphism $\pi: \mathcal{P} \to X$
such that 1) it is locally trivial in the sense that we have a covering of $X$ with (certain type of) morphisms
$\{\iota_{\alpha}:U_{\alpha} \to X\}_{\alpha}$ such that pullback of $\mathcal{P}$ to $U_\alpha$, $\iota_{\alpha}^{*}\mathcal{P}=\mathcal{P}\times_{X} U_{\alpha}$, is 
isomorphic to $G \times U_{\alpha}$ (i.e. trivial), and 2) the morphism $\iota_{\alpha}$ ``agree on intersections'' in the sense that pullback of $\mathcal{P}$
along two morphisms
\begin{align*}
U_{\alpha} \times_X U_{\beta} \xrightarrow{\iota_{\alpha\beta}} U_{\alpha} \xrightarrow{i_{\alpha}} X \\
U_{\alpha} \times_X U_{\beta} \xrightarrow{\iota_{\beta\alpha}} U_{\beta} \xrightarrow{i_{\alpha}} X
\end{align*}
are isomorphic.

Notion of a principal $G$-bundle depends on the topology.
For example, consider a double covering 

\begin{center}
    \begin{tikzcd}
        Y = \mathrm{Spec}\mathbb{C}[y, y^{-1}] \simeq \mathbb{A}^{1} \backslash \{0\}\arrow[d]  \\
        X = \mathrm{Spec}\mathbb{C}[x, x^{-1}] \simeq \mathbb{A}^{1} \backslash \{0\}
    \end{tikzcd}
\end{center}
corresponds to a map $\mathbb{C}[x, x^{-1}] \to \mathbb{C}[y, y^{-1}], x \mapsto y^2$.
We have a $\mathbb{Z}/2$-action on $Y$ given by $y \mapsto -y$.
With \'etale topology, this automorphic imposes a $\mathbb{Z}/2$-bundle structure on $Y\to X$.
However, this can't be a $\mathbb{Z}/2$-bundle with Zariski topology.
In fact, any open subset of $X$ in Zariski topology has a form of $X \backslash S$ for some finite set $S \subset X$, 
and there's no $U\subset X$ such that $Y|_{U} \simeq U \coprod U$ (i.e. it can't be locally trivial).

There are some \emph{special} groups $G$ where local triviality of principal $G$-bundle on Zariski topology is equivalent to
that on \'etale topology, such as $G = \GL_n, \SL_n, \mathrm{Sp}_n$.
For such groups, it is also true that local triviality with Zariski topology is also equivalent to that on fpqc topology.

In case of $\GL_n$, we have the following 1-1 correspondences:
\begin{center}
    \begin{tikzcd}
        \boxed{\text{Locally free sheaves of } \mathcal{O}_X\text{-mod of rank }n } \arrow[d, leftrightarrow] \\
        \boxed{\text{Vector bundles of rank }n} \arrow[d, leftrightarrow]\\
        \boxed{\text{Principal }\GL_n\text{-bundles}}
    \end{tikzcd}
\end{center}
this follows from Grothendieck's faithful flat descent.
Grothendieck also proved that the notion of $G$-bundle coincides on fpqc and \'etale topology
when $G$ is a reductive group.
When $X$ is a projective curve over an algebraically closed field, the notion of $G$-bundle with \'etale, Zariski, and analytic topology all coincides, and this implies that
we have a bijection
\begin{center}
    \begin{tikzcd}
        \boxed{G(F) \backslash G(\mathbb{A}_F) / G(\mathcal{O}_F)} \arrow[d, leftrightarrow] \\
        \boxed{\text{equivalence classes of principal }G\text{-bundles loc. triv. in Zariski topology}}
    \end{tikzcd}
\end{center}
which is due to Drinfeld-Simpson.




\newpage
\section{Principal $G$-bundles and topologies (September 22)}

Let $X$ be a curve over $\mathbb{F}_q$ or $\mathbb{C}$, and
let $G$ be a (split) reductive group over a same ground field.
Our goal is to find a bijection between the double coset space
$$
G(F) \backslash G(\mathbb{A}_F) / G(\mathcal{O}_F)
$$
and the set of equivalence classes of principal $G$-bundles on $X$.
For this, we shall use a covering of $X$ of the form
$$
X = \left(\coprod_{1\leq i \leq n} D_{x_{i}}\right) \cup \left(X \backslash \{x_1, \dots, x_n\}\right)
$$
for a finite set of points $x_1, \dots, x_n \in X$.
Here each $D_{x_{i}}$ is a \emph{small disc} centered at $x_{i}$, which is defined as $D_{x_{i}}:= \Spec \mathcal{O}_{x_{i}}$.
Note that it is a spectrum of formal power series ring, and it is not a Zariski open subset of $X$ but $D_{x_{i}} \hookrightarrow X$ is a fpqc morphism.\footnote{Sometimes $D_{x_{i}}$ is called
\emph{formal disc} centered at $x_{i}$. However, we are not going to use this terminology since it can be confused with the notion of formal scheme.}
Using faithfully flat descent, Grothendieck proved that the following amount of information determines a principal $G$-bundle $\mathcal{P}$ on $X$:\footnote{Note that we can only use faithful descent when a base scheme is Noetherian. Beauville and Laszlo provided an alternative
proof that does not assume Noetherianess of a base scheme.}
\begin{itemize}
    \item $G$-bundle $\mathcal{P}_i$ on each disc $D_{x_i}$,
    \item $G$-bundle $\mathcal{P}_{X^*}$ on $X^{*} = X \backslash \{x_1, \dots, x_n\}$,
    \item identification on overlaps, i.e. a transition function $f_{x_i}: \mathcal{P}_i|_{D_{i}^{*}} \simeq \mathcal{P}_{X^*}|_{D^{*}_i}$
    where $D_{i}^{*} = \Spec F_{x_{i}}$ is a \emph{puctured disc} at $x_i$.
\end{itemize}
It is easy to show that any principal $G$-bundle trivializes when it is restricted to a disc $D_x$ for any $x$.
Suppose that any $G$-bundle on $X$ also trivializes to a sufficiently small Zariski open subset of $X$, i.e.
$X \backslash S$ for a sufficiently large finite subset $S \subset X$.
Such a condition is satisfied when $X$ is a complex curve, or $G = \GL_n$, or $X$ is a curve over a finite field and $G$ is a split semisimple group.
Then, with identification of local trivialization, $f_{x_i}: D_{x_{i}}^{*} \times G \to D_{x_{i}}^{*} \times G$ can be thought as an element of $G(F_{x_i})$, and we get
an associated element
$$
g = (g_{x}) \in G(\mathbb{A}_F) = \prod_{x}{}^{'}G(F_{x}), \quad g_{x} = \begin{cases} f_{x_{i}}& x = x_i \\ 1 & \text{otherwise}\end{cases}.
$$
Changing the trivialization on $D_{x_i}$ has an effect of multiplying an element of $G(\mathcal{O}_F) = \prod_x G(\mathcal{O}_x)$.
Also, changing the trivialization on $X^{*}$ has an effect of multiplying an element in $G(F)$ on left.
Hence this gives a map from the set of equivalence classes of principal $G$-bundles on $X$ to the double coset space
$G(F) \backslash G(\mathbb{A}_F) / G(\mathcal{O}_F)$, which is a bijection.
We already saw the case when $G = \GL_1$ before.
Note that $\GL_1$-bundle (or equivalently a vector bundle) is Zariski locally trivial if and only if its restriction on a generic point is trivial.

Based on the bijection, for a curve $X$ over a finite field, we get a 1-1 correspondence between equivalence classes of (unramified) Weil group representations
\begin{align*}
    \boxed{
        \sigma: W(F^{\mathrm{un}} /F) \to {}^{L}G
    }
\end{align*}
and 
\begin{align*}
    \boxed{
        \text{Hecke eigenfunctions on the set of equivalence classes of Zariski }G\text{-bundles on $X$}
    }
\end{align*}
for the everywhere unramified case.
How about the curves over $\mathbb{C}$?
We saw that the objects on Galois side becomes
\begin{align*}
    \boxed{
        \text{hol. (or alg.) }{}^{L}G\text{-bundles on }X\text{ with hol. (or alg.) connection}
    }
\end{align*}
and on the automorphc side, we expect something with $G$-bundles on $X$.
In this case, there's no function-like object, but there \emph{is} a sheaf-like object which would be the candidate for the objects on the automorphic side.
(We'll see next time.)

When $G = \GL_1$ and $X$ is a curve over finite field, we can describe the correspondence between invariants (conjugacy classes) as follows.
Since $G$ is abelian, any irreducible representation of Weil group $W(F^{\mathrm{un}} / F)$ is a character that factors through $W(F^{\mathrm{ab}, \mathrm{un}} / F)$
(Weil group of maximal unramified abelian extension of $F = \mathbb{F}_q(X)$).
The equivalence classes of line bundles on $X$ forms a variaty over $\mathbb{F}_q$ called \emph{Picard variety}, and we denote it by $\mathrm{Pic}_X$.
Then the automorphic sides is just a set of Hecke eigenfunctions on $\Pic_X(\mathbb{F}_q)$.
We can decompose $\Pic_X$ as
$$
\Pic_X = \coprod_{n \in \mathbb{Z}} \Pic_X^{n}
$$
in terms of degree of a line bundle. For a fixed point $x\in X$, it defins a map
$$
h_{x}: \Pic_{X}^{n} \to \Pic_X^{n+1}, \quad \mathcal{L} \mapsto \mathcal{L}(x).
$$
It is known that the Hecke operator $H_x$ at $x \in X$ that acts on the space of functions on $\Pic_X(\mathbb{F}_q)$ is just $h_{x}^{*}$, the pullback of $h_{x}$ (as a map $\Pic_X \to \Pic_X$).
Now let $f$ be a Hecke eigenfunction with eigenvalues $(a_x)_{x \in X}$.
Then
$$
(H_x \cdot f)(\mathcal{L}) = f(\mathcal{L}(x)) = a_{x} f(\mathcal{L}),
$$
and repeating this gives
$$
f(\mathcal{O}_X(D)) = \prod_{i} a_{x_i}^{n_{x_i}}, \quad D = \sum_{i} n_{x_i} [x_i]
$$
Here we used the identification between line bundles and divisors.
Also, we assume that $f$ is normalized as $f(\mathcal{O}_X) = 1$.
Since it is a function on $\Pic_X = \mathrm{Div}_X /\mathrm{PDiv}_X$, it should be trivial on $\mathrm{PDiv}_X$.
In other words, for any $g \in F^{\times}$ with $(g) = \sum_{x}  \mathrm{ord}_x(g)[x]$, 
the set of eigenvalues $(a_x)_{x\in X}$ should satisfy
$$
\prod_x a_{x}^{\mathrm{ord}_{x}(g)} = 1. \quad (*)
$$
So Hecke eigenfunctions are just a set of eigenvalues $(a_x)$ satisfying the condition (*).
On the Galois side, we have a set of Frobenius conjugacy classes $(\sigma(\mathrm{Fr}_x))_{x \in X}$ (which are just numbers in $\overline{\mathbb{Q}}_{\ell}^\times$),
and it is a nontrivial fact that we have 
$$
\prod_{x\in X} \sigma(\mathrm{Fr}_x)^{\mathrm{ord}_{x}g} = 1
$$
for all $g \in F$ and $\sigma: W(F^{\mathrm{ab,un}}/F) \to \overline{\mathbb{Q}}_{\ell}^{\times}$.

\newpage
\section{Abelian Class Field Theory, Grothendieck's Dictionary (September 27)}

Last time, we saw the example on \emph{abelian class field theory},
which is Langlands correspondence for $G = \mathrm{GL}_1 = \mathbb{G}_m$ over a 
function field $F = \mathbb{F}_q(X)$.
It is almost equivalent to the following statement:
a collection $\{b_{x}\}_{x\in |X|}$ of numbers in $\overline{\mathbb{Q}}_{\ell}^{\times}$
corresponds to an 1-dimensional representation (character) $\sigma: W(F^{\mathrm{ab,un}}/F) \to \overline{\mathbb{Q}}_{\ell}^{\times}$
by $b_{x} = \sigma(\mathrm{Fr}_{x})$ if and only if for all $g \in F^{\times}$, we have
$$
\prod_{x} b_{x}^{\mathrm{ord}_{x}(g)} = 1.\qquad (*)
$$
Note that the set $\{\sigma(\mathrm{Fr}_{x})\}_{x}$ determines $\sigma$ by Chebotarev's density theorem.

This is very close to the following theorem due to Lang and Rosenlicht.
\begin{theorem}[Lang, Rosenlicht]
    Let $X$ be a smooth projective geometrically connected curve over a perfect field $k$, and let $A$ be an 
    abelian algebraic group over $k$. Assume that $X(k)\neq\emptyset$ and choose $x_{0} \in X(k)$.
    Then any morphism $\psi: X \to A$ factors through the Jacobian variety $\mathrm{Jac}_X = \Pic_{X}^{\circ}$
    (neutral componenet of $\Pic_X$) via Abel-Jacobi map $X\to \mathrm{Jac}_X, x \mapsto \mathcal{O}([x]-[x_0])$,
    and the corresponding map $\mathrm{Jac}_X \to A$ is also a homomorphism up to translation.
    In other words, there exists $a \in A$ and $\tilde{\psi}: \mathrm{Jac}_{X} \to A$ such that $\psi(x) = \tilde{\psi}(\iota_X(x)) + a$
    for all $x\in X$, where $\iota_X: X\to \mathrm{Jac}_X$ is an Abel-Jacobi map.
\end{theorem}
Once we ``apply'' the theorem to $|X|\to \overline{\mathbb{Q}}_{\ell}^{\times}, x\mapsto \sigma(\mathrm{Fr}_{x})$, we get the one direction of ACFT (although we can't do this naively).

Here's another example of a consequence of the theorem.
Let both $X$ and $A$ equals to an Elliptic curve $E$ over $\mathbb{C}$, and let $\phi= \mathrm{id}_E:E \to E$ be the identity function.
Then the Jacobian of $E$ is just $E$ itself, and the theorem implies that, for any nonzero rational function $g$ on $E$ with $(g) = \sum_{i}n_{i}[x_i]$,
we have $\sum_i n_i \cdot x_i = 0$ (in an additive form).
This is quite non-trivial and it could be also deduced from residue theorem:
when $E \simeq \mathbb{C}/\Lambda$ for a suitable lattice $\Lambda \subset \mathbb{C}$, applying residue theorem to
the integration of $\frac{zg'}{g}$ along a boundary of fundamental domain gives the result.

How can we understand the equation (*)? This follows from \emph{Grothendieck's dictionary}
that relates functions and sheaves on a curve.
Let $Y$ be an algebraic variety over $\mathbb{F}_{q}$ and let $\ell$ be a prime coprime to $q$. 
Let $\mathbf{Sh}_{\ell}(Y)$ be a category of $\ell$-adic sheaves on $Y$, which are inverse limit of sheaves of $\mathbb{Z}/\ell^n$-modules (tensored with $\mathbb{Q}_\ell$).
Then this category has usual operations like direct sum, tensor product, pullback, and pushforward.
Now let $y \in Y(\mathbb{F}_{q})$ be a $\mathbb{F}_{q}$-point of $Y$, which can be regarded as a map $y: \Spec \mathbb{F}_q \to Y$.
For an $\ell$-adic sheaf $\mathcal{F}$ on $Y$, the pullback $y^{*}(\mathcal{F})$, which is a sheaf on $y$ is a ``stalk of $\mathcal{F}$ at $y$''.
This is an $\overline{\mathbb{Q}}_{\ell}$-vector space with Frobenius action.
Then taking a trace of the Frobenius defines a function 
$$
Y(\mathbb{F}_q) \to \overline{\mathbb{Q}}_{\ell}, \qquad y\mapsto \mathrm{Tr}(\mathrm{Fr}_{y}|_{\mathcal{F}_{y}}).
$$
This function nicely behaves under the operations in $\mathbf{Sh}_{\ell}(Y)$: for example, if we have a short exact sequence
$$
0 \to \mathcal{F} \to \mathcal{G} \to \mathcal{H} \to 0
$$
of $\ell$-adic sheaves, then the additivity holds: 
$$
\mathrm{Tr}(\mathrm{Fr}_{y}|_{\mathcal{G}_{y}}) = \mathrm{Tr}(\mathrm{Fr}_{y}|_{\mathcal{F}_{y}}) + \mathrm{Tr}(\mathrm{Fr}_{y}|_{\mathcal{H}_{y}}).
$$
It also behaves nicely under tensor product, pullback, and pushforward.\footnote{We need Grothendieck-Lefschetz formular to prove that it behaves well under pushforward.}

Then we can ask a following question. 
If we have a Hecke eigenfunction $f$ on $\Pic_X(\mathbb{F}_{q})$ (that was defined above), can we find an $\ell$-adic sheaf $\mathcal{F}$ whose corresponding function
equals to $f$?
In other words, does there exists $\mathcal{F}$ such that $\mathrm{Tr}(\mathrm{Fr}_x|_{\mathcal{F}_x}) = f(x)$ for all $x\in \Pic_X (\mathbb{F}_q)$?
Suprisingly, this is true! Deligne proved that we can find such $\mathcal{F}$ which is a local system on $\Pic_X$ and locally constant of rank 1.
Such a sheaf is called \emph{Hecke eigensheaf}, and we'll see more about this in the next lecture.



\newpage
\section{Grothendieck's dictionary, continued (September 29)}

Today's goal is to define the notion of Hecke eigen\emph{sheaf} and their construction, relation to ACFT.
Recall that, for a given $\ell$-adic sheaf $\mathcal{F}$ on $\Pic_X$, 
one can assign a function $f_{\mathcal{F}}: \Pic_X(\mathbb{F}_q) \to \overline{\mathbb{Q}}_{\ell}$
defined as ``trace of Frobenius at a stalk": $f_{\mathcal{F}}(x) = \mathrm{Tr}(\mathrm{Fr}_{x}|\mathcal{F}_{x})$,
and Deligne proved that one can find an $\ell$-adic sheaf, which is even a local system of rank 1, that corresponds to a given function on $\Pic_X(\mathbb{F}_{q})$.
If $f$ is a Hecke eigenfunction, we want a corresponding function to be a Hecke eigensheaf.
To define it, consider a map $h_{x}: \Pic_X(\mathbb{F}_{q}) \to \Pic_X(\mathbb{F}_{q}), \mathcal{L}\mapsto \mathcal{L}(x)$
defined for each $x \in |X|$.
In the last lecture, we defined the Hecke operator $H_{x}$ as $H_{x}:= h_{x}^{*}$.
In fact, $h_{x}$ can be defined as a morphism of Picard variety $\tilde{h}_{x}: \Pic_X \to \Pic_X$,
and pullback along the morphism define Hecke \emph{functor}
$$
\mathbb{H}_{x}:= (\tilde{h}_{x})^{*}: \mathbf{Sh}_{\ell}(\Pic_X) \to \mathbf{Sh}_{\ell}(\Pic_X)
$$
This is compatible with sheaf-function assignment in the sense of the following commuting diagram:
\begin{center}
    \begin{tikzcd}
        \mathbf{Sh}_{\ell}(\Pic_X) \arrow[d, "\mathcal{F}\mapsto f_{\mathcal{F}}", swap] \arrow[r,"\mathbb{H}_{x}"] & \mathbf{Sh}_{\ell}(\Pic_X) \arrow[d, "\mathcal{F} \mapsto f_{\mathcal{F}}"] \\
        \mathrm{Fun}(\Pic_X(\mathbb{F}_{q}), \overline{\mathbb{Q}}_{\ell}) \arrow[r, "H_{x}"]& \mathrm{Fun}(\Pic_X(\mathbb{F}_q), \overline{\mathbb{Q}}_{\ell})
    \end{tikzcd}
\end{center}
This is an example of \emph{categorification}. We have the following dictionary of correspondences:
\begin{center}
    
\end{center}

\begin{definition} An $\ell$-adic sheaf $\mathcal{F}$ on $\Pic_X$ is called \emph{Hecke eigensheaf} if
$\mathbb{H}_{x}\mathcal{F} \simeq \mathcal{A}_{x} \otimes \mathcal{F}$ for some $\mathcal{A}_{x}$ for all $x\in |X|$.\footnote{To be more precise, it is a sheaf with \emph{compatible} collection of isomorphisms $\iota_{x}: \mathbb{H}_{x}\mathcal{F} \simeq \mathcal{A}_{x}\otimes \mathcal{F}$.}
\end{definition}
In this case, the corresponding function becomes Hecke eigenfunction as $H_{x}f_{\mathcal{F}} = a_{x}f_{\mathcal{F}}$ where $a_{x} = \mathrm{Tr}(\mathrm{Fr}_{x}|\mathcal{F}_x)$.
However, this does not tell us how the eigenvalues $a_{x}$'s are related.
Instead, we consider a family of line bundles: define a map $\tilde{h}$ as 
$$
\tilde{h}: X \times \Pic_X \to \Pic_X, \quad (x, \mathcal{L}) \mapsto \mathcal{L}(x).
$$
Then its pullback defines a functor $\mathbb{H} = \tilde{h}^{*}: \mathbf{Sh}_{\ell}(\Pic_X) \to \mathbf{Sh}_{\ell}(X \times \Pic_X)$, and being $\mathcal{F}$ an $\ell$-adic sheaf
on $\Pic_X$ is equivalent to have a sheaf $\mathcal{A}$ on $X$ such that $\mathbb{H}\mathcal{F} \simeq \mathcal{A} \boxtimes \mathcal{F}$.
This formulation is much better and natural since one can prove that $\mathcal{A}$ is actually a rank 1 local system on $X$.
This also tells us how such an $\ell$-adic sheaf would corresponds to a Galois representation
$\sigma: W(F^{\mathrm{ab, un}}/F) \to \overline{\mathbb{Q}}_{\ell}^{\times}$ - which has 1-to-1 correspondence with local system on $X$.

Now we propose Deligne's construction of Hecke eigensheaves using Abel-Jacobi map.
It is a map from $\Sym^{d}X = X^{d} / S_{d}$ (the space of unordered $d$-tuple of points on $X$, which is smooth when $X$ is a curve) to $\Pic_{X}^{d}$ (degree $d$ divisors) defined in an obvious way.
Let's call it $P_{d}$.
Then we have a following commutative diagram on Hecke operators and Abel-Jacobi maps:
\begin{center}
    \begin{tikzcd}
        X \times \Sym^{d}X \arrow[r, "\tilde{h}_{d}"] \arrow[d, "\mathrm{id}_{X} \times P_{d}", swap] & \Sym^{d+1}X \arrow[d, "P_{d+1}"] \\
        X \times \Pic^{d}X \arrow[r, "\tilde{h}"] & \Pic^{d+1}X
    \end{tikzcd}
\end{center}
If we pullback a sheaf on $\Pic_X$ along $P_{d}$'s, then we get a sheaf $\mathcal{G} = \{\mathcal{G}_{d}\}_{d>0}$ on $\coprod_{d>0} \Sym^{d}X$.
Such a sheaf becomes ``eigensheaf'' if $\tilde{h}_{d}^{*}\mathcal{G}_{d} \simeq \mathcal{A} \boxtimes \mathcal{G}_{d+1}$.
For given rank 1 local system on $X$, we can construct such a collection of sheaves $\{\mathcal{G}_{d}\}$ as
$$
\mathcal{G}_{1} := \mathcal{A}, \quad \mathcal{G}_{d}:= \pi^{d}_{*}(\mathcal{A}^{\boxtimes d})^{S_d}
$$
where $\pi^{d}: X^{d} \twoheadrightarrow \Sym^{d}X$.
The stalk of this sheaf at $D = \sum_i n_i [x_i]$ is $\otimes_{i} \mathcal{A}_{x_i}^{\otimes n_i}$, which has rank 1.
Note that if $\mathcal{A}$ has rank $>$ 1, then there's no chance for $\mathcal{G}_{d}$ to be a local system because of dimensions.
One can also check that $\tilde{h}_{d}^{*}\mathcal{G}_{d+1} \simeq \mathcal{A} \boxtimes \mathcal{G}_{d}$.
Now the point is that, $\mathcal{G}_{d}$ descends to a sheaf on $\Pic_X^{d}$.
\begin{theorem}
    For $d > 2g - 2$, there is a rank 1 $\ell$-adic local system $\mathcal{F}_{d}$ on $\Pic_X^{d}$ such that $P_{d}^{*}\mathcal{F}_{d} \simeq \mathcal{G}_{d}$.
\end{theorem}
\begin{proof}
    Since $\mathcal{G}_{d}$ are local system, the are locally constant on fibers. By the way, the map $\Sym^{d}X \twoheadrightarrow \Pic^{d}_X$
    is a projective fibration (with the condition $d > 2g - 2$) and each fibers are isomorphic to $\mathbb{P}^{d-g}$, which is simply connected.
    (This is true for both over $\mathbb{C}$ (topologically simply connected) or $\mathbb{F}_{q}$ (\'etale fundamental group is trivial)).
    So it is constant on each fiber and descends to $\Pic_X^{d}$.
\end{proof}
Thus we have rank 1 $\ell$-adic local systems on $\coprod_{d > 2g - 2}\Pic_{X}^{d}$ namely $\{\mathcal{F}_{d}\}_{d > 2g - 2}$.
Now we can uniquely extend to all $d$ by using the property of eigensheaf.
Choose a point $x \in X(\mathbb{F}_{q})$ and define $\mathcal{F}_{2g-2}, \mathcal{F}_{2g-3}, \dots$ as $\mathcal{F}_{d} := \mathcal{A}_{x}^{\vee} \otimes h_{d+1}^{*}(\mathcal{F}_{d+1})$, 
and one can show that this is independent of the choice of $x$.\footnote{
    If there's no $\mathbb{F}_{q}$-point on $X$, we can still apply the similar argument by choosing any $x \in X(\overline{\mathbb{F}}_{q})$.
    In this case, the ``shifting'' occurs by the degree of residue field $[\mathbb{F}_{q, x}: \mathbb{F}_{q}]$.
}
\newpage
\section{Geometric Langlands correspondence for $\mathrm{GL}_{1}/\mathbb{C}$ (October 4)}

We are going to see what is a (geometric) Langlands correspondence for $\mathrm{GL}_1$ for
curves over $\mathbb{C}$.
Recall that there are 3 equivalent objects on Galois side for complex curves:

\begin{center}
    \begin{tikzcd}
        \boxed{\mathcal{E} = (E, \nabla), \text{ hol. (or $C^\infty$) line bundle with hol. (or flat) connection }} \arrow[d, leftrightarrow] \\
        \boxed{\text{locally constant sheaves of local systems of rank 1}} \arrow[d, leftrightarrow] \\
        \boxed{\pi_1(X, x_0) \to \GL_1(\mathbb{C})}
    \end{tikzcd}
\end{center}
here the correspondence between the first two boxes is often called \emph{Riemann-Hilbert correspondence}.
Note that the flat line bundles has algebraic nature (can be defined over Zariski topology), where
the local systems has analytic nature (need to be defined over analytic topology).
The correspondence between second and third is via monodromy action.

The automorphic side is the same as curves over finite fields cases.
For a given flat/holomorphic line bundle $\mathcal{E}$, one can associate a Hecke eigensheaf
$\mathcal{F}_{\mathcal{E}}$ satisfying $\mathbf{H}\mathcal{F}_{\mathcal{E}} \simeq \mathcal{E} \boxtimes \mathcal{F}_{\mathcal{E}}$,
where $\mathbf{H}$ is the Hecke functor defined as a pullback of $h: X \times \Pic_X \to \Pic_X, (x, \mathcal{L}) \mapsto \mathcal{L}(x)$.
By its property, the sheaf is determined by its restriction on $\Jac_X = \Pic_X^0$, the Jacobian variety of $X$
(or the neutral component of $\Pic_X$).
In case of curves over $\mathbb{C}$, there's much more elegant way to define $\mathcal{F}_{\mathcal{E}}^{0} = \mathcal{F}_{\mathcal{E}} |_{\Jac_X}$,
which we are going to introduce from now on.

First, we have an isomorphism
$$
\Jac_X \simeq \rH^{1}(X, \mathcal{O}_X) / \rH^{1}(X, \mathbb{Z})
$$
which can be obtained as follows.
We have a short exact sequence
$$
0 \to \mathbb{Z} \to \mathcal{O}_X \xrightarrow{\exp(2\pi i\cdot)} \mathcal{O}_X^{\times} \to 1
$$
which induces a long exact sequence of cohomology
$$
0 \to \rH^{1}(X, \mathbb{Z}) \to \rH^1(X, \mathcal{O}_X) \to \rH^{1}(X, \mathcal{O}_X^{\times}) \xrightarrow{\mathrm{deg}} \rH^{2}(X, \mathbb{Z}) \to 0
$$
where $\rH^{2}(X, \mathbb{Z}) \simeq \mathbb{Z}$ is generated by chern class.
Then we get the isomorphism from $\rH^{1}(X, \mathcal{O}_X^{\times}) \simeq \Pic_X$.
Note that $\rH^{1}(X, \mathcal{O}_X) \simeq \mathbb{C}^{g}$ for a complex curve $X$ with genus $g$,
and $\rH^{1}(X, \mathbb{Z})$ is a lattice in it.

Let's get back to the Galois side.
1-dimensional representation of $\pi_1(X, x_0)$ factors through the abelianization of it, which is $\rH_1(X, \mathbb{Z})$.
Also, for $\pi_1(\Jac_X, x_0) \simeq \pi_1(\rH^{1}(X, \mathcal{O}_X) / \rH^{1}(X, \mathbb{Z})) \simeq \rH^{1}(X, \mathbb{Z})$.
Combining with the Poincare duality $\rH^1(X, \mathbb{Z}) \simeq \rH_1(X, \mathbb{Z})$ we get the following diagram:
\begin{center}
    \begin{tikzcd}
        \mathcal{E} \arrow[r, leftrightarrow] & \pi_1(X, x) \arrow[rd, twoheadrightarrow]\arrow[rr] & & \GL_1(\mathbb{C}) \\
        & & \rH_1(X, \mathbb{Z}) \arrow[ru] \arrow[d, "\cong"] \\
        & & \rH^1(X, \mathbb{Z}) \arrow[rd] \\
        \mathcal{F}_\mathcal{E}^{0} \arrow[r, leftrightarrow] & \pi_1(\Jac_X, \tilde{x}) \arrow[ru, twoheadrightarrow] \arrow[rr] & & \GL_1(\mathbb{C})
    \end{tikzcd}
\end{center}
and this describes a nicer (functorial) way to construct Hecke eigensheaf $\mathcal{F}_{\mathcal{E}}^{0}$ on $\Jac_X$ (so on $\Pic_X$) corresponding to $\mathcal{E}$ 
(the ``eigenvalue'' of $\mathcal{F}_\mathcal{E}$ is $\mathcal{E}$).

Hence we have a following geometric Langlands correspondence over $\GL_1$ for complex curves:
\begin{center}
    \begin{tikzcd}
        \boxed{\mathcal{E}=(E, \nabla)\text{: holomorphic line bundles on }X} \arrow[d, leftrightarrow]\\
        \boxed{\mathcal{F}_\mathcal{E}^{0}\text{: Hecke eigensheaves on $\Jac_X = \Pic_X^0$}}
    \end{tikzcd}
\end{center}
Now, we'll figure out what is the top collections, i.e. the moduli space of holomorphic line bundles on $X$
(i.e. moduli space of rank 1 local systems on $X$), which we'll denote as $\Loc_{\GL_1}(X)$.
First we have a map $\Loc_{\GL_1}(X) \to \Jac_X$ defined as a ``forgetful'' map $(E, \nabla) \mapsto E$.
The fiber of this (surjective) map for $E \in \Jac_X$ is a set of all holomorphic connections on $E$.
For any such connection $\nabla$, we can obtain another connection $\nabla' = \nabla + \eta$ for any $\eta \in \rH^1(X, K_X)$ 
(here $K_X$ is a canonical line bundle on $X$).
Hence the fiber becomes a $\rH^{1}(X, K_X)$-torsor, and $\Loc_{\GL_1}(X)$ becomes an affine bundle over $\Jac_X$.
Note that the equivalence classes of $\mathbb{G}_a$-bundles on some $Y$ is $\rH^{1}(Y, \mathcal{O}_Y)$, 
and more generally, the equivalence classes of $V$-bundles on $Y$ (where $V \simeq \mathbb{G}_{a}^{n}$ for some $n$) is in bijection with
$\rH^{1}(Y, \mathcal{O}_Y) \otimes V$.
When $Y = \Jac_X$, we have
$$
\rH^{1}(\Jac_X, \mathcal{O}_{\Jac_X}) \simeq \rH^{1}(X, \mathcal{O}_X) \simeq \rH^{0}(X, K_X)^{*}
$$
where the last isomorphism follows form Serre duality. Hence there exists a canonical element in 
$$
\rH^{1}(\Jac_X, \mathcal{O}_{\Jac_X}) \otimes \rH^{0}(X, K_X)
$$
which gives rise to a \emph{universal} extension
$$
0 \to \rH^{0}(X, K_X) \to \Loc_{\GL_1}(X) \to \Jac_X \to 0
$$
which becomes the moduli space $\Loc_{\GL_1}(X)$.
Universality implies that, for any $\varphi \in \rH^{1}(X, \mathcal{O}_X) \simeq \rH^{0}(X, K_X)^{*}$, 
this gives a $\mathbb{G}_a$-bundle on $\Jac_X$ as follows:
\begin{center}
    \begin{tikzcd}
        0 \arrow[r] & \mathbb{C} \arrow[r] & B_{\varphi} \arrow[r] & \Jac_X \arrow[r] \arrow[d, equal]& 0 \\
        0 \arrow[r] & \rH^{0}(X, K_X)\arrow[u, "\varphi"] \arrow[r] & \Loc_{\GL_1}(X) \arrow[u, "\exists!\tilde{\varphi}"] \arrow[r] & \Jac_X \arrow[r] & 0
    \end{tikzcd}
\end{center}

In fact, the equivalence (Langlands correspondence) between local systems and Hecke eigensheaves are also equivalent
in a derived sense: there's an equivalence between the bounded derived category of $\mathcal{O}$-modules on $\Loc_{\GL_1}(X)$
and the bounded derived category of $D$-modules on $\Jac_X$.
This is an example of \emph{Fourier-Mukai transform} (categorical version of the Fourier transform), which we are going to study in the next lecture.
\newpage
\section{Fourier-Mukai transform for $\mathrm{GL}_{1}/\mathbb{C}$ (October 11)}

Last time, we see how the geometric Langlands correspondence for $\GL_1/\mathbb{C}$
for complex curves, via the sequence of isomorphisms
$$
\pi_1(X, x)^{\mathrm{ab}} \cong \rH_1(X, \mathbb{Z}) \cong \rH^{1}(X, \mathbb{Z}) \cong \pi_1(\Jac_X, \tilde{x}).
$$
(The first isomorphism is Hurwicz' theorem, the second one is from Poincare duality, and the third one follows from
$\Jac_X \simeq \rH^{1}(X, \mathcal{O}_x) / \rH^{1}(X,\mathbb{Z})$.)
Also, we studied about the modulispace of rank 1 local systems on $X$, $\Loc_{\GL_1}(X)$, 
and view it as the universal extension of $\Jac_X$ by a vector space:
$$
0 \to \rH^{0}(X, K_X) \to \Loc_{\GL_1}(X) \to \Jac_X \to 0.
$$
The \emph{moduli space} of some objects means an object that represents some functor that describes given moduli problem.
For example, $\Loc_{\GL_1}(X)$ is a scheme\footnote{it might be a stack - I'm not sure about this point.} that represents the functor
from $\mathbb{C}$-schemes $\mathbf{Sch}_{\mathbb{C}}$ to the category of abelian groups $\mathbf{Ab}$,
$$
S \mapsto \{\text{abelian group of line bundles on }S \times X\text{ with partial connection along }X\}.
$$
(In other words, the above functor isomorphic to the hom functor $\mathrm{Hom}(S, \Loc_{\GL_1}(X))$.)
By the (geometric) Langlands correspondence for $\GL_1$, we have a bijection between
flat line bundles on $X$ and those on $\Jac_X$, so that we can interprete $\Loc_{\GL_1}(X)$ as a moduli space
of flat line bundles on $\Jac_X$.
Now, when $S = \Loc_{\GL_1}(X)$, there's the universal bundle $\mathcal{P}$ on $\Loc_{\GL_1}(X) \times \Jac_X$
that corresponds to the identity map in $\mathrm{Hom}(\Loc_{\GL_1}(X), \Loc_{\GL_1}(X))$.
We have the following diagram:
\begin{center}
    \begin{tikzcd}
        & \mathcal{P} \arrow[d] & \\
        & \Loc_{\GL_1}(X) \times \Jac_X \arrow[dl, "p_1", swap] \arrow[dr, "p_2"] & \\
        \Loc_{\GL_1}(X) & & \Jac_X
    \end{tikzcd}
\end{center}
Fourier-Mukai transform gives us the following \emph{derived} Langlands correspondence:
\begin{theorem}[Mukai, Laumon, Rothstein]
    The functors
    \begin{align*}
        F(\mathcal{M}) &= Rp_{1*}(p_{2}^{*}(\mathcal{M}) \otimes \mathcal{P}) \\
        G(\mathcal{K}) &= Rp_{2*}(p_{1}^{*}(\mathcal{K}) \otimes \mathcal{P}) \\
    \end{align*}
    give an equivalence of two derived categories
    $$
    D^{b}(\mathcal{O}_{\Loc_{\GL_1}(X)}\text{-}\mathbf{Mod}) \simeq D^{b}(D_{\Jac_X}\text{-}\mathbf{Mod})
    $$
    (up to sign and cohomological shift).
\end{theorem}
The transform sends skyscraper sheaves $\mathcal{O}_{(L, \nabla)}$ of $\Loc_{\GL_1}(X)$ supported at $(L, \nabla)$ to 
the corresponding $\mathcal{F}_{(L,\nabla)}^{0}$ on $\Jac_X$.
Mukai's original transformation is defined between the categories of coherent sheaves on an abelian 
variety $A$ and its dual $A^{\vee}$, and there's a categorical generalization between $\mathcal{O}$-modules and $D$-modules.
The above theorem is the special case when $A = \Jac_X$.
\newpage
\section{Frobenius functor and statement for general $G$ (October 13)}

As we saw before, Fourier-Mukai transform gives equivalence between two bounded derived categories, $\mathcal{O}$-modules on 
$\Loc_{\GL_1}(X)$ and $D$-modules on $\Jac_X$.
Also, it sends a skyscraper sheaf $\mathcal{O}_{\mathcal{E}} =\mathcal{O}_{(L, \nabla)}$ for $\mathcal{E} = (L, \nabla)$ (on $\Loc_{\GL_1}(X)$) to a line bundle $\mathcal{F}_{\mathcal{E}}^{0}$ on $\Jac_X$,
which extends to $\mathcal{F}_{\mathcal{E}}$ as a Hecke eigensheaf.
Then we can ask the following question - what is the corresponding operator on $\Loc_{\GL_1}(X)$-side that is compatible with Fourier-Mukai transform?
The answer is the \emph{Frobenius functor}, which we'll explain now.

Choose a point $p \in X$, and consider a map $\bar{h}_p: X \times \Jac_X \to \Jac_X$ defined by
$(x, L) \mapsto L(x-p)$.
Then $\mathcal{F}_{\mathcal{E}}^{0}$ is a Hecke eigensheaf in the sense that $\bar{\mathbf{H}}_p\mathcal{F}_{\mathcal{E}} \simeq \mathcal{E} \boxtimes \mathcal{F}_{\mathcal{E}}^{0}$.
\footnote{Note that we can choose suitable trivialization at a point $p \in X$ so that we can ignore $\otimes \mathcal{L}_p^{-1}$.}
where $\bar{\mathbf{H}}_p = \bar{h}_p^{*}$.
This also induces a functor from $D^{b}(D_{\Jac_X}\text{-}\mathbf{Mod})$ to $D^{b}(D_{X \times \Jac_X}\text{-}\mathbf{Mod})$.
Now, the Frobenius functor
$$
\mathbf{F}_p: D^{b}(\mathcal{O}_{\Loc_{\GL_1}(X)}\text{-}\mathbf{Mod}) \to D^{b}(D_X \boxtimes \mathcal{O}_{\Loc_{\GL_1}(X)}\text{-}\mathbf{Mod})
$$
is defined as
$$
\mathbf{F}_p(\mathcal{K}):= (\mathcal{O}_X \boxtimes \mathcal{K}) \otimes \left(\mathcal{P}|_{X \times \Loc_{\GL_1}(X)}\right)
$$
for $\mathcal{O}_{\Loc_{\GL_1}(X)}$-module $\mathcal{K}$.
Note that $\mathcal{O}_X$ is a $D_X$-module with de Rham differential, and $\mathcal{P}$ is the Poincare bundle
on $\Jac_X \times \Loc_{\GL_1}(X)$ (that is the universal bundle for $\Loc_{\GL_1}(X)$, where the order is swapped compared to the last note)
and we restrict it to $X \times \Loc_{\GL_1}(X)$ via $X \hookrightarrow \Jac_X$.
Especially when $\mathcal{K} = \mathcal{O}_{\mathcal{E}}$ for $\mathcal{E} = (L, \nabla)$, we have
$$
\mathbf{F}_p(\mathcal{O}_{\mathcal{E}}) = \mathcal{E} \boxtimes \mathcal{O}_{\mathcal{E}}
$$
which follows from universality of $\mathcal{P}$ (this is a tautological statement).
Hence $\mathcal{O}_{\mathcal{E}}$ is a Frobenius eigensheaf and we have a following commutative diagram:
\begin{center}
    \begin{tikzcd}
        D^{b}(\mathcal{O}_{\Loc_{\GL_1}(X)}\text{-}\mathbf{Mod}) \arrow[d, "\mathbf{F}_p", swap] \arrow[r, "G", swap, shift right=1] & D^{b}(D_{\Jac_X}\text{-}\mathbf{Mod}) \arrow[l, "F", swap, shift right=1] \arrow[d, "\bar{\mathbf{H}}_p"]\\
        D^{b}(D_X \boxtimes \mathcal{O}_{\Loc_{\GL_1}(X)}\text{-}\mathbf{Mod}) \arrow[r, "G", swap, shift right = 1]& D^{b}(D_{X\times \Jac_X}\text{-}\mathbf{Mod}) \arrow[l, "F", swap, shift right=1]
    \end{tikzcd}
\end{center}

For general reductive group $G$, this can be generalized as follows (which is proven by Gaitsgory and some other people).
Instead of $\Jac_X$, we consider $\Bun_G(X)$, the moduli stack of principal $G$-bundles on $X$ (we have $\Bun_{\GL_1}(X) = \Jac_X$).
On the other side, we consider $\Loc_{{}^{L}G}(X)$ - the moduli stack of flat ${}^{L}G$-bundles on $X$ where ${}^{L}G$ is the (Langlands) dual group of $G$.
Then for a chosen ${}^{L}G$-representation $V$, one have Hecke functor $\mathbf{H}_V$ and Frobenius functor $\mathbf{F}_V$
\begin{align*}
    \mathbf{H}_V &: D^{b}(D_{\Bun_G(X)}\text{-}\mathbf{Mod}) \to D^{b}(D_{X \times \Bun_G(X)}\text{-}\mathbf{Mod}) \\
    \mathbf{F}_V &: D^{b}(\mathcal{O}_{\Loc_{{}^{L}G}(X)}\text{-}\mathbf{Mod}) \to D^{b}(D_X \boxtimes \mathcal{O}_{\Loc_{{}^{L}G}(X)}\text{-}\mathbf{Mod})
\end{align*}
that are compatible under Fourier-Mukai transform (for $G$):

\begin{center}
    \begin{tikzcd}
        D^{b}(\mathcal{O}_{\Loc_{{}^{L}G}(X)}\text{-}\mathbf{Mod}) \arrow[d, "\mathbf{F}_V", swap] \arrow[r, swap, leftrightarrow] & D^{b}(D_{\Bun_G(X)}\text{-}\mathbf{Mod}) \arrow[d, "\mathbf{H}_V"]\\
        D^{b}(D_X \boxtimes \mathcal{O}_{\Loc_{{}^{L}G}(X)}\text{-}\mathbf{Mod}) \arrow[r, swap, leftrightarrow]& D^{b}(D_{X\times \Bun_G(X)}\text{-}\mathbf{Mod})
    \end{tikzcd}
\end{center}
\newpage
\section{Enhanced Fourier-Mukai transform (October 18)}

We'll explain about enhanced Fourier-Mukai transform in detail.
For an abelian variety $A$ over $\mathbb{C}$, the original Fourier-Mukai transform is between 
bounded derived category of coherent sheaves of $\mathcal{O}_A$-modules and that of $\mathcal{O}_{A^\vee}$-modules.
It is defined in a same way as before (using Poincare bundle, pullback, and pushforward).
For a line bundle $L$ on $A$, a sheaf of sections of $L$ (which is a coherent $\mathcal{O}_A$-module)
sends to the skyscraper sheaf $\mathcal{O}_L$ supported at $L \in A^\vee$, that is a coherent
$\mathcal{O}_{A^\vee}$-module.

For \emph{enhanced} version of Fouier-Mukai transform, we need to define $A^\natural$ first.
It is a moduli space of line bundles on $A$ with flat connection. 
For such a bundle $(L, \nabla)$, degree of $L$ is always zero.
Then one has (enhanced) Poincare bunde $\mathcal{P}$ on $A^\natural \times A$
defined by the universal property of $A^\natural$ (the bundle corresponds to the identity map in $\mathrm{Hom}(A^\natural, A^\natural)$).
For the below diagram
\begin{center}
    \begin{tikzcd}
        & \mathcal{P} \arrow[d] & \\
        & A^\natural \times A \arrow[dl, "p_1", swap] \arrow[dr, "p_2"] & \\
        A^\natural & & A
    \end{tikzcd}
\end{center}
we can define functors between the bounded derived category of coherent $\mathcal{O}_{A^\natural}$-modules and  
that of $D$-modules on $A$.
We obtain geometric Langlands for $\GL_1$ when $A = \Jac_X$.

As we did for $\Loc_{\GL_1}(X)$, we have a map $A^\natural \to A^\vee$ defined by $(L, \nabla)$, 
and the fiber of the map is the space of all flat connections on $L$, which is a torsor over $\rH^0(A, \Omega_A^{1, 0})$.
In other words, $A^\natural$ is an affine bundle on $A^\vee$ modeled on $\rH^0(A, \Omega_A^{1, 0}) = T_{0}^*A$.
When $A = V /\Lambda$ for $g$-dimensional complex vector space $V \simeq \mathbb{C}^g$ and a lattice
$\Lambda$ (rank $2g$ free $\mathbb{Z}$-module) in $V$, we can express $A^\vee$ and $A^\natural$ as follows.

\begin{theorem}[Arinkin]
Let $A = V / \Lambda$ be an abelian variety of dimension $g$.
\begin{enumerate}
    \item $A^\vee \simeq \overline{V}^{*} / \Lambda^{*}$, where $\Lambda^{*}$ is a dual lattice of $\Lambda$ in $\overline{V}^{*}$.
    \item $A^\natural \simeq (\overline{V}^{*} \oplus V) / \Lambda^{*}_{\mathrm{diag}}$, where $\Lambda^{*}_{\mathrm{diag}}$ is a diagonal lattice
    defined as an image of $\Lambda^{*} \hookrightarrow \overline{V}^{*} \oplus V^{*}$
    \item The natural extension 
    \begin{center}
        \begin{tikzcd}
            0 \arrow[r] & V^{*} \arrow[r] \arrow[d, "\simeq"] & (\overline{V}^{*} \oplus V^{*}) / \Lambda_{\mathrm{diag}}^{*} \arrow[r] \arrow[d, "\simeq"] & \overline{V}^{*} / \Lambda^{*} \arrow[r] \arrow[d, "\simeq"] & 0 \\
            0 \arrow[r] & \rH^{0}(A, \Omega_{A}^{1, 0}) \arrow[r] & A^{\natural} \arrow[r] & A^{\vee} \arrow[r] & 0
        \end{tikzcd}
    \end{center}
    is a universal extension.
\end{enumerate}
\end{theorem}
Note that when $A = V /\Lambda$, $V \simeq T_{0}A$ and $V^{*} \simeq \rH^{0}(A, \Omega^{1,0})$ and $\overline{V}^{*} \simeq \rH^{0}(A, \Omega^{0,1}) \simeq \rH^{1}(A, \mathcal{O}_A)$
and $\Lambda^{*} \simeq \rH_1(A, \mathbb{Z})^{*} \simeq \rH^{1}(A, \mathbb{Z})$ by Poincare duality.
This is compatible with $A^\vee \simeq \rH^{1}(A, \mathcal{O}_A) / \rH^{1}(A, \mathbb{Z})$ obtained by cohomology sequence
induced from $0 \to \mathbb{Z} \to \mathcal{O}_A \to \mathcal{O}_A^{\times} \to 0$.


\newpage
\section{Proof of enhanced Fourier-Mukai transform (October 20)}

To be added later.
\newpage
\section{Sneak peek of the analytic Langlands (October 25)}

Finally, analytic Langlands.
The goal of analytic Langlands correspondence is the following.
Let $X$ be a complex curve.
For $G = \GL_1$, we have found a correspondence between holomorphic/algebraic line bundles with (flat) connections
on $X$ and Hecke eigen\emph{sheaves} on $\Jac_X = \Pic_X^{0}$.
This is a special case of enhanced Fourier-Mukai transform applied to $A = \Jac_X$.
Now, here's a question: can we develop function-theoretic correspondence for curves over $\mathbb{C}$?
In other words, can we find some sort of Hecke eigen\emph{functions} corresponds 
to the line bundles, instead of sheaves?
The answer would be YES, and we are going to see this from now.

Our guess is that the function-theoretic side would contain
Hecke eigenfunctions on $\Pic_X$, or $\Jac_X = \Pic_X^0$.
Hecke operators are defined almost the same as before, but for simplicity, we are going to 
normalize them by choosing a reference point $p_0 \in X$.
Define $h_p : \Pic^{d}_X \to \Pic^{d+1}_X$ as $h_p(\mathcal{L}) = \mathcal{L}(p)$,
and define Hecke operator $H_p$ as a pullback $h_p^*$, which is a map from the space of 
functions on $\Pic^{d+1}_X$ to that on $\Pic^{d}_X$.
Then any function $f$ on $\Pic_X$ can be decomposed as $f = \sum_d f_d$ for $f_d = f|_{\Pic^{d}_X}$, 
and it becomes Hecke eigenfunction if $H_p(f) = \mu_p f \Leftrightarrow h_{p}^{*}(f_{d+1}) = \mu_p f_{d}$
for all $d \in \mathbb{Z}$, for some nonzero constant $\mu_p \in \mathbb{C}^{\times}$.
By choosing a reference point $p_0 \in X$, we can identify $\Pic^d_X$ and $\Pic^0_X$ via 
$i_d: \Pic_X^d \to \Pic^0_X$, $\mathcal{L} \mapsto \mathcal{L}(-dp_0)$.
Define ${}_{p_0}h_{p}: \Pic_X^0 \to \Pic_X^0$ as $\mathcal{L} \mapsto \mathcal{L}(p - p_0)$, 
and the corresponding normalized Hecke operator ${}_{p_{0}}H_p$ as the pullback of it.
Then $f_0 = f|_{\Pic^0_X}$ is an eigenfunction of ${}_{p_0}H_p$ with eigenvalue $\lambda_p = \mu_p\mu_{p_0}^{-1}$.
This allows us to consider eigenfunctions on the neutral component $\Jac_X = \Pic_X^0$ instead of whole $\Pic_X$.

Let's try to compute eigenfunctions for the cases when $X$ is an elliptic curve.
For example, let's assume that $X = E_i = \mathbb{C} / (\mathbb{Z} + i\mathbb{Z})$.
We chooose $p_0 = 0 \in E_i$, and the Jacobian of $E_i$ is the same as $E_i$ itself.
Any points on $E_i$ can be written as $z_p = x_p + iy_p$ for $x_p$, $y_p \in \mathbb{R}$, 
and Hecke operators on the space of functions on $E_i$ are just translations.
Hence they are functions on $\mathbb{C} / (\mathbb{Z} + i\mathbb{Z})$ such that 
$H_p(f)(x + iy) = f(x + x_i + i(y + y_p)) = \lambda_p f(x + iy)$ for all $x, y, x_p, y_p \in \mathbb{R}$ and a nonzero constant $\lambda_p$.
We can regard functions on $E_i$ as functions on $\mathbb{C}$ that are $(\mathbb{Z} + i\mathbb{Z})$-invariant, and 
one can prove that such a functions are the exponentials
$$
f_{m, n}(x + iy) = e^{2\pi i mx + 2\pi i ny}
$$
for each $m, n \in \mathbb{Z}$.
There eigenvalues are just the functions values at a point $p$.
We can also consider the $L^2$ space on $E_i$ (using Haar measure on $E_i$),
and the above functions $\{f_{m, n}\}_{m, n \in \mathbb{Z}}$ form a basis of $L^2(E_i)$.
For the latter purpose, we express $f_{m, n}$ as functions of $z, \bar{z}$ instead of $x, y$:
$$
f_{m,n}(z, \bar z) = e^{\pi z(n + im)}e^{-\pi \bar z (n - im)}.
$$
More generally, if $X = E_\tau:= \mathbb{C}/(\mathbb{Z} + \tau \mathbb{Z})$ is an elliptic curve associated to lattice $\mathbb{Z} + \tau\mathbb{Z}$,
the Hecke eigenfunctions are
$$
f_{m,n}^\tau = e^{2\pi i m \left(\frac{z \bar \tau - \bar z \tau}{\tau - \bar \tau}\right)} e^{2\pi i n \left(\frac{z - \bar z}{\tau - \bar \tau}\right)}
$$
for $m, n \in \mathbb{Z}$.
From this, our (conjectural) goal would be: associating flat holomorphic line bundles on $E_\tau$ to these eigenfunctions.

When $X$ is a curve with genus $g$ (not necessarily 1), 
then $A = \Jac_X$ will be a higher dimensional abelian variety.
In this case, how could we write down the explicit formula of the eigenfunctions?
Let $\gamma \in \rH^1(A, \mathbb{Z})$.
Then its image in $\rH^1_{\mathrm{dR}}(A, \mathbb{C})$ is a harmonic 1-form, and it has a Hodge decomposition
$\alpha_\gamma +\bar\alpha_\gamma \in \rH^0(A, \Omega_A^{1, 0})\oplus \rH^0(A, \Omega_A^{0, 1})$.
Also, we have an isomorphism $\rH^0(A, \Omega^{1, 0}_A) \simeq \rH^0(X, \Omega_X^{1, 0})$, and let $\omega_\gamma \in \rH^0(X, \Omega_X^{1, 0})$
be the holomorphic form which is the image of $\alpha_\gamma$ via the isomorphism.
Then the following theorem (that will be proved in future) describes Hecke eigenfunctions on $\Jac_X$.
\begin{theorem}
    Hecke eigenfunctions on $\Jac_X$ are
    $$
    \phi_\gamma(Q) = \exp \bigg(2 \pi i \int_0^Q (\alpha_\gamma + \bar\alpha_\gamma)\bigg)
    $$
    for each $\gamma \in \rH^1(A, \mathbb{Z})$.
    The eigenvalue of $H_p$ for $\phi_\gamma$ is given by the similar function on the curve $X$:
    $$
    f_\gamma(p) = \exp \bigg(2\pi i \int_{p_0}^{p} (\omega_\gamma + \bar\omega_\gamma)\bigg).
    $$
\end{theorem}
\newpage
\section{Proof of the theorem (October 27)}

Last time we defined functions $\phi_\gamma : \Jac_X \to \mathbb{C}$ and $f_\gamma : \Jac_X \to \mathbb{C}$
for each $\gamma \in \rH^1(X, \mathbb{Z})$.
Our goal is to show that $\phi_\gamma$'s are Hecke eigenfunctions of $H_p$ with eigenvalue $f_\gamma(p)$.
They were defined as follows:
\begin{align*}
    \phi_\gamma(L) &= \exp \bigg(2 \pi i \int_0^{L} (\alpha_\gamma + \bar\alpha_\gamma) \bigg) \\
    f_\gamma(p) &= \exp \bigg(2 \pi i \int_{p_0}^{p} (\omega_\gamma + \bar\omega_\gamma) \bigg).
\end{align*}
We'll use \emph{Abel-Jacobi map} for the proof. 
Recall that we can describe $\Jac_X$ using divisors as $\mathrm{Div}(X) / \mathrm{PDiv}(X)$.
Let $X^{(d)}:= \Sym^d X$ be a symmetric $d$-th power of $X$.
This is a space of unordered (multi-)set of $d$ points on $X$, which can be thought as a set of degree $d$ effective divisors on $X$.
For each $d \in \mathbb{Z}_{\geq 1}$, Abel-Jacobi map $\AJ_d: X^{(d)} \to \Jac_X$ is defined as
\[
    \AJ_d\left(\sum_{i=1}^{d} [x_i]\right) = \mathcal{O}_X\left(\sum_{i=1}^{d} [x_i] - d[p_0]\right)
\]
for a reference $p_0$ we have chosen.
Then this map is compatible with Hecke operators - we have a following commutative diagram
\begin{center}
    \begin{tikzcd}
        X^{(d)} \arrow[r, "\widetilde{h}_p"] \arrow[d, swap, "\AJ_d"] & X^{(d+1)} \arrow[d, "\AJ_{d+1}"] \\
        \Jac_X \arrow[r, "{}_{p_0}h_{p}"] & \Jac_X
    \end{tikzcd}.
\end{center}
Using this, we can reduce our theorem as the following proposition.

\begin{proposition}
    Define $\widetilde{f}_{\gamma, d}: X^{(d)} \to \mathbb{C}$ as
    \[
    \widetilde{f}_{\gamma, d}\left(\sum_{i=1}^{d} [x_i]\right) = \prod_{i=1}^{d} f_{\gamma}(x_i)
    \]
    Then $\AJ_{d}^{*}(\phi_\gamma) = \widetilde{f}_{\gamma,d}$.
\end{proposition}

Hecke operators can be also defined on $X^{(d)}$'s as a pullback of $\widetilde{h}_p$, and one can 
easily show that $\widetilde{f}_{\gamma, d}$ is an eigenfunction with eigenvalue $f_{\gamma}(p)$.
Since $\AJ_d$ is surjective for sufficiently large $d$ (in fact, for $d \geq g$), the above Proposition implies that
$\phi_\gamma$ is also an eigenfunction with same eigenvalue as $\widetilde{f}_{\gamma, d}$.
To prove it, we'll use Abel-Jacobi theorem, which describes $\AJ_d$ in terms of integration of holomorphic forms.

\begin{theorem}[Abel-Jacobi]
Let $\Jac_X = \rH^{0}(X, K_X)^{*} / \rH_1(X, \mathbb{Z})$.
For $\eta \in \rH^{0}(X, K_X)$, we have
\[
    \left(\AJ_d \left(\sum_{i=1}^{d} [x_i]\right)\right)(\eta) = \sum_{i=1}^{d} \int_{p_0}^{x_i} \eta
\]
where $\AJ_d(\sum_{i=1}^{d}[x_i])$ is well-defined function on $\rH^{0}(X, K_X)$ up to addition of $\rH_1(X, \mathbb{Z})$.
\end{theorem}
Proof can be found in Bost \cite{bost1992introduction}.

\newpage
\section{Proof of the theorem, continued (Nov 1)}
\newpage

\bibliographystyle{acm}
\bibliography{ref}
\end{document}
