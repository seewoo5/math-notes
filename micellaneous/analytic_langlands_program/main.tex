\documentclass{amsart}
\usepackage[utf8]{inputenc}
\usepackage{amsmath, amsthm, amssymb}
\usepackage{url}
\usepackage{mathrsfs}
\usepackage{empheq}
\usepackage{tikz-cd}

\newtheorem{theorem}{Theorem}[section]
\newtheorem{conjecture}{Conjecture}[section]
% \newtheorem*{conjecture*}{Conjecture}[section]
\newtheorem{lemma}{Lemma}
\newtheorem{proposition}{Proposition}[section]
\newtheorem{corollary}{Corollary}[section]
\newtheorem{definition}{Definition}[section]

\DeclareMathOperator{\GL}{\mathrm{GL}}
\DeclareMathOperator{\SL}{\mathrm{SL}}
\DeclareMathOperator{\PGL}{\mathrm{PGL}}
\DeclareMathOperator{\Ad}{\mathrm{Ad}}
\DeclareMathOperator{\Oo}{\mathrm{O}}
\DeclareMathOperator{\Gal}{\mathrm{Gal}}
\DeclareMathOperator{\AI}{\mathrm{AI}}
\DeclareMathOperator{\L2}{\mathrm{L}^{2}}
\DeclareMathOperator{\cusp}{\mathrm{cusp}}

\newcommand{\xrightarrowdbl}[2][]{%
    \xrightarrow[#1]{#2}\mathrel{\mkern-14mu}\rightarrow
}

\title{Analytic Langlands Program}
\author{Seewoo Lee}


\begin{document}

\begin{abstract}
This is a \LaTeX-ed note for the special lecture on \emph{Analytic Langlands Program} by Edward Frankel at UC Berkeley in 2022 Fall.

\end{abstract}

\maketitle
\tableofcontents

\newpage
\section{Introduction to Langlands correspondence (August 25)}

This course is on a new aspect of Langlands program, so-called \emph{Analytic Langlands Program}.
The classical Langlands program is originated from Langlands' letter to Andr\'e Weil in 1967,
and also from Andr\'e Weil's letter to his sister  (Simone Weil) on his conjecture (Weil's conjecture on
zeta functions of curves over finite fields, which was resolved by Dwork, Grothendieck, and Deligne) in 1940.
Weil's \emph{Rosetta stone} relates two different topics in mathematics: number theory and complex curves (Riemann surfaces).
A goal is to find something happens in parallel between two, and we need another bridge - curves over finite fields.
The difference between complex curves and curves over finite fields is the fact that those are defined over different fields.
A similarity between number theory sied and the curves over finite fields side is that 
the number fields (finite extensions of $\mathbb{Q}$) are similar to the function fields of curves (over finite fields - we denote it as $\mathbb{F}_{q}(X)$ for a curve $X/\mathbb{F}_{q}$).
The most simplest example is a comparison between $\mathbb{Q}$ and $\mathbb{F}_{q}(\mathbb{P}^{1}) \simeq \mathbb{F}_{q}(t)$:
\begin{align*}
    \mathbb{Q} = \left\{ \frac{p}{q}\,:\,p, q\,\text{rel. prime} \in \mathbb{Z}\right\} &\leftrightarrow \mathbb{F}_{q}(t) = \left\{ \frac{P(t)}{Q(t)}\,:\, P, Q\,\text{rel. prime} \in \mathbb{F}_{q}[t]\right\} \\
    \text{ring of integers: }\mathbb{Z} &\leftrightarrow \mathbb{F}_{q}[t] \\
    \text{completions: }\mathbb{Q}_{p} &\leftrightarrow \mathbb{F}_{q}((t))
\end{align*}
Sometimes we include one more topic in this Rosetta stone, which originates from Physics -
Quantum Field Theory, Electro-Magnetic Duality, and Gauge Theory (developed by Edward Witten and other physicists).

The (classical) Langlands correspondence is about interplays between the Galois representations and Automorphic representations.\footnote{The \emph{geometric} Langlands correspondence is about curves over $\mathbb{C}$, which is mainly developed by Drinfeld, Laumon, Beilinson, Gaitsgory, ...}
It dealts with two different (but similar) types of fields - number fields and function fields of curves over finite fields.
Fix such a field $F$.
Then we can describe a Langlands correspondence for $\GL_{n}$ over $F$ as follows:
\begin{enumerate}
    \item \textbf{Galois side:} Let $\overline{F}$ be a (separable) algebraic closure of $F$, and $\Gal(\overline{F}/F)$ 
    be the absolute Galois group of $F$ (i.e. the group of automorphisms of $\overline{F}$ that fix $F$ pointwisely).
    This is one of the most important groups in number theory, and its structure is higly complicated.
    Hence, instead of studying the group $\Gal(\overline{F}/F)$ directly, we study the representations of it.
    Especially, we are going to consider the (equivalence) classes of $n$-dimensional representations of $\Gal(\overline{F} / F)$ over some field that would be determined later.    
    This is just an equivalence class of homomorphisms
    $$
        \sigma: \Gal(\overline{F} / F) \to \GL_{n}(?).
    $$
    \item \textbf{Automorphic side:} We first need to define the notion of \emph{Adele}.
    Let $\mathscr{V} = \mathscr{V}_{F}$ be the set of equivalent classes of the norms (places) on $F$.
    For each $v \in \mathscr{V}$, we can define a completion $F_{v}$ with respect to $v$.
    For example, when $F = \mathbb{Q}$, Ostrowski's theorem states that the places of $\mathbb{Q}$ corresponds to
    the set of primes (each prime $p$ gives $p$-adic norms, which is non-archimedean) along with the ``infinite'' prime (corresponds to the usual archimedean norm).
    In this case, we have two types of completions, either $p$-adic numbers $\mathbb{Q}_{p}$ or real numbers $\mathbb{R}$.
    And we have the ring of $p$-adic integers $\mathbb{Z}_{p}$ as a subring of $\mathbb{Q}_{p}$.

    In case of function field $F =\mathbb{F}_{q}(\mathbb{P}^{1}) \simeq \mathbb{F}_{q}(t)$ of $X = \mathbb{P}^{1}$,
    the places of $F$ corresponds to the closed points of $X$, which again corresponds to
    the maximal ideals of $\mathbb{F}_{q}[t]$ (and the point at infinity).
    For example, any $a \in \mathbb{F}_{q}$ actually gives a closed point that corresponds to
    the maximal ideal $(x-a)$.
    Any other irreducible polynomials over $\mathbb{F}_{q}$ of higher degree also give closed points in $X$.
    Completion of $F$ at $x \in X$ is isomorphic to the field of formal Laurent series $(\mathbb{F}_{q})_{x}((t_{x}))$,
    where $(\mathbb{F}_{q})_x$ is the residue field at $x$ and $t_{x}$ is some parameter.
    We also have a ring of integers in these completions, which is $(\mathbb{F}_{q})_{x}[[t_{x}]]$.

    The ring of adeles is defined as a restricted product of all completions of $F$, which is
    $$
        \quad\quad\quad\quad \mathbb{A}_{F} = \prod_{v \in \mathscr{V}_{F}} F_{v} = \left\{(f_{v})\,:\, f_{v} \in F_{v}, f_{v} \in \mathcal{O}_{v}\text{ for all but finitely many $v$.}\right\}
    $$
    where $\mathcal{O}_{v} \subset F_{v}$ is the ring of integers of $F_{v}$, which is the set of elements with norm at most 1.
    Then we have a diagonal embedding $F \hookrightarrow \mathbb{A}_{F}$ that sends $a \in F$ to $(a, a, \dots) \in \mathbb{A}_{F}$,
    and this induces an embedding $\GL_{n}(F) \hookrightarrow \GL_{n}(\mathbb{A}_{F})$.
    Now we can think of a Hilbert space $\mathscr{H}(\GL_{n}(F) \backslash \GL_{n}(\mathbb{A}_{F}))$ of $\L2$-functions on the quotient space $\GL_{n}(F) \backslash \GL_{n}(\mathbb{A}_{F})$,
    with the Haar measure on the quotient space.
    Then we have a right regular representation of $\GL_{n}(\mathbb{A}_{F})$ on $\mathscr{H}$, and it is known that this representation decomposes
    into continuous part and discrete part:
    $$
        \mathscr{H} = \mathscr{H}_{\mathrm{cont}} \oplus \mathscr{H}_{\mathrm{disc}}
    $$
    and the discrete part decomposes into irreducible representations as 
    $$\mathscr{H}_{\mathrm{disc}} = \oplus_{\pi}\pi$$
    without multiplicity (multiplicity one theorem).\footnote{
        It is not always the case that any representation decomposes into irredubiles - 
        consider the regular representation of $\mathbb{R}$ on $\L2(\mathbb{R})$ that acts as a translation.
        The irreducible sub-representations of it correspons to the exponential function $\exp(i\lambda x)$ for $\lambda \in \mathbb{R}$,
        but we can't write a function $f(x)$ as a discrete sum of these in general. 
        We can only write it as an integral of these, which is the Fourier transform.
        Note that the regular representation on $\L2(\mathbb{Z}\backslash \mathbb{R})$ decomposes into irreducibles (which gives Fourier series),
        and the reason behind is that the circle group $\mathbb{S}^{1} = \mathbb{Z} \backslash\mathbb{R}$ is compact.
    }
    The irreducible constituents of $\mathscr{H}_{\mathrm{disc}}$ is called \emph{cuspidal automorphic rerpesentations of $\GL_{n}(\mathbb{A}_{F})$},
    up to technical conditions on the center of the group and the archimedean places.
    \item \textbf{Correspondence:} The Langlands correspondence states that there is a one-to-one correspondence between these two different objects that preserves some special invariants.
    More precisely, the equivalence classes of $n$-dimensional Galois representation $\sigma$ of $\Gal(\overline{F}/F)$ (over some field)
    corresponds to an irreducible automorphic representation $\pi = \pi(\sigma)$ of $\GL_{n}(\mathbb{A}_{F})$, and vice versa.
    The irreducibility of $\sigma$ corresponds to cuspidality of $\pi$.

    One of the invariant of Galois representation $\sigma$ is the conjugacy classes of $\sigma(\mathrm{Fr}_{x})$ in $\GL_{n}$, where
    $\mathrm{Fr}_{x}$ is the \emph{Frobenius} element corresponds to a closed point $x \in X$ (for almost all $x$).
    On the automorphic side, there is so-called \emph{Hecke operators} $h_{x}$ for each $x \in X$.
    Then the conjectural Langlands correspondence should give correspondences between these two invariants.
    
\end{enumerate}

A celebrated example of Langland's correspondence is the Shimura-Taniyama-Weil conjecture, which is now a theorem by Andrew Wiles and Richard Taylor (modularity theorem).
It is a special case of Langland's correspondence for $n = 2$ that  relates an elliptic curve and a modular form.

Let $E$ be an elliptic curve over $\mathbb{Q}$, i.e. a smooth projective curve defined over $\mathbb{Q}$ by equation
$$
    y^{2} = x^{3} + ax + b
$$
for $a, b \in \mathbb{Q}$ and $\Delta = 4a^{3} + 27b^{2} \neq 0$ (the discriminant of $E$).
Then for each prime $\ell$ not dividing $\Delta$ (or any $\ell \neq p = \mathrm{char}(F)$ when $F$ is a function field),  the first \'etale cohomology $\mathrm{H}^{1}_{\text{\'et}}(E_{\overline{\mathbb{Q}}}, \mathbb{Q}_{\ell})$ is isomorphic to $\mathbb{Q}_{\ell}^{2}$,
2-dimensional vector space over $\mathbb{Q}_{\ell}$.
Since $E$ is defined over $\mathbb{Q}$, we have a natural action of $\Gal(\overline{\mathbb{Q}}/\mathbb{Q})$ on $E_{\mathbb{Q}} = E \otimes_{\mathbb{Q}}\overline{\mathbb{Q}}$ which
induces an action on $\mathrm{H}^{1}_{\text{\'et}}(E_{\overline{\mathbb{Q}}}, \mathbb{Q}_{\ell})$i, i.e. a 2-dimensional representation $\sigma_{E, \ell}$ over $\mathbb{Q}_{\ell}$
(so the mysterious field where the representation is defined that we didn't defined before is $\mathbb{Q}_{\ell}$ in this case).
For $p \nmid \Delta$, Frobenious conjugacy class $\sigma_{E, \ell}(\mathrm{Fr}_{p})$ has a trace
$$
    \mathrm{Tr}(\sigma_{E, \ell}(\mathrm{Fr}_{p})) = p + 1 - \# E(\mathbb{F}_{p})
$$
and determinant $p$, which completely determines the conjugacy class of $\sigma_{E, \ell}(\mathrm{Fr}_{p})$ for $\GL_{2}$.
Here $\#E(\mathbb{F}_{p})$ is the number of $\mathbb{F}_{p}$-points on $E$.

Assuming Langlands' correspondence, such $\sigma_{E, \ell}$ (or family of $\sigma_{E, \ell})$ for varying $\ell$'s) should corresponds
to an irreducible automorphic representation of $\GL_{2}(\mathbb{A}_{\mathbb{Q}})$.
In case of $\GL_{2}$, an automorphic representation $\pi$ corresponds to certain holomorphic function $f_{\pi}$ called \emph{modular form} on
the complex upper half plane $\mathfrak{H}$ satisfying a functional equation
$$
    f_{\pi} \left(\frac{a\tau + b}{c\tau + d}\right) = (c\tau + d)^{2}f_{\pi}(\tau)
$$
for $(\begin{smallmatrix} a & b \\ c & d\end{smallmatrix}) \in \Gamma_{0}(N) = \{(\begin{smallmatrix} a &b \\ c & d\end{smallmatrix}) \in \mathrm{SL}_{2}(\mathbb{Z}), c\equiv 0\,(\mathrm{mod}\,N)\}$
where $N = N_{E}$ is the conductor of $E$.
Also, $f_{\pi}(\tau)  \to 0$ as $\tau \to \infty$ (in other words, $f_{\pi}$ is a cusp form).
It admits a Fourier expansion
$$
    f_{\pi}(\tau) = \sum_{n\geq 1} a_{n} q^{n}, \quad q = e^{2 \pi i \tau}
$$
and the Langlands correspondence for this case becomes the equality
$$
a_{p} = \mathrm{Tr}(\sigma_{E, \ell}(\mathrm{Fr}_p)) = p + 1 - \$ E(\mathbb{F}_{p}).
$$
for all $p \nmid \ell N$.

Langlands correspondence for $\GL_n$ is now a theorem when $F$ is a function field of some curve, and this is proven by Drinfeld ($n = 2$) and Laurent Lafforgue ($n>2$).\footnote{Lafforgue got a Fields medal for this work.}

\newpage
\section{More on Classical Langlands Correspondence (August 30)}

We are going to give more detailed explanations on the classical Langlands correspondence
and give an explicit example of a correspondence between elliptic curves and modular forms
(Taniyama-Shimura-Weil conjecture, now a theorem by Wiles-Taylor and Breuil-Conrad-Diamond-Taylor).

First, irreducible cuspidal automorphic representations $\pi$ of $\GL_{n}(\mathbb{A}_{F})$
always decomposes into \emph{local} representations as\footnote{this is called Flath's theorem.}
$$
\pi = \bigotimes_{v \in \mathscr{V}}\pi_{v}
$$
(this is also a kind of restricted product).
When $F = \mathbb{F}_{q}(X)$ is a function field, then there is a 1-1 correspondence between
the set of places (completions) $\mathscr{V} = \mathscr{V}_{F}$ and the set of closed points $|X|$ of a curve $X$.
(There are only non-archimedean places.)
If a place $v \in \mathscr{V}$ corresponds to a point $x\in |X|$, and the completion of $F$
by $v$ is isomorphic to $(\mathbb{F}_{q})_{x}((t_{x}))$, where $t_{x}$ is a local coordinate at $x$.
In this case, each $\pi_{v}$ becomes a representation of $\GL_{n}(F_{v})$.
When $F$ is a number field, there exist archimedean places, which has a different nature from nonarchimedean places.
For example, when $F = \mathbb{Q}$, we have $\mathscr{V}_{\mathbb{Q}} = \{p\,:\,p\text{ prime}\} \cup \{\infty\}$, and
$\pi$ decomposes as
$$
\pi = \left(\bigotimes_{p < \infty} \pi_{p}\right) \otimes \pi_{\infty}.
$$
Although $\pi_{p}$'s are representations of $\GL_{2}(\mathbb{Q}_{p})$, $\pi_{\infty}$ is \emph{not} an
irreducible representation of $\GL_{2}(\mathbb{R})$.
It is acually a representation of $(\mathfrak{gl}_{2}(\mathbb{R}), \mathrm{O}_{2}(\mathbb{R}))$ - in other words,
it is a representation of Lie algebra $\mathfrak{gl}_{2}(\mathbb{R)}$ and a (maximal compact subgroup) $\mathrm{O}_{2}(\mathbb{R})$
with compatibility condition on their actions.

Recall that the classical Langlands correspondence for $\GL_{n}$ is a correspondence between (equivalence classes of)
$n$-dimensional irreducible ($\ell$-adic) Galois representations $\sigma$ of $\Gal(\overline{F}/F)$
and (equivalence classes of ) cuspidal automorphic representations of $\GL_{n}(\mathbb{A}_{F})$.
It is not an arbitrary 1-1 correspondence - certain \emph{invariants} should match.
The Galois-side invariant is semisimple Frobenius conjugacy classes in $\GL_{n}(\overline{\mathbb{Q}_{\ell}})$: it is
$$
    \{\sigma(\mathrm{Fr}_v), v\in \mathscr{V}\backslash S_{\sigma}\}
$$
where $S_{\sigma}$ is a finite subset of $\mathscr{V}$.
Note that the topology matters for Galois side - we have Krull topology (profinite topology) on $\Gal(\overline{F}/F)$
and we only consider continuous representations.
On the automorphic side, there are certain semisimple conjugacy classes in $\GL_{n}(\mathbb{C})$, which we call
Hecke conjugacy classes.
These record eigenvalues of the (spherical) Hecke algebra associated to each $v\in \mathscr{V}$.
We denote it as
$$
    \{\pi(h_{v}),v\in \mathscr{V}\backslash S_{\pi}\}
$$
where $S_{\pi}$ is a finite subset of $\mathscr{V}$.
Note that we can identify $\overline{\mathbb{Q}_{\ell}}$ and $\mathbb{C}$ since they have
the same transcendence degree over $\overline{\mathbb{Q}}$, and the correspondence is independent of
the choice of identification.
Also, the invariants uniquely determine representation themselves.


Now, we will introduce an explicit correspondence between a certain elliptic curve and a modular form.
Let $E$ be an elliptic curve over $\mathbb{Q}$ defined by
$$
y^{2} + y = x^{3} - x^{2}.
$$
Then the only bad prime of reduction is 11, and the conductor of the elliptic curve is also 11.
We can count the number of $\mathbb{F}_{p}$-points on the curve.
For example, when $p = 5$, there are exactly 5 points: $\{(0, 0), (1, 0), (0, 4), (1, 4), \infty\}$.
Now, consider the following function defined as an infinite product:
$$
    f(\tau) = q\prod_{n=1}^{\infty} (1 - q^{n})^{2}(1-q^{11n})^{2}, \quad q = e^{2\pi i \tau}.
$$
It turns out that this is a modular frm of weight 2 and level 11. Its expansion is
$$
    f(\tau) = q - 2q^{2} - q^{3} + 2q^{r} + q^{5} + 2q^{6} - 2q^{7} + \cdots
$$
and the 5th coefficient of $f$ is $a_{5}(f) = 1$, which equals to $a_{5}(E) = 5 + 1 - 5 = 1$.
In fact, this is the modular form corresponds to $E$, and $a_{p}(E) = a_{p}(f)$ holds for all $p\neq 11$.

As an aside, Langlands correspondence for $\GL_{1}$ has long been known as \emph{abelian class field theory}.
Since $\GL_{1}$ is an abelian group, 1-dimensional Galois represention should factor through $\Gal(F^{\mathrm{ab}}/F)$
and the structure of the latter group is well known for some cases.
For example, we have a Kronecker-Weber theorem when $\mathbb{Q}$, which states that $\mathbb{Q}^{\mathrm{ab}} = \cup_{n\geq 1}\mathbb{Q}(\zeta_{n})$.


Now we will explain Frobenius automorphisms and conjugacy classes in detail.
The Galois group of finite extension of finite fields has a simple structure.
For the extension $\mathbb{F}_{q^{n}} /\mathbb{F}_{q}$, its Galois group is just
a cyclic group of order $n$ generated by the Frobenius automorphism $x \mapsto x^{q}$.
Now let $K/F$ be a finite extension of number fields, and $\mathcal{O}_{F} \subset \mathcal{O}_{K}$ be the ring of integers.
These are Dedekind domain: any ideal admits a prime ideal factorization.
For a prime ideal $v \subset \mathcal{O}_{F}$, regarding it as an ideal $\mathcal{O}_{K}$, it splits as a
product of prime ideals in $\mathcal{O}_{K}$ as $v = w_{1}\cdots w_{g}$.
Then $\mathcal{O}_{F}/v \subset \mathcal{O}_{K} / w_{j}$ is a finite extension of finite fields, so is cyclic.
Although we can't directly link $\Gal(K/F)$ with $\Gal((\mathcal{O}_{K}/w_{j})/(\mathcal{O}_{F}/v))$, there exists
a \emph{decomposition group} $D_{w_{j}} \subset \Gal(K/F)$ defined as
$$
    D_{w_{j}}:=\{g \in \Gal(K/F)\,:\, gw_{j} = w_{j}\} \xrightarrow{\alpha_{w_{j}}} \Gal((\mathcal{O}_{K}/w_{j})/(\mathcal{O}_{F}/v))
$$
where $\alpha_{w_j}$ is surjective.
We also define \emph{inertia subgroup} $I_{w_{j}}$ as $\ker \alpha_{w_{j}}$, so that $D_{w_{1}}/I_{w_{1}} \simeq \mathbb{Z}/n\mathbb{Z}$
for some $n$.
Now, when $I_{w_{j}} = 1$, we have $D_{w_{j}} \simeq \mathbb{Z}/n\mathbb{Z}$ and we can define a Frobenius
conjugacy class in $\GL_{n}(\overline{\mathbb{Q}_{\ell}})$ by composing the isomorphism with $\sigma|_{D_{w_{j}}}$.
It is known that $I_{w_{j}} = 1$ for all but finitely many $v$ (we call such $v$ \emph{unramified}),
and since different choices of $w_{j}$ gives conjugated decomposition groups,
the Frobenius conjugacy class $\sigma(\mathrm{Fr}_{w_{j}})$ does not depend on the choice of $w_{j}$ and only on $v$. 
\newpage
\section{Classical Langlands correspondence over function fields (September 1)}


We are going to explain classical Langlands correspondence over function fields in (more) detail.
Let $X$ be a smooth, geometrically irreducible, projective curve over $\mathbb{F}_{q}$ and $F = \mathbb{F}_{q}(X)$ be a function field.
Let $|X|$ be a set of closed points of $X$, which has a 1-1 correspondence with $\mathscr{V}$ - the set of
places (completions) of $F$.
Recall that the completion $F_{x}$ at $x \in |X|$ is isomorphic to $(\mathbb{F}_{q})_{x}((t_{x}))$, where 
$(\mathbb{F}_{q})_{x}$ residue field at $x$ and $t_{x}$ is a rational functino on $X$ with order 1 zero at $x$
(In other words, it is a generator of maximal ideal $\mathfrak{m}_{x}$ corresponds to $x$).
Then we have a ring of integer $\mathcal{O}_{x} \subset F_{x}$ isomorphic to the ring of
formal power series $(\mathbb{F}_{q})_{x}[[t_{x}]]$.
We also defined the ad\'ele ring $\mathbb{A}_{F}$ for $F$.

Now we define the \emph{Weil group} $W(\overline{F}/F)$ as follows.
Let $\overline{F}$ be a (separable) algebraic closure of $F$, then we have the action 
of $\Gal(\overline{F}/F)$ on the subfield $\overline{\mathbb{F}_{q}}$ (the field of constants) that fixes $\mathbb{F}_{q}$.
Then we have a surjective map 
$$\Gal(\overline{F}/F) \xrightarrowdbl{\mathrm{res}} \Gal(\overline{\mathbb{F}_{q}}/\mathbb{F}_{q})$$
and the latter group is an inverse limit of Galois groups of finite extensions of $\mathbb{F}_{q}$, so
$$
\Gal(\overline{\mathbb{F}_{q}}/\mathbb{F}_{q}) \simeq \varprojlim \Gal(\mathbb{F}_{q^{n}} / \mathbb{F}_{q}) \simeq \varprojlim \mathbb{Z}/n\mathbb{Z} =: \widehat{\mathbb{Z}},
$$
which is the profinite completion of $\mathbb{Z}$.
It is topologically generated by Frobenius automorphism $\mathrm{Fr}$, and it has a subgroup isomorphic to $\mathbb{Z}$ generated (not topologically, but just algebraically) by $\mathrm{Fr}$.
Then we define the \emph{Weil group} $W(\overline{F}/F)$ as an inverse image of $\mathbb{Z} \simeq \langle \mathrm{Fr} \rangle \subset \Gal(\overline{\mathbb{F}_{q}} / \mathbb{F}_{q})$
of restriction map, which is a subgroup of $\Gal(\overline{F}/F)$.
For Galois side of Langlands correspondence over function field, we are going to consider irreducible representations of
$W(\overline{F}/F)$ instead of $\Gal(\overline{F}/F)$.
More precisely, we consider the (equivalence classes of) irreducible $n$-dimensional $\ell$-adic representations of $W(\overline{F}/F)$,
$$
\sigma: W(\overline{F}/F) \to \GL_{n}(\overline{\mathbb{Q}_{\ell}})
$$
such that 
\begin{enumerate}
    \item Image of $\sigma$ in $\GL_{n}(\overline{\mathbb{Q}_{\ell}})$ is in $\GL_{n}(E)$ for some finite extension $E/\mathbb{Q}_{\ell}$.
    \item $\sigma$ is continuous where $W(\overline{F}/F)$ is given Krull topology (profinite toppology) and $\GL_{n}(E)$ is given subspace topology of $M_{n}(E)$.\footnote{
        This explains somehow why we are considering $\ell$-adic representations instead of complex representations.
        As a toy example, consider continuous 1-dimensional complex representations of $(\mathbb{Z}_{\ell}, +)$, i.e. an additive character $\sigma: \mathbb{Z}_{\ell} \to \GL_{1}(\mathbb{C}) = \mathbb{C}^{\times}$.
        Then it should factor through $\mathbb{Z}_{\ell} / \ell^{n}\mathbb{Z}_{\ell}$ for some $n$, so that the image is always finite.
        However, if we consider $\ell$-adic characters $\sigma:\mathbb{Z}_{\ell} \to \mathbb{Q}_{\ell}^{\times}$, then there are non-trivial characters with infinite image, e.g. $x \mapsto \exp_{\ell}(\ell x)$ 
        where $\exp_{ell}$ is an $\ell$-adic exponential function.
    }
    \item $\sigma$ is unramified for all but finitely many $x \in |X|$.
    Note that the unramifiedness is defined using decomposition group and inertia group as before.
\end{enumerate}

On the automorphic side, we wiil explain cuspidality and unramifiedness in more detail.
The space of cusp forms $\mathrm{L}^{2}_{\cusp}(\GL_{n}(F) \backslash \GL_{n}(\mathbb{A}_{F}), \chi)$\footnote{Here $\chi$ is a continuous unitary character on center $Z(\mathbb{A}_{F})$ trivial on $Z(F)$,
and $\mathrm{L}^{2}(\GL_{n}(F)\backslash\GL_{n}(\mathbb{A}_{F}), \chi)$ is a space of functions where the center acts as the character $\chi$.
}
are functions satisfying the following vanishing condition: for $0 < n_{1}, n_{2} < n$ with $n = n_{1} + n_{2}$,
we have
$$
\int_{N_{n_{1}, n_{2}}(F) \backslash N_{n_{1}, n_{2}}(\mathbb{A}_{F})} f(ng)dn = 0
$$
for all $g\in \GL_{n}(\mathbb{A}_{F})$, where $N_{n_{1}, n_{2}} < \GL_{n}$ is the unipotent group of matrices of the form
$$
\begin{pmatrix}
    I_{n_{1}} & * \\ \mathbf{0} & I_{n_2}
\end{pmatrix}
$$
Note that non-example of cuspidal represenetation is Eisenstein series representation, which is obtained from
two representations $\pi_{1}, \pi_{2}$ of $\GL_{n_{1}}(\mathbb{A}_{F})$ and $\GL_{n_2}(\mathbb{A}_F)$ respectively, by inflation and (parabolic induction).
Then it is a theorem (from Flath) that any irreducible cuspidal represenetations of $\GL_{n}(\mathbb{A}_F)$ decomposes as restricted product
of local representations,
$$
\pi \simeq \bigotimes_{x\in |X|} \pi_{x}
$$
where each $\pi_{x}$ are irreducible representation of $\GL_{n}(F_x)$.
In this case, for all but finitely many $x$, $\GL_{n}(\mathcal{O}_{x})$-fixed subspace
$\pi^{\GL_{n}(\mathcal{O}_x)}$ is non-trivial and one-dimensional.
We call that $\pi$ is \emph{unramified at $x$} for such $x$.
For $x$ where $pi$ is unramified, we have a represenetation of \emph{spherical Hecke algebra $\mathcal{H}_{x}$}, which 
is a sub-algebra of compactly supported functions on $\GL_{n}(F_{x})$ that are $\GL_{n}(\mathcal{O}_{x})$-biinvariant.
Then $\mathcal{H}_{x}$ is a convolution algebra which is commutative and 
isomorphic to $\mathbb{C}[x_{1}^{\pm}, \dots, x_{n}^{\pm}]^{S_{n}}$
and corresponds to semisimple conjugacy classes in $\GL_{n}(\mathbb{C})$, which we will denote $\pi(h_{x})$.

Also, as in the case of Galois side, we impose some conditions on the automorphic side.
We will only consider automorphic representations of $\GL_{n}(\mathbb{A}_{F})$ with some finiteness conditions, i.e.
for any compact subgroup $K$ of $\GL_{n}(\mathbb{A}_F)$, the translates of any $f\in \pi$ span a
finite dimensional vector space.

Then the Langlands correspondence becomes as follows.
It is a 1-1 correspondence between the irreducible $\ell$-adic $n$-dimensional representations of Weil group $W(\overline{F}/F)$
(with some conditions) and irreducible cupsidal automorphic representations of $\GL_{n}(\mathbb{A}_{F})$ (with some conditions).
The invariants, Frobenius conjugacy classes $\{\sigma(\mathrm{Fr}_{x})\}$ on the Galois side, matches with
the Hecke conjugacy classes $\{\pi(h_{x})\}$, for all $x \not \in S_{\sigma} \cup S_{\pi}$.
Here $S_{\sigma}$ (resp. $S_{\pi}$) is the set of unramified places for $\sigma$ (resp. $\pi$), and
we actually have $S_{\sigma} = S_{\pi}$ for corresponding $\sigma - \pi$ pairs.
\newpage
\section{More on Hecke algebra and Langlands correspondence for general reductive groups (September 6)}

We said that Langlands correspondence for $\GL_{n}$ gives a correspondence between invariants,
which are the Frobenius conjugacy classes (on Galois side) and the Hecke conjugacy classes (on automorphic side).
We are going to explain about Hecke conjugacy classes more in detail.

Let $F = \mathbb{F}_{q}(X)$ be a function field for a curve over finite field.
Let $\pi$ be an automorphic representation of $\GL_n(\mathbb{A}_F)$.
It decomposes as a restricted product of local representations of $\GL_{n}(F_x)$ as $\pi = \otimes'_{x\in |X|}\pi_x$.
Then there exists a finite set of (closed) points $S_\pi \subset |X|$ such that
$\pi_{x}^{\GL_{n}(\mathcal{O}_x)}\neq 0$, i.e. there exists $\GL_{n}(\mathcal{O}_{x})$-fixed vector in $\pi_x$.
We define Hecke algebra $\mathcal{H}_x$ as a convolution algebra on the set of 
compactly supported $\GL_{n}(\mathcal{O}_{x})$-bi-invariant functions with Haar measure on $\GL_n(F_x)$
normalized by $\mu(\GL_n(\mathcal{O}_x))=1$.
Then the Hecke algebra is actually commutative, and using this we can show that $\pi_x^{\GL_n(\mathcal{O}_x)}$ is actually 1-dimensional,
i.e. there exists a unique vector (up to scaling) $v_x \in \pi_x$ fixed by $\GL_n(\mathcal{O}_x)$.
The restricted product $\otimes_{x\in |X|}' \pi_x$ of local representations are defined as a span of vectors $\otimes_x w_x$
where $w_x\in \pi_x$ and $w_x = v_x$ for all but finitely many $x$.
In this case, the group $\GL_n(\mathbb{A}_F) = \otimes_x' \GL_n(F_x)$ acts on the space componenti-wise
$g.\otimes w_x := \otimes_x (g_x.w_x)$
and the previous argument this actually gives an action on the space $\otimes_{x}'\pi_x$.

Now for given local representation $\pi_x$, we can attach a representation of the Hecke algebra $\mathcal{H}_x$
where the representation space is $\pi_x^{\GL_n(\mathcal{O}_x)}$ action is given by (we use the same notation $\pi_x$ for the representation of $\mathcal{H}_x$)
$$
f \mapsto \pi_x(f): v \mapsto \int_{\GL_n(F_x)} f(g) \pi_x(g)v dg.
$$
The integral is well-defined since $f$ is compactly supported and $\GL_n(\mathcal{O}_x)$ preserves $\pi^{\GL_n(\mathcal{O}_x)}$.
This gives a functor from the category of representations of $\GL_n(F_x)$ and the category of representations of $\mathcal{H}_x$.
In fact, this sets up bijection between irreducible unramified representations of $\GL_n(F_x)$
and irreducible representations of $\mathcal{H}_x$.
Since $\mathcal{H}_x$ is commutative, irreducible representations of $\mathcal{H}_x$ are just characters of $\mathcal{H}_x$,
which we denote it as $\chi_x: \mathcal{H}_x \to \mathbb{C}$.

We can describe the structure of $\mathcal{H}_x$ more precisely as follows, which is a special case of so-called \emph{Satake isomorphism}.

\begin{theorem}
    $$
    \mathcal{H}_x \simeq \mathbb{C}[x_1^{\pm}, \dots, x_n^{\pm}]^{S_n}
    $$
    where the RHS is a space of symmetric Laurent polynonmials in $n$-variables.
\end{theorem}
Here's a sketch of proof.
First, the double coset space $\GL_n(\mathcal{O}_x) \backslash \GL_n(F_x) / \GL_n(\mathcal{O}_x)$ can be identified with $\mathbb{Z}^{n}/S_n$ as follows.
We have a map 
$$
\mathbb{Z}^{n} \to \GL_n(\mathcal{O}_x) \backslash \GL_n(F_x) / \GL_n(\mathcal{O}_x)
$$
that maps $(\lambda_1, \dots, \lambda_n) \in \mathbb{Z}^{n}$ to the double coset of the diagonal matrix
$$
\begin{pmatrix}
    t_{x}^{\lambda_1} & & & \\
    & t_{x}^{\lambda_@} & & \\
    & & \ddots & \\
    & & & t_{x}^{\lambda_n}
\end{pmatrix}
$$
where $t_x$ is a choosen uniformizer of $F_x$, so that $F_x \simeq (\mathbb{F}_q)_x((t_x))$ 
and $\mathcal{O}_x \simeq (\mathbb{F}_{q})_x[[t_x]]$.
Such a map is well-defined in the sense tha the double coset is independent of the choice of uniformizer $t_x$ - any other choice $t_x'$ satisfies $t_x'/t_x \in \mathcal{O}_x$.
The map is surjective, and it factors through $\mathbb{Z}^{n}/S_n$ with permutation on $\mathbb{Z}^n$ since
$(\lambda_1, \dots, \lambda_n)$ and $(\lambda_1', \dots, \lambda_n')$ with $\tau\lambda = \lambda'$ maps to the diagonal matrices
that are conjugate to each other by permutation matrix (corresponds to $\tau\in S_n$).

Using the identification, we can get the Satake isomorphism as follows.
Any $f \in \mathcal{H}_{x}$ as a form of $\sum_{\lambda} a_{\lambda} c_{\lambda}$
where $c_{\lambda}$ is a characteristic function on a double coset corresponds to an unordered set $\lambda = (\lambda_1, \dots, \lambda_n)$.
This sum is a finite sum since $f$ is compactly supported.
And the function $f$ corresponds to a symmetrized Laurent polynomial
$\sum_{\tau\in S_n} \tau(\sum a_{\lambda}x_1^{\lambda_1}\cdots x_{n}^{\lambda_n})$.

As we said before, Langlands correspondence for $\GL_n$ over a global function field is now a theorem.
\begin{theorem}[Deligne, Lafforgue]
   Let $F$ be a function field of a curve over a finite field.
   There is such a bijection between $n$-dimensional continuous irreducible representations of the Weil group $W(\overline{F}/F)$
   (with some technical conditions mentioned before) and irreducible cuspidal automorphic representations of $\GL_n(\mathbb{A}_F)$.
   This also gives a bijection between the Frobenius conjugacy classes and the Hecke conjugacy classes, and also these conjugacy classes
   are actually in $\GL_n(\overline{\mathbb{Q}})$.
\end{theorem}
$n=2$ case is proven by Delign, and $n > 2$ is by L. Lafforgue.
The difference between two cases are on the existence of \emph{nice} moduli space.
Lafforgue invented objects called \emph{Shutuka} and use moduli space of them for the cases $n >2$.


How can we state the Langlands correspondence for general reductive group $G$ (over function fields)?
First, the concepts we defined (Galois representations, automorphic representations, Hecke algebra, Frobenius conjugacy classes, ...)
generalizes to general reductive groups.
For example, Hecke algebra $\mathcal{H}_{x}$ is a convolution algebra of compactly supported functions on $G(F_x)$
which is bi-invariant under $G(\mathcal{O}_{x})$.
Then it is a commutative algebra for $x \in |X|$ with $\pi_x^{G(\mathcal{O}_x)}\neq 0$ (unramified), and
the invariant subspace $\pi_{x}^{G(\mathcal{O}_{x})}$ is 1-dimensional.
Also, we have a Satake isomorphism for $G$.

But if $\chi_x : \mathcal{H}_x \to \mathbb{C}$ is a character of $\mathcal{H}_{x}$, then where the corresponding
conjugacy class would live in?
For general reductive group $G$, Hecke conjugacy classes does not live in $G(\mathbb{C})$, but in different group called \emph{dual group} of $G$.
To define the notion of dual group, we have to define a \emph{root datum} first.

Let $T$ be a maximal torus of $G$, i.e. maximal commutative subgroup of $G$.
For $\GL_n$, it is a set of diagonal matrices.
For now, we will regard $G$ and $T$ as a group over $\mathbb{C}$.
Then we define the latteices of characters and cocharacters of $T$ as
\begin{align*}
X^{*}(T) = \mathrm{Hom}(T, \mathbb{G}_{m}) \\
X_{*}(T) = \mathrm{Hom}(\mathbb{G}_{m}, T)
\end{align*}
which are free abelian groups of finite rank.
We have a pairing $X^{*}(T)\times X_{*}(T) \to \mathbb{Z}$ defined by composition (note that all the morphisms $\mathbb{G}_{m} \to \mathbb{G}_{m}$
has a form of $x \mapsto x^{n}$).
We also have roots $\Delta \subset X^{*}(T)$ and coroots $\Delta ^{\vee} \subset X_{*}(T)$
which are nonzero eigenvalues of adjoint action of $T$ (and their duals).
Then we can associate a quadruple $(X^{*}(T), X_{*}(T), \Delta, \Delta^{\vee})$, a \emph{root datum} of $G$,
and it determines a group $G$ upto isomorphism when $G$ is split (i.e. admits a split maximal torus).
By simply flipping a root datum, we get another root datum
$$
(X_*(T), X^{*}(T), \Delta^{\vee}, \Delta)
$$
called dual root datum, and the group determined by this new root datum is called the \emph{dual} group of $G$, denoted by $\widehat{G}$.

Now let's get back to the Hecke algebra.
In case of $\GL_{n}$, we have a map $\mathbb{Z}^{n} \to \GL_2(\mathcal{O}_{x}) \backslash \GL_2(F_{x}) / \GL_2(\mathcal{O}_x)$ 
that induces an isomorphism 
$$
\mathbb{Z}^{n}/S_{n} \to \GL_2(\mathcal{O}_x) \backslash \GL_2(F_x) / \GL_2(\mathcal{O}_x).
$$
For general reductive group $G$, we have a map $X_{*}(T) \to G(\mathcal{O}_x) \backslash G(F_x) / G(\mathcal{O}_x)$ defined as an evaluation of character at uniformizer $t_{x}$.
Then this map factors through the quotient of $X_{*}(T)$ by the \emph{Weyl group} $W = W(G, T)$, the symmetry group of a root datum.
This induces an isomorphism 
$$
X_{*}(T) / W \simeq G(\mathcal{O}_{x}) \backslash G(F_x) / G(\mathcal{O}_x)
$$
which gives $\mathcal{H}_{x} \simeq \mathbb{C}[X_{*}(T)/W]$.
This saids that the character of $\mathcal{H}_x$ corresponds to an element in $X_{*}(T)/W$.
In this case, which kind of a conjugacy would corresponds to the element?
In fact, we have a canonical isomorphism $\mathbb{C}[X^{*}(T)/W] \simeq \mathbb{C}[T]^{W}$.
which gives a correspondence between $X^{*}(T)/W$ and semisimple conjugacy class in $G$.
However, we have $X_{*}(T)/W$ instead, and since $X_{*}(T)$ is a character group of the dual group $\widehat{G}$, 
we can concludes that $X_{*}(T)/W$ corresponds to semisimple conjugacy classes in $\widehat{G}$.

Based on this observation, the Langlands correspondence for a reductive group $G$ (over a function field) 
is a correspondence between
\begin{align*}
    \boxed{
        \text{irreducible representation } \sigma:W(\overline{F}/F) \to \widehat{G}(\overline{\mathbb{Q}_{\ell}})
    }
\end{align*}
and
\begin{align*}
    \boxed{
        \text{irreducible cuspidal automorphic representation of }G(\mathbb{A}_F)
    }
\end{align*}
where the invariants match: the Frobenius conjugacy classes 
$$
[\sigma(\mathrm{Fr}_{x})] \subset \widehat{G}(\overline{\mathbb{Q}_{\ell}}), \quad x \in |X| \backslash S_{\sigma}
$$
and the Hecke conjugacy classes (corresponds to characters of Hecke algebra $\mathcal{H}_x$)
$$
\pi(h_x) \subset \widehat{G}(\mathbb{C}),\quad x \in |X| \backslash S_\pi
$$
with a choice of identification $\iota: \mathbb{C} \simeq \overline{\mathbb{Q}_{\ell}}$, where $S_\sigma = S_\pi$ is a finite subset of $|X|$.
\newpage
\section{Satake isomorphism (September 8)}

% \subsection{Preliminaries on affine group schemes}
% Before we rigorously state the Satake isomorphism, we'll go through some basics on affine group schemes.


\subsection{Satake isomorphism}
Recall that for given cocharacter $\lambda \in X_{*}(T)$, we can associate an element in $T(F_x)$ 
by evaluating $\lambda$ at chosen uniformizer $t_x \in F_x$.
Then it defines a well-defined double coset in $G(\mathcal{O}_x) \backslash G(F_x) / G(\mathcal{O}_x)$
since the choice of uniformizer is only differ by $\mathcal{O}_x^{\times}$.
Now we can state the Satake isomorphism for genral (split) reductive groups.
\begin{theorem}
We have an isomorphism of $\mathbb{C}$-algebra
$$
\mathcal{H}_{x} \simeq \mathbb{C}[X_{*}(T)]^{W} \simeq \mathbb{C}[X^{*}(\widehat{T})]^{W}.
$$
\end{theorem}
As a consequence, Hecke algebras are commutative.
\begin{proof}
Satake provided explicit isomorphism from $\mathcal{H}_{x}$ to $\mathbb{C}[X_{*}(T)]^{W}$, which is
$$
f \mapsto \sum_{\lambda \in X_{*}(T)} \left(q_{x}^{\langle \rho, \lambda\rangle} \int_{N(F_x)}f(u \lambda(t_x))du \right) \lambda
$$
Here $\rho = (1/2)\sum_{\alpha \in \Delta^{+}}\alpha$ is the half-sum of the positive roots
and $du$ is the Haar measure on $N(F_x)$ normalized as $du(N(\mathcal{O}_x)) = 1$.
Since $f \in \mathcal{H}_x$ is compactly supported, the image under Satake's map is actually a finite sum.
He showed that this is an injective algebra homomorphism whose image is $\mathbb{C}[X_*(T)]^W$.
\end{proof}
Now, observe that we have following isomorphisms
$$
\mathbb{C}[X_{*}(T)]^{W} \simeq \mathbb{C}[T]^{W} \simeq \mathbb{C}[G]^{G}
$$
where the last space is the space of class functions on $G$, i.e. functions on $G$ that are invariant under conjugation.
The second isomorphism comes from Chevalley's restriction theorem.
Hence the characters of $\mathcal{H}_x$ correspond to the characters of $\mathbb{C}[G]^{G}$, which again correspond to
the semisimple conjugacy classes in $G$.
This is the way how we associate ``Hecke conjugacy class $\pi(h_x)$''\footnote{Usually, it is called as \emph{Satake parameters}.} to an automorphic representation $\pi$.

\subsection{Why Weil group?}
On the Galois side, why do we care about representations of Weil groups, not the whole Galois group?
The reason can be found from the $n=1$ case, i.e. Langlands correspondence for $\GL_1 = \mathbb{G}_m$, which is also
called \emph{Abelian Class Field Theory}.

Let $\rho : \Gal(\overline{F}/F) \to \GL_1(\overline{\mathbb{Q}}_\ell) = \overline{\mathbb{Q}_{\ell}}^{\times}$
be an irreducible representation.
Since $\GL_1$ is abelian, the map factors through the abelianization of $\Gal(\overline{F}/F)$, which equals
$\Gal(F^{\mathrm{ab}}/F)$, the Galois group of maximal abelian extension.
On the automorphic side, (cuspidal) automorphic representations of $\GL_1$ are essentially the characters of $\GL_1(F) \backslash \GL_1(\mathbb{A}_F) = F^{\times} \backslash \mathbb{A}_F^{\times}$.
Then the Langlands correspondence is given through Artin reciprocity map
$$
\theta: F^{\times} \backslash \mathbb{A}_F^{\times} \to \Gal(F^{\mathrm{ab}} / F).
$$
By the way, this map is injective but not surjective, and the image is exactly the ablianization of Weil group, $W(F^{\mathrm{ab}}/F)$.
Hence, generalizing this idea for $n > 1$ makes us to consider representations of Weil group instead of the
full absolute Galois group.
\newpage
\section{Glimpse on analytic langlands (September 13)}

Now, our goal is to formulate Langlands correspondence in geometric terms, so that we could
find their analogues when a curve $X$ is over $\mathbb{C}$, rather than a finite field $\mathbb{F}_q$.
For simplicity, we'll assume that everything is unramified, i.e. the ``Galois representations'' and ``automorphic representations''
are unramified everywhere ($S_\sigma = S_\pi = \emptyset$), whatever it means.

On the Galois side, our slogan is the following: the geometric object corresponds to 
Galois group is essentially the (\'etale) \emph{fundamental group} of $X$.
In this case, extensions of a function field corresponds to (\'etale) coverings $Y \to X$.
For example, if we have a curve $X$ (over a base field $k = \mathbb{C}$ or $\mathbb{F}_q$) and a covering $Y \to X$,
then we get a field extension $k(X) \hookrightarrow k(Y)$.
Then the Galois group $\Gal(k(Y) / k(X))$ is essentially the group of deck transformations of this covering.
Hence the Galois group of maximal unramified extension of $F = k(X)$, $\Gal(F^{\mathrm{un}}/F)$,
is the group of deck transformations of maximal unramified cover $\widetilde{X}$, which is a fundamental group $\pi_1(X)$ of $X$.

% Recall that, for topological spaces and their (universal) coverings, fundamental group is defined with a base point

Now, let $X$ be a smooth projective connected algebraic curve over $\mathbb{C}$, or in other words, 
a compact Riemann surface.
Then our analogy for the Galois representations becomes
\begin{align*}
    \boxed{
        \text{Equivalence class of } \sigma: \pi_1(X) \to \widehat{G}(\overline{\mathbb{Q}_{\ell}})
    }
\end{align*}
Now, such a collection has a correspondence with flat connections:
\begin{align*}
    \boxed{
        (E, \nabla): C^{\infty} \text{ principal } \widehat{G} \text{-bundles with } C^{\infty}\text{ flat connection}
    }
\end{align*}
where $C^{\infty}$ principal $G$-bundle is a principal $G$-bundle where transition maps are smooth.
By decomposing it into holomorphic and anti-holomorphic part ($\nabla = \nabla^{(1, 0)} + \nabla^{(0, 1)}$), one can see that this also has a correspondence with
\begin{align*}
    \boxed{
        (E^{\mathrm{hol}}, \nabla^{(1, 0)}): \text{ flat holomorphic $\widehat{G}$-bundle with a holomorphic connection $\nabla^{(1, 0)}$}
    }
\end{align*}
Note that, on a curve, any such connection becomes automatically flat.

For example, let $X = \mathbb{C}$ and consider trivial $\GL_1 = \mathbb{G}_{m}$-bundle on $X$, $\mathbb{C}^{\times} \times \mathbb{C}$.
Let $\nabla^{(1, 0)} = \frac{\partial}{\partial z} + \frac{\lambda}{z}$ be a connection on $X$.
Then the solution of $\nabla^{(1, 0)}\phi = 0$ is $\phi(z) = z^{-\lambda} = e^{-\lambda \log z}$, 
and this gives a monodromy action where a generator of $\pi(\mathbb{C}^{\times}, 1) \simeq \mathbb{Z}$ acts as multiplication by $e^{-2\pi i \lambda}$.

Now let's move on to the automorphic side.
Recall that automorphic representation $\pi$ can be thought as a space of compactly supported
smooth (i.e. locally constant) functions on $G(F)\backslash G(\mathbb{A}_F)$.
If it is unramified everywhere, then its $G(\mathcal{O}_F) = \prod_{v}' G(\mathcal{O}_{F_v})$-invariant subspace 
$\pi^{G(\mathcal{O}_F)}$ is a subspace of functions on a double coset 
$$
G(F) \backslash G(\mathbb{A}_F) / G(\mathcal{O}_F).
$$
Upshot is, the above double coset space is on bijection with the set of isomorphism lasses of principal (holomorphic or algebraic) $G$-bundles on $X$.
For example, in case of $G = \mathrm{GL}_1$, the double coset space is
$$
F^{\times} \backslash \mathbb{A}_F^{\times} / \sideset{}{'}\prod_{x} \mathcal{O}_{F_x}^{\times} \simeq F^{\times} \backslash \sideset{}{'}\prod_{x}(F_{x}^{\times} / \mathcal{O}_{F_{x}}^{\times}) \simeq F^{\times} \backslash \sideset{}{'}\prod_{v}\mathbb{Z}.
$$
On the other hand, $\GL_1$-bundles on $X$ are the same as line bundles on $X$, 
and this again corresponds to (Weil) divisors on $X$.
Divisor is just a formal $\mathbb{Z}$ combination of closed points on $X$.
To divisors are equivalent when the difference is principal divisor, i.e. divisor of a form
$$
\mathrm{div}(f):= \sum_{x \in |X|} \mathrm{ord}_{x}(f)\cdot [x]
$$
and the set of equivalence classes of divisors is in bijection with the divisor class group, which is
$$
\mathrm{Cl}(X) = \mathrm{Div}(X) / \mathrm{PDiv}(X)
$$
(quotient of group of divisors by group of principal divisors).
One can find natural bijection between $\mathrm{Cl}(X)$ and the above double coset space.

\newpage
\section{On vector bundles and flat connections (September 15)}


Let $X$ be a smooth real manifold.
A (complex) rank $n$ vector bundle $\mathcal{P}$ on $X$ is a manifold with projection map $\mathcal{P} \to X$ such that
there exists a covering $\{ U_{\alpha}\}$ of $X$ with local trivializations $u_{\alpha}: \mathcal{P}|_{U_\alpha} \to U_\alpha \times \mathbb{C}^{n}$.
On the intersections $U_\alpha \cap U_\beta$, we have two trivializations
$u_\alpha|_{U_\alpha \cap U_\beta}$ and $u_\beta|_{U_\alpha \cap U_\beta}$, and this gives $\GL_n(\mathbb{C})$-valued functions
$g_{\alpha\beta}$ comes from $u_{\beta}u_{\alpha}^{-1}: (U_\alpha \cap U_\beta) \times \mathbb{C}^n \to (U_\alpha\cap U_\beta)\times\mathbb{C}^n$.
We assume that such a transition function is smooth. Also, on triple intersections $U_\alpha \cap U_\beta \cap U_\gamma$, 
we assume that transition functions are compatible, in the sense that they satisfy cocycle conditions $g_{\alpha\gamma} = g_{\beta\gamma}g_{\alpha \beta}$.
Hence giving a vector bundle is equivalent to giving a covering $\{U_\alpha\}$ and smooth transition functions $\{g_{\alpha\beta}\}$
satisfying cocycle conditions.

We call that a vector bundle $\mathcal{P}$ is \emph{flat} if $g_{\alpha\beta}:U_{\alpha}\cap U_{\beta}\to \GL_n(\mathbb{C})$is a \emph{constant} function.
It means that we have an identification of all nearby fibers $\mathcal{P}_{x} \simeq \mathcal{P}_{x'}$ for all $x, x'\in U_\alpha$.
Equivalent way to say this is using a \emph{flat connection}.
Connection is a way of differentiating sections of vector bundles.
If a vector bundle is trivial, then this corresponds to an ordinary differentiation of a vector-valued function on $X$.
Hence we can do the same thing for general vector bundles locally, but this may depends on the choice of local trivializations.

\emph{Connection} is a map between between a sheaf of smooth vector fields on $X$ and a sheaf of smooth sections
of $\mathrm{End}(\mathcal{P})$, satisfying some linearity and Leibniz rules.
More precisely, it is a map
$$
\nabla: \mathcal{T}\to \mathcal{E}nd_{\mathbb{C}}(\mathcal{P})
$$
with $\xi \mapsto \nabla_{\xi}$, such that 
\begin{enumerate}
    \item ($\mathcal{O}_X$-linear) $\nabla_{\xi + \eta} = \nabla_{\xi} + \nabla_{\eta}$ and $\nabla_{f\xi} = f\nabla_{\xi}$
    for all $f \in \mathcal{O}_{X}$ and $\xi \in \mathcal{T}$.
    \item (Leibniz rule) $\nabla_{\xi}(f\cdot \phi) = f\cdot \nabla_{\xi}(\phi) + (\xi\cdot f)\phi$
    for all $f\in\mathscr{O}_X, \xi \in\mathcal{T}$, and $\phi \in \mathcal{P}$.
\end{enumerate}
A connection is called \emph{flat} if $\nabla$ is a Lie algebra homomorphism, i.e. 
$$
\nabla_{[\xi, \eta]} = [\nabla_{\xi}, \nabla_{\eta}].
$$
In other words, the connection form $R_{\xi\eta}:= \nabla_{[\xi, \eta]} - [\nabla_{\xi}, \nabla_{\eta}]$ is identically 0.
If $U\subset X$ is an open subset fih local coordinates $x_1, \dots, x_k$ and trivialization $\mathcal{P}|_U \simeq U \times \mathbb{C}^n$, it can be written as
$$
\nabla_{\frac{\partial}{\partial x_i}} = \frac{\partial}{\partial x_i} + A_i
$$
fomr some $\mathrm{M}_n(\mathbb{C})$-valued function $A_i$.
Then flatness of $\nabla$ is equivalent to 
$$
\frac{\partial A_{j}}{\partial x_{i}} - \frac{\partial A_{i}}{\partial x_{j}} + [A_i, A_j] =0
$$
for all $i, j$.

Let $\mathcal{S}_U$ be a set of all trivializations of $\mathcal{P}|_U$.
Then $\mathcal{G}_U:=\{f:U \to \GL_n(\mathbb{C})\}$ acts simply transitively on $\mathcal{S}_U$ (i.e. $\mathcal{S}_U$
is $\mathcal{G}_U$-torsor) via conjugation:
$$
g\left(\frac{\partial}{\partial x_i} + A_i\right)g^{-1} = \frac{\partial}{\partial x_i} + gA_i g^{-1} - \left(\frac{\partial g}{\partial x_i}\right)g^{-1},
$$
we call $A_i \mapsto gA_i g^{-1} - \left(\frac{\partial g}{\partial x_i}\right)g^{-1}$ as \emph{gauge} transform.

One can prove that flatness of a vector bundle is equivalent to imposing a flat connection on the vector bundle.
For example, if a flat connection $\nabla$ of $\mathcal{P}$ is given, then we can obtain local identification of nearby fibers
by solving a differential equation $\nabla \phi = 0$ locally, and connect from $p \in \mathcal{P}_{x}$ to $p' \in \mathcal{P}_{x'}$ along $\phi$.
By the theory of PDE, such $\phi$ (called \emph{horizontal section}) always uniquely exists (locally) for given initial condition.
Also, this gives a monodromy action of fundamental group.
For $p \in \mathcal{P}_x$ and a loop $\gamma$, we can ``solve'' $\nabla \phi = 0$ locally along a loop
and coming back to $x$, which gives $\rho_{\mathcal{P}}: \pi_1(X, x) \to \GL_n(\mathbb{C})$.
Since changing initialization gives conjugated representation, and we get a bijection between
equivalence classes of flat vector bundles of rank $n$ and equivalence classes of representations $\pi_1(X, x) \to \GL_n(\mathbb{C})$.

\newpage
\section{Principal $G$-bundles and topologies (September 20)}

Let $G$ be a reductive group over $\mathbb{C}$, and let $\mathcal{P}$ be a principal
$G$-bundle on a curve $X$ over $\mathbb{C}$ (which is also a $G$-torsor on $X$).
Tannaka's theorem tells us that we can reconstruct $G$ from its category of representations $\mathsf{Rep}_{G}$
which is naturally a tensor category (for given two representations of $G$, we have a tensor product representation of them).
Using this, we can prove that there's a natural bijection between 
\begin{align*}
    \boxed{
        \text{principal $G$-bundles on $X$ (w.r.t. particular topology) with connection}
    }
\end{align*}
and
\begin{align*}
    \boxed{
        \text{Tensor functors $\mathsf{Rep}_{G}(\mathbb{C}) \to \mathsf{VecBun}^{\nabla}(X)$ (w.r.t. same topology)}
    }
\end{align*}
where the map is given by
$$
(\mathcal{P}, \nabla) \mapsto \mathcal{F}_{\mathcal{P}}: (\rho, V) \mapsto \mathcal{P} \times V
$$
where $\mathcal{P}$ acts on $V$ via $\rho$.
Note that tensor product on the category $\mathsf{VecBun}^{\nabla}$ (vector bundle with connection) is defined as
$$
(V, \nabla_V) \otimes (W, \nabla_W) := (V \otimes W, \nabla_V \otimes \mathrm{id}_W + \mathrm{id}_V \otimes \nabla_W).
$$

Last week, we gave a correspondence between (flat) holomorphic/algebraic $\hat{G}$-bundles on $X$ with 
a double coset space $G(F)\backslash G(\mathbb{A}_F) / G(\mathcal{O}_F)$, for everywhere-unramified cases.
What if we allow ramifications? What kind of geometric objects would correspond to the flat $\hat{G}$-bundles?
Before we give an answer, we need to specify what kind of topology are we going to impose on $X$.

In algebraic geometry, there's a versatile notion of \emph{Grothendieck topology}
in which open set $U\subset X$ is replaced by morphisms $U \to X$ of certain type.\footnote{To be more precise, Grothendieck topology is a topology
on a certain category of morphisms associated to $X$}
For example, Grothendieck topology defined with \'etale morphisms $U \to X$ is called \emph{\'etale topology}.
We have the following inclusion of topologies: $A \rightarrow B$ means that $B$ is finer than $A$.\footnote{fppf: flat and locally of finite presentation, fpqc: faithfully-flat and quasi-compact.}
\begin{align*}
    \boxed{
        \text{Zariski $\rightarrow$ \'etale $\rightarrow$ fppf $\rightarrow$ fpqc}
    }
\end{align*}
With Grothendieck topology on $X$, principal $G$-bundle is a scheme $\mathcal{P}$ with $G$-equivariant morphism $\pi: \mathcal{P} \to X$
such that 1) it is locally trivial in the sense that we have a covering of $X$ with (certain type of) morphisms
$\{\iota_{\alpha}:U_{\alpha} \to X\}_{\alpha}$ such that pullback of $\mathcal{P}$ to $U_\alpha$, $\iota_{\alpha}^{*}\mathcal{P}=\mathcal{P}\times_{X} U_{\alpha}$, is 
isomorphic to $G \times U_{\alpha}$ (i.e. trivial), and 2) the morphism $\iota_{\alpha}$ ``agree on intersections'' in the sense that pullback of $\mathcal{P}$
along two morphisms
\begin{align*}
U_{\alpha} \times_X U_{\beta} \xrightarrow{\iota_{\alpha\beta}} U_{\alpha} \xrightarrow{i_{\alpha}} X \\
U_{\alpha} \times_X U_{\beta} \xrightarrow{\iota_{\beta\alpha}} U_{\beta} \xrightarrow{i_{\alpha}} X
\end{align*}
are isomorphic.

Notion of a principal $G$-bundle depends on the topology.
For example, consider a double covering 

\begin{center}
    \begin{tikzcd}
        Y = \mathrm{Spec}\mathbb{C}[y, y^{-1}] \simeq \mathbb{A}^{1} \backslash \{0\}\arrow[d]  \\
        X = \mathrm{Spec}\mathbb{C}[x, x^{-1}] \simeq \mathbb{A}^{1} \backslash \{0\}
    \end{tikzcd}
\end{center}
corresponds to a map $\mathbb{C}[x, x^{-1}] \to \mathbb{C}[y, y^{-1}], x \mapsto y^2$.
We have a $\mathbb{Z}/2$-action on $Y$ given by $y \mapsto -y$.
With \'etale topology, this automorphic imposes a $\mathbb{Z}/2$-bundle structure on $Y\to X$.
However, this can't be a $\mathbb{Z}/2$-bundle with Zariski topology.
In fact, any open subset of $X$ in Zariski topology has a form of $X \backslash S$ for some finite set $S \subset X$, 
and there's no $U\subset X$ such that $Y|_{U} \simeq U \coprod U$ (i.e. it can't be locally trivial).

There are some \emph{special} groups $G$ where local triviality of principal $G$-bundle on Zariski topology is equivalent to
that on \'etale topology, such as $G = \GL_n, \SL_n, \mathrm{Sp}_n$.
For such groups, it is also true that local triviality with Zariski topology is also equivalent to that on fpqc topology.

In case of $\GL_n$, we have the following 1-1 correspondences:
\begin{center}
    \begin{tikzcd}
        \boxed{\text{Locally free sheaves of } \mathcal{O}_X\text{-mod of rank }n } \arrow[d, leftrightarrow] \\
        \boxed{\text{Vector bundles of rank }n} \arrow[d, leftrightarrow]\\
        \boxed{\text{Principal }\GL_n\text{-bundles}}
    \end{tikzcd}
\end{center}
this follows from Grothendieck's faithful flat descent.
Grothendieck also proved that the notion of $G$-bundle coincides on fpqc and \'etale topology
when $G$ is a reductive group.
When $X$ is a projective curve over an algebraically closed field, the notion of $G$-bundle with \'etale, Zariski, and analytic topology all coincides, and this implies that
we have a bijection
\begin{center}
    \begin{tikzcd}
        \boxed{G(F) \backslash G(\mathbb{A}_F) / G(\mathcal{O}_F)} \arrow[d, leftrightarrow] \\
        \boxed{\text{equivalence classes of principal }G\text{-bundles loc. triv. in Zariski topology}}
    \end{tikzcd}
\end{center}
which is due to Drinfeld-Simpson.





\bibliographystyle{acm}
\bibliography{ref}
\end{document}
