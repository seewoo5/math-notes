\newpage
\section{Satake isomorphism (September 8)}

% \subsection{Preliminaries on affine group schemes}
% Before we rigorously state the Satake isomorphism, we'll go through some basics on affine group schemes.


\subsection{Satake isomorphism}
Recall that for given cocharacter $\lambda \in X_{*}(T)$, we can associate an element in $T(F_x)$ 
by evaluating $\lambda$ at chosen uniformizer $t_x \in F_x$.
Then it defines a well-defined double coset in $G(\mathcal{O}_x) \backslash G(F_x) / G(\mathcal{O}_x)$
since the choice of uniformizer is only differ by $\mathcal{O}_x^{\times}$.
Now we can state the Satake isomorphism for genral (split) reductive groups.
\begin{theorem}
We have an isomorphism of $\mathbb{C}$-algebra
$$
\mathcal{H}_{x} \simeq \mathbb{C}[X_{*}(T)]^{W} \simeq \mathbb{C}[X^{*}(\widehat{T})]^{W}.
$$
\end{theorem}
As a consequence, Hecke algebras are commutative.
\begin{proof}
Satake provided explicit isomorphism from $\mathcal{H}_{x}$ to $\mathbb{C}[X_{*}(T)]^{W}$, which is
$$
f \mapsto \sum_{\lambda \in X_{*}(T)} \left(q_{x}^{\langle \rho, \lambda\rangle} \int_{N(F_x)}f(u \lambda(t_x))du \right) \lambda
$$
Here $\rho = (1/2)\sum_{\alpha \in \Delta^{+}}\alpha$ is the half-sum of the positive roots
and $du$ is the Haar measure on $N(F_x)$ normalized as $du(N(\mathcal{O}_x)) = 1$.
Since $f \in \mathcal{H}_x$ is compactly supported, the image under Satake's map is actually a finite sum.
He showed that this is an injective algebra homomorphism whose image is $\mathbb{C}[X_*(T)]^W$.
\end{proof}
Now, observe that we have following isomorphisms
$$
\mathbb{C}[X_{*}(T)]^{W} \simeq \mathbb{C}[T]^{W} \simeq \mathbb{C}[G]^{G}
$$
where the last space is the space of class functions on $G$, i.e. functions on $G$ that are invariant under conjugation.
The second isomorphism comes from Chevalley's restriction theorem.
Hence the characters of $\mathcal{H}_x$ correspond to the characters of $\mathbb{C}[G]^{G}$, which again correspond to
the semisimple conjugacy classes in $G$.
This is the way how we associate ``Hecke conjugacy class $\pi(h_x)$''\footnote{Usually, it is called as \emph{Satake parameters}.} to an automorphic representation $\pi$.

\subsection{Why Weil group?}
On the Galois side, why do we care about representations of Weil groups, not the whole Galois group?
The reason can be found from the $n=1$ case, i.e. Langlands correspondence for $\GL_1 = \mathbb{G}_m$, which is also
called \emph{Abelian Class Field Theory}.

Let $\rho : \Gal(\overline{F}/F) \to \GL_1(\overline{\mathbb{Q}}_\ell) = \overline{\mathbb{Q}_{\ell}}^{\times}$
be an irreducible representation.
Since $\GL_1$ is abelian, the map factors through the abelianization of $\Gal(\overline{F}/F)$, which equals
$\Gal(F^{\mathrm{ab}}/F)$, the Galois group of maximal abelian extension.
On the automorphic side, (cuspidal) automorphic representations of $\GL_1$ are essentially the characters of $\GL_1(F) \backslash \GL_1(\mathbb{A}_F) = F^{\times} \backslash \mathbb{A}_F^{\times}$.
Then the Langlands correspondence is given through Artin reciprocity map
$$
\theta: F^{\times} \backslash \mathbb{A}_F^{\times} \to \Gal(F^{\mathrm{ab}} / F).
$$
By the way, this map is injective but not surjective, and the image is exactly the ablianization of Weil group, $W(F^{\mathrm{ab}}/F)$.
Hence, generalizing this idea for $n > 1$ makes us to consider representations of Weil group instead of the
full absolute Galois group.