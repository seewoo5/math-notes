\newpage
\section{Glimpse on analytic langlands (September 13)}

Now, our goal is to formulate Langlands correspondence in geometric terms, so that we could
find their analogues when a curve $X$ is over $\mathbb{C}$, rather than a finite field $\mathbb{F}_q$.
For simplicity, we'll assume that everything is unramified, i.e. the ``Galois representations'' and ``automorphic representations''
are unramified everywhere ($S_\sigma = S_\pi = \emptyset$), whatever it means.

On the Galois side, our slogan is the following: the geometric object corresponds to 
Galois group is essentially the (\'etale) \emph{fundamental group} of $X$.
In this case, extensions of a function field corresponds to (\'etale) coverings $Y \to X$.
For example, if we have a curve $X$ (over a base field $k = \mathbb{C}$ or $\mathbb{F}_q$) and a covering $Y \to X$,
then we get a field extension $k(X) \hookrightarrow k(Y)$.
Then the Galois group $\Gal(k(Y) / k(X))$ is essentially the group of deck transformations of this covering.
Hence the Galois group of maximal unramified extension of $F = k(X)$, $\Gal(F^{\mathrm{un}}/F)$,
is the group of deck transformations of maximal unramified cover $\widetilde{X}$, which is a fundamental group $\pi_1(X)$ of $X$.

% Recall that, for topological spaces and their (universal) coverings, fundamental group is defined with a base point

Now, let $X$ be a smooth projective connected algebraic curve over $\mathbb{C}$, or in other words, 
a compact Riemann surface.
Then our analogy for the Galois representations becomes
\begin{align*}
    \boxed{
        \text{Equivalence class of } \sigma: \pi_1(X) \to \widehat{G}(\overline{\mathbb{Q}_{\ell}})
    }
\end{align*}
Now, such a collection has a correspondence with flat connections:
\begin{align*}
    \boxed{
        (E, \nabla): C^{\infty} \text{ principal } \widehat{G} \text{-bundles with } C^{\infty}\text{ flat connection}
    }
\end{align*}
where $C^{\infty}$ principal $G$-bundle is a principal $G$-bundle where transition maps are smooth.
By decomposing it into holomorphic and anti-holomorphic part ($\nabla = \nabla^{(1, 0)} + \nabla^{(0, 1)}$), one can see that this also has a correspondence with
\begin{align*}
    \boxed{
        (E^{\mathrm{hol}}, \nabla^{(1, 0)}): \text{ flat holomorphic $\widehat{G}$-bundle with a holomorphic connection $\nabla^{(1, 0)}$}
    }
\end{align*}
Note that, on a curve, any such connection becomes automatically flat.

For example, let $X = \mathbb{C}$ and consider trivial $\GL_1 = \mathbb{G}_{m}$-bundle on $X$, $\mathbb{C}^{\times} \times \mathbb{C}$.
Let $\nabla^{(1, 0)} = \frac{\partial}{\partial z} + \frac{\lambda}{z}$ be a connection on $X$.
Then the solution of $\nabla^{(1, 0)}\phi = 0$ is $\phi(z) = z^{-\lambda} = e^{-\lambda \log z}$, 
and this gives a monodromy action where a generator of $\pi(\mathbb{C}^{\times}, 1) \simeq \mathbb{Z}$ acts as multiplication by $e^{-2\pi i \lambda}$.

Now let's move on to the automorphic side.
Recall that automorphic representation $\pi$ can be thought as a space of compactly supported
smooth (i.e. locally constant) functions on $G(F)\backslash G(\mathbb{A}_F)$.
If it is unramified everywhere, then its $G(\mathcal{O}_F) = \prod_{v}' G(\mathcal{O}_{F_v})$-invariant subspace 
$\pi^{G(\mathcal{O}_F)}$ is a subspace of functions on a double coset 
$$
G(F) \backslash G(\mathbb{A}_F) / G(\mathcal{O}_F).
$$
Upshot is, the above double coset space is on bijection with the set of isomorphism lasses of principal (holomorphic or algebraic) $G$-bundles on $X$.
For example, in case of $G = \mathrm{GL}_1$, the double coset space is
$$
F^{\times} \backslash \mathbb{A}_F^{\times} / \sideset{}{'}\prod_{x} \mathcal{O}_{F_x}^{\times} \simeq F^{\times} \backslash \sideset{}{'}\prod_{x}(F_{x}^{\times} / \mathcal{O}_{F_{x}}^{\times}) \simeq F^{\times} \backslash \sideset{}{'}\prod_{v}\mathbb{Z}.
$$
On the other hand, $\GL_1$-bundles on $X$ are the same as line bundles on $X$, 
and this again corresponds to (Weil) divisors on $X$.
Divisor is just a formal $\mathbb{Z}$ combination of closed points on $X$.
To divisors are equivalent when the difference is principal divisor, i.e. divisor of a form
$$
\mathrm{div}(f):= \sum_{x \in |X|} \mathrm{ord}_{x}(f)\cdot [x]
$$
and the set of equivalence classes of divisors is in bijection with the divisor class group, which is
$$
\mathrm{Cl}(X) = \mathrm{Div}(X) / \mathrm{PDiv}(X)
$$
(quotient of group of divisors by group of principal divisors).
One can find natural bijection between $\mathrm{Cl}(X)$ and the above double coset space.
