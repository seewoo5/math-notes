\newpage
\section{Introduction to Langlands correspondence (August 25)}

This course is on a new aspect of Langlands program, so-called \emph{Analytic Langlands Program}.
The classical Langlands program is originated from Langlands' letter to Andr\'e Weil in 1967,
and also from Andr\'e Weil's letter to his sister  (Simone Weil) on his conjecture (Weil's conjecture on
zeta functions of curves over finite fields, which was resolved by Dwork, Grothendieck, and Deligne) in 1940.
Weil's \emph{Rosetta stone} relates two different topics in mathematics: number theory and complex curves (Riemann surfaces).
A goal is to find something happens in parallel between two, and we need another bridge - curves over finite fields.
The difference between complex curves and curves over finite fields is the fact that those are defined over different fields.
A similarity between number theory sied and the curves over finite fields side is that 
the number fields (finite extensions of $\mathbb{Q}$) are similar to the function fields of curves (over finite fields - we denote it as $\mathbb{F}_{q}(X)$ for a curve $X/\mathbb{F}_{q}$).
The most simplest example is a comparison between $\mathbb{Q}$ and $\mathbb{F}_{q}(\mathbb{P}^{1}) \simeq \mathbb{F}_{q}(t)$:
\begin{align*}
    \mathbb{Q} = \left\{ \frac{p}{q}\,:\,p, q\,\text{rel. prime} \in \mathbb{Z}\right\} &\leftrightarrow \mathbb{F}_{q}(t) = \left\{ \frac{P(t)}{Q(t)}\,:\, P, Q\,\text{rel. prime} \in \mathbb{F}_{q}[t]\right\} \\
    \text{ring of integers: }\mathbb{Z} &\leftrightarrow \mathbb{F}_{q}[t] \\
    \text{completions: }\mathbb{Q}_{p} &\leftrightarrow \mathbb{F}_{q}((t))
\end{align*}
Sometimes we include one more topic in this Rosetta stone, which originates from Physics -
Quantum Field Theory, Electro-Magnetic Duality, and Gauge Theory (developed by Edward Witten and other physicists).

The (classical) Langlands correspondence is about interplays between the Galois representations and Automorphic representations.\footnote{The \emph{geometric} Langlands correspondence is about curves over $\mathbb{C}$, which is mainly developed by Drinfeld, Laumon, Beilinson, Gaitsgory, ...}
It dealts with two different (but similar) types of fields - number fields and function fields of curves over finite fields.
Fix such a field $F$.
Then we can describe a Langlands correspondence for $\GL_{n}$ over $F$ as follows:
\begin{enumerate}
    \item \textbf{Galois side:} Let $\overline{F}$ be a (separable) algebraic closure of $F$, and $\Gal(\overline{F}/F)$ 
    be the absolute Galois group of $F$ (i.e. the group of automorphisms of $\overline{F}$ that fix $F$ pointwisely).
    This is one of the most important groups in number theory, and its structure is higly complicated.
    Hence, instead of studying the group $\Gal(\overline{F}/F)$ directly, we study the representations of it.
    Especially, we are going to consider the (equivalence) classes of $n$-dimensional representations of $\Gal(\overline{F} / F)$ over some field that would be determined later.    
    This is just an equivalence class of homomorphisms
    $$
        \sigma: \Gal(\overline{F} / F) \to \GL_{n}(?).
    $$
    \item \textbf{Automorphic side:} We first need to define the notion of \emph{Adele}.
    Let $\mathscr{V} = \mathscr{V}_{F}$ be the set of equivalent classes of the norms (places) on $F$.
    For each $v \in \mathscr{V}$, we can define a completion $F_{v}$ with respect to $v$.
    For example, when $F = \mathbb{Q}$, Ostrowski's theorem states that the places of $\mathbb{Q}$ corresponds to
    the set of primes (each prime $p$ gives $p$-adic norms, which is non-archimedean) along with the ``infinite'' prime (corresponds to the usual archimedean norm).
    In this case, we have two types of completions, either $p$-adic numbers $\mathbb{Q}_{p}$ or real numbers $\mathbb{R}$.
    And we have the ring of $p$-adic integers $\mathbb{Z}_{p}$ as a subring of $\mathbb{Q}_{p}$.

    In case of function field $F =\mathbb{F}_{q}(\mathbb{P}^{1}) \simeq \mathbb{F}_{q}(t)$ of $X = \mathbb{P}^{1}$,
    the places of $F$ corresponds to the closed points of $X$, which again corresponds to
    the maximal ideals of $\mathbb{F}_{q}[t]$ (and the point at infinity).
    For example, any $a \in \mathbb{F}_{q}$ actually gives a closed point that corresponds to
    the maximal ideal $(x-a)$.
    Any other irreducible polynomials over $\mathbb{F}_{q}$ of higher degree also give closed points in $X$.
    Completion of $F$ at $x \in X$ is isomorphic to the field of formal Laurent series $(\mathbb{F}_{q})_{x}((t_{x}))$,
    where $(\mathbb{F}_{q})_x$ is the residue field at $x$ and $t_{x}$ is some parameter.
    We also have a ring of integers in these completions, which is $(\mathbb{F}_{q})_{x}[[t_{x}]]$.

    The ring of adeles is defined as a restricted product of all completions of $F$, which is
    $$
        \quad\quad\quad\quad \mathbb{A}_{F} = \prod_{v \in \mathscr{V}_{F}} F_{v} = \left\{(f_{v})\,:\, f_{v} \in F_{v}, f_{v} \in \mathcal{O}_{v}\text{ for all but finitely many $v$.}\right\}
    $$
    where $\mathcal{O}_{v} \subset F_{v}$ is the ring of integers of $F_{v}$, which is the set of elements with norm at most 1.
    Then we have a diagonal embedding $F \hookrightarrow \mathbb{A}_{F}$ that sends $a \in F$ to $(a, a, \dots) \in \mathbb{A}_{F}$,
    and this induces an embedding $\GL_{n}(F) \hookrightarrow \GL_{n}(\mathbb{A}_{F})$.
    Now we can think of a Hilbert space $\mathscr{H}(\GL_{n}(F) \backslash \GL_{n}(\mathbb{A}_{F}))$ of $\L2$-functions on the quotient space $\GL_{n}(F) \backslash \GL_{n}(\mathbb{A}_{F})$,
    with the Haar measure on the quotient space.
    Then we have a right regular representation of $\GL_{n}(\mathbb{A}_{F})$ on $\mathscr{H}$, and it is known that this representation decomposes
    into continuous part and discrete part:
    $$
        \mathscr{H} = \mathscr{H}_{\mathrm{cont}} \oplus \mathscr{H}_{\mathrm{disc}}
    $$
    and the discrete part decomposes into irreducible representations as 
    $$\mathscr{H}_{\mathrm{disc}} = \oplus_{\pi}\pi$$
    without multiplicity (multiplicity one theorem).\footnote{
        It is not always the case that any representation decomposes into irredubiles - 
        consider the regular representation of $\mathbb{R}$ on $\L2(\mathbb{R})$ that acts as a translation.
        The irreducible sub-representations of it correspons to the exponential function $\exp(i\lambda x)$ for $\lambda \in \mathbb{R}$,
        but we can't write a function $f(x)$ as a discrete sum of these in general. 
        We can only write it as an integral of these, which is the Fourier transform.
        Note that the regular representation on $\L2(\mathbb{Z}\backslash \mathbb{R})$ decomposes into irreducibles (which gives Fourier series),
        and the reason behind is that the circle group $\mathbb{S}^{1} = \mathbb{Z} \backslash\mathbb{R}$ is compact.
    }
    The irreducible constituents of $\mathscr{H}_{\mathrm{disc}}$ is called \emph{cuspidal automorphic rerpesentations of $\GL_{n}(\mathbb{A}_{F})$},
    up to technical conditions on the center of the group and the archimedean places.
    \item \textbf{Correspondence:} The Langlands correspondence states that there is a one-to-one correspondence between these two different objects that preserves some special invariants.
    More precisely, the equivalence classes of $n$-dimensional Galois representation $\sigma$ of $\Gal(\overline{F}/F)$ (over some field)
    corresponds to an irreducible automorphic representation $\pi = \pi(\sigma)$ of $\GL_{n}(\mathbb{A}_{F})$, and vice versa.
    The irreducibility of $\sigma$ corresponds to cuspidality of $\pi$.

    One of the invariant of Galois representation $\sigma$ is the conjugacy classes of $\sigma(\mathrm{Fr}_{x})$ in $\GL_{n}$, where
    $\mathrm{Fr}_{x}$ is the \emph{Frobenius} element corresponds to a closed point $x \in X$ (for almost all $x$).
    On the automorphic side, there is so-called \emph{Hecke operators} $h_{x}$ for each $x \in X$.
    Then the conjectural Langlands correspondence should give correspondences between these two invariants.
    
\end{enumerate}

A celebrated example of Langland's correspondence is the Shimura-Taniyama-Weil conjecture, which is now a theorem by Andrew Wiles and Richard Taylor (modularity theorem).
It is a special case of Langland's correspondence for $n = 2$ that  relates an elliptic curve and a modular form.

Let $E$ be an elliptic curve over $\mathbb{Q}$, i.e. a smooth projective curve defined over $\mathbb{Q}$ by equation
$$
    y^{2} = x^{3} + ax + b
$$
for $a, b \in \mathbb{Q}$ and $\Delta = 4a^{3} + 27b^{2} \neq 0$ (the discriminant of $E$).
Then for each prime $\ell$ not dividing $\Delta$ (or any $\ell \neq p = \mathrm{char}(F)$ when $F$ is a function field),  the first \'etale cohomology $\mathrm{H}^{1}_{\text{\'et}}(E_{\overline{\mathbb{Q}}}, \mathbb{Q}_{\ell})$ is isomorphic to $\mathbb{Q}_{\ell}^{2}$,
2-dimensional vector space over $\mathbb{Q}_{\ell}$.
Since $E$ is defined over $\mathbb{Q}$, we have a natural action of $\Gal(\overline{\mathbb{Q}}/\mathbb{Q})$ on $E_{\mathbb{Q}} = E \otimes_{\mathbb{Q}}\overline{\mathbb{Q}}$ which
induces an action on $\mathrm{H}^{1}_{\text{\'et}}(E_{\overline{\mathbb{Q}}}, \mathbb{Q}_{\ell})$i, i.e. a 2-dimensional representation $\sigma_{E, \ell}$ over $\mathbb{Q}_{\ell}$
(so the mysterious field where the representation is defined that we didn't defined before is $\mathbb{Q}_{\ell}$ in this case).
For $p \nmid \Delta$, Frobenious conjugacy class $\sigma_{E, \ell}(\mathrm{Fr}_{p})$ has a trace
$$
    \mathrm{Tr}(\sigma_{E, \ell}(\mathrm{Fr}_{p})) = p + 1 - \# E(\mathbb{F}_{p})
$$
and determinant $p$, which completely determines the conjugacy class of $\sigma_{E, \ell}(\mathrm{Fr}_{p})$ for $\GL_{2}$.
Here $\#E(\mathbb{F}_{p})$ is the number of $\mathbb{F}_{p}$-points on $E$.

Assuming Langlands' correspondence, such $\sigma_{E, \ell}$ (or family of $\sigma_{E, \ell})$ for varying $\ell$'s) should corresponds
to an irreducible automorphic representation of $\GL_{2}(\mathbb{A}_{\mathbb{Q}})$.
In case of $\GL_{2}$, an automorphic representation $\pi$ corresponds to certain holomorphic function $f_{\pi}$ called \emph{modular form} on
the complex upper half plane $\mathfrak{H}$ satisfying a functional equation
$$
    f_{\pi} \left(\frac{a\tau + b}{c\tau + d}\right) = (c\tau + d)^{2}f_{\pi}(\tau)
$$
for $(\begin{smallmatrix} a & b \\ c & d\end{smallmatrix}) \in \Gamma_{0}(N) = \{(\begin{smallmatrix} a &b \\ c & d\end{smallmatrix}) \in \mathrm{SL}_{2}(\mathbb{Z}), c\equiv 0\,(\mathrm{mod}\,N)\}$
where $N = N_{E}$ is the conductor of $E$.
Also, $f_{\pi}(\tau)  \to 0$ as $\tau \to \infty$ (in other words, $f_{\pi}$ is a cusp form).
It admits a Fourier expansion
$$
    f_{\pi}(\tau) = \sum_{n\geq 1} a_{n} q^{n}, \quad q = e^{2 \pi i \tau}
$$
and the Langlands correspondence for this case becomes the equality
$$
a_{p} = \mathrm{Tr}(\sigma_{E, \ell}(\mathrm{Fr}_p)) = p + 1 - \# E(\mathbb{F}_{p}).
$$
for all $p \nmid \ell N$.

Langlands correspondence for $\GL_n$ is now a theorem when $F$ is a function field of some curve, and this is proven by Drinfeld ($n = 2$) and Laurent Lafforgue ($n>2$).\footnote{Lafforgue got a Fields medal for this work.}
