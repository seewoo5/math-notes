\newpage
\section{Fourier-Mukai transform for $\mathrm{GL}_{1}/\mathbb{C}$ (October 11)}

Last time, we see how the geometric Langlands correspondence for $\GL_1/\mathbb{C}$
for complex curves, via the sequence of isomorphisms
$$
\pi_1(X, x)^{\mathrm{ab}} \cong \rH_1(X, \mathbb{Z}) \cong \rH^{1}(X, \mathbb{Z}) \cong \pi_1(\Jac_X, \tilde{x}).
$$
(The first isomorphism is Hurwicz' theorem, the second one is from Poincare duality, and the third one follows from
$\Jac_X \simeq \rH^{1}(X, \mathcal{O}_x) / \rH^{1}(X,\mathbb{Z})$.)
Also, we studied about the modulispace of rank 1 local systems on $X$, $\Loc_{\GL_1, X}$, 
and view it as the universal extension of $\Jac_X$ by a vector space:
$$
0 \to \rH^{0}(X, K_X) \to \Loc_{\GL_1, X} \to \Jac_X \to 0.
$$
The \emph{moduli space} of some objects means an object that represents some functor that describes given moduli problem.
For example, $\Loc_{\GL_1, X}$ is a scheme\footnote{it might be a stack - I'm not sure about this point.} that represents the functor
from $\mathbb{C}$-schemes $\mathbf{Sch}_{\mathbb{C}}$ to the category of abelian groups $\mathbf{Ab}$,
$$
S \mapsto \{\text{abelian group of line bundles on }S \times X\text{ with partial connection along }X\}.
$$
(In other words, the above functor isomorphic to the hom functor $\mathrm{Hom}(S, \Loc_{\GL_1, X})$.)
By the (geometric) Langlands correspondence for $\GL_1$, we have a bijection between
flat line bundles on $X$ and those on $\Jac_X$, so that we can interprete $\Loc_{\GL_1, X}$ as a moduli space
of flat line bundles on $\Jac_X$.
Now, when $S = \Loc_{\GL_1, X}$, there's the universal bundle $\mathcal{P}$ on $\Loc_{\GL_1, X} \times \Jac_X$
that corresponds to the identity map in $\mathrm{Hom}(\Loc_{\GL_1, X}, \Loc_{\GL_1, X})$.
We have the following diagram:
\begin{center}
    \begin{tikzcd}
        & \mathcal{P} \arrow[d] & \\
        & \Loc_{\GL_1, X} \times \Jac_X \arrow[dl, "p_1", swap] \arrow[dr, "p_2"] & \\
        \Loc_{\GL_1, X} & & \Jac_X
    \end{tikzcd}
\end{center}
Fourier-Mukai transform gives us the following \emph{derived} Langlands correspondence:
\begin{theorem}[Mukai, Laumon, Rothstein]
    The functors
    \begin{align*}
        F(\mathcal{M}) &= Rp_{1*}(p_{2}^{*}(\mathcal{M}) \otimes \mathcal{P}) \\
        G(\mathcal{K}) &= Rp_{2*}(p_{1}^{*}(\mathcal{K}) \otimes \mathcal{P}) \\
    \end{align*}
    give an equivalence of two derived categories
    $$
    D^{b}(\mathcal{O}_{\Loc_{\GL_1, X}}\text{-}\mathbf{Mod}) \simeq D^{b}(D_{\Jac_X}\text{-}\mathbf{Mod})
    $$
    (up to sign and cohomological shift).
\end{theorem}
The transform sends skyscraper sheaves $\mathcal{O}_{(L, \nabla)}$ of $\Loc_{\GL_1, X}$ supported at $(L, \nabla)$ to 
the corresponding $\mathcal{F}_{(L,\nabla)}^{0}$ on $\Jac_X$.
Mukai's original transformation is defined between the categories of coherent sheaves on an abelian 
variety $A$ and its dual $A^{\vee}$, and there's a categorical generalization between $\mathcal{O}$-modules and $D$-modules.
The above theorem is the special case when $A = \Jac_X$.