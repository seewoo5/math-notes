\newpage
\section{Classical Langlands correspondence over function fields (September 1)}


We are going to explain classical Langlands correspondence over function fields in (more) detail.
Let $X$ be a smooth, geometrically irreducible, projective curve over $\mathbb{F}_{q}$ and $F = \mathbb{F}_{q}(X)$ be a function field.
Let $|X|$ be a set of closed points of $X$, which has a 1-1 correspondence with $\mathscr{V}$ - the set of
places (completions) of $F$.
Recall that the completion $F_{x}$ at $x \in |X|$ is isomorphic to $(\mathbb{F}_{q})_{x}((t_{x}))$, where 
$(\mathbb{F}_{q})_{x}$ residue field at $x$ and $t_{x}$ is a rational functino on $X$ with order 1 zero at $x$
(In other words, it is a generator of maximal ideal $\mathfrak{m}_{x}$ corresponds to $x$).
Then we have a ring of integer $\mathcal{O}_{x} \subset F_{x}$ isomorphic to the ring of
formal power series $(\mathbb{F}_{q})_{x}[[t_{x}]]$.
We also defined the ad\'ele ring $\mathbb{A}_{F}$ for $F$.

Now we define the \emph{Weil group} $W(\overline{F}/F)$ as follows.
Let $\overline{F}$ be a (separable) algebraic closure of $F$, then we have the action 
of $\Gal(\overline{F}/F)$ on the subfield $\overline{\mathbb{F}_{q}}$ (the field of constants) that fixes $\mathbb{F}_{q}$.
Then we have a surjective map 
$$\Gal(\overline{F}/F) \xrightarrowdbl{\mathrm{res}} \Gal(\overline{\mathbb{F}_{q}}/\mathbb{F}_{q})$$
and the latter group is an inverse limit of Galois groups of finite extensions of $\mathbb{F}_{q}$, so
$$
\Gal(\overline{\mathbb{F}_{q}}/\mathbb{F}_{q}) \simeq \varprojlim \Gal(\mathbb{F}_{q^{n}} / \mathbb{F}_{q}) \simeq \varprojlim \mathbb{Z}/n\mathbb{Z} =: \widehat{\mathbb{Z}},
$$
which is the profinite completion of $\mathbb{Z}$.
It is topologically generated by Frobenius automorphism $\mathrm{Fr}$, and it has a subgroup isomorphic to $\mathbb{Z}$ generated (not topologically, but just algebraically) by $\mathrm{Fr}$.
Then we define the \emph{Weil group} $W(\overline{F}/F)$ as an inverse image of $\mathbb{Z} \simeq \langle \mathrm{Fr} \rangle \subset \Gal(\overline{\mathbb{F}_{q}} / \mathbb{F}_{q})$
of restriction map, which is a subgroup of $\Gal(\overline{F}/F)$.
For Galois side of Langlands correspondence over function field, we are going to consider irreducible representations of
$W(\overline{F}/F)$ instead of $\Gal(\overline{F}/F)$.
More precisely, we consider the (equivalence classes of) irreducible $n$-dimensional $\ell$-adic representations of $W(\overline{F}/F)$,
$$
\sigma: W(\overline{F}/F) \to \GL_{n}(\overline{\mathbb{Q}_{\ell}})
$$
such that 
\begin{enumerate}
    \item Image of $\sigma$ in $\GL_{n}(\overline{\mathbb{Q}_{\ell}})$ is in $\GL_{n}(E)$ for some finite extension $E/\mathbb{Q}_{\ell}$.
    \item $\sigma$ is continuous where $W(\overline{F}/F)$ is given Krull topology (profinite toppology) and $\GL_{n}(E)$ is given subspace topology of $M_{n}(E)$.\footnote{
        This explains somehow why we are considering $\ell$-adic representations instead of complex representations.
        As a toy example, consider continuous 1-dimensional complex representations of $(\mathbb{Z}_{\ell}, +)$, i.e. an additive character $\sigma: \mathbb{Z}_{\ell} \to \GL_{1}(\mathbb{C}) = \mathbb{C}^{\times}$.
        Then it should factor through $\mathbb{Z}_{\ell} / \ell^{n}\mathbb{Z}_{\ell}$ for some $n$, so that the image is always finite.
        However, if we consider $\ell$-adic characters $\sigma:\mathbb{Z}_{\ell} \to \mathbb{Q}_{\ell}^{\times}$, then there are non-trivial characters with infinite image, e.g. $x \mapsto \exp_{\ell}(\ell x)$ 
        where $\exp_{ell}$ is an $\ell$-adic exponential function.
    }
    \item $\sigma$ is unramified for all but finitely many $x \in |X|$.
    Note that the unramifiedness is defined using decomposition group and inertia group as before.
\end{enumerate}

On the automorphic side, we wiil explain cuspidality and unramifiedness in more detail.
The space of cusp forms $\mathrm{L}^{2}_{\cusp}(\GL_{n}(F) \backslash \GL_{n}(\mathbb{A}_{F}), \chi)$\footnote{Here $\chi$ is a continuous unitary character on center $Z(\mathbb{A}_{F})$ trivial on $Z(F)$,
and $\mathrm{L}^{2}(\GL_{n}(F)\backslash\GL_{n}(\mathbb{A}_{F}), \chi)$ is a space of functions where the center acts as the character $\chi$.
}
are functions satisfying the following vanishing condition: for $0 < n_{1}, n_{2} < n$ with $n = n_{1} + n_{2}$,
we have
$$
\int_{N_{n_{1}, n_{2}}(F) \backslash N_{n_{1}, n_{2}}(\mathbb{A}_{F})} f(ng)dn = 0
$$
for all $g\in \GL_{n}(\mathbb{A}_{F})$, where $N_{n_{1}, n_{2}} < \GL_{n}$ is the unipotent group of matrices of the form
$$
\begin{pmatrix}
    I_{n_{1}} & * \\ \mathbf{0} & I_{n_2}
\end{pmatrix}
$$
Note that non-example of cuspidal represenetation is Eisenstein series representation, which is obtained from
two representations $\pi_{1}, \pi_{2}$ of $\GL_{n_{1}}(\mathbb{A}_{F})$ and $\GL_{n_2}(\mathbb{A}_F)$ respectively, by inflation and (parabolic induction).
Then it is a theorem (from Flath) that any irreducible cuspidal represenetations of $\GL_{n}(\mathbb{A}_F)$ decomposes as restricted product
of local representations,
$$
\pi \simeq \bigotimes_{x\in |X|} \pi_{x}
$$
where each $\pi_{x}$ are irreducible representation of $\GL_{n}(F_x)$.
In this case, for all but finitely many $x$, $\GL_{n}(\mathcal{O}_{x})$-fixed subspace
$\pi^{\GL_{n}(\mathcal{O}_x)}$ is non-trivial and one-dimensional.
We call that $\pi$ is \emph{unramified at $x$} for such $x$.
For $x$ where $pi$ is unramified, we have a represenetation of \emph{spherical Hecke algebra $\mathcal{H}_{x}$}, which 
is a sub-algebra of compactly supported functions on $\GL_{n}(F_{x})$ that are $\GL_{n}(\mathcal{O}_{x})$-biinvariant.
Then $\mathcal{H}_{x}$ is a convolution algebra which is commutative and 
isomorphic to $\mathbb{C}[x_{1}^{\pm}, \dots, x_{n}^{\pm}]^{S_{n}}$
and corresponds to semisimple conjugacy classes in $\GL_{n}(\mathbb{C})$, which we will denote $\pi(h_{x})$.

Also, as in the case of Galois side, we impose some conditions on the automorphic side.
We will only consider automorphic representations of $\GL_{n}(\mathbb{A}_{F})$ with some finiteness conditions, i.e.
for any compact subgroup $K$ of $\GL_{n}(\mathbb{A}_F)$, the translates of any $f\in \pi$ span a
finite dimensional vector space.

Then the Langlands correspondence becomes as follows.
It is a 1-1 correspondence between the irreducible $\ell$-adic $n$-dimensional representations of Weil group $W(\overline{F}/F)$
(with some conditions) and irreducible cupsidal automorphic representations of $\GL_{n}(\mathbb{A}_{F})$ (with some conditions).
The invariants, Frobenius conjugacy classes $\{\sigma(\mathrm{Fr}_{x})\}$ on the Galois side, matches with
the Hecke conjugacy classes $\{\pi(h_{x})\}$, for all $x \not \in S_{\sigma} \cup S_{\pi}$.
Here $S_{\sigma}$ (resp. $S_{\pi}$) is the set of unramified places for $\sigma$ (resp. $\pi$), and
we actually have $S_{\sigma} = S_{\pi}$ for corresponding $\sigma - \pi$ pairs.