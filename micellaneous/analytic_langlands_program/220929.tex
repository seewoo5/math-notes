\newpage
\section{Grothendieck's dictionary, continued (September 29)}

Today's goal is to define the notion of Hecke eigen\emph{sheaf} and their construction, relation to ACFT.
Recall that, for a given $\ell$-adic sheaf $\mathcal{F}$ on $\Pic_X$, 
one can assign a function $f_{\mathcal{F}}: \Pic_X(\mathbb{F}_q) \to \overline{\mathbb{Q}}_{\ell}$
defined as ``trace of Frobenius at a stalk": $f_{\mathcal{F}}(x) = \mathrm{Tr}(\mathrm{Fr}_{x}|\mathcal{F}_{x})$,
and Deligne proved that one can find an $\ell$-adic sheaf, which is even a local system of rank 1, that corresponds to a given function on $\Pic_X(\mathbb{F}_{q})$.
If $f$ is a Hecke eigenfunction, we want a corresponding function to be a Hecke eigensheaf.
To define it, consider a map $h_{x}: \Pic_X(\mathbb{F}_{q}) \to \Pic_X(\mathbb{F}_{q}), \mathcal{L}\mapsto \mathcal{L}(x)$
defined for each $x \in |X|$.
In the last lecture, we defined the Hecke operator $H_{x}$ as $H_{x}:= h_{x}^{*}$.
In fact, $h_{x}$ can be defined as a morphism of Picard variety $\tilde{h}_{x}: \Pic_X \to \Pic_X$,
and pullback along the morphism define Hecke \emph{functor}
$$
\mathbf{H}_{x}:= (\tilde{h}_{x})^{*}: \mathbf{Sh}_{\ell}(\Pic_X) \to \mathbf{Sh}_{\ell}(\Pic_X)
$$
This is compatible with sheaf-function assignment in the sense of the following commuting diagram:
\begin{center}
    \begin{tikzcd}
        \mathbf{Sh}_{\ell}(\Pic_X) \arrow[d, "\mathcal{F}\mapsto f_{\mathcal{F}}", swap] \arrow[r,"\mathbf{H}_{x}"] & \mathbf{Sh}_{\ell}(\Pic_X) \arrow[d, "\mathcal{F} \mapsto f_{\mathcal{F}}"] \\
        \mathrm{Fun}(\Pic_X(\mathbb{F}_{q}), \overline{\mathbb{Q}}_{\ell}) \arrow[r, "H_{x}"]& \mathrm{Fun}(\Pic_X(\mathbb{F}_q), \overline{\mathbb{Q}}_{\ell})
    \end{tikzcd}
\end{center}
This is an example of \emph{categorification}. We have the following dictionary of correspondences:
\begin{center}
    
\end{center}

\begin{definition} An $\ell$-adic sheaf $\mathcal{F}$ on $\Pic_X$ is called \emph{Hecke eigensheaf} if
$\mathbf{H}_{x}\mathcal{F} \simeq \mathcal{A}_{x} \otimes \mathcal{F}$ for some $\mathcal{A}_{x}$ for all $x\in |X|$.\footnote{To be more precise, it is a sheaf with \emph{compatible} collection of isomorphisms $\iota_{x}: \mathbf{H}_{x}\mathcal{F} \simeq \mathcal{A}_{x}\otimes \mathcal{F}$.}
\end{definition}
In this case, the corresponding function becomes Hecke eigenfunction as $H_{x}f_{\mathcal{F}} = a_{x}f_{\mathcal{F}}$ where $a_{x} = \mathrm{Tr}(\mathrm{Fr}_{x}|\mathcal{F}_x)$.
However, this does not tell us how the eigenvalues $a_{x}$'s are related.
Instead, we consider a family of line bundles: define a map $\tilde{h}$ as 
$$
\tilde{h}: X \times \Pic_X \to \Pic_X, \quad (x, \mathcal{L}) \mapsto \mathcal{L}(x).
$$
Then its pullback defines a functor $\mathbf{H} = \tilde{h}^{*}: \mathbf{Sh}_{\ell}(\Pic_X) \to \mathbf{Sh}_{\ell}(X \times \Pic_X)$, and being $\mathcal{F}$ an $\ell$-adic sheaf
on $\Pic_X$ is equivalent to have a sheaf $\mathcal{A}$ on $X$ such that $\mathbf{H}\mathcal{F} \simeq \mathcal{A} \boxtimes \mathcal{F}$.
This formulation is much better and natural since one can prove that $\mathcal{A}$ is actually a rank 1 local system on $X$.
This also tells us how such an $\ell$-adic sheaf would corresponds to a Galois representation
$\sigma: W(F^{\mathrm{ab, un}}/F) \to \overline{\mathbb{Q}}_{\ell}^{\times}$ - which has 1-to-1 correspondence with local system on $X$.

Now we propose Deligne's construction of Hecke eigensheaves using Abel-Jacobi map.
It is a map from $\Sym^{d}X = X^{d} / S_{d}$ (the space of unordered $d$-tuple of points on $X$, which is smooth when $X$ is a curve) to $\Pic_{X}^{d}$ (degree $d$ divisors) defined in an obvious way.
Let's call it $P_{d}$.
Then we have a following commutative diagram on Hecke operators and Abel-Jacobi maps:
\begin{center}
    \begin{tikzcd}
        X \times \Sym^{d}X \arrow[r, "\tilde{h}_{d}"] \arrow[d, "\mathrm{id}_{X} \times P_{d}", swap] & \Sym^{d+1}X \arrow[d, "P_{d+1}"] \\
        X \times \Pic^{d}X \arrow[r, "\tilde{h}"] & \Pic^{d+1}X
    \end{tikzcd}
\end{center}
If we pullback a sheaf on $\Pic_X$ along $P_{d}$'s, then we get a sheaf $\mathcal{G} = \{\mathcal{G}_{d}\}_{d>0}$ on $\coprod_{d>0} \Sym^{d}X$.
Such a sheaf becomes ``eigensheaf'' if $\tilde{h}_{d}^{*}\mathcal{G}_{d} \simeq \mathcal{A} \boxtimes \mathcal{G}_{d+1}$.
For given rank 1 local system on $X$, we can construct such a collection of sheaves $\{\mathcal{G}_{d}\}$ as
$$
\mathcal{G}_{1} := \mathcal{A}, \quad \mathcal{G}_{d}:= \pi^{d}_{*}(\mathcal{A}^{\boxtimes d})^{S_d}
$$
where $\pi^{d}: X^{d} \twoheadrightarrow \Sym^{d}X$.
The stalk of this sheaf at $D = \sum_i n_i [x_i]$ is $\otimes_{i} \mathcal{A}_{x_i}^{\otimes n_i}$, which has rank 1.
Note that if $\mathcal{A}$ has rank $>$ 1, then there's no chance for $\mathcal{G}_{d}$ to be a local system because of dimensions.
One can also check that $\tilde{h}_{d}^{*}\mathcal{G}_{d+1} \simeq \mathcal{A} \boxtimes \mathcal{G}_{d}$.
Now the point is that, $\mathcal{G}_{d}$ descends to a sheaf on $\Pic_X^{d}$.
\begin{theorem}
    For $d > 2g - 2$, there is a rank 1 $\ell$-adic local system $\mathcal{F}_{d}$ on $\Pic_X^{d}$ such that $P_{d}^{*}\mathcal{F}_{d} \simeq \mathcal{G}_{d}$.
\end{theorem}
\begin{proof}
    Since $\mathcal{G}_{d}$ are local system, the are locally constant on fibers. By the way, the map $\Sym^{d}X \twoheadrightarrow \Pic^{d}_X$
    is a projective fibration (with the condition $d > 2g - 2$) and each fibers are isomorphic to $\mathbb{P}^{d-g}$, which is simply connected.
    (This is true for both over $\mathbb{C}$ (topologically simply connected) or $\mathbb{F}_{q}$ (\'etale fundamental group is trivial)).
    So it is constant on each fiber and descends to $\Pic_X^{d}$.
\end{proof}
Thus we have rank 1 $\ell$-adic local systems on $\coprod_{d > 2g - 2}\Pic_{X}^{d}$ namely $\{\mathcal{F}_{d}\}_{d > 2g - 2}$.
Now we can uniquely extend to all $d$ by using the property of eigensheaf.
Choose a point $x \in X(\mathbb{F}_{q})$ and define $\mathcal{F}_{2g-2}, \mathcal{F}_{2g-3}, \dots$ as $\mathcal{F}_{d} := \mathcal{A}_{x}^{\vee} \otimes h_{d+1}^{*}(\mathcal{F}_{d+1})$, 
and one can show that this is independent of the choice of $x$.\footnote{
    If there's no $\mathbb{F}_{q}$-point on $X$, we can still apply the similar argument by choosing any $x \in X(\overline{\mathbb{F}}_{q})$.
    In this case, the ``shifting'' occurs by the degree of residue field $[\mathbb{F}_{q, x}: \mathbb{F}_{q}]$.
}