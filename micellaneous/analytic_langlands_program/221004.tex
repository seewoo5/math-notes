\newpage
\section{Geometric Langlands correspondence for $\mathrm{GL}_{1}/\mathbb{C}$ (October 5)}

We are going to see what is a (geometric) Langlands correspondence for $\mathrm{GL}_1$ for
curves over $\mathbb{C}$.
Recall that there are 3 equivalent objects on Galois side for complex curves:

\begin{center}
    \begin{tikzcd}
        \boxed{\mathcal{E} = (E, \nabla), \text{ hol. (or $C^\infty$) line bundle with hol. (or flat) connection }} \arrow[d, leftrightarrow] \\
        \boxed{\text{locally constant sheaves of local systems of rank 1}} \arrow[d, leftrightarrow] \\
        \boxed{\pi_1(X, x_0) \to \GL_1(\mathbb{C})}
    \end{tikzcd}
\end{center}
here the correspondence between the first two boxes is often called \emph{Riemann-Hilbert correspondence}.
Note that the flat line bundles has algebraic nature (can be defined over Zariski topology), where
the local systems has analytic nature (need to be defined over analytic topology).
The correspondence between second and third is via monodromy action.

The automorphic side is the same as curves over finite fields cases.
For a given flat/holomorphic line bundle $\mathcal{E}$, one can associate a Hecke eigensheaf
$\mathcal{F}_{\mathcal{E}}$ satisfying $\mathbb{H}\mathcal{F}_{\mathcal{E}} \simeq \mathcal{E} \boxtimes \mathcal{F}_{\mathcal{E}}$,
where $\mathbb{H}$ is the Hecke functor defined as a pullback of $h: X \times \Pic_X \to \Pic_X, (x, \mathcal{L}) \mapsto \mathcal{L}(x)$.
By its property, the sheaf is determined by its restriction on $\Jac_X = \Pic_X^0$, the Jacobian variety of $X$
(or the neutral component of $\Pic_X$).
In case of curves over $\mathbb{C}$, there's much more elegant way to define $\mathcal{F}_{\mathcal{E}}^{0} = \mathcal{F}_{\mathcal{E}} |_{\Jac_X}$,
which we are going to introduce from now on.

First, we have an isomorphism
$$
\Jac_X \simeq \rH^{1}(X, \mathcal{O}_X) / \rH^{1}(X, \mathbb{Z})
$$
which can be obtained as follows.
We have a short exact sequence
$$
0 \to \mathbb{Z} \to \mathcal{O}_X \xrightarrow{\exp(2\pi i\cdot)} \mathcal{O}_X^{\times} \to 1
$$
which induces a long exact sequence of cohomology
$$
0 \to \rH^{1}(X, \mathbb{Z}) \to \rH^1(X, \mathcal{O}_X) \to \rH^{1}(X, \mathcal{O}_X^{\times}) \xrightarrow{\mathrm{deg}} \rH^{2}(X, \mathbb{Z}) \to 0
$$
where $\rH^{2}(X, \mathbb{Z}) \simeq \mathbb{Z}$ is generated by chern class.
Then we get the isomorphism from $\rH^{1}(X, \mathcal{O}_X^{\times}) \simeq \Pic_X$.
Note that $\rH^{1}(X, \mathcal{O}_X) \simeq \mathbb{C}^{g}$ for a complex curve $X$ with genus $g$,
and $\rH^{1}(X, \mathbb{Z})$ is a lattice in it.

Let's get back to the Galois side.
1-dimensional representation of $\pi_1(X, x_0)$ factors through the abelianization of it, which is $\rH_1(X, \mathbb{Z})$.
Also, for $\pi_1(\Jac_X, x_0) \simeq \pi_1(\rH^{1}(X, \mathcal{O}_X) / \rH^{1}(X, \mathbb{Z})) \simeq \rH^{1}(X, \mathbb{Z})$.
Combining with the Poincare duality $\rH^1(X, \mathbb{Z}) \simeq \rH_1(X, \mathbb{Z})$ we get the following diagram:
\begin{center}
    \begin{tikzcd}
        \mathcal{E} \arrow[r, leftrightarrow] & \pi_1(X, x) \arrow[rd, twoheadrightarrow]\arrow[rr] & & \GL_1(\mathbb{C}) \\
        & & \rH_1(X, \mathbb{Z}) \arrow[ru] \arrow[d, "\cong"] \\
        & & \rH^1(X, \mathbb{Z}) \arrow[rd] \\
        \mathcal{F}_\mathcal{E}^{0} \arrow[r, leftrightarrow] & \pi_1(\Jac_X, \tilde{x}) \arrow[ru, twoheadrightarrow] \arrow[rr] & & \GL_1(\mathbb{C})
    \end{tikzcd}
\end{center}
and this describes a nicer (functorial) way to construct Hecke eigensheaf $\mathcal{F}_{\mathcal{E}}^{0}$ on $\Jac_X$ (so on $\Pic_X$) corresponding to $\mathcal{E}$ 
(the ``eigenvalue'' of $\mathcal{F}_\mathcal{E}$ is $\mathcal{E}$).

Hence we have a following geometric Langlands correspondence over $\GL_1$ for complex curves:
\begin{center}
    \begin{tikzcd}
        \boxed{\mathcal{E}=(E, \nabla)\text{: holomorphic line bundles on }X} \arrow[d, leftrightarrow]\\
        \boxed{\mathcal{F}_\mathcal{E}^{0}\text{: Hecke eigensheaves on $\Jac_X = \Pic_X^0$}}
    \end{tikzcd}
\end{center}
Now, we'll figure out what is the top collections, i.e. the moduli space of holomorphic line bundles on $X$
(i.e. moduli space of rank 1 local systems on $X$), which we'll denote as $\mathsf{Loc}_{\GL_1, X}$.
First we have a map $\mathsf{Loc}_{\GL_1, X} \to \Jac_X$ defined as a ``forgetful'' map $(E, \nabla) \mapsto E$.
The fiber of this (surjective) map for $E \in \Jac_X$ is a set of all holomorphic connections on $E$.
For any such connection $\nabla$, we can obtain another connection $\nabla' = \nabla + \eta$ for any $\eta \in \rH^1(X, K_X)$ 
(here $K_X$ is a canonical line bundle on $X$).
Hence the fiber becomes a $\rH^{1}(X, K_X)$-torsor, and $\mathsf{Loc}_{\GL_1, X}$ becomes an affine bundle over $\Jac_X$.
Note that the equivalence classes of $\mathbb{G}_a$-bundles on some $Y$ is $\rH^{1}(Y, \mathcal{O}_Y)$, 
and more generally, the equivalence classes of $V$-bundles on $Y$ (where $V \simeq \mathbb{G}_{a}^{n}$ for some $n$) is in bijection with
$\rH^{1}(Y, \mathcal{O}_Y) \otimes V$.
When $Y = \Jac_X$, we have
$$
\rH^{1}(\Jac_X, \mathcal{O}_{\Jac_X}) \simeq \rH^{1}(X, \mathcal{O}_X) \simeq \rH^{0}(X, K_X)^{*}
$$
where the last isomorphism follows form Serre duality. Hence there exists a canonical element in 
$$
\rH^{1}(\Jac_X, \mathcal{O}_{\Jac_X}) \otimes \rH^{0}(X, K_X)
$$
which gives rise to a \emph{universal} extension
$$
0 \to \rH^{0}(X, K_X) \to \mathsf{Loc}_{\GL_1, X} \to \Jac_X \to 0
$$
which becomes the moduli space $\mathsf{Loc}_{\GL_1, X}$.
Universality implies that, for any $\varphi \in \rH^{1}(X, \mathcal{O}_X) \simeq \rH^{0}(X, K_X)^{*}$, 
this gives a $\mathbb{G}_a$-bundle on $\Jac_X$ as follows:
\begin{center}
    \begin{tikzcd}
        0 \arrow[r] & \mathbb{C} \arrow[r] & B_{\varphi} \arrow[r] & \Jac_X \arrow[r] \arrow[d, equal]& 0 \\
        0 \arrow[r] & \rH^{0}(X, K_X)\arrow[u, "\varphi"] \arrow[r] & \mathsf{Loc}_{\GL_1, X} \arrow[u, "\exists!\tilde{\varphi}"] \arrow[r] & \Jac_X \arrow[r] & 0
    \end{tikzcd}
\end{center}

In fact, the equivalence (Langlands correspondence) between local systems and Hecke eigensheaves are also equivalent
in a derived sense: there's an equivalence between the bounded derived category of $\mathcal{O}$-modules on $\mathsf{Loc}_{\GL_1, X}$
and the bounded derived category of $D$-modules on $\Jac_X$.
This is an example of \emph{Fourier-Mukai transform} (categorical version of the Fourier transform), which we are going to study in the next lecture.