\newpage
\section{More on Hecke algebra and Langlands correspondence for general reductive groups (September 6)}

We said that Langlands correspondence for $\GL_{n}$ gives a correspondence between invariants,
which are the Frobenius conjugacy classes (on Galois side) and the Hecke conjugacy classes (on automorphic side).
We are going to explain about Hecke conjugacy classes more in detail.

Let $F = \mathbb{F}_{q}(X)$ be a function field for a curve over finite field.
Let $\pi$ be an automorphic representation of $\GL_n(\mathbb{A}_F)$.
It decomposes as a restricted product of local representations of $\GL_{n}(F_x)$ as $\pi = \otimes'_{x\in |X|}\pi_x$.
Then there exists a finite set of (closed) points $S_\pi \subset |X|$ such that
$\pi_{x}^{\GL_{n}(\mathcal{O}_x)}\neq 0$, i.e. there exists $\GL_{n}(\mathcal{O}_{x})$-fixed vector in $\pi_x$.
We define Hecke algebra $\mathcal{H}_x$ as a convolution algebra on the set of 
compactly supported $\GL_{n}(\mathcal{O}_{x})$-bi-invariant functions with Haar measure on $\GL_n(F_x)$
normalized by $\mu(\GL_n(\mathcal{O}_x))=1$.
Then the Hecke algebra is actually commutative, and using this we can show that $\pi_x^{\GL_n(\mathcal{O}_x)}$ is actually 1-dimensional,
i.e. there exists a unique vector (up to scaling) $v_x \in \pi_x$ fixed by $\GL_n(\mathcal{O}_x)$.
The restricted product $\otimes_{x\in |X|}' \pi_x$ of local representations are defined as a span of vectors $\otimes_x w_x$
where $w_x\in \pi_x$ and $w_x = v_x$ for all but finitely many $x$.
In this case, the group $\GL_n(\mathbb{A}_F) = \otimes_x' \GL_n(F_x)$ acts on the space componenti-wise
$g.\otimes w_x := \otimes_x (g_x.w_x)$
and the previous argument this actually gives an action on the space $\otimes_{x}'\pi_x$.

Now for given local representation $\pi_x$, we can attach a representation of the Hecke algebra $\mathcal{H}_x$
where the representation space is $\pi_x^{\GL_n(\mathcal{O}_x)}$ action is given by (we use the same notation $\pi_x$ for the representation of $\mathcal{H}_x$)
$$
f \mapsto \pi_x(f): v \mapsto \int_{\GL_n(F_x)} f(g) \pi_x(g)v dg.
$$
The integral is well-defined since $f$ is compactly supported and $\GL_n(\mathcal{O}_x)$ preserves $\pi^{\GL_n(\mathcal{O}_x)}$.
This gives a functor from the category of representations of $\GL_n(F_x)$ and the category of representations of $\mathcal{H}_x$.
In fact, this sets up bijection between irreducible unramified representations of $\GL_n(F_x)$
and irreducible representations of $\mathcal{H}_x$.
Since $\mathcal{H}_x$ is commutative, irreducible representations of $\mathcal{H}_x$ are just characters of $\mathcal{H}_x$,
which we denote it as $\chi_x: \mathcal{H}_x \to \mathbb{C}$.

We can describe the structure of $\mathcal{H}_x$ more precisely as follows, which is a special case of so-called \emph{Satake isomorphism}.

\begin{theorem}
    $$
    \mathcal{H}_x \simeq \mathbb{C}[x_1^{\pm}, \dots, x_n^{\pm}]^{S_n}
    $$
    where the RHS is a space of symmetric Laurent polynonmials in $n$-variables.
\end{theorem}
Here's a sketch of proof.
First, the double coset space $\GL_n(\mathcal{O}_x) \backslash \GL_n(F_x) / \GL_n(\mathcal{O}_x)$ can be identified with $\mathbb{Z}^{n}/S_n$ as follows.
We have a map 
$$
\mathbb{Z}^{n} \to \GL_n(\mathcal{O}_x) \backslash \GL_n(F_x) / \GL_n(\mathcal{O}_x)
$$
that maps $(\lambda_1, \dots, \lambda_n) \in \mathbb{Z}^{n}$ to the double coset of the diagonal matrix
$$
\begin{pmatrix}
    t_{x}^{\lambda_1} & & & \\
    & t_{x}^{\lambda_@} & & \\
    & & \ddots & \\
    & & & t_{x}^{\lambda_n}
\end{pmatrix}
$$
where $t_x$ is a choosen uniformizer of $F_x$, so that $F_x \simeq (\mathbb{F}_q)_x((t_x))$ 
and $\mathcal{O}_x \simeq (\mathbb{F}_{q})_x[[t_x]]$.
Such a map is well-defined in the sense tha the double coset is independent of the choice of uniformizer $t_x$ - any other choice $t_x'$ satisfies $t_x'/t_x \in \mathcal{O}_x$.
The map is surjective, and it factors through $\mathbb{Z}^{n}/S_n$ with permutation on $\mathbb{Z}^n$ since
$(\lambda_1, \dots, \lambda_n)$ and $(\lambda_1', \dots, \lambda_n')$ with $\tau\lambda = \lambda'$ maps to the diagonal matrices
that are conjugate to each other by permutation matrix (corresponds to $\tau\in S_n$).

Using the identification, we can get the Satake isomorphism as follows.
Any $f \in \mathcal{H}_{x}$ as a form of $\sum_{\lambda} a_{\lambda} c_{\lambda}$
where $c_{\lambda}$ is a characteristic function on a double coset corresponds to an unordered set $\lambda = (\lambda_1, \dots, \lambda_n)$.
This sum is a finite sum since $f$ is compactly supported.
And the function $f$ corresponds to a symmetrized Laurent polynomial
$\sum_{\tau\in S_n} \tau(\sum a_{\lambda}x_1^{\lambda_1}\cdots x_{n}^{\lambda_n})$.

As we said before, Langlands correspondence for $\GL_n$ over a global function field is now a theorem.
\begin{theorem}[Deligne, Lafforgue]
   Let $F$ be a function field of a curve over a finite field.
   There is such a bijection between $n$-dimensional continuous irreducible representations of the Weil group $W(\overline{F}/F)$
   (with some technical conditions mentioned before) and irreducible cuspidal automorphic representations of $\GL_n(\mathbb{A}_F)$.
   This also gives a bijection between the Frobenius conjugacy classes and the Hecke conjugacy classes, and also these conjugacy classes
   are actually in $\GL_n(\overline{\mathbb{Q}})$.
\end{theorem}
$n=2$ case is proven by Delign, and $n > 2$ is by L. Lafforgue.
The difference between two cases are on the existence of \emph{nice} moduli space.
Lafforgue invented objects called \emph{Shutuka} and use moduli space of them for the cases $n >2$.


How can we state the Langlands correspondence for general reductive group $G$ (over function fields)?
First, the concepts we defined (Galois representations, automorphic representations, Hecke algebra, Frobenius conjugacy classes, ...)
generalizes to general reductive groups.
For example, Hecke algebra $\mathcal{H}_{x}$ is a convolution algebra of compactly supported functions on $G(F_x)$
which is bi-invariant under $G(\mathcal{O}_{x})$.
Then it is a commutative algebra for $x \in |X|$ with $\pi_x^{G(\mathcal{O}_x)}\neq 0$ (unramified), and
the invariant subspace $\pi_{x}^{G(\mathcal{O}_{x})}$ is 1-dimensional.
Also, we have a Satake isomorphism for $G$.

But if $\chi_x : \mathcal{H}_x \to \mathbb{C}$ is a character of $\mathcal{H}_{x}$, then where the corresponding
conjugacy class would live in?
For general reductive group $G$, Hecke conjugacy classes does not live in $G(\mathbb{C})$, but in different group called \emph{dual group} of $G$.
To define the notion of dual group, we have to define a \emph{root datum} first.

Let $T$ be a maximal torus of $G$, i.e. maximal commutative subgroup of $G$.
For $\GL_n$, it is a set of diagonal matrices.
For now, we will regard $G$ and $T$ as a group over $\mathbb{C}$.
Then we define the latteices of characters and cocharacters of $T$ as
\begin{align*}
X^{*}(T) = \mathrm{Hom}(T, \mathbb{G}_{m}) \\
X_{*}(T) = \mathrm{Hom}(\mathbb{G}_{m}, T)
\end{align*}
which are free abelian groups of finite rank.
We have a pairing $X^{*}(T)\times X_{*}(T) \to \mathbb{Z}$ defined by composition (note that all the morphisms $\mathbb{G}_{m} \to \mathbb{G}_{m}$
has a form of $x \mapsto x^{n}$).
We also have roots $\Delta \subset X^{*}(T)$ and coroots $\Delta ^{\vee} \subset X_{*}(T)$
which are nonzero eigenvalues of adjoint action of $T$ (and their duals).
Then we can associate a quadruple $(X^{*}(T), X_{*}(T), \Delta, \Delta^{\vee})$, a \emph{root datum} of $G$,
and it determines a group $G$ upto isomorphism when $G$ is split (i.e. admits a split maximal torus).
By simply flipping a root datum, we get another root datum
$$
(X_*(T), X^{*}(T), \Delta^{\vee}, \Delta)
$$
called dual root datum, and the group determined by this new root datum is called the \emph{dual} group of $G$, denoted by $\widehat{G}$.

Now let's get back to the Hecke algebra.
In case of $\GL_{n}$, we have a map $\mathbb{Z}^{n} \to \GL_2(\mathcal{O}_{x}) \backslash \GL_2(F_{x}) / \GL_2(\mathcal{O}_x)$ 
that induces an isomorphism 
$$
\mathbb{Z}^{n}/S_{n} \to \GL_2(\mathcal{O}_x) \backslash \GL_2(F_x) / \GL_2(\mathcal{O}_x).
$$
For general reductive group $G$, we have a map $X_{*}(T) \to G(\mathcal{O}_x) \backslash G(F_x) / G(\mathcal{O}_x)$ defined as an evaluation of character at uniformizer $t_{x}$.
Then this map factors through the quotient of $X_{*}(T)$ by the \emph{Weyl group} $W = W(G, T)$, the symmetry group of a root datum.
This induces an isomorphism 
$$
X_{*}(T) / W \simeq G(\mathcal{O}_{x}) \backslash G(F_x) / G(\mathcal{O}_x)
$$
which gives $\mathcal{H}_{x} \simeq \mathbb{C}[X_{*}(T)/W]$.
This saids that the character of $\mathcal{H}_x$ corresponds to an element in $X_{*}(T)/W$.
In this case, which kind of a conjugacy would corresponds to the element?
In fact, we have a canonical isomorphism $\mathbb{C}[X^{*}(T)/W] \simeq \mathbb{C}[T]^{W}$.
which gives a correspondence between $X^{*}(T)/W$ and semisimple conjugacy class in $G$.
However, we have $X_{*}(T)/W$ instead, and since $X_{*}(T)$ is a character group of the dual group $\widehat{G}$, 
we can concludes that $X_{*}(T)/W$ corresponds to semisimple conjugacy classes in $\widehat{G}$.

Based on this observation, the Langlands correspondence for a reductive group $G$ (over a function field) 
is a correspondence between
\begin{align*}
    \boxed{
        \text{irreducible representation } \sigma:W(\overline{F}/F) \to \widehat{G}(\overline{\mathbb{Q}_{\ell}})
    }
\end{align*}
and
\begin{align*}
    \boxed{
        \text{irreducible cuspidal automorphic representation of }G(\mathbb{A}_F)
    }
\end{align*}
where the invariants match: the Frobenius conjugacy classes 
$$
[\sigma(\mathrm{Fr}_{x})] \subset \widehat{G}(\overline{\mathbb{Q}_{\ell}}), \quad x \in |X| \backslash S_{\sigma}
$$
and the Hecke conjugacy classes (corresponds to characters of Hecke algebra $\mathcal{H}_x$)
$$
\pi(h_x) \subset \widehat{G}(\mathbb{C}),\quad x \in |X| \backslash S_\pi
$$
with a choice of identification $\iota: \mathbb{C} \simeq \overline{\mathbb{Q}_{\ell}}$, where $S_\sigma = S_\pi$ is a finite subset of $|X|$.