\newpage
\section{On vector bundles and flat connections (September 15)}


Let $X$ be a smooth real manifold.
A (complex) rank $n$ vector bundle $\mathcal{P}$ on $X$ is a manifold with projection map $\mathcal{P} \to X$ such that
there exists a covering $\{ U_{\alpha}\}$ of $X$ with local trivializations $u_{\alpha}: \mathcal{P}|_{U_\alpha} \to U_\alpha \times \mathbb{C}^{n}$.
On the intersections $U_\alpha \cap U_\beta$, we have two trivializations
$u_\alpha|_{U_\alpha \cap U_\beta}$ and $u_\beta|_{U_\alpha \cap U_\beta}$, and this gives $\GL_n(\mathbb{C})$-valued functions
$g_{\alpha\beta}$ comes from $u_{\beta}u_{\alpha}^{-1}: (U_\alpha \cap U_\beta) \times \mathbb{C}^n \to (U_\alpha\cap U_\beta)\times\mathbb{C}^n$.
We assume that such a transition function is smooth. Also, on triple intersections $U_\alpha \cap U_\beta \cap U_\gamma$, 
we assume that transition functions are compatible, in the sense that they satisfy cocycle conditions $g_{\alpha\gamma} = g_{\beta\gamma}g_{\alpha \beta}$.
Hence giving a vector bundle is equivalent to giving a covering $\{U_\alpha\}$ and smooth transition functions $\{g_{\alpha\beta}\}$
satisfying cocycle conditions.

We call that a vector bundle $\mathcal{P}$ is \emph{flat} if $g_{\alpha\beta}:U_{\alpha}\cap U_{\beta}\to \GL_n(\mathbb{C})$is a \emph{constant} function.
It means that we have an identification of all nearby fibers $\mathcal{P}_{x} \simeq \mathcal{P}_{x'}$ for all $x, x'\in U_\alpha$.
Equivalent way to say this is using a \emph{flat connection}.
Connection is a way of differentiating sections of vector bundles.
If a vector bundle is trivial, then this corresponds to an ordinary differentiation of a vector-valued function on $X$.
Hence we can do the same thing for general vector bundles locally, but this may depends on the choice of local trivializations.

\emph{Connection} is a map between between a sheaf of smooth vector fields on $X$ and a sheaf of smooth sections
of $\mathrm{End}(\mathcal{P})$, satisfying some linearity and Leibniz rules.
More precisely, it is a map
$$
\nabla: \mathcal{T}\to \mathcal{E}nd_{\mathbb{C}}(\mathcal{P})
$$
with $\xi \mapsto \nabla_{\xi}$, such that 
\begin{enumerate}
    \item ($\mathcal{O}_X$-linear) $\nabla_{\xi + \eta} = \nabla_{\xi} + \nabla_{\eta}$ and $\nabla_{f\xi} = f\nabla_{\xi}$
    for all $f \in \mathcal{O}_{X}$ and $\xi \in \mathcal{T}$.
    \item (Leibniz rule) $\nabla_{\xi}(f\cdot \phi) = f\cdot \nabla_{\xi}(\phi) + (\xi\cdot f)\phi$
    for all $f\in\mathscr{O}_X, \xi \in\mathcal{T}$, and $\phi \in \mathcal{P}$.
\end{enumerate}
A connection is called \emph{flat} if $\nabla$ is a Lie algebra homomorphism, i.e. 
$$
\nabla_{[\xi, \eta]} = [\nabla_{\xi}, \nabla_{\eta}].
$$
In other words, the connection form $R_{\xi\eta}:= \nabla_{[\xi, \eta]} - [\nabla_{\xi}, \nabla_{\eta}]$ is identically 0.
If $U\subset X$ is an open subset fih local coordinates $x_1, \dots, x_k$ and trivialization $\mathcal{P}|_U \simeq U \times \mathbb{C}^n$, it can be written as
$$
\nabla_{\frac{\partial}{\partial x_i}} = \frac{\partial}{\partial x_i} + A_i
$$
fomr some $\mathrm{M}_n(\mathbb{C})$-valued function $A_i$.
Then flatness of $\nabla$ is equivalent to 
$$
\frac{\partial A_{j}}{\partial x_{i}} - \frac{\partial A_{i}}{\partial x_{j}} + [A_i, A_j] =0
$$
for all $i, j$.

Let $\mathcal{S}_U$ be a set of all trivializations of $\mathcal{P}|_U$.
Then $\mathcal{G}_U:=\{f:U \to \GL_n(\mathbb{C})\}$ acts simply transitively on $\mathcal{S}_U$ (i.e. $\mathcal{S}_U$
is $\mathcal{G}_U$-torsor) via conjugation:
$$
g\left(\frac{\partial}{\partial x_i} + A_i\right)g^{-1} = \frac{\partial}{\partial x_i} + gA_i g^{-1} - \left(\frac{\partial g}{\partial x_i}\right)g^{-1},
$$
we call $A_i \mapsto gA_i g^{-1} - \left(\frac{\partial g}{\partial x_i}\right)g^{-1}$ as \emph{gauge} transform.

One can prove that flatness of a vector bundle is equivalent to imposing a flat connection on the vector bundle.
For example, if a flat connection $\nabla$ of $\mathcal{P}$ is given, then we can obtain local identification of nearby fibers
by solving a differential equation $\nabla \phi = 0$ locally, and connect from $p \in \mathcal{P}_{x}$ to $p' \in \mathcal{P}_{x'}$ along $\phi$.
By the theory of PDE, such $\phi$ (called \emph{horizontal section}) always uniquely exists (locally) for given initial condition.
Also, this gives a monodromy action of fundamental group.
For $p \in \mathcal{P}_x$ and a loop $\gamma$, we can ``solve'' $\nabla \phi = 0$ locally along a loop
and coming back to $x$, which gives $\rho_{\mathcal{P}}: \pi_1(X, x) \to \GL_n(\mathbb{C})$.
Since changing initialization gives conjugated representation, and we get a bijection between
equivalence classes of flat vector bundles of rank $n$ and equivalence classes of representations $\pi_1(X, x) \to \GL_n(\mathbb{C})$.
